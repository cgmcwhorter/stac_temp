\documentclass[11pt,letterpaper]{article}
\usepackage[lmargin=1in,rmargin=1in,bmargin=1in,tmargin=1in]{geometry}
\usepackage{style}

\pagenumbering{gobble}


% -------------------
% Content
% -------------------
\begin{document}

% TItle
\begin{center} 
\bfseries
\color{stacred}
\LARGE Syllabus Quick Facts \par\vspace{0.2\baselineskip}
\Large MATH 100: Fundamentals of Algebra --- Fall 2023 
\end{center} \pspace


% Course Information
\mysection{0.27}{Course Information}{course_info}
\hspace{0.53cm} {\itshape Instructor Email}: \href{mailto:cmcwhort@stac.edu}{cmcwhort@stac.edu} \par
\hspace{0.53cm} {\itshape Course Webpage}: \href{https://coffeeintotheorems.com/courses/2023-2/fall/math-100/}{https://coffeeintotheorems.com/courses/2023-2/fall/math-100/} \par
\hspace{0.53cm} {\itshape Office Hours}: 	\par \vspace{-0.3cm}
	\begin{table}[!ht]
	\centering
	\begin{tabular}{l || l}
	Mon. & 11:30~am -- 12:30~pm \\
	Tues. & 4:00~pm -- 5:00~pm \\
	Wed. & 11:30~am -- 12:30~pm \\
	Thurs. & 4:00~pm -- 5:00~pm \\
	Fri. & 11:30~am -- 1:30~pm
	\end{tabular}
	\end{table}


% Grading Components
\mysection{0.27}{Grading Components}{grade_comp}
Course grades are determined by the following components: \par \vspace{-0.3cm}
	\begin{table}[!ht]
        \begin{tabular}{clr}
	& Participation & 5\% \\
        & Activities & 5\% \\
	& Project & 10\% \\
	& Quizzes & 10\% \\
	& Exams & 30\% \\
	& Homework & 40\% 
        \end{tabular} 
        \end{table}


% Attendance & Participation
\mysection{0.27}{Attendance \& Participation}{attendace}
Attend each lecture and show up on time. Anticipated absences should be addressed with the instructor in advance of the absence. Address any absences---anticipated or otherwise---with the instructor. If you miss a lecture, you are responsible for any material covered, any work assigned, any course changes made, etc. during the class. Four or more unexcused absences from lectures could result in receiving a grade penalty per additional absence or an `F' in the course. Furthermore, excessive lateness will also count as an absence. \pspace


% Events
\mysection{0.27}{Events}{events}
At STAC, we are committed to fostering a student-focused, inclusive, and engaging environment. This course should help to foster community building. You will be required to attend at least 10~different approved college or community events by the end of the semester. These events could include convocation, seminars or other college presentations, sporting events, college social nights, community volunteerism events, etc. If you are unsure whether a particular activity is appropriate, consult with your instructor before assuming that it will be counted. You must submit proof of attendance and participation for each of these events to the instructor. This could be a photo or video of event participation (including the student), a signed form/affidavit, etc. The submission should include the name of the event and the day/time of the event. Each event will be weighted equally. Students are especially encouraged to attend events with others---especially from the course! While the same verification, e.g. the same photo of the students at the event, may be used by multiple students, each student need submit the verification separately. \pspace


% Project
\mysection{0.27}{Project}{project}
Contemporary students should be familiar with what data science is, why it is so ubiquitous, where it is being and might be used, the pros and cons of its use, and how to make use of current technology created by its practitioners in a way that would be attractive to future employers. Therefore, each student in the course will produce some creative endeavor using data science related technologies of their choosing. Specifically, each student will have an ultimate product or goal in mind. They then find several technologies---open source, paid, or otherwise---to design, create, alter, implement, etc. their `product' or achieve their goal. The specifics of this aspect of the project are intentionally vague as each student may choose anything of interest or anything related to their career goals; however, the student should be sure that their project idea is manageably achievable given time constraints, technological skills, availability of programs or services, costs, etc. The project should make use of several distinctive technologies, e.g. computer programs, apps, websites, etc., in their product production or goal achievement. Furthermore, the project should not be a `single step' and instead involve several distinct, substantive processes; for example, an inappropriate project would be to choose to write a poem or short story and simply prompt ChatGPT, ``write me a poem or short story.'' The project may be \textit{somewhat} hypothetical in nature, so long as a solid proof of concept is demonstrated with at least one concrete example output. All work related to the project will be submitted via Canvas. However, if there are submission issues, then discuss this with the instructor as soon as possible. Finally, each student will give a 7--10 minute presentation of their project, where they will `pitch' their idea, overview their project, present their outputs, and discuss the societal implications of their project. This presentation might be thought of as a longer, more ethics focused, elevator pitch. \pspace

Each student will work independently on a unique project that they propose. Before starting the project, students should submit their project proposal to the course instructor. Students cannot work on the same project. Project proposals are approved on a first-come, first-serve basis. Do not wait until the last minute to think of an idea and propose it to the instructor. Project proposals can be made to the instructor via email, during office hours, or through an individual appointment. If you are a struggling coming up with a proposal or experiencing difficulties with the project, do not hesitate or delay in discussing this with the course instructor. Projects will be submitted the last calendar week of the semester and project presentations will be the last week of classes. \pspace


% Quizzes 
\mysection{0.27}{Quizzes}{quizzes}
There will be a quiz \textit{every} class, typically at the start of class. Because solutions will often then be immediately discussed, no make-up quizzes will be given (except under extraordinary circumstances). \pspace


% Homeworks 
\mysection{0.27}{Homeworks}{homeworks}
There will typically be a homework assigned each class, due the next class. Homework is a large portion of your grade, so your best work should be put into them. Your solutions should be neat, organized, display effort and clear mathematical thinking, and they should be submitted using the homework packets. Assignments should be started as soon as possible; it is easier to keep up than it is to catch up. You may request extensions on homework assignments (possibly incurring a grade penalty). Requests for extensions should be submitted to the instructor in a timely fashion---do not delay! However, do not simply assume that you will be able to receive extra time on an assignment and plan your schedule carefully. You are encouraged to work with others on homeworks; however, be sure to carefully abide by the academic integrity standards excepted by the college and instructor. \pspace


% Exams 
\mysection{0.27}{Exams}{exams}
There will be a total of 3 exams that are each worth 10\% of the course grade for a total of 30\%. The tentative schedule for these exams can be found below. Each exam covers course material, up until the exam preceding it. While the exams are not cumulative, topics from previous exams can appear in an exam if the material is relevant---but it will not be the focus of the exam. You should be present, seated, and prepared for a scheduled exam before the exam begins. If you are late, you should not expect extra exam time. There are no make-up exams except under extraordinary circumstances. \pspace


% Course Schedule 
\mysection{0.27}{Course Schedule}
The following is a \emph{tentative} schedule for the course and is subject to change. 
        \begin{table}[!ht]
        \centering
        \scalebox{1}{%
        \begin{tabular}{ll || ll}
        Date & Topic(s) & Date & Topic(s) \\ \hline 
	09/06 & Percentages & 10/30 & Review \\
	09/11 & Weighted Averages \& Unit Conversion & 11/01 & Data Science \\
	09/13 & Geometry & 11/06 & Data Science \\
	09/18 & Geometry \& Rates & 11/08 & Exam 2 \\
	09/20 & Fermi Estimation & 11/13 & Probability \\
	09/25 & Functions & 11/15 & Probability \\
	09/27 & Linear Functions & 11/20 & Statistics \\
	10/02 & Other Functions & 11/22 & Thanksgiving Break \\
	10/04 & Review & 11/27 & Statistics \\
	10/09 & Study Day & 11/29 & Statistics \\
	10/11 & Exam 1 & 12/04 & Statistics \\
	10/16 & Financial Mathematics & 12/06 & Review \\
	10/18 & Financial Mathematics & 12/11 & Project Presentations \\
	10/23 & Financial Mathematics & 12/13 & Exam 3 \\
	10/25 & Financial Mathematics & 
        \end{tabular}
        }
        \end{table}


\end{document}