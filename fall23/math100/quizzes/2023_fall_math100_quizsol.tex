\documentclass[11pt,letterpaper]{article}
\usepackage[lmargin=1in,rmargin=1in,bmargin=1in,tmargin=1in]{geometry}
\usepackage{style/quiz}
\usepackage{style/commands}

% -------------------
% Content
% -------------------
\begin{document}
\thispagestyle{title}

% Quiz 1
\quizsol \textit{True/False}: The product $57(1.08)$ can be interpreted either as both finding 8\% of 57 and increasing 57 by 8\%. \pspace

\sol The statement is \textit{false}. To find a percentage $\%$ of a number $N$, we compute $N \cdot \%_d$, where $\%_d$ is the percentage written as a decimal. But then finding 8\% of 57 is $57(0.08)$, not $57(1.08)$. The product $57(1.08)$ would represent finding 108\% of 57. To find the result of a number $N$ increased or decreased by a percentage $\%$, we compute $N(1 \pm \%_d)$, where $\%_d$ is the percentage written as a decimal, and we choose `$+$' if a percentage increase and `$-$' if a percentage decrease. But then to compute 57 increased by 8\%, we compute $57(1 + 0.08)= 57(1.08)$, as stated in the quiz. Therefore, the quiz statement is false. \pvspace{1.3cm}



% Quiz 2
\quizsol \textit{True/False}: Your GPA after the end of your Freshman year (30~credits) was 3.217. If you took 16 credits in the Fall of your Sophomore year and had a semester GPA of 3.615, then your current GPA is $\frac{30(3.217) + 16(3.615)}{30+16}= \frac{154.35}{46} \approx 3.355$. \pspace

\sol The statement is \textit{true}. To compute ones new GPA, one computes\dots
	\[
	\begin{aligned}
	\text{Overall GPA}&= \dfrac{\text{Previous Credits} \cdot \text{Previous GPA} + \text{Semester Credits} \cdot \text{Semester GPA}}{\text{Total Credits}} \\
	&= \dfrac{30 \cdot 3.217 + 16 \cdot 3.615}{30 + 16} \\
	&= \dfrac{96.51 + 57.84}{30 + 16} \\
	&= \dfrac{154.35}{46} \\
	&\approx 3.355
	\end{aligned}
	\] \pvspace{1cm}



% Quiz 3
\quizsol \textit{True/False}: The distance between the points $(4, 3)$ and $(-1, 6)$ is $\sqrt{(-1 - 4)^2 + (6 - 3)^2}= \sqrt{(-5)^2 + 3^2}= \sqrt{25 + 9}= \sqrt{34} \approx 5.83095$. \pspace

\sol The statement is \textit{true}. The distance between two points $(x, y)$ and $(a, b)$ is $d= \sqrt{(x - a)^2 + (y - b)^2}$. But then taking $(x, y)= (-1, 6)$ and $(a, b)= (4, 3)$, we have\dots
	\[
	d= \sqrt{(-1 - 4)^2 + (6 - 3)^2}= \sqrt{(-5)^2 + 3^2}= \sqrt{25 + 9}= \sqrt{34} \approx 5.83095
	\] \pvspace{1.3cm}



% Quiz 4
\quizsol \textit{True/False}: A 10~ft $\times$ 10~ft $\times$ 20~ft container is filled with 500~ft$^3$ of syrup. More syrup is flowing in at a rate of 40~ft$^3$ per minute. Because the volume is $10 \cdot 10 \cdot 20= 2000 \text{ ft}^3$, the time it will take to fill the container is $t= \frac{2000 \text{ ft}^3}{40 \text{ ft}^3/\text{min}}= 50 \text{ min}$. \pspace

\sol The statement is \textit{true}.


% Quiz 5
\quizsol \textit{True/False}: Alice and Bob and start at the same location. Alice travels north at 3~mph and Bob travels east at 4~mph. Two hours later, Alice has traveled $3 \text{ mph} \cdot 2 \text{ hr}= 6 \text{ mi}$, Bob has traveled $4 \text{ mph} \cdot 2 \text{ hr}= 8 \text{ mi}$, and Alice and Bob are $6 \text{ mi} + 8 \text{ mi}= 14 \text{ mi}$ apart. \pspace

% Quiz 6
\quizsol \textit{True/False}: If $f(x)= 5x - 4$ and $g(x)= 6x$, then $(f \circ g)(2)= f(2) \cdot g(2)= 6 \cdot 12= 72$. \pspace

%The function $f(x)= \dfrac{x + 1}{x - 3}$ is linear. 
%In `real-life', exponential growth is not sustainable and `most often' logistic models have to be used.
%The current CPI is $307.789$ and if next year it is $318.438$, then the inflation rate was $\frac{318.438}{307.789}= 1.0346$; hence, the inflation rate from this year to next year would be 3.46\%. 
%If you invest \$250 at 4.8\% annual interest, compounded quarterly for 8 years, then the investment is worth $\$250 \left(1 + \dfrac{0.048}{4} \right)^8 \approx \$275.03$. 

%A series of equal payments at equal time intervals is an annuity. 


%One will always make more interest compounding continuously than compounding $k$ times per year at the same interest rate, no matter the value for $k$. 

%The amount you would have to invest at 1.3\% annual interest, compounded continuously to have \$8,000 in 5 years is $\frac{\$8000}{e^{0.13 \cdot 5}} \approx \$4176.37$. 


%Because artificial intelligence algorithms execute code to learn from data, they are free from bias. 

% You have 5 coins in your pocket: two dimes and three nickels. If you choose two coins from your pocket, one after the other, the probability of getting 20\cents is $\frac{2}{5} \cdot \frac{3}{5}= \frac{6}{25} \approx 0.24$. 

\end{document}