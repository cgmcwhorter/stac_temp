\documentclass[12pt,letterpaper]{exam}
\usepackage[lmargin=1in,rmargin=1in,tmargin=1in,bmargin=1in]{geometry}
\usepackage{../style/exams}

% -------------------
% Course & Exam Information
% -------------------
\newcommand{\course}{MAT 100: Exam 1}
\newcommand{\term}{Fall -- 2023}
\newcommand{\examdate}{11/08/2023}
\newcommand{\timelimit}{85 Minutes}

\setbool{hideans}{true} % Student: True; Instructor: False

% -------------------
% Content
% -------------------
\begin{document}

\examtitle
\instructions{Write your name on the appropriate line on the exam cover sheet. This exam contains \numpages\ pages (including this cover page) and \numquestions\ questions. Check that you have every page of the exam. Answer the questions in the spaces provided on the question sheets. Be sure to answer every part of each question and show all your work. If you run out of room for an answer, continue on the back of the page --- being sure to indicate the problem number.} 
\scores
\bottomline
\newpage

% ---------
% Questions
% ---------
\begin{questions}

% Question 1
\newpage
\question[10] Ryker is a muckbanger and his YerbTurd channel has taken off. Over the last year, he made \$187,000. He is preparing his 2023 federal tax burden. Ryker will take the 2023 standard deduction for single filers of \#13,850. What will Ryker pay in federal taxes? 
	\begin{table}[H]
	\centering
	\begin{tabular}{|l|l|} \hline
	Tax Rate & Taxable Income \\ \hline \hline
	10\% & Up to \$11,000 \\ \hline
	12\% & \$11,001 -- \$44,725 \\ \hline
	22\% & \$44,726 -- \$95,375 \\ \hline
	24\% & \$95,376 -- \$182,100 \\ \hline
	32\% & \$182,101 -- \$231,250 \\ \hline
	35\% & \$231,251 -- \$578,125 \\ \hline
	37\% & $\geq$ \$578,126 \\ \hline
	\end{tabular}
	\end{table}



% Question 2
\newpage
\question[10] Cashanova is a local savings bank for the average pre-millionaire. Their savings account offer an annual interest rate of 5.75\%, compounded monthly. 
	\begin{enumerate}[(a)]
	\item What is the nominal interest rate for Cashanova's savings account?
	\item What is the effective interest rate for Cashanova's savings account?
	\end{enumerate}



% Question 3
\newpage
\question[10] It was 1984 and inflation was on the rise. The CPI in 1983 was 105.313, whereas the CPI in 1984 was 111.039. 
	\begin{enumerate}[(a)]
	\item Find the inflation rate from 1983 to 1984.
	\item If an item cost \$0.94 in 1984, assuming the inflation rate from (a) remains the same, estimate the cost of the item today. 
	\end{enumerate}
	
	

% Question 4
\newpage
\question[10] Warren T. is saving to replace his washer/dryer should they ever break. He places \$150 in an account that earns 1.03\% annual interest, compounded continuously. The money sits there for 8~years until the current washer/dryer break. How much is in the account?



% Question 5
\newpage
\question[10] You are going to take out a loan so that you can travel across the United States to the Grand Canyon to see the faces of the presidents. The bank offers you a simple discount note for \$1,500 at 8.4\% annual interest for a period of 6~months. 
	\begin{enumerate}[(a)]
	\item What is the discount for this loan?
	\item Find the maturity and proceeds for this loan. 
	\item How much do you owe the bank at the end of the six months?
	\end{enumerate}



% Question 6
\newpage
\question[10] Your favorite anime, \textit{Attack on Behemoth}, just ended. You have the entire manga collection, currently worth \$275. You estimate that the value of the collection will increase by 5\% in value every four months. How long until the collection is worth \$1,000?



% Question 7
\newpage
\question[10] April begins a business selling paintings of people walking other people's dogs. The studio she rents costs \$3,600 per month to rent. It costs her \$8.20 to make each painting; however, each painting sells for \$150. Find the minimum number of paintings she must make and sell each month to turn a profit. 



% Question 8
\newpage
\question[10] You are saving to spend on a lavish meal at Hell's Kitchen in the hopes that you can watch Gordon Ramsay yell at someone. You place \$20 at the end of the month into an account that earns 2.3\% annual interest, compounded monthly. After 18~months, how much have you saved?



% Question 9
\newpage
\question[10] When you were born, your meemaw placed \$8,000 into an account that earned 4.1\% annual interest, compounded monthly. Now 18~years later, she is going to empty the account to give to you for your high school graduation. How much money will you receive? 



% Question 10
\newpage
\question[10] While watching \textit{Hope Floats}, you come up with a brilliant idea---invest in cryptocurrency. You hear that HuskyCoin offers an annual return in value equivalent to 3.9\% annual interest, compounded continuously. If you want to have \$10,000 in two years, how much money should you invest in the totally non-scam coin now?


\end{questions}
\end{document}