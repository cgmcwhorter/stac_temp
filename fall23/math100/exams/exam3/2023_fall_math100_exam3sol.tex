\documentclass[12pt,letterpaper]{exam}
\usepackage[lmargin=1in,rmargin=1in,tmargin=1in,bmargin=1in]{geometry}
\usepackage{../style/exams}

% -------------------
% Course & Exam Information
% -------------------
\newcommand{\course}{MAT 100: Exam 3}
\newcommand{\term}{Fall -- 2023}
\newcommand{\examdate}{12/13/2023}
\newcommand{\timelimit}{85 Minutes}

\setbool{hideans}{false} % Student: True; Instructor: False

% -------------------
% Content
% -------------------
\begin{document}

\examtitle
\instructions{Write your name on the appropriate line on the exam cover sheet. This exam contains \numpages\ pages (including this cover page) and \numquestions\ questions. Check that you have every page of the exam. Answer the questions in the spaces provided on the question sheets. Be sure to answer every part of each question and show all your work. If you run out of room for an answer, continue on the back of the page --- being sure to indicate the problem number.} 
\scores
\bottomline
\newpage

% ---------
% Questions
% ---------
\begin{questions}

% Question 1
\newpage
\question[10] Indicate whether the underlined measurement represents a quantitative or categorical variable.
	\begin{enumerate}[(a)]
	\item I would describe their relationship as more of a \underline{situationship}. 
	\item Dr. Draye entered the trial participants as \underline{0} for placebo and \underline{1} for active drug. 
	\item Troye easily spend \underline{nineteen} hours this week watching \textit{Jingle All the Way}. 
	\item My venti, seven pump pumpkin spice latte with eight shots of espresso and one pump of maple pecan sauce cost me \underline{\$27.85}. 
	\end{enumerate} \pspace

\sol 
\begin{enumerate}[(a)]
\item Categorical \pspace

\item Categorical \pspace

\item Quantitative \pspace

\item Quantitative 
\end{enumerate}



% Question 2
\newpage
\question[10] Indicate whether the described experiment represents a convenience, systematic, stratified, or cluster sampling. 
	\begin{enumerate}[(a)]
	\item Saint Tom Aquiner College surveyed fifty people from each dorm room to determine whether their students had holiday spirit. 
	\item Ben surveyed every AI programmer from twenty of the top fifty AI companies about their views on AI safety. 
	\item Tequila Mockingbird liquor store surveyed every twentieth customer purchasing nog about their customer satisfaction.  
	\item You ask every member of your fam whether or not you got the drip. 
	\end{enumerate} \pspace

\sol 
\begin{enumerate}[(a)]
\item Stratified \pspace

\item Cluster \pspace

\item Systematic \pspace

\item Convenience 
\end{enumerate}



% Question 3
\newpage
\question[10] Indicate whether the underlined measurement represents a nominal, ordinal, interval, or ratio level measurement.
	\begin{enumerate}[(a)]
	\item The \underline{eighth} Final Fantasy game is the best one. 
	\item The weather outside is \underline{frightful}. 	
	\item Despite it being the Christmas season, it is \underline{43$^\circ$F} outside. 
	\item There are \underline{57} people in line to try fish sticks at the new restaurant \textit{Frying Nemo}. 
	\end{enumerate} \pspace

\sol 
\begin{enumerate}[(a)]
\item Ordinal \pspace

\item Nominal \pspace

\item Interval \pspace

\item Ratio
\end{enumerate}



% Question 4
\newpage
\question[10] A professor surveyed students on the number of breakdowns they had during a single math homework assignment. The results are found below. 
	\[
	1, \quad 10, \quad 17, \quad 21, \quad 23, \quad 24, \quad 24, \quad 25, \quad 25, \quad 27, \quad 30, \quad 34, \quad 39
	\]

\begin{enumerate}[(a)]
\item Compute the median for this data set. 
\item Compute the 5-number summary for this data set. 
\item Find the IQR for this dataset. 
\item Sketch a box plot for this dataset. 
\end{enumerate} \pspace

\sol Observe that the data has already been placed in ascending order. 
\begin{enumerate}[(a)]
\item The median for this dataset will be the $\frac{13}{2} = 6.5 \squiggle 7$th value in this dataset. Therefore, the median is $24$. \pspace

\item The five-number summary for a dataset consists of the min, $Q_1$, median, $Q_3$, and max. Recall that $Q_1$ is the $25$th percentile and $Q_3$ is the $75$th percentile. Therefore, $Q_1$ is the $0.25 \cdot 13= 3.25 \squiggle 4$th value in the dataset, which is $21$, and $Q_3$ is the $0.75 \cdot 13= 9.75 \squiggle 10$th value in the dataset, which is $27$. Therefore, the five-number summary is\dots
	\begin{table}[ht]
	\centering
	\begin{tabular}{ccccc}
	Min & $Q_1$ & Median & $Q_3$ & Max \\ \hline 
	$1$ & $21$ & $24$ & $27$ & $39$
	\end{tabular}
	\end{table} \par

\item We have\dots
	\[
	\text{I.Q.R.} = Q_3 - Q_1 = 27 - 21 = 6
	\] \pspace

\item \phantom{.} \par
	\begin{center}
	\begin{tikzpicture}
	% Number Line
	\draw (0,0) -- (11,0); 			% Main Line
	\draw (0,-0.15) -- (0,0.15); 	% Left Tick
	\draw (11,-0.15) -- (11,0.15); 	% Right Tick
	
	% BoxPlot
	\draw[line width=0.02cm] (0.22,0.5) -- (0.22,1.5); % Min
	\draw[line width=0.02cm] (4.62,0.5) -- (4.62,1.5); % Q1
	\draw[line width=0.02cm] (5.28,0.5) -- (5.28,1.5); % Median
	\draw[line width=0.02cm] (5.94,0.5) -- (5.94,1.5); % Q3
	\draw[line width=0.02cm] (8.58,0.5) -- (8.58,1.5); % Max
	
	\draw[line width=0.02cm] (4.62,1.5) -- (5.94,1.5); % Top Line
	\draw[line width=0.02cm] (4.62,0.5) -- (5.94,0.5); % Bottom Line
	\draw[line width=0.02cm] (0.22,1) -- (4.62,1); 	  % Left Line
	\draw[line width=0.02cm] (5.94,1) -- (8.58,1); 	  % Right Line
	
	% Labels
	\node at (0,-0.5) {\footnotesize $0$}; 		% 0
	\node at (11,-0.5) {\footnotesize $50$}; 	% 50 
	
	\node at (0.22,-0.5) {\footnotesize $1$};	\draw (0.22,-0.25) -- (0.22,0.25); % 1
	\node at (4.62,-0.5) {\footnotesize $21$};	\draw (4.62,-0.25) -- (4.62,0.25); % 21
	\node at (5.28,-0.5) {\footnotesize $24$};	\draw (5.28,-0.25) -- (5.28,0.25); % 24
	\node at (5.94,-0.5) {\footnotesize $27$};	\draw (5.94,-0.25) -- (5.94,0.25); % 27
	\node at (8.58,-0.5) {\footnotesize $39$};	\draw (8.58,-0.25) -- (8.58,0.25); % 39
	\end{tikzpicture}
	\end{center}
\end{enumerate}



% Question 5
\newpage
\question[10] The number of minutes four students spent studying for a math final exam are found below. 
	\[
	1, \quad 6, \quad 6, \quad 12
	\]

\begin{enumerate}[(a)]
\item Compute the mean for this data set.
\item Compute the standard deviation for this data set. 
\item Is the median or mean a more robust measure of center for this data set? Explain. 
\end{enumerate} \pspace

\sol 
\begin{enumerate}[(a)]
\item The mean is\dots
	\[
	\overline{x} = \dfrac{\sum x_i}{n} = \dfrac{1 + 6 + 6 + 12}{4} = \dfrac{25}{4} = 6.25
	\] \pspace

\item We compute all the individual terms to compute the standard deviation: \par
	\begin{table}[ht]
	\centering
	\begin{tabular}{ccr}
	$x_i$ & $x_i - \overline{x}$ & $(x_i - \overline{x})^2$ \\ \hline
	$1$ & $-5.25$ & $27.5625$ \\
	$6$ & $-0.25$ & $0.0625$ \\
	$6$ & $-0.25$ & $0.0625$ \\
	$12$ & $\phantom{-}5.75$ & $33.0625$ \\ \cline{3-3}
		& 			& Total: $60.75$
	\end{tabular}
	\end{table} \par
But then the standard deviation is\dots
	\[
	\sigma = \sqrt{ \dfrac{1}{n - 1} \sum (x_i - \overline{x})^2} = \sqrt{ \dfrac{1}{3} \cdot 60.72} = \sqrt{20.25} = 4.5
	\] \pspace
 
\item The median is always a more robust measure of center than the mean as it is resistant to outliers whereas the mean is not resistant to outliers. The median of this dataset is clearly 6. 
\end{enumerate}



% Question 6
\newpage
\question[10] Though one's man may not walk 500~miles for you, the number of miles a man is willing to walk is found to be normally distributed with mean 42.1~miles and standard deviation 8.6~miles
	\begin{enumerate}[(a)]
	\item Find the percent of men that will walk less than 40~miles for you.
	\item Find the percent of men that will walk more than 40~miles for you.
	\item Find the number of miles a man needs to walk to be in the top 1\% of men that will walk for you.
	\end{enumerate} \pspace

\sol 
\begin{enumerate}[(a)]
\item 
	\[
	z_{40}= \dfrac{40 - 42.1}{8.6}= \dfrac{-2.1}{8.6}= -0.24 \squiggle 0.4052
	\]
Therefore, $P(x \leq 40) = 0.4052$. \pspace

\item We have\dots
	\[
	P(x \geq 40) = 1 - P(x \leq 40) = 1 - 0.4052 = 0.5948
	\] \pspace

\item If a man walked the minimal number of miles to be in the top 1\% of men that will walk for you, then this man has walked more miles than 99\% of the men have walked for you. But then if $x$ were this minimal number of miles, we know that $z_x \squiggle 0.99$. Examining a $z$-score chart, it is clear that $z_x \approx 2.33$. But then\dots
	\[
	\begin{gathered}
	z_x = \dfrac{x - \mu}{\sigma} \\[0.3cm]
	2.33 = \dfrac{x - 42.1}{8.6} \\[0.3cm]
	20.038 = x - 42.1 \\[0.3cm]
	x = 62.038
	\end{gathered}
	\]
Therefore, a man needs to walk at least 62.038~miles to be in the top 1\% of men that would walk for you. 
\end{enumerate}



% Question 7
\newpage
\question[10] Scientists have put their top minds on the problem of what the average `rizz' rating is for current celebrities. Participants were asked to rate the `rizz' of various Hollywoo stars and celebrities\footnote{What do they know? Do they know things? Let's find out.} on a scale of 0 to 100. The study examined 37~individuals and found an average `rizz' rating of 72.4. The standard deviation for `rizz' ratings for the population was known to be 13.9. Find a 98\% confidence interval for the average `rizz' rating of Hollywoo stars and celebrities. \pspace

\sol We assume that the sample was a simple random sample. The study examined $n= 37 > 30$ individuals. Therefore, this sample is `sufficiently large' for the Central Limit Theorem to apply. If the confidence interval contains 98\% of the values, there are 2\% of values outside of this interval. Because the normal distribution is symmetric, this leaves 1\% of values above/below the confidence interval.  But then the upper value in this confidence interval is greater than 98\% + 1\%= 99\% of values in the distribution. If $x$ were this upper value, then $z_x \squiggle 0.99$. Examining a $z$-score chart, we have $z_x \approx 2.33$. But then the confidence interval is given by\dots
	\[
	\begin{gathered}
	\overline{x} \quad \pm \quad z^* \, \dfrac{\sigma}{\sqrt{n}} \\[0.3cm]
	72.4 \quad \pm \quad 2.33 \cdot \dfrac{13.9}{\sqrt{37}} \\[0.3cm]
	72.4 \quad \pm \quad 2.33 \cdot 2.28515 \\[0.3cm]
	72.4 \quad \pm \quad 5.32
	\end{gathered}
	\]
Therefore, the confidence interval is $(67.08, 77.72)$, i.e. ``there is a 98\% chance that the average `rizz' rating for Hollywoo stars and celebrities is between 67.08\% and 77.72\%.''\footnote{Strictly speaking, a 98\% chance that this interval contains the true mean.} 



% Question 8
\newpage
\question[10] Visitors to a winter wonderland festival at a local college were asked what their favorite Christmas movie was. The results of the survey, broken down by age, are given below. \par
	\begin{table}[H]
	\centering
	\begin{tabular}{|l|cccc|r|} \hline
	 & 18 -- 21 & 22 -- 35 & 35 -- 50 & 50+ & \multicolumn{1}{c|}{Total} \\ \hline
	\textit{Elf} & 40 & 18 & 30 & 32 & 120 \\
	\textit{Home Alone} & 20 & 38 & 42 & 14 & 114 \\
	\textit{Muppet Christmas Carol} & 24 & 42 & 39 & 33 & 138 \\
	\textit{A Christmas Story} & 2 & 7 & 23 & 50 & 82 \\
	\textit{Gremlins} & 0 & 0 & 1 & 3 & 4 \\ \hline
	Total & 86 & 105 & 135 & 132 & \multicolumn{1}{c|}{458} \\ \hline
	\end{tabular}
	\end{table} \par

\begin{enumerate}[(a)]
\item What is the probability that an individual surveyed was 35 -- 50 or preferred \textit{Muppet Christmas Carol}?
\item What is the probability that an individual surveyed was 22 -- 35 and preferred \textit{Elf}?
\item What is the probability that an individual that was 51+ preferred \textit{A Christmas Story}?
\item Are preferring \textit{Gremlins} and being 18 -- 21 disjoint events? Can these events be independent? 
\end{enumerate} \pspace

\sol 
\begin{enumerate}[(a)]
\item 
	\[
	P(\text{35 -- 50 or } \textit{Muppet Christmas Carol}) = \dfrac{135 + 138 - 39}{458} = \dfrac{234}{458} \approx 0.5109
	\] \pspace

\item 
	\[
	P(\text{22 -- 35 and } \textit{Elf}) = \dfrac{18}{458} \approx 0.0393
	\] \pspace

\item 
	\[
	P(\textit{A Christmas Story} \;|\; 51+)= \dfrac{50}{132} \approx 0.3788
	\] \pspace

\item Yes, there were no people that preferred \textit{Gremlins} and were 18 -- 21. But then these events cannot be independent because they are disjoint. [Disjoint events are \textit{never} independent.]
\end{enumerate}


\end{questions}
\end{document}