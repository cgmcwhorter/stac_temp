\documentclass[11pt,letterpaper]{article}
\usepackage[lmargin=1in,rmargin=1in,tmargin=1in,bmargin=1in]{geometry}
\usepackage{../style/homework}
\usepackage{../style/commands}
\setbool{quotetype}{false} % True: Side; False: Under
\setbool{hideans}{false} % Student: True; Instructor: False

% -------------------
% Content
% -------------------
\begin{document}

\homework{9: Due 10/30}{Whether we are working to pay off student loans, credit card debt, paying for elder or childcare, or even trying to save for retirement, the idea of the American dream still remains just that---a dream.}{Adrienne Moore}

% Problem 1
\problem{10} Robert Schock is taking out a loan to finance an expansion for his business, \textit{Electricity Bill}. The bank offers him a \$37,000 discount note for 10~months at 10.2\% annual interest. 
	\begin{enumerate}[(a)]
	\item What are the maturity and discount for this loan?
	\item What are the proceeds for this loan?
	\item At the end of the 10~months, how much does Robert owe the bank?
	\item How much does Robert pay in total for this loan?
	\end{enumerate} \pspace

\sol 
\begin{enumerate}[(a)]
\item The maturity is the loan amount, which is \$37,000. The discount is the interest that is paid up-front. We know that $D= Mrt$. But then we have\dots
	\[
	D= Mrt= \$37,\!000 (0.102)\, \frac{10}{12}= \$3,\!145.00
	\] \pspace

\item The proceeds are the amount received from the bank after the up-front interest is paid. But then the proceeds, $P$, are $P= M - D$. Therefore, \dots
	\[
	P= M - D= \$37,\!000 - \$3,\!145.00= \$33,\!855
	\] \pspace

\item Robert only ever has to pay the bank the loan amount (the maturity, $M$) and any interest on the loan. However, the interest (the discount, $D$) is paid up-front. Therefore, at the end of the 10~months, Robert only owes the loan amount---\$37,000. \pspace

\item Robert only ever pays the loan amount (the maturity, $M$) and any interest on the loan (the discount, $D$). Therefore, the total amount paid on this loan is\dots
	\[
	\text{Total}= M + D= \$37,\!000 + \$3,\!145.00= \$40,\!145
	\]
\end{enumerate}



\newpage



% Problem 2
\problem{10} Inna Vesta Moore places \$7,800 of her savings into a savings account that earns 0.57\% annual interest, compounded monthly. 
	\begin{enumerate}[(a)]
	\item How much will be in the account after 9~years?
	\item How much interest has the account earned after 9~years?
	\item How long would it take the account to have \$50,000?
	\end{enumerate} \pspace

\sol This savings account compounds interest discretely. We know that the annual interest rate is $r= 0.0057$ and that the interest is compounded $k= 12$~times per year. We know that Inna has invested $P= \$7,\!800$. 
\begin{enumerate}[(a)]
\item The amount in the account after $t= 9$~years, $F$, is\dots
	\[
	F= P \left(1 + \dfrac{r}{k} \right)^{kt}= \$7,\!800 \left(1 + \dfrac{0.0057}{12} \right)^{12 \cdot 9}= \$7,\!800 (1.000475)^{108}= \$7,\!800 (1.05262582)= \$8,\!210.48
	\] 
Therefore, the account will have \$8,210.48 after 9~years. \pspace

\item The only money in the account is money Inna placed in the account (the principal, $P$) and interest earned. Therefore, we must have\dots
	\[
	\text{Interest Earned}= \$8,\!210.48 - \$7,\!800= \$410.48
	\]
Therefore, Inna has earned \$410.48 in interest after 9~years. \pspace

\item We know that the time (in years), $t$, it takes a principal $P$ to increase to a future value $F$ at an annual interest rate $r$, compounded $k$~times per year is given by $t= \frac{\ln(F/P)}{k \ln(1 + \frac{r}{k})}$. But then\dots
	\[
	t= \dfrac{\ln(F/P)}{k \ln(1 + \frac{r}{k})}= \dfrac{\ln(\$50,\!000/\$7,\!800)}{12 \ln(1 + \frac{0.0057}{12})}= \dfrac{\ln(6.410256410)}{12 \ln(1.000475)}= \dfrac{1.8578993}{0.00569865} \approx 326.024
	\]
Therefore, it will take 326.02~years for the account to have \$50,000. 
\end{enumerate}



\newpage



% Problem 3
\problem{10} Annita needs a loan. After discussing with a loan officer, she is offered a loan with an annual interest rate of 11.43\%, compounded monthly. 
	\begin{enumerate}[(a)]
	\item What is the nominal interest rate?
	\item What is the effective interest rate for this loan?
	\item If Annita will take out the loan for 4~years and will not be able to pay back more than \$11,000, what is the most she can take out now? That is, what is the amount that she could borrow now so that after 4~years, she would owe \$11,000?
	\end{enumerate} \pspace

\sol 
\begin{enumerate}[(a)]
\item The nominal interest rate is the advertised interest rate, which is 11.43\%. \pspace

\item For discrete compounded interest, we know the effective annual interest rate is given by $(1 + \frac{r}{k})^k - 1$, where $r$ is the nominal interest rate and $k$ is the number of interest compounds per year. But then\dots
	\[
	r_{\text{eff}}= \left(1 + \dfrac{r}{k} \right)^k - 1= \left(1 + \dfrac{0.1143}{12} \right)^{12} - 1= (1.009525)^{12} - 1= 1.120482144 - 1= 0.120482144
	\]
Therefore, the effective interest rate is 12.05\%. \pspace

\item The initial investment (principal), $P$, required to grow to a future value, $F$, after $t$~years at an annual interest rate $r$, compounded $k$~times per year is given by $P= \frac{F}{\left(1 + \frac{r}{k} \right)^{kt}}$. But then\dots
	\[
	P= \dfrac{F}{\left(1 + \frac{r}{k} \right)^{kt}}= \dfrac{\$11,\!000}{\left(1 + \dfrac{0.1143}{12} \right)^{12 \cdot 4}}= \dfrac{\$11,\!000}{(1.009525)^{48}}= \dfrac{\$11,\!000}{1.576230621} \approx \$6,\!978.67
	\]
Therefore, the most Annita can borrow right now is \$6,978.67. 
\end{enumerate}


\end{document}