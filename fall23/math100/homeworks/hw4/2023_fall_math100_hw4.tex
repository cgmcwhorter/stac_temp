\documentclass[11pt,letterpaper]{article}
\usepackage[lmargin=1in,rmargin=1in,tmargin=1in,bmargin=1in]{geometry}
\usepackage{../style/homework}
\usepackage{../style/commands}
\setbool{quotetype}{true} % True: Side; False: Under
\setbool{hideans}{true} % Student: True; Instructor: False

% -------------------
% Content
% -------------------
\begin{document}

\homework{4: Due 09/18}{She was born in the '80s. She still uses her phone as a phone!}{Troy Barnes, Community}

% Problem 1
\problem{10} Compute the following:
	\begin{enumerate}[(a)]
	\item Find the missing leg of a right triangle with leg 7 and hypotenuse 34.
	\item The distance between the points $(-3, 8)$ and $(6,2)$. 
	\item The area of a parallelogram with base 44.3~ft and height 13.9~ft. 
	\item The volume of a sphere with diameter 0.86~in.
	\item The surface area of a building that is 300~ft long, 95~ft deep, and 18~ft tall. 
	\end{enumerate}



\newpage



% Problem 2
\problem{10} You are going to paint a barn silo. The silo is 14~ft across and 80~ft high and is approximately shaped like a cylinder. 
	\begin{enumerate}[(a)]
	\item What is the surface area of the barn silo?
	\item If the paint can costs \$19 per gallon and each can covers 350~square feet, how many cans will you need to complete this job? How much will the cost be?
	\item If you were to also paint the `rectangular shaped' barn next to the silo (180~ft long, 60~ft wide, and 40~ft tall), how many more paint cans would you need? [Assume that you will not paint the barn roof or floor.]
	\end{enumerate} 



\newpage



% Problem 3
\problem{10} Consider the region shown below:
	\[
	\begin{tikzpicture}
	\draw[line width=0.03cm] (7,0) -- (0,0) -- (0,5) -- (3,5) -- (3,2) -- (5,2) -- (5,3) -- (7,3) -- (7,2);
	\draw[line width=0.03cm] (7,0) arc(270:450:1);
	
	\node at (-0.3,2.5) {$50$};
	\node at (1.5,5.3) {$28$};
	\node at (3.3,3.5) {$32$};
	\node at (4.75,2.5) {$11$};
	\node at (4.1,1.7) {$16$};
	\node at (6,3.2) {$18$};
	\end{tikzpicture}
	\]

\begin{enumerate}[(a)]
\item Find the perimeter of the region. 
\item Find the area of the region. 
\item If the region were actually the shape of a building, as viewed from above, assuming that the building is `regularly shaped', 38~ft high, and all the measurements in the figure shown above were in feet, what is the volume of the building?
\end{enumerate}


\end{document}