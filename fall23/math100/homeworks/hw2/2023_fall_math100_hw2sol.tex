\documentclass[11pt,letterpaper]{article}
\usepackage[lmargin=1in,rmargin=1in,tmargin=1in,bmargin=1in]{geometry}
\usepackage{../style/homework}
\usepackage{../style/commands}
\setbool{quotetype}{true} % True: Side; False: Under
\setbool{hideans}{false} % Student: True; Instructor: False

% -------------------
% Content
% -------------------
\begin{document}

\homework{2: Due 09/13}{I wasn't a failed DJ. I was pre-successful.}{Jason Mendoza, The Good Place}

% Problem 1
\problem{10} Your history class course grade is determined by the following components: \par
	\begin{table}[h]
	\centering
	\begin{tabular}{lr}
	Homeworks & 25\% \\
	Paper I & 10\% \\
	Paper II & 10\% \\
	Paper III & 10\% \\
	Project & 30\% \\
	Quizzes & 15\%
	\end{tabular}
	\end{table} \par
Suppose that your homework average was 82\%, your grades on the papers were 68\%, 84\%, and 76\%, respectively, your project grade was 88\%, and your quiz average was 94\%. 
	\begin{enumerate}[(a)]
	\item Compute your course average.
	\item If the project was a final project that had not yet occurred, i.e. you had not yet received the 88\%, but all the other course grades were as listed above, then what is your current course average?
	\end{enumerate} 

{\small
\sol 
\begin{enumerate}[(a)]
\item The course average will be a weighted average where each course component grade is weighted by the amount of the course grade that component is worth. But then we have\dots
	\[
	\begin{aligned}
	\text{Course Grade}=& \sum \text{Course Worth} \cdot \text{Component Grade} \\
	&= 0.25 (82\%) + 0.10 (68\%) + 0.10 (84\%) + 0.10 (76\%) + 0.30 (88\%) + 0.15 (94\%) \\
	&= 20.5\% + 6.8\% + 8.4\% + 7.6\% + 26.4\% + 14.1\% \\
	&= 83.8\%
	\end{aligned}
	\]

\item We can compute the percentage earned towards the course average thus far the same way we computed the course average in (a):
	\[
	\begin{aligned}
	\text{Course Grade}=& \sum \text{Course Worth} \cdot \text{Component Grade} \\
	&= 0.25 (82\%) + 0.10 (68\%) + 0.10 (84\%) + 0.10 (76\%) + 0.15 (94\%) \\
	&= 20.5\% + 6.8\% + 8.4\% + 7.6\% + 14.1\% \\
	&= 57.4\%
	\end{aligned}
	\]
The percentage of the course grade counted thus far is $25\% + 10\% + 10\% + 10\% + 15\%= 70\%$. But then the work above shows that the student has earned 57.4\% from the 70\% of the course grade counted thus far. But then the students course average is\dots
	\[
	\text{Course Average}= \dfrac{\text{Percentage Earned}}{\text{Percentage Counted}}= \dfrac{0.574}{0.70}= 0.82 \squiggle 82\%
	\]
{\itshape Note. This was implicit in (a), where you have earned 83.8\% of the 100\% of the course grade counted thus far: $\text{Course Average}= \frac{\text{Percentage Earned}}{\text{Percentage Counted}}= \dfrac{0.838}{1.00}= 0.838 \squiggle 83.8\%$. Note also it is also possible to compute the average directly by treating this as a `true' weighted average: $\frac{\sum \text{weight} \cdot \text{value}}{\sum \text{weight}}$, which in our case would yield $\frac{57.4\%}{70\%}= 0.82 \squiggle 82\%$.}
\end{enumerate}
}



\newpage



% Problem 2
\problem{10} Suppose you received the following grades this semester: \par
	\begin{table}[h]
	\centering
	\begin{tabular}{lrc}
	Course & Credits & Grade \\ \hline
	BIO 101: Human Biology & 3 & B \\
	MATH 104: Precalculus & 4 & B+ \\
	RELS 102: Religion and Society & 3 & A-- \\
	ENG 207: Writing about World Mythology & 3 & B-- \\
	ECON 102: Principles of Microeconomics & 3 & C+
	\end{tabular}
	\end{table} \par
Given the following grade values, 
	\begin{table}[h]
	\centering
	\begin{tabular}{lrclr}
	Grade & Values & & Grade & Values \\ \hline
	A & 4.0 & \hspace{1cm} & C+ & 2.3 \\
	A-- & 3.7 & & C & 2.0 \\
	B+ & 3.3 & & C-- & 1.7 \\
	B & 3.0 & & D & 1.0 \\
	B-- & 2.7 & & F & 0
	\end{tabular}
	\end{table}

\begin{enumerate}[(a)]
\item Compute your semester GPA. 
\item If your previous GPA based on 67 credits was 3.308, what is your current overall GPA?
\end{enumerate} \pspace

\sol 
\begin{enumerate}[(a)]
\item GPA is a weighted average where each letter grade earned is weighted by the number of credits for the corresponding course. But then we have\dots
	\[
	\begin{aligned}
	\text{GPA}&= \dfrac{\sum \text{Credits} \cdot \text{Letter Grade}}{\sum \text{Credits}} \\
	&= \dfrac{3(3.0) + 4(3.3) + 3(3.7) + 3(2.7) + 3(2.3)}{3 + 4 + 3 + 3 + 3} \\
	&= \dfrac{9 + 13.2 + 11.1 + 8.1 + 6.9}{3 + 4 + 3 + 3 + 3} \\
	&= \dfrac{48.3}{16} \\
	&\approx 3.019
	\end{aligned}
	\]

\item The new GPA will be the weighted average of the previous GPA with the current semester GPA, where each is weighted by the number of credits. Therefore, we have\dots
	\[
	\begin{aligned}
	\text{New GPA}&= \dfrac{\text{Previous Credits} \cdot \text{Previous GPA} + \text{Current Credits} \cdot \text{Current GPA}}{\text{Total Credits}} \\
	&= \dfrac{67 \cdot 3.308 + 16 \cdot 3.019}{67 + 16} \\
	&= \dfrac{221.636 + 48.304}{67 + 16} \\
	&= \dfrac{269.94}{83} \\
	&\approx 3.252
	\end{aligned}
	\]
\end{enumerate}



\newpage



% Problem 3
\problem{10} Showing all your work, convert the following:
	\begin{enumerate}[(a)]
	\item 0.4 megagrams to centigrams
	\item 87~oz to stones [1~oz = 28.35~g, 1~stone = 6.35~kg]
	\item \$98 per hour to MXP per minute [\$0.057 = 1 MXP]
	\item 6,000~ft$^2$ to mi$^2$ [5280~ft = 1~mi.]
	\item 0.05 meters per square second to feet per square hour [0.3048~m = 1~ft]
	\end{enumerate} \pspace

\sol 
\begin{enumerate}[(a)]
\item We have\dots \par
	\begin{table}[H]
	\centering
	\begin{tabular}{c||c|c}
	0.4~Mg & 1000000~g & 100~cg \\ \hline
		    & 1~Mg         & 1~g
	\end{tabular}\;= 40,000,000~cg
	\end{table} \pspace

\item We have\dots \par
	\begin{table}[H]
	\centering
	\begin{tabular}{c||c|c|c}
	87~oz & 28.35~g & 1~kg	   & 1~stone \\ \hline
		  & 1~oz	    & 1000~g & 66.35~kg
	\end{tabular}\;= 0.0371733233~stone
	\end{table} \pspace

\item We have\dots \par
	\begin{table}[H]
	\centering
	\begin{tabular}{c||c|c}
	\$98 & 1~MXP & 1~hr \\ \hline
	1~hr	& \$0.057 & 60~min
	\end{tabular}\;= \$28.655 MXP per minute
	\end{table} \pspace
	
\item We have\dots \par
	\begin{table}[H]
	\centering
	\begin{tabular}{c||c|c}
	6000~ft$^2$ & 1~mi      & 1~mi \\ \hline
			    & 5280~ft  & 5280~ft
	\end{tabular}\;= 0.00021522~mi$^2$
	\end{table} \pspace

\item We have\dots \par
	\begin{table}[H]
	\centering
	\begin{tabular}{c||c|c|c|c|c}
	0.05~m & 1~ft		& 60~s & 60~s	     & 60~min & 60~min \\ \hline
	1~s$^2$ & 0.3048~m & 1~min & 1~min & 1~hr & 1~hr
	\end{tabular}\;$\approx$ 2,125,984.25~ft/hr$^2$
	\end{table}
\end{enumerate}



\newpage



% Problem 4
\problem{10} Suppose you are talking with your friend who has moved to Germany. The conversation has drifted to NYC housing. Currently, the cost of space in NYC is approximately \$1,600 per square foot. 
	\begin{enumerate}[(a)]
	\item For your friend, convert this to Euros per square meter. [\texteuro 1= \$1.07; 1~ft = 0.3048~m]
	\item Using (a), find the conversion factor from dollars per square foot to Euros per square meter. 
	\item Use your answer from (b) to convert \$1,845 per square foot to Euros per square meter. 
	\end{enumerate} \pspace

\sol 
\begin{enumerate}[(a)]
\item We have\dots \par
	\begin{table}[H]
	\centering
	\begin{tabular}{c||c|c|c}
	\$1600	& \texteuro 1	& 1~ft		& 1~ft \\ \hline
	1~ft$^2$   & \$1.07 		& 0.3048~m     & 0.3048~m
	\end{tabular}\;= \texteuro 16,095.57 per square meter
	\end{table} \pspace

\item From the work above, we can see that the conversion factor is\dots
	\[
	\dfrac{1}{1.07} \cdot \dfrac{1}{0.3048} \cdot \dfrac{1}{0.3048}= 10.059729361
	\] \pspace

\item We have\dots 
	\[
	\$1845 \text{ per square foot } \cdot 10.059729361 \approx \text{\texteuro }18,560.20 \text{ per square meter}
	\] \pspace
We can also check this directly: \par \pspace
	\begin{table}[H]
	\centering
	\begin{tabular}{c||c|c|c}
	\$1845	& \texteuro 1	& 1~ft		& 1~ft \\ \hline
	1~ft$^2$   & \$1.07 		& 0.3048~m     & 0.3048~m
	\end{tabular}\;= \texteuro 18,560.2 per square meter
	\end{table} \pspace
\end{enumerate}


\end{document}