\documentclass[11pt,letterpaper]{article}
\usepackage[lmargin=1in,rmargin=1in,tmargin=1in,bmargin=1in]{geometry}
\usepackage{../style/homework}
\usepackage{../style/commands}
\setbool{quotetype}{true} % True: Side; False: Under
\setbool{hideans}{false} % Student: True; Instructor: False

% -------------------
% Content
% -------------------
\begin{document}

\homework{1: Due 09/11}{Everything hurts and I'm dying.}{Leslie Knope, Parks and Recreation}

% Problem 1
\problem{10} Compute the following:
	\begin{enumerate}[(a)]
	\item 47\% of 120
	\item 96\% of 55
	\item 1\% of 870
	\item 130\% of 94
	\item 285\% of 16
	\end{enumerate} \pspace

\sol We use the fact that to compute a \% of some number $N$, we need only compute $N \cdot \%_d$, where $\%_d$ is the percentage written as a decimal. \pspace

\begin{enumerate}[(a)]
\item 
	\[
	\text{47\% of 120}= 120(0.47)= 56.4
	\] \pspace

\item 
	\[
	\text{96\% of 55}= 55(0.96)= 52.8
	\] \pspace

\item 
	\[
	\text{1\% of 870}= 870(0.01)= 8.7
	\] \pspace

\item 
	\[
	\text{130\% of 94}= 94(1.30)= 122.2
	\] \pspace

\item 
	\[
	\text{285\% of 16}= 16(2.85)= 45.6
	\] 
\end{enumerate}



\newpage



% Problem 2
\problem{10} Compute the following:
	\begin{enumerate}[(a)]
	\item 96 increased by 12\%
	\item 620 decreased by 24\%
	\item 45 increased by 4\% 
	\item 59 increased by 108\%
	\item 200 increased by 160\%
	\end{enumerate} \pspace

\sol We use the fact that if we want to compute $N$ increased or decreased by a \%, we compute $N \cdot (1 \pm \%_d)$, where $\%_d$ is the percentage written as a decimal and we choose `$+$' if it is a percentage increase and choose `$-$' if it is a percentage decrease. \pspace

\begin{enumerate}[(a)]
\item 
	\[
	\text{96 increased by 12\%}= 96(1 + 0.12)= 96(1.12)= 107.52
	\] \pspace

\item 
	\[
	\text{620 decreased by 24\%}= 620(1 - 0.24)= 620(0.76)= 471.2
	\] \pspace

\item 
	\[
	\text{45 increased by 4\%}= 45(1 + 0.04)= 45(1.04)= 46.8
	\] \pspace

\item 
	\[
	\text{59 increased by 108\%}= 59(1 + 1.08)= 59(2.08)= 122.72
	\] \pspace

\item 
	\[
	\text{200 increased by 160\%}= 200(1 + 1.60)= 200(2.60)= 520
	\] 
\end{enumerate}



\newpage



% Problem 3
\problem{10} You are out to dinner with your friends. The check comes and you all have decided that you will pay and they will `Venmo you.' The bill comes and it is \$187.45. You will tip the server 20\%. 
	\begin{enumerate}[(a)]
	\item What is the tip amount?
	\item What is the total bill?
	\item If you are dividing the cost equally amongst \textit{you and your four friends}, what is the amount each person should pay you?
	\end{enumerate} \pspace

\sol 
\begin{enumerate}[(a)]
\item We need to compute 20\% of \$187.45. We use the fact that to compute a \% of some number $N$, we need only compute $N \cdot \%_d$, where $\%_d$ is the percentage written as a decimal. So the tip amount is\dots
	\[
	\$187.45(0.20)= \$37.49
	\] \pspace

\item We can add the tip amount we found in (a) to the total bill. So the total bill is $\$187.45 + \$37.49= \$224.94$. \pspace

Alternatively, we need to increase the bill by 20\%. We use the fact that if we want to compute $N$ increased or decreased by a \%, we compute $N \cdot (1 \pm \%_d)$, where $\%_d$ is the percentage written as a decimal and we choose `$+$' if it is a percentage increase and choose `$-$' if it is a percentage decrease. So the total bill is\dots
	\[
	\$187.45(1 + 0.20)= \$187.45(1.20)= \$224.94
	\] \pspace

\item From (b), we know that the total bill is \$224.94. We are then dividing the bill amongst five people---yourself and your four friends. Therefore, each person should pay\dots
	\[
	\dfrac{\text{Total Bill}}{\text{Total People}}= \dfrac{\$224.94}{5}= \$44.988 \approx \$44.99
	\]
\end{enumerate}



\newpage



% Problem 4
\problem{10} Suppose that Dahlia sells above ground pools. Due to inflation, she plans on raising her prices slowly. The cost of one particular above ground pool is \$3,800. Over the next three months, she will raise the price 10\%. 
	\begin{enumerate}[(a)]
	\item Calculate the price of the pool for each of the next three months. 
	\item Using your answer from (a), find by what percentage the pool price has increased. 
	\item What if she had simply raised the price of the pool by 30\%, what would the cost be?
	\item Are your final answers from (a) and (c) the same? Explain. 
	\item By what percentage should she raise the price of the pool each month if she wants the final cost to be \$4,000?
	\end{enumerate} \pspace

{\small
\sol We shall repeatedly use the fact that to compute a \% of some number $N$, we need only compute $N \cdot \%_d$, and if we want to compute $N$ increased or decreased by a \%, we compute $N \cdot (1 \pm \%_d)$, where $\%_d$ is the percentage written as a decimal and we choose `$+$' if it is a percentage increase and choose `$-$' if it is a percentage decrease. \pspace

\begin{enumerate}[(a)]
\item Currently, i.e. at month 0, the price is \$3,800. The price next month is 10\% higher, so that the price will be $\$3800(1 + 0.10)= \$3800(1.10)= \$4180$. The following month is again 10\% higher, so that the price will be $\$4180(1 + 0.10)= \$4180(1.10)= \$4598$. Finally, the third month, the price will increase by 10\%, so that the price will be $\$4598(1 + 0.10)= \$4598(1.10)= \$5057.80$. Of course, if we simply wanted the price after three months of a 10\% price increase, this would be $\$3800(1 + 0.10)^3= \$3800(1.10)^3= \$3800(1.331)= \$5057.80$. \par
	\begin{table}[h]
	\centering
	\begin{tabular}{c||cccc}
	Month & 0 & 1 & 2 & 3 \\ \hline
	Price & \$3,800 & \$4,180 & \$4,598 & \$5,057.80 
	\end{tabular}
	\end{table} \par

\item We know from (a) that the final price is \$5057.80 with initial price \$3800. But then the percent increase is\dots
	\[
	\dfrac{\text{final price} - \text{initial price}}{\text{initial price}}= \dfrac{\$5057.80 - \$3800}{\$3800}= \dfrac{\$1257.80}{\$3800}= 0.331
	\]
Therefore, the final price increase is 33.1\%. Another approach is to recognize the final price will be $\$3800(1 + 0.10)^3= \$3800(1.10)^3= \$3800(1.331)= \$3800(1 + 0.331)$, which we can recognize as representing a 33.1\% increase. As yet another approach, we can solve the equation $\$3800(1 + \%_d)= \$5057.80$ for $\%_d$ and find that $\%_d= 0.331$, i.e. the final price is a 33.1\% increase. 

\item The price would then simply be $\$3800(1 + 0.30)= \$3800(1.30)= \$4940$. 

\item The answers are \textit{not} the same. Percentages are not additive; they are multiplicative. Therefore, a price increase of 10\% three times will not be the same as $3 \cdot 10\%= 30\%$ increase. 

\item Let $x$ be the percentage written as the decimal. Then we know\dots
	\[
	\begin{aligned}
	\text{initial price} (1 + \%_d)^n&= \text{final price} \\
	\$3800 (1 + x)^3&= \$4000 \\
	(1 + x)^3&= 1.0526315789 \\
	\sqrt[3]{(1 + x)^3}&= \sqrt[3]{1.0526315789} \\
	1 + x&= 1.017245 \\
	x&= 0.017245
	\end{aligned}
	\]
Therefore, she should raise the price by 1.7245\% each month. We can check: $\$3800(1 + 0.017245)^3= \$3800(1.017245)^3= \$3800(1.05263) \approx \$4000$. 
\end{enumerate}
}


\end{document}