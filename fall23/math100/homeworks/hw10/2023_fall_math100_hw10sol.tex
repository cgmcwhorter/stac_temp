\documentclass[11pt,letterpaper]{article}
\usepackage[lmargin=1in,rmargin=1in,tmargin=1in,bmargin=1in]{geometry}
\usepackage{../style/homework}
\usepackage{../style/commands}
\setbool{quotetype}{true} % True: Side; False: Under
\setbool{hideans}{false} % Student: True; Instructor: False

% -------------------
% Content
% -------------------
\begin{document}

\homework{10: Due 10/30}{The superior man understands what is right; the inferior man understands what will sell.}{Confucius}

% Problem 1
\problem{10} Lee H. is taking out a loan to help expand his liquor store \textit{Tequila Mockingbird}. Lee decides to borrow \$60,000 at 12.6\% annual interest, compounded monthly. The loan will be repaid with equal end of the month payments over a period of 3~years. 
	\begin{enumerate}[(a)]
	\item What will Lee's monthly payment be?
	\item How much does Lee pay in total for this loan?
	\item How much does Lee pay in interest for this loan?
	\end{enumerate} \pspace

\sol Because Lee will make require monthly payments, this is an annuity. 
\begin{enumerate}[(a)]
\item For a simple annuity immediate, we know that the monthly payments, $R$, are given by $R= \dfrac{P}{\dfrac{1 - (1 + i_p)^{-\text{PM}}}{i_p}}$, where $i_p$ is the interest per period, $P$ is the loan amount, and PM is the number of payments. We know that Lee will make monthly payments for 3~years---a total of $\text{PM}= 12 \cdot 3= 36$~payments. We know that the interest per period is $i_p= \frac{r}{k}$, where $r$ is the nominal interest rate and $k$ is the number of interest compounds per year. But then $i_p= \frac{0.126}{12}= 0.0105$. We know the loan amount is $P= \$60,\!000$. Therefore, Lee's monthly payment is\dots
	\[
	\hspace{-3cm} R= \dfrac{P}{\dfrac{1 - (1 + i_p)^{-\text{PM}}}{i_p}}= \dfrac{\$60,\!000}{\dfrac{1 - (1 + 0.0105)^{-36}}{0.0105}}= \dfrac{\$60,\!000}{\dfrac{1 - (1.0105)^{-36}}{0.0105}}= \dfrac{\$60,\!000}{\dfrac{1 - 0.68658223}{0.0105}}= \dfrac{\$60,\!000}{\dfrac{0.31341777}{0.0105}}= \dfrac{\$60,\!000}{29.849311}= \$2,\!010.10
	\] \pspace

\item Lee makes 36 monthly payments of \$2,010.10. Therefore, Lee pays a total of $36 \cdot \$2,\!010.10= \$72,\!363.60$ on this loan. \pspace

\item Lee only pays back the original loan amount, \$60,000, and any interest on the loan. Because Lee pays a total of \$72,363.60 on this loan, he must pay $\$72,\!363.60 - \$60,\!000= \$12,\!363.60$ in interest on this loan. 
\end{enumerate}



\newpage



% Problem 2
\problem{10} A product has cost function $C(q)= 12.67q + 16200$ and revenue function $R(q)= 29.99q$. 
	\begin{enumerate}[(a)]
	\item What are the fixed costs? 
	\item How much does it cost to produce each product? How much does each product sell for?
	\item Find the break-even point.
	\item What is the minimum number of items that must be made/sold in order to make a profit?
	\end{enumerate} \pspace

\sol 
\begin{enumerate}[(a)]
\item We know that the fixed costs are the costs incurred in production regardless of the amount produced. But then the fixed costs are the costs even when nothing is produced. These costs are $C(0)= 12.67(0) + 16200= 0 + 16200= 16200$. Therefore, the fixed costs are \$16,200. \pspace

\item Because the cost and revenue functions are linear functions, the cost to produce each product and the amount each product sells for is the rate of change of these functions, i.e. the slope. The slope of $C(q)$ is 12.67 and the slope of $R(q)$ is 29.99. Therefore, it costs \$12.67 to produce each product and each product sells for \$29.99. \pspace

\item The break-even point is the production/sale level at which the total cost is the total revenue. But then\dots
	\[
	\begin{gathered}
	C(q)= R(q) \\
	12.67q + 16200= 29.99q \\
	16200= 17.32q \\
	q= \dfrac{16200}{17.32} \\
	q \approx  935.335
	\end{gathered}
	\]
Therefore, the break-even point occurs at a production/sales level of 935.335. \pspace

\item The minimum number of items that must be made/sold to make a profit is 936~items. 
\end{enumerate}



\newpage




% Problem 3
\problem{10} You rent a small studio apartment in NYC for \$3,380 per month to produce social media content. Between hiring actors, purchasing props, travel costs, etc., it costs approximately \$510 to produce a video. However, each video typically makes \$870 in ad revenue and sponsorship deals. Let $C(v)$ and $R(v)$ denote the cost and revenue function to produce $v$~videos. 
	\begin{enumerate}[(a)]
	\item Explain why $C(v)$ and $R(v)$ are approximately linear. 
	\item Find $C(v)$ and $R(v)$. 
	\item What is the minimum number of videos you have to produce to make a profit each month? 
	\end{enumerate} \pspace

\sol 
\begin{enumerate}[(a)]
\item The cost function, $C(v)$, is linear because it costs a constant amount of \$510 to produce each video. The revenue function, $R(v)$, is linear because you make a constant amount of \$870 per video. \pspace

\item Because it costs \$510 to produce each video, the cost incurred from video-making is $510v$, where $v$ is the number of videos. However, you must still rent the studio for \$3,380 per month. Therefore, the total costs to produce $v$~videos each month is $C(v)= 510v + 3380$. Because you only make money from the videos and each video makes \$870, the revenue from making $v$ videos is $R(v)= 870v$. \pspace

\item To find the amount of videos we must make to turn a profit each month, we first find the break-even point, i.e. the number of videos such that $C(v)= R(v)$:
	\[
	\begin{gathered}
	C(v)= R(v) \\
	510v + 3380= 870v \\
	3380= 360v \\
	v= \dfrac{3380}{360} \\
	v \approx 9.38889
	\end{gathered}
	\]
Therefore, you must make at least 10~videos each month to make a profit. 
\end{enumerate}


\end{document}