\documentclass[11pt,letterpaper]{article}
\usepackage[lmargin=1in,rmargin=1in,tmargin=1in,bmargin=1in]{geometry}
\usepackage{../style/homework}
\usepackage{../style/commands}
\setbool{quotetype}{true} % True: Side; False: Under
\setbool{hideans}{true} % Student: True; Instructor: False

% -------------------
% Content
% -------------------
\begin{document}

\homework{10: Due 10/30}{The superior man understands what is right; the inferior man understands what will sell.}{Confucius}

% Problem 1
\problem{10} Lee H. is taking out a loan to help expand his liquor store \textit{Tequila Mockingbird}. Lee decides to borrow \$60,000 at 12.6\% annual interest, compounded monthly. The loan will be repaid with equal end of the month payments over a period of 3~years. 
	\begin{enumerate}[(a)]
	\item What will Lee's monthly payment be?
	\item How much does Lee pay in total for this loan?
	\item How much does Lee pay in interest for this loan?
	\end{enumerate}



\newpage



% Problem 2
\problem{10} A product has cost function $C(q)= 12.67q + 16200$ and revenue function $R(q)= 29.99q$. 
	\begin{enumerate}[(a)]
	\item What are the fixed costs? 
	\item How much does it cost to produce each product? How much does each product sell for?
	\item Find the break-even point.
	\item What is the minimum number of items that must be made/sold in order to make a profit?
	\end{enumerate}



\newpage




% Problem 3
\problem{10} You rent a small studio apartment in NYC for \$3,380 per month to produce social media content. Between hiring actors, purchasing props, travel costs, etc., it costs approximately \$510 to produce a video. However, each video typically makes \$870 in ad revenue and sponsorship deals. Let $C(v)$ and $R(v)$ denote the cost and revenue function to produce $v$~videos. 
	\begin{enumerate}[(a)]
	\item Explain why $C(v)$ and $R(v)$ are approximately linear. 
	\item Find $C(v)$ and $R(v)$. 
	\item What is the minimum number of videos you have to produce to make a profit each month? 
	\end{enumerate}


\end{document}