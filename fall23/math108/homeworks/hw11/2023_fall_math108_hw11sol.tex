\documentclass[11pt,letterpaper]{article}
\usepackage[lmargin=1in,rmargin=1in,tmargin=1in,bmargin=1in]{geometry}
\usepackage{../style/homework}
\usepackage{../style/commands}
\setbool{quotetype}{false} % True: Side; False: Under
\setbool{hideans}{false} % Student: True; Instructor: False

% -------------------
% Content
% -------------------
\begin{document}

\homework{11: Due 11/02}{The 50-50-90 rule: anytime you have a 50-50 change of getting something right, there's a 90\% probability you'll get it wrong.}{Andy Rooney}

% Problem 1
\problem{10} News reporting suggests that 30\% of local residents support a tax increase to fund local beautification improvements. You go out into the community to survey individuals to confirm this reporting. Walking up to random individuals on the street, you ask whether they support the tax increase. 
	\begin{enumerate}[(a)]
	\item Discuss whether the number of people supporting the tax increase is given by a binomial distribution. 
	\item Assuming you survey 12 people and that the number of people supporting the tax increase is given by a binomial distribution and that the news reporting is correct, compute\dots
		\begin{enumerate}[(i)]
		\item Exactly 7 people support the tax increase.
		\item At least 3 people support the tax increase. 
		\item More than 7 people support the tax increase. 
		\item Less than 7 people support the tax increase.
		\end{enumerate}
	\end{enumerate} \pspace

\sol 
\begin{enumerate}[(a)]
\item A binomial distribution is a count $X$ of an event such that: (i) the event either occurs or not with each observation, (ii) the probability of observing the event is constant, (iii) the number of trials is fixed, and (iv) the observations are independent. We check each condition individually: (i) Because you are asking if they support or not, there are only two possible outcomes. (ii) You will ask a fixed number of people by the end of the survey. (iii) The probability of a resident supporting or not supporting the tax is not likely constant across the individuals. (iv) The sample individuals are not likely to be independent---especially if they know each other---because they are all being taken by yourself in a small area. Given all these observations on the binomial criterion, this sampling is not likely to precisely follow a binomial distribution. 

\item Ignoring the issues in (a) and proceeding with a binomial computation using the reported 30\% support and a sample size of 12, we have binomial distribution $B(n, p)= B(12, 0.30)$. But then\dots
	\begin{enumerate}[(i)]
	\item $P(X= 7)= 0.0291$
	\item $P(X \geq 3)= 1 - P(X= 2) - P(X= 1) - P(X= 0)= 1 - 0.1678 - 0.0712 - 0.0138= 0.7472$
	\item $P(X > 7)= P(X= 8) + P(X= 9) + P(X= 10) + P(X= 11) + P(X= 12)= 0.0078 + 0.0015 + 0.0002 + 0 + 0= 0.0095$
	\item $P(X < 7)= 1 - P(X= 7) - P(X > 7)= 1 - 0.0291 - 0.0095= 0.9614$
	\end{enumerate}
\end{enumerate}



\newpage



% Problem 2
\problem{10} Suppose that surveys suggest that 85\% of individuals have taken an Uber at some point. You take a simple random sample of eight individuals. Find the probability that\dots
	\begin{enumerate}[(a)]
	\item At least one individual had taken an Uber. 
	\item Exactly five individuals had taken an Uber. 
	\item More than four individuals had taken an Uber. 
	\item At most three individuals had taken an Uber. 
	\end{enumerate} \pspace

\sol You have a fixed number of trials---namely, $n= 8$. Each individual has either taken an Uber or not. Assuming that the probability of the individuals of having taken an Uber is fixed across the individuals and the samples are independent, the count of those that have taken an Uber, $X$, is given by the binomial distribution $B(n, p)= B(8, 0.85)$. Let $\overline{X}$ be the count of those that have \textit{not} taken an Uber. We have $\overline{X} \sim B(n, p)= B(8, 0.15)$. 

\begin{enumerate}[(a)]
\item 
	\[
	\begin{gathered}
	P(X \geq 1)= 1 - P(X= 0)= 1 - 0= 1 \\[0.3cm]
	\textit{ OR } \\[0.3cm]
	P(\overline{X} \leq 7)= 1 - P(\overline{X}= 8)= 1 - 0= 1
	\end{gathered}
	\] \pspace

\item 
	\[
	\begin{gathered}
	P(X= 5)= 0.0839 \\[0.3cm]
	\textit{ OR } \\[0.3cm]
	P(\overline{X}= 3)= 0.0839
	\end{gathered}
	\] \pspace

\item 
	\[
	\begin{gathered}
	\hspace{-2.8cm} P(X > 4)= P(X \geq 5)= P(X= 5) + P(X= 6) + P(X= 7) + P(X= 8)= 0.0839 + 0.2376 + 0.3847 + 0.2725= 0.9787 \\[0.3cm]
	\textit{ OR } \\[0.3cm]
	\hspace{-2.5cm} P(\overline{X} \leq 3)= P(\overline{X}= 3) + P(\overline{X}= 2) + P(\overline{X}= 1) + P(\overline{X}= 0)= 0.0839 + 0.2376 + 0.3847 + 0.2725= 0.9787
	\end{gathered}
	\] \pspace

\item 
	\[
	\begin{gathered}
	P(X \leq 3)= P(X= 0) + P(X= 1) + P(X= 2) + P(X= 3)= 0 + 0 + 0.0002 + 0.0026= 0.0028 \\[0.3cm]
	\textit{ OR } \\[0.3cm]
	P(\overline{X} \geq 5)= P(\overline{X}= 5) + P(\overline{X}= 6) + P(\overline{X}= 7) + P(\overline{X}= 8)= 0 + 0 + 0.0002 + 0.0026= 0.0028
	\end{gathered}
	\]
\end{enumerate}


\end{document}