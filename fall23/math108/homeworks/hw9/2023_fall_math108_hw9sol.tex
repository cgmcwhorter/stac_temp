\documentclass[11pt,letterpaper]{article}
\usepackage[lmargin=1in,rmargin=1in,tmargin=1in,bmargin=1in]{geometry}
\usepackage{../style/homework}
\usepackage{../style/commands}
\setbool{quotetype}{true} % True: Side; False: Under
\setbool{hideans}{false} % Student: True; Instructor: False

% -------------------
% Content
% -------------------
\begin{document}

\homework{9: Due 10/24}{I like to play blackjack. I'm not addicted to gambling. I'm addicted to sitting in a semicircle.}{Mitch Hedberg}

% Problem 1
\problem{10} Suppose you play a game where you roll a loaded die. The probabilities for this die are (partially) given below. If you roll an even number, you win \$1. If you roll a one, you lose \$5. If you roll a three, you lose \$2. Finally, if you roll a five, you win/lose nothing. 
	\begin{table}[!ht]
	\centering 
	\begin{tabular}{|c||c|c|c|c|c|c|} \hline 
	$n$ & $1$ & $2$ & $3$ & $4$ & $5$ & $6$ \\ \hline 
	$P(n)$ & $\dfrac{\textit{1}}{\textit{12}}$ & $\dfrac{2}{12}$ & $\dfrac{3}{12}$ & $\dfrac{1}{12}$ & $\dfrac{1}{12}$ & $\dfrac{4\rule{0pt}{2.9ex}}{12\rule[-1.3ex]{0pt}{0pt}}$ \\ \hline
	\end{tabular}
	\end{table}

\begin{enumerate}[(a)]
\item Find $P(1)$. 
\item Find the probability that if you roll the die three times, you win \$1 each time. 
\item Find the average amount you win per game. 
\item Should you play this game? Explain.
\end{enumerate} 

\sol 
\begin{enumerate}[(a)]
\item We know the sum of the probabilities for all the possibilities must be 1. But then\dots
	\[
	\hspace{-1.5cm} P(n= 1)= 1 - P(n= 2) - P(n= 3) - P(n= 4) - P(n= 5) - P(n= 6)= 1 - \dfrac{2}{12} - \dfrac{3}{12} - \dfrac{1}{12} - \dfrac{1}{12} - \dfrac{4}{12}= \dfrac{1}{12} \approx 0.0833
	\] 

\item The only way one wins \$1 is by rolling an even number. We know the probability of rolling an even number is $P(\text{even})= P(n= 2) + P(n= 4) + P(n= 6)= \frac{2}{12} + \frac{1}{12} + \frac{4}{12}= \frac{7}{12}$. Because dice rolls are independent, this is\dots
	\[
	P(\$1 \text{three times})= P(\text{Even three times})= P(\text{Even}) P(\text{Even}) P(\text{Even})= \frac{7}{12} \cdot \frac{7}{12} \cdot \frac{7}{12}= \dfrac{343}{1728} \approx 0.1985
	\] \pspace

\item The amount you win on average is the expected value for this game. We construct the random variable, $X$, given by the win/loss amounts: \par
	\begin{table}[!ht]
	\centering 
	\begin{tabular}{|c||c|c|c|c|c|c|} \hline 
	$n$ & $1$ & $2$ & $3$ & $4$ & $5$ & $6$ \\ \hline 
	$P(n)$ & $\frac{1}{12}$ & $\frac{2}{12}$ & $\frac{3}{12}$ & $\frac{1}{12}$ & $\frac{1}{12}$ & $\frac{4\rule{0pt}{2.0ex}}{12\rule[-0.8ex]{0pt}{0pt}}$ \\ \hline
	$X$ & $-\$5$ & $\$1$ & $-\$2$ & $\$1$ & $\$0$ & $\$1$ \\ \hline
	\end{tabular}
	\end{table} \par
We know the expected value for a discrete random variable, $X$, is $EX= \sum X P(x= X)$. But then we have\dots
	\[
	EX= \sum X P(x= X)= -\$5 \cdot \dfrac{1}{12} + \$1 \cdot \dfrac{2}{12} + (-\$2) \cdot \dfrac{3}{12} + \$1 \cdot \dfrac{1}{12} + \$0 \cdot \dfrac{1}{12} + \$1 \cdot \dfrac{4}{12}= -\$ \tfrac{1}{3} \approx -\$0.33
	\]

\item Because the expected value, $EX \approx -\$0.33 < 0$, is negative, one loses money `in the long run' playing this game---even if one experiences initial wins. Therefore, one should not play this game for `long' periods of time. 
\end{enumerate}



\newpage



% Problem 2
\problem{10} Suppose you are designing a game to `reallocate' money from your friends to an account that you control\dots You will have them roll a four-sided dice---each side equally likely to occur. If they roll a four, neither of you wins money. If they roll a two or three, you will pay them \$2 or \$3, respectively. If they roll a one, they will flip a fair coin. If the coin is heads, they win/lose nothing. However, if the coins is tails, they will pay you some amount of money. 
	\begin{enumerate}[(a)]
	\item Find the amount your friend must pay you if they roll a one and then flip a tails so that you will not lose money at this game `in the long run.' 
	\item If your friend plays this game one-hundred times, are you guaranteed to make money? Explain. 
	\end{enumerate} \pspace

\sol 
\begin{enumerate}[(a)]
\item Let $n$ be the number one rolls and let the amount of money you pay your friend if they roll a one followed by flipping a tails $M$. Because each side is equally likely to occur, we know that $P(n= 1)= P(n= 2)= P(n= 3)= P(n= 4)= \frac{1}{4}$. For a coin flip, we know that $P(H)= P(T)= \frac{1}{2}$. Because the dice rolls and coin flips are independent, we know that $P(\text{one and heads})= P(n= 1) P(H)= \frac{1}{4} \cdot \frac{1}{2}= \frac{1}{8}$ and $P(\text{one and tails})= P(n= 1) P(T)= \frac{1}{4} \cdot \frac{1}{2}= \frac{1}{8}$. But then we can construct a table of the outcomes and their associated payouts---the random variable $X$. \par
	\begin{table}[!ht]
	\centering 
	\begin{tabular}{|c||c|c|c|c|c|} \hline 
	$n$ & $1$ \& H & $1$ \& T & $2$ & $3$ & $4$ \\ \hline 
	$P(n)$ & $\frac{1}{12}$ & $\frac{2}{12}$ & $\frac{3}{12}$ & $\frac{1}{12}$ & $\frac{4\rule{0pt}{2.0ex}}{12\rule[-0.8ex]{0pt}{0pt}}$ \\ \hline
	$X$ & $\$0$ & $M$ & $-\$2$ & $-\$3$ & $\$0$ \\ \hline
	\end{tabular}
	\end{table} \par
But then the expected value is\dots
	\[
	EX= \sum X P(x= X)= \$0 \cdot \dfrac{1}{12} + M \cdot \dfrac{2}{12} + (-\$2) \cdot \dfrac{3}{12} + (-\$3) \cdot \dfrac{1}{12} + \$0 \cdot \dfrac{4}{12}= \dfrac{M}{6} -\dfrac{3}{4}= \dfrac{2M - 9}{12}
	\]
If we want to not lose money `in the long run', we want the expected value to be positive, i.e. $EX > 0$. But then\dots
	\[
	\begin{gathered}
	EX > 0 \\
	\dfrac{2M - 9}{12} > 0 \\
	2M - 9 > 0 \\
	2M > 9 \\
	M > \frac{2}{9} \approx 0.22
	\end{gathered}
	\]
Therefore, to win money `in the long run' playing this game with your friend, you need to make the rule that they pay you any amount more than \$0.22 if they roll a one followed by flipping a tail. \pspace

\item No, you are not guaranteed to win money. It is possible that your friend rolls a 3 one-hundred times in a row! However, choosing the amount $M$ as in (a), we know that if one continues to play this game `sufficiently many' times, one will make a profit playing this game. 
\end{enumerate}


\end{document}