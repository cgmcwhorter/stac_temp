\documentclass[11pt,letterpaper]{article}
\usepackage[lmargin=1in,rmargin=1in,tmargin=1in,bmargin=1in]{geometry}
\usepackage{../style/homework}
\usepackage{../style/commands}
\setbool{quotetype}{true} % True: Side; False: Under
\setbool{hideans}{false} % Student: True; Instructor: False

% -------------------
% Content
% -------------------
\begin{document}

\homework{2: Due 09/12}{You only have to do a few things right in your life so long as you don’t do too many things wrong.}{Warren Buffett}

% Problem 1
\problem{10} Suppose that the revenue and cost function for a certain item are given by $R(q)= 45.99q$ and $C(q)= 11.13q + 576000$, respectively. 
	\begin{enumerate}[(a)]
	\item How much does the company sell each item for? How much does it cost to make each item?
	\item What are the fixed costs for the production of this good?
	\item What is the profit or loss if the company produces and sells ten-thousand of these items?
	\item What is the break-even point? At least many items does this company need to sell in order to make a profit on this item?
	\end{enumerate} \pspace

\sol 
\begin{enumerate}[(a)]
\item Observe that both $R(q)$ and $C(q)$ are linear. Because $R(q)$ is linear, the price of each item is the slope of the revenue function, $R(q)$. Therefore, each item sells for \$45.99. Because $C(q)$ is linear, the cost to make each item is the slope of the cost function, $C(q)$. Therefore, the cost to make each item is \$11.13. \pspace

\item The fixed costs are the costs regardless of the level of production. But then we know that the fixed costs are given by $C(0)$. We have $C(0)= 11.13(0) + 576000= 576000$. Therefore, the fixed costs are \$576,000. \pspace

\item We know\dots
	\[
	\begin{aligned}
	R(10000)&= 45.99(10000)= 459900
	C(10000)&= 11.13(10000) + 576000= 111300 + 576000= 687300
	\end{aligned}
	\]
Because the revenue, \$459,900, is less than the costs, \$687,300, the company is experiencing a loss. In fact, we know that $P(10000)= \$459900 - \$687300= -\$227400$, i.e. the company is experiencing a \$227,400 loss. \pspace

\item To find the break-even point, we can solve $R(q)= C(q)$. But then we have\dots
	\[
	\begin{aligned}
	R(q)&= C(q) \\
	45.99q&= 11.13q + 576000 \\
	34.86q&= 5760000 \\
	q&= 16523.2
	\end{aligned}
	\]
But then in order to turn a profit, the company must produce/sell at least 16,524 items. 
\end{enumerate}



\newpage



% Problem 2
\problem{10} Howard just started a small business cleaning service called \textit{Grossbusters}. For now, he is renting a truck for \$1,550 per month. On average, he charges \$110 per cleaning and uses approximately \$4.86 in supplies per cleaning. 
	\begin{enumerate}[(a)]
	\item What are the fixed and variable costs for Howard's cleaning service?
	\item Find the cost function for Howard's business.
	\item Find the revenue function for Howard's business.
	\item Find the break-even point for Howard's business. What is the minimal amount of cleanings Howard must book per month to make a profit?
	\item How many cleanings must Howard book each month to make a monthly profit of \$8,000 (translating to a yearly profit of \$96,000)? Does this seem feasible? 
	\end{enumerate} 

\sol 
\begin{enumerate}[(a)]
\item The fixed costs are the \$1,550 per month to rent the truck, while the variable costs are the \$4.86 used in supplies per cleaning, i.e. $4.86q$. \pspace

\item We know $C(q)= \text{V.C.} + \text{F.C.}$. But then $C(q)= 4.86q + 1550$. \pspace

\item We know that he charges \$110 per cleaning. Therefore, $R(q)= 110q$. \pspace

\item To find the break-even point, one solves $R(q)= C(q)$ or computes $P(q)$ then solves $P(q)= 0$. In this instance, we use the latter method. First, we have\dots
	\[
	P(q)= R(q) - C(q)= 110q - (4.86q + 1550)= 110q - 4.86q - 1550= 105.14q - 1550
	\]
But then solving $P(q)= 0$, we have\dots
	\[
	\begin{aligned}
	P(q)&= 0 \\
	105.14q - 1550&= 0 \\
	105.14q&= 1550 \\
	q&= 14.7422
	\end{aligned}
	\]
Therefore, Howard must perform at least 15~cleanings per month to make a profit. \pspace

\item We need find $q$ such that $P(q)= 8000$. We found $P(q)$ in (d). But then, we have\dots 
	\[
	\begin{aligned}
	P(q)&= 8000 \\
	105.14q - 1550&= 8000 \\
	105.14q&= 9550 \\
	q&= 90.83
	\end{aligned}
	\]
Therefore, to make a profit of at least \$8,000 per month, Howard must perform at least 91~cleanings per month. Working every day (assuming a month of 30~days), this yields 3.033~cleanings per day with an average time of approximately 2~hours 38~minutes per cleaning (assuming an 8~hour work day). If Howard only works 8~hours a day on weekdays, then this is 4.55 cleanings per day with an average time of 1~hour 45~minutes per cleaning. In either case, this seems feasible---so long as he can actually make the bookings. 
\end{enumerate}



\newpage



% Problem 3
\problem{10} Suppose a company produces two items, $q_1$ and $q_2$, and has a cost function given by $C(q_1, q_2)= 56.20q_1 + 19.45q_2 + 7192$. 
	\begin{enumerate}[(a)]
	\item What are the fixed costs for producing these two items?
	\item What is the total cost associated with producing 30 of the first item and 65 of the second item?
	\item How much does it cost to produce the first item? How much does it cost to produce the second item?
	\end{enumerate} \pspace

\sol 
\begin{enumerate}[(a)]
\item We know that the fixed costs are given by $C(0, 0)= 56.20(0) + 19.45(0) + 7192= 0 + 0 + 7192= 7192$. Therefore, the fixed costs are \$7,192. \pspace

\item We have\dots
	\[
	C(30, 65)= 56.20(30) + 19.45(65) + 7192= 1686 + 1264.25 + 7192= 10142.25
	\]
Therefore, the total cost to produce 30 of the first item and 65 of the second item is \$10,142.25. From the work above, we can also see that it costs \$1,686 to produce the first item and \$1264.25 to produce the second item---ignoring the fixed costs. \pspace

\item Because $C(q_1, q_2)$ is (affine) linear, we know that the cost to produce each item is the `slope' in the `direction' of each variable. Therefore, it costs \$56.20 per item to produce the first item and \$19.45 per item to produce the second item. 
\end{enumerate}



\newpage



% Problem 4
\problem{10} Suppose that you have a revenue function given by $R(q)= 120q$ and a cost function given by $C(q)= 70q + 1600$. 
	\begin{enumerate}[(a)]
	\item What are the revenue and cost at a production/sale level of 80~units?
	\item Without finding the profit function, find the break-even point for the production/sale of this item.
	\item Find the profit function, $P(q)$.
	\item Compute $P(80)$. Explain how you could use (a) to find $P(80)$. 
	\end{enumerate} \pspace

\sol 
\begin{enumerate}[(a)]
\item We have\dots
	\[
	\begin{aligned}
	R(80)&= 120(80)= 9600 \\
	C(80)&= 70(80) + 1600= 7200
	\end{aligned}
	\] 
Therefore, at a production/sale level of 80~units, the revenue is \$9,600 and the costs are \$7,200. \pspace

\item To find the break-even point, we solve $R(q)= C(q)$. But then, we have\dots
	\[
	\begin{aligned}
	R(q)&= C(q) \\
	120q&= 70q + 1600 \\
	50q&= 1600 \\
	q&= 32
	\end{aligned}
	\]
Therefore, the break-even point is a production/sale level of 32~units. \pspace

\item We have\dots
	\[
	P(q)= R(q) - C(q)= 120q - (70q + 1600)= 120q - 70q - 1600= 50q - 1600
	\] \pspace

\item From (c), we know that $P(q)= 50q - 1600$. But then, we have\dots
	\[
	P(80)= 50(80) - 1600= 4000 - 1600= 2400
	\]
Therefore, at a production/sale level of 80~units, the profits are \$2,400. Observe that we could have found this in (a). From (a), we know that at a production/sale level of 80~units, the revenue and cost are \$9,600 and \$7,200, respectively. But then the profit is $\$9600 - \$7200= \$2400$. 
\end{enumerate}


\end{document}