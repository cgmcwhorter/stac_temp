\documentclass[11pt,letterpaper]{article}
\usepackage[lmargin=1in,rmargin=1in,tmargin=1in,bmargin=1in]{geometry}
\usepackage{../style/homework}
\usepackage{../style/commands}
\setbool{quotetype}{false} % True: Side; False: Under
\setbool{hideans}{true} % Student: True; Instructor: False

% -------------------
% Content
% -------------------
\begin{document}

\homework{7: Due 09/28}{The people who did the collateralized mortgage obligations sold them to pension funds, then sold them short, then bought credit default swap insurance on them, are just amazing. They are a law unto themselves.}{Ben Stein}

% Problem 1
\problem{10} E.M. Perior wants to take out a loan for\dots let's call it a small construction project.\footnote{It'll be no moon.} He approaches PJMorgan \& Follow Bank for a loan to fund the project. After discussions with the bank, they offer him a \$3~trillion loan at 8.33\% annual interest, compounded monthly over a period of 24~years. Mr. Perior's payments will be due at the end of every month. ``Good\dots \textit{good},'' Mr. Perior stated as he accepted the terms. 
	\begin{enumerate}[(a)]
	\item What are E.M. Perior's monthly payments?
	\item What is the total interest E.M. Perior will pay on this loan?
	\end{enumerate}



\newpage



% Problem 2
\problem{10} Theo Guysel wants to take out a loan to self-publish his new book \textit{Oh, the Places You'll Go\dots after Repaying Student Loans}. Guysel takes out a loan for \$15,000 at 3.7\% yearly interest, compounded quarterly. The loan is for a period of 5~years with end of the quarter payments of \$824.97.
	\begin{enumerate}[(a)]
	\item How much does Guysel still owe after 3~years?
	\item After 3~years, how much of Guysel's next payment will actually go towards paying off the loan? How much is paid in interest for this payment?
	\end{enumerate}


\end{document}