\documentclass[11pt,letterpaper]{article}
\usepackage[lmargin=1in,rmargin=1in,tmargin=1in,bmargin=1in]{geometry}
\usepackage{../style/homework}
\usepackage{../style/commands}
\setbool{quotetype}{true} % True: Side; False: Under
\setbool{hideans}{false} % Student: True; Instructor: False

% -------------------
% Content
% -------------------
\begin{document}

\homework{3: Due 09/19}{Money is not the most important thing in the world. Love is. Fortunately, I love money.}{Jackie Mason}

% Problem 1
\problem{10} The CPI in 2022 was approximately 296.071. According the the US Bureau of Labor Statistics, the current CPI is 307.026.
	\begin{enumerate}[(a)]
	\item Find the inflation rate from 2022 to 2023. 
	\item If the inflation rate in (a) continues from 2023 to 2024, estimate the cost of a good next year that costs \$46.99 this year,
	\item If the inflation rate in (a) remains constant across the next year, what will the increase in prices be from 2023 to 2028? 
	\end{enumerate} \pspace

\sol 
\begin{enumerate}[(a)]
\item Because the current CPI is greater than the CPI last year, we know there has been inflation. We know also that the inflation rate is\dots
	\[
	\left| \dfrac{\text{Current CPI}}{\text{Former CPI}} - 1 \right|= \left| \dfrac{307.026}{296.071} - 1 \right|= |1.037 - 1|= 0.037
	\]
Therefore, the inflation rate was 3.7\%. \pspace

\item If we want to compute $N$ increased or decreased by a \%, we compute $N \cdot (1 \pm \%_d)$, where $\%_d$ is the percentage written as a decimal and we choose `$+$' if it is a percentage increase and choose `$-$' if it is a percentage decrease. Assuming an inflation rate of 3.7\%, we expect a percentage increase of 3.7\%. But then, assuming a constant inflation rate, we approximate that the cost of the good next year will be\dots
	\[
	\$46.99 (1 + 0.037)= \$46.99 (1.037)= \$48.7286 \approx \$48.73
	\] \pspace

\item If we apply the same percentage increase or decrease $n$ times in a row, we multiply by $(1 \pm \%_d)$ a total of $n$ times. Therefore, if we want to compute $N$ increased or decreased by a \% a total of $n$ times, we compute $P(1 + \%_d)^n$. But then the $(1 + \%_d)^n$ factor represents the percentage increase or decrease resulting from applying a percentage increase/decrease of $\%_d$ a total of $n$ times. Assuming a constant inflation rate of 3.7\% over the five years from 2023 to 2028, we have\dots
	\[
	(1 + 0.037)^5= (1.037)^5= 1.19920597= 1 + 0.19920597
	\]
Therefore, we can recognize this as representing a 19.92\% increase, i.e. prices will increase 19.92\% from 2023 to 2028. 
\end{enumerate}



\newpage



% Problem 2
\problem{10} Stevie runs a shop that sells hot teas and baked beans. The shop is, surprisingly, not doing well. She decides to take out a simple discount note to pay for some additional advertising in the hopes that it will drive customers to the store. The note the bank offers is \$8,000 for 7~months at 8.3\% annual interest.
	\begin{enumerate}[(a)]
	\item What is the maturity for this simple discount note?
	\item What is the discount for this note?
	\item What is the interest Stevie pays on this loan?
	\item How much does Stevie receive from the bank?
	\item At the end of the 7~months, how much does Stevie owe the bank?
	\end{enumerate} \pspace

\sol 
\begin{enumerate}[(a)]
\item The maturity for a simple discount note is the loan amount before the interest is taken. Therefore, the maturity, $M$, for this note is \$8,000. \pspace

\item The discount for a simple discount note is the amount of interest paid on the loan---which is paid up-front. Using a rate of 8.3\% annual interest for a period of 7~months, i.e. $\frac{7}{12}$ years, we have
	\[
	D= Mrt= \$8000 \cdot 0.083 \cdot \tfrac{7}{12} \approx \$387.33
	\] \pspace

\item The interest in a simple discount note is the discount. In (b), we found a discount of \$387.33. Therefore, the interest paid is \$387.33. \pspace

\item In a simple discount note, one receives the value of maturity after the discount has been applied, i.e. after the interest has been paid. But then, we have\dots
	\[
	\text{Amount Received}= M - D= \$8000 - \$387.33= \$7,612.67
	\] \pspace

\item In total, Stevie must pay the maturity of the loan plus the discount (the interest). Therefore, a total of $\$8000 + \$387.33= \$8387.33$ is paid on the loan. The discount is paid up-front. Therefore, at the end of the 7~months, Stevie need only repay the \$8,000 maturity of the loan. 
\end{enumerate}


\end{document}