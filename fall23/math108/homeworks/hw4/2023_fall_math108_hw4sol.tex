\documentclass[11pt,letterpaper]{article}
\usepackage[lmargin=1in,rmargin=1in,tmargin=1in,bmargin=1in]{geometry}
\usepackage{../style/homework}
\usepackage{../style/commands}
\setbool{quotetype}{false} % True: Side; False: Under
\setbool{hideans}{false} % Student: True; Instructor: False

% -------------------
% Content
% -------------------
\begin{document}

\homework{4: Due 09/19}{Money will buy you a fine dog, but only love can make it wag its tail.}{Richard Friedman}

% Problem 1
\problem{10} Serge is hoping to make an investment in his future by saving his money so that he can one day invest in NFTs. He deposits \$16,000 into an account that earns 2.37\% annual interest, compounded monthly.
	\begin{enumerate}[(a)]
	\item How much does Serge have after 4~years?
	\item Suppose after 2~years, Serge deposits an additional \$4,000 into this account. How much money will he have after four years, i.e. two years after he makes this additional deposit? 
	\end{enumerate} \pspace

\sol We recognize that this is a discrete compounding interest with a principal of $P= \$16000$ and annual interest rate of $r= 0.0237$. Because we are compounding monthly, we know we are compounding 12 times per year, i.e. $k= 12$. 
\begin{enumerate}[(a)]
\item We want the future value of the principal after 4~years. We have\dots
	\[
	\hspace{-1cm} F= P \left(1 + \dfrac{r}{k} \right)^{kt}= \$16000 \left(1 + \dfrac{0.0237}{12} \right)^{12 \cdot 4}= \$16000 (1.001975)^48= \$16000 (1.0993362) \approx \$17,\!589.38
	\] \pspace

\item Two years after depositing the \$16,000 into the account, the account has\dots
	\[
	\hspace{-1cm} F= P \left(1 + \dfrac{r}{k} \right)^{kt}= \$16000 \left(1 + \dfrac{0.0237}{12} \right)^{12 \cdot 2}= \$16000 (1.001975)^{24}= \$16000 (1.04849233) \approx \$16,775.88
	\]
But then after depositing \$4,000, the account then has $\$16,775.88 + \$4,000= \$20,775.88$. But then after an additional two years of earning interest, the account has\dots
	\[
	\hspace{-1.5cm} F= P \left(1 + \dfrac{r}{k} \right)^{kt}= \$20775.88 \left(1 + \dfrac{0.0237}{12} \right)^{12 \cdot 2}= \$20775.88 (1.001975)^{24}= \$20775.88(1.04849233) \approx \$21,\!783.35
	\]
Alternatively, the only money in the account after the 4~years is the principal amount, the interest the principal earned across the 4~years, the \$4,000 deposited, and the interest the \$4,000 earned across the 2~years from its initial deposit. In (a), we found the value of the principal plus its interest. The value of the \$4,000 deposit plus its interest after 2~years of earning interest is\dots
	\[
	F= P \left(1 + \dfrac{r}{k} \right)^{kt}= \$4000 \left(1 + \dfrac{0.0237}{12} \right)^{12 \cdot 2}= \$4000 (1.001975)^{24}= \$4000 (1.04849233) \approx \$4,\!193.97
	\]
But then the account has $\$17,589.38 + \$4,193.97= \$21,783.35$ in total. 
\end{enumerate}



\newpage



% Problem 2
\problem{10} Piper is tired of her parents yelling at her to start saving for college---mostly because she wants to be a TikTok prank influencer. However, to quell their complaints, she starts setting aside money from her job to save for college textbooks. If she were to go to college, she estimates that she would spend at least \$600 each semester of college on textbooks. 
	\begin{enumerate}[(a)]
	\item What is the minimum amount she should estimate that she will spend on books while in college? 
	\item How much should she deposit into an account earning 1.17\% annual interest, compounded quarterly, so that she will have the minimum amount she estimated in (a) that she would need for college textbooks after 5 years? 
	\end{enumerate} \pspace

\sol 
\begin{enumerate}[(a)]
\item Assuming Piper goes to college and graduates in a total of four years, Piper will be at college for a total of $4 \cdot 2= 8$ semesters. If she spends a minimum of \$600 per semester, then she will spend a minimum of $8 \cdot \$600= \$4,800$ on textbooks while at college. \pspace

\item This is a discrete compounding interest setup with unknown principal, $P$, future value $F= \$4800$, and annual interest rate of $r= 0.0117$. Because the interest is compounded quarterly, i.e. four times per year, we have $k= 4$. But then, if she invests the principal, $P$, for $t= 5$~years, we have\dots
	\[
	\begin{aligned}
	P= \dfrac{F}{\left(1 + \dfrac{r}{k} \right)^{kt}}= \dfrac{\$4800}{\left(1 + \dfrac{0.0117}{4} \right)^{4 \cdot 5}}= \dfrac{\$4800}{(1.002925)^{20}}= \dfrac{\$4800}{1.0601545} \approx \$4,\!527.64
	\end{aligned}
	\]
Therefore, she will have to invest \$4,527.64 to have sufficient funds for her college textbooks after 5~years. 
\end{enumerate}



\newpage



% Problem 3
\problem{10} Ira Flatrow wants to buy an antique radio to adorn his fireplace mantel. The radio costs \$849.13. Working for a non-profit, he currently does not have the budget to set aside money each month. Instead, he will invest an up-front amount into an account that earns 3.3\% annual interest, compounded continuously. He hopes that this money will earn enough interest over time so that he will be able to afford the radio. Regardless, he will purchase the radio after 14~months---regardless of whether he has saved enough. If he does not want to spend any extra money at the end of the 14~months, what is the minimum amount he should invest now? \pspace

\sol This is a compounded interest setup with unknown principal, $P$, future value $F= \$849.13$, and annual yearly interest of $r= 0.033$. If Ira wants to be able to afford the radio after investing $P$ for 14~months, i.e. $t= \frac{14}{12} \approx 1.16667$ years, then we have\dots
	\[
	P= \dfrac{F}{e^{rt}}= \dfrac{\$849.13}{e^{0.033 \cdot 1.16667}}= \dfrac{\$849.13}{e^{0.03850011}}= \dfrac{\$849.13}{1.039251} \approx \$817.06
	\]
Therefore, Ira must deposit \$817.06 into the account for there to be \$849.13 in the account after 14~months. Obviously, placing less than this in the account will result in less than \$849.13 being in the account after the 14~months, and placing more than \$817.06 in the account will result in more than \$849.13 being in the account after the 14~months. Therefore, Ira should deposit a minimum of \$817.06 into the account. 



\newpage



% Problem 4
\problem{10} Kent C. Strate is an optometrist would like to save for a down payment on new office space for his business. He deposits a lump sum of \$175,000 into an account that earns 0.43\% annual interest, compounded semiannually. He will move offices in three years. If he uses all the money in this account for his down payment and only this money, what is the largest down payment that he will be able to afford? \pspace

\sol This is a discrete compounding interest setup. We have principal $P= \$175000$ and annual interest rate $r= 0.0043$. Because the interest is compounded semiannually, the interest is compounded twice per year, i.e. $k= 2$. But then the amount in the account after 3~years is\dots
	\[
	\hspace{-1cm} F= P \left(1 + \dfrac{r}{k} \right)^{kt}= \$175000 \left(1 + \dfrac{0.0043}{2} \right)^{2 \cdot 3}= \$175000 (1.00215)^6= \$175000(1.01296954) \approx \$177,269.67
	\]
Therefore, the most expensive down payment that Kent will be able to afford using this money is \$177,269.67.


\end{document}