\documentclass[11pt,letterpaper]{article}
\usepackage[lmargin=1in,rmargin=1in,tmargin=1in,bmargin=1in]{geometry}
\usepackage{../style/homework}
\usepackage{../style/commands}
\setbool{quotetype}{true} % True: Side; False: Under
\setbool{hideans}{true} % Student: True; Instructor: False

% -------------------
% Content
% -------------------
\begin{document}

\homework{4: Due 09/19}{Money will buy you a fine dog, but only love can make it wag its tail.}{Richard Friedman}

% Problem 1
\problem{10} Serge is hoping to make an investment in his future by saving his money so that he can one day invest in NFTs. He deposits \$16,000 into an account that earns 2.37\% annual interest, compounded monthly.
	\begin{enumerate}[(a)]
	\item How much does Serge have after 4~years?
	\item Suppose after 2~years, Serge deposits an additional \$4,000 into this account. How much money will he have after four years, i.e. two years after he makes this additional deposit? 
	\end{enumerate}



\newpage



% Problem 2
\problem{10} Piper is tired of her parents yelling at her to start saving for college---mostly because she wants to be a TikTok prank influencer. However, to quell their complaints, she starts setting aside money from her job to save for college textbooks. If she were to go to college, she estimates that she would spend at least \$600 each semester of college on textbooks. 
	\begin{enumerate}[(a)]
	\item What is the minimum amount she should estimate that she will spend on books while in college? 
	\item How much should she deposit into an account earning 1.17\% annual interest, compounded quarterly, so that she will have the minimum amount she estimated in (a) that she would need for college textbooks after 5 years? 
	\end{enumerate}



\newpage



% Problem 3
\problem{10} Ira Flatrow wants to buy an antique radio to adorn his fireplace mantel. The radio costs \$849.13. Working for a non-profit, he currently does not have the budget to set aside money each month. Instead, he will invest an up-front amount into an account that earns 3.3\% annual interest, compounded continuously. He hopes that this money will earn enough interest over time so that he will be able to afford the radio. Regardless, he will purchase the radio after 14~months---regardless of whether he has saved enough. If he does not want to spend any extra money at the end of the 14~months, what is the minimum amount he should invest now? 




\newpage



% Problem 4
\problem{10} Kent C. Strate is an optometrist would like to save for a down payment on new office space for his business. He deposits a lump sum of \$175,000 into an account that earns 0.43\% annual interest, compounded semiannually. He will move offices in three years. If he uses all the money in this account for his down payment and only this money, what is the largest down payment that he will be able to afford?


\end{document}