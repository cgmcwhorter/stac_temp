\documentclass[11pt,letterpaper]{article}
\usepackage[lmargin=1in,rmargin=1in,tmargin=1in,bmargin=1in]{geometry}
\usepackage{../style/homework}
\usepackage{../style/commands}
\setbool{quotetype}{true} % True: Side; False: Under
\setbool{hideans}{true} % Student: True; Instructor: False

% -------------------
% Content
% -------------------
\begin{document}

\homework{8: Due 10/17}{The consequences of an act affect the probability of its occurring again.}{B.F. Skinner}

% Problem 1
\problem{10} The probabilities of several events in a finite probability space are given below:
	\[
	\begin{aligned}
	P(A)&= 0.45 & P(D)&= 0.10 \\
	P(B)&= 0.20 & P(A \text{ and } C)&= 0.01 \\
	P(C)&= 0.85 & P(B \text{ and } C)&= 0.10 
	\end{aligned}
	\] 
\begin{enumerate}[(a)]
\item Assuming that $A$ and $B$ are independent, find $P(A \text{ or } B)$.
\item Assuming $C$ and $D$ are disjoint, find $P(C \text{ or } D)$.
\item Are $B$ and $C$ disjoint? Explain.
\item Are $A$ and $C$ independent? Explain. 
\item Find $P(B \;|\; C)$.
\end{enumerate}



\newpage



% Problem 2
\problem{10} A statistician is examining tax rebates for small businesses in the area. She finds that of the 227 small businesses in the county, 109 qualified for a state tax rebate, 80 qualified for a federal tax rebate, and 38 qualified for both. 
	\begin{enumerate}[(a)]
	\item Find the probability that a randomly selected local small business qualified for a state or federal tax rebate. 
	\item Find the probability that a randomly selected local small business qualified for a state and federal tax rebate. 
	\item Find the probability that a randomly selected local small business qualified for neither a state nor a federal tax rebate. 
	\item Find the probability that a randomly selected local small business qualified for only a state tax rebate. 
	\item Find the probability that a randomly selected local small business that qualified for a state tax rebate also qualified for a federal tax rebate. 
	\end{enumerate}



\newpage



% Problem 2
\problem{10} A large accounting class has 156 students. A chart summarizing the pass/fail/withdraw results for students, broken down by class, is given below. \par
	\begin{table}[H]
	\centering
	\begin{tabular}{lccc}
	& Pass & Fail & Withdraw \\ \cline{2-4} 
	\multicolumn{1}{l|}{Freshmen} & \multicolumn{1}{c|}{41} & \multicolumn{1}{c|}{14} & \multicolumn{1}{c|}{6} \\ \cline{2-4} 
	\multicolumn{1}{l|}{Sophomore} & \multicolumn{1}{c|}{56} & \multicolumn{1}{c|}{11} & \multicolumn{1}{c|}{3} \\ \cline{2-4} 
	\multicolumn{1}{l|}{Junior} & \multicolumn{1}{c|}{18} & \multicolumn{1}{c|}{3} & \multicolumn{1}{c|}{1} \\ \cline{2-4} 
	\multicolumn{1}{l|}{Senior} & \multicolumn{1}{c|}{3} & \multicolumn{1}{c|}{0} & \multicolumn{1}{c|}{0} \\ \cline{2-4} 
	\end{tabular}
	\end{table} \pspace
Given the data above, answer the following:
	\begin{enumerate}[(a)]
	\item Find the probability that a randomly selected student failed the course.
	\item Find the probability that a randomly selected student was a sophomore or withdrew from the course.
	\item Find the probability that a randomly selected student was a junior and failed the course.
	\item Find the probability that a randomly selected freshman failed the course.
	\item Are freshmen status and failing the course independent events? Explain. 
	\end{enumerate}



\newpage



% Problem 4
\problem{10} Administrators at a college are examining job placement for their graduates. Only 4\% of their graduates are Computer Science majors. They find that 85\% of their computer science majors obtain a job within 6~months of graduating. For all other majors at the college, 70\% of their graduates find a job within 6~months of graduating. 
	\begin{enumerate}[(a)]
	\item Find the percentage of graduates that received a job within 6~months of graduating. 
	\item Find the percentage of graduates that were a computer science major and obtained a job within 6~months of graduating.
	\item Find the percentage of graduates that obtained a job within 6~months of graduating or were not a computer science major. 
	\item Of the graduates that obtained a job within 6~months of graduating, what percentage were computer science majors?	
	\end{enumerate}


\end{document}