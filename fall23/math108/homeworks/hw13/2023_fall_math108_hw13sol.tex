\documentclass[11pt,letterpaper]{article}
\usepackage[lmargin=1in,rmargin=1in,tmargin=1in,bmargin=1in]{geometry}
\usepackage{../style/homework}
\usepackage{../style/commands}
\setbool{quotetype}{true} % True: Side; False: Under
\setbool{hideans}{false} % Student: True; Instructor: False

% -------------------
% Content
% -------------------
\begin{document}

\homework{13: Due 12/12}{You can't learn too much linear algebra.}{Benedict Gross}

% Problem 1
\problem{10} Define the following:
	\[
	\mathbf{u}= \begin{pmatrix} 1 \\ -1 \\ 5 \\ 7 \end{pmatrix}, \qquad
	\mathbf{v}= \begin{pmatrix} 0 \\ 2 \\ 4 \\ -6 \end{pmatrix}, \qquad
	\mathbf{w}= \begin{pmatrix} 8 \\ 1 \\ 0 \\5 \end{pmatrix}
	\]
Showing all your work, compute the following:
	\begin{enumerate}[(a)]
	\item $-3\mathbf{v}$
	\item $\mathbf{w} - \mathbf{u}$
	\item $\mathbf{v} \cdot \mathbf{w}$
	\end{enumerate} \pspace

\sol 
\begin{enumerate}[(a)]
\item 
	\[
	-3\mathbf{v}= -3 \begin{pmatrix} 0 \\ 2 \\ 4 \\ -6 \end{pmatrix}= \begin{pmatrix} 0 \\ -6 \\ -12 \\ 18 \end{pmatrix}
	\]

\item 
	\[
	\begin{pmatrix} 8 \\ 1 \\ 0 \\5 \end{pmatrix} - \begin{pmatrix} 1 \\ -1 \\ 5 \\ 7 \end{pmatrix}= \begin{pmatrix} 7 \\ 2 \\ -5 \\ -2 \end{pmatrix}
	\]

\item 
	\[
	\begin{pmatrix} 0 \\ 2 \\ 4 \\ -6 \end{pmatrix} \cdot \begin{pmatrix} 8 \\ 1 \\ 0 \\5 \end{pmatrix}= 0(8) + 2(1) + 4(0) + (-6)5= 0 + 2 + 0 - 30= -28
	\]
\end{enumerate}



\newpage



% Problem 2
\problem{10} Define the following:
	\[
	A= \begin{pmatrix} 1 & 3 & 0 \\ -2 & 5 & 2 \end{pmatrix}, \qquad
	B= \begin{pmatrix} 8 & 4 & -1 \\ 2 & 0 & 6 \end{pmatrix}, \qquad
	C= \begin{pmatrix} 1 & 7 & 3 \\ -2 & 6 & 0 \end{pmatrix}
	\]
Showing all your work, compute the following:
	\begin{enumerate}[(a)]
	\item $-4B$
	\item $C - A$
	\item $AB^T$
	\end{enumerate} \pspace
	
\sol 
\begin{enumerate}[(a)]
\item 
	\[
	-4 \begin{pmatrix} 8 & 4 & -1 \\ 2 & 0 & 6 \end{pmatrix}= \begin{pmatrix} -32 & -16 & 4 \\ -8 & 0 & -24 \end{pmatrix}
	\]

\item 
	\[
	\begin{pmatrix} 1 & 7 & 3 \\ -2 & 6 & 0 \end{pmatrix} - \begin{pmatrix} 1 & 3 & 0 \\ -2 & 5 & 2 \end{pmatrix}= \begin{pmatrix} 0 & 4 & 3 \\ 0 & 1 & -2 \end{pmatrix}
	\]

\item 
	\[
	\begin{aligned}
	\begin{pmatrix} 1 & 3 & 0 \\ -2 & 5 & 2 \end{pmatrix} \begin{pmatrix} 8 & 4 & -1 \\ 2 & 0 & 6 \end{pmatrix}^T&= \begin{pmatrix} 1 & 3 & 0 \\ -2 & 5 & 2 \end{pmatrix} \begin{pmatrix} 8 & 2 \\ 4 & 0 \\ -1 & 6 \end{pmatrix} \\[0.3cm]
	&= \begin{pmatrix} 1(8) + 3(4) + 0(-1) & 1(2) + 3(0) + 2(6) \\ -2(8) + 5(4) + 2(-1) & -2(2) + 5(0) + 2(6) \end{pmatrix} \\[0.3cm]
	&= \begin{pmatrix} 8 + 12 + 0 & 2 + 0 + 12 \\ -16 + 20 - 2 & -4 + 0 + 12 \end{pmatrix} \\[0.3cm]
	&= \begin{pmatrix} 20 & 14 \\ 2 & 8 \end{pmatrix}
	\end{aligned}
	\]
\end{enumerate}



\newpage



% Problem 3
\problem{10} Define the following:
	\[
	A= \begin{pmatrix} 4 & 6 & 1 & 0 & 5 \\ -1 & 2 & -3 & 0 & 4 \end{pmatrix}, \qquad
	\mathbf{u}= \begin{pmatrix} 1 \\ 0 \\ 2 \\ 0 \\ -3 \end{pmatrix}
	\]

\begin{enumerate}[(a)]
\item Can one compute $A\mathbf{u}$? If so, compute it. If not, explain why. 
\item Can one compute $A^T\mathbf{u}$? If so, compute it. If not, explain why. 
\end{enumerate} \pspace

\sol 
\begin{enumerate}[(a)]
\item 
	\[
	\hspace{-1.2cm} \begin{pmatrix} 4 & 6 & 1 & 0 & 5 \\ -1 & 2 & -3 & 0 & 4 \end{pmatrix} \begin{pmatrix} 1 \\ 0 \\ 2 \\ 0 \\ -3 \end{pmatrix}= \begin{pmatrix} 4(1) + 6(0) + 1(2) + 0(0) + 5(-3) \\ -1(1) + 2(0) + (-3)2 + 0(0) + 4(-3) \end{pmatrix}= \begin{pmatrix} 4 + 0 + 2 + 0 - 15 \\ -1 + 0 - 6 + 0 - 12 \end{pmatrix}= \begin{pmatrix} -9 \\ -19 \end{pmatrix}
	\] \pspace

\item The matrix $A$ has dimension $2 \times 5$. Because the transpose interchanges rows and columns, $A^T$ has dimension $5 \times 2$, i.e. 5 rows and 2 columns. The vector $\mathbf{u}$ has dimension $5 \times 1$. Because the number of columns of $A^T$ (two) does not match the number of rows of $\mathbf{u}$ (five), one cannot form $A^T \mathbf{u}$. 
\end{enumerate}


\end{document}