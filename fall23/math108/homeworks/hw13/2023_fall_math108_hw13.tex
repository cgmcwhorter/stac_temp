\documentclass[11pt,letterpaper]{article}
\usepackage[lmargin=1in,rmargin=1in,tmargin=1in,bmargin=1in]{geometry}
\usepackage{../style/homework}
\usepackage{../style/commands}
\setbool{quotetype}{true} % True: Side; False: Under
\setbool{hideans}{true} % Student: True; Instructor: False

% -------------------
% Content
% -------------------
\begin{document}

\homework{13: Due 12/12}{You can't learn too much linear algebra.}{Benedict Gross}

% Problem 1
\problem{10} Define the following:
	\[
	\mathbf{u}= \begin{pmatrix} 1 \\ -1 \\ 5 \\ 7 \end{pmatrix}, \qquad
	\mathbf{v}= \begin{pmatrix} 0 \\ 2 \\ 4 \\ -6 \end{pmatrix}, \qquad
	\mathbf{w}= \begin{pmatrix} 8 \\ 1 \\ 0 \\ 5 \end{pmatrix}
	\]
Showing all your work, compute the following:
	\begin{enumerate}[(a)]
	\item $-3\mathbf{v}$
	\item $\mathbf{w} - \mathbf{u}$
	\item $\mathbf{v} \cdot \mathbf{w}$
	\end{enumerate}



\newpage



% Problem 2
\problem{10} Define the following:
	\[
	A= \begin{pmatrix} 1 & 3 & 0 \\ -2 & 5 & 2 \end{pmatrix}, \qquad
	B= \begin{pmatrix} 8 & 4 & -1 \\ 2 & 0 & 6 \end{pmatrix}, \qquad
	C= \begin{pmatrix} 1 & 7 & 3 \\ -2 & 6 & 0 \end{pmatrix}
	\]
Showing all your work, compute the following:
	\begin{enumerate}[(a)]
	\item $-4B$
	\item $C - A$
	\item $AB^T$
	\end{enumerate}



\newpage



% Problem 3
\problem{10} Define the following:
	\[
	A= \begin{pmatrix} 4 & 6 & 1 & 0 & 5 \\ -1 & 2 & -3 & 0 & 4 \end{pmatrix}, \qquad
	\mathbf{u}= \begin{pmatrix} 1 \\ 0 \\ 2 \\ 0 \\ -3 \end{pmatrix}
	\]

\begin{enumerate}[(a)]
\item Can one compute $A\mathbf{u}$? If so, compute it. If not, explain why. 
\item Can one compute $A^T\mathbf{u}$? If so, compute it. If not, explain why. 
\end{enumerate}


\end{document}