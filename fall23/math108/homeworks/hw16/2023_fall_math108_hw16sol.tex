\documentclass[11pt,letterpaper]{article}
\usepackage[lmargin=1in,rmargin=1in,tmargin=1in,bmargin=1in]{geometry}
\usepackage{../style/homework}
\usepackage{../style/commands}
\setbool{quotetype}{false} % True: Side; False: Under
\setbool{hideans}{false} % Student: True; Instructor: False

% -------------------
% Content
% -------------------
\begin{document}

\homework{16: Due 12/12}{Linear programming can be viewed as part of a great revolutionary development which has given mankind the ability to state general goals and to lay out a path of detailed decisions to take in order to 'best' achieve its goals when faced with practical situations of great complexity.}{George Dantzig}

% Problem 1
\problem{10} Find the initial simplex tableau corresponding to the linear programming problem shown below:
	\[
	\begin{gathered}
	\hspace{-0.7cm} \max z= 4x_1 - x_2 + 5x_3 \\
	\begin{cases}
	x_1 + 7x_3 \leq 20 \\
	x_1 - 5x_2 + 9x_3 \geq -20 \\
	-x_1 + x_2 - 5x_3 \geq 30 \\
	x_1, x_2, x_3 \geq 0
	\end{cases}
	\end{gathered}
	\] \pspace

\sol We need all inequalities to have a nonnegative number on the `right side' of the inequality. So we must multiply both sides of the second inequality by $-1$, so that we obtain the following inequalities: 
	\[
	\begin{cases}
	x_1 + 0x_2 + 7x_3 \leq 20 \\
	-x_1 + 5x_2 - 9x_3 \leq 20 \\
	-x_1 + x_2 - 5x_3 \geq 30 \\
	x_1, x_2, x_3 \geq 0	
	\end{cases}
	\]
Observe that we introduce the missing $0x_2$ in the first inequality. We now introduce slack or surplus variables to obtain equalities. We also move everything to `one side' in the function to obtain $z - 4x_1 + x_2 - 5x_3= 0$. Writing all these equalities together, we obtain\dots \par
	\begin{table}[H]
	\centering
	\begin{tabular}{rrrrrrrrrrrrrrr}
	              & & $x_1$ & $+$ & $0x_2$ & $+$ & $7x_3$ & $+$ & $s_1$ & & & & & $=$ & $20$ \\
	              & & $-x_1$ & $+$ & $5x_2$ & $+$ & $-9x_3$ & $+$ & & & $s_2$ & & & $=$ & $20$ \\
	  	      & & $-x_1$ & $+$ & $x_2$ & $+$ & $-5x_3$ & & & & & $-$ &$s_3$ & $=$ & $30$ \\
	 $z$ & $+$ & $-x_1$ & $+$ & $5x_2$ & $+$ & $-9x_3$ & & & & & & & $=$ & $0$ \\
	\end{tabular}
	\end{table} \par
Therefore, the initial simplex tableau is\dots \par
	\begin{table}[H]
	\centering
	\begin{tabular}{rrrrrr|c}
	$1$ & $0$ & $7$ & $1$ & $0$& $0$ & $20$ \\ 
	$-1$ & $5$ & $-9$ & $0$ & $1$ & $0$ & $20$ \\
	$-1$ & $1$ & $-5$ & $0$ & $0$ & $-1$ & $30$ \\ \hline
	$-1$ & $5$ & $-9$ & $0$ & $0$& $0$ & $0$ 
	\end{tabular}
	\end{table}



\newpage



% Problem 2
\problem{10} Below is the initial simplex tableau corresponding to a linear programming maximization problem. Find the initial maximization problem. \par
	\begin{table}[H]
	\centering
	\begin{tabular}{rrrrrrrr}
	$7$ & $9$ & $0$ & $1$ & $1$ & $0$ & $0$ & $21$ \\
	$-1$ & $9$ & $4$ & $2$ & $0$ & $-1$ & $0$ & $37$ \\
	$1$ & $0$ & $-1$ & $3$ & $0$ & $0$ & $1$ & $46$ \\
	$-5$ & $4$ & $-6$ & $1$ & $0$ & $0$ & $0$ & $0$ \\
	\end{tabular}
	\end{table} \pspace

\sol We first add the appropriate horizontal line to separate the function from the inequalities and a vertical line to separate the sides of the equalities. \par
	\begin{table}[H]
	\centering
	\begin{tabular}{rrrrrrr|r}
	$7$ & $9$ & $0$ & $1$ & $1$ & $0$ & $0$ & $21$ \\
	$-1$ & $9$ & $4$ & $2$ & $0$ & $-1$ & $0$ & $37$ \\
	$1$ & $0$ & $-1$ & $3$ & $0$ & $0$ & $1$ & $46$ \\ \hline
	$-5$ & $4$ & $-6$ & $1$ & $0$ & $0$ & $0$ & $0$ \\
	\end{tabular}
	\end{table} \par
The last row corresponds to the function, while the other rows correspond to the inequalities. Therefore, there were three inequalities in the original problem (not including the non-negativity conditions). For each inequality, we introduce a slack or surplus variable. Therefore, three of the variables are slack or surplus variables. Each column---except the last---corresponds to a variable in the system. Therefore, there are 7 total variables. With 3 slack variables, there must then be $7- 3= 4$ original variables in the system. We can then label the variables in our system. \par
	\begin{table}[H]
	\centering
	\begin{tabular}{rrrrrrrr}
	{\footnotesize $x_1$} & {\footnotesize $x_2$} & {\footnotesize $x_3$} & {\footnotesize $x_4$} & {\footnotesize $s_1$} & {\footnotesize $s_2$} & {\footnotesize $s_3$} & \\
	$7$ & $9$ & $0$ & $1$ & $1$ & $0$ & \multicolumn{1}{r|}{$0$} & $21$ \\
	$-1$ & $9$ & $4$ & $2$ & $0$ & $-1$ & \multicolumn{1}{r|}{$0$} & $37$ \\
	$1$ & $0$ & $-1$ & $3$ & $0$ & $0$ & \multicolumn{1}{r|}{$1$} & $46$ \\ \hline
	$-5$ & $4$ & $-6$ & $1$ & $0$ & $0$ & \multicolumn{1}{r|}{$0$} & $0$ \\
	\end{tabular}
	\end{table} \par
We can see that we had to add $s_1, s_3$ to obtain equalities. Therefore, these are slack variables and the corresponding inequalities must have been `$\leq$'. As we had to subtract $s_2$ to obtain an equality, this must have been a surplus variable. Therefore, this corresponding inequality must have been `$\geq$.' We know that $z - 5x_1 + 4x_2 - 6x_3 + x_4= 0$, which implies $z= 5x_1 - 4x_2 + 6x_3 - x_4$. Introducing the condition that the variables are nonnegative, the original optimization problem must have been\dots
	\[
	\begin{gathered}
	\hspace{-0.7cm} \max z= 5x_1 - 4x_2 + 6x_3 - x_4 \\
	\begin{cases}
	7x_1 + 9x_2 + x_4 \leq 21 \\
	-x_1 + 9x_2 + 4x_3 + 2x_4 \geq 37 \\
	x_1 - x_3 + 3x_4 \leq 46 \\
	x_1, x_2, x_3, x_4 \geq 0
	\end{cases}
	\end{gathered}
	\] 
	



\newpage



% Problem 3
\problem{10} Below is the final simplex tableau for a linear programming maximization problem. \par
	\begin{table}[H]
	\centering
	\begin{tabular}{rrrrrrrrrrrr}
	$0$ & $9.3$ & $0$ & $0$ & $-3.31$ & $1$ & $-1.23$ & $-0.11$ & $0.45$ & $0$ & $0$ & $26.79$ \\
	$1$ & $0.39$ & $0$ & $0$ & $3.08$ & $0$ & $0.31$ & $0.18$ & $-0.14$ & $0$ & $0$ & $27.37$ \\
	$0$ & $0.68$ & $1$ & $0$ & $-0.77$ & $0$ & $-0.08$ & $0.05$ & $0.07$ & $0$ & $0$ & $5.1$ \\
	$0$ & $-0.36$ & $0$ & $1$ & $0.08$ & $0$ & $0.31$ & $-0.07$ & $0.11$ & $0$ & $0$ & $28.87$ \\
	$0$ & $-3.99$ & $0$ & $0$ & $-6.46$ & $0$ & $0.15$ & $-1.97$ & $1.24$ & $1$ & $0$ & $61.06$ \\
	$0$ & $-1.79$ & $0$ & $0$ & $-3.15$ & $0$ & $0.38$ & $-1.37$ & $0.29$ & $0$ & $1$ & $31.27$ \\
	$0$ & $2.78$ & $0$ & $0$ & $9.62$ & $0$ & $3.46$ & $0.34$ & $0.47$ & $0$ & $0$ & $355.67$
	\end{tabular}
	\end{table}

\begin{enumerate}[(a)]
\item How many inequalities were considered?
\item How many variables were there in the original inequalities?
\item How many slack/surplus variables were introduced?
\item What was the solution to this maximization problem?
\end{enumerate} 

\sol 
\begin{enumerate}[(a)]
\item Every row in the tableau corresponds to an inequality---except for the last row which corresponds to the function. Because there are $7$ rows, there must have been $7 - 1= 6$ inequalities in the original system (neglecting the non-negativity inequalities). 

\item Every column in the tableau corresponds to a variable---except the last column which corresponds to the `other' side of an equality. Because there are $12$ columns, there are $12 - 1= 11$ variables in the system. Because we introduce a slack or surplus variable to each inequality and by (a) there are $6$ inequalities, $6$ of the variables are slack/surplus variables. Therefore, there were $11 - 6= 5$ `original' variables in the system. 

\item By (b), we know that there were $6$ slack or surplus variables introduced. 

\item Introducing labels for the variables, adding horizontal and vertical lines, and boxing the `pivot positions', we obtain the following tableau: \par
	\begin{table}[H]
	\centering
	\begin{tabular}{rrrrrrrrrrrr}
	{\footnotesize $x_1$} & {\footnotesize $x_2$} & {\footnotesize $x_3$} & {\footnotesize $x_4$} & {\footnotesize $x_5$} & {\footnotesize $s_1$} & {\footnotesize $s_2$} & {\footnotesize $s_3$} & {\footnotesize $s_4$} & {\footnotesize $s_5$} & {\footnotesize $s_6$} & \\ 	
	$0$ & $9.3$ & $0$ & $0$ & $-3.31$ & \boxed{$1$} & $-1.23$ & $-0.11$ & $0.45$ & $0$ & \multicolumn{1}{r|}{$0$} & $26.79$ \\
	\boxed{$1$} & $0.39$ & $0$ & $0$ & $3.08$ & $0$ & $0.31$ & $0.18$ & $-0.14$ & $0$ & \multicolumn{1}{r|}{$0$} & $27.37$ \\
	$0$ & $0.68$ & \boxed{$1$} & $0$ & $-0.77$ & $0$ & $-0.08$ & $0.05$ & $0.07$ & $0$ & \multicolumn{1}{r|}{$0$} & $5.1$ \\
	$0$ & $-0.36$ & $0$ & \boxed{$1$} & $0.08$ & $0$ & $0.31$ & $-0.07$ & $0.11$ & $0$ & \multicolumn{1}{r|}{$0$} & $28.87$ \\
	$0$ & $-3.99$ & $0$ & $0$ & $-6.46$ & $0$ & $0.15$ & $-1.97$ & $1.24$ & \boxed{$1$} & \multicolumn{1}{r|}{$0$} & $61.06$ \\
	$0$ & $-1.79$ & $0$ & $0$ & $-3.15$ & $0$ & $0.38$ & $-1.37$ & $0.29$ & $0$ & \multicolumn{1}{r|}{\boxed{$1$}} & $31.27$ \\ \hline
	$0$ & $2.78$ & $0$ & $0$ & $9.62$ & $0$ & $3.46$ & $0.34$ & $0.47$ & $0$ & \multicolumn{1}{r|}{$0$} & $355.67$
	\end{tabular}
	\end{table}
This gives $x_1= 27.37$, $x_3= 5.1$, $x_4= 28.87$, $s_1= 26.79$, $s_5= 61.06$, and $s_6= 31.27$. All the remaining variables have value $0$. From the bottom-rightmost entry, we see that $\max z= 355.67$. Therefore, the maximum values i$ 355.67$ and occurs at $(x_1, x_2, x_3, x_4, x_5, s_1, s_2, s_3, s_4, s_5, s_6)= (27.37, 0, 5.1, 28.87, 0, 26.79, 0, 0, 0, 61.06, 31.27)$. 
\end{enumerate}



\newpage



% Problem 4
\problem{10} Below is the final simplex tableau for a linear programming minimization problem. \par
	\begin{table}[H]
	\centering
	\begin{tabular}{rrrrr}
	$0$ & $1$ & $0.5$ & $-0.25$ & $0.25$ \\
	$1$ & $0$ & $-0.25$ & $0.38$ & $0.63$ \\
	$0$ & $0$ & $4.75$ & $1.38$ & $13.63$
	\end{tabular}
	\end{table}

\begin{enumerate}[(a)]
\item How many inequalities were considered?
\item How many variables were there in the original inequalities?
\item How many slack/surplus variables were introduced?
\item What was the solution to this minimization problem?
\end{enumerate} \pspace

\sol 
\begin{enumerate}[(a)]
\item Each row of the tableau corresponds to an inequality---except for the last row which corresponds to the function. Because there are $3$ rows, the original system had $3 - 1= 2$~inequalities. 

\item Each column of the tableau corresponds to a variable---except for the last column which corresponds to the `other side' of the equalities. Because there are $5$ columns, there are $5 - 1= 4$~variables in the system. A slack or surplus variable is introduced for each inequality. We know there are two inequalities from (a). Therefore, there are two slack or surplus variables. For a minimization problem, the variables in the original system correspond to the slack or surplus variables in the dual maximization problem. Therefore, there were $2$ variables in the original minimization problem. For the dual maximization problem, we know there were $4$ variables involved and that there are $2$ slack or surplus variables. Therefore, the dual maximization problem has $4 - 2= 2$ `original' variables. 

\item From (b), we know that $2$ slack or surplus variables were introduced into the dual maximization problem. 

\item We add a horizontal line to separate the function from the equalities and a vertical line to separate the variables from the `other side' of the equalities. We also label the $2$ `original' variables and the $2$ slack/surplus variables. \par
	\begin{table}[H]
	\centering
	\begin{tabular}{rrrrr}
	{\footnotesize $x_1$} & {\footnotesize $x_2$} & {\footnotesize $s_1$} & {\footnotesize $s_2$} & \\ 
	$0$ & $1$ & $0.5$ & \multicolumn{1}{r|}{$-0.25$} & $0.25$ \\
	$1$ & $0$ & $-0.25$ & \multicolumn{1}{r|}{$0.38$} & $0.63$ \\ \hline
	$0$ & $0$ & $4.75$ & \multicolumn{1}{r|}{$1.38$} & $13.63$
	\end{tabular}
	\end{table} \par
The minimum value to the original minimization problem is the maximum value for the dual problem. From the table above, we can see that the maximum value is $13.63$. The original minimization problem had two variables, say $y_1, y_2$. The value of the original minimization variables are the `values' of the slack/surplus variables in the dual minimization problem. Therefore, we know $y_1= 4.75$ and $y_2= 1.38$. 
\end{enumerate}


\end{document}