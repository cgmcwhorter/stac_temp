\documentclass[11pt,letterpaper]{article}
\usepackage[lmargin=1in,rmargin=1in,bmargin=1in,tmargin=1in]{geometry}
\usepackage{style/quiz}
\usepackage{style/commands}

% -------------------
% Content
% -------------------
\begin{document}
\thispagestyle{title}

% Quiz 1
\quizsol \textit{True/False}: The product $189.75(1.08)$ can be interpreted as representing finding either 8\% of $189.75$ or increasing $189.75$ by 8\%. \pspace

\sol The statement is false. To compute a \% of some number $N$, we need only compute $N \cdot \%_d$ and if we want to compute $N$ increased or decreased by a \%, we compute $N \cdot (1 \pm \%_d)$, where $\%_d$ is the percentage written as a decimal and we choose `$+$' if it is an increase and choose `$-$' if it is a decrease. So thinking of the product $189.75(1.08)$ as finding a percent of a number, we must have $\#= 189.75$ and $\%_d= 1.08$, i.e. $\%= 108\%$. Then one way of interpreting this is finding 108\% of 189.75---not the 8\% claimed in the quiz statement. Interpreting the product $189.75(1.08)$ as finding a percentage increase/decrease, it must be a percentage increase because $1.08 > 1$. Writing $1.08= 1 + 0.08$, we can see that $\%_d= 0.08$. Therefore, $189.75(1.08)$ can represent finding 189.75 increased by 8\%---which is what is claimed in the quiz statement. 








\end{document}