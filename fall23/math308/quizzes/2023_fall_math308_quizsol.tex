\documentclass[11pt,letterpaper]{article}
\usepackage[lmargin=1in,rmargin=1in,bmargin=1in,tmargin=1in]{geometry}
\usepackage{style/quiz}
\usepackage{style/commands}

% -------------------
% Content
% -------------------
\begin{document}
\thispagestyle{title}

% Quiz 1
\quizsol \textit{True/False}: The expression $P \to Q$ is logically equivalent to $\neg P \vee Q$. \pspace

\sol The statement is true. One method of seeing is this is to compute the truth table for $P \to Q$ and $\neg P \vee Q$ and see that the outputs of $P \to Q$ and $\neg P \vee Q$ match, no matter the inputs for $P, Q$. \par
	\begin{table}[h]
	\centering
	\begin{tabular}{ccccc}
	$P$ & $Q$ & $P \to Q$ & $\neg P$ & $\neg P \vee Q$ \\ \hline 
	$T$ & $T$ & $\mathbf{T}$ & $F$ & $\mathbf{T}$ \\
	$T$ & $F$ & $\mathbf{F}$ & $F$ & $\mathbf{F}$ \\
	$F$ & $T$ & $\mathbf{T}$ & $T$ & $\mathbf{T}$ \\
	$F$ & $F$ & $\mathbf{T}$ & $T$ & $\mathbf{T}$
	\end{tabular}
	\end{table} \par
As we can see, the third and fourth columns corresponding to $P \to Q$ and $\neg P \vee Q$, respectively, are the same, $P \to Q \equiv \neg P \vee Q$. Alternatively, $P \to Q$ will be logically equivalent to $\neg P \vee Q$ if they are always simultaneously true. We know for $P \to Q$ to be true, either $P$ must be false or $P, Q$ must both be true. Observe that if $P$ is false, then $\neg P$ is true so that $\neg P \vee Q$ is true. If $P, Q$ are true, then $\neg P \vee Q$ is true. Loosely, $P \to Q$ is true if either $P$ does not occur or if $Q$ occurs. But this is precisely $\neg P \vee Q$. In any case, it is true that $P \to Q \equiv \neg P \vee Q$. 








\end{document}