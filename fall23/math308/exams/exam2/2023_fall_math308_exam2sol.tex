\documentclass[12pt,letterpaper]{exam}
\usepackage[lmargin=1in,rmargin=1in,tmargin=1in,bmargin=1in]{geometry}
\usepackage{../style/exams}

% -------------------
% Course & Exam Information
% -------------------
\newcommand{\course}{MAT 308: Exam 2}
\renewcommand{\term}{Fall -- 2023}
\newcommand{\examdate}{11/16/2023}
\newcommand{\timelimit}{`$\infty$' Minutes}

\setbool{hideans}{true} % Student: True; Instructor: False


% -------------------
% Content
% -------------------
\begin{document}

\examtitle
\instructions{Write your name on the appropriate line on the exam cover sheet. This exam contains \numpages\ pages (including this cover page) and \numquestions\ questions. Check that you have every page of the exam. Answer the questions in the spaces provided on the question sheets. Be sure to answer every part of each question and show all your work. If you run out of room for an answer, continue on the back of the page --- being sure to indicate the problem number.} 
\scores
\bottomline
\newpage

% ---------
% Questions
% ---------
\begin{questions}

% Question 1
\newpage
\question[10] Showing all your work and fully justifying your reasoning, compute the following:
	\begin{enumerate}[(a)]
	\item $\displaystyle \sum_{k=0}^5 (2k^2 - 10k + 5)$
	\item $\displaystyle \prod_{\substack{j= -3 \\ j \neq -1}}^3 \dfrac{j}{j + 1}$
	\item $\displaystyle \sum_{k=1}^\infty \left( \dfrac{1}{k + 2} - \dfrac{1}{k} \right)$
	\item $\displaystyle \sum_{j=1}^{3} \sum_{i=0}^{3} (ij + i - j + 2)$
	\end{enumerate}



% Question 2
\newpage
\question[10] Showing all your work and fully justifying your reasoning, find a closed form expression for the following:
	\[
	\sum_{i= -1}^n \left( 3(i+1)^2 + 5ni - 3n \right)
	\]



% Question 3
\newpage
\question[10] Define the following vectors and matrices:
	\[
	A= \begin{pmatrix} 1 & -4 & 2 \\ 0 & 3 & -1 \end{pmatrix}, \qquad
	B= \begin{pmatrix} 5 & -3 & 4 \\ 1 & 2 & 1 \end{pmatrix}, \qquad
	\mathbf{u}= \begin{pmatrix} 5 \\ -1 \\ 6 \end{pmatrix}, \qquad
	\mathbf{v}= \begin{pmatrix} -3 \\ 10 \\ 1 \end{pmatrix}
	\]
Showing all your work and fully justifying your reasoning, answer the following:
	\begin{enumerate}[(a)]
	\item Compute $-3\mathbf{v} + \mathbf{u}$.
	\item Compute $\mathbf{u} \cdot \mathbf{v}$.
	\item Compute $2A - B$.
	\item Compute $A\mathbf{u}$.
	\item Only one of $AB$, $B^T\mathbf{u}$, and $A^T B$ can be computed. Explain which cannot and compute the one that can be computed. 
	\end{enumerate}



% Question 4
\newpage
\question[10] Being sure to show all your work and fully justify your logic, complete the following:
	\begin{enumerate}[(a)]
	\item Use the definition of even and odd integers to show that $-198$ is even and $455$ is odd. 
	\item Is $432= 18 \cdot 24$ a factorization of $432$? Is this a prime factorization? Find the prime factorization of $432$. 
	\item Are there integers $x, y$ such that $4x + 6y= 5$? Explain. 
	\item Compute $\gcd(2^{10} \cdot 3^{20} \cdot 5^{30} \cdot 11^{80}, \, 2^{50} \cdot 3^{40} \cdot 5^{30} \cdot 7^{80})$. 
	\item Compute $\lcm(2^{10} \cdot 3^{20} \cdot 5^{30} \cdot 11^{80}, \, 2^{50} \cdot 3^{40} \cdot 5^{30} \cdot 7^{80})$. 
	\end{enumerate}



% Question 5
\newpage
\question[10] Showing all your work, answer the following:
	\begin{enumerate}[(a)]
	\item Express $\dfrac{-1488}{287}$ using the division algorithm. 
	\item Compute $\gcd(287, 1488)$ using the Euclidean algorithm. 
	\item Express $\gcd(287, 1488)$ as a linear combination of $287$ and $1488$. 
	\end{enumerate}



% Question 6
\newpage
\question[10] Showing all your work, compute the following:
	\begin{enumerate}[(a)]
	\item $(15 \cdot 18 + 43) \mod 8$.
	\item $(34 \cdot 17) \mod 45$.
	\item $99^{1234567} \mod 100$.
	\item The last three digits of $44^{200}$.
	\item The number of digits in $99^{1234567}$. 
	\end{enumerate}



% Question 7
\newpage
\question[10] Consider the linear congruence $287x + 584 \equiv 422 \mod 1488$. 
	\begin{enumerate}[(a)]
	\item Explain why there exists a unique solution to this equation modulo $1,\!488$. 
	\item How many integers in the range 1, 2, \ldots, $1,\!488$ are invertible modulo $1,\!488$?
	\item Is $287$ invertible modulo $1488$? Explain. 
	\item Solve the given linear congruence. 
	\end{enumerate}



% Question 8
\newpage
\question[10] Showing all your work, compute the following:
	\begin{enumerate}[(a)]
	\item How many different starting teams of six players can be chosen from a hockey team of $11$ players. 
	\item The number of possible top five finishers from a F$1$ race with $22$ drivers. 
	\item The number of distinct arrangements of the letters in `sassafras.'
	\item How many arrangements of the digits in the number `$12345678$' contain the number `$72$.'
	\end{enumerate}



% Question 9
\newpage
\question[10] Showing all your work, compute the following:
	\begin{enumerate}[(a)]
	\item The coefficient of $x^3y^7$ in $(2x - 5y)^{10}$
	\item The coefficient of $xy^2z^3$ in $(2x - y + 5z)^6$
	\item How many committees with at most two men with a designated chair can be chosen from a collection of five men and 8 women. 
	\item The number of nonnegative integer solutions to $x_1 + x_2 + x_3= 23$ with $x_2 \geq 2$. [Hint: Consider `distributing' twenty-three $1$'s to $x_1$, $x_2$, $x_3$.]
	\end{enumerate}



% Question 10
\newpage
\question[10] Let $S$ be the set of nonnegative integers that are at most one million. 
	\begin{enumerate}[(a)]
	\item Find the number of elements of $S$ that are not divisible by 7 but are divisible by at least one of 3 or 5. 	
	\item Find the number of elements of $S$ that are a multiple of at least one of 4, 6, or 11.
	\item Find the number of elements of $S$ that are divisible by 2 (but not 4), and at least one of 3, 5, 6, or 9.
	\end{enumerate}


\end{questions}
\end{document}