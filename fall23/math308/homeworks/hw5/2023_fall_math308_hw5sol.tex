\documentclass[11pt,letterpaper]{article}
\usepackage[lmargin=1in,rmargin=1in,tmargin=1in,bmargin=1in]{geometry}
\usepackage{../style/homework}
\usepackage{../style/commands}
\setbool{quotetype}{true} % True: Side; False: Under
\setbool{hideans}{true} % Student: True; Instructor: False

% -------------------
% Content
% -------------------
\begin{document}

\homework{5: Due 10/05}{No one shall expel us from the paradise which Cantor has created for us.}{David Hilbert}

% Problem 1
\problem{10} Define the following sets:
        \begin{enumerate}[(a)]
        \item The set of positive odd numbers less than 50 that are not prime. 
        \item The set of real-valued solutions to $\frac{x^2 - 4}{x + 2}= 0$.
        \item The set of integer solutions to $\sqrt{2x - 1} + 4= 12$. 
        \item The set of English sentences containing a homonym. 
        \item The set of linear functions with positive $y$-intercept. 
        \end{enumerate}
For each of the sets described above, do the following:
	\begin{enumerate}[(i)]
	\item Determine if the set is empty or nonempty. If the set is nonempty, give an element and non-element of the set.	
	\item Determine whether the set is finite or infinite. If it is finite, state its cardinality. 
	\item If the set is finite, enumerate all its elements. If the set is infinite, give the set using set-builder notation. 
	\end{enumerate}



\newpage 
\begin{center} {\itshape --- Continued Space for Problem~1 ---} \end{center}
\newpage


% Problem 2
\problem{10} For each of the sets given below, describe the sets in words. Also for each set, give an example of an element and non-element of the set.
	\begin{enumerate}[(a)]
	\item $\{ 3, 5, 7, 11, 13, 17, 19, 23, 29, 31, 37, 41, 43, 47, \ldots \}$
	\item $\{ 0, 1, 4, 9, 16, 25, 36, 49, 81, 100, \ldots \}$
	\item $\{ f(x) \colon f(x) \text{ function}, ( \exists x_0 \in \mathbb{R} ) [f(x_0)= 0] \}$
	\item $\{ f(x) \colon f(x) \text{ function}, f(x) > 0 \text{ for all } x \in \mathbb{R} \}$
	\item $\{ f(x, y) \colon f(2, 3)= 0 \}$
	\end{enumerate}



\newpage



% Problem 3
\problem{10} Define the following sets:
	\[
	\begin{aligned}
	A&= \{ -10, -9, -8, \ldots, 8, 9, 10 \} \\
	B&= \{ -10, -8, -6, \ldots, 6, 8, 10 \} \\
	C&= \{ -9, -7, -5, \ldots, 5, 7, 9 \} \\
	D&= \{ -10, -5, 0, 5, 10 \} \\
	E&= \{ -4, -1, 1, 2, 3, 5, 7 \} \\
	F&= \{ -10, -9, -2, -1, 1, 5, 6, 9 \}
	\end{aligned}
	\]
Consider each of the sets above as coming from the universal set $\mathcal{U}:= A$. Compute the following:
	\begin{2enumerate}
	\item $B \cup C$
	\item $B \cap C$
	\item $E \setminus D$
	\item $F \Delta C$
	\item $E^c$
	\item $(C \cup E) - B$
	\end{2enumerate}



\newpage



% Problem 4
\problem{10} Define the following sets:
	\[
	\begin{aligned}
	A&:= (-10, 10) \\
	B&:= [0, 3] \\
	C&:= (-1, 15] \\
	D&:= (-20, -3] \cup [4, 12)
	\end{aligned}
	\]
Consider each of the sets above as coming from the universal set $\mathbb{R}= (-\infty, \infty)$. Compute the following:
	\begin{enumerate}[(a)]
	\item $D^c$
	\item $B \cap C$
	\item $A \Delta C$
	\item $C - B$
	\item $(A \cap D) \cup C$
	\end{enumerate}


\end{document}