\documentclass[11pt,letterpaper]{article}
\usepackage[lmargin=1in,rmargin=1in,tmargin=1in,bmargin=1in]{geometry}
\usepackage{../style/homework}
\usepackage{../style/commands}
\setbool{quotetype}{true} % True: Side; False: Under
\setbool{hideans}{false} % Student: True; Instructor: False

% -------------------
% Content
% -------------------
\begin{document}

\homework{5: Due 10/05}{No one shall expel us from the paradise which Cantor has created for us.}{David Hilbert}

% Problem 1
\problem{10} Define the following sets:
        \begin{enumerate}[(a)]
        \item The set of positive odd numbers less than 50 that are not prime. 
        \item The set of real-valued solutions to $\frac{x^2 - 4}{x + 2}= 0$.
        \item The set of integer solutions to $\sqrt{2x - 1} + 4= 12$. 
        \item The set of English sentences containing a homonym. 
        \item The set of linear functions with positive $y$-intercept. 
        \end{enumerate}
For each of the sets described above, do the following:
	\begin{enumerate}[(i)]
	\item Determine if the set is empty or nonempty. If the set is nonempty, give an element and non-element of the set.	
	\item Determine whether the set is finite or infinite. If it is finite, state its cardinality. 
	\item If the set is finite, enumerate all its elements. If the set is infinite, give the set using set-builder notation. 
	\end{enumerate} \pspace

\sol 
\begin{enumerate}[(a)]
\item Call this set $A$. 
	\begin{enumerate}[(i)]
	\item The set $A$ is nonempty. We can see that $1, 9, 15 \in A$ and $-4, 2, 5 \notin A$. 
	\item The set $A$ is finite. The cardinality of $A$ is eleven, i.e. $|A|= 11$. 
	\item The set $A$ is finite and we have $A= \{ 1, 9, 15, 21, 25, 27, 33, 35, 39, 45, 49 \}$. 
	\end{enumerate} \pspace

\item Call this set $B$.
	\begin{enumerate}[(i)]
	\item The set $B$ is nonempty. We can see that $2 \in B$ and $-2, 0, 1 \notin B$. 
	\item The set $B$ is finite. The cardinality of $B$ is one, i.e. $|B|= 1$. 
	\item The set $B$ is finite and we have $B= \{ 2 \}$.\footnote{Clearly, the left-side is not defined when $x= -2$. If $x \neq -2$, we have $\frac{x^2 - 4}{x + 2}= \frac{(x - 2)(x + 2)}{x + 2}= x - 2$, which is clearly zero if and only if $x= 2$.}
	\end{enumerate} \pspace

\item Call this set $C$. 
	\begin{enumerate}[(i)]
	\item The set $C$ is empty.\footnote{If $\sqrt{2x - 1} + 4= 12$, then $\sqrt{2x - 1}= 8$. But then $2x - 1= 64$. This implies that $2x= 65$. Hence, $x= \frac{65}{2} \notin \mathbb{Z}$. Therefore, there are no integer solutions to this equation. However, there are rational and real solutions to the equation---namely $x= \frac{65}{2}$.}
	
	\newpage \begin{center} {\itshape --- Continued Space for Problem~1 ---} \end{center} \pspace


	\item The set $C$ is finite. The cardinality of $C$ is zero, i.e. $|C|= 0$. 
	\item The set $C$ is finite. The set $C$ is $C= \{ \,\}= \varnothing$. 
	\end{enumerate} \pspace

\item Call this set $D$. Recall that a homonym are words which are homographs (words with the same spelling) or homophones (words having the pronunciation). For example, bat (the baseball instrument) and bat (the adorable flying mammal) are homonyms. Alternatively, air, ere, heir are all homonyms. 
	\begin{enumerate}[(i)]
	\item The set $D$ is nonempty. The sentence, ``I like the bat,'' and, ``Have some air ere meeting the heir,'' are elements of $D$. The sentence, ``I like turtles,'' is a non-element of $D$. 
	\item The set $D$ is infinite.\footnote{There are many ways to justify this. However, a valid argument cannot suggest that there are infinitely many such sentences simply because there are `lots' of homonyms or `lots' of sentences involving containing them. Even unfathomably large numbers (e.g. $10^{10^{10^{34}}}$) are still finite. One way of justifying that there are infinitely many such sentences is to use the famous example that appeared in Dmitri Borgmann's 1967 book \textit{Beyond Language: Adventures in Word and Though}. The word buffalo can have many meanings. As a verb, buffalo can mean to bully, harass, or intimidate, or buffalo can mean to outwit, fool, confuse, baffle, or deceive. As a noun, buffalo is the large cattle. There is also Buffalo, as in the city Buffalo, New York. Borgmann gave the unusual example of the following sentence: Buffalo buffalo Buffalo buffalo buffalo buffalo Buffalo buffalo; that is, Buffalo buffalo (buffalo from Buffalo) which Buffalo buffalo (buffalo from Buffalo) bully also bully Buffalo buffalo (buffalo from Buffalo). As James Henle, Jay Garfield, and Thomas Tymoczko discussed in their book \textit{Sweet Reason: A Field Guide to Modern Logic}, one can actually write a sentence using only the word buffalo (possibly with capitalizations) to create a grammatically correct English sentence. Because buffalo is a homonym and one, for any natural number $n$, can create a grammatically correct English sentence containing $n$ repetitions of the word buffalo, there are infinitely many sentences containing a homonym.}
	\item The set $D$ is infinite. Let $\mathcal{U}$ be the set of all possible English sentences. We can define $D$ to be $D:= \{ S \in \mathcal{U} \colon S \text{ contains a homonym} \}$. 
	\end{enumerate} \pspace

\item Call this set $E$. 
	\begin{enumerate}[(i)]
	\item The set $E$ is nonempty. Recall that for a linear function $y= mx + b$, the $y$-intercept is the $b$-value. Then $x + 1 \in E$, while $x - 1 \notin E$. 
	\item The set $E$ is infinite. 
	\item The set $E$ can be given as $E= \{ mx + b \colon m,b \in \mathbb{R}, b > 0 \}= \{ mx + b \colon m \in \mathbb{R}, b \in \mathbb{R}^{> 0} \}$. 
	\end{enumerate}
\end{enumerate}



\newpage



% Problem 2
\problem{10} For each of the sets given below, describe the sets in words. Also for each set, give an example of an element and non-element of the set.
	\begin{enumerate}[(a)]
	\item $\{ 3, 5, 7, 11, 13, 17, 19, 23, 29, 31, 37, 41, 43, 47, \ldots \}$
	\item $\{ 0, 1, 4, 9, 16, 25, 36, 49, 81, 100, \ldots \}$
	\item $\{ f(x) \colon f(x) \text{ function}, ( \exists x_0 \in \mathbb{R} ) [f(x_0)= 0] \}$
	\item $\{ f(x) \colon f(x) \text{ function}, f(x) > 0 \text{ for all } x \in \mathbb{R} \}$
	\item $\{ f(x, y) \colon f(2, 3)= 0 \}$
	\end{enumerate} 

\sol 
\begin{enumerate}[(a)]
\item Call this given set $A$. This is the set of odd prime numbers. We can see that $3, 11, 47, 71 \in A$ and $-3, 1, 2, 9 \notin A$. \pspace

\item Call this given set $B$. This is the set of perfect squares. We can see that $0, 1, 25 \in B$ and $-5, -4, 2 \notin B$. \pspace

\item Call this given set $C$. This is the set of functions with a real root. Equivalently, this is the set of functions with a real zero. Equivalently, this is the set of functions with an $x$-intercept. We define $f(x)= 0$, $g(x)= 2x - 1$, and $h(x)= \sin x$. Clearly, $f(x)$ has a root for all $x$. We see that if $x_0= \frac{1}{2}$, then $g(x_0)= 0$, so that $g(x)$ has a real root. Finally, if $x_0= n\pi$ for any integer $n$, then $h(x_0)= 0$, so that $h(x)$ has a real root. Therefore, $f, g, h \in C$. \par\vspace{0.2cm}

Furthermore, we can define $a(x)= 1$ and $b(x)= x^2 + 1$. Clearly, $a(x)= 1 > 0$ for all $x$, so that $a(x)$ does not have a root. If $b(x)= 0$, then $x^2 + 1= 0$. This implies that $x^2= -1$, which is impossible for a real number $x$ because the square of any real number is nonnegative. Therefore, $b(x)$ cannot have a real root. This shows that $a, b \notin C$. \pspace

\item Call this given set $D$. This is the set of positive functions. We can define $f(x)= 1$ and $g(x)= x^2 + 1$. Clearly, $f(x)= 1 > 0$ for all $x$, so that $f(x)$ is a positive function. If $g(x) \leq 0$ for some $x$, then $x^2 + 1 \leq 0$ for some $x$. But then $x^2 \leq -1$ for some $x$. However, the square of any real number is nonnegative. Therefore, $g(x) > 0$ for all $x$, so that $g(x)$ is a positive function. Then we know that $f, g \in D$. \par\vspace{0.2cm}

Furthermore, we can define $a(x)= -1$ and $b(x)= x$. Clearly, $a(x)= -1 \leq 0$ for all $x$, so that $a(x)$ is not a positive function. Observe that while $b(x) > 0$ for $x > 0$, it is not the case that $b(x) > 0$ \textit{for all} $x$. For instance, $b(-1)= -1 < 0$. Then $b(x)$ is not a positive function. Therefore, $a, b \notin D$. \pspace

\item Call this given set $E$. This is the set of relations with a zero at $(x, y)= (2, 3)$. For instance, we can define $f(x, y)= (x - 2) + (y - 3)$, $g(x, y)= x + y - 5$, and $h= \{ \big( (0, 0), 1 \big), \big( (0, 0), 1 \big), \big( (2, 3), 0 \big) \}$. Clearly, $f(2, 3)= 0$, $g(2, 3)= 0$, and $\big( (2, 3), 0 \big) \in h$. Therefore, $f, g, h \in E$. \pspace

Furthermore, we can define $a(x, y)= 1$, $b(x, y)= x - y$, and $c= \{ \big( (0, 0), 1 \big), \big( (0, 0), 1 \big), \big( (2, 3), 1 \big) \}$. Then $a(2, 3)= 1 \neq 0$, $b(2, 3)= -1 \neq 0$, and $\big( (2, 3), 0) \notin c$. Therefore, $a, b, c \notin E$. 
\end{enumerate}



\newpage



% Problem 3
\problem{10} Define the following sets:
	\[
	\begin{aligned}
	A&= \{ -10, -9, -8, \ldots, 8, 9, 10 \} \\
	B&= \{ -10, -8, -6, \ldots, 6, 8, 10 \} \\
	C&= \{ -9, -7, -5, \ldots, 5, 7, 9 \} \\
	D&= \{ -10, -5, 0, 5, 10 \} \\
	E&= \{ -4, -1, 1, 2, 3, 5, 7 \} \\
	F&= \{ -10, -9, -2, -1, 1, 5, 6, 9 \}
	\end{aligned}
	\]
Consider each of the sets above as coming from the universal set $\mathcal{U}:= A$. Compute the following:
	\begin{2enumerate}
	\item $B \cup C$
	\item $B \cap C$
	\item $E \setminus D$
	\item $F \,\Delta\, C$
	\item $E^c$
	\item $(C \cup E) - B$
	\end{2enumerate} \pspace

\sol 
\begin{enumerate}[(a)]
\item $B \cup C= \{ -10, -9, -8, \ldots, 8, 9, 10 \}= \mathcal{U}$
\item $B \cap C= \varnothing$
\item $E \setminus D= \{ -4, -1, 1, 2, 3, 7 \} $
\item $F \,\Delta\, C= \{ -10, -7, -5, -3, -2, 3, 6 \}$
\item $E^c= \{ -10, -9, -8, -7, -6, -5, -3, -2, 0, 4, 6, 8, 9, 10 \}= \mathcal{U} \setminus \{ -4, -1, 1, 2, 3, 5, 7 \}$
\item $(C \cup E) - B= \{ -9, -7, -5, -3, -1, 1, 3, 5, 7, 9 \}$
\end{enumerate}



\newpage



% Problem 4
\problem{10} Define the following sets:
	\[
	\begin{aligned}
	A&:= (-10, 10) \\
	B&:= [0, 3] \\
	C&:= (-1, 15] \\
	D&:= (-20, -3] \cup [4, 12)
	\end{aligned}
	\]
Consider each of the sets above as coming from the universal set $\mathbb{R}= (-\infty, \infty)$. Compute the following:
	\begin{enumerate}[(a)]
	\item $D^c$
	\item $B \cap C$
	\item $A \,\Delta\, C$
	\item $C - B$
	\item $(A \cap D) \cup C$
	\end{enumerate} \pspace

\sol 
\begin{enumerate}[(a)]
\item $D^c= (\infty, -20] \cup (-3, 4) \cup [12, \infty)$
\item $B \cap C= [0, 3]$
\item $A \,\Delta\, C= (-10, -1] \cup [10, 15]$
\item $C - B= (-1, 0) \cup (3, 15)$
\item $(A \cap D) \cup C= (-10, -3] \cup (-1, 15]$
\end{enumerate}


\end{document}