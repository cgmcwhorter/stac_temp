\documentclass[11pt,letterpaper]{article}
\usepackage[lmargin=1in,rmargin=1in,tmargin=1in,bmargin=1in]{geometry}
\usepackage{../style/homework}
\usepackage{../style/commands}
\setbool{quotetype}{true} % True: Side; False: Under
\setbool{hideans}{false} % Student: True; Instructor: False

% -------------------
% Content
% -------------------
\begin{document}

\homework{1: Due 09/07}{And I knew exactly what to do\dots but in a much more real sense, I had no idea what to do.}{Michael Scott, The Office}

% Problem 1
\problem{10} Determine if each of the following are propositions. If the example is a proposition, state its truth value with a brief justification. If the example is \textit{not} a proposition, briefly explain why:
	\begin{enumerate}[(a)]
	\item $3^2 - 15= 6$
	\item The statement in (c) is false. 
	\item George Orwell wrote \textit{A Remembrance of Things Past}. 
	\item There is intelligent life in the universe.  
	\item $x - 3 \leq 10$
	\end{enumerate} \pspace

\sol 
\begin{enumerate}[(a)]
\item This is a proposition---either $3^2 - 15= 6$ or $3^3 - 15 \neq 6$. In fact, because $3^2 - 15= 9 - 15= -6 \neq 6$, the proposition is false. \pspace

\item This is a proposition. So long as (c) is a proposition, then the statement in (c) is either true or false. But this will mean the statement in (b) is either false or true, respectively. Because (c) is a proposition, we know that the statement in (b) is a proposition. Because (c) is false, the statement in (b) is true. \pspace

\item This is a proposition---either George Orwell wrote \textit{A Remembrance of Things Past} or he did not. In fact, \textit{A Remembrance of Things Past} was written by Marcel Proust. Therefore, the proposition is false. \pspace

\item This is a proposition---either there is `intelligent' life in the universe or there is not. If we consider human life to be `intelligent' life,\footnote{There are strong arguments to suggest otherwise\dots} then the proposition is true. \pspace

\item This is \textit{not} a proposition. There is no definite truth value as the veracity of the inequality depends on the value of $x$. For instance, if $x= -6$, then $x - 3= -6 - 3= -9 \leq 10$, so that the expression is true. However, if $x= 20$, then $x - 3= 20 - 3= 17 \not\leq 10$, so that the expression is false. 
\end{enumerate}



\newpage



% Problem 2
\problem{10} For each of the following, either define appropriate primitive propositions (using $P$, $Q$, $R$, etc.) and write the `statement' using logical connectives, or give an English sentence for the given primitives and `translate' the logical `sentence' into an English sentence:
	\begin{enumerate}[(a)]
	\item $P \to (\neg Q \vee R)$
	\item You will succeed, if you believe and work hard. 
	\item $Q \wedge (\neg P \vee Q)$
	\item I pay rent, or I lose my job and starve. 
	\end{enumerate} \pspace

\sol 
\begin{enumerate}[(a)]
\item There are many possible solutions, depending on the choices for the propositions $P$, $Q$, and $R$. For instance, choose $P$ to be the proposition ``I study,'' $Q$ to be the proposition ``I take the exam,'' and $R$ to be the proposition ``I pass.'' The logical expression $P \to (\neg Q \vee R)$ is the statement, ``If I study, then either I do not take the exam or I pass the exam.'' \pspace

\item Let $R$ be the proposition ``You succeed,'' $P$ be the proposition ``You believe,'' and $Q$ be the proposition ``You work hard.'' The statement, ``You will succeed, if you believe and work hard,'' which can also be equivalently stated as, ``If you believe and work hard, then you will succeed,'' can be expressed as a logical statement as $P \wedge Q \to R$. \pspace

\item Let $Q$ be the proposition ``I buy a new TV,'' $P$ be the proposition ``I buy a game station.'' Then the logical expression $Q \wedge (\neg P \vee Q)$ written as a complete English sentence is, ``I buy a new TV, and I do not buy a new game station or I buy a new TV.'' \pspace

\item Let $P$ be the proposition ``I pay rent,'' $Q$ be the proposition ``I lose my job,'' and $R$ be the proposition ``I starve.'' The statement, ``I pay rent, or I lose my job and starve,'' can be expressed as a logical statement as $P \vee (Q \wedge R)$. 
\end{enumerate}



\newpage



% Problem 3
\problem{10} Consider the following compound statement: $\neg (P \to \neg Q) \wedge \neg Q$
	\begin{enumerate}[(a)]
	\item Determine whether the given compound statement is a tautology, contradiction, or neither. Be sure to justify your response. 
	\item Using a truth table, show that the first part of the given compound statement, i.e. $\neg (P \to \neg Q)$, is logically equivalent to $P \wedge Q$. 
	\item By `simplifying' the expression $\neg \big(P \vee \neg (P \wedge Q) \big)$, show that this compound statement is logically equivalent to the compound statement given at the start of the problem.
	\end{enumerate} \pspace

\sol 
\begin{enumerate}[(a)]
\item One method of determining if the given expression is a tautology, contradiction, or neither, we can simply construct a truth table for $\neg (P \to \neg Q) \wedge \neg Q$: \par
	\begin{table}[H]
	\centering
	\begin{tabular}{cc|ccc|c}
	$P$ & $Q$ & $\neg Q$ & $P \to \neg Q$ & $\neg (P \to \neg Q)$ & $\neg (P \to \neg Q) \wedge \neg Q$ \\ \hline
	$T$ & $T$ & $F$ & $F$ & $T$ & $F$ \\
	$T$ & $F$ & $T$ & $T$ & $F$ & $F$ \\
	$F$ & $T$ & $F$ & $T$ & $F$ & $F$ \\
	$F$ & $F$ & $T$ & $T$ & $F$ & $F$
	\end{tabular}
	\end{table} \par
Because the logical expression $\neg (P \to \neg Q) \wedge \neg Q$ is false, independent on the truth values for $P, Q$, the expression $\neg (P \to \neg Q) \wedge \neg Q$ is a contradiction. \pspace

We can simplify the given expression to determine whether it is a tautology, contradiction, or neither:
	\[
	\neg (P \to \neg Q) \wedge \neg Q \equiv \big(P \wedge \neg (\neg Q) \big) \wedge \neg Q \equiv (P \wedge Q) \wedge \neg Q \equiv P \wedge (Q \wedge \neg Q) \equiv P \wedge F_0 \equiv F_0
	\]
Therefore, the statement $\neg (P \to \neg Q) \wedge \neg Q$ must be a contradiction. \pspace

\item We construct the truth table for $\neg (P \to \neg Q)$ and $P \wedge Q$: \par
	\begin{table}[H]
	\centering
	\begin{tabular}{cc|ccc|c}
	$P$ & $Q$ & $\neg Q$ & $P \to \neg Q$ & $\neg (P \to \neg Q)$ & $P \wedge Q$ \\ \hline
	$T$ & $T$ & $F$ & $F$ & $T$ & $T$ \\
	$T$ & $F$ & $T$ & $T$ & $F$ & $F$ \\
	$F$ & $T$ & $F$ & $T$ & $F$ & $F$ \\
	$F$ & $F$ & $T$ & $T$ & $F$ & $F$
	\end{tabular}
	\end{table} \par
Because the columns for $\neg (P \to \neg Q)$ and $P \wedge Q$ have the same outputs for any given inputs of $P, Q$, we know that $\neg (P \to \neg Q)$ and $P \wedge Q$ are logically equivalent, i.e. $\neg (P \to \neg Q) \equiv P \wedge Q$. \pspace


\item We have\dots
	\[
	\neg \big(P \vee \neg (P \wedge Q) \big) \equiv \neg P \wedge (P \wedge Q) \equiv (\neg P \wedge P) \wedge Q \equiv F_0 \wedge Q \equiv F_0
	\]
But then $\neg \big(P \vee \neg (P \wedge Q) \big)$ is a contradiction. Because all contradictions are logically equivalent, it must be that $\neg \big(P \vee \neg (P \wedge Q) \big)$ is logically equivalent to $\neg (P \to \neg Q) \wedge \neg Q$.
\end{enumerate}



\newpage



% Problem 4
\problem{10} Fix a real number $x$. Consider the statement, ``if $x^2 > 4$, then $x > 2$'
	\begin{enumerate}[(a)]
	\item Determine the truth value of this statement with an explanation. 
	\item Rewrite the given statement by defining appropriate primitive propositions and logical connectives. 
	\item Find the negation, converse, and contrapositive of your result from (b).
	\item Rewrite your answers from (c) as English sentences. Then determine the truth value, with explanation, of each of the statements. 
	\end{enumerate} \pspace

\sol 
\begin{enumerate}[(a)]
\item Suppose one chooses $x= 1$. Then the statement is, ``If $1 > 4$, then $1 > 2$,'' which is a true statement because $1 \not> 4$. [Recall that $F_0 \to P$ is true for all propositions $P$.] If one chooses $x= 10$, then the statement is, ``If $100 > 4$, then $10 > 2$.'' This is a true statement because this is equivalent to $T_0 \to T_0$, which is true. However, if one chooses $x= -3$, then the given statement is, ``If $9 > 4$, then $-3 > 2$,'' which is equivalent to $T_0 \to F_0$. Hence, the statement is false when $x= -3$. \pspace

Generally, the given statement will be true when $x \geq -2$ and false for all other values of $x$. Assume that $x$ is a real number. If $x^2 > 4$, then $x^2 - 4 > 0$. But $x^2 - 4= (x - 2)(x + 2)$, so that $(x - 2)(x + 2) > 0$. This implies that either $x - 2 > 0$ and $x + 2 > 0$, or $x - 2 < 0$ and $x + 2 < 0$. This is equivalent to $x > 2$ and $x > -2$, or $x < 2$ and $x < -2$. But these are in turn equivalent to $x > 2$ or $x < -2$. Alternatively, suppose that $x < -2$ or $x > 2$. We can from the fact that $x < -2 < 0$ and $x > 2 > 0$, that `squaring' the inequalities, we obtain $x^2 > 4$.\footnote{There is a lot of nuance here. Why the observation that $x < -2 < 0$ and $x > 2 > 0$? What is meant by `squaring' here and why is this a `valid' operation? Making these steps clear are a good exercise.} But then we know that $x^2 > 4$ if and only if $-2 < x$ and $x > 2$. But then $x^2 > 4$ is false whenever $-2 \leq x \leq 2$. However, if $x^2 > 4$ is false, then ``if $x^2 > 4$, then $x > 2$'' is true. So the statement, ``If $x^2 > 4$, then $x > 2$,'' is true whenever $-2 \leq x \leq 2$. Clearly, if $x > 2$, then $x^2 > 4$ and $x > 2$. But then, ``If $x^2 > 4$, then $x > 2$,'' is true. Finally, if $x < -2$, we know that $x^2 > 4$, but clearly $x \not> 2$. Therefore, in this case, ``If $x^2 > 4$, then $x > 2$,'' is false. Therefore, the statement, ``If $x^2 > 4$, then $x > 2$,'' is true if and only if $x \geq -2$. 

\item For a given fixed $x$, let $P$ be the proposition $P: x^2 > 4$ and $Q$ be the proposition $Q: x > 2$. Then the statement, ``If $x^2 > 4$, then $x > 2$,'' can be written as the logical expression $P \to Q$. \pspace

\item The negation of $P \to Q$ is $\neg (P \to Q) \equiv P \wedge \neg Q$. The converse of $P \to Q$ is $Q \to P$. Finally, the contrapositive of $P \to Q$ is $\neg Q \to \neg P$. \pspace

\item We know the negation is $P \wedge \neg Q$. Observe that $\neg Q: x \leq 2$. Written as an English sentence, the negation is the statement, ``$x^2 > 4$ and $x \leq 2$.'' The converse is $Q \to P$. Written as an English sentence, this is the statement, ``If $x > 2$, then $x^2 > 4$.'' The contrapositive was $\neg Q \to \neg P$. We know that $\neg P: x^2 \leq 4$. But written as a complete English sentence, the contrapositive is, ``If $x \leq 2$, then $x^2 \leq 4$.'' 








\end{enumerate}


\end{document}