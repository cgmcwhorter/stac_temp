\documentclass[11pt,letterpaper]{article}
\usepackage[lmargin=1in,rmargin=1in,tmargin=1in,bmargin=1in]{geometry}
\usepackage{../style/homework}
\usepackage{../style/commands}
\setbool{quotetype}{true} % True: Side; False: Under
\setbool{hideans}{false} % Student: True; Instructor: False

% -------------------
% Content
% -------------------
\begin{document}

\homework{1: Due 09/07}{And I knew exactly what to do\dots but in a much more real sense, I had no idea what to do.}{Michael Scott, The Office}

% Problem 1
\problem{10} Determine if each of the following are propositions. If the example is a proposition, state its truth value with a brief justification. If the example is \textit{not} a proposition, briefly explain why:
	\begin{enumerate}[(a)]
	\item $3^2 - 15= 6$
	\item The statement in (c) is false. 
	\item George Orwell wrote \textit{A Remembrance of Things Past}. 
	\item There is intelligent life in the universe.  
	\item $x - 3 \leq 10$
	\end{enumerate} \pspace

\sol 
\begin{enumerate}[(a)]
\item This is a proposition---either $3^2 - 15= 6$ or $3^3 - 15 \neq 6$. In fact, because $3^2 - 15= 9 - 15= -6 \neq 6$, the proposition is false. \pspace

\item This is a proposition. So long as (c) is a proposition, then the statement in (c) is either true or false. But this will mean the statement in (b) is either false or true, respectively. Because (c) is a proposition, we know that the statement in (b) is a proposition. Because (c) is false, the statement in (b) is true. \pspace

\item This is a proposition---either George Orwell wrote \textit{A Remembrance of Things Past} or he did not. In fact, \textit{A Remembrance of Things Past} was written by Marcel Proust. Therefore, the proposition is false. \pspace

\item This is a proposition---either there is `intelligent' life in the universe or there is not. If we consider human life to be `intelligent' life,\footnote{There are strong arguments to suggest otherwise\dots} then the proposition is true. \pspace

\item This is \textit{not} a proposition. There is no definite truth value as the veracity of the inequality depends on the value of $x$. For instance, if $x= -6$, then $x - 3= -6 - 3= -9 \leq 10$, so that the expression is true. However, if $x= 20$, then $x - 3= 20 - 3= 17 \not\leq 10$, so that the expression is false. 
\end{enumerate}



\newpage



% Problem 2
\problem{10} For each of the following, either define appropriate primitive propositions (using $P$, $Q$, $R$, etc.) and write the `statement' using logical connectives, or give an English sentence for the given primitives and `translate' the logical `sentence' into an English sentence:
	\begin{enumerate}[(a)]
	\item $P \to (\neg Q \vee R)$
	\item You will succeed, if you believe and work hard. 
	\item $Q \wedge (\neg P \vee Q)$
	\item I pay rent, or I lose my job and starve. 
	\end{enumerate} \pspace

\sol 
\begin{enumerate}[(a)]
\item There are many possible solutions, depending on the choices for the propositions $P$, $Q$, and $R$. For instance, 

\item 
\item 
\item 
\end{enumerate}



\newpage



% Problem 3
\problem{10} Consider the following compound statement: $\neg (P \to \neg Q) \wedge \neg Q$
	\begin{enumerate}[(a)]
	\item Determine whether the given compound statement is a tautology, contradiction, or neither. Be sure to justify your response. 
	\item Using a truth table, show that the first part of the given compound statement, i.e. $\neg (P \to \neg Q)$, is logically equivalent to $P \wedge Q$. 
	\item By `simplifying' the expression $\neg \big(P \vee \neg (P \wedge Q) \big)$, show that this compound statement is logically equivalent to the compound statement given at the start of the problem.
	\end{enumerate} \pspace

\sol 
\begin{enumerate}[(a)]
\item 
\item 
\item 
\end{enumerate}



\newpage



% Problem 4
\problem{10} Fix a real number $x$. Consider the statement, ``if $x^2 > 4$, then $x > 2$'
	\begin{enumerate}[(a)]
	\item Determine the truth value of this statement with an explanation. 
	\item Rewrite the given statement by defining appropriate primitive propositions and logical connectives. 
	\item Find the negation, converse, and contrapositive of your result from (b).
	\item Rewrite your answers from (c) as English sentences. Then determine the truth value, with explanation, of each of the statements. 
	\end{enumerate} \pspace

\sol 
\begin{enumerate}[(a)]
\item 
\item 
\item 
\item 
\end{enumerate}


\end{document}