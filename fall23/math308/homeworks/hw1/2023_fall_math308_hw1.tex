\documentclass[11pt,letterpaper]{article}
\usepackage[lmargin=1in,rmargin=1in,tmargin=1in,bmargin=1in]{geometry}
\usepackage{../style/homework}
\usepackage{../style/commands}
\setbool{quotetype}{true} % True: Side; False: Under
\setbool{hideans}{true} % Student: True; Instructor: False

% -------------------
% Content
% -------------------
\begin{document}

\homework{1: Due 09/07}{And I knew exactly what to do\dots but in a much more real sense, I had no idea what to do.}{Michael Scott, The Office}

% Problem 1
\problem{10} Determine if each of the following are propositions. If the example is a proposition, state its truth value with a brief justification. If the example is \textit{not} a proposition, briefly explain why:
	\begin{enumerate}[(a)]
	\item $3^2 - 15= 6$
	\item The statement in (c) is false. 
	\item George Orwell wrote \textit{A Remembrance of Things Past}. 
	\item There is intelligent life in the universe.  
	\item $x - 3 \leq 10$
	\end{enumerate}



\newpage



% Problem 2
\problem{10} For each of the following, either define appropriate primitive propositions (using $P$, $Q$, $R$, etc.) and write the `statement' using logical connectives, or give an English sentence for the given primitives and `translate' the logical `sentence' into an English sentence:
	\begin{enumerate}[(a)]
	\item $P \to (\neg Q \vee R)$
	\item You will succeed, if you believe and work hard. 
	\item $Q \wedge (\neg P \vee Q)$
	\item I pay rent, or I lose my job and starve. 
	\end{enumerate}



\newpage



% Problem 3
\problem{10} Consider the following compound statement: $\neg (P \to \neg Q) \wedge \neg Q$
	\begin{enumerate}[(a)]
	\item Determine whether the given compound statement is a tautology, contradiction, or neither. Be sure to justify your response. 
	\item Using a truth table, show that the first part of the given compound statement, i.e. $\neg (P \to \neg Q)$, is logically equivalent to $P \wedge Q$. 
	\item By `simplifying' the expression $\neg \big(P \vee \neg (P \wedge Q) \big)$, show that this compound statement is logically equivalent to the compound statement given at the start of the problem.
	\end{enumerate}



\newpage



% Problem 4
\problem{10} Fix a real number $x$. Consider the statement, ``if $x^2 > 4$, then $x > 2$'
	\begin{enumerate}[(a)]
	\item Determine the truth value of this statement with an explanation. 
	\item Rewrite the given statement by defining appropriate primitive propositions and logical connectives. 
	\item Find the negation, converse, and contrapositive of your result from (b).
	\item Rewrite your answers from (c) as English sentences. Then determine the truth value, with explanation, of each of the statements. 
	\end{enumerate}


\end{document}