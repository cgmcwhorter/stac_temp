\documentclass[11pt,letterpaper]{article}
\usepackage[lmargin=1in,rmargin=1in,tmargin=1in,bmargin=1in]{geometry}
\usepackage{../style/homework}
\usepackage{../style/commands}
\setbool{quotetype}{false} % True: Side; False: Under
\setbool{hideans}{true} % Student: True; Instructor: False

\newcommand{\blank}[1]{\underline{\hspace{#1}}} % Blank Underline

% -------------------
% Content
% -------------------
\begin{document}

\homework{8: Due 10/12}{The study of Mathematics, like the Nile, begins in minuteness but ends in magnificence.}{Charles Caleb Colton}

% Problem 1
\problem{10} Let $A= \{ 2, 6, 8, 10 \}$, $B$ be the set of nonnegative numbers less than 10, and $C$ be the set of perfect squares less than 10. Define $f: A \to \mathbb{Z}$ and $g: B \setminus C$ via $x \to \frac{15(x + 8)}{x}$ and $x \mapsto \frac{5(x^2 - 16x + 88)}{4}$, respectfully. Fully justifying your answer, determine whether $f= g$. 



\newpage



% Problem 2
\problem{10} Define the following real-valued functions:
	\[
	\begin{aligned}
	f(x)&= 2x - 1 &\qquad j(x)&= \dfrac{x - 1}{x + 2} \\
	g(x)&= x^2 + x + 1 & k(x)&= \sin(\pi x) \\
	h(x)&= x 2^x & \ell(x)&= 1 - x^2
	\end{aligned}
	\]
Showing all your work, for each of the following, either compute the function or find a general rule for the given function operation:
	\begin{enumerate}[(a)]
	\item $(f + g)(0)$
	\item $(j - \ell)(2)$
	\item $(gk)(5)$
	\item $\left( \dfrac{f}{j} \right) (3)$
	\item $(h \circ k)(1)$
	\item $(2f + \ell)(x)$
	\item $(fg)(x)$
	\item $\left( \dfrac{h}{f} \right)(x)$
	\item $(k \circ \ell)(x)$
	\item $(\ell \circ g \circ f)(x)$
	\end{enumerate}



\newpage



% Problem 3
\problem{10} Let $f: \mathbb{R} \to \mathbb{R}$ be given by $x \mapsto x^2 + 4x - 5$.
	\begin{enumerate}[(a)]
	\item Determine $f(-5)$.
	\item Compute $f([0,1])$.
	\item Is $16 \in \im f$? Explain. 
	\item Determine $f^{-1}(0)$.
	\item Find the domain, codomain, and range for $f(x)$. 
	\end{enumerate}



\newpage



% Problem 4
\problem{10} Being sure to justify your answer, complete the following:
	\begin{enumerate}[(a)]
	\item Let $f: \mathbb{R} \to \mathbb{R}$ be given by $f(x)= 5 - x^2$. Is $f$ an increasing function? Explain. Is $f$ decreasing function? Explain. 
	\item Let $g: \mathbb{R} \to \mathbb{R}$ be given by $g(x)= 5x - 8$. Is $g$ a positive function? Explain. Is $g$ a negative function? Explain. 
	\item Let $g$ be as in (b) and define $A= [2, \infty)$ and $B= (\infty, 0)$. Is $g \big|_A$ a positive function? Explain. Is $g \big|_B$ a negative function? Explain. 
	\item Let $h: \mathbb{R} \to \mathbb{R}$ be given by\dots
		\[
		h(x)= 
		\begin{cases}
		1 - x, & x < 2 \\
		3x + 5, & x \geq 2
		\end{cases}
		\]
	Find the largest possible interval $S \subseteq \mathbb{R}$ such that $h \big|_S$ is a nondecreasing function. Is $h$ monotone on $S$? Is $h$ strictly monotone on $S$?
	\end{enumerate}


\end{document}