\documentclass[11pt,letterpaper]{article}
\usepackage[lmargin=1in,rmargin=1in,tmargin=1in,bmargin=1in]{geometry}
\usepackage{../style/homework}
\usepackage{../style/commands}
\setbool{quotetype}{false} % True: Side; False: Under
\setbool{hideans}{false} % Student: True; Instructor: False


% -------------------
% Content
% -------------------
\begin{document}

\homework{8: Due 10/12}{The study of Mathematics, like the Nile, begins in minuteness but ends in magnificence.}{Charles Caleb Colton}

% Problem 1
\problem{10} Let $A= \{ 2, 6, 8, 10 \}$, $B$ be the set of nonnegative even numbers that are at most 10, and $C$ be the set of perfect squares less than 10. Define $f: A \to \mathbb{Z}$ and $g: B \setminus C \to \mathbb{Z}$ via $x \to \frac{15(x + 8)}{x}$ and $x \mapsto \frac{5(x^2 - 16x + 88)}{4}$, respectfully. Fully justifying your answer, determine whether $f \equiv g$. \pspace

\sol To show that two functions $f, g$ are equal, i.e. $f = g$ or $f \equiv g$, we need to show that they have the same domain, the same codomain, and their outputs are the same everywhere on their `common domain.'\footnote{Note: This is not the same as the two functions having the same image. For example, take $A= \{ 1, 2 \}$ and $B= \{ a, b \}$. Define $f, g: A \to B$ via $f(1)= a$, $f(2)= b$, and $g(1)= b$ and $g(2)= a$. Clearly, $f, g$ have the same domain and codomains. The image of both $f$ and $g$ are the same---namely, the set $\{ a, b \}$, but observe $a= f(1) \neq g(1)= b$ and $b= f(2) \neq g(2)= a$.} \pspace

{\itshape Equal Domains, $A= B$:} We need to show $A= B$ that is, we need to show that $A$ and $B$ have all the same elements. We know that $A= \{ 2, 6, 8, 10 \}$. Now $B$ is the set of nonnegative even numbers less than 10, i.e. $B= \{ 0, 2, 4, 6, 8, 10 \}$. Furthermore, $C$ is the set of perfect squares less than 10, i.e. $C= \{ 0, 4, 9 \}$. But then $B \setminus C= \{ 2, 6, 8, 10 \}$. Therefore, $A= B \setminus C$. \pspace

{\itshape Equal Codomains, $\mathbb{Z}= \mathbb{Z}$:} It is immediately clear that $f$ and $g$ have the same codomain---namely, $\mathbb{Z}$. \pspace

{\itshape Equivalent on their Common Domain:} To check whether $f$ and $g$ have the same outputs for every element of their `common domain', we can simply compute $f, g$ for the values in $\{ 2, 6, 8, 10 \}$:
	\[
	\begin{aligned}
	f(2)&= \dfrac{15(2 + 8)}{2}= \dfrac{150}{2}= 75 \qquad\qquad& g(2)&= \frac{5(2^2 - 16(2) + 88)}{4}= \dfrac{300}{4}= 75 \\
	f(6)&= \dfrac{15(6 + 8)}{6}= \dfrac{210}{6}= 35 & g(6)&= \frac{5(6^2 - 16(6) + 88)}{4}= \dfrac{140}{4}= 35 \\
	f(8)&= \dfrac{15(8+ 8)}{8}= \dfrac{240}{8}= 30 & g(8)&= \frac{5(8^2 - 16(8) + 88)}{4}= \dfrac{120}{4}= 30 \\
	f(10)&= \dfrac{15(10 + 8)}{10}= \dfrac{270}{10}= 27 & g(10)&= \frac{5(10^2 - 16(10) + 88)}{4}= \dfrac{140}{4}= 35
	\end{aligned}
	\]
Observe that $f(2)= g(2)= 75$, $f(6)= g(6)= 35$, and $f(8)= g(8)= 30$. However, $f(10)= 27 \neq 35= g(10)$. Therefore, $f$ and $g$ do not agree on their `common domain.' \pspace

Because $f$ and $g$ do not agree on their `common domain', $f$ and $g$ are not equal, i.e. $f \not\equiv g$. 



\newpage



% Problem 2
\problem{10} Define the following real-valued functions:
	\[
	\begin{aligned}
	f(x)&= 2x - 1 &\qquad j(x)&= \dfrac{x - 1}{x + 2} \\
	g(x)&= x^2 + x + 1 & k(x)&= \sin(\pi x) \\
	h(x)&= x 2^x & \ell(x)&= 1 - x^2
	\end{aligned}
	\]
Showing all your work, for each of the following, either compute the function at the specified value or find a general rule for the given function operation:
	\begin{enumerate}[(a)]
	\item $(f + g)(0)$
	\item $(j - \ell)(2)$
	\item $(gk)(5)$
	\item $\left( \dfrac{f}{j} \right) (3)$
	\item $(h \circ k)(1)$
	\item $(2f + \ell)(x)$
	\item $(fg)(x)$
	\item $\left( \dfrac{h}{f} \right)(x)$
	\item $(k \circ \ell)(x)$
	\item $(\ell \circ g \circ f)(x)$
	\end{enumerate} \pspace

\sol

\begin{enumerate}[(a)]
\item $(f + g)(0)$
\item $(j - \ell)(2)$
\item $(gk)(5)$
\item $\left( \dfrac{f}{j} \right) (3)$
\item $(h \circ k)(1)$
\item $(2f + \ell)(x)$
\item $(fg)(x)$
\item $\left( \dfrac{h}{f} \right)(x)$
\item $(k \circ \ell)(x)$
\item $(\ell \circ g \circ f)(x)$
\end{enumerate}



\newpage



% Problem 3
\problem{10} Let $f: \mathbb{R} \to \mathbb{R}$ be given by $x \mapsto x^2 + 4x - 5$.
	\begin{enumerate}[(a)]
	\item Determine $f(-5)$.
	\item Compute $f([0,1])$.
	\item Is $16 \in \im f$? Explain. 
	\item Determine $f^{-1}(0)$.
	\item Find the domain, codomain, and range for $f(x)$. 
	\end{enumerate} \pspace

\sol 
\begin{enumerate}[(a)]
\item 
\item 
\item 
\item 
\item 
\end{enumerate}



\newpage



% Problem 4
\problem{10} Being sure to justify your answer, complete the following:
	\begin{enumerate}[(a)]
	\item Let $f: \mathbb{R} \to \mathbb{R}$ be given by $f(x)= 5 - x^2$. Is $f$ an increasing function? Explain. Is $f$ a decreasing function? Explain. 
	\item Let $g: \mathbb{R} \to \mathbb{R}$ be given by $g(x)= 5x - 8$. Is $g$ a positive function? Explain. Is $g$ a negative function? Explain. 
	\item Let $g$ be as in (b) and define $A= [2, \infty)$ and $B= (-\infty, 0)$. Is $g \big|_A$ a positive function? Explain. Is $g \big|_B$ a negative function? Explain. 
	\item Let $h: \mathbb{R} \to \mathbb{R}$ be given by\dots
		\[
		h(x)= 
		\begin{cases}
		1 - x, & x < 2 \\
		3x + 5, & x \geq 2
		\end{cases}
		\]
	Find the largest possible interval $S \subseteq \mathbb{R}$ such that $h \big|_S$ is a nondecreasing function. Is $h$ monotone on $S$? Is $h$ strictly monotone on $S$?
	\end{enumerate} \pspace

\sol 
\begin{enumerate}[(a)]
\item 
\item 
\item 
\item 
\end{enumerate}


\end{document}