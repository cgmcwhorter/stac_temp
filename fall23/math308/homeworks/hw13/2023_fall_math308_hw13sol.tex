\documentclass[11pt,letterpaper]{article}
\usepackage[lmargin=1in,rmargin=1in,tmargin=1in,bmargin=1in]{geometry}
\usepackage{../style/homework}
\usepackage{../style/commands}
\setbool{quotetype}{false} % True: Side; False: Under
\setbool{hideans}{false} % Student: True; Instructor: False

\newcommand{\blank}[1]{\underline{\hspace{#1}}} % Blank Underline

% -------------------
% Content
% -------------------
\begin{document}

\homework{13: Due 11/10}{The difference between mathematicians and physicists is that after physicists prove a big result they think it is fantastic but after mathematicians prove a big result they think it is trivial.}{Lucien Szpiro}

% Problem 1
\problem{10} Showing all your work, compute the following:
	\begin{enumerate}[(a)]
	\item $45 - 69 \mod 27$
	\item $115 + 82 \mod 6$
	\item $11 \cdot 17 \mod 3$
	\item $2^{100} \mod 5$
	\item $-17 \cdot 14 \mod 8$
	\end{enumerate} \pspace

\sol 
\begin{enumerate}[(a)]
\item We have $45 - 69= -24 \equiv 3 \mod 27$ because $-24= -1(27) + 3$. Alternatively, $45 - 69 \equiv 18 - 15= 3 \mod 27$. \pspace

\item We have $115 + 82= 197 \equiv 5 \mod 6$ because $197= 32(6) + 5$. Alternatively, $115 + 82 \equiv 1 + 4= 5 \mod 6$. \pspace

\item We have $11 \cdot 17= 187 \equiv 1$ because $187= 62(3) + 1$. Alternatively, $11 \cdot 17 \equiv 2 \cdot 2= 4 \equiv 1 \mod 3$. \pspace

\item We have\dots
	\[
	2^1= 2 \equiv 2 \mod 5, \qquad 2^2= (2^1)^2 \equiv 2^2= 4 \equiv 4 \mod 5, \qquad 2^4= (2^2)^2 \equiv 4^2= 16 \equiv 1 \mod 5
	\]
We then have\dots
	\[
	2^{100}= (2^4)^{25} \equiv 1^{25}= 1 \mod 5
	\] \pspace

\item We have $-17 \cdot 14= -238 \equiv 2 \mod 8$ because $-238= -30(8) + 2$. Alternatively, we have $-17 \cdot 14 \equiv 7 \cdot 6= 42 \equiv 2 \mod 8$ or $-17 \cdot 14 \equiv -1 \cdot -2= 2 \mod 8$. 
\end{enumerate}



\newpage



% Problem 2
\problem{10} Showing all your work, compute the following:
	\begin{enumerate}[(a)]
	\item $\phi(143)$
	\item $\phi(64)$
	\item $\phi(660)$
	\end{enumerate} \pspace

\sol Recall that the Euler-$\phi$ function (or Euler totient function, also denoted $\varphi$) is defined as follows: $\phi(n)$ is the number of integers $k$ in the range $1 \leq k \leq n$ such that $\gcd(k, n)= 1$. Recall that if $p$ is prime, then $\phi(p)= p - 1$. Generally, if $p$ is prime and $k \geq 1$, then $\phi(p^k)= p^k - p^{k-1}= p^{k-1} (p - 1)$. Finally, if $a$, $b$ are relatively prime (or coprime), i.e. $\gcd(a, b)= 1$, then $\phi(ab)= \phi(a) \phi(b)$. But if $N= p_1^{a_1} p_2^{a_2} \cdots p_n^{a_n}$ is a prime factorization, then\dots
	\[
	\hspace{-1cm} \phi(N)= \phi(p_1^{a_1} p_2^{a_2} \cdots p_n^{a_n})= \phi(p_1^{a_1}) \phi(p_2^{a_2}) \cdots \phi(p_n^{a_n})= p_1^{a_1 - 1} (p_1 - 1) \cdot p_2^{a_2 - 1} (p_2 - 1) \cdot \cdots \cdot p_n^{a_n - 1} (p_n - 1)= \prod_{i=1}^n p_i^{a_i - 1} (p_i - 1)
	\]

\begin{enumerate}[(a)]
\item We have\dots
	\[
	\phi(143)= \phi(11 \cdot 13)= \phi(11) \phi(13)= (11 - 1)(13 - 1)= 10 \cdot 12= 120 
	\] \pspace

\item We have\dots
	\[
	\phi(64)= \phi(2^6)= 2^{6 - 1} (2 - 1)= 2^5 (2 - 1)= 32 \cdot 1= 32
	\] \pspace

\item We have\dots
	\[
	\hspace{-1cm} \phi(660)= \phi(2^2 \cdot 3^1 \cdot 5^1 \cdot 11^1)= 2^{2 - 1} (2 - 1) \cdot 3^{1 - 1} (3 - 1) \cdot 5^{1 - 1} (5 - 1) \cdot 11^{1 - 1} (11 - 1)= 2(1) \cdot 1(2) \cdot 1(4) \cdot 1(10)= 160 
	\]
\end{enumerate}



\newpage



% Problem 3
\problem{10} Showing all your work, complete the following:
	\begin{enumerate}[(a)]
	\item The number of digits in $96758^{2023}$.
	\item What is the remainder when $19^{115}$ is divided by 5.
	\item What are the last two digits of $178^{996}$?
	\end{enumerate} \pspace

\sol 
\begin{enumerate}[(a)]
\item The number of digits in $N$ when expressed in base-$b$ is $\floor*{\log_b N} + 1$. Recalling the change of base formula $\log_b x= \frac{\ln x}{\ln b}$ and the power formula $\log_b(x^n)= n \log_b x$, we have\dots
	\[
	\begin{aligned}
	\floor*{\log_{10} (96758^{2023}) } + 1&= \floor*{2023 \log_{10} (96758) } + 1 \\
	&= \floor*{2023\; \dfrac{\ln(96758)}{\ln(10)} } + 1 \\
	&= \floor*{2023 \cdot \dfrac{11.47997}{2.302585}} + 1 \\
	&= \floor*{2023 \cdot 4.98569} + 1 \\
	&= \floor*{10086.1} + 1 \\
	&= 10086 + 1 \\
	&= 10087
	\end{aligned}
	\]
Therefore, $96758^{2023}$ has $10,\!087$ digits. \pspace

\item The remainder of $N$ when divided by $b$ is precisely the value of $N \mod b$. Observe that $19= 20 - 1$, which implies $19= 20 - 1 \equiv 0 - 1= -1 \mod 5$. But then we have $19^{115} \equiv (-1)^{115}= -1 \equiv 4 \mod 5$. Therefore, $19^{115}$ has a remainder of 4 when divided by 5. \pspace

\item The last two digits of an integer $N$ in base-$10$ is the remainder of $N$ when divided by 100, i.e. the value of $N \mod 100$. Now observe\dots
	\[
	\begin{aligned}
	178^1&= 178 \equiv 78 \mod 100 &\qquad 178^{32}&= (178^{16})^2 \equiv 96^2= 9216 \equiv 16 \mod 100 \\
	178^2&= (178^1)^2 \equiv 78^2= 6084 \equiv 84 \mod 100 & 178^{64}&= (178^{32})^2 \equiv 16^2= 256 \equiv 56 \mod 100 \\
	178^4&= (178^2)^2 \equiv 84^2= 7056 \equiv 56 \mod 100 & 178^{128}&= (178^{64})^2 \equiv 56^2 \equiv 36 \mod 100 \\
	178^8&= (178^4)^2 \equiv 56^2= 3136 \equiv 36 \mod 100 & 178^{256}&= (178^{128})^2 \equiv 36^2 \equiv 96 \mod 100 \\
	178^{16}&= (178^8)^2 \equiv 36^2= 1296 \equiv 96 \mod 100 & 178^{512}&= (178^{256})^2 \equiv 92^2 \equiv 16 \mod 100
	\end{aligned}
	\]
Now observe that $996= 512 + 256 + 128 + 64 + 32 + 4$. But then\dots
	\[
	\begin{aligned}
	178^{996}&= 178^{4 + 32 + 64 + 128 + 256 + 512} \\
	&= 178^4 \cdot 178^{32} \cdot 178^{64} \cdot 178^{128} \cdot 178^{256} \cdot 178^{512} \\
	&\equiv 56 \cdot 16 \cdot 56 \cdot 36 \cdot 96 \cdot 16 \\
	&= 2774532096 \\
	&\equiv 96
	\end{aligned}
	\]
Therefore, the last two digits of $178^{996}$ are $96$. 
\end{enumerate}



\newpage



% Problem 4
\problem{10} Consider the congruence $18x + 27 \equiv 5 \mod 31$. 
	\begin{enumerate}[(a)]
	\item Explain why the given congruence has a solution. 
	\item Explain why $18^{-1}$ exists mod 31. 
	\item Solve the congruence and give at least three explicit solutions. 
	\item Verify that one of your solutions in (c) is correct. 
	\end{enumerate} \pspace

\sol 
\begin{enumerate}[(a)]
\item Consider the linear congruence $ax \equiv b \mod n$. Let $\gcd(a, n)= d$. If $d \nmid b$, then there are no solutions. However, if $d \mid b$, there are infinitely many solutions and the solutions are $\frac{xb}{d} + \frac{n}{d}\,z$, where $z \in \mathbb{Z}$ and $x$ is such that for some $y$, $d= ax + ny$. When $d= 1$, we can express this simply using the inverse: the solutions modulo $n$ are $x \equiv a^{-1}b$, where $a^{-1}$ is the inverse of $a$ modulo $n$ (which exists because $\gcd(a, n)= 1$). Let $s$ be the integer $1 \leq s \leq n$ such that $s \equiv a^{-1}b$ modulo $n$. Then general solutions are $x= s + zn$, where $z \in \mathbb{Z}$. We have\dots
	\[
	\begin{aligned}
	18x + 27 &\equiv 5 \mod 31 \\
	18x &\equiv -22 \mod 31 \\
	18x &\equiv 9 \mod 31
	\end{aligned}
	\]
We have $\gcd(18, 31)= 1$ and $1 \mid 9$. Therefore, there is a solution to the given linear congruence. \pspace

\item Let $a, n \in \mathbb{Z}$ with $n > 0$. Suppose that $\gcd(a, n)= d > 1$ and let $dk= a$ and $dk'= n$ for some integers $k, k'$. Clearly, $k, k' \neq 0$ and $0 \leq k < a$, $0 \leq k' < n$. We know that $a^{-1}$ cannot exist modulo $n$. If there were an integer, $m$, such that $ma \equiv 1 \mod n$, then\dots
	\[
	k' \equiv k' \cdot 1 \equiv k' \cdot ma= k' \cdot m(dk)= (mk) \cdot (dk')= mk \cdot n \equiv mk \cdot 0 \equiv 0
	\]
Because $k' \equiv 0 \mod n$, we know that $n \mid k'$. But because $0 \leq k' < n$, this is impossible unless $k'= 0$, which is a contradiction. Therefore, there does not exist an integer $m$ such that $ma \equiv 1 \mod n$, i.e. $a^{-1}$ does not exist modulo $n$. 

Now suppose that $\gcd(a, n)= 1$. Given two integers $a, b$, there exist integers $x, y$ such that $\gcd(a, b)= ax + by$. But then there exists integers $x, y$ such that $ax + ny= 1$. Reducing this modulo $n$, we have $1= ax + ny \equiv ax + 0 = ax$. But then $x$ is an integer such that $xa \equiv 1$ modulo $n$. Therefore, $a^{-1}$ exists. 

All the work above shows that $a^{-1}$ exists modulo $n$ if and only if $\gcd(a, n)= 1$. Because $\gcd(18, 31)= 1$, we know that $18^{-1}$ exists modulo $31$. \pspace

\item From (b), we know that $18^{-1}$ exists modulo $31$. Moreover, if $x, y$ are integers such that $ax + ny= 1$, then $x= a^{-1}$ modulo $n$. We need find integers $x, y$ such that $18x + 31y= 1$. We can find these using the (extended) Euclidean algorithm:
	\[
	\begin{aligned}
	31&= 1(18) + 13 &\hspace{2cm} 1&= 3 - 1(2) \\
	18&= 1(13) + 5 & &= 3 - 1 \big(5 - 1(3) \big) \\
	13&= 2(5) + 3 & &= 3 - 1 \cdot 5 + 1 \cdot 3 \\
	5&= 1(3) + 2 & &= 2 \cdot 3 - 1 \cdot 5 \\
	3&= 1(2) + 1 & &= 2 \cdot \big(13 - 2(5) \big) - 1 \cdot 5 \\
	2&= 2(1) & &= 2 \cdot 13 - 4 \cdot 5 - 1 \cdot 5 \\
	&&&= 2 \cdot 13 - 5 \cdot 5 \\
	&&&= 2 \cdot 13 - 5 \big(18 - 1(13) \big) \\
	&&&= 2 \cdot 13 - 5 \cdot 18 + 5 \cdot 13 \\
	&&&= 7 \cdot 13 - 5 \cdot 18 \\
	&&&=  7 \big(31 - 1(18) \big) - 5 \cdot 18 \\
	&&&= 7 \cdot 31 - 7 \cdot 18 - 5 \cdot 18 \\
	&&&= 7 \cdot 31 - 12 \cdot 18
	\end{aligned}
	\]
But then $1= 7(31) + (-12)18 \equiv 0 + (-12)18= (-12)18 \mod 31$. Therefore, $18^{-1}= -12 \equiv 19 \mod 31$. We can verify this easily: $19(18)= 342 \equiv 1 \mod 31$. But then\dots
	\[
	\begin{aligned}
	18x + 27 &\equiv 5 \mod 31 \\
	18x &\equiv -22 \mod 31 \\
	18x &\equiv 9 \mod 31 \\
	18^{-1} \cdot 18x &\equiv 18^{-1} \cdot 9 \mod 31 \\
	x &\equiv -12 \cdot 9 \mod 31 \\
	x &\equiv 19 \cdot 9 \mod 31 \\
	x &\equiv 171 \mod 31 \\
	x &\equiv 16
	\end{aligned}
	\]
Therefore, the solution modulo $31$ is $16$. The general solutions are the integers of the form $16 + 31z$, where $z \in \mathbb{Z}$. But then choosing $k= -3, -2, -1, 0, 1, 2, 3$, we know that $-77, -46, -15, 16, 47, 78, 109$ are all solutions to the given equation, respectively. \pspace

\item We select the general solution $-77$. We have\dots
	\[
	18x + 27= 18(-77) + 27= -1386 + 27= -1359 \equiv 5 \mod 31,
	\]
where $-1359 \equiv 5 \mod 31$ because $-1359= -44(31) + 5$. Alternatively, we could have selected the general solution $109$. We have\dots
	\[
	18x + 27= 18(109) + 27= 1962 + 27= 1989 \equiv 5 \mod 31,
	\]
where $1989 \equiv 5 \mod 31$ because $1989= 64(31) + 5$.
\end{enumerate}


\end{document}