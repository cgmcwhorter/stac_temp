\documentclass[11pt,letterpaper]{article}
\usepackage[lmargin=1in,rmargin=1in,tmargin=1in,bmargin=1in]{geometry}
\usepackage{../style/homework}
\usepackage{../style/commands}
\setbool{quotetype}{false} % True: Side; False: Under
\setbool{hideans}{true} % Student: True; Instructor: False

\newcommand{\blank}[1]{\underline{\hspace{#1}}} % Blank Underline

% -------------------
% Content
% -------------------
\begin{document}

\homework{13: Due 11/10}{The difference between mathematicians and physicists is that after physicists prove a big result they think it is fantastic but after mathematicians prove a big result they think it is trivial.}{Lucien Szpiro}

% Problem 1
\problem{10} Showing all your work, compute the following:
	\begin{enumerate}[(a)]
	\item $45 - 69 \mod 27$
	\item $115 + 82 \mod 6$
	\item $11 \cdot 17 \mod 3$
	\item $2^{100} \mod 5$
	\item $-17 \cdot 14 \mod 8$
	\end{enumerate}



\newpage



% Problem 2
\problem{10} Showing all your work, compute the following:
	\begin{enumerate}[(a)]
	\item $\phi(143)$
	\item $\phi(64)$
	\item $\phi(660)$
	\end{enumerate}



\newpage



% Problem 3
\problem{10} Showing all your work, complete the following:
	\begin{enumerate}[(a)]
	\item The number of digits in $96758^{2023}$.
	\item What is the remainder when $19^{115}$ is divided by 5.
	\item What are the last two digits of $178^{996}$?
	\end{enumerate}



\newpage



% Problem 4
\problem{10} Consider the congruence $18x + 27 \equiv 5 \mod 31$. 
	\begin{enumerate}[(a)]
	\item Explain why the given congruence has a solution. 
	\item Explain why $18^{-1}$ exists mod 31. 
	\item Solve the congruence and give at least three explicit solutions. 
	\item Verify that one of your solutions in (c) is correct. 
	\end{enumerate}


\end{document}