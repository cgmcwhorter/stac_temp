\documentclass[11pt,letterpaper]{article}
\usepackage[lmargin=1in,rmargin=1in,tmargin=1in,bmargin=1in]{geometry}
\usepackage{../style/homework}
\usepackage{../style/commands}
\setbool{quotetype}{false} % True: Side; False: Under
\setbool{hideans}{false} % Student: True; Instructor: False

\newcommand{\blank}[1]{\underline{\hspace{#1}}} % Blank Underline
\usepackage{colortbl}	% Table Cell Shading

% -------------------
% Content
% -------------------
\begin{document}

\homework{12: Due 11/10}{No part of mathematics is ever, in the long run, ``useless.'' Most of number theory has very few ``practical'' applications. That does not reduce its importance, and if anything it enhances its fascination. No one can predict when what seems to be a most obscure theorem may suddenly be called upon to play some vital and hitherto unsuspected role.}{C. Stanley}

% Problem 1
\problem{10} Showing all your work and fully justifying your answer, complete the following:
	\begin{enumerate}[(a)]
	\item From the definition, determine whether 79 is odd or even. 
	\item From the definition, determine whether or not $-343$ a perfect cube.
	\item Find the prime factorization of 840. Find all the divisors of 840. 
	\item Can an integer of the form $n^2 + 7n + 6$, where $n \in \mathbb{N}$, be prime? Explain. 
	\end{enumerate} \pspace

\sol 
\begin{enumerate}[(a)]
\item We know an integer $n$ is odd if there exists an integer $k$ such that $n= 2k + 1$. Observe that taking $k= 39$, we have $2(39) + 1= 79$. Therefore, $79$ is odd. Recall also that an integer $n$ is even if there exists an integer $k$ such that $n= 2k$. Suppose that 79 is even. Then there exists an integer $k$ such that $79= 2k$. As $78 < 2k < 80$, it must be that $39 < k < 40$. But there is no such integer. Therefore, there is no integer $k$ such that $2k= 79$, so that $79$ cannot be even. \pspace

\item An integer $n$ is a perfect cube if there exists an integer $k$ such that $n= k^3$. Observe that taking $k= -7$, we have $(-7)^3= -343$. Therefore, $-343$ is a perfect cube. \pspace

\item Observe that $840= 84 \cdot 10= (4 \cdot 21) \cdot 10= \big( (2 \cdot 2) \cdot (3 \cdot 7) \big) \cdot (2 \cdot 5)= 2^3 \cdot 3 \cdot 5 \cdot 7$. The prime factorization of $840$ is then $2^3 \cdot 3 \cdot 5 \cdot 7$. If $d \mid 840$, then either $d= 1$ or $d$ is a product of prime factors of $840$. But then the divisors of $840$ are the integers of the form $2^a \cdot 3^b \cdot 5^c \cdot 7^d$, where $a \in \{ 0, 1, 2, 3 \}$ and $b, c, d \in \{ 0, 1 \}$. But then the divisors of $840$ are $1$, $2$, $3$, $4$, $5$, $6$, $7$, $8$, $10$, $12$, $14$, $15$, $20$, $21$, $24$, $28$, $30$, $35$, $40$, $42$, $56$, $60$, $70$, $84$, $105$, $120$, $140$, $168$, $210$, $280$, $420$, $840$. \pspace

\item Observe that $n^2 + 7n + 6= (n + 1)(n + 6)$. For a whole number $n \in \mathbb{N}$, $n + 1$ and $n + 6$ are integers. However, clearly, if $n > 0$, $n + 1$ and $n + 6$ are at least 2. But then $n + 1$ and $n + 6$ are either prime or products of primes by the Fundamental Theorem of Arithmetic. But then $n^2 + 7n + 6= (n + 1)(n + 6)$ is a product of primes. Therefore, $n^2 + 7n + 6$ cannot be prime. 
\end{enumerate}



\newpage



% Problem 2
\problem{10} Using the given $a, b$, express the division $\frac{b}{a}$ using the division algorithm. Be sure to show all your work. 
	\begin{enumerate}[(a)]
	\item $a= 16$, $b= 2797$
	\item $a= -29$, $b= -7015$
	\item $a= 56$, $b= 55664$
	\end{enumerate} \pspace

\sol Recall that the Division Algorithm states that for $a, b \in \mathbb{Z}$ with $a \neq 0$, there are unique $q, r \in \mathbb{Z}$ with $0 \leq r < |a|$ such that $b= qa + r$. If $q$ is known, we can take $r= b - qa$. Recall that we can find $q$ via\dots
	\[
	q= 
	\begin{cases}
	\floor*{\dfrac{b}{a}}, & a > 0 \\
	\\
	\ceil*{\dfrac{b}{a}}, & a < 0
	\end{cases}
	\]

\begin{enumerate}[(a)]
\item Because $a > 0$, we have $q= \floor*{\dfrac{2797}{16}}= 174$ so that $r= 2797 - 174 \cdot 16= 2797 - 2784= 13$. But then we have $2797= 174(16) + 13$. \pspace

\item Because $a < 0$, we have $q= \ceil*{\dfrac{-7015}{-29}}= 242$ so that $r= -7015 - 242(-29)= -7015 + 7018= 3$. Therefore, $-7015= 242(-29) + 3$. \pspace

\item Because $a > 0$, we have $q= \floor*{\dfrac{55664}{56}}= 994$ so that $r= 55664 - 994 \cdot 56= 55664 - 55664= 0$. Therefore, $55664= 994(56) + 0$. 
\end{enumerate}



\newpage



% Problem 3
\problem{10} Showing all your work and fully justifying your reasoning, complete the following: 
	\begin{enumerate}[(a)]
	\item Use the Euclidean Algorithm to find $\gcd(459, 303)$. 
	\item Use the extended Euclidean algorithm, express $\gcd(459, 303)$ as a linear combination of 459 and 303. 
	\item Is it possible to find integers $x, y$ such that $459x + 303y= 5$? If not, explain why. If so, find them.  
	\item Is it possible to find integers $x, y$ such that $459x + 303y= 6$? If not, explain why. If so, find them.  
	\end{enumerate} \pspace

\sol 
\begin{enumerate}[(a)]
\item Recall that the Euclidean Algorithm computes $\gcd(a, b)$. Each step of the Euclidean Algorithm uses the division algorithm to express $r_{k-2}= q_k r_{k-1} + r_k$, where $0 \leq r_{k-1} < r_k$, and we take $r_{-1}= b$ and $r_{-2}= a$. The algorithm continues until $r_n= 0$ is obtained. We then have $\gcd(a, b)= r_{n-1}$. 
	\[
	\begin{aligned}
	495&= 1(303) + 192 \\
	303&= 1(192) + 111 \\
	192&= 1(111) + 81 \\
	111&= 1(81) + 30 \\
	81&= 2(30) + 21 \\
	30&= 1(21) + 9 \\
	21&= 2(9) + 3 \\
	9&= 3(3)
	\end{aligned}
	\]
Therefore, $\gcd(459, 303)= 3$. \pspace

\item Given the outputs of the Euclidean Algorithm, terminating with $r_n= 0$, the extended Euclidean algorithm expresses $r_k$ as a linear combination of $r_{k-1}$ and $r_{k-2}$ beginning with $r_{n-1}$. When this process terminates, one has expressed $\gcd(a, b)$ as linear combination of $a, b$. 
	\[
	\begin{aligned}
	3&= 21 - 2(9) \\
	&= 21 - 2 \big(30 - 1(21) \big) \\
	&= 21 - 2 \cdot 30 + 2(21) \\
	&= 3 \cdot 21 - 2 \cdot 30 \\
	&= 3 \big(81 - 2(30) \big) - 2 \cdot 30 \\
	&= 3 \cdot 81 - 6(30) - 2 \cdot 30 \\
	&= 3 \cdot 81 - 8(30) \\
	&= 3 \cdot 81 - 8 \big(111 - 1(81) \big) \\
	&= 3 \cdot 81 - 8 \cdot 111 + 8(81) \\
	&= 11 \cdot 81 - 8 \cdot 111 \\
	&= 11 \big(192 - 1(111) \big) - 8 \cdot 111 \\
	\end{aligned}
	\]
	\[
	\begin{aligned}
	&= 11 \cdot 192 - 11 \cdot 111 - 8 \cdot 111 \\
	&= 11 \cdot 192 - 19 \cdot 111 \\
	&= 11 \cdot 192 - 19 \big(303 - 1(192) \big) \\
	&= 11 \cdot 192 - 19 \cdot 303 + 19 \cdot 192 \\
	&= 30 \cdot 192 - 19 \cdot 303 \\
	&= 30 \big(495 - 1(303) \big) - 19 \cdot 303 \\
	&= 30 \cdot 495 - 30 \cdot 303 - 19 \cdot 303 \\
	&= 30 \cdot 495 - 49 \cdot 303
	\end{aligned}
	\]
But then we have $3= 30 \cdot 495 + (-49) \cdot 303$. \pspace

\item Recall that the smallest possible \textit{positive} integer that is expressible as a linear combination of integers $a, b$ is $\gcd(a, b)$; that is, given $a, b \in \mathbb{Z}$, the smallest positive integer of the form $ax + by$, where $x, y \in \mathbb{Z}$, is $\gcd(a, b)$. Furthermore, if $a, b \in \mathbb{Z}$ are nonzero and $n= ax + by$ for some $x, y \in \mathbb{Z}$, then $\gcd(a, b) \mid n$. 

By the observations above, if there existed $x, y \in \mathbb{Z}$ such that $459x + 303y= 5$, then $\gcd(459, 303) \mid 5$. We know that $\gcd(459, 303)= 3$. As $3 \nmid 5$, there do not exist integers $x, y$ such that $459x + 303y= 5$. \pspace

\item We know that the extended Euclidean algorithm expresses $\gcd(a, b)$ as a linear combination of $a, b$. That is, there exist $x, y \in \mathbb{Z}$ such that $\gcd(a, b)= ax + by$. Suppose $\gcd(a, b) \mid n$, i.e. there exists $k \in \mathbb{Z}$ such that $k \gcd(a, b)= n$. But then $n= k \gcd(a, b)= a(kx) + b(ky)$. So if $\gcd(a, b) \mid n$, $n$ is expressible as a linear combination of $a, b$. We know that $3= 30 \cdot 495 - 49 \cdot 303$. But then\dots
	\[
	6= 2 \cdot 3= 2 \big( 30 \cdot 495 - 49 \cdot 303 \big)= 60 \cdot 495 - 98 \cdot 303
	\]
But then $6= 60 \cdot 495 + (-98) \cdot 303$. Therefore, taking $x= 60$ and $y= -98$, we have $459x + 301y= 6$. 
\end{enumerate}



\newpage



% Problem 4
\problem{10} Showing all your work, convert the given base-10 number, binary number, or hexadecimal to the given base $b$:
	\begin{enumerate}[(a)]
	\item $7187$; $b= 16$
	\item $119$; $b= 2$
	\item $11101_2$; $b= 10$
	\item $801\text{c}$; $b= 10$
	\end{enumerate} \pspace

\sol There are two obvious approaches to express $N \geq 0$ in base-$b$. First, one can use a greedy algorithm approach: compute the powers of $b$ that are at most $N$. Assume these are $b^0, b^1, b^2, \ldots, b^n$. Now for $k \geq 0$, repeatedly apply the division algorithm to express $y_k$ as $y_k= q_k b^{n - k} + r_k$, where $0 \leq r_k < b^{n - k}$, $y_0= N$, and $y_{k+1}= r_k$. This process terminates at step $n+1$. Expressing $q_k$ in base-$b$, we have $N= q_0 q_1 \cdots q_n$, so that $q_k$ is the $b^{n - k}$th's place of $N$ in base-$b$. \pspace

The second method makes this process more `algorithmic.'We iteratively apply the division algorithm: to express $N$ in base $b$, for $k \geq 1$, we express $b_k$ as $b_k= q_k b + r_k$ until $b_k= 0$, where $q_k$, $r_k$ are found by applying the division algorithm to $b_k$ and $16$, $b_1= N$, and $b_{k+1}= q_k$. If this process terminates at step $n$, then in base-$b$, $N= r_n r_{n-1} \cdots r_1$, so that $r_i$ is the $b^{i-1}$th's place of $N$ in base-$b$.

\begin{enumerate}[(a)]
\item Using the first method, observe that $16^0= 1$, $16^1= 16$, $16^2= 256$, and $16^3= 4096$. Now $7187/16^3= 7187/4096 \approx 1.75$ and $7187 - 1(4096)= 3091$. Now $3091/16^2= 3091/256 \approx 12.07$ and $3091 - 12(256)= 19$. [Note that in base-16, $12= \text{c}$.] Now $19/16^1= 19/16 \approx 1.19$ and $19 - 1(16)= 3$. Finally, $3/16^0= 3/1= 3$ and $3 - 3(1)= 0$. Therefore, $7187_{10}= 1\text{c}13_{16}$. Using the second method, we have\dots
	\begin{table}[H]
	\centering
	\begin{tabular}{cc|c}
	\multicolumn{1}{c|}{$16$} & $7187$ & \cellcolor[HTML]{d3d3d3} \\ \cline{2-3} 
	& $449$ & $3$ \\
	& $28$ & $1$ \\
	& $1$ & $12= \text{c}_{16}$ \\
	& $0$ & $1$
	\end{tabular} \\[0.3cm]
	$7187_{10}= 1\text{c}13_{16}$
	\end{table} \pspace

\item Using the first method, observe that $2^0= 1$, $2^1= 2$, $2^2= 4$, $2^3= 8$, $2^4= 16$, $2^5= 32$, and $2^6= 64$. Now $119/2^6= 119/64 \approx 1.86$ and $119 - 1(64)= 55$. Then $55/2^5= 55/32= 1.72$ and $55 - 1(32)= 23$. Then $23/2^4= 23/16= 1.44$ and $23 - 1(16)= 7$. Then $7/2^3= 7/8= 0.875$ and $7 - 0(8)= 7$. Then $7/2^2= 7/4= 1.75$ and $7 - 1(4)= 3$. Then $3/2^1= 3/2= 1.5$ and $3 - 1(2)= 1$. Finally, $1/2^0= 1/2= 0.5$ and $1 - 0(2)= 1$. Therefore, $119_{10}= 1110111_2$. Using the second method, we have\dots
	\begin{table}[H]
	\centering
	\begin{tabular}{cc|c}
	\multicolumn{1}{c|}{$2$} & $119$ & \cellcolor[HTML]{d3d3d3} \\ \cline{2-3} 
	& $59$ & $1$ \\
	& $29$ & $1$ \\
	& $14$ & $1$ \\
	& $7$ & $0$ \\
	& $3$ & $1$ \\
	& $1$ & $1$ \\
	& $0$ & $1$
	\end{tabular} \\[0.3cm]
	$119_{10}= 1110111_2$
	\end{table} \pspace

\item Recall to express $N_b= a_n a_{n-1} \cdots a_1 a_0$ in base-$10$, we compute $\sum_{i=0}^n a_i \cdot b^i$, where $a_i$ and $b_i$ are expressed in base-$10$. We then have\dots 
	\[
	\begin{aligned}
	11101_2&= 1 \cdot 2^0 + 0 \cdot 2^1 + 1 \cdot 2^2 + 1 \cdot 2^3 + 1 \cdot 2^4 \\[0.3cm]
	&= 1 \cdot 1 + 0 \cdot 2 + 1 \cdot 4 + 1 \cdot 8 + 1 \cdot 16 \\[0.3cm]
	&= 1 + 0 + 4 + 8 + 16 \\[0.3cm]
	&= 29
	\end{aligned} 
	\] \pspace

\item Recall to express $N_b= a_n a_{n-1} \cdots a_1 a_0$ in base-$10$, we compute $\sum_{i=0}^n a_i \cdot b^i$, where $a_i$ and $b_i$ are expressed in base-$10$. Recall also that when expressed in base-$10$, the base-$16$ (hexadecimal) numbers a--f are $\text{a}= 10$, $\text{b}= 11$, $\text{c}= 12$, $\text{d}= 13$, $\text{e}= 14$, and $\text{f}= 15$. We then have\dots 
	\[
	\begin{aligned}
	801\text{c}_{16}&= c_{16} \cdot 16^0 + 1 \cdot 16^1 + 0 \cdot 16^2 + 8 \cdot 16^3 \\[0.3cm]
	&= 12 \cdot 1 + 1 \cdot 16 + 0 \cdot 256 + 8 \cdot 4096 \\[0.3cm]
	&= 12 + 16 + 0 + 32768 \\[0.3cm]
	&= 32796
	\end{aligned} 
	\]  
\end{enumerate}


\end{document}