\documentclass[11pt,letterpaper]{article}
\usepackage[lmargin=1in,rmargin=1in,tmargin=1in,bmargin=1in]{geometry}
\usepackage{../style/homework}
\usepackage{../style/commands}
\setbool{quotetype}{false} % True: Side; False: Under
\setbool{hideans}{true} % Student: True; Instructor: False

\newcommand{\xor}{\,\underline{\vee}\,}

% Logical Circuits
\usepackage{circuitikz}
\usetikzlibrary{shapes.gates.logic.US,shapes.gates.logic.IEC}

% -------------------
% Content
% -------------------
\begin{document}

\homework{4: Due 09/21}{Most of the mistakes in thinking are inadequacies of perception rather than mistakes of logic.}{Edward de Bono}

% Problem 1
\problem{10} Suppose that $P(x)$ is a predicate with nonempty universe $\mathcal{U}$. Being sure to justify your answer, explain whether the following statements are true or false.
	\begin{enumerate}[(a)]
	\item There are values of $x$ for which $\forall x, P(x)$ is true and values for which the statement is false. 
	\item If $\forall x, P(x)$ is false, then the statement $\exists x, \neg P(x)$ must be true. 
	\item If $\exists x\, P(x)$ is true, then so is $\forall x\, P(x)$.
	\end{enumerate} 



\newpage



% Problem 2
\problem{10} Showing all your work and simplifying your logical expression as much as possible, negate the following quantified open statements:
	\begin{enumerate}[(a)]
	\item $\forall x\, \exists y\, \forall z P(x, y, z)$
	\item $\exists x \big( \neg P(x) \wedge Q(x) \big)$
	\item $\forall x \big( \neg P(x) \to Q(x) \big)$
	\item $\exists y\, \forall x \big( P(x, y) \wedge \neg Q(x, y) \big)$
	\item $\exists x \big( P(x) \to (1 < x \vee x \geq 3) \big)$
	\end{enumerate} 



\newpage



% Problem 3
\problem{10} Being as clear and as detailed as possible, explain why $\forall\, x\, P(x) \wedge \forall\, x\, Q(x)$ does not imply $\forall\, x \, [P(x) \wedge Q(x)]$. 





\newpage



% Problem 4
\problem{10} A certain computer program has $n$ as an integer variable. Suppose that $A$ is an array of 10~integers values, i.e. $A$ is a `list' of the integer values $A[1], A[2], \ldots, A[10]$. Write the following as quantified open statements using $A[k]$:
	\begin{enumerate}[(a)]
	\item Every entry in the array is greater than 100.
	\item There is an entry in the array that is negative. 
	\item The value $A[10]$ is the largest value in the array.
	\item There are two entries in the array that are the same. 
	\end{enumerate} 



\newpage



% Problem 5
\problem{10} A function $f(x)$ is said to be \textit{uniformly continuous} if for all $\epsilon > 0$, there exists $\delta > 0$ such that $|f(x) - f(y)| < \epsilon$ whenever $|x - y| < \delta$. 
	\begin{enumerate}[(a)]
	\item Write the definition of uniform continuity as a quantified statement. 
	\item Find a quantified statement for a function to be \textit{not} uniformly continuous by negating your answer in (a). 
	\item Write your answer in (b) as an English sentence. 
	\end{enumerate}


\end{document}