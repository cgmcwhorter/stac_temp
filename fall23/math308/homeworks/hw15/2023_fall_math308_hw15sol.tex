\documentclass[11pt,letterpaper]{article}
\usepackage[lmargin=1in,rmargin=1in,tmargin=1in,bmargin=1in]{geometry}
\usepackage{../style/homework}
\usepackage{../style/commands}
\setbool{quotetype}{false} % True: Side; False: Under
\setbool{hideans}{false} % Student: True; Instructor: False

\newcommand{\blank}[1]{\underline{\hspace{#1}}} % Blank Underline

% -------------------
% Content
% -------------------
\begin{document}

\homework{15: Due 11/14}{One reason why Combinatorics has been slow to become accepted as part of mainstream Mathematics is the common belief that it consists of a bag of isolated tricks, a number of areas\dots with little or no connexion between them. We shall see that they have numerous threads weaving them together into a beautifully patterned tapestry.}{Richard K. Guy}

% Problem 1
\problem{10} Showing all your work, complete the following:
	\begin{enumerate}[(a)]
	\item Writing $\left(3x^2 - \frac{1}{x} \right)^{12}$ in descending power of $x$, find the `middle' term. 
	\item Find the coefficient of $x^3y^2$ in $(2x - 3y)^5$.
	\item Find the coefficient of $x^2yz^3$ in $(2x - y + 3z)^6$
	\end{enumerate} \pspace

\sol 

Recall that the Binomial Theorem states that for $n \in \mathbb{Z}_{\geq 0}$, $(x + y)^n= \sum_{k=0}^n \binom{n}{k} x^k y^{n - k}$, i.e. the coefficient of $x^k y^{n - k}$ is $\binom{n}{k}$. Furthermore, by the Multinomial Theorem, if $n_1, n_2, \ldots, n_k \in \mathbb{Z}_{\geq 0}$ and $n= n_1 + n_2 + \cdots + n_k$, the coefficient of $x_1^{n_1} x_2^{n_2} \cdots x_k^{n_k}$ in $(x_1 + x_2 + \cdots + x_k)^n$ is $\binom{n}{n_1, n_2, \ldots, n_k}$. 

\begin{enumerate}[(a)]
\item The expansion of $(3x^2 - \frac{1}{x})^{12}$ has $12 + 1= 13$~terms. The middle term is then the $\frac{13 + 1}{2}= \frac{14}{2}= 7$th term. Writing $(3x^2 - \frac{1}{x})$ in descending powers of $x$, this is the choice of six $3x^2$ and six $-\frac{1}{x}$ terms in the expansion. Using the Binomial Theorem, this term is\dots
	\[
	\binom{12}{6} (3x^2)^6 \left( -\dfrac{1}{x} \right)^6= 924 \cdot 729x^{12} \cdot \dfrac{1}{x^6}= 673596 x^6
	\] \pspace

\item Because there is only one way to obtain $x^3y^2$, by the Binomial Theorem, the term with $x^3y^2$ in the expansion of $(2x - 3y)^5$ is\dots
	\[
	\binom{5}{3} (2x)^3 (-3y)^2= 10 \cdot 8x^3 \cdot 9y^2= 720 x^3 y^2
	\]
Therefore, the coefficient of $x^3y^2$ is $720$. \pspace

\item Because there is only one way to obtain the term $x^2yz^3$, by the Multinomial Theorem, the term with $x^2yz^3$ in $(2x - y + 3z)^6$ is\dots
	\[
	\binom{6}{2, 1, 3} (2x)^2 (-y)^1 (3z)^3= 60 \cdot 4x^2 \cdot -y \cdot 27z^3= -6480 x^2yz^3
	\]
Therefore, the coefficient of $x^2yz^3$ is $-6480$. 
\end{enumerate}



\newpage



% Problem 2
\problem{10} If you distribute 27 tasks amongst 6 people, what is the maximum number of tasks a person can be assigned? What is the minimum number of tasks someone can be assigned to? Does there have to be a person with at least 5 tasks assigned to them? Explain. Does there have to be a person with at most 4 tasks assigned to them? Explain. \pspace

\sol Clearly, if you distribute 27 tasks amongst 6~people, the maximum number of tasks a person is 27 by assigning all tasks to one person. Furthermore, the minimum number of tasks a person can be assigned is 0 by being sure one person receives no tasks. \pspace

By the Pigeonhole Principle, there must be at least one person with at least $\ceil*{\frac{27}{6}}= \ceil*{4.5}= 5$ tasks assigned to them. Alternatively, if this were not the case, then every individual would have to be assigned four or less tasks. But then the most number of tasks that could be assigned would be $6 \cdot 4= 24 < 27$. With $27 - 24= 3$ tasks left to assign, at least one person must receive at least one more tasks, so that this individual is assigned at least 5 tasks. \pspace

Similarly, by the Pigeonhole Principle, there must be a person with at most $\floor*{\frac{27}{6}}= \floor*{4.5}= 4$ tasks assigned to them. Alternatively, if this were not the case, then every individual would be assigned at least 5 tasks. But then the minimum number of tasks assigned that would be assigned is $6 \cdot 5= 30 > 27$, which is impossible. Then there must be an individual assigned less than 5 tasks, i.e. assigned at most 4 tasks. 



\newpage



% Problem 3
\problem{10} Consider the set of positive integers less than 10,000. How many of these numbers are\dots
	\begin{enumerate}[(a)]
	\item \dots divisible by 5 or 7?
	\item \dots divisible by 2 and 5 but not 20?
	\item \dots divisible by 2, 3, or 5?
	\end{enumerate} \pspace

\sol Recall the Principle of Inclusion/Exclusion: if $A_1, A_2, \ldots, A_n$ are finite sets, then\dots
	\[
	\left| \bigcup_{i=1}^n A_i \right|= \sum_{i=1}^n |A_i| - \sum_{1 \leq i < j \leq n} |A_i \cap A_j| + \sum_{1 \leq i < j < k \leq n} |A_i \cap A_j \cap A_k| - \cdots _ (-1)^{n+1} |A_1 \cap \cdots \cap A_n|
	\]
For two sets $A, B$, this is $|A \cup B|= |A| + |B| - |A \cap B|$. Note that if $A, B$ are disjoint finite sets, this implies that $|A \cup B|= |A| + |B| - |A \cap B|= |A| + |B| - |\varnothing|= |A| + |B| - 0= |A| + |B|$. For three sets, $A, B, C$, this is $|A \cup B \cup C|= |A| + |B| + |C| - |A \cap B| - |A \cap C| - |B \cap C| + |A \cap B \cap C|$. Note that for $n, m \geq 1$, the number of positive multiples of $m$ less than $n$ is $\floor*{\frac{n - 1}{m}}$.

\begin{enumerate}[(a)]
\item Let $A$ be the set of multiples of 5 and $B$ be the set of multiples of 7. The set $A \cup B$ is the set of multiples of 5 or 7, i.e. the set of numbers that are divisible by 5 or 7. The set $A \cap B$ is the set of positive integers less than 10,000 that are a multiple of 5 and 7. But any multiples of both 5 and 7 are a multiple of their least common multiple, which is $\lcm(5, 7)= 35$. Therefore, $A \cap B$ is the set of positive integers less than 10,000 that are a multiple of 35. We have\dots
	\[
	\begin{aligned}
	|A|&= \floor*{\dfrac{10000 - 1}{5}}= \floor*{\dfrac{9999}{5}}= \floor*{1999.8}= 1999 \\
	|B|&= \floor*{\dfrac{10000 - 1}{7}}= \floor*{9999}{7}= \floor*{1428.43}= 1428 \\
	|A \cap B|&= \floor*{\dfrac{10000 - 1}{35}}= \floor*{\dfrac{9999}{35}}= \floor*{285.686}= 285
	\end{aligned}
	\]
But then by Inclusion-Exclusion applied to $A, B$, we have\dots
	\[
	|A \cup B|= |A| + |B| - |A \cap B|= 1999 + 1428 - 285= 3142
	\]

\item Let $A$ be the set of positive integers less than 10,000 that are a multiple of 2, i.e. the set of positive integers less than 10,000 that are divisible by 2.  Let $B$ be the set of positive integers less than 10,000 that are a multiple of 5, i.e. the set of positive integers less than 10,000 that are divisible by 5. Finally,  let $C$ be the set of positive integers less than 10,000 that are a multiple of 20, i.e. the set of positive integers less than 10,000 that are divisible by 20. The set $A \cap B$ is the set of positive integers less than 10,000 that are a multiple of 2 and 5, i.e. the set of positive integers less than 10,000 that are divisible by 2 and 5. If an integer is divisible by 2 and 5, then it is divisible by $\lcm(2, 5)= 10$. Therefore, $A \cap B$ is the set of positive integers less than 10,000 that are divisible by 10. Because every element of $A \cap B$ is either in $C$ or not, we must have $A \cap B= \big( (A \cap B) \cap C \big) \cup \big( (A \cap B) \cap C^c \big)$. The sets $(A \cap B) \cap C= A \cap B \cap C$ and $(A \cap B) \cap C^c= A \cap B \cap C^c$ are disjoint because $(A \cap B \cap C) \cap (A \cap B \cap C^c)= (A \cap B) \cap (C \cap C^c)= (A \cap B) \cap \varnothing= \varnothing$. Now the set $(A \cap B) \cap C$ is the set of positive integers less than 10,000 that are divisible by 2 and 5, i.e. divisible by 10, that are divisible by 20. But an integer divisible by 10 and 20 is divisible by $\lcm(10, 20)= 20$. Therefore, $(A \cap B) \cap C$ is the set of positive integers less than 10,000 that are divisible by 20, i.e. $(A \cap B) \cap C= C$ (which also follows from the fact that $C \subseteq A \cap B$). The set $(A \cap B) \cap C^c$ are the set of positive integers less than 10,000 that are divisible by 2 and 5, i.e. divisible by 10, that are not divisible by 20. We have\dots
	\[
	\begin{aligned}
	|A \cap B|&= \floor*{\dfrac{10000 - 1}{10}}= \floor*{\dfrac{9999}{10}}= \floor*{999.9}= 999 \\
	|(A \cap B) \cap C|&= |C|= \floor*{\dfrac{10000 - 1}{20}}= \floor*{\dfrac{9999}{20}}= \floor*{499.95}= 499
	\end{aligned}
	\]
Then by the Inclusion-Exclusion Principle and the comments before (a), we have\dots
	\[
	\begin{gathered}
	|A \cap B|= |\big( (A \cap B) \cap C \big)| + |\big( (A \cap B) \cap C^c \big)| \\
	|A \cap B|= |C| + |A \cap B \cap C^c| \\
	999= 499 + (A \cap B \cap C^c) \\
	A \cap B \cap C^c= 500
	\end{gathered}
	\]

\item Let $A, B, C$ be the set of positive integers less than 10,000 that are a multiple of 2, 3, and 5, respectively, i.e. the set of positive integers less than 10,000 that are divisible by 2, 3, and 5, respectively. The set $A \cup B \cup C$ is the set of positive integers less than 10,000 that are a multiple of 2, 3, or 5, i.e. the set of positive integers less than 10,000 divisible by 2, 3, or 5. The set $A \cap B$ is the set of positive integers that are less than 10,000 that are a multiple of 2 and 3, i.e. the set of positive integers less than 10,000 divisible by 2 and 3. If an integer is divisible by 2 and 3, then it is divisible by $\lcm(2, 3)= 6$. Therefore, $A \cap B$ is the set of positive integers less than 10,000 divisible by 6. Similarly, $A \cap C$ is the set of positive integers less than 10,000 divisible by $\lcm(2, 5)= 10$, $B \cap C$ is the set of positive integers less than 10,000 divisible by $\lcm(3, 5)= 15$, and $A \cap B \cap C$ is the set of positive integers less than 10,000 divisible by $\lcm(2, 3, 5)= 30$. We have\dots
	\[
	\begin{aligned}
	|A|&= \floor*{\dfrac{10000 - 1}{2}}= \floor*{\dfrac{9999}{2}}= \floor*{4999.5}= 4999 \\
	|B|&= \floor*{\dfrac{10000 - 1}{3}}= \floor*{\dfrac{9999}{3}}= \floor*{3333}= 3333 \\
	|C|&= \floor*{\dfrac{10000 - 1}{5}}= \floor*{\dfrac{9999}{5}}= \floor*{1999.8}= 1999 \\
	|A \cap B|&= \floor*{\dfrac{10000 - 1}{6}}= \floor*{\dfrac{9999}{6}}= \floor*{1666.5}= 1666 \\
	|A \cap C|&= \floor*{\dfrac{10000 - 1}{10}}= \floor*{\dfrac{9999}{10}}= \floor*{999.9}= 999 \\
	|B \cap C|&= \floor*{\dfrac{10000 - 1}{15}}= \floor*{\dfrac{9999}{15}}= \floor*{666.6}= 666 \\
	|A \cap B \cap C|&= \floor*{\dfrac{10000 - 1}{30}}= \floor*{\dfrac{9999}{30}}= \floor*{333.3}= 333
	\end{aligned}
	\]
Applying the Inclusion-Exclusion Principle to $A, B, C$, we have\dots
	\[
	\hspace{-1.5cm} |A \cup B \cup C|= |A| + |B| + |C| - |A \cap B| - |A \cap C| - |B \cap C| + |A \cap B \cap C|= 4999 + 3333 + 1999 - 1666 - 999 - 666 + 333= 7333
	\]
\end{enumerate}


\end{document}