\documentclass[11pt,letterpaper]{article}
\usepackage[lmargin=1in,rmargin=1in,tmargin=1in,bmargin=1in]{geometry}
\usepackage{../style/homework}
\usepackage{../style/commands}
\setbool{quotetype}{false} % True: Side; False: Under
\setbool{hideans}{false} % Student: True; Instructor: False

% -------------------
% Content
% -------------------
\begin{document}

\homework{6: Due 10/05}{I know that the great Hilbert said, `We will not be driven out of the paradise Cantor has created for us,' and I reply, `I see no reason for walking in!' \,}{Richard Hamming}

% Problem 1
\problem{10} Let $A$ and $B$ be sets. For each of the following sets, compute the \textit{complement} of the given set. Be sure to show all your work and simplify your set expression as much as possible. 
	\begin{enumerate}[(a)]
	\item $(A \Delta B) \cup B^c$
	\item $(A \cup B^c) \cap (A \cap B)^c$
	\item$A - (A - B)$
	\end{enumerate} \pspace

\sol 
\begin{enumerate}[(a)]
\item We use the fact that $A \Delta B= (A \cup B) - (A \cap B)= (A \cap B^c) \cup (B \cap A^c)$. 
	\[
	\begin{aligned}
	\big( (A \Delta B) \cup B^c \big)^c&= (A \Delta B)^c \cap (B^c)^c \\
	&= (A \Delta B)^c \cap B \\
	&= \big( (A \cap B^c) \cup (B \cap A^c) \big)^c \cap B \\
	&= (A \cap B^c)^c \cap (B \cap A^c)^c \cap B \\
	&= (A^c \cup B) \cap (B^c \cup A) \cap B \\
	&= \left[ (A^c \cup B) \cap B \right] \cap (B^c \cup A) \\
	&= B \cap (B^c \cup A) \\
	&= (B \cap B^c) \cup (B \cap A) \\
	&= \varnothing \cup (A \cap B) \\
	&= A \cap B
	\end{aligned}
	\] \pspace

\item 
	\[
	\begin{aligned}
	\big( (A \cup B^c) \cap (A \cap B)^c \big)^c&= (A \cup B^c)^c \cup (A \cap B) \\
	&= (A^c \cap B) \cup (A \cap B) \\
	&= \big( (A^c \cap B) \cup A \big) \cap \big( (A^c \cap B) \cup B \big) \\
	&= \big( (A^c \cap B) \cup A \big) \cap B \\
	&= \big( (A^c \cup A) \cap (B \cup A) \big) \cap B \\
	&= \big( \mathcal{U} \cap (B \cup A) \big) \cap B \\
	&= (B \cup A) \cap B \\
	&= B
	\end{aligned}
	\] 

\item We use the fact that $A - B= A \cap B^c$.
	\[
	\begin{aligned}
	\big( A - (A - B) \big)^c&= \big( A \cap (A - B)^c \big)^c \\
	&= A^c \cup (A - B) \\
	&= A^c \cup (A \cap B^c) \\
	&= (A^c \cup A) \cap (A^c \cup B^c) \\
	&= \mathcal{U} \cap (A^c \cup B^c) \\
	&= A^c \cup B^c \\
	&= (A \cap B)^c
	\end{aligned}
	\] 
\end{enumerate}



\newpage



% Problem 2
\problem{10} Let $X= \{ a, \{ b \}, \{ a, b \} \}$. 
	\begin{enumerate}[(a)]
	\item Compute $\mathcal{P}(X)$. What is the cardinality of this set?
	\item Determine whether the following are true or false---no justification is necessary:
		\begin{multicols}{2}
		\begin{enumerate}[(i)]
		\item $\varnothing \in X$
		\item $\varnothing \subseteq X$
		\item $a \in X$
		\item $\{ a \} \in X$
		\item $\{ a \} \subseteq X$
		% 
		\item $\varnothing \in \mathcal{P}(X)$
		\item $\mathcal{P}(X) \subseteq \mathcal{P}(X)$
		\item $\{ a, b \} \in \mathcal{P}(X)$
		\item $\{ a, b \} \subseteq \mathcal{P}(X)$
		\item $\{ \{ a, b \} \} \subseteq \mathcal{P}(X)$
		\end{enumerate}
		\end{multicols}
	\end{enumerate}

\sol
\begin{enumerate}[(a)]
\item 
 It would be useful to write $S$ and compute $\mathcal{P}(S)$:
	\[
	\mathcal{P}(X)= 
	\left\{
	\begin{matrix}
	\varnothing, \\
	\{ a \}, & \{ \{ b \} \}, & \{ \{ a, b \} \}, \\
	\{ a, \{ b \} \}, & \{ a, \{ a, b \} \}, & \{ \{ b \}, \{ a, b \} \}, \\
	X= \{ a, \{ b \}, \{ a, b \} \}
	\end{matrix}
	\right\}
	\] \pspace

\item 
	\begin{multicols}{2}
	\begin{enumerate}[(i)]
	\item \textit{F}
	\item \textit{T}
	\item \textit{T}
	\item \textit{F}
	\item \textit{T}
	% 
	\item \textit{T}
	\item \textit{T}
	\item \textit{F}
	\item \textit{F}
	\item \textit{F}
	\end{enumerate}
	\end{multicols}
\end{enumerate}



\newpage



% Problem 3
\problem{10} For integers $n$, let $X_n= (n, n + 1)$, and for natural numbers $m$, let $Y_m= \left[ \frac{1}{m}, m \right)$. Compute the following:
	\begin{enumerate}[(a)]
	\item $\displaystyle \bigcup_{i= -1}^2 X_i$
	\item $\displaystyle \bigcap_{k=2}^5 Y_k$
	\item $\displaystyle \bigcup_{n \in \mathbb{Z}} X_n$
	\item $\displaystyle \bigcup_{m \in \mathbb{N}} Y_m$
	\item $\displaystyle \left( \bigcup_{m \in \mathbb{N}} Y_m \right)^c$
	\end{enumerate} \pspace

\sol 
\begin{enumerate}[(a)]
\item $\displaystyle \bigcup_{i= -1}^2 X_i= (-1, 0) \cup (0, 1) \cup (1, 2) \cup (2, 3)$
\item $\displaystyle \bigcap_{k=2}^5 Y_k= [ \tfrac{1}{2}, 2)$
\item $\displaystyle \bigcup_{n \in \mathbb{Z}} X_n= \mathbb{R} - \mathbb{Z}$
\item $\displaystyle \bigcup_{m \in \mathbb{N}} Y_m= (0, \infty)$
\item $\displaystyle \left( \bigcup_{m \in \mathbb{N}} Y_m \right)^c= (0, \infty)^c= (-\infty, 0]$
\end{enumerate}



\newpage



% Problem 4
\problem{10} Let $A= \{ -1, 0, 1 \}$, $B= \{ a, b \}$, and $C= \{ \sqrt{2}, \pi \}$. 
	\begin{enumerate}[(a)]
	\item Compute $A \times B$.
	\item Is $A \times B= B \times A$? Explain. 
	\item Compute $\mathcal{P}(B \times C)$.
	\item If $X= \varnothing$, what is $X \times Y$ for any set $Y$?
	\item If $X, Y$ are sets and $X \times Y= Y \times X$, is it necessarily true that $X= Y$? Explain. [Hint: Use part (d).]
	\end{enumerate} \pspace

\sol 
\begin{enumerate}[(a)]
\item 
	\[
	A \times B= 
	\left\{
	\begin{matrix}
	(-1, a), & (-1, b), \\
	(0, a), & (0, b), \\
	(1, a), & (1, b)
	\end{matrix}
	\right\}
	\] \pspace

\item No, $A \times B \neq B \times A$. For instance, $(-1, a) \in A \times B$ because $-1 \in A$ and $a \in B$; however, $(-1, a) \notin B \times A$ because $-1 \notin B$, $a \notin A$. In fact, we have\dots
	\[
	B \times A= 
	\left\{
	\begin{matrix}
	(a, -1), & (a, 0), & (a, 1), \\
	(b, -1), & (b, 0), & (b, 1)
	\end{matrix}
	\right\}
	\] \pspace

\item First, observe that we have $B \times C= \{ (a, \sqrt{2}), (a, \pi), (b, \sqrt{2}), (b, \pi) \}$. But then we have\dots
	\[
	\hspace{-1.5cm} \mathcal{P}(B \times C)= 
	\left\{
	\begin{matrix}
	\varnothing, \\ \\
	\{ (a, \sqrt{2}) \}, & \{ (a, \pi) \}, & \{ (b, \sqrt{2}) \}, & \{ (b, \pi) \}, \\ \\
	\{ (a, \sqrt{2}), (a, \pi) \}, & \{ (a, \sqrt{2}), (b, \sqrt{2}) \}, & \{ (a, \sqrt{2}), (b, \pi) \}, \\
	\{ (a, \pi), (b, \sqrt{2}) \}, & \{ (a, \pi), (b, \pi) \}, \\
	\{ (b, \sqrt{2}), (b, \pi) \}, \\ \\
	\{ (a, \sqrt{2}), (a, \pi), (b, \sqrt{2}) \}, & \{ (a, \sqrt{2}), (a, \pi), (b, \pi) \}, \\
	\{ (a, \sqrt{2}), (b, \sqrt{2}), (b, \pi) \}, \\
	\{ (a, \pi), (b, \sqrt{2}), (b, \pi) \}, \\ \\
	B \times A=  \{ (a, \sqrt{2}), (a, \pi), (b, \sqrt{2}), (b, \pi) \}
	\end{matrix}
	\right\}
	\] \pspace

\item If $X= \varnothing$, then $X \times Y= \varnothing$. We know that $X \times Y= \{ (x, y) \colon x \in X, y \in Y \}$. But if $X= \varnothing$, then there is no $x \in X$. Then there can be no $(x, y) \in X \times Y$, which proves that $X \times Y= \varnothing$. This holds mutatis mutandis for $Y= \varnothing$. Therefore, for all sets $X$, $X \times \varnothing= \varnothing$ and $\varnothing \times X= \varnothing$. \pspace

\item We know from (d) that if $X= \varnothing$, then $X \times Y= \varnothing= Y \times X$ for all sets $Y$. But then taking $Y$ to be any nonempty set, we see that if $X= \varnothing$, then $X \times Y= \varnothing= Y \times X$ but $X= \varnothing \neq Y$. Therefore, it is not true that if $X \times Y= Y \times X$ that $X= Y$. \pspace

However, if $X, Y$ are nonempty sets and $X \times Y= Y \times X$ does imply that $X= Y$. To see this, suppose that $X, Y$ are nonempty sets with $X \times Y= Y \times X$. Choose any $x \in X$ and $y \in Y$. By definition, $(x, y) \in X \times Y$. But $X \times Y= Y \times X$ so that $(x, y) \in Y \times X$. This implies that $x \in Y$ and $y \in X$. Therefore, for all $x \in X, y \in Y$, if $x \in X$, then $x \in Y$, which implies that $X \subseteq Y$, and if $y \in Y$, then $y \in X$, which implies that $Y \subseteq X$. Because $X \subseteq Y$ and $Y \subseteq X$, we know that $X= Y$. Obviously, $X \times Y= Y \times X$, if $X= Y= \varnothing$. 
\end{enumerate}


\end{document}