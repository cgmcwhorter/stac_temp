\documentclass[11pt,letterpaper]{article}
\usepackage[lmargin=1in,rmargin=1in,tmargin=1in,bmargin=1in]{geometry}
\usepackage{../style/homework}
\usepackage{../style/commands}
\setbool{quotetype}{true} % True: Side; False: Under
\setbool{hideans}{false} % Student: True; Instructor: False

% -------------------
% Content
% -------------------
\begin{document}

\homework{5: Due 09/25}{I turned myself into a pickle, Morty! I'm Pickle Rick!}{Rich Sanchez, Rick \& Morty}

% Problem 1
\problem{10} Express each of the following decimal numbers as a rational number in simplest form and express each of the rational numbers as a decimal number:
	\begin{enumerate}[(a)]
	\item $0.85$
	\item $\frac{5}{12}$
	\item $1.12$
	\item $\frac{11}{6}$
	\end{enumerate}



\newpage



% Problem 2
\problem{10} Showing all your work, express the number $0.\overline{2023}$ as a rational number. 



\newpage



% Problem 3
\problem{10} Perform the following operations in $\mathbb{C}$:
	\begin{enumerate}[(a)]
	\item $(\frac{2}{3} + 5i) + (\frac{1}{2} - \frac{3}{4} i)$
	\item $(15 + 6i) - (9 - 4i)$
	\item $(6 - 3i)(8 + 5i)$
	\item $\dfrac{5 - 7i}{4 + 3i}$
	\item $(1 + 2i) (\overline{1 + 2i})$
	\end{enumerate}



\newpage



% Problem 4
\problem{10} Every quadratic equation $ax^2 + bx + c= 0$ has exactly two (not necessarily distinct) solutions when the solutions are allowed to be complex numbers. Without explicitly solving the equation, verify that the two solutions to $x^2 - 2x + 5= 0$ are $x_0= 1 \pm 2i$; that is, substitute both $x= 1 + 2i$ and $x= 1 - 2i$ into $x^2 - 2x + 5$ and show that one obtains a zero for this function in each case. 


\end{document}