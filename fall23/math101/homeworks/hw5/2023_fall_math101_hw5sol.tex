\documentclass[11pt,letterpaper]{article}
\usepackage[lmargin=1in,rmargin=1in,tmargin=1in,bmargin=1in]{geometry}
\usepackage{../style/homework}
\usepackage{../style/commands}
\setbool{quotetype}{true} % True: Side; False: Under
\setbool{hideans}{false} % Student: True; Instructor: False

% Fraction Long Division
\usepackage{longdivision}

% -------------------
% Content
% -------------------
\begin{document}

\homework{5: Due 09/25}{I turned myself into a pickle, Morty! I'm Pickle Rick!}{Rich Sanchez, Rick \& Morty}

% Problem 1
\problem{10} Express each of the following decimal numbers as a rational number in simplest form and express each of the rational numbers as a decimal number:
	\begin{enumerate}[(a)]
	\item $0.85$
	\item $\frac{5}{12}$
	\item $1.12$
	\item $\frac{11}{6}$
	\end{enumerate} \pspace

\sol 
\begin{enumerate}[(a)]
\item 
	\[
	0.85= \dfrac{85}{100}= \dfrac{5 \cdot 17}{2^2 \cdot 5^2}= \dfrac{17}{2^2 \cdot 5}= \dfrac{17}{20}
	\] \pspace

\item 
	\[
	\longdivision{5}{12} % Compare to \intlongdivision
	\] \pspace
 
\item 
	\[
	1.12= \dfrac{112}{100}= \dfrac{2^4 \cdot 7}{2^2 \cdot 5^2}= \dfrac{2^2 \cdot 7}{5^2}= \dfrac{28}{25}
	\] \pspace
 
\item 
	\[
	\longdivision{11}{6} % Compare to \intlongdivision
	\]  
\end{enumerate}



\newpage



% Problem 2
\problem{10} Showing all your work, express the number $0.\overline{2023}$ as a rational number. \pspace

\sol Suppose that $N= 0.\overline{2023}= 0.2023202320232023\overline{2023}$. We have\dots
	\begin{table}[!ht]
	\centering\small
	\begin{tabular}{rccc}
	& $10000N$ & $=$ & $2023.2023202320232023\overline{2023}$ \\ 
	$-$ & $N$ & $=$ & $\phantom{202}0.2023202320232023\overline{2023}$ \\ \hline
	& $9999N$ & $=$ & $2023$ \\[0.1cm]
	& $N$ & $=$ & $\frac{2023}{9999}$
	\end{tabular}
	\end{table} \par

	\[
	0.\overline{2023}= \dfrac{2023}{9999}
	\]



\newpage



% Problem 3
\problem{10} Perform the following operations in $\mathbb{C}$:
	\begin{enumerate}[(a)]
	\item $(\frac{2}{3} + 5i) + (\frac{1}{2} - \frac{3}{4} i)$
	\item $(15 + 6i) - (9 - 4i)$
	\item $(6 - 3i)(8 + 5i)$
	\item $\dfrac{5 - 7i}{4 + 3i}$
	\item $(1 + 2i) (\overline{1 + 2i})$
	\end{enumerate} \pspace

\sol 
\begin{enumerate}[(a)]
\item 
	\[
	\left(\frac{2}{3} + 5i \right) + \left(\frac{1}{2} - \frac{3}{4} i \right)= \left( \dfrac{2}{3} + \dfrac{1}{2} \right) + \left( 5i - \dfrac{3}{4} \,i \right)= \left( \dfrac{4}{6} + \dfrac{3}{6} \right) + \left( \dfrac{20}{4}\,i - \dfrac{3}{4} \,i \right)= \dfrac{7}{6} + \dfrac{17}{4}\,i
	\] \pspace

\item 
	\[
	(15 + 6i) - (9 - 4i)= 15 + 6i - 9 + 4i= (15 - 9) + (6i + 4i)= 6 + 10i
	\] \pspace

\item 
	\[
	(6 - 3i)(8 + 5i)= 48 + 30i - 24i - 15 i^2= 48 + 30i - 24i - 15(-1)= 48 + 30i - 24i + 15= 63 + 6i
	\] \pspace

\item 
	\[
	\dfrac{5 - 7i}{4 + 3i}= \dfrac{5 - 7i}{4 + 3i} \cdot \dfrac{4 - 3i}{4 - 3i}= \dfrac{20 - 15i - 28i + 21i^2}{16 - 12i + 12i - 9i^2}= \dfrac{(20 - 21) + (-15i - 28i)}{(16 + 9) + (-12i + 12i)}= \dfrac{-1 - 43i}{25}= -\frac{1}{25} - \frac{43}{25}\,i
	\] \pspace

\item 
	\[
	(1 + 2i) (\overline{1 + 2i})= (1 + 2i)(1 - 2i)= 1 - 2i + 2i - 4i^2= 1 - 4(-1)= 1 + 4= 5
	\]
\end{enumerate}



\newpage



% Problem 4
\problem{10} Every quadratic equation $ax^2 + bx + c= 0$ has exactly two (not necessarily distinct) solutions when the solutions are allowed to be complex numbers. Without explicitly solving the equation, verify that the two solutions to $x^2 - 2x + 5= 0$ are $x_0= 1 \pm 2i$; that is, substitute both $x= 1 + 2i$ and $x= 1 - 2i$ into $x^2 - 2x + 5$ and show that one obtains a zero for this function in each case. \pspace

\sol We have\dots
	\[
	\begin{aligned}
	(x^2 - 2x + 5) \bigg|_{x= 1 + 2i}&= (1 + 2i)^2 - 2(1 + 2i) + 5 \\
	&= (1 + 2i + 2i + 4i^2) - 2(1 + 2i) + 5 \\
	&= (1 + 4i - 4) - 2 - 4i + 5 \\
	&= -3 + 4i - 2 - 4i + 5 \\
	&= (-3 - 2 + 5) + (4i - 4i) \\
	&= 0 
	\end{aligned}
	\] 
and also\dots
	\[
	\begin{aligned}
	(x^2 - 2x + 5) \bigg|_{x= 1 - 2i}&= (1 - 2i)^2 - 2(1 - 2i) + 5 \\
	&= (1 - 2i - 2i + 4i^2) - 2(1 - 2i) + 5 \\
	&= (1 - 4i - 4) - 2 + 4i + 5 \\
	&= -3 - 4i - 2 + 4i + 5 \\
	&= (-3 - 2 + 5) + (-4i + 4i) \\
	&= 0 
	\end{aligned}
	\] 


\end{document}