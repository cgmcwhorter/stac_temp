\documentclass[11pt,letterpaper]{article}
\usepackage[lmargin=1in,rmargin=1in,tmargin=1in,bmargin=1in]{geometry}
\usepackage{../style/homework}
\usepackage{../style/commands}
\setbool{quotetype}{true} % True: Side; False: Under
\setbool{hideans}{false} % Student: True; Instructor: False

% -------------------
% Content
% -------------------
\begin{document}

\homework{4: Due 09/20}{I'm not afraid of hard work. I just don't like it.}{Bob Belcher, Bob's Burgers}

% Problem 1
\problem{10} Showing all your work, compute the following ``without a calculator'': 
	\begin{enumerate}[(a)]
	\item $\sqrt[4]{256}$
	\item $\sqrt[3]{-125}$
	\item $\left( \dfrac{49}{36} \right)^{-1/2}$
	\item $\sqrt{\dfrac{1}{4}}$
	\item $216^{2/3}$
	\end{enumerate} \pspace

\sol 
\begin{enumerate}[(a)]
\item The prime factorization of 256 is $256= 2^8$. But then\dots
	\[
	\sqrt[4]{256}= \sqrt[4]{2^8}= (2^8)^{1/4}= 2^2= 4
	\] \pspace

\item The prime factorization of $125= 5^3$. But then\dots
	\[
	\sqrt[3]{-125}= \sqrt[3]{-5^3}= -5
	\] \pspace

\item The prime factorizations of 49 and 36 are $49= 7^2$ and $36= 2^2 \cdot 3^2$. But then\dots
	\[
	\left( \dfrac{49}{36} \right)^{-1/2}= \left( \dfrac{36}{49} \right)^{1/2}= \dfrac{36^{1/2}}{49^{1/2}}= \dfrac{(2^2 \cdot 3^2)^{1/2}}{(7^2)^{1/2}}= \dfrac{2 \cdot 3}{7}= \dfrac{6}{7}
	\] \pspace

\item The prime factorization of 4 is $4= 2^2$. But then\dots
	\[
	\sqrt{\dfrac{1}{4}}= \dfrac{\sqrt{1}}{\sqrt{4}}= \dfrac{1}{\sqrt{2^2}}= \dfrac{1}{2}
	\] \pspace

\item The prime factorization of 216 is $216= 2^3 \cdot 3^3$. But then\dots
	\[
	216^{2/3}= (2^3 \cdot 3^3)^{2/3}= \big( (2^3 \cdot 3^3)^{1/3} \big)^2= (2 \cdot 3)^2= 6^2= 36
	\]
\end{enumerate}



\newpage



% Problem 2
\problem{10} Showing all your work and completely justifying your reasoning, estimate $\sqrt[4]{101}$ without a calculator. \pspace

\sol If $x= \sqrt[4]{101}$, then $x^4= 101$. Now observe\dots 
	\[
	\begin{aligned}
	1^4&= 1 \\
	2^4&= 16 \\
	3^4&= 81 \\
	4^4&= 256
	\end{aligned}
	\]
But then $3^4= 81 < x^4= 101 < 256= 4^4$. This shows that $3 < x < 4$, i.e. $3 < \sqrt[4]{101} < 4$. \pspace 

We can further estimate $\sqrt[4]{101}$ by bisecting this interval. We have $3.5= \frac{35}{10}$, so that $3.5^4= \frac{35^4}{10^4}= \frac{1500620}{10000}= 150.0625 > 101$. But then we know that $3^4= 81 < x^4= 101 < 150.0625= 3.5^4$, which implies $3 < \sqrt[4]{101} < 3.5$. \pspace

Yet again, we can further estimate $\sqrt[4]{101}$ by bisecting this interval. We have $3.25= \frac{325}{100}$, so that $3.25^4= \frac{325^4}{100^4}= \frac{11156640625}{100000000}= 111.56640625 > 101$. But then we know that $3^4= 81 < x^4= 101 < 111.56640625= 3.25^4$, which implies $3 < \sqrt[4]{101} < 3.25$. \pspace

We can continue this process ad infinitum. However, stopping here, we can estimate $\sqrt[4]{101} \approx \dfrac{3 + 3.25}{2}= 3.125$. The true value of $\sqrt[4]{101}$ is $\approx 3.17015388$. But then we have estimated $\sqrt[4]{101}$ with an error of only $\approx 0.0451539$---a percentage error of only $\approx 1.42\%$. 



\newpage



% Problem 3
\problem{10} Simplify the following:
	\begin{enumerate}[(a)]
	\item $\sqrt{\dfrac{(x y^2)^3}{x y^{-8}}}$
	\item $\left( \dfrac{x^9 y^{-1} (x y^5)^2}{x^{-1} y} \right)^{-1/2}$
	\item $\left( \sqrt[3]{\dfrac{xy (x^{-3} y^5)^{-2}}{x^{-2} y^5}} \right)^{-2}$
	\end{enumerate}   	



\newpage



% Problem 4
\problem{10} Simplify the following:
	\begin{enumerate}[(a)]
	\item $\dfrac{10}{\sqrt{72}}$
	\item $\sqrt{300}$
	\item $\sqrt[3]{360}$
	\item $\sqrt{2^{10} \cdot 3^5 \cdot 5^2 \cdot 11^{3}}$
	\item $\sqrt[5]{2^{12} \cdot 3^9 \cdot 5^1 \cdot 7^5}$
	\end{enumerate}   


\end{document}