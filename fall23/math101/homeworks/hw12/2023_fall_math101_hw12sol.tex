\documentclass[11pt,letterpaper]{article}
\usepackage[lmargin=1in,rmargin=1in,tmargin=1in,bmargin=1in]{geometry}
\usepackage{../style/homework}
\usepackage{../style/commands}
\setbool{quotetype}{true} % True: Side; False: Under
\setbool{hideans}{false} % Student: True; Instructor: False

% -------------------
% Content
% -------------------
\begin{document}

\homework{12: Due 11/06}{If you think education is expensive, try ignorance.}{Jeff Rich}

% Problem 1
\problem{10} Consider the linear function $\ell(x)= \frac{13 - 11x}{5}$. 
	\begin{enumerate}[(a)]
	\item Find the slope of this function.
	\item Find the $y$-intercept of this function.
	\item Find the $x$-intercept of this function.
	\item Does the graph of this function contain the point $(6, -8)$? Explain. 
	\end{enumerate} \pspace

\sol 
\begin{enumerate}[(a)]
\item Given a linear function $y= mx + b$, we know that $m$ is the slope and $b$ is the $y$-intercept. We have $\ell(x)= \frac{13 - 11x}{5}= \frac{13}{5} - \frac{11}{5}\,x$. Therefore, the slope is $m= -\frac{11}{5}$. 

\item From (a), we know that $\ell(x)= \frac{13 - 11x}{5}= \frac{13}{5} - \frac{11}{5}\,x$. Therefore, the $y$-intercept is $\frac{13}{5}$, i.e. the point $(0, \frac{13}{5})$. Equivalently, we know the $y$-intercept is the value of the function when $x= 0$. But then we have $\ell(0)= \frac{13 - 11(0)}{5}= \frac{13 - 0}{5}= \frac{13}{5}$, i.e. the $y$-intercept is $(0, \frac{13}{5})$. \pspace

\item The $x$-intercept(s) occurs at the $x$-value(s) when the output is 0. But then\dots
	\[
	\ell(x)= 0 
	\frac{13 - 11x}{5}= 0 \\
	13 - 11x= 0 \\
	13= 11x \\
	x= \frac{13}{11} \approx 1.18182
	\]
That is, the $x$-intercept is the point $(\frac{13}{11}, 0)$. \pspace

\item If the graph of $\ell$ contains the point $(6, -8)$, then $\ell(6)= -8$. But we have\dots
	\[
	\ell(6)= \dfrac{13 - 11(6)}{5}= \dfrac{13 - 66}{5}= -\dfrac{53}{5} \approx -10.6 \neq -8
	\]
Therefore, the graph of $\ell$ does not contain the point $(6, -8)$. Alternatively, the graph of $\ell$ contains the point $(6, -8)$ if the point satisfies the equation of $\ell$. But then\dots
	\[
	\begin{gathered}
	\ell(x)= \dfrac{13 - 11x}{5} \\
	\ell(6) \stackrel{?}{=} \dfrac{13 - 11(6)}{5} \\
	-8 \stackrel{?}{=} \dfrac{13 - 66}{5} \\
	-8 \neq -\dfrac{53}{5}
	\end{gathered}
	\]
Therefore, the graph of $\ell$ does not contain the point $(6, -8)$. 
\end{enumerate}



\newpage



% Problem 2
\problem{10} Solve the following equation and verify that your solution is correct:
	\[
	9 - 3(x + 1)= \dfrac{6 - x}{2}
	\] \pspace

\sol We have\dots
	\[
	\begin{gathered}
	9 - 3(x + 1)= \dfrac{6 - x}{2} \\[0.3cm]
	9 - 3x - 3= \dfrac{6 - x}{2} \\[0.3cm]
	6 - 3x= \dfrac{6 - x}{2} \\[0.3cm]
	2(6 - 3x)= 2 \left( \dfrac{6 - x}{2} \right) \\[0.3cm]
	12 - 6x= 6 - x \\[0.3cm]
	12= 6 + 5x \\[0.3cm]
	5x= 6 \\[0.3cm]
	x= \dfrac{6}{5}
	\end{gathered}
	\] \pspace
We can now verify this solution:
	\[
	\begin{gathered}
	9 - 3(x + 1)= \dfrac{6 - x}{2} \\
	9 - 3 \left( \dfrac{6}{5} + 1 \right) \stackrel{?}{=} \dfrac{6 - \frac{6}{5}}{2} \\[0.3cm]
	9 - 3 \cdot \dfrac{11}{5} \stackrel{?}{=} \dfrac{\frac{24}{5}}{2} \\[0.3cm]
	9 - \dfrac{33}{5} \stackrel{?}{=} \dfrac{24}{5} \cdot \dfrac{1}{2} \\[0.3cm]
	\dfrac{12}{5}= \dfrac{12}{5}
	\end{gathered}
	\]
Therefore, the solution $x= \frac{6}{5}$ is correct. 



\newpage



% Problem 3
\problem{10} Solve the following equation:
	\[
	5\sqrt{2}\, x + 8= -3(1 - 2x)
	\] \pspace

\sol We have\dots
	\[
	\begin{gathered}
	5\sqrt{2}\, x + 8= -3(1 - 2x) \\[0.3cm]
	5\sqrt{2}\,x + 8= -3 + 6x \\[0.3cm]
	5\sqrt{2}\,x - 6x + 8= -3 \\[0.3cm]
	5\sqrt{2}\,x - 6x= -11 \\[0.3cm]
	(5\sqrt{2} - 6)x= -11 \\[0.3cm]
	x= \dfrac{-11}{5 \sqrt{2} - 6} \\[0.3cm]
	x= \dfrac{11}{6 - 5\sqrt{2}} \approx -10.2701
	\end{gathered}
	\] \pspace
We can also numerically verify this solution:
	\[
	\begin{gathered}
	5\sqrt{2}\, x + 8= -3(1 - 2x) \\[0.3cm]
	5\sqrt{2} (-10.2701) + 8 \stackrel{?}{=} -3( 1 - 2 \cdot -10.2701) \\[0.3cm]
	-72.6206 + 8 \stackrel{?}{=} -3 (1 - (-20.5402)) \\[0.3cm]
	-64.6206 \stackrel{?}{=} -3(21.5402) \\[0.3cm]
	-64.6206= -64.6206
	\end{gathered}
	\] \pspace



\newpage



% Problem 4
\problem{10} Find the equation of the line perpendicular to the line $y= \pi$ with $x$-intercept $\sqrt{2}$. \pspace

\sol The line $y= \pi$ is horizontal. Because the line in question is perpendicular to a horizontal line, the line must be vertical. Therefore, the line has the form $x= a$ for some number $a$. We know that the $x$-intercept of the line is $\sqrt{2}$, I.e. the point $(\sqrt{2}, 0)$. But then it must be that $x= \sqrt{2}$. 


\end{document}