\documentclass[11pt,letterpaper]{article}
\usepackage[lmargin=1in,rmargin=1in,tmargin=1in,bmargin=1in]{geometry}
\usepackage{../style/homework}
\usepackage{../style/commands}
\setbool{quotetype}{true} % True: Side; False: Under
\setbool{hideans}{false} % Student: True; Instructor: False

% -------------------
% Content
% -------------------
\begin{document}

\homework{9: Due 10/30}{Mathematics is the most beautiful and most powerful creation of human spirit.}{Stefan Banach}

% Problem 1
\problem{10} Values for several functions are given in the table below. 
        \begin{table}[!ht]
        \centering
        \begin{tabular}{| c || c | c | c | c | c | c | c |} \hline
	$x$ & $-3$ & $-2$ & $-1$ & $\phantom{-}0$ & $\phantom{-}1$ & $\phantom{-}2$ & $\phantom{-}3$ \\ \hline \hline
	$f(x)$ & $\phantom{-}4$ & $8$ & $-1$ & $\phantom{-}5$ & $-3$ & $\phantom{-}0$ & $-2$ \\ \hline
	$g(x)$ & $\phantom{-}1$ & $6$ & $\phantom{-}0$ & $-6$ & $-7$ & $-3$ & $\phantom{-}1$ \\ \hline
	$h(x)$ & $-4$ & $0$ & $\phantom{-}3$ & $\phantom{-}5$ & $10$ & $\phantom{-}3$ & $\phantom{-}9$ \\ \hline
        \end{tabular}
        \end{table}

Given the data above, compute the following: 
        \begin{enumerate}[(a)]
        \item $(h + g)(-2)= h(-2) + g(-2)= 0 + 6= 6$ \vfill
        \item $(f - g)(0)= f(0) - g(0)= 5 - (-6)= 5 + 6= 11$ \vfill
        \item $(5h)(1)= 5h(1)= 5 \cdot 10= 50$ \vfill
        \item $\left(\dfrac{h}{f}\right)(1)= \dfrac{h(1)}{f(1)}= \dfrac{10}{-3}= -\dfrac{10}{3}$ \vfill
        \item $g(-3)\, h(3)= 1 \cdot 9= 9$ \vfill
        \item $g \big(-1 - f(3) \big)= g\big(-1 - (-2) \big)= g(-1 + 2)= g(1)= -7$ \vfill
        \item $(h \circ g)(2)= h \big( g(2) \big)= h(-3)= -4$ \vfill
	\item $(g \circ h)(2)= g \big( h(2) \big)= g(3)= 1$ \vfill
        \item $(f \circ g)(-1)= f \big( g(-1) \big)= f(0)= 5$ \vfill
	\item $(h \circ g \circ f)(1)= h \big( g \big( f(1) \big) \big)= h \big( g(-3) \big)= h(1)= 10$ \vfill
        \end{enumerate}



\newpage



% Problem 2
\problem{10} Suppose $f(x)$ and $g(x)$ are the functions given below. 
	\[
	\begin{aligned}
	f(x)&= 2x - 3 \\[0.3cm]
	g(x)&= x^2 + 2x - 1
	\end{aligned}
	\]

Compute the following: \pspace
        \begin{enumerate}[(a)]
        \item $f(5)= 2(5) - 3= 10 - 3= 7$ \vfill
        \item $g(-2)= (-2)^2 + 2(-2) - 1= 4 - 4 - 1= -1 $ \vfill
        \item $f(0) - 3g(2)= (2 \cdot 0 - 3) - 3 \big( 2^2 + 2(2) - 1 \big)= -3 - 3(7)= -3 - 21= -24$ \vfill
        \item $(f - g)(x)= f(x) - g(x)= (2x - 3) - (x^2 + 2x - 1)= 2x - 3 - x^2 - 2x + 1= -x^2 - 2$ \vfill
        \item $(fg)(x)= f(x) \, g(x)= (2x - 3)(x^2 + 2x - 1)= 2x^3 + 4x^2 - 2x - 3x^2 - 6x + 3= 2x^3 + x^2 - 8x + 3$ \vfill
        \item $\left( \dfrac{f}{g} \right)(x)= \dfrac{f(x)}{g(x)}= \dfrac{2x - 3}{x^2 + 2x - 1}$ \vfill
        \item $(f \circ g)(0)= f\big( g(0) \big)= f(0^2 + 2(0) - 1)= f(0 + 0 - 1)= f(-1)= 2(-1) - 3= -2 - 3= -5$ \vfill
        \item $(g \circ f)(0)= g\big( f(0) \big)= g( 2(0) - 3)= g(0 - 3)= g(-3)= (-3)^2 + 2(-3) - 1= 9 - 6 - 1= 2$ \vfill
        \item $(f \circ g)(x)= f\big( g(x) \big)= f(x^2 + 2x - 1)= 2(x^2 + 2x - 1) - 3= 2x^2 + 4x - 2 - 3= 2x^2 + 4x - 5$ \vfill
        \item $(g \circ f)(x)= g\big( f(x) \big)= g(2x - 3)= (2x - 3)^2 + 2(2x - 3) - 1= (4x^2 - 12x + 9) + (4x - 6) - 1= 4x^2 - 8x + 2$ \vfill
        \end{enumerate} 



\newpage



% Problem 3
\problem{10} Let $f(x)$ be the function given by $f(x)= 3x - 7$. 
	\begin{enumerate}[(a)]
	\item Find a value in the range of $f$. Be sure to justify why the value is in the range. 
	\item Compute $f(4)$. Is $(4, 1)$ on the graph of $f$? Explain. 
	\item Is there an $x$ such that $f(x)= 11$? Explain. 
	\item Is $1 \in f^{-1}(3)$? Explain. 
	\item Assuming $f^{-1}$ exists, what is $f(f^{-1}(\pi))$ and $f^{-1}(f(\sqrt{2}))$?
	\end{enumerate} \pspace

\sol 
\begin{enumerate}[(a)]
\item We know that the range of $f$ is the set of outputs of $f$. Therefore, we can obtain an output by evaluating $f$ at any value in its domain. For example, $f(0)= 3(0) - 7= -7$, $f(10)= 3(10) - 7= 23$, and $f(-5)= 3(-5) - 7= -22$ are all values in the range of $f$. \pspace

\item We have $f(4)= 3(4) - 7= 12 - 7= 5$. This implies that $(4, 5)$ is a point on the graph. Therefore, $(4, 1)$ cannot be on the graph of $f$. If it were on the graph, then we would know that $f(4)= 1$. But we know $f(4)= 5 \neq 1$. \pspace

\item If there were $x$ such that $f(x)= 11$, then\dots
	\[
	\begin{gathered}
	f(x)= 11 \\
	3x - 7= 11 \\
	3x= 18 \\
	x= 6
	\end{gathered}
	\]
Of course, this assumes there is an $x$ such that $f(x)= 11$; that is, we have shown that $x= 6$ is the only \textit{possible} value. We can verify this possible solution: $f(6)= 3(6) - 7= 18 - 7= 11$. Therefore, there is such an $x$-value---namely, $x= 6$. \pspace

\item If $1 \in f^{-1}(3)$, then $f(1)= 3$. We have $f(1)= 3(1) - 7= 3 - 7= -4$. Therefore, $1 \notin f^{-1}(3)$. \pspace

\item If $f^{-1}$ exists, then we know that $(f \circ f^{-1})(x)= f \big( f^{-1}(x) \big)$ and $(f^{-1} \circ f)(x)= f^{-1} \big( f(x) \big)$ for all $x$. But then we would have $f(f^{-1}(\pi))= \pi$ and $f^{-1}(f(\sqrt{2}))= \sqrt{2}$.
\end{enumerate}


\end{document}