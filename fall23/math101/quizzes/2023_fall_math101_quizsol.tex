\documentclass[11pt,letterpaper]{article}
\usepackage[lmargin=1in,rmargin=1in,bmargin=1in,tmargin=1in]{geometry}
\usepackage{style/quiz}
\usepackage{style/commands}

\usepackage{cancel} 	% Use Cancels

% -------------------
% Content
% -------------------
\begin{document}
\thispagestyle{title}

% Quiz 1
\quizsol \textit{True/False}: The integer 45 has prime factorization $45= 3 \cdot 15$, which shows that 3 and 15 are divisors of 45. Furthermore, we know that 1 is a multiple of 45. \pspace

\sol The statement is \textit{false}. While it is true that $45= 3 \cdot 15$ is a \textit{factorization} of 45, it is not a \textit{prime factorization} of 45 because $15= 3 \cdot 5$. The prime factorization of 45 is $45= 3^2 \cdot 5$. It is true that if $45= 3 \cdot 15$, then 3 and 15 are divisors of 45. Finally, while 1 is a divisor of 45 because $45= 45 \cdot 1$, 1 is not a multiple of 45 because there is not an integer $k$ such that $1= 45k$. \pvspace{1.3cm}



% Quiz 2
\quizsol \textit{True/False}: $\dfrac{\frac{2a}{b}}{\frac{4a}{bc}}= 8c$ \pspace

\sol The statement is \textit{false}. We have\dots
	\[
	\dfrac{\;\;\dfrac{2a}{b}\;\;}{\dfrac{4a}{bc}}= \dfrac{2a}{b} \cdot \dfrac{bc}{4a}= \dfrac{\cancel{2} \cancel{a}}{\cancel{b}} \cdot \dfrac{\cancel{b}c}{\cancel{4}^{\,2} \cancel{a}}= \dfrac{c}{2}
	\] \pvspace{1.3cm}



% Quiz 3
\quizsol \textit{True/False}: The expression $\dfrac{(x y^3)^{-2}}{(x^{-3} y^8)^2}$ when fully simplified is $\dfrac{x^4}{y^{22}}$. \pspace

\sol The statement is \textit{true}. We have\dots
	\[
	\dfrac{(x y^3)^{-2}}{(x^{-3} y^8)^2}= \dfrac{x^{-2} y^{-6}}{x^{-6} y^{16}}= \dfrac{x^6}{x^2 y^6 y^{16}}= \dfrac{x^6}{x^2 y^{22}}= \dfrac{x^4}{y^{22}}
	\] \pvspace{1.3cm}



% Quiz 4
\quizsol \textit{True/False}: $\left( \dfrac{(x^2 y^3)^4}{x^{-3} y^8} \right)^{-1/2}= \dfrac{1}{y^2 \sqrt[11]{x^2}}$ \pspace

\sol The statement is \textit{true}. We have\dots
	\[
	\hspace{-1.7cm} \left( \dfrac{(x^2 y^3)^4}{x^{-3} y^8} \right)^{-1/2}= \left( \dfrac{x^{-3} y^8}{(x^2 y^3)^4} \right)^{1/2}= \left( \dfrac{x^{-3} y^8}{x^8 y^{12}} \right)^{1/2}= \left( \dfrac{y^8}{x^3 x^8 y^{12}} \right)^{1/2}= \left( \dfrac{y^8}{x^{11} y^{12}} \right)^{1/2}= \left( \dfrac{1}{x^{11} y^4} \right)^{1/2}= \dfrac{1}{x^{11/2} y^{4/2}}= \dfrac{1}{y^2 \sqrt{x^{11}}}
	\]
Therefore, the quiz statement is false. The quiz statement has $\sqrt[11]{x^2}= x^{2/11}$ instead of $\sqrt{x^{11}}= x^{11/2}$. 



\newpage



% Quiz 5
\quizsol \textit{True/False}: The real number $0.123412341234\ldots$ is a rational number; therefore, one can find integers $a, b$ such that $\frac{a}{b}= 0.123412341234\ldots$. \pspace

\sol The statement is \textit{true}. A rational number is a real number of the form $\frac{a}{b}$, where $a, b$ are integers and $b \neq 0$. Equivalently, a rational number is a real number whose decimal expansion either terminates or repeats. Because the decimal expansion of $0.123412341234\ldots$ repeats, it must be that $0.123412341234\ldots$ is rational. Therefore, there must be integers $a, b$ such that $\frac{a}{b}= 0.123412341234\dots$. In fact, if $N= 0.123412341234\dots$, we have\ldots
	\begin{table}[!ht]
	\centering\small
	\begin{tabular}{rccc}
	& $10000N$ & $=$ & $1234.123412341234\overline{1234}$ \\ 
	$-$ & $N$ & $=$ & $\phantom{123}0.123412341234\overline{1234}$ \\ \hline
	& $9999N$ & $=$ & $1234$ \\[0.1cm]
	& $N$ & $=$ & $\frac{1234}{9999}$
	\end{tabular}
	\end{table} \pvspace{1.3cm}



% Quiz 6
\quizsol \textit{True/False}: Suppose a course has grade components of homework (50\%), quizzes (10\%), a midterm (20\%), and a final (20\%). If you had a 80\% homework average, 75\% quiz average, and received a 60\% on the midterm, then your average is\dots
	\[
	0.50(80\%) + 0.10(75\%) + 0.20(60\%)= 40\% + 7.5\% + 12\%= 59.5\%
	\] \pspace

\sol The statement is \textit{false}. One's course average is a weighted average where each percentage earned is weighted by the components worth. But then\dots
	\[
	\text{Course Average}= \dfrac{\sum w_i x_i}{\sum w_i}= \dfrac{0.50 \cdot 0.80 + 0.10 \cdot 0.75 + 0.20 \cdot 0.60}{0.50 + 0.10 + 0.20}= \dfrac{0.40 + 0.075 + 0.12}{0.80}= \dfrac{0.595}{0.80}= 0.74375
	\] \pvspace{1.3cm}


%The real number $0.1 \cdot 10^3$ is in scientific notation.
%The surface area of a box that is open at the top with dimensions 1~ft $\times$ 8~in $\times$ 5~in is $\text{SA}= 12 \cdot 8 + 2(8 \cdot 5) + 2(12 \cdot 5)= 296 \text{ in}^2$.
%The relation with domain $\mathbb{R}^3$ and codomain $\mathbb{R}$ given by $f(x,y,z)= x^2yz - yz^2 + 6$ is a function. 
%If $\psi$ is a function and $\psi(4)= 10= \psi(-2)$, then $\psi^{-1}$ exists and $\psi^{-1}(10)= 4$. 
%The point $(-\frac{1}{2}, 3)$ is on the graph of $f(x)= 4x + 5$. 
%There exists a function, $f$, with $x$-intercepts $-1, 0, 1$ such that $f^{-1}$ exists. 

%If you are driving down the highway at 65~mph from Albany to NYC, then your distance from NYC is given by a linear function. 
%There exists a horizontal line that is perpendicular to $y= 5x - 3$. 

% Three lines, none of which are parallel to the others, must intersect at a distinct point. 




















\end{document}