\documentclass[11pt,letterpaper]{article}
\usepackage[lmargin=1in,rmargin=1in,bmargin=1in,tmargin=1in]{geometry}
\usepackage{style/quiz}
\usepackage{style/commands}

% -------------------
% Content
% -------------------
\begin{document}
\thispagestyle{title}

% Quiz 1
\quizsol \textit{True/False}: The function $f(x)= 9 - 5x$ is a linear function with slope $5$ and $y$-intercept $9$. \pspace

\sol The statement is \textit{false}. We know a function of the form $f(x)= mx + b$ is a linear function with slope $m$ and $y$-intercept $b$. Because we have $f(x)= 9 - 5x= -5x + 9$, we have $m= -5$, i.e. slope $-5$, and $y$-intercept $9$, i.e. $(0, 9)$. But then the slope is $-5$, not the given value of $5$. \pvspace{1.3cm}



% Quiz 2
\quizsol \textit{True/False}: If $f(x)= 2x - 1$ and $g(x)= 3 - x$, then $(f \circ g)(0)= f(0)g(0)= -1 \cdot 3= -3$. \pspace

\sol The statement is \textit{false}. First, note that $f(0)= 2(0) - 1= -1$, $g(0)= 3 - 0= 3$, and $f(3)= 2(3) - 1= 6 - 1= 5$. What was given was function multiplication, i.e. what was computed was $(fg)(0)= f(0) g(0)= -1 \cdot 3= -3$. What was originally written was function composition. We have $(f \circ g)(0)= f\big(g(0) \big)= f(3)= 5$. \pvspace{1.3cm}



% Quiz 3
\quizsol \textit{True/False}: Compared to the graph of $f(x)$, the graph of $5 - 3f(x + 2)$ is stretched by a factor of 3, then shifted to the right by 2 and up by 5. \pspace

\sol The statement is \textit{false}. We know that $f(x + 2)$ is the graph of $f(x)$ shifted 2 to the \textit{left}. The graph of $-3f(x + 2)$ is then the graph of $f(x)$ shifted two to the left, stretched by a factor of 3, and reflected across the $x$-axis. Finally, the graph of $5 - 3f(x + 2)$ is the graph of $f(x)$ shifted two to the left, stretched by a factor of 3, reflected across the $x$-axis, then shifted upwards by 5. \pvspace{1.3cm}



% Quiz 4
\quizsol \textit{True/False}: The function $f(x)= 4(5^{-x})$ is a concave up, decreasing, exponential function. \pspace

\sol The statement is \textit{true}. A function of the form $f(x)= Ab^x$ is an exponential function. We can summarize whether $f(x)$ is increasing or decreasing and concave up or down as follows:
        \begin{table}[!ht]
        \centering
        \begin{tabular}{cll}
        \multicolumn{1}{l}{} & \multicolumn{1}{c}{$0 < b < 1$} & \multicolumn{1}{c}{$b > 1$} \\ \cline{2-3} 
        \multicolumn{1}{c|}{$A > 0$} & \multicolumn{1}{l|}{Decreasing, Concave Up} & \multicolumn{1}{l|}{Increasing, Concave Up} \\ \cline{2-3} 
        \multicolumn{1}{c|}{$A < 0$} & \multicolumn{1}{l|}{Increasing, Concave Down} & \multicolumn{1}{l|}{Decreasing, Concave Down} \\ \cline{2-3} 
        \end{tabular}
        \end{table}	
But we have $f(x)= 4(5^{-x})= 4 (5^{-1})^x= 4 \left( \dfrac{1}{5} \right)^x$. Therefore, $f(x)$ is exponential with $A= 4 > 0$ and $0 < b= \frac{1}{5} < 1$. Therefore, $f(x)$ is a decreasing, concave up, exponential function. \pvspace{1.3cm}



% Quiz 5
\quizsol \textit{True/False}: The function $f(x)= 5(2^{1 - 2x})$ is equal to the function $g(x)= 10 \left( \dfrac{1}{4} \right)^x$. \pspace

\sol The statement is \textit{true}. Observe that we have\dots
	\[
	f(x)= 5(2^{1 - 2x})= 5 \cdot 2^1 \cdot 2^{-2x}= 10 \cdot 2^{-2x}= 10 (2^{-2})^x= 10 \left( \dfrac{1}{2^2} \right)^x= 10 \left( \dfrac{1}{4} \right)^x= g(x)
	\]




% log_5(4 ^-4)= -3


% $\ln\left( \dfrac{x^5}{\sqrt[3]{y} \right)= 5 \ln(x) - \frac{1}{3} \ln(y)$


%If $2^{\sqrt{x}} - 5= 3$, then $x= 9$. 


% $\tan(\theta) \cot(\theta)= 1$






















\end{document}