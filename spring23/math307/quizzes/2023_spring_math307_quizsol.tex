\documentclass[11pt,letterpaper]{article}
\usepackage[lmargin=1in,rmargin=1in,bmargin=1in,tmargin=1in]{geometry}
\usepackage{style/quiz}
\usepackage{style/commands}

% -------------------
% Content
% -------------------
\begin{document}
\thispagestyle{title}

TBA

%% Quiz 1
%\quizsol \textit{True/False}: Any function of one-variable which has a constant rate of change can be written in the form $f(x)= mx + b$ for some values $m$ and $b$. \pspace
%
%\sol The statement is \textit{true}. 

% Prunella is trying to find the cardinality of the set $\{ 1, 2, 1, 4, 1, 8 \}$. She counts how many numbers are in the set and finds that there are six numbers. Therefore, the cardinality of the set is 6. 

% If $A$ is a set with $5$ elements and $B$ is a set with $3$ elements, then $A - B$ is a set with $2$ elements. 

% The number of ways of choosing three candle sticks from a collection of five to arrange on a mantle is $_5P_3= 5 \cdot 4 \cdot 3= 60$ possible arrangements. 

\end{document}