\documentclass[11pt,letterpaper]{article}
\usepackage[lmargin=1in,rmargin=1in,tmargin=1in,bmargin=1in]{geometry}
\usepackage{../style/homework}
\usepackage{../style/commands}
\setbool{quotetype}{false} % True: Side; False: Under
\setbool{hideans}{true} % Student: True; Instructor: False

% -------------------
% Content
% -------------------
\begin{document}

\homework{8: Due 03/01 (02)}{Statistically, the probability of any one of us being here is so small that you'd think the mere fact of existing would keep us all in a contented dazzlement of surprise.}{Lewis Thomas}

% Problem 1
\problem{10} The number of hours Elizabeth has spent on TokTik over the past few days is given below:
	\[
	0 \qquad 1 \qquad 6 \qquad 1 \qquad 3 \qquad 6 \qquad 2 \qquad 6 \qquad 9 \qquad 7 \qquad 8  
	\] 

\begin{enumerate}[(a)]
\item Find the median of this dataset.
\item Find the IQR of this dataset. 
\item A 5-number summary for a dataset consists of the min, $Q_1$, median, $Q_3$, and max for the data. Find the 5-number summary for this dataset. 
\end{enumerate}



\newpage



% Problem 2
\problem{10} Will took the SAT and received a 1650 while Chris took the ACT and received a 23. Suppose that both the SAT and the ACT had scores which were normally distributed. Furthermore, suppose that the SAT had a mean score of 1500 and standard deviation 300 while the ACT had a mean score of 21 and standard deviation of 5. Relative to their own exams, who did better? Be sure to justify your answer. \pspace



\newpage



% Problem 3
\problem{10} STACKS is a local college. At the school, GPAs are approximately normally distributed with mean 3.205 and standard deviation 0.27. 
	\begin{enumerate}[(a)]
	\item Find the percentage of students that have a GPA lower than 2.8.
	\item Find the percentage of students that have a GPA greater than 3.5. 
	\item Find the percentage of students that have a GPA between 2.8 and 3.5. 
	\end{enumerate}



\newpage



% Problem 4
\problem{10} Margarita claims that you can generate a random sample of the numbers 2 through 12 by continuously rolling two die and taking the sum of the numbers that appear.\footnote{By a random sample of a finite set of numbers, we mean that the probability of all elements of the sample space are equal.} Explain why Margarita is incorrect. Be sure to include the concept of bias in your response. How would you help Margarita understand why she is wrong? \pspace


\end{document}