\documentclass[11pt,letterpaper]{article}
\usepackage[lmargin=1in,rmargin=1in,tmargin=1in,bmargin=1in]{geometry}
\usepackage{../style/homework}
\usepackage{../style/commands}
\setbool{quotetype}{false} % True: Side; False: Under
\setbool{hideans}{false} % Student: True; Instructor: False

% -------------------
% Content
% -------------------
\begin{document}

\homework{3: Due 02/15 (16)}{For the things we have to learn before we can do them, we learn by doing them.}{Aristotle}

% Problem 1
\problem{10} A box of candy contains 30~different pieces of candy. Of these pieces, 10 are filled with chocolate, 10 are filled with nougat, and 6 are filled with both. 
	\begin{enumerate}[(a)]
	\item How many ways can you select a candy that contains chocolate or nougat?
	\item How many pieces of candy contain neither chocolate nor nougat?
	\item How many ways can you select a candy that is filled with chocolate or something other than chocolate or nougat?
	\end{enumerate} \pspace

\sol 
\begin{enumerate}[(a)]
\item There are 10 candies filled with chocolate and 10 candies filled with nougat. However, there are 6 filled with both. Then there are $10 + 10 - 6= 14$ candies filled with chocolate or nougat. There are then 14 ways to select a candy with chocolate or nougat. Using counting principles, let $A$ be the set of chocolate filled candies and $B$ be the set of nougat filled candies. Then $|A|= 10$, $|B|= 10$, and $|A \cap B|= 6$ (because there are 6 candies with both chocolate and nougat). Therefore, the amount of candies filled with either chocolate or nougat, i.e. $|A \cup B|$, is\dots
	\[
	|A \cup B|= |A| + |B| - |A \cap B|= 10 + 10 - 6= 14
	\] \pspace

\item From (a), we know there are 14 candies filled with either chocolate or nougat. There are 30 pieces of candy in total. Therefore, there are $30 - 14= 16$ candies filled with neither chocolate nor nougat. Using counting principles, let $C$ be the set of candies, $A$ be the set of chocolate filled candies, and $B$ be the set of nougat filled candies. From (a), we know $|A \cup B|= 14$. We know also that $|C|= 30$. The set $(A \cup B)^c$ is the set of candies that contain neither chocolate nor nougat. Because $A \cup B \subseteq C$, we have\dots
	\[
	|(A \cup B)^c|= |C| - |A \cup B|= 30 - 14= 16
	\] \pspace

\item There are 10~candies filled with chocolate. From (b), we know there are 16 candies filled with neither chocolate nor nougat. Therefore, there are $10 + 16= 26$ candies either filled with chocolate or neither chocolate nor nougat. Using counting principles, let $A$ be the set of candies filled with chocolate and $B$ be the set of candies filled with neither chocolate nor nougat. Using the problem statement and (b), we know $|A|= 10$ and $|(A \cup B)^c|= 16$. Because $A$ and $(A \cup B)^c$ are disjoint, we have\dots
	\[
	|A \cup (A \cup B)^c|= |A| + |(A \cup B)^c| - |A \cap (A \cup B)^c|= 10 + 16 - 0= 26
	\]
\end{enumerate}



\newpage



% Problem 2
\problem{10} A restaurant has a breakfast special. The special consists of a selection of a sandwich, a side, and some beverage. The restaurant offers 5 different breakfast sandwiches, 8 different sides, and 3 different beverages. You must choose two different sides. How many possible breakfast orders are there? \pspace

\sol To choose a meal, you must choose a sandwich, two \textit{different} sides, and a beverage. There are 5 choices for sandwich. One must choose two sides but they must be different, i.e. the sides cannot repeat. The order one selects the side does not matter. Therefore, there are $_8C_2= 28$ different ways to choose two different sides.\footnote{One can also count this as follows: there are 8 choices for the first side. Because the second side must be different, there are then 7 possible choices for the second side. Then there are $8 \cdot 7= 56$ different ways to select two sides. But if one chose the sides in the opposite order, it still results in the same selection of sides. For instance, if one chooses bacon from the 8 choices for the first side then home fries from the 7 remaining choices for the second choice. However, this is the same selection of sides if one chose home fries from the 8 choices for the first selection and then bacon from the remaining 7 choices for the second side. So one has doubled the number of actual possible selections. Therefore, there are $\frac{56}{2}= 28$ total possible combinations of sides.} There are 3 choices of beverage. Therefore, the number of ways to choose a sandwich \textit{and} a two different sides \textit{and} a beverage is\dots
	\[
	5 \cdot 28 \cdot 3= 420
	\]



\newpage



% Problem 3
\problem{10} Showing all necessary work, compute the following:
	\begin{enumerate}[(a)]
	\item $8!$
	\item $0!$
	\item $_9P_2$
	\item $\binom{6}{4}$
	\item $_{15}C_3$
	\end{enumerate} \pspace

\sol 
\begin{enumerate}[(a)]
\item 
	\[
	8!= 8 \cdot 7 \cdot 6 \cdot 5 \cdot 4 \cdot 3 \cdot 2 \cdot 1= 40320
	\] \pspace

\item 
	\[
	0!= 1
	\] \pspace

\item 
	\[
	_9P_2= \dfrac{9!}{(9 - 2)!}= \dfrac{9!}{7!}= \dfrac{9 \cdot 8 \cdot 7!}{7!}= \dfrac{9 \cdot 8 \cdot \cancel{7!}}{\cancel{7!}}= 9 \cdot 8= 72
	\] \pspace

\item 
	\[
	\binom{6}{4}= \dfrac{6!}{4! (6 - 4)!}= \dfrac{6!}{4! 2!}= \dfrac{6 \cdot 5 \cdot 4!}{4! 2!}= \dfrac{6 \cdot 5 \cdot \cancel{4!}}{\cancel{4!} 2!}= \dfrac{6 \cdot 5}{2 \cdot 1}= \dfrac{\cancel{6}^3 \cdot 5}{\cancel{2} \cdot 1}= 15
	\] \pspace

\item 
	\[
	\hspace{-1cm} _{15}C_3= \dfrac{15!}{3! (15 - 3)!}= \dfrac{15!}{3! 12!}= \dfrac{15 \cdot 14 \cdot 13 \cdot 12!}{3! 12!}=  \dfrac{15 \cdot 14 \cdot 13 \cdot \cancel{12!}}{3! \cancel{12!}}= \dfrac{15 \cdot 14 \cdot 13}{3 \cdot 2 \cdot 1}= \dfrac{\cancel{15}^5 \cdot \cancel{14}^7 \cdot 13}{\cancel{3} \cdot \cancel{2} \cdot 1}= 5 \cdot 7 \cdot 13= 455
	\]
\end{enumerate}



\newpage



% Problem 4
\problem{10} There are 19 students in a homeroom. Of these students, 8 are in choir, 5 are in orchestra, and 1 is in both. A student claims that there are $8 + 1= 9$ students in choir and that there are a total of 5 students that are in neither orchestra nor choir. Is the student correct? Explain. \pspace

\sol The student is not correct. If one is adding counts for collections of objects, one need worry that one is overcounting. If one states that there are $8 + 5= 13$ people in choir and orchestra, they may have counted a person in choir once in the 8~people and again when one counted the 5~people in orchestra if there are people that are in both choir and orchestra. If one adds counts for collections of objects, one need subtract anything that might have been counted multiple times. So if one adds the number of people in choir and the number of people in orchestra, one need subtract the amount of people that are in both to obtain the correct number of people that are in choir or orchestra. Therefore, the correct number of people in choir or orchestra is $8 + 5 - 1= 12$. 


\end{document}