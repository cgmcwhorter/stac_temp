\documentclass[11pt,letterpaper]{article}
\usepackage[lmargin=1in,rmargin=1in,tmargin=1in,bmargin=1in]{geometry}
\usepackage{../style/homework}
\usepackage{../style/commands}
\setbool{quotetype}{true} % True: Side; False: Under
\setbool{hideans}{false} % Student: True; Instructor: False

% -------------------
% Content
% -------------------
\begin{document}

\homework{2: Due 02/13 (14)}{I am always doing that which I cannot do, in order that I may learn how to do it.}{Pablo Picasso}

% Problem 1
\problem{10} Define the following sets:
	\[
	\begin{aligned}
	A&= \text{the set of geometric objects colored blue} \\
	B&= \text{the set of triangles} \\
	C&= \text{the set of circles} \\
	D&= \text{the set of `large' geometric objects} \\
	E&= \text{the set geometric objects colored red}
	\end{aligned}
	\]
Describe the shapes found in the following sets:
	\begin{enumerate}[(a)]
	\item $A \cap C$
	\item $B \cap D$
	\item $B \cup C$
	\item $C \cap E^c$
	\item $B \cap A \cap D$
	\end{enumerate} \pspace

\sol 
\begin{enumerate}[(a)]
\item The set $A \cap C$ is the collection of elements that are in \textit{both} $A$ and $C$. The elements of $A$ are blue geometric objects and the elements of $C$ are circles. But then $A \cap C$ consists of objects which are blue and circles. Therefore, we have\dots
	\[
	A \cap C= \{ \text{ blue circles } \}
	\] \pspace

\item The set $B \cap D$ is the collection of elements that are in \textit{both} $B$ and $D$. The elements of $B$ are triangles and the elements of $D$ are the set of `large' geometric objects. But then $B \cap D$ consists of objects that are triangles and that are `large' geometric objects. Therefore, we have\dots
	\[
	B \cap D= \{ \text{ large triangles } \}
	\] \pspace
 
\item The set $B \cup C$ is the collection of elements that are in $B$ or in $C$. The elements of $B$ are triangles and the elements of $C$ are circles. But then $B \cup C$ consists of triangles or circles. Therefore, we have\dots
	\[
	B \cup C= \{ \text{ triangles or circles } \}
	\] \pspace

\item The set $C \cap E^c$ is the collection of elements that are in $C$ and $E^c$. The elements of $C$ are circles and the elements of $E^c$ are elements that are not in $E$, i.e. geometric objects that are not colored red. But then $C \cap E^c$ consists of objects that are circles and are geometric objects that are not red. Therefore, we have\dots
	\[
	C \cap E^c= \{ \text{ circles that are not red } \} 
	\] \pspace

\item The set $B \cap A \cap D$ is the collection of elements that are in $B$, $A$, and $D$. The elements of $A$ are blue geometric objects. The elements of $B$ are triangles. The elements of $D$ are `large' geometric objects. But then $B \cap A \cap D$ consists of objects that are triangles and are blue geometric objects and are `large' geometric objects. Therefore, we have\dots
	\[
	B \cap A \cap D= \{ \text{ large blue triangles } \} 
	\]
\end{enumerate}



\newpage



% Problem 2
\problem{10} Define the following subsets of the integers: $S= \{ n \colon n= 3k + 1 \text{ for some integer } k \}$, $T$ is the set of even numbers, and $U= \{ 1, 2, 3, 4, 5, 6 \}$. 
	\begin{enumerate}[(a)]
	\item Is $5 \in S$? Explain.
	\item Is $10 \in S$? Explain.
	\item Find $S \cap U$.
	\item What is the value of $|U - T|$?
	\end{enumerate} \pspace

\sol 
\begin{enumerate}[(a)]
\item If $5 \in S$, then $n= 5$ and there is a $k$ such that $5= 3k + 1$ for some integer $k$. But then\dots
	\[
	\begin{aligned}
	5&= 3k + 1 \\[0.3cm]
	4&= 3k \\[0.3cm]
	k&= \dfrac{4}{3}
	\end{aligned}
	\]
But $k= \frac{4}{3}$ is not an integer. Therefore, $5 \notin S$. \pspace

\item Let $k= 3$. Then $3(3) + 1= 9 + 1= 10 \in S$. Therefore, $10 \in S$. Alternatively, if $10 \in S$, then $n= 10$ and $10= 3k + 1$ for some integer $k$. But then\dots
	\[
	\begin{aligned}
	10&= 3k + 1 \\[0.3cm]
	9&= 3k \\[0.3cm]
	k&= 3
	\end{aligned}
	\]
Because $k= 3$ is an integer, we know that $10 \in S$. \pspace

\item The objects of $S \cap U$ are the elements that are in \textit{both} $S$ and $U$. We know that $U= \{ 1, 2, 3, 4, 5, 6 \}$. The integers $3k$, where $k$ is an integer, are the multiples of 3. But then the numbers $3k + 1$ are the integers that are `1 above' a multiple of 3. Then the elements of $S \cap U$ are the elements of $U$ that are `1 above' a multiple of 3. Therefore, we have\dots
	\[
	S \cap U= \{ 1, 4 \} 
	\]

\item The set $U - T$ is the collection of elements of $U$ that are \textit{not} elements of $T$. The elements of $U$ are $\{ 1, 2, 3, 4, 5, 6 \}$. The elements of $T$ are even numbers. Therefore, the elements of $U - T$ are the elements of $\{ 1, 2, 3, 4, 5, 6 \}$ that are not even. Therefore, $U - T= \{ 1, 3, 5 \}$. We know $|U - T|$ is the cardinality, or size of the set $U - T$; that is, $|U - T|$ is the number of distinct elements of $U - T$. Therefore, we have\dots
	\[
	|U - T|= 3
	\]
\end{enumerate}



\newpage



% Problem 3
\problem{10} You are teaching a class where you are introducing the concept of cardinality or `size' of a set. 
	\begin{enumerate}[(a)]
	\item You define a set $S$ to be the set of grains of sand found in all the beaches across the world. You ask students whether or not $S$ is finite. Your class decides that the set $S$ is infinite. Are they correct? Explain. 
	\item You ask your students whether the set $[1, 6)$ is finite or infinite. You break your students into groups. One of the groups states that because $[1, 6)= \{ 1, 2, 3, 4, 5, 6 \}$ that the set is finite. Are they correct? If they are correct, explain why. If they are not correct, state everything they have done incorrectly and give the correct answer. 
	\end{enumerate} \pspace

\sol 
\begin{enumerate}[(a)]
\item They are not correct. They are confusing `very large' with infinite. There are indeed many, many, \textit{many} grains of sand on planet Earth. Some estimates of the number of grains of sand on Earth are $7.5 \cdot 10^{18}$. In any case, while the number is large, it is still finite. \pspace

\item First, the interval $[1, 6)$ is the set $[1, 6)= \{ x \colon 1 \leq x < 6 \}$; that is, $[1, 6)$ is the set of numbers greater than or equal to 1 but \textit{less than} 6. But then $6 \notin [1, 6)$ because $6 \not< 6$ (but it is true that $6 \leq 6$). They have included 6 in the set $[1, 6)$, which is incorrect. However, they forgot that while there are only a few `nice' numbers, i.e. integers, at least 1 but less than 6 (the integers 1, 2, 3, 4, 5), there are infinitely many numbers in the interval $[1, 6)$. For instance, $1.11341$, $5.003$, $\frac{14}{3}$, $\pi$, etc. are all in the interval $[1, 6)$. To `easily' see that $[1, 6)$ contains infinitely many numbers. Observe that $1.0, 1.1, 1.11, 1.111, \ldots$ are all distinct and in the interval $[1, 6)$. 
\end{enumerate}



\newpage



% Problem 4
\problem{10} Students are studying for their state exams. They are given the sets $A= \{ 1, 2, 3, 4, 2, 3 \}$, $B= \{ 1, 2, 3, 4 \}$, and $C= \{ 5, 4, 3, 2, 1 \}$. 
	\begin{enumerate}[(a)]
	\item Students claim that $A \neq B$ because $A$ has more elements than $B$. Are they correct? Explain. 
	\item Students claim that $B \not\subseteq C$ because the elements of $C$ appear in the reverse order of the elements of $B$. Are they correct? Explain. 
	\end{enumerate} \pspace

\sol 
\begin{enumerate}[(a)]
\item Recall that a set is about the \textit{collection} of objects. Repetition or order of elements in a set does not matter. Then $A= \{ 1, 2, 3, 4, 2, 3 \}= \{ 1, 2, 3, 4 \}$. But this is exactly the set $B$. Therefore, $A= B$. The students have not understood that the order of the elements and repetition of elements of a set do not change the set. \pspace

\item The order of the elements of a set does not matter. Therefore, $C= \{ 5, 4, 3, 2, 1 \}= \{ 1, 2, 3, 4, 5 \}$. Now $X \subseteq Y$ if every element of $X$ is an element of $Y$. Observe that each of the integers 1, 2, 3, and 4 are elements of $B$. The integers 1, 2, 3, and 4 are elements of $C$. But then every element of $B$ is an element of $C$. Therefore, $B \subseteq C$. The students have misunderstood the fact that the elements of a subset of another set need not appear in the same order in both sets. 
\end{enumerate}




















\end{document}