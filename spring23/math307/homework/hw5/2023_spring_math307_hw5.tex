\documentclass[11pt,letterpaper]{article}
\usepackage[lmargin=1in,rmargin=1in,tmargin=1in,bmargin=1in]{geometry}
\usepackage{../style/homework}
\usepackage{../style/commands}
\setbool{quotetype}{true} % True: Side; False: Under
\setbool{hideans}{true} % Student: True; Instructor: False

% -------------------
% Content
% -------------------
\begin{document}

\homework{5: Due 02/27 (28)}{From principles is derived probability, but truth or certainty is obtained only from facts.}{Tom Stoppard}

% Problem 1
\problem{10} Joey is drawing marbles from a bag. The bag contains 50 marbles. Of these marbles, 25 are red, 10 are blue, and 15 are green. He will draw two random marbles sequentially from the bag. After drawing a marble, Joey will put the marble back in the bag.
	\begin{enumerate}[(a)]
	\item What is the probability that both marbles are red?
	\item What is the probability that neither marble is red?
	\item What is the probability that he draws a blue and green marble, in either order?
	\item What is the probability that both marbles have the same color?
	\end{enumerate}



\newpage



% Problem 2
\problem{10} A standard deck of cards consists of 52 cards. There are 13 of each of the different suits of cards: hearts, diamonds, spades, and clubs. Each suit consists of cards numbered 2 through 10 along with the cards jack, queen, king, and ace. Hearts and diamonds are colored red while spades and clubs are colored black. Suppose that you will draw two random cards from the deck without replacement.
	\begin{enumerate}[(a)]
	\item What is the probability that you draw two aces?
	\item What is the probability that you draw two black cards?
	\item What is the probability that you draw two aces or two black cards?
	\item What is the probability that both cards are from the same suit?
	\end{enumerate}



\newpage



% Problem 3
\problem{10} Suppose you are going to invite two of your friends to the movies. You decide that you are going to choose from your friends Alice, Susan, Sally, Bill, and Joe. Because you cannot decide, you will choose the friends that go with you randomly. 
	\begin{enumerate}[(a)]
	\item How many total ways could you choose the friends that go with you?
	\item What is the probability that you go with Alice and Bill?
	\item What is the probability that you go with only girls?
	\item What is the probability that you go with one boy and one girl?
	\end{enumerate}


\newpage



% Problem 4
\problem{10} Tina B. is trying to determine the probability that if you draw two cards from a deck of cards, one after the other, that they are both red. She states that because there are 26 red cards in a deck of cards, the probability you draw a red card is $\frac{26}{52}= \frac{1}{2}$. Tina then claims the probability that both the cards are red is then $\frac{1}{2} \cdot \frac{1}{2}= \frac{1}{4}$. Explain why Tina is wrong using the description of the card drawing. Explain why Tina is wrong using the statistical concept of independence. 


\end{document}