\documentclass[11pt,letterpaper]{article}
\usepackage[lmargin=1in,rmargin=1in,bmargin=1in,tmargin=1in]{geometry}
\usepackage{style}

\setlength{\parindent}{0ex}

% -------------------
% Content
% -------------------
\begin{document}

\begin{center} {\bfseries \Large MATH 307 --- Spring 2023 \par \vspace{0.3cm} Paper \& Poster} \end{center}

\noindent {\itshape Purpose:} MATH 307 is a gateway course. Of course, this requires you to write a paper that demonstrates the academic sophistication that you have developed as a student at STAC thus far. Of course, this course being preparation for the mathematics portion of your certification exams as well as continued preparation for your future careers as educators, the project will be on your ability to apply course concepts in an educational environment. This will allow you to demonstrate your mastery of not only the course material but your command of teaching pedagogy. \pspace

\noindent {\itshape What:} Students will choose a topic related to the course materials in Probability, Statistics, or Data Science. For a topic area in Probability, one might choose probabilistic independence, Law of Large Numbers, fundamental laws of probability, expected value, etc. For a topic area in Statistics, one might choose statistical versus practical significance, Central Limit Theorem, data collection/sampling, etc. For a topic area in Data Science, one might choose general AI, machine learning, big data, etc. The topic should have `real world implications.' Because of this, it might be easiest to choose a topic by finding news articles, papers, interviews/discussions, etc. on `real world' issues related to these areas. Finding a common theme related to course materials among several of them, one can then synthesize these together in a common theme. The purpose of the paper is then to inform a reader on these course material related issues and the real world implications and propose a lecture, appropriate for some K--12 grade, that would help prepare students to engage with these issues when they leave school. Students will also create a short `poster' that could serve as an `advertisement' highlighting the issue as well as how people should think about or approach these issue. \pspace

\noindent {\itshape When:} The project and poster will be due by the end of the day Thursday, May 4th, 2023. \pspace

\noindent {\itshape Who:} Students are allowed to work in small groups to help collect and collate materials to help write the paper. Students are also allowed to discuss materials to facilitate writing a clear paper and creating an engaging and useful poster. They are also free to read and make suggestions on each others papers and posters. However, students' paper and poster should be their own and should not have `significant' similarity to other students' papers. Using others, including person(s) outside the college, any artificial intelligence, etc. to write the paper (in part or whole) or create the poster (in part or whole) is a violation of STAC's academic integrity policies. \pspace

\noindent {\itshape Requirements:} The paper should be on a topic related to the course material in Probability, Statistics, or Data Science. The paper should be no less than 3~pages, single spaced, size 11 font, with 1'' margins. Papers should begin with an introduction that captures readers attention and introduces the issue(s) that will be discussed in the paper. The paper should then give a brief summary of the mathematics that will underline the paper, assuming that the reader has little to no background in the material. Then students should introduce three to five occurrences of the chosen topic in the `real world', e.g. through news articles. The materials presented should be thorough and from `reputable' sources. Each of these should include a summary of the occurrence as well as the mathematics related to the issue. One should then briefly discuss how all the issues discussed are mathematically related. The paper should then discuss an engaging lecture (or series of lectures), appropriate for some K--12 classroom, that would inform students on these issues and the mathematics related to them. The proposal should be able to be easily adapted for other educators in a variety of settings. This lecture should ultimately provide a framework that, if retained, would better prepare the students to be better engaged citizens once they leave school. Finally, the paper should have a conclusion that summarizes the issues presented, mathematics discussed, and educational proposals made. The paper should end with a bibliography (not included in the length of the paper). \pspace

The `poster' created should be based on the paper written, i.e. it cannot be on a `tangental' topic. The poster should fill a standard 8.5" x 11" paper. The poster should capture a viewers attention, i.e. contain something to engage a viewer and make them want to stop and read the content. The main content of the poster should inform a reader to a real world issue from the paper, help prepare a reader to be alert to the issue going forward, and give a reader a more informed view to engage with the issue(s) going forward. The poster content should be clear, concise, and look `professional.' The poster should not be overly dense nor sparse and should be visually interesting. It is strongly encouraged to include easily accessible resources in the poster, e.g. a QR code linking to a resource where the reader could be more informed. \pspace

\noindent {\itshape Grading Rubric:} The grading rubric for the paper and poster are found below. \pspace

\indent Paper Grading Rubric

\begin{itemize}

\item\textbf{Overall Organization and Coherence}
\begin{itemize}
\item[\underline{\hspace{.2in}}] Demonstrates a logical plan of organization and coherence 
in the development of ideas (3 points)
\item[\underline{\hspace{.2in}}] Has a plan of organization and a satisfactory development of ideas (2 points)
\item[\underline{\hspace{.2in}}]  Demonstrates weakness in organization and the development of ideas (1 point)
\item[\underline{\hspace{.2in}}]  Shows lack of organization and 
development of ideas (0 points)
\end{itemize}

\item  \textbf{Articulation of Main Mathematical Ideas}
\begin{itemize}
\item[\underline{\hspace{.2in}}]  clear articulation of main ideas and arguments (3 points)
\item[\underline{\hspace{.2in}}] some articulation of main ideas (2 points)
\item[\underline{\hspace{.2in}}]  poor articulation of main ideas (1 point)
\item[\underline{\hspace{.2in}}] no articulation of main ideas (0 points)
\end{itemize}

\item  \textbf{Use of Examples and Supporting Data}
\begin{itemize}
\item[\underline{\hspace{.2in}}] strong usage of examples and evidence for the issue(s) as well as the mathematics and teaching pedagogy; evidence and examples always appear when the reader asks; uses more than the required examples from reputable sources (4 points)
\item[\underline{\hspace{.2in}}]  frequent or consistent use of examples and evidence; example or evidence appears whenever the reader asks, ``For instance?" Uses at least 3 to 5 examples from reputable sources (3 points)
\item[\underline{\hspace{.2in}}] some use of examples and/or evidence, but not consistent; no examples or evidence in places where they are needed. Uses less than the required 3 examples (2 points)
\item[\underline{\hspace{.2in}}] little use of examples (1 point)
\item[\underline{\hspace{.2in}}] no use of examples (0 points)
\end{itemize}

\item \textbf{Use of Mathematical Reasoning}
\begin{itemize}
\item[\underline{\hspace{.2in}}] identification of mathematics involved is supported with logic throughout the report; thoughtful mathematics is posited and explained clearly and correctly (6 points)
\item[\underline{\hspace{.2in}}] identification of mathematics involved is supported with logic throughout the report; thoughtful mathematics are posited and explained, but lack complete clarity (5 points)
\item[\underline{\hspace{.2in}}] some identification of mathematics involved; some mathematics supported with mathematical logic, while others are not; a small amount of incorrect reasoning applied (4 points)
\item[\underline{\hspace{.2in}}] some  identification of mathematics involved; several pieces of mathematics supported with correct logical or data, but one or two ideas have significant logical/reasoning errors (3 points)
\item[\underline{\hspace{.2in}}] some  identification of mathematics involved; little attempt at reasoning or significant errors throughout (2 points)
\item[\underline{\hspace{.2in}}] little or no identification of mathematics involved; just an idea dump (1 point)
\item[\underline{\hspace{.2in}}] no clear usage of reasoning (0 points)
\end{itemize}

\item \textbf{Educational Discussion}
\begin{itemize}
\item[\underline{\hspace{.2in}}] the lecture suggested is clear, adaptable, and well designed; the lecture is related to the issue(s) discussed and is mathematically sound. (6 points)
\item[\underline{\hspace{.2in}}] the lecture suggested is mostly clear, adaptable, and well designed but may have some small issues; the lecture is related to the issue(s) discussed and is mostly mathematically correct, but may have some issues. (5 points)
\item[\underline{\hspace{.2in}}] the lecture suggested is mostly clear, adaptable, and well designed but may have several issues; the lecture is related to the issue(s) discussed and is mostly mathematically correct, but may have several issues. (4 points)
\item[\underline{\hspace{.2in}}] the lecture suggested has several issues or is not clearly related to the issue(s) discussed and there may be several mathematical issues (3 points)
\item[\underline{\hspace{.2in}}] the lecture suggested has many issues or is not related to the issue(s) discussed and there are several mathematical issues (2 points)
\item[\underline{\hspace{.2in}}] the lecture suggested is seriously flawed, either in the quality of the lecture proposed or because of its mathematical content (1 point)
\item[\underline{\hspace{.2in}}] the lecture suggested is incoherent or mathematically flawed. (0 points)
\end{itemize}

\item \textbf{Coherence}
\begin{itemize}
\item[\underline{\hspace{.2in}}] every paragraph works to support the thesis (3 points)
\item[\underline{\hspace{.2in}}] occasional tangents; repetition (2 points)
\item[\underline{\hspace{.2in}}] lack of coherence; that is, the thesis and the body do not match; goes off on several tangents (1 point)
\item[\underline{\hspace{.2in}}] there is little to no coherence; that is, the paper seems a collection of unrelated ideas/themes (0 points)
\end{itemize}

\item \textbf{Writing Mechanics}
\begin{itemize}
\item[\underline{\hspace{.2in}}]  makes very few or no mechanical errors (3 points)
\item[\underline{\hspace{.2in}}] makes several mechanical errors, but these do not interfere with communication (2 points)
\item[\underline{\hspace{.2in}}] makes mechanical errors that do interfere with communication (1 point)
\item[\underline{\hspace{.2in}}] makes mechanical errors that seriously interfere with communication (0 points)
\end{itemize}

\item \textbf{Paragraph development} 
\begin{itemize}
\item[\underline{\hspace{.2in}}]  paragraphs are consistently well developed, with a clear topic sentence and appropriate number of sentences that provide examples and develop points (3 points)
\item[\underline{\hspace{.2in}}]  some structure and development of paragraphs and/or some with clear topic sentences or focus, but not consistently (2 points)
\item[\underline{\hspace{.2in}}] poor paragraphs with no clear topic sentence; multiple topics; little or no 
development (1 point)
\item[\underline{\hspace{.2in}}] very poor paragraph structure that greatly affects readability (0 points)
\end{itemize}  
 
\item \textbf{Transitions}
\begin{itemize}
\item[\underline{\hspace{.2in}}]  strong and/or consistent transition between points in lab; strong flow  (3 points)
\item[\underline{\hspace{.2in}}]  some transition or flow between paragraphs; partial structure to argument (2 points)
\item[\underline{\hspace{.2in}}] little or no transition between paragraphs; poor flow (1 point) 
\item[\underline{\hspace{.2in}}] no transitions between paragraph or ideas (0 points)
\end{itemize}

\item \textbf{Introduction}
\begin{itemize}
\item[\underline{\hspace{.2in}}]  introduction grasps reader's attention, engages the reader and foreshadows major points (3 points)
\item[\underline{\hspace{.2in}}] some introduction; nothing beyond a list of what is to come (2 points)
\item[\underline{\hspace{.2in}}] poor introduction; does not adequately relate to the body of the report (1 point)
\item[\underline{\hspace{.2in}}] no introduction (0 points)
\end{itemize}
 
\item  \textbf{Conclusion} 
\begin{itemize}
\item[\underline{\hspace{.2in}}] a conclusion going beyond summary of what was written in the body of the report; describes what more could be learned (3 points)
\item[\underline{\hspace{.2in}}] some summary of points made, but nothing beyond summary; no broad conclusions/lessons (2 points)
\item[\underline{\hspace{.2in}}]  poor conclusion or summary of argument (1 point) 
\item[\underline{\hspace{.2in}}]  no conclusion (0 points) 
\end{itemize}
\end{itemize}


\indent Poster Grading Rubric

\begin{itemize}
\item  \textbf{Informative}
\begin{itemize}
\item[\underline{\hspace{.2in}}] the poster well informs readers about the issues and take-away points (3 points) 
\item[\underline{\hspace{.2in}}] the poster somewhat informs readers about the issues and take-away points (2 points) 
\item[\underline{\hspace{.2in}}] the poster does not clearly inform readers about the issues and take-away points (1 point) 
\item[\underline{\hspace{.2in}}] the poster is confusing or has other serious flaws (0 points) 
\end{itemize}

\item  \textbf{Presentation}
\begin{itemize}
\item[\underline{\hspace{.2in}}] the poster is professional, creative, and captures attention (3 points)
\item[\underline{\hspace{.2in}}] the poster is creative, and captures attention but may suffer some issues (2 points)
\item[\underline{\hspace{.2in}}] the poster has several presentation issues (1 point)
\item[\underline{\hspace{.2in}}] the poster has serious presentation issues (0 points)
\end{itemize}

\item  \textbf{Articulation of Main Mathematical Ideas}
\begin{itemize}
\item[\underline{\hspace{.2in}}]  clear articulation of main ideas and arguments (3 points)
\item[\underline{\hspace{.2in}}] some articulation of main ideas (2 points)
\item[\underline{\hspace{.2in}}]  poor articulation of main ideas (1 point)
\item[\underline{\hspace{.2in}}] no articulation of main ideas (0 points)
\end{itemize}

\item \textbf{Writing Mechanics}
\begin{itemize}
\item[\underline{\hspace{.2in}}]  makes very few or no mechanical errors (3 points)
\item[\underline{\hspace{.2in}}] makes several mechanical errors, but these do not interfere with communication (2 points)
\item[\underline{\hspace{.2in}}] makes mechanical errors that do interfere with communication (1 point)
\item[\underline{\hspace{.2in}}] makes mechanical errors that seriously interfere with communication (0 points)
\end{itemize}
\end{itemize}



\end{document}