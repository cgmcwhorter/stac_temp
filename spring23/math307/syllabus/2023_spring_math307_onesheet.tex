\documentclass[11pt,letterpaper]{article}
\usepackage[lmargin=1in,rmargin=1in,bmargin=1in,tmargin=1in]{geometry}
\usepackage{style}

\pagenumbering{gobble}


% -------------------
% Content
% -------------------
\begin{document}

% TItle
\begin{center} 
\bfseries
\color{stacred}
\LARGE Syllabus Quick Facts \par\vspace{0.2\baselineskip}
\Large MATH 307: Quantitative Literacy II for Educators --- Spring 2023
\end{center} 


% Course Information
\mysection{0.27}{Course Information}
\hspace{0.53cm} {\itshape Instructor Email}: \href{mailto:cmcwhort@stac.edu}{cmcwhort@stac.edu} \par
\hspace{0.53cm} {\itshape Course Webpage}: \href{https://coffeeintotheorems.com/courses/2023-2/spring/math-307/}{https://coffeeintotheorems.com/courses/2023-2/spring/math-307/} \par
\hspace{0.53cm} {\itshape Office Hours}: 	\par \vspace{-0.3cm}
	\begin{table}[!ht]
	\centering
	\begin{tabular}{l || l}
	Mon. & 11:30~am -- 12:30~pm \\
	Tues. & 1:00~pm -- 2:00~pm \\
	Wed. & 11:30~am -- 12:30~pm \\
	Thurs. & 1:00~pm -- 2:00~pm \\
	Fri. & 11:30~am -- 1:30~pm
	\end{tabular}
	\end{table} 


% Grading Components
\mysection{0.27}{Grading Components}
Course grades are determined by the following components: \par \vspace{-0.3cm}
	\begin{table}[!ht]
        \begin{tabular}{clr}
        & Reflections & 10\% \\
        & Quizzes & 10\% \\
        & Project & 20\% \\
        & Homework & 30\% \\
        & Exams & 30\% 
        \end{tabular} 
        \end{table}


% Attendance 
\mysection{0.27}{Attendance}
Attend each lecture and show up on time. Anticipated absences should be addressed with the instructor in advance of the absence. Address any absences---anticipated or otherwise---with the instructor. If you miss a lecture, you are responsible for any material covered, any work assigned, any course changes made, etc. during the class. Four or more unexcused absences from lectures could result in receiving a grade penalty per additional absence or an `F' in the course. Furthermore, excessive lateness will also count as an absence. \pspace


\mysection{0.18}{Reflections}
Students will be assigned reflections throughout the course. These may be self-reflections, videos, articles, etc. Each reflection will be an approximately one-half to one-page paper (single spaced, size 11~font, 1~in margins) addressing questions related to the source(s) assigned. Reflections will be submitted electronically via a GoogleDrive folder shared with you. Students should be sure to actively engage with these source(s) and respond deeply and reflectively with the assigned question(s). \textit{Do not simply repeat words, phrases, ideas, etc. in a `word dump' to fill space.} Is it better to have a (slightly) shorter reflection that is meaningful than a long reflection that is hollow. Late submissions may not be accepted. Any extensions, due dates, and grade penalties for late assignments will be determined by the instructor on a student-by-student basis. \pspace 


% Quizzes 
\mysection{0.27}{Quizzes}
There will be a quiz \textit{every} class, typically at the start of class. Because solutions will often then be immediately discussed, no make-up quizzes will be given (except under extraordinary circumstances). \pspace


% Project
\mysection{0.18}{Project\label{project}}
Students will both write a paper and create a poster during this course. The topic for both the paper and the poster must be the same but should not contain the exact same information `verbatim.' The topic of both the paper and the poster bust be a topic related to Probability, Statistics, or Data Science and to fulfill the course gateway requirement must also be related to global learning or social responsibility. The exact topic(s) will be chosen by the student(s) and approved by the instructor. With permission of the instructor, it may be possible to work in \textit{small} groups for the poster. It is strongly suggested that you work to make your poster `Ignite quality' and apply to present during Ignite. The paper portion of the project must be written individually. Details about the projects, e.g. grading guidelines, suggested topics, etc., will be announced after the corresponding course topic lectures---likely during lecture but possibly via email. \pspace


% Homeworks 
\mysection{0.27}{Homeworks}
There will typically be a homework assigned each class/week, due the next class/week. Homework is a large portion of your grade, so your best work should be put into them. Your solutions should be neat, organized, display effort and clear mathematical thinking, and they should be submitted using the homework packets. You may request extensions on homework assignments in a timely fashion (possibly incurring a grade penalty) but do not simply assume that you will be able to receive extra time on an assignment and plan your schedule carefully. You are encouraged to work with others on homeworks; however, be sure to carefully abide by the academic integrity standards excepted by the college and instructor. \pspace


% Exams 
\mysection{0.27}{Exams}
There will be a total of 3 exams that are each worth 10\% of the course grade for a total of 30\%. While the exams are not cumulative, topics from previous exams can appear in an exam if the material is relevant---but it will not be the focus of the exam. You should be present, seated, and prepared for a scheduled exam before the exam begins. If you are late, you should not expect extra exam time. There are no make-up exams except under extraordinary circumstances. \pspace


% Course Schedule 
\mysection{0.27}{Course Schedule}
The following is a \emph{tentative} schedule for the course and is subject to change. 
        \begin{table}[!ht]
        \centering
        \scalebox{0.83}{%
        \begin{tabular}{ll || ll}
        Date & Topic(s) & Date & Topic(s) \\ \hline 
	01/23 & Rates, \%'s, Functions & 03/15 & Spring Break \\
	01/25 & Cost \& Revenue, Surplus \& Demand & 03/20 & Normal Distributions \\
	01/30 & Simple Interest, Discount Notes, Inflation & 03/22 & Binomial Distributions \\
	02/01 & Discrete \& Compound Interest & 03/27 & Binomial Distributions \\
	02/06 & Annuities & 03/29 & Central Limit Theorem \\
	02/08 & Amortizations & 04/03 & Central Limit Theorem \& Review \\
	02/13 & Review & 04/05 & Exam 2 \\
	02/15 & Exam 1 & 04/10 & Introductory Linear Algebra \\
	02/20 & Introduction to Probability & 04/12 & Linear Algebra \& Regressions \\
	02/22 & Applications of Probability & 04/17 & Applications of Linear Algebra \\
	02/27 & Review & 04/19 & Linear Programming \\
	03/01 & Random Variables & 04/24 & Linear Programming \\
	03/06 & Topics in Probability/Statistics & 04/26 & Linear Programming \\
	03/08 & Normal Distributions & 05/01 & Review \\
	03/13 & Spring Break & 05/03 & Exam 3
        \end{tabular}
        }
        \end{table}


\end{document}