\documentclass[11pt,letterpaper]{article}
\usepackage[lmargin=1in,rmargin=1in,bmargin=1in,tmargin=1in]{geometry}
\usepackage{style}

% Course Subject Abbreviation
\newcommand{\coursenumber}{MAT 307}
% Course Title
\newcommand{\coursetitle}{Quantitative Literacy~II for Educators}
% Section
\newcommand{\coursesection}{}
% Term
\newcommand{\semester}{Spring}
% Class Dates
\newcommand{\classdates}{January 23 -- May 5}
% Class Time
\newcommand{\classtimes}{A: MW 1:00~pm -- 2:25~pm; B: TR 9:50~am -- 11:15~am}
% Classroom
\newcommand{\classroom}{A: MAGR G 13; B: MAGR G 17}


% Instructor
\newcommand{\instructor}{Dr. Caleb McWhorter}
% Instructor Office
\newcommand{\office}{Maguire 129}
% Instructor Number
\newcommand{\phone}{845.398.4077}
% Instructor Email
\newcommand{\email}{cmcwhort@stac.edu}
% Instructor Website
\newcommand{\website}{http://coffeeintotheorems.com}
% Instructor Office Hours
\newcommand{\officehours}{See \textit{`Mathematics Help'}}


% -------------------
% Content
% -------------------
\begin{document}

% Title
\mytitle



% Table of Contents
\largeheader{0cm}{Table of Contents}

\begin{minipage}[t]{0.45\textwidth}
{\bfseries\color{stacred} Course Information} \dotfill \pageref{course_info} \par
\hspace{0.3cm} Instructor Information \dotfill \pageref{instr_info} \par
\hspace{0.3cm} Class Information \dotfill \pageref{class_info} \par
\hspace{0.3cm} Course Description \dotfill \pageref{course_desc} \par
\hspace{0.3cm} Course Objectives \dotfill \pageref{course_obj} \par
\hspace{0.3cm} Course Materials \dotfill \pageref{course_mat} \par
{\bfseries\color{stacred} Course Policies} \dotfill \pageref{course_polc} \par
\hspace{0.3cm} Grading Components \dotfill \pageref{grade_comp} \par
\hspace{0.3cm} Grading Scale \dotfill \pageref{grade_scale} \par
\hspace{0.3cm} Course Format \dotfill \pageref{course_form} \par
\hspace{0.3cm} Attendance \& Participation \dotfill \pageref{attend} \par
\hspace{0.3cm} Reflections \dotfill \pageref{reflections} \par
\hspace{0.3cm} Quizzes \dotfill \pageref{quiz} \par
\hspace{0.3cm} Project \dotfill \pageref{project} \par
\hspace{0.3cm} Homework \dotfill \pageref{hw} \par
\hspace{0.3cm} Exams \dotfill \pageref{exams} \par
\hspace{0.3cm} Mathematics Help \dotfill \pageref{help} \par
\hspace{0.3cm} Respect Policy \dotfill \pageref{respect} \par
\hspace{0.3cm} Email Policy \dotfill \pageref{email_policy}
%
\end{minipage}\hfill\begin{minipage}[t]{0.45\textwidth} \par
\hspace{0.3cm} Electronic Device Policy \dotfill \pageref{electronic} \par
\hspace{0.3cm} Mental Health \& Counseling Services \dotfill \pageref{mental_health} \par
\hspace{0.3cm} Faith/Tradition Observances Policy \dotfill \pageref{faith} \par
\hspace{0.3cm} Use of Student Work \dotfill \pageref{std_work} \par
\hspace{0.3cm} Course Materials Policy \dotfill \pageref{copyright} \par
\hspace{0.3cm} Syllabus Policy \dotfill \pageref{syllabus} \par
\hspace{0.3cm} Tips for Success \dotfill \pageref{tips} \par
\hspace{0.3cm} Important Dates \dotfill \pageref{imp_dates} \par
{\bfseries\color{stacred} College Policies} \dotfill \pageref{college_polc} \par
\hspace{0.3cm} Academic Integrity \dotfill \pageref{college_acadint} \par
\hspace{0.3cm} Academic Dishonesty \dotfill \pageref{college_acaddis} \par
\hspace{0.3cm} Electronic Use Policy \dotfill \pageref{college_elecuse} \par
\hspace{0.3cm} Academic Accommodations for Students \par
\hspace{0.6cm} with Disabilities Statement \dotfill \pageref{college_acadacc} \par
\hspace{0.3cm} Sexual Misconduct Policy \dotfill \pageref{college_sexmisconduct} \par
\hspace{0.3cm} COVID-19 Policies/Procedures \dotfill \pageref{college_healthsafety} \par
\hspace{0.3cm} Diversity and Inclusivity Statement \dotfill \pageref{college_inclusive} \par
\hspace{0.3cm} Mental Health \& Wellness \dotfill \pageref{mental_wellness} \par
{\bfseries\color{stacred} Course Schedule} \dotfill \pageref{schd} \par
\hfill {\bfseries\color{stacred} Total Pages:} \pageref*{LastPage}
\end{minipage}
\sectionbreak





% -----
% Course Information
% -----

% Course Information
\largeheader{0.3cm}{Course Information\label{course_info}}



% Instructor Information
\mysection{0.27}{Instructor Information\label{instr_info}}
\textit{Name:} \instructor \par
\textit{Office:} \office \par
\textit{Phone:} \phone \par
\textit{Email:} \href{mailto:\email}{\email} \par
\textit{Office Hours:} \officehours 
\sectionbreak



% Class Information
\mysection{0.27}{Class Information\label{class_info}}
\textit{Dates:} \classdates \par
\textit{Time:} \classtimes \par
\textit{Classroom:} \classroom \par
\textit{Course Webpage:} \href{\website}{\website}
\sectionbreak



% Course Description 
\mysection{0.27}{Course Description\label{course_desc}}
Problem solving techniques, number systems---including past systems, non-decimal systems, and the decimal number system—and arithmetic in such, the basics of geometric reasoning and problem solving, techniques and uses for measurements in geometry, transformations, symmetries, tilings, congruence and similarity of figures, introductory statistics and statistical inference, and introductory probability. This class may not be used in any major to satisfy a ``MATH 300 or above'' requirement. {\itshape Prerequisites: MATH 180 and Childhood Education majors.}
\sectionbreak



% Course Objectives
\mysection{0.27}{Course Objectives\label{course_obj}}
After this course, among other things, students should be able to\dots
	\begin{itemize} \itemsep=0.3ex
	\item Give the basic definitions, along with examples, relating to Set Theory. 
	\item Understand apply basic counting principles, e.g. addition/subtraction principle, permutations, combinations, etc. 
	\item Understand and be able to communicate the connection between counting and probabilities. 
	\item Understand and apply the definitions and rules of basic probability theory. 
	\item Use the rules of probability to solve real-world problems.
	\item Recognize and understand common misconceptions in Probability and Statistics. 
	\item Understand and articulate different types of data, e.g. ordinal, ratio, etc. 
	\item Understand the uses for (and abuses using) and construct different representations of data, e.g. histograms, pie charts, etc. 
	\item Compute `elementary' metrics in Statistics, e.g. 5-number summaries, means, standard deviations, expected values, etc. 
	\item Understand and articulate informations about sampling in Statistics. 
	\item Understand bias in Statistics and articulate possible sources for bias. 
	
	\item Use terminology associated with lines in the plane. 
	\item Compute angle measurements in polygons. 
	\item Identify various types of polygons and the properties of various types of curves. 
	\item Apply the classification of quadrilaterals. 
	\item Identify, sketch, and work with various polyhedron.
	\item State and use Euler's Formula for Polyhedra. 
	\item Understand and use the basic definitions and notions for graphs. 
	\item Compute degrees for vertices in graphs and apply these to various problems.
	\item Understand, compute, and articulate various measurements, e.g. lengths, areas, volumes, etc. 
	\item Give formulas for the perimeters, areas, surface areas, volumes, etc. of various geometric objects. 
	\item State and apply the Pythagorean Theorem to both mathematical and real-world problems. 
	\item Articulate, sketch, and apply various rigid motions in the plane. 
	\item Identify various types of symmetries in geometric objects. 
	\item Articulate the applications of mathematical symmetries to the `real-world.' 
	\item Perform a number of geometric constructions in the plane, e.g. angle bisection, perpendiculars, etc. 
	\item Verify whether given triangles are similar, congruent, or neither. 
	\item Apply similar triangles to various mathematical and `real-world' problems.  
	\end{itemize}
Furthermore, students should\dots
	\begin{itemize} \itemsep=0.3ex
	\item  Improve their ability to engage in mathematical thinking, reasoning, communication, and problem solving.
	\item Develop a matured perspective on how to approach mathematical problems and concepts.
	\item Be able to state ways Mathematics applies to real world problems.
	\item Refine their cognitive skills by improving their ability to learn independently, approach problems imaginatively, solve problems methodically, and communicate solutions intelligibly.
	\item Reflect on the issues arising during teaching and be able to propose remedies for these issues. 
	\item Develop a more sophisticated approach to communication and problem solving in STEM. 
	\end{itemize}
\sectionbreak



% Course Materials
\mysection{0.22}{Course Materials\label{course_mat}}
{\itshape\bfseries\color{stacred}Textbook.} The textbook for this course is \textit{Mathematical Reasoning for Elementary Teachers} by Calvin T. Long, Duane W. De Temple, and Richard S. Millman. While there are many different editions of this textbook, \textit{any} edition of this textbook should suffice for the course. The current edition is the 7th edition. If you are concerned that your edition will not suffice, contact the instructor. The other primary reference for the course will be lecture notes and related materials provided by the instructor. Be sure to regularly check the course webpage and GoogleDrive for the course. \pspace

{\itshape\bfseries\color{stacred}Calculators.} Basic graphing calculators will be allowed during the course, unless otherwise instructed. However, these will not be required. The course will make use of the computational engine Mathematica via the WolframAlpha website: \url{https://www.wolframalpha.com}. Although WolframAlpha does have a paid account option for additional resources, the course will not make use of these features and students {\itshape will not} be required to setup an account or make any kind of payment. \pspace

{\itshape\bfseries\color{stacred}Spreadsheet Software.} 
The course may also make use spreadsheet software such as Excel and GoogleSheets. However, Excel homeworks should be submitted as a \texttt{.xlsx} file. Students should have free access to these resources via their STAC accounts as well as on any campus computer. When using GoogleSheets, students should use their STAC provided Google account. Other online resources may be used in the course and will be provided by the instructor. 
\sectionbreak





% -----
% Course Policies
% -----

% Course Policies
\largeheader{0.3cm}{Course Policies\label{course_polc}}



% Grading Components
\mysection{0.27}{Grading Components\label{grade_comp}}
Course grades are determined by the following components: \par
	\begin{table}[!ht]
        \begin{tabular}{clr}
        & Reflections & 10\% \\
        & Quizzes & 10\% \\
        & Project & 20\% \\
        & Homework & 30\% \\
        & Exams & 30\% 
        \end{tabular} 
        \end{table}
\sectionbreak



% Grading Scale
\mysection{0.18}{Grading Scale\label{grade_scale}}
The grade scale is the standard St. Thomas Aquinas College grading scale and is as follows: \par
        \begin{table}[!ht]
        \centering
        \begin{tabular}{|l||c|l||c|} \hline
        A & 95 -- 100 & C+ & 77 -- 79 \\ \hline
        A-- & 90 -- 94 & C & 73 -- 76 \\ \hline
        B+ & 87 -- 89 & C-- & 70 -- 72 \\ \hline
        B & 83 -- 86 & D & 65 -- 69 \\ \hline
        B-- & 80 -- 82 & F & 0 -- 64 \\ \hline
        \end{tabular}
        \end{table}
\sectionbreak



% Course Format
\mysection{0.19}{Course Format\label{course_form}}
The course consists of two lectures per week. Each class will begin with a short quiz. This will typically be followed by a lecture period. However, there will be many classes where the quiz is then followed by either one of two things: a special activity or a discussion. These special activities will serve to motivate course material and help deepen mathematical understanding. Discussions will be focused on previously assigned reflections (see the reflection section of the syllabus). In either case, \textit{every} student is expected to participate. The remainder of lecture will be focused on new course material. While some new instruction will occur during lecture, this time will typically be focused on learning/reviewing the course material through problem solving, especially group problem solving. Due to the number and `depth' of course topics, not every concept or problem type can be covered during class. Students are expected to read chapters assigned to them, especially focusing on definitions, theorems, examples, and problem-solving methods. Furthermore, students are expected to spend outside of class studying extra materials and solving additional problems. Students are expected to typically spend approximately 3~hours per credit outside of class on course materials. However, some weeks this may be more or less. \sectionbreak



% Attendance and Participation
\mysection{0.36}{Attendance and Participation\label{attend}}
{\itshape\bfseries\color{stacred}Attendance.} It is essential to your success in this course that you attend each lecture and participate in class discussions. It is also a federal requirement that students who do not attend or stop attending a class be reported at the time of determination by the faculty that the student never attended or stopped attending the class. Therefore, you are expected to attend each lecture and to show up on time. Address any absence(s), anticipated or unanticipated, with the instructor as soon as possible. Should you anticipate an absence, you are to contact the instructor as soon as possible---at least twenty-four hours before the class, if possible.  Certain absences from lecture(s) may be excused, depending on the reason for the absence. Determinations are made on a case-by-case basis at the discretion of the instructor. The student should discuss the issue with the instructor as soon as possible; however, to excuse an absence, the reason(s) for missing lecture(s) must be documentable and presented, if requested. \pspace

If you miss a lecture, you are responsible for any material covered, any work assigned, any course changes made, etc. during the class. Do not assume or expect the instructor to provide you with anything, particularly lecture notes, from the class(es) missed. {\itshape Four or more unexcused absences from lectures could result in receiving a grade penalty per additional absence or an `F' in the course.} Furthermore, excessive lateness will also count as an absence. If you are dismissed from lecture due to problems during the lecture, e.g. disruptive behavior or unauthorized cell phone use, then this dismissal will be recorded as an absence for the lecture. If you cannot attend a class due to a mandated quarantine, inform your instructor immediately so that arrangements can be made. In this case, the student may be required to participate in lectures virtually and submit assignments online. \pspace

{\itshape\bfseries\color{stacred}Participation.} 
Students are expected to participate in the course---both inside and outside the classroom. Inside the classroom, this means attending class, paying attention, taking notes, asking and answering questions when appropriate, etc. However, course participation does not begin and end at the classroom door. Students are expected to review course material and complete course assignments. Typically, students can expect to spend approximately 3~hours per credit outside of class working for the course---although some weeks this could be more or less. \pspace

Students are highly encouraged to form study groups to help support themselves and their fellow students' learning. These groups can be used to review notes or additional resources, work on class activities, discuss homework problems, etc. However, these groups {\itshape should not} be used to simply solve problems for others or have others solve your problems for you. For instance, students may not `assign' homework problems to each other to solve in order to complete assignments. Using study groups in this or similar manners is an academic integrity violation that will be dealt with harshly. If you are unsure if what plan on doing or are doing in study groups is appropriate, discuss this with your instructor. \sectionbreak



% Reflections
\mysection{0.18}{Reflections\label{reflections}}
While a primary focus of the course is to deepen students' understanding of mathematical definitions, theorems, problem solving skills, thinking and reasoning, etc., i.e. quantitative literacy, this is not the only focus of the course. Students should also leave the course with a deeper understanding of how to communicate and teach mathematics as well as related, broader educational issues. Therefore, students will be assigned reflections throughout the course. These reflections will come in a variety of sources: self-reflections, videos, articles, etc. Each reflection will be an approximately one-half to one-page paper (single spaced, size 11~font, 1~in margins) addressing questions related to the source(s) assigned. Reflections may be directly related to course material. However, reflections will typically be `adjacent' to course material, i.e. related to educational issues more broadly. Reflections will be submitted electronically via a GoogleDrive folder shared with you. Be sure to start reflections early to give yourself time to engage with the material. Reflections will be primarily be graded on effort and `thoughtfulness,' Students should be sure to actively engage with these source(s) and respond deeply and reflectively with the assigned question(s). \textit{Do not simply repeat words, phrases, ideas, etc. in a `word dump' to fill space.} Is it better to have a (slightly) shorter reflection that is meaningful than a long reflection that is hollow. Be sure to begin these early---do not put them off until the last minute---and submit them on-time. Late submissions may not be accepted. Any extensions, due dates, and grade penalties for late assignments will be determined by the instructor on a student-by-student basis. \sectionbreak 



% Quizzes
\mysection{0.18}{Quizzes\label{quiz}}
There will be a quiz \textit{every} class. Quizzes are meant to be short and simple. These quizzes serve more as a method of gauging whether you are keeping up with the material. It is important that if you are late that you obtain a copy of the quiz immediately. Quiz solutions will often be discussed following the quiz. Because quiz solutions will often be discussed in class, no make-up quizzes will be given except under extraordinary circumstances determined on a case-by-case basis at the discretion of the instructor. Unless otherwise instructed, there are no calculators, computational devices, notes, or outside assistance of any kind allowed on quizzes. \pspace



% Project
\mysection{0.18}{Project\label{project}}
As part of the gateway requirement for this course, students are required to show what they have learned in the course as well as demonstrate the skills they have developed thus far at STAC through a course project/paper. For this course, students will fulfill this requirement through a paper and the creation of a poster corresponding to this paper. The topic for both the paper and the poster must be the same; however, they need not (nor should they) contain the exact same information `verbatim.' The topic of both the paper and the poster bust be a topic related to Probability, Statistics, or Data Science. Furthermore, as required by the gateway requirement, the paper and poster must also be related to or have some impact on global learning or social responsibility. The exact topic(s) will be chosen by the student(s) and approved by the instructor. \pspace

While students must write their paper individually, they may work together in the creation of the poster. Students wishing to work in a small group (no more than three) for the poster may do so with permission of the instructor. Each student should contribute equally to the poster and the final product (whether working alone or in a group) should be of the quality of posters presented during Ignite. While you cannot be \textit{required} to present your poster at Ignite, it is strongly suggested that you work to make your poster `Ignite quality' and apply to present during Ignite. While students' papers will reflect their posters content, it should be far more expansive; that is, in a sense, the poster is a summary of the poster. Students working in small groups for the poster should be especially careful that their papers are independent from each other. The papers must be written individually and may not have `significant' portions of the paper `similar'; that is, a casual reader should not be able to identify similarities between group members' papers nor tell that they had at any point communicated with each other. If you have questions or concerns about the distinctness of your paper, a certain type/level of collaboration, or whether a course of action is appropriate when working with other students, consult the instructor \textit{before} proceeding. Details about the projects, e.g. grading guidelines, suggested topics, etc., will be announced after the corresponding course topic lectures---likely during lecture but possibly via email. \sectionbreak



% Homework
\mysection{0.18}{Homework\label{hw}}
The only way to learn Mathematics is to do Mathematics! Therefore, there will be weekly homework assignments. It is essential for students to complete all of the assignments for the course. Working on homework is the best way of engaging with course concepts and gauging one's mastery of the material. Moreover, homework is an essential portion of the course grade. Assignments should be started as soon as possible. Do not delay working through homework; it is easier to keep up than it is to catch up. Students may request extensions on homework assignments. Requests for extensions should be submitted to the instructor in a timely fashion---do not delay! However, do not simply assume that you will be able to receive extra time on an assignment and plan your schedule carefully. Except in exceptional circumstances, homework extensions on topics included in an exam will not be granted beyond that exam date. Any extensions, due dates, and grade penalties for late assignments will be determined by the instructor on a student-by-student basis. \pspace

You are encouraged to work with others on homeworks. Mathematics is a social activity! The purpose of working together on assignments is to engage with course topics, see different perspectives, ask questions, and have others look over your work. However, do not simply use others to do your assignments. You should also not allow other students to use you to complete their assignments. Of course, using online solutions is a violation of the St. Thomas Aquinas College academic integrity policies. If you are unsure of whether a particular resource is appropriate to use on an assignment, consult with your instructor first. \pspace

Homeworks may entail software or programming components. Access to any datasets required for these portions will be provided by the instructor. These portions may require a fair amount of independence on the part of the student. However, there are resources available to help you with these problems. Should you have difficulty with these problems, ask your instructor for help! Be aware that many of your fellow students may be more technologically literate and ask them for help as well! Anticipate that there may be technological issues and always start these problems early! Do not wait until the problem(s) are due to try to complete or submit them. You are responsible for submitting solutions and any files for computer-based problems on-time and in the proper format. Always check the file(s) after submission. Failure to adhere to these guidelines may result in grade deductions or rejection of submissions. There is no guarantee that any late solution(s) or file(s) will be accepted. However, if you experience technical difficulties, document the issues thoroughly. 
\sectionbreak



% Exams
\mysection{0.18}{Exams\label{exams}}
There will be three exams in this course, each worth 10\% of the total course grade for a total of 30\% of the course grade. The schedule of the exams can be found in the `Course Schedule' section of the syllabus. However, these exam dates are subject to change. Students should not make plans to leave campus before May~5th or otherwise have conflicts on/before that date. Each of the exams covers approximately the third of the course material proceeding the exam date. However, any course topics may appear on any exam. Students should be present, seated, and prepared for a scheduled exam before the exam begins. Students who are late should not expect extra exam time. Furthermore, students who miss an exam should not expect to receive a make-up exam. There will be no make-up exams except under extraordinary circumstances, e.g. in the case of an emergency. However, determinations for make-up exams or other substitutions, with possible grade deductions, are made at the discretion of the instructor on a case-by-case basis. Unless otherwise instructed, no devices or materials other than those provided by the instructor are allowed on any exam. Exams may involve out-of-class portions, which will be submitted at a time and manner specified in lecture. Furthermore, it may be possible that any exam will be a take-home exam. In this case, the exam procedure and schedule will be announced in advance during lecture. 
\sectionbreak



% Mathematics Help
\mysection{0.24}{Mathematics Help\label{help}}
Be proactive about your success in the course! If you need help, there are many resources available to help you. Your first primary contact for help is the instructor. If you are struggling, attend office hours or send an email. The instructors office hours for this semester can be found below: \par
	\begin{table}[!ht]
	\centering
	\begin{tabular}{l || l}
	Mon. & 11:30~am -- 12:30~pm \\
	Tues. & 1:00~pm -- 2:00~pm \\
	Wed. & 11:30~am -- 12:30~pm \\
	Thurs. & 1:00~pm -- 2:00~pm \\
	Fri. & 11:30~am -- 1:30~pm
	\end{tabular}
	\end{table}
Do not wait to bring issues, course related or otherwise, to the attention of the instructor. If you cannot attend office hours, send an email to the instructor to try to make other arrangements. There are also a number of resources available to you at St. Thomas Aquinas College: Center for Student Success, Academic Recovery Program, Writing Center, etc. Students looking for extra mathematics help should consult with the Academic Services Office in Spellman~106, via email at \href{mailto:AcademicServices@stac.edu}{academicservices@stac.edu}, or on the web at \href{https://www.stac.edu/academics/academic-services}{https://www.stac.edu/academics/academic-services}. The Center for Student Success website is \url{https://www.stac.edu/academics/academic-services/center-student-success} and can be found at Spellman~111 or contacted at 845.398.4090. \sectionbreak



% Respect Policy
\mysection{0.19}{Respect Policy\label{respect}}
Learning requires a healthy academic environment. A key component to this is respecting everyone's time---especially giving everyone time to fail, ask questions, and learn. Therefore, everyone should abide by the following respect policies: \pspace

The instructor will respect student's time:
	\begin{itemize}
	\item They will come prepared to help you understand the course material and prepare students for quizzes/exams. 
	\item They will listen to student feedback on how to best help them succeed. 
	\item They will return assignments, respond to emails, and give feedback in a timely fashion. 
	\item They will be patient during the student learning process and will treat all students fairly. 
	\end{itemize} \pspace

Students will respect the instructor's time:
	\begin{itemize}
	\item They will be on time to class. Moreover, they will come prepared and pay attention during class. 
	\item They will ask for help and communicate with the instructor in a timely fashion. 
	\item They will keep track of assignments---completing them on time and to the best of their ability.  
	\item They will read and follow course policies. 
	\end{itemize} \pspace

Students will respect each other's time:
	\begin{itemize}
	\item They will not be disruptive in class. If you need to call or text someone, take it outside of the classroom. 
	\item They will work with each other to find solutions and understand course material. However, they will not simply solve problems. 
	\item They will allow each other to make mistakes, ask questions, and participate in the learning process. 
	\item They will use respectful language when speaking to or about one another. 
	\end{itemize}
\sectionbreak



%% COVID Discussion Policy
%\mysection{0.31}{COVID Discussion Policy\label{covid_disc}}
%At the time of writing, there have been over 94.1~million cases of COVID-19 in the United States with over 1,043,803 deaths; moreover, there have been over 600~million cases with 6.4~million deaths worldwide. It is an understatement to say that these are trying times not just for students, including their friends and family, but for our broader community. While many of us use humor to cope with difficult situations, we are often able to do so without great offense because we can choose our words and our actions to fit an audience with which we are familiar---be it friends or family. \pspace
%
%However, this luxury may not be available to us in the classroom. You will likely not know all your classmates and their circumstances. It is not unlikely that at least one of your classmates in at least one of your classes has lost an acquaintance, friend, or family member to COVID-19. Worse yet, because of social distancing, they may not have been able to properly mourn them. Even if a classmate has not lost someone, they or someone in their life may be experiencing financial hardships or other crises due to COVID-19. \pspace
%
%All students are expected to respect and protect each other by abiding by the college's vaccination and mask policies. But protecting the health of others is the minimum that one can do during a pandemic. We need to go beyond basic physical health and support our community's mental health. By enrolling in this course, you agree to refrain from making jokes or other trivialization of the COVID-19 pandemic while participating in the course, both online and in-person.
%\sectionbreak



% Email Policy
\mysection{0.18}{Email Policy\label{email_policy}}
All email communication in this course should be done using your @stac.edu email account. Similarly, any digital course access and file submissions should be made using your @stac.edu email account. Abiding by federal guidelines, emails coming from a non-STAC email may not receive a response. Emails should be properly written: contain appropriate subject line, possess an opening and closing address, be understandable and contain appropriate language, be grammatically correct, have appropriate font style and size, etc. Emails which do not follow these guidelines may not receive a response.
\sectionbreak



% Electronic Device Policy
\mysection{0.31}{Electronic Device Policy\label{electronic}}
Students are expected to complete the course without the use of calculators or other computational devices on assignments, quizzes, exams, etc., unless otherwise instructed. Any unauthorized use of such devices are considered a violation of the academic integrity policies. During the course, \href{http://www.wolframalpha.com/}{http://www.wolframalpha.com/}, \href{https://www.symbolab.com/}{https://www.symbolab.com/}, and Mathematica will be used to demonstrate concepts give students an opportunity to be able to check work. However, these should only be used as instructed, and never during a quiz or exam. All electronic devices should be turned off and put away during class unless otherwise instructed or given specific permission. Use of such devices can result in dismissal from class.
\sectionbreak



% Mental Health & Counseling Services
\mysection{0.49}{Mental Health and Counseling Services\label{mental_health}}
If at any point during the semester, you feel overwhelmed with your class work, feel thoughts of depression/suicide, experience sexual assault/rape, experience problems with substance abuse or relationship abuse, or have any other struggles with physical/mental health, \underline{\bfseries\itshape please seek help}! The Counseling \& Psychological Services (CAPS) at St. Thomas Aquinas College is a resource offering assistance with any issue you might have. There is \underline{\bfseries\itshape never} any shame in seeking help. If you or someone you know is struggling with any of these issues, {\itshape speak out}! The CAPS website can be found at \href{https://www.stac.edu/student-life/counseling-psychological-services}{https://www.stac.edu/student-life/counseling-psychological-services}. CAPS is located in the upper level of the Romano Student Alumni Center and can be contacted at 845.398.4065. If you or someone you know is having issues with gender or sexual identity issues, CAPS is also there to create a safe space for those with marginalized genders and sexualities or those who might be struggling with these issues. Know that my office is a safe space and should you prefer any gender specific pronoun/name, please be sure to make me aware! Students may also make use of the College Health \& Wellness Services located in the McNelis Commons Residence Life Complex, Apartment~2B which can also be contacted at \href{mailto:stachealth@stac.edu}{stachealth@stac.edu} or 845.398.4242, as well as the Campus Ministry and Volunteer Services, directed by Nick Migliorino, located in the Romano Student Alumni Center and can be contacted at \href{mailto:nmiglior@stac.edu}{nmiglior@stac.edu} or 845.398.4084.
\sectionbreak



% Faith/Tradition Observances Policy
\mysection{0.44}{Faith/Tradition Observances Policy\label{faith}}
The instructor recognizes the diversity of faiths and traditions represented in the campus community. Students should have the right to observe religious holy days according to their faith and traditions. Accordingly, students may notify their instructor, no later than the end of the second week of classes, of any classes that they will be missing due to religious or traditional observances. Students following this guideline will be excused from these classes. Under this policy, students should have an opportunity to make up any examination, study, or work missed due to these observances or have an equitable and appropriate substitution made. All policy and procedural decisions are made at the discretion of the instructor on a student-by-student basis. 
\sectionbreak



% Use of Student Work
\mysection{0.27}{Use of Student Work\label{std_work}}
In compliance with the federal Family Educational Rights and Privacy Act (FERPA), registration in this class is understood as permission for assignments prepared for this class to be used anonymously in the future for educational purposes.
\sectionbreak



% Course Materials Policy
\mysection{0.30}{Course Materials Policy\label{copyright}}
All course materials (defined to include, but not limited to, course handouts, video/audio lectures, assignments, quizzes, exams, etc.) are the intellectual property of the instructor or St. Thomas Aquinas College, unless the copyright is already explicitly held by some other individual, group, or other entity. Therefore, course materials are protected by United States copyright law, see Title~17~USC. Students in this course are permitted to download some course materials for personal use. \pspace

However, students are not permitted to (in print, digitally, or otherwise) share, distribute, sell, or publish course materials, either in part or in whole, without the instructors explicit written and signed permission along with a personal usage code. Unauthorized reproduction or distribution of course materials is a violation of intellectual property law, and is a violation of the student code of conduct. The instructor, or agent acting on behalf of the instructor with written and signed permission, also reserves the right to delete or disable any link to any course materials. In enrolling in the course, the student agrees to abide by this course materials policy in perpetuity.
\sectionbreak



% Syllabus Policy
\mysection{0.20}{Syllabus Policy\label{syllabus}}
The instructor reserves the right to revise, including substantially revise, the course syllabus at any time---with or without notification. By enrolling in this course, students agree to all the policies found in the syllabus. Wherever applicable, students also agree to follow syllabus policies in perpetuity, e.g. students may not provide unauthorized assistance, materials, etc. to students enrolled in future versions of this course. 
\sectionbreak



% Tips for Success
\mysection{0.21}{Tips for Success\label{tips}}
\begin{itemize} \itemsep=0.3ex
\item Be proactive about your success in the course.
\item Do not procrastinate! Begin your assignments and studying early!
\item Attend every lecture.
\item Address issues immediately. Ask questions during class, recitation, office hours, etc. 
\item Form a study group! Working together will help you and others better understand the course material as you can work through different difficulties and offer each other clarifications on concepts.
\item Do problems! Reading through your notes is not enough. Seek out new problems and work through them carefully. When you are done, check your answer. If you are wrong, examine carefully what misunderstanding occurred and how to avoid it in the future. If you were correct, examine if there was a faster way, check to see if your solution `flowed' and was easy to read, and think over what concepts/computations were used and what `type' of problem was the exercise.
\end{itemize}
\sectionbreak



% Important Dates
\mysection{0.22}{Important Dates\label{imp_dates}}
\begin{itemize} \itemsep=0.3ex
\item 01/27: Academic Add/Drop Deadline
\item 03/10: Mid-semester
\item 03/13 -- 03/17: Spring Break
\item 04/05: Academic Withdrawal Deadline
\item 04/07: Good Friday (No Classes)
\item 05/05: Last day of classes/exams
\end{itemize}
\sectionbreak





% -----
% College Policies
% -----

% College Policies
\largeheader{0.5cm}{College Policies\label{college_polc}}

% Academic Integrity
\mysection{0.25}{Academic Integrity\label{college_acadint}}

Academic integrity is a commitment to honesty, trust, fairness, respect, and responsibility within an academic community. An academic community of integrity advances the quest for truth and knowledge by requiring intellectual and personal honesty in learning, teaching, research, and service. Honesty begins with oneself and extends to others. Such a community also fosters a climate of mutual trust, encourages the free exchange of ideas, and enables all to reach their highest potential. \pspace

A college community of integrity upholds personal accountability and shared responsibility, and ensures fairness in all academic interactions of students, faculty, and administrators. While we recognize the participatory and collaborative nature of the learning process, faculty and students alike must show respect for the work of others by adhering to the clear standards, practices, and procedures contained in the policy described below. \pspace

Academic integrity is essential to St. Thomas Aquinas College's mission to educate in an atmosphere of mutual understanding, concern, cooperation, and respect. All members of the College community are expected to possess and embrace academic integrity. \sectionbreak



% Academic Dishonesty
\mysection{0.28}{Academic Dishonesty\label{college_acaddis}}

Academic dishonesty is defined as any behavior that violates the principles outlined above. St. Thomas Aquinas College strictly prohibits academic dishonesty. Any violation of academic integrity policies that constitutes academic dishonesty will be subject to harsh penalties, ranging up to and including dismissal from the College. \pspace

For all Academic Integrity violations, faculty must file a Student Conduct Academic Dishonesty Report, which will be shared with the Dean of the appropriate School, the Provost, and the student. The student will also have to file a Student Academic Integrity Violation Report. Please view the full policy and the associated forms at \url{https://www.stac.edu/academics/academic-integrity-policy}. \sectionbreak



% Electronic Use Policy
\mysection{0.27}{Electronic Use Policy\label{college_elecuse}}

Faculty members at St. Thomas Aquinas College have the discretion to regulate the use of electronic devices in their classes, and students should not use such devices without the expressed permission of the professor. This policy covers cell phones, tablets, laptop computers, or any other device the use of which might constitute a distraction to the professor or to the other students in the class, as determined by the professor. Students with documented disabilities should discuss the use of laptops and/or other electronic devices with their professor at the beginning of the semester. \pspace

When a professor designates a time during which electronic devices may be used, they are only to be used at the discretion of the faculty member and in accordance with the mission of the college. Professors may develop specific and reasonable penalties to deal with violations of these general policies. For more extreme cases of classroom disruption, refer to the College's Disruptive Student Policy. \pspace

Please note that a browser lockdown system may be implemented in order to prevent cheating during assessments such as exams and quizzes. Faculty are expected to confirm that these systems will work with students' laptops before requiring their use. \pspace





\newpage





{\itshape Recording of Lectures:} Class meetings that include course content or identifiable student information are protected by the Family Education Rights and Privacy Act (FERPA), found at \url{https://www2.ed.gov/policy/gen/guid/fpco/ferpa/index.html}. At times throughout the semester, the faculty member may record their lecture. It is a best practice for faculty to notify participants that their session is going to be recorded. This recording \textit{\textbf{CANNOT}} be shared with anyone who is not enrolled in this specific course section. \pspace

Students cannot personally record class sessions and then share them outside of the course, although they can maintain them for personal use. \sectionbreak



% Academic Accommodations for Students with Disabilities Statement
\mysection{0.83}{Academic Accommodations for Students with Disabilities Statement\label{college_acadacc}}

St. Thomas Aquinas College values diverse types of learners and is committed to ensuring that each student is afforded equal access to participate in all learning experiences. If you have a learning difference or a disability---including a mental health, medical, or physical impairment---that would hinder your access to learning in this class, please contact Disability Services. They will confidentially explain the accommodation request process and the type of documentation that may be needed to determine your eligibility for reasonable accommodations. To learn more about academic accommodations for students with disabilities, please contact Anne Schlinck, Director of Disability Services, at \href{mailto:aschlinc@stac.edu}{aschlinc@stac.edu} or call/text 845.398.4087. Disability Services is located in Room~L102 in the lower level of Spellman Hall. \pspace

If you have already been granted academic accommodations at St. Thomas Aquinas College, you have the right to receive the academic accommodations that are listed on your Letter of Accommodation. Please understand that it is your responsibility as a student registered with Disability Services to provide your Letter of Accommodation to your instructor if you wish to use your accommodations in this course. If you will need to use your testing accommodations, please be sure to review the Disability Services Testing Accommodation Policies---Academic Year 2022--2023 found at \href{https://docs.google.com/document/d/1V5iUtgypiS8kClqhSLPde7AOSZPoLu6CsIDcpiEic2w/edit?usp=sharing}{Disability Services Testing Accommodation Policies}. \sectionbreak



% Gender- or Sex-Based Misconduct Policy
\mysection{0.50}{Sexual Misconduct Policy\label{college_sexmisconduct}}

Students should be aware that faculty members are responsible employees and are required to report certain information to the STAC’s Title IX Coordinator. If you inform your instructor about, or that person witnesses, gender- or sex-based misconduct, which includes sexual harassment, sexual assault, intimate partner or domestic violence, stalking, or any gender- or sex-based discrimination, the faculty member will keep the information as private as possible, but must bring it to the attention of STAC’s Title~IX Coordinator. 
Students should also be aware that disclosing such experiences in course assignments does NOT put the College on notice and will NOT begin the process of STAC providing assistance or response to those experiences. \pspace

Please remember that instances of gender- and sex-based misconduct that occur in virtual/online environments are covered by STAC’s Title~IX, Student Code of Conduct, and Faculty/Employee Conduct policies. \pspace





\newpage





The College encourages individuals who experience, witness or become aware of alleged sexual misconduct to report the incident to the Title~IX Coordinator. The College will assist individuals in contacting law enforcement, if desired. The College also provides individuals the opportunity to discuss alleged incidents with a trained professional on campus with the assurance that the discussion will be confidential. \pspace

The following reporting processes are:
	\begin{itemize}
	\item {\bfseries Non-Confidential Reporting Resources}: If you would like to talk to the Title~IX Coordinator directly, you can contact Mr. Norman Huling (\href{mailto:nhuling@stac.edu}{nhuling@stac.edu}, 845.398.4068). Additionally, you also may report incidents or complaints to Title IX Deputy Coordinators, Ms. Nicole Ryan (\href{mailto:nryan@stac.edu}{nryan@stac.edu}, 845.398.4163) or Dr. Benjamin Wagner (\href{mailto:bwagner@stac.edu}{bwagner@stac.edu}, 845.398.4212), or you can contact the Office of Campus Safety and Security (845.398.4080). You can find more information at \url{www.stac.edu/titleix}.
	
	\item {\bfseries Confidential Reporting Resources}: If you would like to report  to a confidential counseling resource who is not required to initiate a Title~IX report, you may contact the following people on a confidential basis: 

	\hfill\begin{minipage}[t]{0.49\textwidth}
	{\bfseries Ms. Anne Walsh RN, BSN} \par
	Director, Health and \par 
	Wellness Services \par
	845.398.4242 \par
	\href{mailto:awalsh@stac.edu}{awalsh@stac.edu}
	\end{minipage}\begin{minipage}[t]{0.49\textwidth}
	{\bfseries Dr. Lou Muggeo} \par
	Director, Counseling \& \par
	Psychological Services \par
	845.398.4174 \par
	\href{mailto:lmuggeo@stac.edu}{lmuggeo@stac.edu}
        \end{minipage} \pspace
        \hfill\begin{minipage}[t]{0.49\textwidth} 
 	{\bfseries Dr. Alexa Gaydos} \par
	Licensed Clinical Psychologist, \par
	Counseling \& Psychological Services \par
	845.398.4065 \par
	\href{mailto:agaydos@stac.edu}{agaydos@stac.edu}
	\end{minipage}\begin{minipage}[t]{0.49\textwidth} 
	{\bfseries Elysse Sellers, LCSW} \par
	Licensed Clinical Social Worker, \par
	Counseling \& Psychological Services \par
	845.398.4065 \par
	\href{mailto:esellers@stac.edu}{esellers@stac.edu}
        \end{minipage} 
        
        \begin{center}
        \begin{minipage}[t]{0.50\textwidth} 
	{\bfseries Center for Safety and Change} \par
	\url{http://centerforsafetyandchange.org/} \par
	9 Johnsons Lane, New City, NY 10956 \par
	845.634.3344 \par \vspace{0.3cm} 
	\hspace{1cm} {\itshape Academic Semester} \par
	\hspace{0.5cm} {\itshape On-Campus Office Hours} \par\vspace{0.1cm}
	\hspace{0.5cm} Thursdays, 1~pm -- 5~pm \par
	\hspace{0.5cm} Romano Center, \par
	\hspace{0.5cm} Alumni Center Room~21
        \end{minipage} 
        \end{center}
\end{itemize} \sectionbreak



% Classroom Health and Safety Protocols
\mysection{0.49}{COVID-19 Related Policies and Procedures\label{college_healthsafety}}

% Classroom Health and Safety Protocols
\noindent {\bfseries Classroom Health and Safety Protocols}

The health and safety of students, faculty, and staff on our campus is our top priority. In response to the ongoing COVID-19 pandemic, the STAC community will continue to work together to support compliance with recommended health and safety standards to optimize the learning experience while minimizing health risks.   

	\begin{enumerate}[1.]
	\item {\bfseries Follow quarantine and isolation guidelines.} If you feel ill, have recently tested positive for COVID-19, or have come into contact with someone who has tested positive for COVID-19, {\itshape do not} come to campus or leave your residence hall until you have been cleared to do so by STAC Health Services (\url{stachealth@stac.edu}). It is important that you always contact STAC Health Services in any of these circumstances and follow the quarantine and isolation instructions given. Please also let your professor know if you cannot attend class. 

	\item {\bfseries Mask policy.} Mask-wearing is optional on STAC's campus except for in the STAC Health and Wellness Center, where masks must be worn; however, you must wear a mask if you are directed to do so under our isolation/quarantine policy. Please note that individual professors may encourage students to wear masks in their classrooms, but masks cannot be required. In the event of an increase in COVID cases on campus, the College may decide to return to a mask requirement. 

	\item {\bfseries Minimize shared equipment.} Individuals should avoid sharing equipment where possible. However, if equipment does need to be shared, please wipe it down with provided disinfecting wipes in between users and maintain physical distancing as much as possible.

	\item {\bfseries Disinfect your classroom space.} Students and faculty are encouraged to disinfect areas within their workspaces by cleaning these at the beginning and end of each class. This includes desk tops, seats, and equipment used during class. Disinfectant supplies will be provided. 

	\item {\bfseries Practice good hand hygiene.} Individuals should wash their hands with soap and water for at least 20~seconds as often as possible or use personal hand sanitizers. Hand sanitizer stations are available throughout the campus.

	\item {\bfseries Respect each other.} Show concern for each other's health and safety, and remember that this is a stressful time for everyone.
	\end{enumerate} \sectionbreak



% COVID-19, Other Illness, and Absences
\noindent {\bfseries COVID-19, Other Illness, and Absences} \par

As stated above, for the health and safety of the campus community, students who are ill {\itshape should not} attend classes. Students in the following situations must contact Health Services as soon as possible: 

	\begin{itemize}
	\item Those who have tested positive for COVID-19 or are exhibiting COVID-19 related symptoms. 
	\item Those who have been instructed to quarantine because of close contact with someone who has tested positive for COVID-19.
	\end{itemize}

If a student cannot attend classes for any of the above reasons, they should:

	\begin{enumerate}[1.]
	\item Communicate this change with their instructor(s) via email. Contact instructors as soon as possible, preferably within 24~hours.
	\item Keep up with coursework and participate in class activities as much as possible. Students are responsible for completing any work that they might miss due to illness, including assignments, quizzes, tests, and exams.
	\item Reach out to the instructor if illness will require late submission or modifications of assignments; work with the instructor to reschedule exams and other critical academic activities before they are due.
	\end{enumerate}
\sectionbreak



% Diversity and Inclusivity Statement
\mysection{0.44}{Diversity and Inclusivity Statement\label{college_inclusive}}

St. Thomas Aquinas College is committed to creating an inclusive environment. Our community actively seeks the inclusion and full participation of individuals from groups that have historically experienced discrimination and prejudice. We are committed to a climate of mutual respect and inclusion, one in which diversity is a source of pride rather than a source of division. We encourage all persons---students, faculty, and staff alike---to reflect on their own experiences to explore the ways in which others' experiences can and do differ; the goal is to use this reflection to learn about different values, cultures, and ways of thinking. Ultimately, a just and equitable society will be easier to realize if we do not exclude those who are different from us and instead practice empathy and inclusivity. \pspace

To that end, if you experience or are aware of bias, mistreatment, or discrimination based on a person's (or your own) membership in a historically under-privileged or marginalized group, please contact one of the following individuals to share your concerns: \pspace

	\hfill\begin{minipage}[t]{0.33\textwidth}
	{\bfseries Samantha Bazile} \par
	Director of Admissions \& \par
	Chief Diversity Officer \par
	845.398.4104 \par
	\href{mailto:sbazile@stac.edu}{sbazile@stac.edu}
	 \end{minipage}\begin{minipage}[t]{0.33\textwidth}
	{\bfseries Cindy Garvey} \par
	Associate Director \par
	of Financial Aid \par
	845.398.4098 \par
	\href{mailto:cgarvey@stac.edu}{cgarvey@stac.edu}
	\end{minipage} 
	\hfill\begin{minipage}[t]{0.33\textwidth}
	{\bfseries Ryan Gasser} \par
	Associate Director, \par
	Student Development \par
	845.398.4108 \par
	\href{mailto:rgasser@stac.edu}{rgasser@stac.edu}
	\end{minipage}\hfill \pspace

Faculty reserve the right to provide open and honest readings and discussions in their classes about personal and institutional biases and prejudices and other topics that may cause discomfort to some. \pspace

More detailed information about the College’s expectations and policies related to these matters can be found in the Student Handbook, specifically in the Student Code of Conduct, Section D: Harassment and Abuse, the Anti-Harassment Policy, and Rules and Regulations for Maintenance of Order. \sectionbreak




% Mental Health & Wellness
\mysection{0.44}{Mental Health \& Wellness\label{mental_wellness}}

As a student, you may experience a range of issues that can cause barriers to learning, such as strained relationships, increased anxiety, alcohol/drug problems, feeling down, difficulty concentrating and/or lack of motivation. These mental health concerns or stressful events may lead to diminished academic performance or reduce a student's ability to participate in daily activities. \pspace

St. Thomas Aquinas College offers services to assist you with addressing these and other concerns you may be experiencing. If you or someone you know are suffering from any of the aforementioned conditions, you can learn more about the broad range of confidential mental health services available on campus via the Office of Student Development’s Counseling and Psychological Services (CAPS) by visiting \url{https://www.stac.edu/student-life/counseling-psychological-services} or calling 845.398.4065. CAPS is located on the 2nd Floor of the Romano Student Alumni Center (RSAC). \pspace

If you or someone you care about requires immediate assistance during the hours when CAPS is closed, you may reach out to Campus Safety at 845.398.4080. You can also reach an on-call mental health professional by dialing 988 on your phone or visiting the 24-hour emergency help service website at \url{https://988lifeline.org/}. \sectionbreak



% -----
% Course Schedule
% -----
\largeheader{0.3cm}{Course Schedule\label{schd}}
The following is a \emph{tentative} schedule for the course and is subject to change. 
        \begin{table}[!ht]
        \centering
        \scalebox{1}{%
        \begin{tabular}{ll || ll}
        Date & Topic(s) & Date & Topic(s) \\ \hline 
	01/23 & Introduction & 03/15 & Spring Break \\
	01/25 & Set Theory, \S2.1 & 03/20 & Measurements, \S10.1, 10.2 \\
	01/30 & Combinatorics, \S14.2 -- 14.3 & 03/22 & Measurements, \S10.2, 10.3 \\
	02/01 & Combinatorics, \S14.3 & 03/27 & Measurements, \S10.3, 10.4 \\
	02/06 & Probability, \S14.1, 14.4 & 03/29 & Measurements, \S10.4, 10.5 \\
	02/08 & Probability, \S14.1, 14.4 & 04/03 & Review \\
	02/13 & Statistics, \S13.1, 13.2 & 04/05 & Exam 2 \\
	02/15 & Statistics, \S13.2, 13.3 & 04/10 & Transformations \& Symmetry, \S11.1 \\
	02/20 & Review & 04/12 & Transformations \& Symmetry,  \S11.2 \\
	02/22 & Exam 1 & 04/17 & Transformations \& Symmetry, \S11.3 \\
	02/27 & Geometry, \S9.1, 9.2 & 04/19 & Transformations \& Symmetry,  \S12.1 \\
	03/01 & Geometry, \S9.3 & 04/24 & Transformations \& Symmetry, \S12.2 \\
	03/06 & Graphs, \S9.4 & 04/26 & Transformations \& Symmetry, \S12.3 \\
	03/08 & Review & 05/01 & Review \\
	03/13 & Spring Break & 05/03 & Exam 3
        \end{tabular}
        }
        \end{table}


\end{document}