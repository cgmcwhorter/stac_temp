\documentclass[11pt,letterpaper]{article}
\usepackage[lmargin=1in,rmargin=1in,tmargin=1in,bmargin=1in]{geometry}
\usepackage{../style/homework}
\usepackage{../style/commands}
\setbool{quotetype}{false} % True: Side; False: Under
\setbool{hideans}{true} % Student: True; Instructor: False

% -------------------
% Content
% -------------------
\begin{document}

\reflection{\!\!1: \!Due \!05/03 \!(04)}{A great discovery solves a great problem, but there is a grain of discovery in the solution of any problem. Your problem may be modest, but if it challenges your curiosity and brings into play your inventive faculties, and if you solve it by your own means, you may experience the tension and enjoy the triumph of discovery.}{George P\'olya}

\begin{center} {\Large \bfseries P\'olya's Problem Solving Techniques} \end{center}

The textbook discusses P\'olya's problem solving techniques not only as a great approach to mathematical problem solving but also good teaching pedagogy. Recall P\'olya's principles were:
	\begin{enumerate}[I.]
	\item Understand the problem: Do you understand the words used in the problem? What are you asked to find or show? Can you restate the problem in your own words? Can you think of a picture or diagram that might help you understand the problem? Is there enough information to solve the problem?
	\item Devise a plan: Have you tried guessing and checking, making an orderly list, eliminating possibilities, using symmetry, trying special cases, using direct reasoning, solving an equation, looking for patterns, drawing a picture, solving a simpler problem, using a model, working backwards, using a formula, being `creative', etc.?
	\item Carry out the plan: Implement your plan. If it works, wonderful! If the plan does not work, discard it and choose another plan.
	\item Look back: What did you try? What worked and what did not? Are there other ways of solving the problem? Is there a `better' way? Is there a shorter way? What were the underlying principles? What have you learned that might solve `similar' problems in the future?
	\end{enumerate}

Watch 3Blue1Brown's video \href{https://www.youtube.com/watch?v=M64HUIJFTZM&ab_channel=3Blue1Brown}{`The unexpectedly hard windmill question (2011 IMO, Q2)'}. How did Grant Sanderson use (intentionally or unintentionally) P\'olya's problem solving techniques in presenting a solution to the problem? What worked well and what worked poorly in the video? How might you bring similar techniques to your own teaching---in Mathematics or other subjects?

\newpage

\phantom{.}


























\end{document}