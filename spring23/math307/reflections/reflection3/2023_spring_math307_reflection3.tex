\documentclass[11pt,letterpaper]{article}
\usepackage[lmargin=1in,rmargin=1in,tmargin=1in,bmargin=1in]{geometry}
\usepackage{../style/homework}
\usepackage{../style/commands}
\setbool{quotetype}{false} % True: Side; False: Under
\setbool{hideans}{true} % Student: True; Instructor: False

% -------------------
% Content
% -------------------
\begin{document}

\reflection{\!3: \!Due \!05/03 \!(04)}{Do not train children in learning by force and harshness, but direct them to it by what amuses their minds, so that you may be better able to discover with accuracy the peculiar bent of the genius of each.}{Plato}

You have probably had the experience of having trouble understanding the material in a class or lecture. But when you worked one-on-one with the instructor or another student, the material suddenly `clicked'. Learning almost always works best when it is tailored to the individual. Research validates these experiences. Read Benjamin S. Bloom's \textit{The 2 Sigma Problem: The Search for Methods of Group Instruction as Effective as One-to-One Tutoring}. Respond to the following problems:
	\begin{itemize}
	\item What is the `2 Sigma Problem'? What are barriers to achieving this in the classroom?
	\item What are other techniques you have seen or proposals you have to addressing student learning and `The 2 Sigma Problem'? 
	\item What does the paper suggest is the least effective method for learning? Knowing this, address the ethics of continuing to teach using these paradigms. 
	\item Give a topic in your own subject that you might teach and outline a lecture designed to address the issues discussed in `The 2 Sigma Problem.' 
	\end{itemize}

\newpage

\phantom{.}


\end{document}