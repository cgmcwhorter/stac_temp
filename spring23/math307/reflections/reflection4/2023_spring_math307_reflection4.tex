\documentclass[11pt,letterpaper]{article}
\usepackage[lmargin=1in,rmargin=1in,tmargin=1in,bmargin=1in]{geometry}
\usepackage{../style/homework}
\usepackage{../style/commands}
\setbool{quotetype}{true} % True: Side; False: Under
\setbool{hideans}{true} % Student: True; Instructor: False

% -------------------
% Content
% -------------------
\begin{document}

\reflection{\!4: Due 05/03 (04)}{The mediocre teacher tells. The good teacher explains. The superior teacher demonstrates. The great teacher inspires.}{William A. Ward}

A common opinion is that Mathematics is an `unemotional' subject and that mathematicians are a `special type' of person; that is, it takes special talents to be a `great' mathematician. Arguably, one of the most well-regarded modern mathematicians is Steven Strogatz. Strogatz graduated from Princeton in 1980 and then studied at Trinity College Cambridge. He then went on to receive a Ph.D in applied mathematics from Harvard University in 1986. Strogatz has served as a professor at MIT and is currently a distinguished faculty member at Cornell University. He has won numerous awards, written several books (including  NYT best sellers), and is regarded as an excellent educator in mathematics. That is, Steven Strogatz is an excellent mathematician. Listen to Grant Sanderson's interview with Steven Strogatz titled, \href{https://www.youtube.com/watch?v=SUMLKweFAYk&ab_channel=GrantSanderson}{`Steven Strogatz: In and out of love with math | 3b1b podcast \#3'} and respond to the following prompts:
	\begin{itemize}
	\item What surprising aspects of Strogatz's mathematical journey and his mathematical knowledge? Why do you think he ultimately succeeded?
	\item How can teaching the `beauty' of mathematics be inspiring or instructive? How can this be isolating or elitist? 
	\item What are the pros/cons for teaching `historically' versus teaching something in a `logical' order were discussed? How might this play a role in your own instruction?
	\item What roles did Strogatz's teachers play a role in his intellectual and professional development?
	\item What were suggestions for the most important aspects of good mathematical exposition? How might you incorporate these principles into teaching your own subject?
	\end{itemize}

\newpage

\phantom{.}


\end{document}