\documentclass[11pt,letterpaper]{article}
\usepackage[lmargin=1in,rmargin=1in,tmargin=1in,bmargin=1in]{geometry}
\usepackage{../style/homework}
\usepackage{../style/commands}
\setbool{quotetype}{false} % True: Side; False: Under
\setbool{hideans}{true} % Student: True; Instructor: False

% -------------------
% Content
% -------------------
\begin{document}

\reflection{\!2: \!Due \!05/03 \!(04)}{If I am walking with two other men, each of them will serve as my teacher. I will pick out the good points of the one and imitate them, and the bad points of the other and correct them in myself.}{Confucius}

\begin{center} {\Large \bfseries Teaching Pedagogy in STEM} \end{center}

Teachers in STEM (Science, Technology, Engineering, and Mathematics), regardless of the level, meet unique challenges compared to teaching in other subjects. It is difficult to help students reach mastery of the concepts, computations, and problem solving skills involved in STEM simultaneously while actively engaging them in the learning process. Watch either \href{https://www.youtube.com/watch?v=wy-LqFDwMuM&list=PLB1304385546D6F86&index=2&ab_channel=MITOpenCourseWare}{Lec 1 | MIT 5.95J Teaching College-Level Science and Engineering, Spring 2009} or \href{https://www.youtube.com/watch?v=gyboshu425k&list=PLB1304385546D6F86&index=4&ab_channel=MITOpenCourseWare}{Lec 2 | MIT 5.95J Teaching College-Level Science and Engineering, Spring 2009}. Respond to the following prompts for the lecture that you chose:
	\begin{itemize}
	\item Lecture~I: Explain the use of introducing `tension' in STEM teaching. How does this enhance learning and engagement?
	\item Lecture~I: What were the suggestions for addressing when a student `ruins' an element of the lecture? How might you handle this in your own classroom?
	\item Lecture~I: What were the suggestions and reasonings for `good' STEM teaching? How might you integrate these into teaching in your own discipline? 
	\item Lecture~I: What were the main themes of the lecture? How can you bring these themes into teaching your own subject?
	\item Lecture II: What are the pros/cons of teaching using Option~A and Option~B?
	\item Lecture II: Explain the use of introducing `tension' in STEM teaching. How does this enhance learning and engagement?
	\item Lecture II: What is `chunking'? How might these issues arise in your own teaching and how might you address them?
	\item Lecture II: What were the overall themes and take-aways for the lecture? How might you incorporate these themes/suggestions into your own teaching?
	\end{itemize}

\newpage

\phantom{.}


\end{document}