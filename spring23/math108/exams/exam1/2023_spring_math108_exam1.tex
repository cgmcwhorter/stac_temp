\documentclass[12pt,letterpaper]{exam}
\usepackage[lmargin=1in,rmargin=1in,tmargin=1in,bmargin=1in]{geometry}
\usepackage{../style/exams}

% -------------------
% Course & Exam Information
% -------------------
\newcommand{\course}{MAT 108: Exam 1}
\renewcommand{\term}{Spring -- 2023}
\newcommand{\examdate}{02/22/2023}
\newcommand{\timelimit}{85 Minutes}

\setbool{hideans}{true} % Student: True; Instructor: False

% -------------------
% Content
% -------------------
\begin{document}

\examtitle
\instructions{Write your name on the appropriate line on the exam cover sheet. This exam contains \numpages\ pages (including this cover page) and \numquestions\ questions. Check that you have every page of the exam. Answer the questions in the spaces provided on the question sheets. Be sure to answer every part of each question and show all your work. If you run out of room for an answer, continue on the back of the page --- being sure to indicate the problem number.} 
\scores
\bottomline
\newpage

% ---------
% Questions
% ---------
\begin{questions}

% Question 1
\newpage
\question[10] Suppose that the CPI for urban US cities in 2013 was 230.280 while it was 233.916 in 2014. 
	\begin{enumerate}[(a)]
	\item What was the inflation rate from 2013 to 2014?
	\item Using (a), what would you predict a good that cost \$100 in 2013 would cost in 2014?
	\item If in the inflation rate from (a) stayed constant, what would you predict a good that cost \$100 in 2013 would cost in 2023?
	\end{enumerate}



% Question 2
\newpage
\question[10] Pepe Roni owns an Asian fusion restaurant called 9021 Pho. One of the items on the menu has a daily revenue function given by $R(q)= 15.99q$ and a daily cost function given by $C(q)= 5.47q + 287.50$. 
	\begin{enumerate}[(a)]
	\item How much does each item cost to make?
	\item What are the fixed costs for producing this item?
	\item How much is this item sold for?
	\item What is the minimum number of sales of this item that Pepe needs to make in order to turn a profit on its sale?
	\end{enumerate}



% Question 3
\newpage
\question[10] Walter Malone is taking out a short term loan to help finance a used car. He makes an arrangement with a bank to take out a 7~month simple discount note for \$4,000 at 9.4\% annual interest. 
	\begin{enumerate}[(a)]
	\item What are the maturity and proceeds for this note?
	\item How much does Walter pay in interest on this loan?
	\item How much does Walter owe at the end of the 7~months?
	\end{enumerate}



% Question 4
\newpage
\question[10] Abe Hines has taken out a loan to open up his dream business---a vape shop called `Darth Vaper.' The bank Abe negotiated with has offered him a loan of \$760,000 at 7.84\% annual interest, compounded monthly. Abe would have to pay the loan off over the next 15~years by making regular, end of the month payments.
	\begin{enumerate}[(a)]
	\item How much would Abe's monthly payment be?
	\item How much would Abe pay in total on this loan?
	\item How much interest would Abe pay in total on this loan?
	\end{enumerate}



% Question 5
\newpage
\question[10] Lee Reese has invested money in a local company called `Pane in the Glass.' He has given the company \$12,000. The company promises a return on his investment of 3.7\% annual interest, compounded quarterly. 
	\begin{enumerate}[(a)]
	\item How much is Lee's investment worth after 3~years?
	\item If Lee had wanted the investment to be worth \$20,000 after 5~years, how much should he have initially invested? 
	\end{enumerate}



 % Question 6
\newpage
\question[10] Donna Mite owns a hair salon called `Hairway to Heaven.' On average, each customer that visits the store spends approximately \$50.97. However, the average cost of serving that customer is approximately \$28.21. The salon has rent and utility costs of approximately \$518.35 to keep open per day. 
	\begin{enumerate}[(a)]
	\item Find the revenue function as a function of the number of daily customers. 
	\item Find the cost function as a function of the daily number of customers. 
	\item What is the minimal number of customers Donna needs to attract to the store each day to turn a profit? 
	\end{enumerate}



% Question 7
\newpage
\question[10] Herb Atake is saving for his children's college education. Upon the birth of his daughter 18~years ago, he placed \$85 at the start of each month into a savings account that earned 2.31\% annual interest, compounded monthly. 
	\begin{enumerate}[(a)]
	\item How much did Herb deposit in total?
	\item How much is in the account?
	\item How much interest did Herb earn?
	\end{enumerate}



% Question 8
\newpage
\question[10] Kiki Cate has retired from her job at a small computer company called `Tech It Easy.' She has saved \$1.3~million for retirement. She has cashed out the retirement and placed it into an account which earns 3.15\% annual interest, compounded quarterly. She will live off this money by making regular, equal, end of the month withdrawals. She wants the money to last at least 20~years. What is the most Kiki can withdraw each month?



% Question 9
\newpage
\question[10] Petra Fried has taken out a \$27,000 loan with a bank to go in on a business called `Amy's Winehouse' with her friend Amy. The bank charges Petra 8.91\% annual interest, compounded continuously. 
	\begin{enumerate}[(a)]
	\item How much will Petra owe the bank after 2~years?
	\item How long until Petra owes the bank \$50,000? 
	\end{enumerate}



% Question 10
\newpage
\question[10] Rhoda Chamel has taken out a 30-year, \$335,000 mortgage to purchase her dream home. The mortgage rate Rhoda secured with the bank was 7.316\% annual interest, compounded monthly. Rhoda's end of the month payments will be \$2,300.31. 
	\begin{enumerate}[(a)]
	\item How much will Rhoda still owe on the house after 15~years of payments?
	\item After 15~years of making payments, how much less will Rhoda owe after her next payment, i.e. how much of her next payment goes towards the principal? 
	\end{enumerate}


\end{questions}
\end{document}