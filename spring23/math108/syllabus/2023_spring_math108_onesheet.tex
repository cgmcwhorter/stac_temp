\documentclass[11pt,letterpaper]{article}
\usepackage[lmargin=1in,rmargin=1in,bmargin=1in,tmargin=1in]{geometry}
\usepackage{style}

\pagenumbering{gobble}


% -------------------
% Content
% -------------------
\begin{document}

% TItle
\begin{center} 
\bfseries
\color{stacred}
\LARGE Syllabus Quick Facts \par\vspace{0.2\baselineskip}
\Large MATH 108: Quantitative Methods in \par Business and Social Studies --- Spring 2023
\end{center} \pspace


% Course Information
\mysection{0.27}{Course Information}
\hspace{0.53cm} {\itshape Instructor Email}: \href{mailto:cmcwhort@stac.edu}{cmcwhort@stac.edu} \par
\hspace{0.53cm} {\itshape Course Webpage}: \href{https://coffeeintotheorems.com/courses/2023-2/spring/math-108/}{https://coffeeintotheorems.com/courses/2023-2/spring/math-108/} \par
\hspace{0.53cm} {\itshape Office Hours}: 	\par \vspace{-0.3cm}
	\begin{table}[!ht]
	\centering
	\begin{tabular}{l || l}
	Mon. & 11:30~am -- 12:30~pm \\
	Tues. & 1:00~pm -- 2:00~pm \\
	Wed. & 11:30~am -- 12:30~pm \\
	Thurs. & 1:00~pm -- 2:00~pm \\
	Fri. & 11:30~am -- 1:30~pm
	\end{tabular}
	\end{table}


% Grading Components
\mysection{0.27}{Grading Components\label{grade_comp}}
Course grades are determined by the following components: \par \vspace{-0.3cm}
	\begin{table}[!ht]
        \begin{tabular}{clr}
	& Quizzes & 15\% \\
	& Homework & 40\% \\
	& Exams & 45\%
        \end{tabular} 
        \end{table}


% Attendance 
\mysection{0.27}{Attendance}
Attend each lecture and show up on time. Anticipated absences should be addressed with the instructor in advance of the absence. Address any absences---anticipated or otherwise---with the instructor. If you miss a lecture, you are responsible for any material covered, any work assigned, any course changes made, etc. during the class. Four or more unexcused absences from lectures could result in receiving a grade penalty per additional absence or an `F' in the course. Furthermore, excessive lateness will also count as an absence. \pspace


% Quizzes 
\mysection{0.27}{Quizzes}
There will be a quiz \textit{every} class, typically at the start of class. Because solutions will often then be immediately discussed, no make-up quizzes will be given (except under extraordinary circumstances). \pspace


% Homeworks 
\mysection{0.27}{Homeworks}
There will typically be a homework assigned each class, due the next class. Homework is a large portion of your grade, so your best work should be put into them. Your solutions should be neat, organized, display effort and clear mathematical thinking, and they should be submitted using the homework packets. You will also have homework sets which involve using Excel. If you have difficulties, do not hesitate to ask for help! These homeworks will likely be submitted virtually. Whenever possible, in-cell solutions should be coded and not entered or copy/pasted, i.e. the cell should compute the solution. Assignments should be started as soon as possible; it is easier to keep up than it is to catch up. You may request extensions on homework assignments (possibly incurring a grade penalty). Requests for extensions should be submitted to the instructor in a timely fashion---do not delay! However, do not simply assume that you will be able to receive extra time on an assignment and plan your schedule carefully. You are encouraged to work with others on homeworks; however, be sure to carefully abide by the academic integrity standards excepted by the college and instructor. \pspace


% Exams 
\mysection{0.27}{Exams}
There will be a total of 3 exams that are each worth 15\% of the course grade for a total of 45\%. The tentative schedule for these exams can be found below. Each exam covers course material, up until the exam preceding it. While the exams are not cumulative, topics from previous exams can appear in an exam if the material is relevant---but it will not be the focus of the exam. You should be present, seated, and prepared for a scheduled exam before the exam begins. If you are late, you should not expect extra exam time. There are no make-up exams except under extraordinary circumstances. \pspace


% Course Schedule 
\mysection{0.27}{Course Schedule}
The following is a \emph{tentative} schedule for the course and is subject to change. 
        \begin{table}[!ht]
        \centering
        \scalebox{1}{%
        \begin{tabular}{ll || ll}
        Date & Topic(s) & Date & Topic(s) \\ \hline 
	01/23 & Rates, \%'s, Functions & 03/15 & Spring Break \\
	01/25 & Cost \& Revenue, Surplus \& Demand & 03/20 & Normal Distributions \\
	01/30 & Simple Interest, Discount Notes, Inflation & 03/22 & Binomial Distributions \\
	02/01 & Discrete \& Compound Interest & 03/27 & Binomial Distributions \\
	02/06 & Annuities & 03/29 & Central Limit Theorem \\
	02/08 & Amortizations & 04/03 & Central Limit Theorem \& Review \\
	02/13 & Review & 04/05 & Exam 2 \\
	02/15 & Exam 1 & 04/10 & Introductory Linear Algebra \\
	02/20 & Introduction to Probability & 04/12 & Linear Algebra \& Regressions \\
	02/22 & Applications of Probability & 04/17 & Applications of Linear Algebra \\
	02/27 & Review & 04/19 & Linear Programming \\
	03/01 & Random Variables & 04/24 & Linear Programming \\
	03/06 & Topics in Probability/Statistics & 04/26 & Linear Programming \\
	03/08 & Normal Distributions & 05/01 & Review \\
	03/13 & Spring Break & 05/03 & Exam 3
        \end{tabular}
        }
        \end{table}


\end{document}