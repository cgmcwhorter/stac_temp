\documentclass[11pt,letterpaper]{article}
\usepackage[lmargin=1in,rmargin=1in,bmargin=1in,tmargin=1in]{geometry}
\usepackage{style/quiz}
\usepackage{style/commands}

% -------------------
% Content
% -------------------
\begin{document}
\thispagestyle{title}

% Quiz 1
\quizsol \textit{True/False}: Ethan is saving for a new car. The car he is saving for costs \$9,800. Over the next two months, the cost of the car rises by 8\% each month. Therefore, the car now costs 16\% more and thus costs $\$9800(0.16)= \$1568$ more than it did when he started saving. \pspace

\sol The statement is \textit{false}. Repeated percent increase/decrease are not additive; that is, applying a $P$\% increase/decrease to a number $X$ a total of $n$ times is not the same thing as finding a $nP$\% increase/decrease. In this case, raising the price 8\% twice is not the same thing as raising the price by 16\%. The price increase after the first month is $\$9800(0.08)= \$784$. The new price is then $\$9800 + \$784= \$10584$. The price increase the second month is then $\$10584(0.08)= \$846.72$. The final price is then $\$10584 + \$846.72= \$11430.70$. [Note: One could compute this immediately via $\$9800(1.08)^2= \$9800(1.1664)= \$11430.70$.] But then the car costs $\$11430.70 - \$9800= \$1630.72$ more than when he bought it. Alternatively, we know that the original price was $\$9800$. After increasing the price twice by 8\%, the final price will be $\$9800(1.08)^2= \$9800(1.1664)$. We can recognize this as a 16.64\% increase from the original price. But this is $\$9800(0.1664)= \$1630.72$ increase in price. \pspace


% Quiz 2
\quizsol \textit{True/False}: Ellen has been hired as a financial analyst at a company. The previous analyst modeled the cost of producing one of their products as $C(q)= 43.27q + 13296$. Therefore, Ellen can deduce that the model predicted that the production cost of each item was $\$43.27$ and the fixed costs were $\$13296$. \pspace

\sol The statement is \textit{true}. We know that the cost of production per unit (at a production level of $q$~items) is the rate of change of $C(q)$. Because $C(q)$ is linear, this is the slope of $C(q)$. The slope of $C(q)$ is $43.27$. Therefore, the item costs $\$43.27$ per item to produce, as claimed. Finally, we know that the fixed costs are the costs regardless the level of production. But then the fixed costs will be the costs associated with producing 0~units. This is $C(0)= 43.27(0) + 13296= 13296$. Therefore, the fixed costs are \$13,296, as claimed. 



% If the CPI last year was \$248.110 and this year it is \$251.608, then the inflation rate from last year to this year was 1.4\%. Moreover, if the inflation rate stays constant, in 5~years, goods will cost approximately 7.2\% more.


% The expression $8000 \left(1 + \dfrac{0.051}{4} \right)^{12}$ could represent the amount of money in a savings account earning 5.1\% annual interest, compounded quarterly after 12~years if the savings account initially had \$8,000 deposited into it. 



\end{document}