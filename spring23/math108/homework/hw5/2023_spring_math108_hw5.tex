\documentclass[11pt,letterpaper]{article}
\usepackage[lmargin=1in,rmargin=1in,tmargin=1in,bmargin=1in]{geometry}
\usepackage{../style/homework}
\usepackage{../style/commands}
\setbool{quotetype}{true} % True: Side; False: Under
\setbool{hideans}{true} % Student: True; Instructor: False

% -------------------
% Content
% -------------------
\begin{document}

\homework{5: Due 02/13}{A penny saved is a penny earned.}{Benjamin Franklin}

% Problem 1
\problem{10} Claire is hoping to save for some basement renovations. Speaking with a contractor, they estimate that the desired renovations will cost approximately \$12,000. Claire sets aside money in a savings account that earns 1.2\% annual interest, compounded monthly. She plans on depositing money at the end of each month to save for the build. Claire hopes to have enough saved in 3~years. How much should Claire's monthly payments be? How much will she have earned in interest? 



\newpage



% Problem 2
\problem{10} Phil wants to expand his new reality business. To help finance his expansions, he negotiates a loan for \$450,000 at 8.2\% annual interest, compounded quarterly. He will make equal, end of the month payments to repay this loan. How much will these monthly payments be? How much will he pay in total on this loan? 



\newpage



% Problem 3
\problem{10} Luke is saving for a go-cart. He puts all the money he earns from his paper route---which is \$647---into a saving account at the start of each month. The account earns 2.6\% annual interest, compounded monthly. How much will he have saved after 2~years? How much did Luke deposit in total? 



\newpage



% Problem 4
\problem{10} Alex is taking out a loan to help pay for the textbooks for her college courses. She takes out a loan for \$4,000 at 5.3\% annual interest, compounded semiannually. She is required to make a payment at the start of each month. How much are Alex's monthly payments? How much will she pay in total for this loan?


\end{document}