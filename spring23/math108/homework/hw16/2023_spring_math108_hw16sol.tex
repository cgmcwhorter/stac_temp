\documentclass[11pt,letterpaper]{article}
\usepackage[lmargin=1in,rmargin=1in,tmargin=1in,bmargin=1in]{geometry}
\usepackage{../style/homework}
\usepackage{../style/commands}
\setbool{quotetype}{false} % True: Side; False: Under
\setbool{hideans}{false} % Student: True; Instructor: False

% -------------------
% Content
% -------------------
\begin{document}

\homework{16: Due 05/01}{Everyday life is like programming, I guess. If you love something, you can put beauty into it.}{Donald Knuth}

% Problem 1
\problem{10} Write down the initial simplex tableau for the following optimization problem:
	\[
	\begin{gathered}
	\max z= 3.1x_1 - 4.7x_2 + 5.9x_3 \\
	1.1x_1 - 5.7x_2 + 4.0x_3 \leq 10.4 \\
	6.7x_1 - 0.8x_2 - 8.8x_3 \geq -8.8 \\
	-9.1x_1 + 7.3x_2 - 9.1x_3 \leq 11.7 \\
	x_1, x_2, x_3 \geq 0
	\end{gathered}
	\] \pspace

\sol First, observe that this optimization is not in standard form because the third inequality is not `$\leq$' and the constant term is negative. Multiplying both sides of this third inequality by $-1$, we have\dots 
	\[
	\begin{gathered}
	\max z= 3.1x_1 - 4.7x_2 + 5.9x_3 \\
	1.1x_1 - 5.7x_2 + 4.0x_3 \leq 10.4 \\
	-6.7x_1 + 0.8x_2 + 8.8x_3 \leq 8.8 \\
	-9.1x_1 + 7.3x_2 - 9.1x_3 \leq 11.7 \\
	x_1, x_2, x_3 \geq 0
	\end{gathered}
	\] 
Now introducing slack variables into each inequality (except the last non-negativity inequality) to obtain equalities, we have\dots \par
	\begin{table}[!ht]
	\centering
	\begin{tabular}{rrrrrrrrrrrrr}
	$1.1x_1$ & $-$ & $5.7x_2$ & $+$ & $4.0x_3$ & $+$ & $s_1$ & & & & & $=$ & $10.4$ \\
	$-6.7x_1$ & $+$ & $0.8x_2$ & $+$ & $8.8x_3$ & $+$ & & & $s_2$ & & & $=$ & $8.8$ \\
	$-9.1x_1$ & $+$ & $7.3x_2$ & $-$ & $9.1x_3$ & $+$ & & & & & $s_3$ & $=$ & $11.7$ \\
	\end{tabular}
	\end{table} \par
Moving things to the `$z$'-side of the equality in the function, we have $z - 4.6x_1 - 3.1x_2 - 7.9x_3= 0$. Adding this to the table yields\dots \par
	\begin{table}[!ht]
	\centering
	\begin{tabular}{rrrrrrrrrrrrrrr}
	&& $1.1x_1$ & $-$ & $5.7x_2$ & $+$ & $4.0x_3$ & $+$ & $s_1$ & & & & & $=$ & $10.4$ \\
	&& $-6.7x_1$ & $+$ & $0.8x_2$ & $+$ & $8.8x_3$ & $+$ & & & $s_2$ & & & $=$ & $8.8$ \\
	&& $-9.1x_1$ & $+$ & $7.3x_2$ & $-$ & $9.1x_3$ & $+$ & & & & & $s_3$ & $=$ & $11.7$ \\
	$z$ & $-$ & $3.1x_1$ & $+$ & $4.7x_2$ & $-$ & $5.9x_3$ & & & & & & & $=$ & $0$ \\
	\end{tabular}
	\end{table} \par
This yields the following initial simplex tableau: \par
	\begin{table}[!ht]
	\centering
	\begin{tabular}{rrrrrr|r}
	$1.1$ & $-5.7$ & $4.0$ & $1$ & $0$ & $0$ & $10.4$ \\ 
	$-6.7$ & $0.8$ & $8.8$ & $0$ & $1$ & $0$ & $8.8$ \\ 
	$-9.1$ & $7.3$ & $-9.1$ & $0$ & $0$ & $1$ & $11.7$ \\ \hline
	$-3.1$ & $4.7$ & $-5.9$ & $0$ & $0$ & $0$ & $0$ \\ 
	\end{tabular}
	\end{table}



\newpage



% Problem 2
\problem{10} Suppose that the final simplex tableau associated to a maximization problem was the following: \par
	\begin{table}[!ht]
	\centering
	\begin{tabular}{rrrrrrrrr}
	$1$ & $1.77$ & $0$ & $0$ & $0.74$ & $0$ & $0.26$ & $0.29$ & $208.57$ \\
	$0$ & $0.57$ & $0$ & $1$ & $0.14$ & $0$ & $-0.14$ & $0.29$ & $28.57$ \\
	$0$ & $2.51$ & $0$ & $0$ & $0.83$ & $1$ & $1.17$ & $-0.14$ & $605.71$ \\
	$0$ & $0.09$ & $1$ & $0$ & $-0.03$ & $0$ & $0.03$ & $0.14$ & $34.29$ \\
	$0$ & $3$ & $0$& $0$& $2$ & $0$ & $0$ & $2$ & $600$
	\end{tabular}
	\end{table} \par

\begin{enumerate}[(a)]
\item How many inequalities were considered?
\item How many variables were there in the original inequalities?
\item How many slack/surplus variables were introduced?
\item What was the solution to this maximization problem?
\end{enumerate} \pspace

\sol 
\begin{enumerate}[(a)]
\item Each row of the tableau `corresponds' to an inequality with the exception of the last row which `corresponds to the function.' But then there were $5 - 1= 4$ inequalities in the original system (ignoring the non-negativity inequality). \pspace

\item Each column of the tableau `corresponds' to a variable in the system with the exception of the last column which `corresponds to the solutions.' Therefore, there were $9 - 1= 8$ variables in the system. Note by (c), there are 4 slack/surplus variables. Therefore, there were $8 - 4= 4$ `original' variables in the system of inequalities. \pspace

\item Because we introduce a slack/surplus variable for each inequality and by (a) there were 4 inequalities in the original system, there were 4 slack/surplus variables. \pspace

\item By (b) and (c), there were 5 `original' variables and 4 slack/surplus variables. Therefore, we need find the maximum value along with the values of the variables----namely, the values for $(x_1, x_2, x_3, x_4, x_5, s_1, s_2, s_3, s_4)$. Adding `dividers' to the tableau and `naming' the columns, we have\dots \par
	\begin{table}[!ht]
	\centering
	\begin{tabular}{rrrrrrrr|r}
	{\small $x_1$} & {\small $x_2$} & {\small $x_3$} & {\small $x_4$} & {\small $s_1$} & {\small $s_2$} & {\small $s_3$} & {\small $s_4$} & \\
	$\boxed{1}$ & $1.77$ & $0$ & $0$ & $0.74$ & $0$ & $0.26$ & $0.29$ & $208.57$ \\
	$0$ & $0.57$ & $0$ & $\boxed{1}$ & $0.14$ & $0$ & $-0.14$ & $0.29$ & $28.57$ \\
	$0$ & $2.51$ & $0$ & $0$ & $0.83$ & $\boxed{1}$ & $1.17$ & $-0.14$ & $605.71$ \\
	$0$ & $0.09$ & $\boxed{1}$ & $0$ & $-0.03$ & $0$ & $0.03$ & $0.14$ & $34.29$ \\ \hline
	$0$ & $3$ & $0$& $0$& $2$ & $0$ & $0$ & $2$ & $600$
	\end{tabular}
	\end{table} \par
We indicate the pivot positions above. This yields $x_1= 208.57$, $x_3= 34.39$, $x_4= 28.57$, and $s_2= 605.71$. All remaining variables have value 0. The maximum value is $600$. Therefore, the maximum value is $600$ and occurs at $(x_1, x_2, x_3, x_4, s_1, s_2, s_3, s_4)= (208.57, 0, 34.29, 28.57, 0, 605.71, 0, 0)$. 
\end{enumerate} 




\newpage



% Problem 3
\problem{10} Find the dual problem to the minimization problem below. 
	\[
	\begin{gathered}
	\text{min } z= 2x_1 + 6x_2 \\
	6x_1 + 5x_2 \leq 10 \\
	x_1 + 3x_2 \leq 9 \\ 
	x_1, x_2 \geq 0
	\end{gathered}
	\] \pspace

\sol First, we write the `matrix associated' to this minimization; that is, we create a matrix with rows corresponding to the equality version of the inequalities (with the exception of the non-negativity inequality) with the function being the last row. This yields matrix:
	\[
	\begin{pmatrix}
	6 & 5 & 10 \\
	1 & 3 & 9 \\
	2 & 6 & 0 
	\end{pmatrix}
	\]
We now find the transpose of this matrix: 
	\[
	\begin{pmatrix}
	6 & 5 & 10 \\
	1 & 3 & 9 \\
	2 & 6 & 0 
	\end{pmatrix}^T= 
	\begin{pmatrix}
	6 & 1 & 2 \\
	5 & 3 & 6 \\
	10 & 9 & 0 
	\end{pmatrix}
	\]
We now find the standard maximization problem corresponding to this matrix:
	\[
	\begin{gathered}
	\max w= 10y_1 + 9y_2 \\
	6y_1 + y_2 \leq 2 \\
	5y_1 + 3y_2 \leq 6 \\
	y_1, y_2 \geq 0
	\end{gathered}
	\]


\end{document}