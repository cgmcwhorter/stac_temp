\documentclass[11pt,letterpaper]{article}
\usepackage[lmargin=1in,rmargin=1in,tmargin=1in,bmargin=1in]{geometry}
\usepackage{../style/homework}
\usepackage{../style/commands}
\setbool{quotetype}{true} % True: Side; False: Under
\setbool{hideans}{false} % Student: True; Instructor: False

% -------------------
% Content
% -------------------
\begin{document}

\homework{3: Due 02/06}{A billion here, a billion there, and pretty soon you're talking about real money.}{Everett Dirksen}

% Problem 1
\problem{10} Suppose that the CPI in 2021 was approximately 278.802 while the CPI in 2022 was approximately 296.797. 
	\begin{enumerate}[(a)]
	\item Find the inflation rate from 2021 to 2022. 
	\item Assuming that the CPI accurately reflects the prices of goods as a whole, how much would you estimate a good that cost \$69.99 in 2021 would cost in 2022?
	\item Assuming that the inflation rate from 2021 to 2022 stayed constant, how much more do you estimate goods would cost in 2030 than in 2022?
	\end{enumerate} \pspace

\sol 
\begin{enumerate}[(a)]
\item We have\dots
	\[
	\dfrac{296.797}{278.802}= 1.064544= 1 + 0.064544
	\]
Therefore, the inflation rate was approximately 6.45\%. \pspace

\item We would expect the good would cost approximately 6.45\% more, i.e. that it would cost\dots
	\[
	\$89.99(1 + 0.064544)= \$89.99 (1.064544) \approx \$95.80
	\] \pspace

\item If the inflation rate stayed constant, the cost of goods would rise by approximately 6.45\% each year for the next 8~years. But then a good that cost $P$~dollars this year would then cost $P(1.064544)^8= P(1.64934)= P(1 + 0.64934)$. Therefore, goods would cost 64.93\% more in 2030 than in 2022. 
\end{enumerate}



\newpage



% Problem 2
\problem{10} Anita needs a short term loan to cover some temporary expense increases. She takes out a short-term, 5~month loan for \$1100 that earns 9.4\% annual simple interest. 
	\begin{enumerate}[(a)]
	\item How much will she owe in total at the end of the five months?
	\item If Anita knows she will only be able to pay at most \$1300 at the end of the five months, what is the most she can take out on the loan now?
	\end{enumerate} \pspace

\sol 
\begin{enumerate}[(a)]
\item We have\dots
	\[
	F= P(1 + rt)= \$1100 \left(1 + 0.094 \cdot \frac{5}{12} \right)= \$1100(1.03917)= \$1143.08
	\] 
Therefore, she will owe \$1,1143.08 at the end of the five months. \pspace

\item We have\dots
	\[
	P= \dfrac{F}{1 + rt}= \dfrac{\$1300}{1 + 0.094 \cdot \dfrac{5}{12}}= \dfrac{\$1300}{1.03917}= \$1251
	\]
Therefore, the most she can take out on the loan is \$1,251. 
\end{enumerate}



\newpage



% Problem 3
\problem{10} Emmanuel is taking out a loan to help purchase a new delivery truck to help expand his small business. He decides on a truck that costs \$67,049. There is a \$160 processing/service fee for the truck. After this is added, he will pay 7\% sales tax. He will take out a simple discount note to pay for this delivery truck. The note will have a period of 9~months at an annual interest rate of 8.3\%.
	\begin{enumerate}[(a)]
	\item How much will the truck cost in total?
	\item What are the maturity and proceeds for this note?
	\item How much does he owe at the end of the 9~months?
	\item How much does he pay for the truck in the end, i.e. how much in total does he pay for the loan?
	\end{enumerate} \pspace

\sol 
\begin{enumerate}[(a)]
\item After the \$160 processing/service fee, the truck costs $\$67049 + \$160= \$67209$. But then there is a 7\% sales tax, after which the price is $\$67209(1 + 0.07)= \$67209(1.07)= \$71931.63$. Therefore, the truck will cost \$71,913.63. \pspace

\item The maturity for this loan is the loan amount, which will be the total amount for the truck. By (a), this is \$71,913.63. Emmanuel will receive this amount minus the discount (interest) for this loan. The discount is $D= Mrt= 71913.63(0.083) \frac{9}{12}= \$4476.62$. Then the amount Emmanuel receives is $\$71913.63 - \$4476.62= \$67437.01$. \pspace

\item After 9~months, Emmanuel must pay back the original loan amount (he has already paid the discount, i.e. the interest), which is \$71,913.63. \pspace

\item In total, he pays the loan plus the discount (the interest). Therefore, the total is $\$71913.63 + \$4476.62= \$76390.25$. 
\end{enumerate}


\end{document}