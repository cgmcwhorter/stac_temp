\documentclass[11pt,letterpaper]{article}
\usepackage[lmargin=1in,rmargin=1in,tmargin=1in,bmargin=1in]{geometry}
\usepackage{../style/homework}
\usepackage{../style/commands}
\setbool{quotetype}{false} % True: Side; False: Under
\setbool{hideans}{false} % Student: True; Instructor: False

% -------------------
% Content
% -------------------
\begin{document}

\homework{14: Due 05/01}{Of the many forms of false culture, a premature converse with abstractions is perhaps the most likely to prove fatal to the growth of a masculine vigour of intellect.}{George Boole}

% Problem 1
\problem{10} Consider the following system of equations:
	\[
	\begin{aligned}
	3x - 2y&= -8 \\
	-x + 3y&= 5
	\end{aligned}
	\]

\begin{enumerate}[(a)]
\item Find the coefficient matrix, $A$. 
\item Show that $A$ has an inverse. 
\item Use your answer from (b) to find the solution to the system of equations. 
\end{enumerate} \pspace

\sol 
\begin{enumerate}[(a)]
\item The coefficient matrix is the matrix of column-by-column coefficients for the variables---properly aligned. Because the variables are already aligned, we have\dots
	\[
	A= \begin{pmatrix} 3 & -2 \\ -1 & 3 \end{pmatrix}
	\] \pspace

\item We know that $A$ has an inverse, i.e. that $A^{-1}$ exists, if and only if $\det A \neq 0$. We have\dots
	\[
	\det A= \det \begin{pmatrix} 3 & -2 \\ -1 & 3 \end{pmatrix}= 3(3) - (-2)(-1)= 9 - 2= 7 \neq 0
	\]
Because $\det A \neq 0$, we know that $A^{-1}$ exists. \pspace

\item Recall that that when written in vector form, i.e. $A\mathbf{x}= \mathbf{b}$, the matrix $A$ is the coefficient matrix (written column-by-column in the same order as the variable vector), $\mathbf{x}$ is the variable vector, and $\mathbf{b}$ is the constant vector. If $A^{-1}$ exists, multiplying both sides of $A\mathbf{x}= \mathbf{b}$ on the left by $A^{-1}$, we have\dots
	\[
	\begin{aligned}
	A\mathbf{x}&= \mathbf{b} \\[0.3cm]
	A^{-1}A \mathbf{x}&= A^{-1} \mathbf{b} \\[0.3cm]
	\mathbf{x}&= A^{-1} \mathbf{b}
	\end{aligned}
	\]
From (b), we know that $A^{-1}$ exists. We need to find $A^{-1}$. But we know how to find the inverse of a $2 \times 2$ matrix:
	\[
	\text{If } A= \begin{pmatrix} a & b \\ c & d \end{pmatrix} \text{ and } \det A \neq 0, \text{ then } A^{-1}= \dfrac{1}{\det A} \begin{pmatrix} d & -b \\ -c & a \end{pmatrix}
	\]
But then we have\dots
	\[
	A^{-1}= \dfrac{1}{7} \begin{pmatrix} 3 & 2 \\ 1 & 3 \end{pmatrix}
	\]
Therefore, we know\dots
	\[
	\mathbf{x}= A^{-1} \mathbf{b}= \dfrac{1}{7} \begin{pmatrix} 3 & 2 \\ 1 & 3 \end{pmatrix} \begin{pmatrix} -8 \\ 5 \end{pmatrix}= \dfrac{1}{7} \begin{pmatrix} 3(-8) + 2(5) \\ 1(-8) + 3(5) \end{pmatrix} \dfrac{1}{7} \begin{pmatrix} -24 + 10 \\ -8 + 15 \end{pmatrix}= \dfrac{1}{7} \begin{pmatrix} -14 \\ 7 \end{pmatrix}= \begin{pmatrix} -2 \\ 1 \end{pmatrix}
	\]
Then the solution is\dots
	\[
	\mathbf{x}= \begin{pmatrix} x \\ y \end{pmatrix}= \begin{pmatrix} -2 \\ 1 \end{pmatrix}
	\]
That is, the solution is $x= -2$ and $y= 1$. 
\end{enumerate}



\newpage



% Problem 2
\problem{10} The RREF form of a matrix coming from a system of equations is shown below. Determine if there is a solution. If so, find the solution(s). If not, explain why the system does not have a solution. 
	\[
	\begin{pmatrix}
	1 & 0 & 0 & 0 & 0 \\
	0 & 1 & 0 & 0 & -4 \\
	0 & 0 & 1 & 0 & 6 \\
	0 & 0 & 0 & 1 & 4
	\end{pmatrix}
	\]



\newpage



% Problem 3
\problem{10} The RREF form of a matrix coming from a system of equations is shown below. Determine if there is a solution. If so, find the solution(s). If not, explain why the system does not have a solution. 
	\[
	\begin{pmatrix}
	1 & 0 & 0 & 0 & 4 \\
	0 & 1 & 3 & 0 & 5 \\
	0 & 0 & 0 & 1 & -2 \\
	0 & 0 & 0 & 0 & 0
	\end{pmatrix}
	\]


\end{document}