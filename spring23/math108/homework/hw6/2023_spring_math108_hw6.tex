\documentclass[11pt,letterpaper]{article}
\usepackage[lmargin=1in,rmargin=1in,tmargin=1in,bmargin=1in]{geometry}
\usepackage{../style/homework}
\usepackage{../style/commands}
\setbool{quotetype}{false} % True: Side; False: Under
\setbool{hideans}{true} % Student: True; Instructor: False

% -------------------
% Content
% -------------------
\begin{document}

\homework{6: Due 02/13}{When I was young I used to think that money was the most important thing in life; now that I am old, I know it is.}{Oscar Wilde}

% Problem 1
\problem{10} Sue A. is taking out a loan to purchase a collection of vintage comics, which she hopes will rise in value so that later she can sell them for a profit. She takes out a loan for \$84,000 at 8.7\% annual interest, compounded monthly. The term of the loan will be 5~years. Find Sue's end of the month payments. How much will she pay in total?



\newpage



% Problem 2
\problem{10} Herbert H. has just accepted a new job in a different school district. After searching for a new home, he finds the perfect one for \$225,000. He takes out a 30-year mortgage for this new home at 4.25\% annual interest, compounded quarterly. His mortgage payments will be at the end of each month. How much are his monthly payments? How much will he owe on the home after 10~years of payments?



\newpage



% Problem 3
\problem{10} Catherine F. is looking to purchase her first home. However, she is unsure what price of home she is able to afford. She discuses with a bank and finds that the best mortgage she is likely to qualify for is a 25~year, 5.6\% annual interest, compounded monthly. If she can only afford payments at the start of each month of at most \$4,600, what is the most expensive advertised price for a home she can afford?



\newpage 



% Problem 4
\problem{10} Bitzi is purchased her home 20~years ago. When she originally purchased her home, she took out a 30-year mortgage at 4.71\% annual interest, compounded quarterly. She has made her beginning of the month payments of \$2,000 faithfully all this time. How much of her next payment actually goes towards repaying her loan? 


\end{document}