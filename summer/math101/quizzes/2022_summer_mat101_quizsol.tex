\documentclass[11pt,letterpaper]{article}
\usepackage[lmargin=1in,rmargin=1in,bmargin=1in,tmargin=1in]{geometry}
\usepackage{style/quiz}
\usepackage{style/commands}

% -------------------
% Content
% -------------------
\begin{document}
\thispagestyle{title}

% Quiz 1
\quizsol \textit{True/False}: If $a$ and $b$ are distinct primes, then $\gcd(a, b)= 1$. \pspace

\sol The statement is \textit{true}. Because $a$ is prime, the only divisors of $a$ are 1 and $a$. But then if $a$ and $b$ are distinct primes, then $a \neq b$. But then the only common divisor for $a$ and $b$ is 1. Therefore, $\gcd(a, b)= 1$. For instance, if $a= 3$ and $b= 5$, we have $\gcd(a, b)= \gcd(3, 5)= 1$. \pvspace{1.3cm}



% Quiz 2
\quizsol \textit{True/False}: There is a rational number equal to the following decimal:
	\[
	0.123456789101112131415\ldots
	\]

\sol The statement is \textit{false}. A rational number is a real number, i.e. decimal number, whose decimal expansion either terminates or repeats. The decimal number above simply `counts out' all the integers. Observe that the decimal expansion is 1, 2, 3, 4, \dots. But then the decimal expansion does not terminate nor does it repeat. Therefore, the decimal number above cannot be rational. \pvspace{1.3cm}



% Quiz 3
\quizsol \textit{True/False}: $\sqrt[4]{2^{100} \cdot 3^{21} \cdot 5^{42}}= 2^{25} \cdot 3^5 \cdot 5^{10} \sqrt[4]{3^1 \cdot 5^2}$ \pspace

\sol The statement is \textit{true}. We have\dots
	\[
	\sqrt[4]{2^{100} \cdot 3^{21} \cdot 5^{42}}= \sqrt[4]{2^{100} \cdot (3^{20} \cdot 3^1) \cdot (5^{40} \cdot 5^2)}= 2^{100/4} \cdot 3^{20/4} \cdot 5^{40/4} \sqrt[4]{3^1 \cdot 5^2}= 2^{25} \cdot 3^5 \cdot 5^{10} \sqrt[4]{3^1 \cdot 5^2}
	\]
Alternatively, observe\dots
	\[
	\sqrt[4]{2^{100} \cdot 3^{21} \cdot 5^{42}}= (2^{100} \cdot 3^{21} \cdot 5^{42})^{1/4}= 2^{25} \cdot 3^{4 + 1/4} \cdot 5^{10 + 2/4}= 2^{25} \cdot 3^4 \cdot 5^{10} \sqrt[4]{3^1 \cdot 5^2}
	\] \pvspace{1.3cm}



% Quiz 4
\quizsol \textit{True/False}: In a course, the first exam is worth 15\% of your course grade. If you receives an 80\% on that exam, then your maximum possible course grade is a 88\%. \pspace

\sol The statement is \textit{false}. Because you received an 80\%, you only received 80\% of the 15~total points possible. Therefore, you have earned $15(0.80)= 12$ points toward your average. But then you have lost $15 - 12= 3$ of your course grade. Therefore, your maximum possible grade is $100 - 3= 97$. Alternatively, by receiving an 80\%, you have lost $100\% - 80\%= 20\%$ of the 15~points of your course grade. Therefore, you have lost $15(0.20)= 3$ points of your overall average. Therefore, your maximum possible course grade is $100 - 3= 97$. \pvspace{1.3cm}



% Quiz 5
\quizsol \textit{True/False}: A function can have more than one $y$-intercept. \pspace

\sol The statement is \textit{false}. Suppose the function had more than one $y$-intercept. Say $A$ and $B$ are $y$-intercepts, i.e. $(0, A)$ and $(0, B)$ are $y$-intercepts. But then the vertical line at $x= 0$ intersects the graph of the function in at least two points---namely, $(0, A)$ and $(0, B)$. But then the function fails the vertical line test. Therefore, the `function' cannot actually be a function. Alternatively, Say $A$ and $B$ are $y$-intercepts, i.e. $(0, A)$ and $(0, B)$ are $y$-intercepts. But then the `function' is not well defined at $x= 0$. Therefore, the `function' is not actually a function. \pvspace{1.3cm}



% Quiz 6
\quizsol \textit{True/False}: If $f(x)= 6x - 5$ and $f^{-1}(x)$ exists, then $f^{-1}(3)= 13$. \pspace

\sol The statement is \textit{false}. Recall that if $f^{-1}(y)= x$, then $f(x)= y$. So if $f^{-1}(3)= 13$, then $f(13)= 3$. However, $f(13)= 6(13) - 5= 78 - 5= 73 \neq 3$. We can even find the correct value. Suppose that $x= f^{-1}(3)$. Then we have\dots
	\[
	\begin{aligned}
	x&= f^{-1}(3) \\
	f(x)&= f(f^{-1}(3)) \\
	f(x)&= 3 \\
	6x - 5&= 3 \\
	6x&= 8 \\
	x&= \dfrac{8}{6} \\
	x&= \dfrac{4}{3}
	\end{aligned}
	\]
We can check this as $f(\frac{4}{3})= 6 \cdot \frac{4}{3} - 5= 8 - 5= 3$. Therefore, $f^{-1}(3)= \frac{4}{3}$. 



\end{document}