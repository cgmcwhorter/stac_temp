\documentclass[11pt,letterpaper]{article}
\usepackage[lmargin=1in,rmargin=1in,tmargin=1in,bmargin=1in]{geometry}
\usepackage{../style/homework}
\usepackage{../style/commands}
\setbool{quotetype}{false} % True: Side; False: Under
\setbool{hideans}{false} % Student: True; Instructor: False

% -------------------
% Content
% -------------------
\begin{document}

\homework{6: Due 06/02}{The fact that we live at the bottom of a deep gravity well, on the surface of a gas covered planet going around a nuclear fireball 90~million miles away and think this to be normal is obviously some indication of how skewed our perspective tends to be.}{Douglas Adams}

% Problem 1
\problem{10} Determine whether the following lines are parallel, perpendicular, or neither. Be sure to justify your answer. \pvspace{0.1cm}
	\[
	\begin{aligned}
	\ell_1 &\colon y= \dfrac{2}{3}\,x + 5 \\[0.3cm]
	\ell_2 &\colon 3x - 2y= 8
	\end{aligned}
	\] \pspace

\sol Putting each line in the form $y= mx + b$, we have\dots
	\[
	\begin{aligned}
	y&= \dfrac{2}{3}\,x + 5 &\qquad 3x - 2y&= 8 \\[0.3cm]
	& & -2y&= -3x + 8 \\[0.3cm]
	& & y&= \dfrac{3}{2}\,x - 4
	\end{aligned}
	\]
The slope of the first line is $m_1= \frac{2}{3}$ and the slope of the second line is $m_2= \frac{3}{2}$. Because $m_1 \neq m_2$, the lines are not parallel. Therefore, the lines intersect. Because the negative reciprocal of $m_1$ is $-\frac{3}{2} \neq m_2$, the lines are not perpendicular. Therefore, the lines are neither parallel nor perpendicular. 



\newpage



% Problem 2
\problem{10} Determine whether the following lines are parallel, perpendicular, or neither. Be sure to justify your answer. \pvspace{0.1cm}
	\[
	\begin{aligned}
	\ell_1 &\colon -5x + 6y= 6 \\[0.3cm]
	\ell_2 &\colon 5x + 6y= -12
	\end{aligned}
	\] \pspace

\sol Putting each line in the form $y= mx + b$, we have\dots
	\[
	\begin{aligned}
	-5x + 6y&= 6 &\qquad 5x + 6y&= -12 \\[0.3cm]
	6y&= 5x + 6 & 6y&= -5x - 12 \\[0.3cm]
	y&= \dfrac{5}{6}\,x + 1 & y&= -\dfrac{5}{6}\,x - 2
	\end{aligned}
	\]
The slope of the first line is $m_1= \frac{5}{6}$ and the slope of the second line is $m_2= -\frac{5}{6}$. Because $m_1 \neq m_2$, the lines are not parallel. Therefore, the lines intersect. Because the negative reciprocal of $m_1$ is $-\frac{6}{5} \neq m_2$, the lines are not perpendicular. Therefore, the lines are neither parallel nor perpendicular. 



\newpage



% Problem 3
\problem{10} Find the equation of the line with $x$-intercept $(6, 0)$ and passing through the point $(-1, 10)$. \pspace

\sol Because the desired line is not vertical, we know that it has the form $y= mx + b$. Because the line passes through the points $(6, 0)$ and $(-1, 10)$, we have\dots
	\[
	\begin{aligned}
	m= \dfrac{0 - 10}{6 - (-1)}= \dfrac{-10}{7}= -\dfrac{10}{7}
	\end{aligned}
	\]
We know that $y= -\frac{10}{7}\,x + b$. But the line contains the point $(6, 0)$ so that\dots
	\[
	\begin{aligned}
	y&= -\dfrac{10}{7}\,x + b \\[0.3cm]
	0&= -\dfrac{10}{7} \cdot 6 + b \\[0.3cm]
	0&= -\dfrac{60}{7} + b \\[0.3cm]
	b&= \dfrac{60}{7}
	\end{aligned}
	\]
Therefore, the line is $y= -\frac{10}{7}\,x + \frac{60}{7}= \dfrac{60 - 10x}{7}$. 



\newpage



% Problem 4
\problem{10} Find the equation of the line perpendicular to the line $2x - 3y= 5$ that passes through the origin. \pspace

\sol Because the desired line is not vertical, we know that it has the form $y= mx + b$. Because the line is perpendicular to the line $2x - 3y= 5$, its slope is the negative reciprocal of the slope of the line $2x - 3y= 5$. We know\dots
	\[
	\begin{aligned}
	2x - 3y&= 5 \\[0.3cm]
	-3y&= -2x + 5 \\[0.3cm]
	y&= \dfrac{2}{3}\,x - \dfrac{5}{3}
	\end{aligned}
	\] 
so that the line has slope $\frac{2}{3}$. Therefore, the slope of the desired line is $m= -\frac{3}{2}$. We then know that $y= -\frac{3}{2}x + b$. Because the line contains the origin, i.e. the point $(0, 0)$, we have\dots
	\[
	\begin{aligned}
	y&= -\dfrac{3}{2}\,x + b \\[0.3cm]
	0&= -\dfrac{3}{2} \cdot 0 + b \\[0.3cm]
	b&= 0
	\end{aligned}
	\] 
Therefore, the line is $y= -\frac{3}{2}\,x$. 



\newpage



% Problem 5
\problem{10} Find the equation of the line that contains $(1, -1)$ and is parallel to the line $3x + y= 11$. \pspace

\sol Because the desired line is not vertical, we know that the line has the form $y= mx + b$. The desired line is parallel to the line $3x + y= 11$, implying that they have the same slope. We know\dots
	\[
	\begin{aligned}
	3x + y&= 11 \\[0.3cm]
	y&= -3x + 11
	\end{aligned}
	\]
so that the line has slope $-3$. Therefore, the slope of the desired line is $m= -3$. We then have $y= -3x + b$. The line contains the point $(1, -1)$ so that when $x= 1$, we know that $y= -1$. But then\dots
	\[
	\begin{aligned}
	y&= -3x + b \\[0.3cm]
	-1&= -3(1) + b \\[0.3cm]
	-1&= -3 + b \\[0.3cm]
	b&= 2
	\end{aligned}
	\]
Therefore, $y= -3x + 2$. 



\newpage



% Problem 6
\problem{10} Showing all your work, solve the following equation and verify that your solution is correct:
	\[
	5x - 7= 7 - 2x
	\] \pspace

\sol We have\dots
	\[
	\begin{aligned}
	5x - 7&= 7 - 2x \\[0.3cm]
	7x - 7&= 7 \\[0.3cm]
	7x&= 14 \\[0.3cm]
	x&= 2
	\end{aligned}
	\] \pspace
We verify the solution:
	\[
	\begin{aligned}
	5x - 7&= 7 - 2x \\[0.3cm]
	5(2) - 7&\stackrel{?}{=} 7 - 2(2) \\[0.3cm]
	10 - 7&\stackrel{?}{=} 7 - 4 \\[0.3cm]
	3&= 3 \\
	&\text{ \cmark}
	\end{aligned}
	\]



\newpage



% Problem 7
\problem{10} Showing all your work, solve the following equation and verify that your solution is correct:
	\[
	2(1 - x)= 6x + 11
	\] \pspace

\sol We have\dots
	\[
	\begin{aligned}
	2(1 - x)&= 6x + 11 \\[0.3cm]
	2 - 2x&= 6x + 11 \\[0.3cm]
	2&= 8x + 11 \\[0.3cm]
	-9&= 8x \\[0.3cm]
	x&= -\dfrac{9}{8}
	\end{aligned}
	\] \pspace
We verify the solution:
	\[
	\begin{aligned}
	2(1 - x)&= 6x + 11 \\[0.3cm]
	2 \left( 1 - \dfrac{-9}{8} \right)&\stackrel{?}{=} 6 \cdot -\dfrac{9}{8} + 11 \\[0.3cm]
	2 \left( \dfrac{8}{8} - \dfrac{-9}{8} \right)&\stackrel{?}{=} 3 \cdot -\dfrac{9}{4} + 11 \\[0.3cm]
	2 \cdot \dfrac{17}{8}&\stackrel{?}{=} -\dfrac{27}{4} + 11 \\[0.3cm]
	\dfrac{17}{4}&\stackrel{?}{=} -\dfrac{27}{4} + \dfrac{44}{4} \\[0.3cm]
	\dfrac{17}{4}&= \dfrac{17}{4} \\
	&\text{ \cmark}
	\end{aligned}
	\]



\newpage



% Problem 8
\problem{10} Showing all your work, solve the following equation and verify that your solution is correct:
	\[
	\dfrac{x - 1}{x + 3}= 5
	\] \pspace

\sol We have\dots
	\[
	\begin{aligned}
	\dfrac{x - 1}{x + 3}&= 5 \\[0.3cm]
	x - 1&= 5(x + 3) \\[0.3cm]
	x - 1&= 5x + 15 \\[0.3cm]
	-1&= 4x + 15 \\[0.3cm]
	-16&= 4x \\[0.3cm]
	x&= -4 
	\end{aligned}
	\] \pspace
We verify the solution:
	\[
	\begin{aligned}
	\dfrac{x - 1}{x + 3}&= 5 \\[0.3cm]
	\dfrac{-4 - 1}{-4 + 3}&\stackrel{?}{=} 5 \\[0.3cm]
	\dfrac{-5}{-1}&\stackrel{?}{=} 5 \\[0.3cm]
	5&= 5 \\
	&\text{ \cmark}
	\end{aligned}
	\]



\newpage



% Problem 9
\problem{10} Suppose you sell automobiles. You earn a weekly baseline salary of \$820 per week and make 3\% commission on your sales. Let $I(s)$ denote your weekly income if you make $s$~dollars in sales. 
	\begin{enumerate}[(a)]
	\item Explain why $I(s)$ is linear. 
	\item Find $I(s)$.
	\item Find an interpret the slope and $y$-intercept of $I(s)$ in context, if possible. 
	\item How much in sales do you have to make in a given week to have made \$1,500?
	\end{enumerate} \pspace

\sol 
\begin{enumerate}[(a)]
\item The only money earned comes from the baseline salary and commission. Because you get paid a constant baseline salary and earn a constant commission rate, the total amount you make each week is constant. Therefore, the amount of money you earn each week after $s$~dollars in sales, $I(s)$, is linear. 

\item Each week, you make \$820. If you sell $s$~dollars, you earn 3\% commission, i.e. 3\% of the total sales value. This is $0.03 \cdot s= 0.03s$. Therefore, you make $0.03s + 820$ each week, i.e. $I(s)= 0.03s + 820$. 

\item Because $I(s)= 0.03s + 820$ is in the form $y= mx + b$, we have $m= 0.03$ and $b= 820$. The slope, $m= 0.03$, is the commission you make on $s$~dollars in sales. The $y$-intercept, $b= 820$ or $(0, 820)$, represents the amount you are paid each week---regardless of the amount in sales you make. 

\item If you sell $s$~dollars in automobiles, you make $I(s)$ total that week. But then we want $I(s)= 1500$. Then we have\dots
	\[
	\begin{aligned}
	I(s)&= 1500 \\[0.3cm]
	0.03s + 820&= 1500 \\[0.3cm]
	0.03s&= 680 \\[0.3cm]
	s&= 22666.67
	\end{aligned}
	\]
Therefore, to make \$1,500 in a week, you have to sell \$22,666.67 in automobiles. 
\end{enumerate}



\newpage



% Problem 10
\problem{10} The amount of people, on average, that have entered a store $t$~hours after it has opened, $P(t)$, can be modeled by $P(t)= 30.5t - 4$. 
	\begin{enumerate}[(a)]
	\item What does $P(t)$ being linear imply about the rate that people enter the store?
	\item Find an interpret the slope and $y$-intercept of $I(s)$ in context, if possible.
	\item Find $P(2)$ and interpret the value. 
	\item How long after opening until 400~people have entered the store? 
	\end{enumerate} \pspace

\sol
\begin{enumerate}[(a)]
\item Because the amount of people having entered the store $t$ hours after opening, $P(t)$, on average is linear, we know that the rate of people entering the store is constant, on average. Because $P(t)= 30.5t - 4$ has the form $y= mx + b$ with $m= 30.5$ and $b= -4$, we know that, on average, 30.5~people enter the store every hour. 

\item Because $P(t)= 30.5t - 4$ has the form $y= mx + b$ with $m= 30.5$ and $b= -4$, we know that the $y$-intercept is $-4$, i.e $(0, -4)$. This would imply that when $t= 0$, zero hours from when the store opens, i.e. at opening, $-4$~people will enter the store, on average. As it is impossible for there to be $-4$~people entering the store on average (unless one wants this to mean, on average, 4~people leave the store at opening), there is no in-context interpretation for the $y$-intercept. 

\item We have $P(2)= 30.5(2) - 4= 61 - 4= 57$; that is, two hours after the store opens, 57~people have entered the store, on average. 

\item If 400~people have entered the store, then $P(t)= 400$. But then we have\dots
	\[
	\begin{aligned}
	P(t)&= 400 \\[0.3cm]
	30.5t - 4&= 400 \\[0.3cm]
	30.5t&= 404 \\[0.3cm]
	t&= 13.25
	\end{aligned}
	\]
Therefore, on average, after 13.25~hours after opening, i.e. 13~hours and 15~minutes after opening, 400 people will have entered the store. 
\end{enumerate}


\end{document}