\documentclass[11pt,letterpaper]{article}
\usepackage[lmargin=1in,rmargin=1in,tmargin=1in,bmargin=1in]{geometry}
\usepackage{../style/homework}
\usepackage{../style/commands}
\setbool{quotetype}{true} % True: Side; False: Under
\setbool{hideans}{true} % Student: True; Instructor: False

% -------------------
% Content
% -------------------
\begin{document}

\homework{11: Due 06/15}{It is strange that only extraordinary men make the discoveries, which later appear so easy and simple.}{Georg C. Lichtenberg}

% Problem 1
\problem{10} Write the following exponential functions in the form $y= Ab^x$:
	\begin{enumerate}[(a)]
	\item $y= -3(2^{3x})$
	\item $f(x)= 4 \left( \dfrac{5}{7} \right)^{-x}$
	\item $g(x)= -6 (5^{1 - 3x})$
	\item $h(x)= 9 \left( \dfrac{3}{2} \right)^{2x - 1}$
	\end{enumerate}



\newpage



% Problem 2
\problem{10} Write the following exponential functions in the form $y= Ab^{-x}$:
	\begin{enumerate}[(a)]
	\item $y= 6 (2^x)$
	\item $f(x)= -7 \left( \dfrac{1}{3} \right)^x$
	\item $g(x)= 5 \left( \dfrac{1}{6} \right)^{2x}$
	\item $h(x)= 3^{3x + 1}$
	\end{enumerate}



\newpage



% Problem 3
\problem{10} Find an integer $n$ so that each of the following logarithms are between $n$ and $n + 1$, i.e. estimate the logarithm without the use of a calculator. Be sure to show all your work.
	\begin{enumerate}[(a)]
	\item $\log_2(11)$
	\item $\log_3(187)$
	\item $\log_{1/2}(5)$
	\item $\log_5(\frac{1}{20})$
	\end{enumerate}



\newpage



% Problem 4
\problem{10} For each of the following, either express the given exponential equation in terms of logarithms or express the given logarithmic equation in terms of exponentials:
	\begin{enumerate}[(a)]
	\item $5^x= 9$
	\item $\log_3(x)= 4$
	\item $2^3= x$
	\item $\log_7(2)= x$
	\end{enumerate}



\newpage



% Problem 5
\problem{10} Showing all your work, compute the following exactly:
	\begin{enumerate}[(a)]
	\item $\log_2(64)$
	\item $\log_3 \left( \dfrac{1}{27} \right)$
	\item $\ln(1)$
	\item $\log_{2/3} \left( \frac{3}{2} \right)$
	\item $\log_8(8)$
	\end{enumerate}



\newpage



% Problem 6
\problem{10} For each of the following, express the given logarithm in terms of $\log_b$ for the given base $b$:
	\begin{enumerate}[(a)]
	\item $\log_5(25)$, $b= 2$
	\item $\log_7(64)$, $b= 8$
	\item $\log_3(10)$, $b= e$
	\item $\log_{20}(6)$, $b= 6$
	\end{enumerate}



\newpage



% Problem 7
\problem{10} Express each of the following logarithms in terms of $\log x$, $\log y$, $\log z$, and a constant term:
	\begin{enumerate}[(a)]
	\item $\log_2(x^2y)$
	\item $\log_7 \left(\dfrac{x y^2}{z^3} \right)$
	\item $\ln \left( \dfrac{x z^{-1}}{\sqrt[3]{y}} \right)$
	\item $\log_5(25 x \sqrt{y})$
	\end{enumerate}



\newpage



% Problem 8
\problem{10} Express each of the following logarithms in terms of a single logarithm involving no negative powers:
	\begin{enumerate}[(a)]
	\item $\log_2(x) - 5 \log_2(y)$
	\item $-\dfrac{1}{2} \big( 6 \log_3(x) - \log_3(y) \big)$
	\item $5\ln(x^2) - 2 \ln \left( \dfrac{1}{y} \right)$
	\item $\log_6(x) - 5 \log_6(y) + 2$
	\end{enumerate}



\newpage



% Problem 9
\problem{10} Showing all your work, solve the following equation: 
	\[
	15^x + 10= 20
	\]



\newpage



% Problem 10
\problem{10} Showing all your work, solve the following equation: 
	\[
	6(2^{3x}) - 2= 34
	\]


\end{document}