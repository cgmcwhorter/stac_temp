\documentclass[11pt,letterpaper]{article}
\usepackage[lmargin=1in,rmargin=1in,tmargin=1in,bmargin=1in]{geometry}
\usepackage{../style/homework}
\usepackage{../style/commands}
\setbool{quotetype}{true} % True: Side; False: Under
\setbool{hideans}{false} % Student: True; Instructor: False

% -------------------
% Content
% -------------------
\begin{document}

\homework{11: Due 06/15}{It is strange that only extraordinary men make the discoveries, which later appear so easy and simple.}{Georg C. Lichtenberg}

% Problem 1
\problem{10} Write the following exponential functions in the form $y= Ab^x$:
	\begin{enumerate}[(a)]
	\item $y= -3(2^{3x})$
	\item $f(x)= 4 \left( \dfrac{5}{7} \right)^{-x}$
	\item $g(x)= -6 (5^{1 - 3x})$
	\item $h(x)= 9 \left( \dfrac{3}{2} \right)^{2x - 1}$
	\end{enumerate} \pspace

\sol
\begin{enumerate}[(a)]
\item 
	\[
	y= -3(2^{3x})= -3 (2^3)^x= -3 (8^x)
	\] \pspace

\item 
	\[
	f(x)= 4 \left( \dfrac{5}{7} \right)^{-x}= 4 \left( \left( \dfrac{5}{7}\right)^{-1} \right)^x= 4 \left( \dfrac{7}{5} \right)^x
	\] \pspace

\item 
	\[
	g(x)= -6 (5^{1 - 3x})= -6 \cdot 5^1 \cdot 5^{-3x}= -30 (5^{-3})^x= -30 \left( \dfrac{1}{125} \right)^x
	\] \pspace

\item 
	\[
	h(x)= 9 \left( \dfrac{3}{2} \right)^{2x - 1}= 9 \cdot \left( \dfrac{3}{2} \right)^{2x} \cdot \left( \dfrac{3}{2} \right)^{-1}= 9 \cdot \left( \left( \dfrac{3}{2} \right)^2 \right)^x \cdot \dfrac{2}{3}= 6 \cdot \left( \dfrac{9}{4} \right)^x 
	\]
\end{enumerate}



\newpage



% Problem 2
\problem{10} Write the following exponential functions in the form $y= Ab^{-x}$:
	\begin{enumerate}[(a)]
	\item $y= 6 (2^x)$
	\item $f(x)= -7 \left( \dfrac{1}{3} \right)^x$
	\item $g(x)= 5 \left( \dfrac{1}{6} \right)^{2x}$
	\item $h(x)= 3^{3x + 1}$
	\end{enumerate} \pspace

\sol 
\begin{enumerate}[(a)]
\item 
	\[
	y= 6 (2^x)= 6 (2^{-1})^{-x}= 6 \left( \dfrac{1}{2} \right)^{-x}
	\] \pspace

\item 
	\[
	f(x)= -7 \left( \dfrac{1}{3} \right)^x= -7 \left( \left( \dfrac{1}{3} \right)^{-1} \right)^{-x}= -7 (3^{-x}) 
	\] \pspace

\item 
	\[
	g(x)= 5 \left( \dfrac{1}{6} \right)^{2x}= 5 \left( \left( \dfrac{1}{6} \right)^2 \right)^x= 5 \left( \dfrac{1}{36} \right)^x= 5 \left( \left( \dfrac{1}{36} \right)^{-1} \right)^{-x}= 5 (36^{-x})
	\] \pspace

\item 
	\[
	h(x)= 3^{3x + 1}= 3^{3x} \cdot 3^1= 3 (3^3)^x= 3(27^x)= 3(27^{-1})^{-x}= 3 \left( \dfrac{1}{27} \right)^{-x}
	\]
\end{enumerate}



\newpage



% Problem 3
\problem{10} Find an integer $n$ so that each of the following logarithms are between $n$ and $n + 1$, i.e. estimate the logarithm without the use of a calculator. Be sure to show all your work.
	\begin{enumerate}[(a)]
	\item $\log_2(11)$
	\item $\log_3(187)$
	\item $\log_{1/2}(5)$
	\item $\log_5(\frac{1}{20})$
	\end{enumerate} \pspace

\sol 
\begin{enumerate}[(a)]
\item Because $2^3= 8 < 11$ and $2^4= 16 > 11$, we know that $3 < \log_2(11) < 4$. \pspace

\item Because $3^4= 81 < 187$ and $3^5= 243 > 187$, we know that $4 < \log_3(187) < 5$. \pspace

\item Because $(\frac{1}{2})^{-2}= (2^{-1})^{-2}= 2^2= 4 < 5$ and $(\frac{1}{2})^{-3}= (2^{-1})^{-3}= 2^3= 8 > 5$, we know that $-3 < \log_{1/2}(5) < -2$. \pspace

\item Because $5^{-2}= \frac{1}{25} < \frac{1}{20}$ and $5^{-1}= \frac{1}{5} > \frac{1}{20}$, we know that $-2 < \log_5(\frac{1}{20}) < -1$. 
\end{enumerate}



\newpage



% Problem 4
\problem{10} For each of the following, either express the given exponential equation in terms of logarithms or express the given logarithmic equation in terms of exponentials:
	\begin{enumerate}[(a)]
	\item $5^x= 9$
	\item $\log_3(x)= 4$
	\item $2^3= x$
	\item $\log_7(2)= x$
	\end{enumerate} \pspace

\sol 
\begin{enumerate}[(a)]
\item $5^x= 9 \Longleftrightarrow \log_5(9)= x$ \pspace

\item $\log_3(x)= 4 \Longleftrightarrow 3^4= x$ \pspace

\item $2^3= x \Longleftrightarrow \log_2(x)= 3$ \pspace

\item $\log_7(2)= x \Longleftrightarrow 7^x= 2$
\end{enumerate}



\newpage



% Problem 5
\problem{10} Showing all your work, compute the following exactly:
	\begin{enumerate}[(a)]
	\item $\log_2(64)$
	\item $\log_3 \left( \dfrac{1}{27} \right)$
	\item $\ln(1)$
	\item $\log_{2/3} \left( \frac{3}{2} \right)$
	\item $\log_8(8)$
	\end{enumerate} \pspace

\sol 
\begin{enumerate}[(a)]
\item $\log_2(64)= \log_2(2^6)= 6$ \pspace

\item $\log_3 \left( \dfrac{1}{27} \right)= \log_3 \left( \dfrac{1}{3^3} \right)= \log_3(3^{-3})= -3$ \pspace

\item $\ln(1)= 0$ \pspace

\item $\log_{2/3} \left( \frac{3}{2} \right)= \log_{2/3} \left( \left( \frac{2}{3} \right)^{-1} \right)= -1$ \pspace

\item $\log_8(8)= 1$
\end{enumerate}



\newpage



% Problem 6
\problem{10} For each of the following, express the given logarithm in terms of $\log_b$ for the given base $b$:
	\begin{enumerate}[(a)]
	\item $\log_5(25)$, $b= 2$
	\item $\log_7(64)$, $b= 8$
	\item $\log_3(10)$, $b= e$
	\item $\log_{20}(6)$, $b= 6$
	\end{enumerate} \pspace

\sol 
\begin{enumerate}[(a)]
\item 
	\[
	\log_5(25)= \dfrac{\log_2(25)}{\log_2(5)}
	\] \pspace

\item 
	\[
	\log_7(64)= \dfrac{\log_8(64)}{\log_8(7)}= \dfrac{2}{\log_8(7)}
	\] \pspace

\item 
	\[
	\log_3(10)= \dfrac{\log_e(10)}{\log_e(3)}= \dfrac{\ln(10)}{\ln(3)}
	\] \pspace

\item 
	\[
	\log_{20}(6)= \dfrac{\log_6(6)}{\log_6(20)}= \dfrac{1}{\log_6(20)}
	\]
\end{enumerate}



\newpage



% Problem 7
\problem{10} Express each of the following logarithms in terms of $\log x$, $\log y$, $\log z$, and a constant term:
	\begin{enumerate}[(a)]
	\item $\log_2(x^2y)$
	\item $\log_7 \left(\dfrac{x y^2}{z^3} \right)$
	\item $\ln \left( \dfrac{x z^{-1}}{\sqrt[3]{y}} \right)$
	\item $\log_5(25 x \sqrt{y})$
	\end{enumerate} \pspace

\sol 
\begin{enumerate}[(a)]
\item 
	\[
	\log_2(x^2y)= \log_2(x^2) + \log_2(y)= 2 \log_2(x) + \log_2(y)
	\] \pspace

\item 
	\[
	\log_7 \left(\dfrac{x y^2}{z^3} \right)= \log_7(xy^2) - \log_7(z^3)= \log_7(x) + \log_7(y^2) - \log_7(z^3)= \log_7(x) + 2\log_7(y) - 3\log_7(z)
	\] \pspace

\item 
	\[
	\ln \left( \dfrac{x z^{-1}}{\sqrt[3]{y}} \right)= \ln(xz^{-1}) - \ln(\sqrt[3]{y})= \ln(x) + \ln(z^{-1}) - \ln(y^{1/3})= \ln(x) - \ln(z) - \dfrac{1}{3}\, \ln(y)
	\] \pspace

\item 
	\[
	\log_5(25 x \sqrt{y})= \log_5(25) + \log_5(x) + \log_5(\sqrt{y})= 2 + \log_5(x) + \dfrac{1}{2}\, \log_5(y)
	\]
\end{enumerate}



\newpage



% Problem 8
\problem{10} Express each of the following logarithms in terms of a single logarithm involving no negative powers:
	\begin{enumerate}[(a)]
	\item $\log_2(x) - 5 \log_2(y)$
	\item $-\dfrac{1}{2} \big( 6 \log_3(x) - \log_3(y) \big)$
	\item $5\ln(x^2) - 2 \ln \left( \dfrac{1}{y} \right)$
	\item $\log_6(x) - 5 \log_6(y) + 2$
	\end{enumerate} \pspace

\sol 
\begin{enumerate}[(a)]
\item 
	\[
	\log_2(x) - 5 \log_2(y)= \log_2(x) - \log_2(y^5)= \log_2 \left( \dfrac{x}{y^5} \right)
	\] \pspace

\item 
	\[
	-\dfrac{1}{2} \big( 6 \log_3(x) - \log_3(y) \big)= -3 \log_3(x) + \dfrac{1}{2}\, \log_3(y)= \log_3(x^{-3}) + \log_3( \sqrt{y} )= \log_3( x^{-3} \sqrt{y} )= \log_3 \left( \dfrac{\sqrt{y}}{x^3} \right)
	\] \pspace

\item 
	\[
	5\ln(x^2) - 2 \ln \left( \dfrac{1}{y} \right)= \ln \big( (x^2)^5 \big) -  \ln \left( \left( \dfrac{1}{y}  \right)^{1/2} \right)= \ln(x^{10}) - \ln \left( \dfrac{1}{\sqrt{y}} \right)= \ln \left( \dfrac{x^{10}}{1/\sqrt{y}} \right)= \ln( x^{10} \sqrt{y} )
	\] \pspace

\item 
	\[
	\log_6(x) - 5 \log_6(y) + 2= \log_6(x) - 5 \log_6(y) + \log_6(6^2)= \log_6(x) - \log_6(y^5) + \log_6(36)= \log_6 \left( \dfrac{36x}{y^5} \right)
	\]
\end{enumerate}



\newpage



% Problem 9
\problem{10} Showing all your work, solve the following equation: 
	\[
	15^x + 10= 20
	\] \pspace

\sol
	\begin{gather*}
	15^x + 10= 20 \\[0.3cm]
	15^x= 10 \\[0.3cm]
	\log_15(15^x)= \log_{15}(10) \\[0.3cm]
	x= \log_{15}(10)
	\end{gather*}



\newpage



% Problem 10
\problem{10} Showing all your work, solve the following equation: 
	\[
	6(2^{3x}) - 2= 34
	\] \pspace

\sol
	\begin{gather*}
	6(2^{3x})= 36 \\[0.3cm]
	2^{3x}= 6 \\[0.3cm]
	\log_2(2^{3x})= \log_2(6) \\[0.3cm]
	3x= \log_2(6) \\[0.3cm]
	x= \dfrac{\log_2(6)}{3}
	\end{gather*}


\end{document}