\documentclass[11pt,letterpaper]{article}
\usepackage[lmargin=1in,rmargin=1in,tmargin=1in,bmargin=1in]{geometry}
\usepackage{../style/homework}
\usepackage{../style/commands}
\setbool{quotetype}{true} % True: Side; False: Under
\setbool{hideans}{false} % Student: True; Instructor: False

% -------------------
% Content
% -------------------
\begin{document}

\homework{9: Due 06/09}{Science is organized knowledge. Wisdom is organized life.}{Immanuel Kant}

% Problem 1
\problem{10} Find the domain of the rational function below. What are the vertical asymptotes of the given rational function? Also, simplify the rational function. 
	\[
	\dfrac{x^2 - 36}{x^2 - 2x - 24}
	\] \pspace

\sol The domain of a rational function $\frac{p(x)}{q(x)}$ are the values for which $q(x) \neq 0$. We have\dots \pspace
	\[
	\begin{aligned}
	q(x)&= 0 \\[0.3cm]
	x^2 - 2x - 24&= 0 \\[0.3cm]
	(x + 4)(x - 6)&= 0 \\[0.3cm]
	x + 4= 0 &\text{  or  } x - 6= 0 \\[0.3cm]
	x= -4 &\text{  or  } x= 6
	\end{aligned}
	\] \pspace
Therefore, the domain of the function is $x \neq -4$, $6$, i.e. all real numbers except for $x= -4$ and $x= 6$. For $x \neq -4$, $6$, we have\dots \pspace
	\[
	\dfrac{x^2 - 36}{x^2 - 2x - 24}= \dfrac{(x - 6)(x + 6)}{(x + 4)(x - 6)}= \dfrac{\cancel{(x - 6)}(x + 6)}{(x + 4)\cancel{(x - 6)}}= \dfrac{x + 6}{x + 4}
	\] \pspace
Therefore, the vertical asymptotes of $\frac{x^2 - 36}{x^2 - 2x - 24}$ are $x= -4$. Because $\frac{x + 6}{x + 4}$ evaluated at $x= 6$ is $\frac{6 + 6}{6 + 4}= \frac{12}{10}= \frac{6}{5}$, the value $x= 6$ corresponds to the hole $(6, \frac{6}{5})$ for $\frac{x^2 - 36}{x^2 - 2x - 24}$.  
	\[
	\boxed{
	\begin{aligned}
	\text{Domain: }& x \in \mathbb{R}, x \neq -4, 6 \\
	\text{Vertical Asymptotes: }& x= -4
	\end{aligned}
	}
	\]



\newpage



% Problem 2
\problem{10} Simplifying as much as possible, compute the following:
	\[
	\dfrac{3x + 1}{x^2 - 1} - \dfrac{x + 5}{x + 1}
	\] \pspace

\sol
	\[
	\begin{aligned}
	\dfrac{3x + 1}{x^2 - 1} - \dfrac{x + 5}{x + 1}&= \dfrac{3x + 1}{(x - 1)(x + 1)} - \dfrac{x + 5}{x + 1} \\[0.3cm]
	&= \dfrac{3x + 1}{(x - 1)(x + 1)} - \dfrac{(x - 1)(x + 5)}{(x - 1)(x + 1)} \\[0.3cm]
	&= \dfrac{3x + 1}{(x - 1)(x + 1)} - \dfrac{x^2 + 5x - x - 5}{(x - 1)(x + 1)} \\[0.3cm]
	&= \dfrac{3x + 1 - (x^2 + 4x - 5)}{(x - 1)(x + 1)} \\[0.3cm]
	&= \dfrac{3x + 1 - x^2 - 4x + 5}{(x - 1)(x + 1)} \\[0.3cm]
	&= \dfrac{-x^2 - x + 6}{(x - 1)(x + 1)} \\[0.3cm]
	&= \dfrac{-(x^2 + x - 6)}{(x - 1)(x + 1)} \\[0.3cm]
	&= \dfrac{-(x + 3)(x - 2)}{(x - 1)(x + 1)} \\[0.3cm]
	\end{aligned}
	\]



\newpage



% Problem 3
\problem{10} Simplifying as much as possible, compute the following:
	\[
	\dfrac{7 - x}{x^2 + 8x + 12} + \dfrac{x}{x^2 + x - 30}
	\] \pspace

\sol
	\[
	\begin{aligned}
	\dfrac{7 - x}{x^2 + 8x + 12} + \dfrac{x}{x^2 + x - 30}&= \dfrac{7 - x}{(x + 2)(x + 6)} + \dfrac{x}{(x + 6)(x - 5)} \\[0.3cm]
	&= \dfrac{(7 - x)(x - 5)}{(x + 2)(x + 6)(x - 5)} + \dfrac{x(x + 2)}{(x + 6)(x - 5)(x + 2)} \\[0.3cm]
	&= \dfrac{7x - 35 - x^2 + 5x}{(x + 2)(x + 6)(x - 5)} + \dfrac{x^2 + 2x}{(x + 6)(x - 5)(x + 2)} \\[0.3cm]
	&= \dfrac{-x^2 + 12x - 35}{(x + 2)(x + 6)(x - 5)} + \dfrac{x^2 + 2x}{(x + 6)(x - 5)(x + 2)} \\[0.3cm]
	&= \dfrac{-x^2 + 12x - 35 + x^2 + 2x}{(x + 2)(x + 6)(x - 5)} \\[0.3cm]
	&= \dfrac{14x - 35}{(x + 2)(x + 6)(x - 5)} \\[0.3cm]
	&= \dfrac{7(2x - 5)}{(x + 2)(x + 6)(x - 5)}
	\end{aligned}
	\]



\newpage



% Problem 4
\problem{10} Simplifying as much as possible, compute the following:
	\[
	\dfrac{x^2 - 4}{x^3 - 9x} \cdot \dfrac{x^2 - 2x - 3}{x^2 - 3x - 10}
	\] \pspace

\sol
	\[
	\begin{aligned}
	\dfrac{x^2 - 4}{x^3 - 9x} \cdot \dfrac{x^2 - 2x - 3}{x^2 - 3x - 10}&= \dfrac{(x - 2)(x + 2)}{x(x^2 - 9)} \cdot \dfrac{(x - 3)(x + 1)}{(x - 5)(x + 2)} \\[0.3cm]
	&= \dfrac{(x - 2)(x + 2)}{x(x - 3)(x + 3)} \cdot \dfrac{(x - 3)(x + 1)}{(x - 5)(x + 2)} \\[0.3cm]
	&= \dfrac{(x - 2)\cancel{(x + 2)}}{x \cancel{(x - 3)}(x + 3)} \cdot \dfrac{\cancel{(x - 3)}(x + 1)}{(x - 5)\cancel{(x + 2)}} \\[0.3cm]
	&= \dfrac{(x - 2)(x + 1)}{x (x + 3)(x - 5)}
	\end{aligned}
	\]



\newpage



% Problem 5
\problem{10} Simplifying as much as possible, compute the following:
	\[
	\dfrac{\;\;\dfrac{2x^2 + 8x}{x^2 - 6x - 7}\;\;}{\;\;\dfrac{x^2 + 9x + 20}{x^2 - 4x - 5}\;\;}
	\] \pspace

\sol
	\[
	\begin{aligned}
	\dfrac{\;\;\dfrac{2x^2 + 8x}{x^2 - 6x - 7}\;\;}{\;\;\dfrac{x^2 + 9x + 20}{x^2 - 4x - 5}\;\;}&= \dfrac{2x^2 + 8x}{x^2 - 6x - 7} \cdot \dfrac{x^2 - 4x - 5}{x^2 + 9x + 20} \\[0.3cm]
	&= \dfrac{2x(x + 4)}{(x - 7)(x + 1)} \cdot \dfrac{(x - 5)(x + 1)}{(x + 4)(x + 5)} \\[0.3cm]
	&= \dfrac{2x \cancel{(x + 4)}}{(x - 7) \cancel{(x + 1)}} \cdot \dfrac{(x - 5) \cancel{(x + 1)}}{\cancel{(x + 4)}(x + 5)} \\[0.3cm]
	&= \dfrac{2x(x - 5)}{(x - 7)(x + 5)}
	\end{aligned}
	\]



\newpage



% Problem 6
\problem{10} Simplifying as much as possible, compute the following:
	\[
	\dfrac{\;\;\dfrac{x^2 + 2x - 3}{x^2 + 11x + 10}\;\;}{\;\;\dfrac{x^2 + 8x - 9}{x^2 + 8x - 20}\;\;}
	\] \pspace

\sol
	\[
	\begin{aligned}
	\dfrac{\;\;\dfrac{x^2 + 2x - 3}{x^2 + 11x + 10}\;\;}{\;\;\dfrac{x^2 + 8x - 9}{x^2 + 8x - 20}\;\;}&= \dfrac{x^2 + 2x - 3}{x^2 + 11x + 10} \cdot \dfrac{x^2 + 8x - 20}{x^2 + 8x - 9} \\[0.3cm]
	&= \dfrac{(x + 3)(x - 1)}{(x + 1)(x + 10)} \cdot \dfrac{(x - 2)(x + 10)}{(x + 9)(x - 1)} \\[0.3cm]
	&= \dfrac{(x + 3) \cancel{(x - 1)}}{(x + 1) \cancel{(x + 10)}} \cdot \dfrac{(x - 2) \cancel{(x + 10)}}{(x + 9) \cancel{(x - 1)}} \\[0.3cm]
	&= \dfrac{(x + 3)(x - 2)}{(x + 1)(x + 9)}
	\end{aligned}
	\]



\newpage



% Problem 7
\problem{10} Fully justifying your answer, determine if the point $(-1, 3)$ is a solution to the following system of equations:
	\[
	\begin{cases}
	4x - 7y= -8 \\[0.3cm]
	-3x + 5y= 5
	\end{cases}
	\] \pspace

\sol The point $(-1, 3)$ is a solution to the system of equations if and only if it satisfies both of the equations. We check this:
	\[
	\begin{aligned}
	4x - 7y&= -8 \\
	4(-1) - 7(3)&\stackrel{?}{=} -8 \\
	-4 - 21&\stackrel{?}{=} -8 \\
	-25&\neq -8 \\
	&\;\text{\xmark}
	\end{aligned}
	\]
and
	\[
	\begin{aligned}
	-3x + 5y&= 5 \\
	-3(-1) + 5(3)&\stackrel{?}{=} 5 \\
	3 + 15&\stackrel{?}{=} 5 \\
	18&\neq 5 \\
	&\;\text{\xmark}
	\end{aligned}
	\]
Because $(-1, 3)$ satisfies neither equation, $(-1, 3)$ is not a solution to the system of equations. 



\newpage



% Problem 8
\problem{10} Show that the following system of equations has a solution:
	\[
	\begin{aligned}
	6x - 3y&= 11 \\[0.3cm]
	2x + 5y&= 12
	\end{aligned}
	\] \pspace

\sol This a system of linear equations. The system will have a solution if and only if the lines intersect. But this will only happen if they are not parallel. We find the slopes of each line:
	\[
	\begin{aligned}
	6x &- 3y= 11 &\quad\quad 2x &+ 5y= 12 \\
	-3y&= -6x + 11 & 5y&= -2x + 12 \\
	y&= 2x - \tfrac{11}{3} & y&= -\tfrac{2}{5}x + \tfrac{12}{5}
	\end{aligned}
	\]
The slope of the first line is $m_1= 2$ while the slope of the second line is $m_2= -\frac{2}{5}$. Because $m_1 \neq m_2$, the lines are not parallel. But then the lines intersect so that there is a solution to the system of equations. 



\newpage



% Problem 9
\problem{10} Solve the following system of equations and verify that your solution is valid:
	\[
	\begin{cases}
	6x + 4y= 0 \\[0.3cm]
	-12x + 6y= -7
	\end{cases}
	\] \pspace

\sol This is a system of linear equations. Assuming that there is a solution, it can be found with substitution or elimination. If we use substitution, we can solve for $y$ in the first equation. This yields $4y= -6x$ so that $y= -\frac{3}{2}\,x$. Using this in the second equation, we have\dots
	\[
	\begin{aligned}
	-12x + 6y&= -7 \\
	-12x + 6 \cdot -\frac{3}{2}\,x&= -7 \\
	-12x - 9x&= -7 \\
	-21x&= -7 \\
	x&= \dfrac{1}{3}
	\end{aligned}
	\]
But then we have $y= -\frac{3}{2}\,x= -\frac{3}{2} \cdot \frac{1}{3}= -\frac{1}{2}$. Therefore, the solution is $(\frac{1}{3}, -\frac{1}{2})$. \pspace

Using elimination, suppose we eliminate $x$. Multiplying the first equation by $2$ and adding this to the second equation, we find
	\[
	\begin{aligned}
	12x + 8y&= 0 \\
	-12x + 6y&= -7 \\ \hline
	14y&= -7 \\
	y&= -\dfrac{1}{2}
	\end{aligned}
	\] 
Using this in the first equation, we find
	\[
	\begin{aligned}
	6x + 4y&= 0 \\
	6x + 4 \cdot -\dfrac{1}{2}&= 0 \\
	6x - 2&= 0 \\
	6x&= 2 \\
	x&= \dfrac{1}{3}
	\end{aligned}
	\]
Therefore, the solution is $(\frac{1}{3}, -\frac{1}{2})$. We verify this in each of the two equations: 
	\[
	\begin{aligned}
	6x + 4y&= 0 &\hspace{2cm} -12x + 6y&= - 7 \\
	6 \cdot \tfrac{1}{3} + 4 \cdot -\tfrac{1}{2}&\stackrel{?}{=} 0 & -12 \cdot \tfrac{1}{3} + 6 \cdot -\tfrac{1}{2}&\stackrel{?}{=} -7 \\
	2 - 2&\stackrel{?}{=} 0 & -4 - 3&\stackrel{?}{=} -7 \\
	0&= 0 & -7&= -7 \\
	&\text{\;\cmark} & &\text{\;\cmark}
	\end{aligned}
	\]
Therefore, $(\frac{1}{3}, -\frac{1}{2})$ is the solution to the given system of equations. 



\newpage



% Problem 10
\problem{10} Solve the following system of equations and explain whether your solution is the only one possible:
	\[
	\begin{aligned}
	\frac{1}{2}\,x + 4y&= -1 \\[0.3cm]
	\frac{1}{3}\,x - 5y&= 7
	\end{aligned}
	\] \pspace

\sol This is a system of linear equations. Assuming that there is a solution, it can be found with substitution or elimination. If we use substitution, we can solve for $x$ in the first equation. This yields $\frac{1}{2}\,x= -4y - 1$ so that $x= -8y - 2$. Using this in the second equation, we have\dots
	\[
	\begin{aligned}
	\frac{1}{3}\,x - 5y&= 7 \\
	\frac{1}{3} \cdot (-8y - 2) - 5y&= 7 \\
	-8y - 2 - 15y&= 21 \\
	-23y&= 23 \\
	y&= -1
	\end{aligned}
	\]
But then we have $x= -8y - 2= -8(-1) - 2= 8 - 2= 6$. Therefore, the solution is $(6, -1)$. \pspace

Using elimination, suppose we eliminate $x$. Multiplying the first equation by $2$, the second equation by $-3$, and adding these equations, we find
	\[
	\begin{aligned}
	x + 8y&= -2 \\
	-x + 15y&= -21 \\ \hline
	23y&= -23 \\
	y&= -1
	\end{aligned}
	\] 
Using this in the second equation, we find
	\[
	\begin{aligned}
	\dfrac{1}{3}\,x - 5y&= 7 \\
	x - 15y&= 21 \\
	x - 15(-1)&= 21 \\
	x + 15&= 21 \\
	x&= 6
	\end{aligned}
	\]
Therefore, the solution is $(6, -1)$. We verify this in each of the two equations: 
	\[
	\begin{aligned}
	\tfrac{1}{2}\,x + 4y&= -1 &\hspace{2cm} \tfrac{1}{3}\,x - 5y&= 7 \\
	\tfrac{1}{2} \cdot 6 + 4 \cdot -1&\stackrel{?}{=} -1 & \tfrac{1}{3} \cdot 6 - 5 \cdot -1&\stackrel{?}{=} 7 \\
	3 - 4&\stackrel{?}{=} -1 & 2 + 5&\stackrel{?}{=} 7 \\
	-1&= -1 & 7&= 7 \\
	&\text{\;\cmark} & &\text{\;\cmark}
	\end{aligned}
	\]
Therefore, $(6, -1)$ is the solution to the given system of equations. 


\end{document}