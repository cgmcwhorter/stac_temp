\documentclass[11pt,letterpaper]{article}
\usepackage[lmargin=1in,rmargin=1in,tmargin=1in,bmargin=1in]{geometry}
\usepackage{../style/homework}
\usepackage{../style/commands}
\setbool{quotetype}{false} % True: Side; False: Under
\setbool{hideans}{false} % Student: True; Instructor: False

% -------------------
% Content
% -------------------
\begin{document}

\homework{8: Due 06/08}{I do not know what I may appear to the world, but to myself I seem to have been only like a boy playing on the seashore, and diverting myself in now and then finding a smoother pebble or a prettier shell than ordinary, whilst the great ocean of truth lay all undiscovered before me.}{Issac Newton}

% Problem 1
\problem{10} Use the discriminant to explain why the quadratic function $f(x)= x^2 - 4x + 13$ does not factor `nicely.' Does the function factor `nicely' over the complex numbers? Explain. \pspace

\sol The discriminant of a polynomial $ax^2 + bx + c$ is $D= b^2 - 4ac$. The polynomial factors `nicely' only if $|D|$ is a square, and factors over the real numbers if and only if $D \geq 0$. For $f(x)= x^2 - 4x + 13$, we have\dots \pspace
	\[
	D= b^2 - 4ac = (-4)^2 - 4(1)(13)= 16 - 52= -36
	\] \pspace
Because $|D|= 36= 6^2$ is a square, $f(x)$ factors `nicely.' However, because $D < 0$, we know that $f(x)$ factors `nicely' only over the complex numbers. In fact, we have\dots \pspace
	\[
	f(x)= x^2 - 4x + 13= \big(x - (2 - 3i) \big) \big(x - (2 + 3i) \big)
	\]



\newpage



% Problem 2
\problem{10} Find the factorization of $x^2 + 9x - 36$ the `traditional' way. Then use the quadratic formula to factor $x^2 + 9x - 36$. Confirm that your factorization is correct. \pspace

\sol 
	\begin{table}[!ht]
	\centering
	\underline{\bfseries 36} \pvspace{0.2cm}
	\begin{tabular}{rr}
	$1 \cdot -36$ & $-35$ \\
	$-1 \cdot 36$ $35$ \\
	$2 \cdot -18$ & $-16$ \\
	$-2 \cdot 18$ & $16$ \\
	$3 \cdot -12$ & $-9$ \\ \hline
	\multicolumn{1}{|r}{$-3 \cdot 12$} & \multicolumn{1}{r|}{$9$} \\ \hline
	$4 \cdot -9$ & $-5$ \\
	$-4 \cdot 9$ & $5$ \\
	$6 \cdot -6$ & $0$
	\end{tabular}
	\end{table}

Therefore,
	\[
	x^2 + 9x - 36= (x - 3)(x + 12)
	\]  \pspace
Alternatively, using the quadratic formula, we solve $x^2 + 9x - 36= 0$: \pspace
	\[
	\begin{aligned}
	x&= \dfrac{-b \pm \sqrt{b^2 - 4ac}}{2a} \\[0.3cm]
	x&= \dfrac{-9 \pm \sqrt{9^2 - 4(1)(-36)}}{2(1)} \\[0.3cm]
	x&= \dfrac{-9 \pm \sqrt{81 + 144}}{2} \\[0.3cm]
	x&= \dfrac{-9 \pm \sqrt{225}}{2} \\[0.3cm]
	x&= \dfrac{-9 \pm 15}{2}
	\end{aligned}
	\] \pspace
Therefore, $x= \frac{-9 + 15}{2}= \frac{6}{2}= 3$ or $x= \frac{-9 - 15}{2}= \frac{-24}{2}= -12$. We then have\dots \pspace
	\[
	x^2 + 9x - 36= a(x - r_1)(x - r_2)= 1\big(x - 3 \big) \big(x - (-12) \big)= (x - 3)(x + 12)
	\] \pspace
We can check these factorizations by expanding the factorization: \pspace
	\[
	(x - 3)(x + 12)= x^2 + 12x - 3x - 36= x^2 + 9x - 36
	\]



\newpage



% Problem 3
\problem{10} Use the quadratic formula to factor $2x^2 - 4x - 12$. \pspace

\sol We know $2x^2 - 4x - 12$ is of the form $ax^2 + bx + c$ with $a= 2$, $b= - 4$, and $c= -12$. Now we find the roots of $2x^2 - 4x - 12$, i.e. solve the equation $2x^2 - 4x - 12= 0$: \pspace
	\[
	\begin{aligned}
	x&= \dfrac{-b \pm \sqrt{b^2 - 4ac}}{2a} \\[0.3cm]
	x&= \dfrac{-(-4) \pm \sqrt{(-4)^2 - 4(2)(-12)}}{2(2)} \\[0.3cm]
	x&= \dfrac{4 \pm \sqrt{16 + 96}}{4} \\[0.3cm]
	x&= \dfrac{4 \pm \sqrt{112}}{4} \\[0.3cm]
	x&= \dfrac{4 \pm \sqrt{16 \cdot 7}}{4} \\[0.3cm]
	x&= \dfrac{4 \pm 4\sqrt{7}}{4} \\[0.3cm]
	x&= 1 \pm \sqrt{7}
	\end{aligned}
	\] \pspace
Therefore, we have\dots \pspace
	\[
	2x^2 - 4x - 12= a(x - r_1)(x - r_2)= 2 \big(x - (1 - \sqrt{7}) \big) \big(x - (1 + \sqrt{7}) \big)
	\]



\newpage



% Problem 4
\problem{10} Use the quadratic formula to factor $x^2 - 10x + 34$. \pspace

\sol We know $x^2 - 10x + 34$ is of the form $ax^2 + bx + c$ with $a= 1$, $b= -10$, and $c= 34$. Now we find the roots of $x^2 - 10x + 34$, i.e. solve the equation $x^2 - 10x + 34= 0$: \pspace
	\[
	\begin{aligned}
	x&= \dfrac{-b \pm \sqrt{b^2 - 4ac}}{2a} \\[0.3cm]
	x&= \dfrac{-(-10) \pm \sqrt{(-10)^2 - 4(1)(34)}}{2(1)} \\[0.3cm]
	x&= \dfrac{10 \pm \sqrt{100 - 136}}{2} \\[0.3cm]
	x&= \dfrac{10 \pm \sqrt{-36}}{2} \\[0.3cm]
	x&= \dfrac{10 \pm \sqrt{36}\,i}{2} \\[0.3cm]
	x&= \dfrac{10 \pm 6i}{2} \\[0.3cm]
	x&= 5 \pm 3i
	\end{aligned}
	\]  \pspace
Therefore, we have\dots \pspace
	\[
	x^2 - 10x + 34= a(x - r_1)(x - r_2)= \big(x - (5 - 3i) \big) \big(x - (5 + 3i) \big)
	\] 



\newpage



% Problem 5
\problem{10} Use the quadratic formula to factor $60x^2 - 2615x + 24200$. \pspace

\sol We know $60x^2 - 2615x + 24200$ is of the form $ax^2 + bx + c$ with $a= 60$, $b= -2615$, and $c= 24200$. Now we find the roots of $60x^2 - 2615x + 24200$, i.e. solve the equation $60x^2 - 2615x + 24200= 0$: \pspace
	\[
	\begin{aligned}
	x&= \dfrac{-b \pm \sqrt{b^2 - 4ac}}{2a} \\[0.3cm]
	x&= \dfrac{-(-2615) \pm \sqrt{(-2615)^2 - 4(60)(24200)}}{2(60)} \\[0.3cm]
	x&= \dfrac{2615 \pm \sqrt{6838225 - 5808000}}{120} \\[0.3cm]
	x&= \dfrac{2615 \pm \sqrt{1030225}}{120} \\[0.3cm]
	x&= \dfrac{2615 \pm 1015}{120} \\[0.3cm]
	x&= \dfrac{523 \pm 203}{24}
	\end{aligned}
	\] \pspace
Then $x= \dfrac{523 - 203}{24}= \dfrac{320}{24}= \dfrac{40}{3}$ or $x= \dfrac{523 + 203}{24}= \dfrac{726}{24}= \dfrac{121}{4}$. Therefore, we have\dots \pspace
	\[
	\begin{aligned}
	60x^2 - 2615x + 24200&= a(x - r_1)(x - r_2) \\[0.3cm]
	&= 60 \left(x - \dfrac{40}{3} \right) \left(x - \dfrac{121}{4} \right) \\[0.3cm]
	&= 5 \cdot 3 \left(x - \dfrac{40}{3} \right) \cdot 4 \left(x - \dfrac{121}{4} \right) \\[0.3cm]
	&= 5(3x - 40)(4x - 121)
	\end{aligned}
	\] 



\newpage



% Problem 6
\problem{10} Showing all your work, solve the following equation:
	\[
	9x - x^2= -10
	\] \pspace

\sol
	\[
	\begin{aligned}
	9x - x^2&= -10 \\[0.3cm]
	x^2 - &9x - 10= 0 \\[0.3cm]
	(x + 1)&(x - 10)= 0 \\[0.3cm]
	x + 1= 0 &\text{  or  } x - 10= 0 \\[0.3cm]
	x= -1 &\text{  or  } x= 10 
	\end{aligned}
	\]



\newpage



% Problem 7
\problem{10} Showing all your work, solve the following equation:
	\[
	2(x^2 - 3)= -11x
	\] \pspace

\sol
	\[
	\begin{aligned}
	2(x^2 - 3)&= -11x \\[0.3cm]
	2x^2 - 6&= -11x \\[0.3cm]
	2x^2 + &11x - 6= 0 \\[0.3cm]
	(2x - 1)&(x + 6)= 0 \\[0.3cm]
	2x - 1= 0 &\text{  or  } x + 6= 0 \\[0.3cm]
	x= \tfrac{1}{2} &\text{  or  } x= -6
	\end{aligned}
	\]



\newpage



% Problem 8
\problem{10} Showing all your work, solve the following equation:
	\[
	x^2= 6x - 7
	\] \pspace

\sol We have\dots \pspace
	\[
	\begin{aligned}
	x^2&= 6x - 7 \\[0.3cm]
	x^2 - 6x + 7&= 0 
	\end{aligned}
	\] \pspace
We know $x^2 - 6x + 7$ has the form $ax^2 + bx + c$ with $a= 1$, $b= -6$, and $c= 7$. Then using the quadratic formula, we then have\dots \pspace
	\[
	\begin{aligned}
	x&= \dfrac{-b \pm \sqrt{b^2 - 4ac}}{2a} \\[0.3cm]
	x&= \dfrac{-(-6) \pm \sqrt{(-6)^2 - 4(1)(7)}}{2(1)} \\[0.3cm]
	x&= \dfrac{6 \pm \sqrt{36 - 28}}{2} \\[0.3cm]
	x&= \dfrac{6 \pm \sqrt{8}}{2} \\[0.3cm]
	x&= \dfrac{6 \pm \sqrt{4 \cdot 2}}{2} \\[0.3cm]
	x&= \dfrac{6 \pm 2\sqrt{2}}{2} \\[0.3cm]
	x&= 3 \pm \sqrt{2}
	\end{aligned}
	\] 



\newpage



% Problem 9
\problem{10} Showing all your work, solve the following equation:
	\[
	x(2 - x)= 2
	\] \pspace

\sol We have\dots \pspace
	\[
	\begin{aligned}
	x(2 - x)&= 2 \\[0.3cm]
	2x - x^2&= 2 \\[0.3cm]
	x^2 - 2x + 2&= 0 
	\end{aligned}
	\] \pspace
We know $x^2 - 2x + 2$ has the form $ax^2 + bx + c$ with $a= 1$, $b= -2$, and $c= 2$. Then using the quadratic formula, we then have\dots \pspace
	\[
	\begin{aligned}
	x&= \dfrac{-b \pm \sqrt{b^2 - 4ac}}{2a} \\[0.3cm]
	x&= \dfrac{-(-2) \pm \sqrt{(-2)^2 - 4(1)(2)}}{2(1)} \\[0.3cm]
	x&= \dfrac{2 \pm \sqrt{4 - 8}}{2} \\[0.3cm]
	x&= \dfrac{2 \pm \sqrt{-4}}{2} \\[0.3cm]
	x&= \dfrac{2 \pm \sqrt{4}\,i}{2} \\[0.3cm]
	x&= \dfrac{2 \pm 2i}{2} \\[0.3cm]
	x&= 1 \pm i 
	\end{aligned}
	\] 



\newpage


% Problem 10
\problem{10} Showing all your work, solve the following equation:
	\[
	\dfrac{x + 1}{x - 3}= x + 1
	\] \pspace

\sol
	\[
	\begin{aligned}
	\dfrac{x + 1}{x - 3}&= x + 1 \\[0.3cm]
	x + 1&= (x - 3)(x + 1) \\[0.3cm]
	x + 1&= x^2 + x - 3x - 3 \\[0.3cm]
	x + 1&= x^2 - 2x - 3 \\[0.3cm]
	x^2 - &3x - 4= 0 \\[0.3cm]
	(x + 1)&(x - 4)= 0 \\[0.3cm]
	x + 1= 0 &\text{  or  } x - 4= 0 \\[0.3cm]
	x= -1 &\text{  or  } x= 4 
	\end{aligned}
	\]


\end{document}