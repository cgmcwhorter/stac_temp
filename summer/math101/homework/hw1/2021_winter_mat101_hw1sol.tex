\documentclass[11pt,letterpaper]{article}
\usepackage[lmargin=1in,rmargin=1in,tmargin=1in,bmargin=1in]{geometry}
\usepackage{../style/homework}
\usepackage{../style/commands}
\setbool{quotetype}{false} % True: Side; False: Under
\setbool{hideans}{false} % Student: True; Instructor: False

% -------------------
% Content
% -------------------
\begin{document}

\homework{1: Due 01/04}{I knew exactly what to do. But in a much more real sense, I had no idea what to do.}{Michael Scott, The Office}

% Problem 1
\problem{10} Showing all your work, compute each of the following:
\begin{enumerate}[(a)]
\item $20 - 5(3 - 6)$ 
\item $30/3 - 3 + (-1)^3$
\item $5(1 - 3)^2/4$
\item $4 - 2(3^2 - 10)$
\item $\dfrac{10 - 2^2}{5 - 6} + 10/2 - 5$
\end{enumerate} 

\sol
\begin{enumerate}[(a)]
\item 
	\[
	\begin{aligned}
	20 - 5(3 - 6)&= 20 - 5(-3)= 20 - (-15)= 20 + 15= 35
	\end{aligned}
	\]

\item 
	\[
	\begin{aligned}
	30/3 - 3 + (-1)^3&= 30/3 - 3 - 1 \\
	&= 10 - 3 - 1 \\
	&= 7 - 1 \\
	&= 6
	\end{aligned}
	\]

\item 
	\[
	\begin{aligned}
	5(1 - 3)^2/4&= 5(-2)^2/4 \\
	&= 5(4)/4 \\
	&= 20/4 \\
	&= 5
	\end{aligned}
	\]

\item 
	\[
	\begin{aligned}
	4 - 2(3^2 - 10)&= 4 - 2(9 - 10) \\
	&= 4 - 2(-1) \\
	&= 4 - (-2) \\
	&= 4 + 2 \\
	&= 6
	\end{aligned}
	\]

\item 
	\[
	\begin{aligned}
	\dfrac{10 - 2^2}{5 - 6} + 10/2 - 5&= \dfrac{10 - 4}{5 - 6} + 10/2 \\
	&= \dfrac{6}{-1} + 10/2 - 5 \\
	&= -6 + 10/2 - 5 \\
	&= -6 + 5 - 5 \\
	&= -1 - 5 \\
	&= -6
	\end{aligned}
	\]
\end{enumerate}



\newpage



% Problem 2
\problem{10} Showing all your work, compute each of the following:
\begin{enumerate}[(a)]
\item $6/2(1 + 2)$
\item $6/(2(1+2))$
\item $\dfrac{10^2/5 - 6 + 4 \cdot 2}{(5 + 1)(3 - 8)}$
\item $\dfrac{10 - 6}{-2^2} + 12$
\item $5(-1)^3 - 4(-1)^3 + 3 \cdot 16/4$
\end{enumerate} \pspace

\sol
\begin{enumerate}[(a)]
\item 
	\[
	6/2(1 + 2)= 6/2(3)= 3(3)= 9
	\]

\item 
	\[
	6/(2(1+2))= 6/(2(3))= 6/6= 1
	\]

\item 
	\[
	\begin{aligned}
	\dfrac{10^2/5 - 6 + 4 \cdot 2}{(5 + 1)(3 - 8)}&= \dfrac{10^2/5 - 6 + 4 \cdot 2}{6 \cdot -5} \\
	&= \dfrac{100/5 - 6 + 4 \cdot 2}{6 \cdot -5} \\
	&= \dfrac{20 - 6 + 8}{-30} \\
	&= \dfrac{22}{-30} \\
	&= -\dfrac{11}{15}
	\end{aligned}
	\]

\item 
	\[
	\begin{aligned}
	\dfrac{10 - 6}{-2^2} + 12&= \dfrac{10 - 6}{-4} + 12 \\
	&= \dfrac{4}{-4} + 12 \\
	&= -1 + 12 \\
	&= 11
	\end{aligned}
	\]

\item 
	\[
	\begin{aligned}
	5(-1)^3 - 4(-1)^3 + 3 \cdot 16/4&= 5(-1) - 4(-1) + 3 \cdot 16/4 \\
	&= -5 - (-4) + 48/4 \\
	&= -5 - (-4) + 12 \\
	&= -5 + 4 + 12 \\
	&= -1 + 12 \\
	&= 11
	\end{aligned}
	\]
\end{enumerate}



\newpage



% Problem 3 
\problem{10} Showing all your work, find the prime factorizations of the following integers:
\begin{enumerate}[(a)]
\item $60$
\item $61$
\item $132$
\item $125$
\item $147$
\end{enumerate} \pspace

\sol
\begin{enumerate}[(a)]
\item $60= 2^2 \cdot 3 \cdot 5$
\item $61= 61^1$
\item $132= 2^2 \cdot 3 \cdot 11$
\item $125= 5^3$
\item $147= 3 \cdot 7^2$
\end{enumerate}



\newpage



% Problem 4
\problem{10} Using the relationship between factors of integers and their square root, explain why the integer 79 is prime. \pspace

\sol We know that if a number $n$ is composite, then it must have a prime factor between 1 and $\sqrt{n}$. We know that $\sqrt{79} \approx 8.8$. [Even if we did not know this, we know that $8^2= 64$ and $9^2= 81$ so that $8 < \sqrt{79} < 9$.] Therefore, if 79 were composite, it would have a factor between 1 and 8. But none of the numbers 2, 3, 5, 7 (the primes at most 8) divide 79. Therefore, it must be that 79 is prime. 



\newpage



% Problem 5
\problem{10} Compute each of the following by finding the divisors/multiples of the given integers:
\begin{enumerate}[(a)]
\item $\gcd(12, 20)$
\item $\gcd(18, 27)$
\item $\lcm(12, 30)$
\item $\lcm(8, 15)$
\end{enumerate} \pspace

\sol
\begin{enumerate}[(a)]
\item \phantom{.}\par
	\begin{table}[!ht]
	\centering
	\begin{tabular}{rl}
	12: & 1, 2, 3, \textbf{4}, 6, 12 \\
	20: & 1, 2, \textbf{4}, 5, 10, 20 
	\end{tabular}
	\end{table}

	\[
	\gcd(12, 20)= 4 
	\] \pspace

\item \phantom{.}\par
	\begin{table}[!ht]
	\centering
	\begin{tabular}{rl}
	18: & 1, 2, 3, 6, \textbf{9}, 18 \\
	27: & 1, 3, \textbf{9}, 27
	\end{tabular}
	\end{table}

	\[
	\gcd(18, 27)= 9
	\] \pspace

\item \phantom{.}\par
	\begin{table}[!ht]
	\centering
	\begin{tabular}{rl}
	12: & 12, 24, 36, 48, \textbf{60}, 72, 84, 96 \\
	30: & 30, \textbf{60} 90, 120, 150, 180, 210, 240
	\end{tabular}
	\end{table}

	\[
	\lcm(12, 30)= 60 
	\] \pspace

\item \phantom{.}\par
	\begin{table}[!ht]
	\centering
	\begin{tabular}{rl}
	8: & 8, 16, 24, 32, 40, 48, 56, 64, 72, 80, 88, 96, 104, 112, \textbf{120}, 128 \\
	15: & 15, 30, 45, 60, 75, 90, 105, \textbf{120}, 135
	\end{tabular}
	\end{table}

	\[
	\lcm(8, 15)= 120
	\]
\end{enumerate}



\newpage



% Problem 6
\problem{10} Use the prime factorizations of the given integers to compute each of the following:
\begin{enumerate}[(a)]
\item $\gcd(124, 144)$
\item $\lcm(128, 146)$
\item $\gcd(2^3 \cdot 3 \cdot 7^8 \cdot 17^6, 2^2 \cdot 3^5 \cdot 5^2 \cdot 11^9)$
\item $\lcm(2^3 \cdot 3 \cdot 7^8 \cdot 17^6, 2^2 \cdot 3^5 \cdot 5^2 \cdot 11^9)$
\end{enumerate} \pspace

\begin{enumerate}[(a)]
\item 
	\[
	\gcd(124, 144)= \gcd(2^2 \cdot 31, 2^2 \cdot 3^2)= 2^2= 4
	\]

\item 
	\[
	\lcm(128, 146)= \lcm(2^7, 2^1 \cdot 73)= 2^7 \cdot 73= 9344
	\]

\item 
	\[
	\gcd(2^3 \cdot 3 \cdot 7^8 \cdot 17^6, 2^2 \cdot 3^5 \cdot 5^2 \cdot 11^9)= 2^2 \cdot 3^1= 12 
	\]

\item 
	\[
	\lcm(2^3 \cdot 3 \cdot 7^8 \cdot 17^6, 2^2 \cdot 3^5 \cdot 5^2 \cdot 11^9)= 2^3 \cdot 3^5 \cdot 5^2 \cdot 7^8 \cdot 11^9 \cdot 17^6= 15\;945\;872\;383\;623\;188\;299\;435\;619\;400
	\]
\end{enumerate}



\newpage



% Problem 7
\problem{10} Lena is making gift baskets for an event. Each basket will contain at least one jam jar, candy box, and gift card. Suppose there are 120 jam jars, 280 candy boxes, and 360 gift cards with which to make the baskets. 
\begin{enumerate}[(a)]
\item If she is trying to make the most number of gift baskets, how many can she make? Explain. 
\item If she is trying to put the most number of items in each basket with each basket having the same number of each item, how many baskets can she make? Explain. 
\item If she were making gift baskets that each contained 4 jam jars, 1 candy box, and 6 gift cards each, what is the fewest number of each item she would have to purchase if she was going to purchase an equal number of each item.
\end{enumerate} \pspace

\sol
\begin{enumerate}[(a)]
\item Because each basket must have at least one of each item, you can only make as many gift baskets as the least numerous among those three items. Because there are only 120 jam jars, Lena can make at most 120 gift baskets. \pspace

\item Suppose there are $N$ of each item in the gift basket. Then the total amount of jam jars, candy boxes, and gift cards must be a multiple of $N$. Therefore, $N$ is a divisor of 120, 280, and 360. If we want $N$ to be as large as possible, then $N$ will be the gcd of 120, 280, and 360. But $\gcd(120, 280, 360)= 40$. But then she can only make $120/40= 3$ gift baskets. \pspace

\item Because Lena will purchase an equal amount of each item and she wants to have none left over at the end, it must be that the total number of items purchased is a multiple of 4, 1, and 6. To purchase the fewest number of each item, this must be the smallest multiple of 4, 1, and 6, i.e. the lcm. But we know $\lcm(4, 1, 6)= 12$. Therefore, she should purchase 12 of each item. 
\end{enumerate}



\newpage



% Problem 8 
\problem{10} Showing all your work and being sure to reduce as much as possible, compute the following:
\begin{enumerate}[(a)]
\item $\dfrac{4}{5} + \dfrac{9}{7}$
\item $\dfrac{9}{4} - \dfrac{7}{12}$
\item $\dfrac{6}{35} + \dfrac{4}{15}$
\item $\dfrac{17}{24} - \dfrac{13}{40}$
\item $4\frac{1}{2} + 3\frac{1}{3}$
\end{enumerate} \pspace

\sol
\begin{enumerate}[(a)]
\item 
	\[
	\dfrac{4}{5} + \dfrac{9}{7}= \dfrac{28}{35} + \dfrac{45}{35}= \dfrac{28 + 45}{35}= \dfrac{73}{35}
	\] \pspace

\item
	\[
	\dfrac{9}{4} - \dfrac{7}{12}= \dfrac{27}{12} - \dfrac{7}{12}= \dfrac{27 - 7}{12}= \dfrac{20}{12}= \dfrac{5}{3}
	\] \pspace
 
\item 
	\[
	\dfrac{6}{35} + \dfrac{4}{15}= \dfrac{18}{105} + \dfrac{28}{105}= \dfrac{18 + 28}{105}= \dfrac{46}{105}
	\] \pspace

\item 
	\[
	\dfrac{17}{24} - \dfrac{13}{40}= \dfrac{85}{120} - \dfrac{39}{120}= \dfrac{85 - 39}{120}= \dfrac{46}{120}= \dfrac{23}{60}
	\] \pspace

\item 
	\[
	4\frac{1}{2} + 3\frac{1}{3}= \dfrac{8 + 1}{2} + \dfrac{9 + 1}{3}= \dfrac{9}{2} + \dfrac{10}{3}= \dfrac{27}{6} + \dfrac{20}{6}= \dfrac{27 + 20}{6}= \dfrac{47}{6}
	\]
\end{enumerate}



\newpage



% Problem 9
\problem{10} Showing all your work and being sure to reduce as much as possible, compute the following:
\begin{enumerate}[(a)]
\item $\dfrac{9}{10} \cdot \dfrac{-6}{33}$
\item $\dfrac{12}{35} \cdot \dfrac{25}{2}$
\item $\dfrac{100}{3} \cdot \dfrac{27}{25}$
\item $\dfrac{-11}{68} \cdot \dfrac{4}{55}$
\item $2 \frac{2}{3} \cdot -4 \frac{5}{8}$
\end{enumerate} \pspace

\sol
\begin{enumerate}[(a)]
\item 
	\[
	\dfrac{9}{10} \cdot \dfrac{-6}{33}= \dfrac{\cancel{9}^3}{\cancel{10}^5} \cdot \dfrac{\cancel{-6}^{-3}}{\cancel{33}^{11}}= \dfrac{3}{5} \cdot \dfrac{-3}{11}= -\dfrac{9}{55}
	\] \pspace

\item 
	\[
	\dfrac{12}{35} \cdot \dfrac{25}{2}= \dfrac{\cancel{12}^6}{\cancel{35}^7} \cdot \dfrac{\cancel{25}^5}{\cancel{2}^1}= \dfrac{6}{7} \cdot \dfrac{5}{1}= \dfrac{30}{7}
	\] \pspace

\item 
	\[
	\dfrac{100}{3} \cdot \dfrac{27}{25}= \dfrac{\cancel{100}^4}{\cancel{3}^1} \cdot \dfrac{\cancel{27}^9}{\cancel{25}^1}= \dfrac{4}{1} \cdot \dfrac{9}{1}= \dfrac{36}{1}= 36
	\] \pspace

\item 
	\[
	\dfrac{-11}{68} \cdot \dfrac{4}{55}= \dfrac{\cancel{-11}^{-1}}{\cancel{68}^{17}} \cdot \dfrac{\cancel{4}^1}{\cancel{55}^5}= \dfrac{-1}{17} \cdot \dfrac{1}{5}= -\dfrac{1}{85}
	\] \pspace

\item 
	\[
	2 \frac{2}{3} \cdot -4 \frac{5}{8}= \dfrac{8}{3} \cdot \dfrac{-37}{8}= \dfrac{\cancel{8}^1}{3} \cdot \dfrac{-37}{\cancel{8}^1}= \dfrac{1}{3} \cdot \dfrac{-37}{1}= -\dfrac{37}{3}
	\]
\end{enumerate}



\newpage



% Problem 10 
\problem{10} Showing all your work and being sure to reduce as much as possible, compute the following:
\begin{enumerate}[(a)]
\item $\dfrac{\dfrac{10}{21}}{\dfrac{6}{7}}$
\item $\dfrac{\dfrac{5}{4}}{\dfrac{3}{10}}$
\item $\dfrac{-\dfrac{9}{44}}{\dfrac{84}{11}}$
\item $\dfrac{\dfrac{3}{5}}{\dfrac{10}{9}}$
\item $\dfrac{12\frac{1}{3}}{3\frac{1}{2}}$
\end{enumerate} \pspace

\sol {\footnotesize
\begin{enumerate}[(a)]
\item 
	\[
	\dfrac{\dfrac{10}{21}}{\dfrac{6}{7}}= \dfrac{10}{21} \cdot \dfrac{7}{6}= \dfrac{\cancel{10}^5}{\cancel{21}^3} \cdot \dfrac{\cancel{7}^1}{\cancel{6}^3}= \dfrac{5}{3} \cdot \dfrac{1}{3}= \dfrac{5}{9}
	\] \pspace

\item 
	\[
	\dfrac{\dfrac{5}{4}}{\dfrac{3}{10}}= \dfrac{5}{\cancel{4}^2} \cdot \dfrac{\cancel{10}^5}{3}= \dfrac{5}{2} \cdot \dfrac{5}{3}= \dfrac{25}{6}
	\] \pspace

\item 
	\[
	\dfrac{-\dfrac{9}{44}}{\dfrac{84}{11}}= -\dfrac{9}{44} \cdot \dfrac{11}{84}= -\dfrac{\cancel{9}^3}{\cancel{44}^4} \cdot \dfrac{\cancel{11}^1}{\cancel{84}^{28}}= -\dfrac{3}{4} \cdot \dfrac{1}{28}= -\dfrac{3}{112}
	\] \pspace

\item 
	\[
	\dfrac{\dfrac{3}{5}}{\dfrac{10}{9}}= \dfrac{3}{5} \cdot \dfrac{9}{10}= \dfrac{27}{50}
	\] \pspace

\item 
	\[
	\dfrac{12\frac{1}{3}}{3\frac{1}{2}}= \dfrac{\dfrac{37}{3}}{\dfrac{7}{2}}= \dfrac{37}{3} \cdot \dfrac{2}{7}= \dfrac{74}{21}
	\]
\end{enumerate}
}


\end{document}