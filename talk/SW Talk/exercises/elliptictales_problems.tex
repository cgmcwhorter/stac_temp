\documentclass[11pt,letterpaper]{article}
\usepackage[lmargin=1in,rmargin=1in,bmargin=1in,tmargin=1in]{geometry}
\usepackage{style}

\setlength{\parindent}{0ex}

% -------------------
% Content
% -------------------
\begin{document}

NOTE

% Use Rational Roots Theorem
% Find solutions to ax + by= g

% show ax + by= c$ has infinitely many rational solutions, parametrize all the solutions

% show only solutiosn y^2= x^3 - 2 are (3, \pm 5)
	% Z(sqrt(-2)) is a UFD
	% Units are \pm 1
	% Rewrite to make use of the factorization $r^2 + 2= (r + sqrt(-2)) (r - sqrt(-2))
	% Show these terms relatively prime [Consider irreducible dividing them both
	% Explain why y + sqrt(-2) is a cube of a product of irreducibles
	% Show that y + sqrt(-2)= a^3 - 6ab^2 + (3a^2b - 2b^3) sqrt(-2) for some integer a, b
	% Show that this is implies only solutions (3, \pm 5)

% rational parametrize circle, use to integrate secant

% Show no rational solutions to x^2 + y^2= 3. [Clear denominators, work mod 4] How does this differ to the above method?

% apply falting's theorem


% cubes mod 9: 0, 1, -1 (i.e. 8) ---> 4, 5, 13, 14 cannot be written as sum of three cubes. Is 18 sum of cubes? [-1, -2, 3]
% x = 9b^4, y= 3b - 9b^4, z= 1- 9b^3 give infinite class of representation of 1, K. Mahler 1936, give expression for cube.
















\end{document}