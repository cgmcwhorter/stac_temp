\documentclass[11pt,letterpaper]{article}
\usepackage[lmargin=1in,rmargin=1in,bmargin=1in,tmargin=1in]{geometry}
\usepackage{quiz}


% -------------------
% Content
% -------------------
\begin{document}
\thispagestyle{title}

% Quiz 1
\quizsol \textit{True/False}: The expression $12 \div 6 \cdot 2 + (-1)^3$ is the same as $\frac{12}{6 \cdot 2} + (-1)^3$ and both are equal to $0$. \pspace

\sol The statement is \textit{false}. We can compute both, following order of operations (PEMDAS, applied carefully left-to-right), and show that the expressions evaluate to different values:
	\[
	\begin{gathered}
	12 \div 6 \cdot 2 + (-1)^3 \\
	12 \div 6 \cdot 2 - 1 \\
	2 \cdot 2 - 1 \\
	4 - 1 \\
	3
	\end{gathered} \hspace{4cm}
	\begin{gathered}
	\tfrac{12}{6 \cdot 2} + (-1)^3 \\
	\tfrac{12}{6 \cdot 2} - 1 \\
	\tfrac{12}{12} - 1 \\
	1 - 1 \\
	0
	\end{gathered}
	\]
For these two expressions to be the same, the first needs a set of parentheses around the $6 \cdot 2$: $12 \div (6 \cdot 2) + (-1)^3$.



















\end{document}