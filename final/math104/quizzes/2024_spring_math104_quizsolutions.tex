\documentclass[11pt,letterpaper]{article}
\usepackage[lmargin=1in,rmargin=1in,bmargin=1in,tmargin=1in]{geometry}
\usepackage{quiz}

\DeclareMathOperator{\disc}{disc}	% Discriminant

\usepackage{pifont}				% Use Check/X mark
\newcommand{\cmark}{\ding{51}}	% Checkmark
\newcommand{\xmark}{\ding{55}}	% X-mark

% -------------------
% Content
% -------------------
\begin{document}
\thispagestyle{title}

% Quiz 1
\quizsol \textit{True/False}: The expression $12 \div 6 \cdot 2 + (-1)^3$ is the same as $\frac{12}{6 \cdot 2} + (-1)^3$ and both are equal to $0$. \pspace

\sol The statement is \textit{false}. We can compute both, following order of operations (PEMDAS, applied carefully left-to-right), and show that the expressions evaluate to different values:
	\[
	\begin{gathered}
	12 \div 6 \cdot 2 + (-1)^3 \\
	12 \div 6 \cdot 2 - 1 \\
	2 \cdot 2 - 1 \\
	4 - 1 \\
	3
	\end{gathered} \hspace{4cm}
	\begin{gathered}
	\tfrac{12}{6 \cdot 2} + (-1)^3 \\
	\tfrac{12}{6 \cdot 2} - 1 \\
	\tfrac{12}{12} - 1 \\
	1 - 1 \\
	0
	\end{gathered}
	\]
For these two expressions to be the same, the first needs a set of parentheses around the $6 \cdot 2$: $12 \div (6 \cdot 2) + (-1)^3$. \pvspace{1.3cm}



% Quiz 2
\quizsol \textit{True/False}: The point $(2, -6)$ is in the third quadrant and is a distance of 2 away from the $x$-axis and a distance of $6$ away from the $y$-axis. \pspace

\sol The statement is \textit{false}. Because $x= 2 > 0$ and $y= -6 < 0$, the point $(2, -6)$ is in Quadrant~IV. Moreover, plotting the point $(2, -6)$, we can see that the point is a distance of $|2|= 2$ away from the $x$-axis and a distance of $|-6|= 6$ away from the $y$-axis. 
	\[
	\fbox{
	\begin{tikzpicture}[scale=1,every node/.style={scale=0.5}]
	\begin{axis}[
	grid=both,
	axis lines=middle,
	ticklabel style={fill=blue!5!white},
	xmin= -10.5, xmax=10.5,
	ymin= -10.5, ymax=10.5,
	xtick={-10,-8,-6,-4,-2,0,2,4,6,8,10},
	ytick={-10,-8,-6,-4,-2,0,2,4,6,8,10},
	minor tick = {-10,-9,...,10},
	xlabel=\(x\),ylabel=\(y\),
	]
	\draw[dotted,thick,red] (2,-6) -- (0,-6);
	\draw[dotted,thick,red] (2,-6) -- (2,0);
	\draw[fill=black] (2,-6) circle (0.2);
	\end{axis}
	\end{tikzpicture} 
	}
	\] \pvspace{1.3cm}



% Quiz 3
\quizsol \textit{True/False}: The average rate of change of a function is the quotient of the change in output by the change in input. If the average rate of change is positive, the function increased. If the average rate of change is negative, the function decreased. \pspace

\sol The statement is \textit{true}. We know that the average rate of change of a function $f(x)$ over the interval $[a, b]$ is given by $\frac{f(b) - f(a)}{b - a}$. Because $x$'s are the inputs and $f(x)$'s are the outputs, this is $\frac{\Delta \text{output}}{\Delta \text{input}}$. Now because $b > a$, we know that $b - a > 0$. So the sign of the average rate of change depends only on the sign of $f(b) - f(a)$. If $f(b) - f(a) > 0$, then $f(b) > f(a)$. But then the function increased in value from the value at $x= a$ to the value at $x= b$. If $f(b) - f(a) < 0$, then $f(b) < f(a)$. But then the function decreased in value from the value at $x= a$ to the value at $x= b$. Neither imply that the function was increasing or decreasing, respectively, over the \textit{entire} interval $[a, b]$. \pvspace{1.3cm}



% Quiz 4
\quizsol \textit{True/False}: The average rate of change of a function $f(x)$ on an interval $[a, b]$ is the slope of the secant through the points $\big(a, f(a) \big)$ and $\big(b, f(b) \big)$. \pspace

\sol The statement is \textit{true}. We know that the average rate of change of a function $f(x)$ over the interval $[a, b]$ is given by $\frac{f(b) - f(a)}{b - a}$. Given the points $\big(a, f(a) \big)$ and $\big(b, f(b) \big)$, the slope of the line through them is $m= \frac{\Delta y}{\Delta x}=  \frac{f(b) - f(a)}{b - a}$. 
	\[
	\fbox{
	\begin{tikzpicture}[scale=1,every node/.style={scale=0.5}]
	\begin{axis}[
	grid=both,
	axis lines=middle,
	ticklabel style={fill=blue!5!white},
	xmin= -10.5, xmax=10.5,
	ymin= -10.5, ymax=10.5,
	xtick={-10,-8,-6,-4,-2,0,2,4,6,8,10},
	ytick={-10,-8,-6,-4,-2,0,2,4,6,8,10},
	minor tick = {-10,-9,...,10},
	xlabel=\(x\),ylabel=\(y\),
	]
	\addplot[line width=0.03cm, domain= -2:6,samples=100] ({x},{x^3 - 6*x^2 + 4*x + 8});
	\addplot[line width=0.03cm, domain= -9:10.5,samples=2,red] ({x},{x - 2});
	\draw[fill=black,blue] (-1,-3) circle (0.2);
	\draw[fill=black,blue] (5,3) circle (0.2);
	\end{axis}
	\end{tikzpicture} 
	}
	\] \pvspace{1.3cm}



% Quiz 5
\quizsol \textit{True/False}: If $P(t)= 5300t - 1480$ is a linear model representing the population in a small town $t$~years from now, then $m= 5300 > 0$ means that the model says that the population is growing at a rate of 5,300 people per year. Furthermore, the $y$-intercept $b= 1480$ represents the exact initial population of the town. \pspace

\sol The statement is \textit{false}. We know that the function $P(t)$ is linear, i.e. the graph of $P(t)$ is a line. We know that $m= 5300$. Recalling that $m= \frac{\Delta P}{\Delta t}$ and writing $5300= \frac{5300}{1}$, we can see that for every one increase in $t$ results in a 5300 increase in $P$, i.e. the population is growing at a rate of 5,300~people per year. Of course, this is just what the model says happens, on average. The $y$-intercept is $P(0)= 5300(0) - 1480= -1480$. But then $b \neq 1480$. Because $b= P(0)= -1480 < 0$, this cannot possibly represent a population. Furthermore, $b= P(0)$ need not represent the \textit{exact} initial population, merely what the model predicts is the initial population. \pvspace{1.3cm}



% Quiz 6
\quizsol \textit{True/False}: The line $y= 5 - 3x$ and the line through $(0, 5)$ with slope $-3$ must be the same line. \pspace

\sol The statement is \textit{true}. There is a unique line with a specified slope through any point, i.e. all that is needed to determine a line is a point and a slope. The line $y= 5 - 3x$ has slope $-3$. Furthermore, because $5 - 3(0)= 5$, the line $y= 5 - 3x$ contains the point $(0, 5)$. Therefore, these lines must be the same. Alternatively, we can find the line through $(0, 5)$ with slope $-3$ using the point-slope form: $y= y_0 + m(x - x_0)= 5 - 3(x - 0)= 5 - 3x$. One can also use the fact that $(0, 5)$ is a $y$-intercept so that this forces $b= 5$. Because the line has slope $-3$, we must have $y= -3x + 5= 5 - 3x$. In either case, both lines are $y= 5 - 3x$. \pvspace{1.3cm}



% Quiz 7
\quizsol \textit{True/False}: The quadratic function $f(x)= 5 - (x + 6)^2$ opens upwards and has vertex $(6, 5)$. \pspace

\sol The statement is \textit{false}. Recall that the vertex form of a quadratic function is $a(x - P)^2 + Q$, where $a$ is the $a$ from the standard form $ax^2 + bx + c$ and $(P, Q)$ is the vertex. We have $f(x)= 5 - (x + 6)^2= -\big(x - (-6) \big)^2 + 5$. Therefore, $a= -1 < 0$ so that the quadratic function opens downwards. Furthermore, the vertex is $(-6, 5)$. The given solution incorrectly identifies the $a$-value and vertex. \pvspace{1.3cm}



% Quiz 8
\quizsol \textit{True/False}: $\dfrac{2.2 \cdot 10^8}{5.5 \cdot 10^{-3}}= 0.4 \cdot 10^{11}$ \pspace

\sol To compute products and quotients of real numbers in scientific notation, one need compute the product/quotient of the mantissa/significand and the exponential portion and then make an necessary adjustments to the power based on the resulting mantissa/significand. We then have\dots
	\[
	\dfrac{2.2 \cdot 10^8}{5.5 \cdot 10^{-3}}= \dfrac{2.2}{5.5} \cdot \dfrac{10^8}{10^{-3}}= 0.4 \cdot 10^{8 - (-3)}= 0.4 \cdot 10^{11}= 4.0 \cdot 10^{12}
	\] 
The given solution forgets to correctly place the resulting real number in scientific notation by adjusting the power after dividing the mantissas. \pvspace{1.3cm}



% Quiz 9
\quizsol \textit{True/False}: $\dfrac{x^5}{(x^2)^4}= x^{-3}$ \pspace

\sol The statement is \textit{true}. Recall that $x^a x^b= x^{a+b}$, $\frac{x^a}{x^b}= x^{a-b}$, and $(x^a)^b= x^{ab}$. But then\dots
	\[
	\dfrac{x^5}{(x^2)^4}= \dfrac{x^5}{x^8}= x^{-3}
	\]
Of course, $x^{-3}= \frac{1}{x^3}$. \pvspace{1.3cm}



% Quiz 10
\quizsol \textit{True/False}: $\sqrt[5]{x^{10}}= x^{10/5}= x^2$ \pspace

\sol The statement is \textit{true}. Recall that $\sqrt[b]{x^a}= \left( \sqrt[b]{x} \right)^a= x^{a/b}$. But then\dots
	\[
	\sqrt[5]{x^{10}}= x^{10/5}= x^2
	\] \pvspace{1.3cm}



% Quiz 11
\quizsol \textit{True/False}: If $a, b$ are constants, then $f(x)= x^3 + ax + b$ is a polynomial and it is possible for $f(x)$ to have four zeros. \pspace

\sol The statement is \textit{false}. Recall the Fundamental Theorem of Algebra states that a nonconstant polynomial of degree $n$ has at most $n$ zeros (and exactly $n$ zeros if only allows complex numbers and counts with multiplicity). The degree of $f(x)$ is three. Therefore, it is not possible for $f(x)$ to have four zeros. \pvspace{1.3cm}



% Quiz 12
\quizsol \textit{True/False}: Factoring $16x^4 - 1$ as completely as possible (using rational numbers) results in the factorization $(4x^2 - 1)(4x^2 + 1)$. \pspace

\sol The statement is \textit{false}. Recall the factorization of the difference of perfect squares: $a^2 - b^2= (a - b)(a + b)$. Observe that $1^2= 1$, $(4x^2)^2= 16x^4$, and $(2x)^2= 4x^2$ are perfect squares. But then\dots
	\[
	16x^4 - 1= (4x^2 - 1)(4x^2 + 1)= (2x - 1)(2x + 1)(4x^2 + 1)
	\]
The error in the statement of the quiz is failing to recognize that the given factorization can be further factorized by recognizing $4x^2 - 1$ as a difference of perfect squares. \pvspace{1.1cm}



% Quiz 13
\quizsol \textit{True/False}: Let $f(x)$ be a quadratic function. The function $f(x)$ will factor `nicely' if and only if $\text{disc } f(x)$ is a perfect square. \pspace

\sol The statement is \textit{true}. Let $f(x)= ax^2 + bx + c$ be a quadratic function. The discriminant of $f(x)$ is $\disc f= b^2 - 4ac$. The function $f(x)$ will factor `nicely' if and only if $\disc f$ is a perfect square. For instance, $x^2 + 5x + 6= (x + 2)(x + 3)$ factors `nicely' and $\disc(x^2 + 5x + 6)= 5^2 - 4(1)6= 25 - 24= 1= 1^2$ is a perfect square, while $x^2 - 2= (x - \sqrt{2})(x + \sqrt{2})$ does not factor `nicely' and $\disc(x^2 - 2)= 0^2 - 4(1)(-2)= 0 + 8= 8$ is not a perfect square. \pvspace{1.1cm}



% Quiz 14
\quizsol \textit{True/False}: The polynomial $2x^2 - 16x + 22$ has roots $4 \pm \sqrt{5}$. Therefore, it factors as $\big(x - (4 - \sqrt{5}) \big) \big(x - (4 + \sqrt{5}) \big)$. \pspace

\sol The statement is \textit{false}. Recall that if the quadratic function $f(x)= ax^2 + bx + c$ has roots $r_1, r_2$, then $f(x)$ factors as $a(x - r_1)(x - r_2)$. Here we have roots $4 - \sqrt{5}$ and $4 + \sqrt{5}$ along with $a= 2$. But then\dots
	\[
	2x^2 - 16x + 22= 2 \big(x - (4 - \sqrt{5}) \big) \big(x - (4 + \sqrt{5}) \big)
	\]
The statement in the square has forgotten the $a$-value in this factorization. We can also see that the quiz statement is false because expanding $\big(x - (4 - \sqrt{5}) \big) \big(x - (4 + \sqrt{5}) \big)$, the $x^2$-term is clearly $x^2$ and not $2x^2$. We can also expand the given factorization directly to see that the factorization is not valid:
	\[
	\begin{gathered}
	\big(x - (4 - \sqrt{5}) \big) \big(x - (4 + \sqrt{5}) \big) \\
	x^2 - (4 + \sqrt{5}) x - (4 - \sqrt{5}) x + (4 - \sqrt{5}) (4 + \sqrt{5}) \\
	x^2 - 4x - \sqrt{5} \, x - 4x + \sqrt{5} \, x + 16 + 4 \sqrt{5} - 4 \sqrt{5} - 5 \\
	x^2 - 8x + 11
	\end{gathered}
	\]



% Quiz 15
\quizsol \textit{True/False}: Because the only factors of $1$ are $\pm 1$ and none of those add to $-4$, the polynomial $4x^2 - 4x + 1$ does not factor. \pspace

\sol The statement is \textit{false}. If we wish to factor a \textit{monic} quadratic polynomial $x^2 + bx + c$, i.e. a polynomial with leading coefficient $a= 1$, we look for factors of $c$ that add to $b$. However, if the quadratic polynomial is not monic, i.e. $a \neq 1$, we look for factors of $ac$ that add to $b$ and use factor-by-grouping. In this case, we see factors of $ac= 4(1)= 4$ that to $-4$. Because $ac= 4 > 0$, the factors must have the same sign. Observe that $(-2)(-2)= 4$ and $-2 + (-2)= -4$. Therefore, we have\dots
	\[
	4x^2 - 4x + 1= 4x^2 - 2x - 2x + 1= 2x(2x - 1) - (2x - 1)= (2x - 1)(2x - 1)= (2x - 1)^2
	\] 
The statement of the quiz has used the factorization technique for a \textit{monic} quadratic rather than a non-monic quadratic. \pvspace{1.3cm}



% Quiz 16
\quizsol \textit{True/False}: $\log_4(2)= 2$ \pspace 

\sol The statement is \textit{false}. Recall that $\log_b(y)= x$ if and only if $b^x= y$. That is, $\log_b(y)$ is the power one needs to raise $b$ to in order to obtain $y$. If $\log_4(2)= 2$, then $4^2= 2$, which is clearly not the case. However, $4^{1/2}= \sqrt{4}= 2$, so that $\log_4(2)= \log_4 \big( 4^{1/2} \big)= \frac{1}{2}$. Observe that generally, we have $\log_b(b^x)= x$. \pvspace{1.3cm}



% Quiz 17
\quizsol \textit{True/False}: For any base $b > 0$, $\log_b(1 + 2 + 3)= \log_b(1) + \log_b(2) + \log_b(3)$. \pspace

\sol The statement is \textit{true}. Recall that $\log_b(xy)= \log_b(x) + \log_b(y)$ and that $\log_b(1)= 0$. But then\dots
	\[
	\log_b(1 + 2 + 3)= \log_b(6)= \log_b(2 \cdot 3)= \log_b 2 + \log_b 3= 0 + \log_b(2) + \log_b(3)= \log_b(1) + \log_b(2) + \log_b(3)
	\] \pvspace{1.3cm}



% Quiz 18
\quizsol \textit{True/False}: The solution to $2^{2x} + 1= 11$ is $x= \frac{\ln(10)}{\ln 4}$. \pspace

\sol The statement is \textit{true}. We can solve this directly, making use of the property $\log_b(a^x)= x \log_b(a)$:
	\[
	\begin{gathered}
	2^{2x} + 1= 11 \\[0.1cm]
	2^{2x}= 10 \\[0.1cm]
	\ln(2^{2x})= \ln(10) \\[0.1cm]
	2x \ln(2)= \ln(10) \\[0.1cm]
	x= \dfrac{\ln(10)}{2 \ln(2)} \\[0.1cm]
	\end{gathered}
	\]
	\[
	\begin{gathered}
	x= \dfrac{\ln(10)}{\ln(2^2)} \\[0.1cm]
	x= \dfrac{\ln(10)}{\ln(4)}
	\end{gathered}
	\]
Alternatively, we can solve this more `naturally' using base-$2$, where we use the fact $\log_b(b^x)= x$: 
	\[
	\begin{gathered}
	2^{2x} + 1= 11 \\[0.1cm]
	2^{2x}= 10 \\[0.1cm]
	\log_2(2^{2x})= \log_2(10) \\[0.1cm]
	2x= \log_2(10) \\[0.1cm]
	x= \dfrac{\log_2(10)}{2}
	\end{gathered}
	\]
To see that this is equivalent to the given solution, we use change of base, i.e. $\log_b(x)= \frac{\log_c(x)}{\log_c(b)}$:
	\[
	\dfrac{\log_2(10)}{2}= \dfrac{\frac{\log_e(10)}{\log_e(2)}}{2}= \dfrac{\log_e(10)}{2 \log_e(2)}= \dfrac{\ln(10)}{2 \ln(2)}= \dfrac{\ln(10)}{\ln(2^2)}= \dfrac{\ln(10)}{\ln(4)}
	\]
Alternatively, we can check the given solution (assuming there is only one solution, which we can see graphically), making use of the fact that $\log_b(b)= 1$ and change of base $\log_b(x)= \frac{\log_c(x)}{\log_c(b)}$: 
	\[
	\begin{gathered}
	2^{2x} + 1= 11 \\[0.1cm]
	2^{2 \cdot \frac{\ln(10)}{\ln(4)}} + 1 \stackrel{?}{=} 11 \\[0.1cm]
	2^{2 \cdot \frac{\ln(10)}{\ln(2^2)}} + 1 \stackrel{?}{=} 11 \\[0.1cm]
	2^{2 \cdot \frac{\ln(10)}{2 \ln(2)}} + 1 \stackrel{?}{=} 11 \\[0.1cm]
	2^{\frac{\ln(10)}{\ln(2)}} + 1 \stackrel{?}{=} 11 \\[0.1cm]
	2^{\frac{\log_2(10)/\log_2(e)}{\log_2(2)/\log_2(e)}} + 1 \stackrel{?}{=} 11 \\[0.1cm]
	2^{\frac{\log_2(10)}{\log_2(2)}} + 1 \stackrel{?}{=} 11 \\[0.1cm]
	2^{\frac{\log_2(10)}{1}} + 1 \stackrel{?}{=} 11 \\[0.1cm]
	2^{\log_2(10)} + 1 \stackrel{?}{=} 11 \\[0.1cm]
	10 + 1 \stackrel{?}{=} 11 \\[0.1cm]
	11= 11 \\[0.1cm]
	\text{\cmark}
	\end{gathered}
	\]































\end{document}