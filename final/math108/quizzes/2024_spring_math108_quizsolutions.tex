\documentclass[11pt,letterpaper]{article}
\usepackage[lmargin=1in,rmargin=1in,bmargin=1in,tmargin=1in]{geometry}
\usepackage{quiz}


% -------------------
% Content
% -------------------
\begin{document}
\thispagestyle{title}

% Quiz 1
\quizsol \textit{True/False}: If you decrease 178 by 20\% consecutively three times, the result is given by $178 (1 - 3 \cdot 0.20)= 178(1 - 0.60)= 178(0.40)= 71.2$.  \pspace

\sol The statement is \textit{false}. If we want to compute $N$ increased or decreased by a \%, we compute $N \cdot (1 \pm \%_d)$, where $\%_d$ is the percentage written as a decimal and we choose `$+$' if it is a percentage increase and choose `$-$' if it is a percentage decrease. Then to compute $178$ decreased by 20\% consecutively three times, we need take $N= 178$, $\%_d= 0.20$, and choose `$-$'. Therefore, we have\dots
	\[
	N \cdot (1 \pm \%_d)= 178 (1 - 0.20)^3= 178(0.80)^3= 178(0.512)= 91.136
	\]
From the $178(0.512)$ portion from the computation above, we can see that decreasing a number by 20\% consecutively three times actually results in a 48.8\% decrease in the original numbers value because $1 - 0.512= 0.488$. The mistake made in the quiz is thinking that repeated percentage increases or decreases are additive. A decrease of 20\% three times \textit{does not} result in a $3 \cdot 20\%= 60\%$ decrease, which was the percentage decrease computed in the quiz statement. 


















\end{document}