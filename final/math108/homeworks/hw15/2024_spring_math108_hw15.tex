\documentclass[11pt,letterpaper]{article}
\usepackage[lmargin=1in,rmargin=1in,tmargin=1in,bmargin=1in]{geometry}
\usepackage{../style/homework}
\setbool{quotetype}{false} % True: Side; False: Under
\setbool{hideans}{true} % Student: True; Instructor: False

% -------------------
% Content
% -------------------
\begin{document}

\homework{15: Due 04/08}{Yeah. Yeah, I\dots I can see this. I mean, it's not for me, but people will like it. It's Starbucks. It's what American wants.}{Matthew MacDell, Big Mouth}

% Problem 1
\problem{10} Define the following:
	\[
	\mathbf{u}= \begin{pmatrix} 1 \\ 0 \\ -1 \\ 4 \end{pmatrix}, \qquad
	\mathbf{v}= \begin{pmatrix} 1 \\ -3 \\ 8 \\ 2 \end{pmatrix}, \qquad
	\mathbf{w}= \begin{pmatrix} 6 \\ -2 \\ -1 \\ 0 \end{pmatrix}
	\]
Showing all your work, compute the following:
	\begin{enumerate}[(a)]
	\item $-3\mathbf{w}$
	\item $\mathbf{v} - \mathbf{u}$
	\item $\mathbf{u} \cdot \mathbf{w}$
	\end{enumerate}



\newpage



% Problem 2
\problem{10} Define the following:
	\[
	A= \begin{pmatrix} -1 & 2 & 0 \\ 0 & 6 & -2 \end{pmatrix}, \qquad
	B= \begin{pmatrix} 6 & -3 & -1 \\ 1 & 1 & 0 \end{pmatrix}, \qquad
	C= \begin{pmatrix} 0 & 2 & -5 \\ 6 & 0 & 4 \end{pmatrix}
	\]
Showing all your work, compute the following:
	\begin{enumerate}[(a)]
	\item $3A$
	\item $B - A$
	\item $CA^T$
	\end{enumerate}



\newpage



% Problem 3
\problem{10} Define the following:
	\[
	A= \begin{pmatrix} 1 & -1 \\ 0 & 3 \\ -4 & 2 \\ 0 & 6 \end{pmatrix}, \qquad
	\mathbf{u}= \begin{pmatrix} 4 \\ -2 \\ 0 \\ 1 \end{pmatrix}
	\]

\begin{enumerate}[(a)]
\item Can one compute $A\mathbf{u}$? If so, compute it. If not, explain why. 
\item Can one compute $A^T\mathbf{u}$? If so, compute it. If not, explain why. 
\end{enumerate}


\end{document}