\documentclass[11pt,letterpaper]{article}
\usepackage[lmargin=1in,rmargin=1in,tmargin=1in,bmargin=1in]{geometry}
\usepackage{../style/homework}
\setbool{quotetype}{false} % True: Side; False: Under
\setbool{hideans}{false} % Student: True; Instructor: False

% -------------------
% Content
% -------------------
\begin{document}

\homework{15: Due 04/08}{Yeah. Yeah, I\dots I can see this. I mean, it's not for me, but people will like it. It's Starbucks. It's what American wants.}{Matthew MacDell, Big Mouth}

% Problem 1
\problem{10} Define the following:
	\[
	\mathbf{u}= \begin{pmatrix} 1 \\ 0 \\ -1 \\ 4 \end{pmatrix}, \qquad
	\mathbf{v}= \begin{pmatrix} 1 \\ -3 \\ 8 \\ 2 \end{pmatrix}, \qquad
	\mathbf{w}= \begin{pmatrix} 6 \\ -2 \\ -1 \\ 0 \end{pmatrix}
	\]
Showing all your work, compute the following:
	\begin{enumerate}[(a)]
	\item $-3\mathbf{w}$
	\item $\mathbf{v} - \mathbf{u}$
	\item $\mathbf{u} \cdot \mathbf{w}$
	\end{enumerate} \pspace

\sol 
\begin{enumerate}[(a)]
\item 
	\[
	-3 \mathbf{w}= -3 \begin{pmatrix} 6 \\ -2 \\ -1 \\ 0 \end{pmatrix}= \begin{pmatrix} -18 \\ 6 \\ 3 \\ 0 \end{pmatrix}
	\] \pspace

\item 
	\[
	\mathbf{v} - \mathbf{u}= \begin{pmatrix} 1 \\ -3 \\ 8 \\ 2 \end{pmatrix} - \begin{pmatrix} 1 \\ 0 \\ -1 \\ 4 \end{pmatrix}= \begin{pmatrix} 1 - 1 \\ -3 - 0 \\ 8 - (-1) \\ 2 - 4 \end{pmatrix}= \begin{pmatrix} 0 \\ -3 \\ 9 \\ -2 \end{pmatrix}
	\] \pspace

\item 
	\[
	\mathbf{u} \cdot \mathbf{w}= \begin{pmatrix} 1 \\ 0 \\ -1 \\ 4 \end{pmatrix} \cdot \begin{pmatrix} 6 \\ -2 \\ -1 \\ 0 \end{pmatrix}= 1(6) + 0(-2) + (-1)(-1) + 4(0)= 6 + 0 + 1 + 0= 7
	\]
\end{enumerate}



\newpage



% Problem 2
\problem{10} Define the following:
	\[
	A= \begin{pmatrix} -1 & 2 & 0 \\ 0 & 6 & -2 \end{pmatrix}, \qquad
	B= \begin{pmatrix} 6 & -3 & -1 \\ 1 & 1 & 0 \end{pmatrix}, \qquad
	C= \begin{pmatrix} 0 & 2 & -5 \\ 6 & 0 & 4 \end{pmatrix}
	\]
Showing all your work, compute the following:
	\begin{enumerate}[(a)]
	\item $3A$
	\item $B - A$
	\item $CA^T$
	\end{enumerate} \pspace

\sol 
\begin{enumerate}[(a)]
\item 
	\[
	3A= 3 \begin{pmatrix} -1 & 2 & 0 \\ 0 & 6 & -2 \end{pmatrix}= \begin{pmatrix} -3 & 6 & 0 \\ 0 & 18 & -6 \end{pmatrix}
	\] \pspace

\item 
	\[
	B - A= \begin{pmatrix} 6 & -3 & -1 \\ 1 & 1 & 0 \end{pmatrix} - \begin{pmatrix} -1 & 2 & 0 \\ 0 & 6 & -2 \end{pmatrix}= \begin{pmatrix} 6 - (-1) & -3 - 2 & -1 - 0 \\ 1 - 0 & 1 - 6 & 0 - (-2) \end{pmatrix}= \begin{pmatrix} 7 & -5 & -1 \\ 1 & -5 & 2 \end{pmatrix}
	\] \pspace

\item 
	\[
	\begin{aligned}
	CA^T&= \begin{pmatrix} 0 & 2 & -5 \\ 6 & 0 & 4 \end{pmatrix} \begin{pmatrix} -1 & 2 & 0 \\ 0 & 6 & -2 \end{pmatrix}^T \\[0.3cm]
	&= \begin{pmatrix} 0 & 2 & -5 \\ 6 & 0 & 4 \end{pmatrix} \begin{pmatrix} -1 & 0 \\ 2 & 6 \\ 0 & -2 \end{pmatrix} \\[0.3cm]
	&= \begin{pmatrix} 
	0(-1) + 2(2) + (-5)0 & 0(0) + 2(6) + (-5)(-2) \\
	6(-1) + 0(2) + 4(0) & 6(0) + 0(6) + 4(-2)
	\end{pmatrix} \\[0.3cm]
	&= \begin{pmatrix}
	0 + 4 + 0 & 0 + 12 + 10 \\
	-6 + 0 + 0 & 0 + 0 - 8
	\end{pmatrix} \\[0.3cm]
	&= \begin{pmatrix} 4 & 22 \\ -6 & -8 \end{pmatrix}
	\end{aligned}
	\]
\end{enumerate}



\newpage



% Problem 3
\problem{10} Define the following:
	\[
	A= \begin{pmatrix} 1 & -1 \\ 0 & 3 \\ -4 & 2 \\ 0 & 6 \end{pmatrix}, \qquad
	\mathbf{u}= \begin{pmatrix} 4 \\ -2 \\ 0 \\ 1 \end{pmatrix}
	\]

\begin{enumerate}[(a)]
\item Can one compute $A\mathbf{u}$? If so, compute it. If not, explain why. 
\item Can one compute $A^T\mathbf{u}$? If so, compute it. If not, explain why. 
\end{enumerate} \pspace

\sol 
\begin{enumerate}[(a)]
\item No, we cannot compute $A \mathbf{u}$. To multiply a $m \times n$ matrix with a $r \times s$ matrix, it must be that $n= r$, i.e. the number of columns of the first must be the number of rows of the second. The matrix $A$ is $4 \times 2$ and the matrix/vector $\mathbf{u}$ is $4 \times 1$. Because $n= 2 \neq 4= r$, we cannot form $A \mathbf{u}$. \pspace

\item The matrix $A^T$ switches the rows and columns of $A$. But then $A^T$ is a $2 \times 4$ matrix. But then one can form $A^T \mathbf{u}$. We have\dots
	\[
	\begin{aligned}
	A^T\mathbf{u}&= \begin{pmatrix} 1 & -1 \\ 0 & 3 \\ -4 & 2 \\ 0 & 6 \end{pmatrix}^T \begin{pmatrix} 4 \\ -2 \\ 0 \\ 1 \end{pmatrix} \\[0.3cm]
	&= \begin{pmatrix} 1 & 0 & -4 & 0 \\ -1 & 3 & 2 & 6 \end{pmatrix} \begin{pmatrix} 4 \\ -2 \\ 0 \\ 1 \end{pmatrix} \\[0.3cm]
	&= \begin{pmatrix} 1(4) + 0(-2) + (-4)0 + 0(1) \\ -1(4) + 3(-2) + 2(0) + 6(1) \end{pmatrix} \\[0.3cm]
	&= \begin{pmatrix} 4 + 0 + 0 + 0 \\ -4 - 6 + 0 + 6 \end{pmatrix} \\[0.3cm]
	&= \begin{pmatrix} 4 \\ -4 \end{pmatrix}
	\end{aligned}
	\]
\end{enumerate}


\end{document}