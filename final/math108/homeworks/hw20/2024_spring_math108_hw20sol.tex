\documentclass[11pt,letterpaper]{article}
\usepackage[lmargin=1in,rmargin=1in,tmargin=1in,bmargin=1in]{geometry}
\usepackage{../style/homework}
\setbool{quotetype}{true} % True: Side; False: Under
\setbool{hideans}{false} % Student: True; Instructor: False

\usepackage{float}	% Force Table Placement

% -------------------
% Content
% -------------------
\begin{document}

\homework{20: Due 04/24}{Okay. No hard feelings, but I hate you. Not joking. Bye.}{Gina Linetti, Brooklyn 99}

% Problem 1
\problem{10} Find the initial simplex tableau corresponding to the linear programming problem shown below:
	\[
	\begin{gathered}
	\max z= 6x_1 + 9 x_2 \\
	\begin{cases}
	x_1 + x_2 \leq 100 \\
	-x_1 + 7x_2 \geq 10 \\
	-6x_1 + x_2 \leq -70 \\
	x_1 + 7x_2 \leq 80 \\
	x_1, x_2 \geq 0
	\end{cases}
	\end{gathered}
	\] \pspace

\sol We need all inequalities to have a nonnegative number on the `right side' of the inequality. So we must multiply both sides of the third inequality by $-1$, so that we obtain the following inequalities: 
	\[
	\begin{cases}
	x_1 + x_2 \leq 100 \\
	-x_1 + 7x_2 \geq 10 \\
	6x_1 - x_2 \geq 70 \\
	x_1 + 7x_2 \leq 80 \\
	x_1, x_2 \geq 0
	\end{cases}
	\]
We now introduce slack or surplus variables to obtain equalities. We also move everything to `one side' in the function to obtain $z - 6x_1 - 9x_2= 0$. Writing all these equalities together, we obtain\dots \par
	\begin{table}[H]
	\centering
	\begin{tabular}{rrrrrrrrrrrrrrrr}
	              & & $x_1$ & $+$ & $x_2$ & $+$ & $s_1$ & & & & & & & & $=$ & $100$ \\
	              & & $-x_1$ & $+$ & $7x_2$ & & & $-$ & $s_2$ & & & & & & $=$ & $10$ \\
	  	      & & $6x_1$ & $+$ & $-1x_2$ & & & & & & $-$ & $s_3$ & & & $=$ & $70$ \\
		      & & $x_1$ & $+$ & $7x_2$ & $+$ & & & & & & & & $s_4$ & $=$ & $80$ \\
	 $z$ & $+$ & $-6x_1$ & $+$ & $-9x_2$ & & & & & & & & & & $=$ & $0$ \\
	\end{tabular}
	\end{table} \par
Therefore, the initial simplex tableau is\dots \par
	\begin{table}[H]
	\centering
	\begin{tabular}{rrrrrr|c}
	$1$ & $1$ & $1$ & $0$ & $0$ & $0$ & $100$ \\
	$-1$ & $7$ & $0$ & $-1$ & $0$ & $0$ & $10$ \\
	$6$ & $-1$ & $0$ & $0$ & $-1$ & $0$ & $70$ \\
	$1$ & $7$ & $0$ & $0$ & $0$ & $1$ & $80$ \\ \hline
	$-6$ & $-9$ & $0$ & $0$ & $0$ & $0$ & $0$ \\
	\end{tabular}
	\end{table}



\newpage



% Problem 2
\problem{10} Below is the initial simplex tableau corresponding to a linear programming maximization problem. Find the initial maximization problem. \par
	\begin{table}[H]
	\centering
	\begin{tabular}{rrrrrrr}
	$4$ & $-1$ & $2$ & $1$ & $0$ & $0$ & $82$ \\
	$-1$ & $5$ & $9$ & $0$ & $1$ & $0$ & $55$ \\
	$7$ & $-1$ & $4$ & $0$ & $0$ & $1$ & $68$ \\
	$-2$ & $-3$ & $-1$ & $0$ & $0$ & $0$ & $0$ 
	\end{tabular}
	\end{table} \pspace

\sol We first add the appropriate horizontal line to separate the function from the inequalities and a vertical line to separate the sides of the equalities. \par
	\begin{table}[H]
	\centering
	\begin{tabular}{rrrrrr|r}
	$4$ & $-1$ & $2$ & $1$ & $0$ & $0$ & $82$ \\
	$-1$ & $5$ & $9$ & $0$ & $1$ & $0$ & $55$ \\
	$7$ & $-1$ & $4$ & $0$ & $0$ & $1$ & $68$ \\ \hline
	$-2$ & $-3$ & $-1$ & $0$ & $0$ & $0$ & $0$ 
	\end{tabular}
	\end{table}  \par
The last row corresponds to the function, while the other rows correspond to the inequalities. Therefore, there were three inequalities in the original problem (not including the non-negativity conditions). For each inequality, we introduce a slack or surplus variable. Therefore, three of the variables are slack or surplus variables. Each column---except the last---corresponds to a variable in the system. Therefore, there are 6 total variables. With 3 slack variables, there must then be $6 - 3= 3$ original variables in the system. We can then label the variables in our system. \par
	\begin{table}[H]
	\centering
	\begin{tabular}{rrrrrrr}
	{\footnotesize$x_1$} & {\footnotesize$x_2$} & {\footnotesize$x_3$} & {\footnotesize$s_1$} & {\footnotesize$s_2$} & {\footnotesize$s_3$} & \\
	$4$ & $-1$ & $2$ & $1$ & $0$ & \multicolumn{1}{r|}{$0$} & $82$ \\
	$-1$ & $5$ & $9$ & $0$ & $1$ & \multicolumn{1}{r|}{$0$} & $55$ \\
	$7$ & $-1$ & $4$ & $0$ & $0$ & \multicolumn{1}{r|}{$1$} & $68$ \\ \hline
	$-2$ & $-3$ & $-1$ & $0$ & $0$ & \multicolumn{1}{r|}{$0$} & $0$ 
	\end{tabular}
	\end{table} \par
We can see that we had to add $s_1, s_2, s_3$ to obtain equalities. Therefore, these are slack variables and the corresponding inequalities must have been `$\leq$'. We know that $z - 2x_1 - 3x_2 - x_3$, which implies $z= 2x_1 + 3x_2 + x_3$. Introducing the condition that the variables are nonnegative, the original optimization problem must have been\dots
	\[
	\begin{gathered}
	\hspace{-0.3cm} \max z= 2x_1 + 3x_2 + x_3 \\
	\begin{cases}
	4x_1 - x_2 + 2x_3 \leq 82 \\
	-x_1 + 5x_2 + 9x_3 \leq 55 \\
	7x_1 - x_2 + 4x_3 \leq 68 \\
	x_1, x_2, x_3 \geq 0
	\end{cases}
	\end{gathered}
	\] 



\newpage



% Problem 3
\problem{10} Below is the final simplex tableau for a linear programming maximization problem. \par
	\begin{table}[H]
	\centering
	\begin{tabular}{rrrrrrrrr}
	$0$ & $1$ & $0$ & $0$ & $0.03$ & $0.12$ & $-0.1$ & $0.04$ & $8.89$ \\
	$0$ & $0$ & $1$ & $0$ & $-0.05$ & $0.03$ & $0.01$ & $0.05$ & $2.19$ \\
	$0$ & $0$ & $0$ & $1$ & $0.05$ & $-0.01$ & $0.05$ & $0.02$ & $7.31$ \\
	$1$ & $0$ & $0$ & $0$ & $0.01$ & $-0.06$ & $-0.03$ & $0.1$ & $1.27$ \\
	$0$ & $0$ & $0$ & $0$ & $0.25$ & $0.86$ & $0.18$ & $0.6$ & $123.14$
	\end{tabular}
	\end{table}

\begin{enumerate}[(a)]
\item How many inequalities were considered?
\item How many variables were there in the original inequalities?
\item How many slack/surplus variables were introduced?
\item What was the solution to this maximization problem?
\end{enumerate} \pspace

\sol 
\begin{enumerate}[(a)]
\item Every row in the tableau corresponds to an inequality---except for the last row which corresponds to the function. Because there are $5$ rows, there must have been $5 - 1= 4$ inequalities in the original system (neglecting the non-negativity inequalities). 

\item Every column in the tableau corresponds to a variable---except the last column which corresponds to the `other' side of an equality. Because there are $9$ columns, there are $9 - 1= 8$ variables in the system. Because we introduce a slack or surplus variable to each inequality and by (a) there are $4$ inequalities, $4$ of the variables are slack/surplus variables. Therefore, there were $8 - 4= 4$ `original' variables in the system. 

\item By (b), we know that there were $4$ slack or surplus variables introduced. 

\item Introducing labels for the variables, adding horizontal and vertical lines, and boxing the `pivot positions', we obtain the following tableau: \par
	\begin{table}[H]
	\centering
	\begin{tabular}{rrrrrrrrr}
	{\footnotesize $x_1$} & {\footnotesize $x_2$} & {\footnotesize $x_3$} & {\footnotesize $x_4$} & {\footnotesize $s_1$} & {\footnotesize $s_2$} & {\footnotesize $s_3$} & {\footnotesize $s_4$} & \\
	$0$ & $\boxed{1}$ & $0$ & $0$ & $0.03$ & $0.12$ & $-0.1$ & \multicolumn{1}{r|}{$0.04$} & $8.89$ \\
	$0$ & $0$ & $\boxed{1}$ & $0$ & $-0.05$ & $0.03$ & $0.01$ & \multicolumn{1}{r|}{$0.05$} & $2.19$ \\
	$0$ & $0$ & $0$ & $\boxed{1}$ & $0.05$ & $-0.01$ & $0.05$ & \multicolumn{1}{r|}{$0.02$} & $7.31$ \\
	$\boxed{1}$ & $0$ & $0$ & $0$ & $0.01$ & $-0.06$ & $-0.03$ & \multicolumn{1}{r|}{$0.1$} & $1.27$ \\ \hline
	$0$ & $0$ & $0$ & $0$ & $0.25$ & $0.86$ & $0.18$ & \multicolumn{1}{r|}{$0.6$} & $123.14$
	\end{tabular}
	\end{table} \par
This gives $x_1= 1.27$, $x_2= 8.89$, $x_3= 2.19$, and $x_4= 7.31$. All the remaining variables have value $0$. From the bottom-rightmost entry, we see that $\max z= 123.14$. Therefore, the maximum values is $123.14$ and occurs at $(x_1, x_2, x_3, x_4, s_1, s_2, s_3, s_4)= (1.27, 8.89, 2.19, 7.31, 0, 0, 0, 0)$. 
\end{enumerate}



\newpage



% Problem 4
\problem{10} Below is the final simplex tableau for a linear programming minimization problem. \par
	\begin{table}[H]
	\centering
	\begin{tabular}{rrrrrrr}
	$3$ & $9$ & $0$ & $1$ & $0$ & $2$ & $5$ \\
	$0$ & $-5$ & $0$ & $0$ & $1$ & $-1$ & $2$ \\
	$1$ & $4$ & $1$ & $0$ & $0$ & $1$ & $2$ \\
	$5$ & $48$ & $0$ & $0$ & $0$ & $15$ & $30$
	\end{tabular}
	\end{table}

\begin{enumerate}[(a)]
\item How many inequalities were considered?
\item How many variables were there in the original inequalities?
\item How many slack/surplus variables were introduced?
\item What was the solution to this minimization problem?
\end{enumerate} \pspace

\sol 
\begin{enumerate}[(a)]
\item Each row of the tableau corresponds to an inequality---except for the last row which corresponds to the function. Because there are $4$ rows, the original system had $4 - 1= 3$~inequalities. 

\item Each column of the tableau corresponds to a variable---except for the last column which corresponds to the `other side' of the equalities. Because there are $7$ columns, there are $7 - 1= 6$~variables in the system. A slack or surplus variable is introduced for each inequality. We know there are three inequalities from (a). Therefore, there are three slack or surplus variables. For a minimization problem, the variables in the original system correspond to the slack or surplus variables in the dual maximization problem. Therefore, there were $3$ variables in the original minimization problem. For the dual maximization problem, we know there were $3$ variables involved and that there are $3$ slack or surplus variables. Therefore, the dual maximization problem has $6 - 3= 3$ `original' variables. 

\item From (b), we know that $3$ slack or surplus variables were introduced into the dual maximization problem. 

\item We add a horizontal line to separate the function from the equalities and a vertical line to separate the variables from the `other side' of the equalities. We also label the $3$ `original' variables and the $3$ slack/surplus variables. \par
	\begin{table}[H]
	\centering
	\begin{tabular}{rrrrrrr}
	{\footnotesize$x_1$} & {\footnotesize$x_2$} & {\footnotesize$x_3$} & {\footnotesize$s_1$} & {\footnotesize$s_2$} & {\footnotesize$s_3$} & \\
	$3$ & $9$ & $0$ & $1$ & $0$ & \multicolumn{1}{r|}{$2$} & $5$ \\
	$0$ & $-5$ & $0$ & $0$ & $1$ & \multicolumn{1}{r|}{$-1$} & $2$ \\
	$1$ & $4$ & $1$ & $0$ & $0$ & \multicolumn{1}{r|}{$1$} & $2$ \\ \hline
	$5$ & $48$ & $0$ & $0$ & $0$ & \multicolumn{1}{r|}{$15$} & $30$
	\end{tabular}
	\end{table} \par
The minimum value to the original minimization problem is the maximum value for the dual problem. From the table above, we can see that the maximum value is $30$. The original minimization problem had two variables, say $y_1, y_2, y_3$. The value of the original minimization variables are the `values' of the slack/surplus variables in the dual minimization problem. Therefore, we know $y_1= 0$, $y_2= 0$, and $y_3= 15$. Therefore, the minimum value is $30$ and occurs at the point $(y_1, y_2, y_3)= (0, 0, 15)$. 
\end{enumerate}


\end{document}