\documentclass[11pt,letterpaper]{article}
\usepackage[lmargin=1in,rmargin=1in,tmargin=1in,bmargin=1in]{geometry}
\usepackage{../style/homework}
\setbool{quotetype}{true} % True: Side; False: Under
\setbool{hideans}{false} % Student: True; Instructor: False

% -------------------
% Content
% -------------------
\begin{document}

\homework{5: Due 02/07}{The answer was so simple, I was too smart to see it!}{Princess Bubblegum, Adventure Time}

% Problem 1
\problem{10} You want to purchase a collection of Bob Ross paintings. Bank Sinatra offers you two different loan options: a loan with 3.2\% annual interest, compounded semiannually or a loan at 3.18\% annual interest, compounded continuously. 
	\begin{enumerate}[(a)]
	\item Which loan appears to be the `better deal'? Explain. 
	\item Compute the effective interest for both loan setups. Which loan setup is better? Explain. 
	\item Compute the doubling time for both loan setups. Which loan setup is better? Explain.
	\end{enumerate} \pspace

\sol 
\begin{enumerate}[(a)]
\item Because the 3.18\% annual interest rate is lower than the 3.2\% annual interest rate, the 3.18\% annual interest, compounded continuously appears to be the better deal. \pspace

\item We have\dots
	\[
	\begin{aligned}
	r_{\text{eff, DC}}&= \left(1 + \dfrac{r}{k} \right)^k - 1= \left(1 + \dfrac{0.032}{2} \right)^2 - 1= 1.016^2 - 1= 1.032256 - 1= 0.032256 \\[0.3cm]
	r_{\text{eff, CC}}&= e^r - 1= e^{0.0318} - 1= 1.032311 - 1= 0.032311
	\end{aligned}
	\]
Because the discrete compounded loan, i.e. the 3.2\% annual interest, compounded semiannually, has the lower effective interest, this loan is the better deal. \pspace
 
\item We have\dots
	\[
	\begin{aligned}
	t_{D, DC}&= \dfrac{\ln(2)}{k \ln \left(1 + \dfrac{r}{k} \right)}= \dfrac{\ln(2)}{2 \ln(1.016)}= \dfrac{0.69314718}{0.0317467}= 21.8337 \\[0.3cm]
	t_{D, CC}&= \dfrac{\ln(2)}{r}= \dfrac{0.69314718}{0.0318}= 21.7971
	\end{aligned}
	\]
Because the discrete compounded loan, i.e. the 3.2\% annual interest, compounded semiannually, has the longer doubling time, this loan is the better deal.  
\end{enumerate}



\newpage



% Problem 2
\problem{10} Reed wants to buy a tablet for his books while he travels. He finds one that he likes for \$340. Reed places \$240 into an account that earns 1.02\% annual interest, compounded continuously. Assume that he makes no additional deposits. 
	\begin{enumerate}[(a)]
	\item How long until Reed has enough money for the tablet?
	\item How long until Reed would have doubled his money?
	\end{enumerate} \pspace

\sol This is a continuous compounding interest problem. We have a principal of $P= \$240$. The annual interest rate is $r= 0.0102$. 

\begin{enumerate}[(a)]
\item We want to know the time that it takes Reed's investment of \$240 to become \$340. This is\dots
	\[
	t= \dfrac{\ln(F/P)}{r}= \dfrac{\ln(\$340/\$240)}{0.0102}= \dfrac{\ln(1.4166667)}{0.0102}= \dfrac{0.3483067}{0.0102}= 34.1477
	\]
Therefore, it will take Reed 34.15~years to save for the tablet. \pspace

\item The amount of time it will take Reed to double his money is\dots
	\[
	t_D= \dfrac{\ln(2)}{r}= \dfrac{0.69314718}{0.0102}= 67.9556 \text{ years}
	\]
\end{enumerate}



\newpage



% Problem 3
\problem{10} Spencer sold his painting of a man dressed as a T-Rex walking someone else's dog for \$2,427. He places all the money into an account that earns 4.7\% annual interest, compounded quarterly. 
	\begin{enumerate}[(a)]
	\item How long until he has double this amount in savings?
	\item How long until this money in the account has increased in value to \$200,000?
	\end{enumerate} \pspace

\sol Because the amount is simply sitting and earning interest---with no additional deposits---this is a simple discrete compounded interest problem. We have a principal amount of $P= \$2,\!427$ with a nominal interest rate of $r= 0.047$, compounded $k= 4$~times per year. 

\begin{enumerate}[(a)]
\item The amount of time until Spencer doubles his savings is\dots
	\[
	t_D= \dfrac{\ln(2)}{k \ln \left(1 + \dfrac{r}{k} \right)}= \dfrac{\ln(2)}{4 \ln \left(1 + \dfrac{0.047}{4} \right)}= \dfrac{\ln(2)}{4 \ln(1.01175)}= \dfrac{0.6931471805599453}{0.04672601909535132}= 14.8343 \text{ years}
	\] \pspace

\item The amount of time it will take for Spencer's investment to grow to \$200,000 is\dots
	\[
	t= \dfrac{\ln(F/P)}{k \ln \left(1 + \dfrac{r}{k} \right)}= \dfrac{\ln(\$200,\!000/\$2,\!427)}{4 \ln \left(1 + \dfrac{0.047}{4} \right)}= \dfrac{\ln(82.406262876)}{4 \ln(1.01175)}= \dfrac{4.411661439803832}{0.04672601909535132}= 94.4155 \text{ years}
	\]
\end{enumerate}


\end{document}