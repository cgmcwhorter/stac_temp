\documentclass[11pt,letterpaper]{article}
\usepackage[lmargin=1in,rmargin=1in,tmargin=1in,bmargin=1in]{geometry}
\usepackage{../style/homework}
\setbool{quotetype}{true} % True: Side; False: Under
\setbool{hideans}{false} % Student: True; Instructor: False

% -------------------
% Content
% -------------------
\begin{document}

\homework{2: Due 01/29}{All wish to possess knowledge, but few, comparatively speaking, are willing to pay the price.}{Decimus Junius Juvenalis}

% Problem 1
\problem{10} Suppose that the revenue and cost function for a certain item are given by $R(q)= 67.99q$ and $C(q)= 13.47q + 495000$, respectively. 
	\begin{enumerate}[(a)]
	\item How much does the company sell each item for? How much does it cost to make each item?
	\item What are the fixed costs for the production of this good?
	\item What is the profit or loss if the company produces and sells ten-thousand of these items?
	\item What is the break-even point? At least many items does this company need to sell in order to make a profit on this item?
	\end{enumerate} \pspace

\sol 
\begin{enumerate}[(a)]
\item Both $R(q)$ and $C(q)$ are linear. Therefore, the amount each item sells for and the cost to produce each item requires to make are the slopes of $R(q)$ and $C(q)$, respectively. Therefore, each item sells for \$67.99 and costs \$13.47 to make. \pspace

\item The fixed costs are the costs not associated with the amount of production, i.e. the costs when one still has even when producing nothing. But this must be $C(0)= 13.47(0) + 495000= 0 + 495000= \$495,\!000$. \pspace

\item We know the profit is revenue minus any associated costs. We have\dots
	\[
	\begin{aligned}
	R(10,\!000)&= 67.99(10000)= \$679,\!900 \\
	C(10,\!000)&= 13.47(10000) + 495000= \$629,\!700 \\
	P(10,\!000)&= R(10,\!000) - C(10,\!000)= \$679,\!900 - \$629,\!700= \$50,\!200
	\end{aligned}
	\]
Therefore, if the company produces and sells 10,000 items, the company has a profit of \$50,200. \pspace

\item The break-even point is the point where revenue is equal to cost, i.e. where the profit is 0. But then\dots
	\[
	\begin{gathered}
	R(q)= C(q) \\
	67.99q= 13.47q + 495000 \\
	54.52q= 495000 \\
	q= 9,\!079.24
	\end{gathered}
	\]
Therefore, to turn a profit, the company must make and sell at least 9,080~items. 
\end{enumerate}



\newpage



% Problem 2
\problem{10} Leslie owns a wine and spirit store called \textit{Planet of the Grapes}. She rents the building for \$24,730 per month. The average bottle of wine or spirit at her store sells for \$11.56. The average cost of ordering, stocking, and selling these wines/spirits is \$5.21 per bottle. 
	\begin{enumerate}[(a)]
	\item What are the fixed and variable costs for Leslie's business?
	\item Find the cost function for Leslie's business.
	\item Find the revenue function for Leslie's business.
	\item Find the break-even point for Leslie's business. 
	\item What is the minimal average amount of bottles Leslie must sell per month to make a profit?
	\item How many bottles must Leslie sell each month on average to make a profit of \$15,000 (translating to a yearly profit of \$180,000)? Does this seem feasible?	
	\end{enumerate} \pspace

\sol 
\begin{enumerate}[(a)]
\item The fixed costs are the costs not associated with the level of production. In this case, this is the \$24,730 in rent Leslie pays each month. The variable costs are the costs associated with production. Each bottle costs \$5.21, on average, to order, stock, and sell. Therefore, if $q$ bottles are sold, the variable costs would be $5.21q$. \pspace

\item We know the costs are the total of the variable and fixed costs. Therefore, $C(q)= \text{V.C.} + \text{F.C.}= 5.21q + 24730$. 

\item Because each bottle sells for an average of \$11.56, if Leslie sells $q$ bottles, the amount of money brought in would be $11.56q$. Therefore, $R(q)= 11.56q$. \pspace

\item The break-even point is the point where revenue and cost are equal. But then\dots
	\[
	\begin{gathered}
	R(q)= C(q) \\
	11.56q= 5.21q + 24730 \\
	6.35q= 24730 \\
	q= 3,\!894.49
	\end{gathered}
	\]

\item From (d), we can see that the minimal average number of bottles Leslie has to sell each month to make a profit is 3,895. 

\item We know that $P(q)= R(q) - C(q)= 11.56q - (5.21q + 24730)= 11.56q - 5.21q - 24730= 6.35q - 24730$. To make a profit of \$15,000, we require\dots
	\[
	\begin{gathered}
	P(q)= \$15000 \\
	6.35q - 24730= 15000 \\
	6.35q= 39730 \\
	q= 6,\!256.69
	\end{gathered}
	\]
Therefore, at least 6,257~bottles would need to be sold. If the store was open from 8~am to 10~pm, seven days a week for a month (31~days), this would mean that on average 2.06~bottles were sold every hour. This seems like a reasonable rate---even though the sales would not be distributed as described. 
\end{enumerate}



\newpage



% Problem 3
\problem{10} Suppose a company produces two items, $q_1$ and $q_2$, and has a cost function given by $C(q_1, q_2)= 7.23 q_1 + 82.56 q_2 + 15721.12$. 
	\begin{enumerate}[(a)]
	\item What are the fixed costs for producing these two items?
	\item What is the total cost associated with producing 30 of the first item and 65 of the second item?
	\item How much does it cost to produce the first item? How much does it cost to produce the second item?
	\end{enumerate} \pspace

\sol 
\begin{enumerate}[(a)]
\item The fixed costs are the costs not associated with the amount of production, i.e. the costs when one still has even when producing nothing. But this is $C(0, 0)= 7.23(0) + 82.56(0) + 15721.12= 0 + 0 + 15721.12= \$15,\!721.12$. \pspace

\item This is precisely\dots
	\[
	C(30, 65)= 7.23 (30) + 82.56 (65) + 15721.12= 216.90 + 5366.40 + 15721.12= \$21,\!304.42
	\] \pspace

\item Because the function $C(q_1, q_2)$ is `linear' in $q_1, q_2$, the cost to produce the items are the `slopes' for each of the variables. But then it costs \$7.23 per item to produce the first item and \$82.56 per item to produce the second item. 
\end{enumerate}



\newpage



% Problem 4
\problem{10} Suppose that you have a revenue function given by $R(q)= 89q$ and a cost function given by $C(q)= 45q + 7200$. 
	\begin{enumerate}[(a)]
	\item What are the revenue and cost at a production/sale level of 60~units?
	\item Without finding the profit function, find the break-even point for the production/sale of this item.
	\item Find the profit function, $P(q)$.
	\item Compute $P(60)$. Explain how you could use (a) to find $P(60)$. 
	\end{enumerate} \pspace

\sol 
\begin{enumerate}[(a)]
\item We have\dots
	\[
	\begin{aligned}
	R(60)&= 89(60)= \$5,\!340 \\[0.3cm]
	C(60)&= 45(60) + 7200= 2700 + 7200= \$9,\!900
	\end{aligned}
	\] \pspace

\item We simply set $R(q)= C(q)$. But then\dots
	\[
	\begin{gathered}
	R(q)= C(q) \\
	89q= 45q + 7200 \\
	44q= 7200 \\
	q= 163.636
	\end{gathered}
	\] \pspace

\item The profit function is the difference of the revenue and cost functions:
	\[
	P(q)= R(q) - C(q)= 89q - (45q + 7200)= 89q - 45q - 7200= 44q - 7200
	\] \pspace

\item We have\dots
	\[
	P(60)= 44(60) - 7200= 2640 - 7200= -\$4,\!560
	\] \pspace
Alternatively, from (a), we know that $R(60)= \$5,\!340$ and $C(60)= \$9,\!900$. But then $P(60)= \$5,\!340 - \$9,\!900= -\$4,\!560$. 
\end{enumerate}


\end{document}