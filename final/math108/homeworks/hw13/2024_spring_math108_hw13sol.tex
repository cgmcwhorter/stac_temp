\documentclass[11pt,letterpaper]{article}
\usepackage[lmargin=1in,rmargin=1in,tmargin=1in,bmargin=1in]{geometry}
\usepackage{../style/homework}
\setbool{quotetype}{false} % True: Side; False: Under
\setbool{hideans}{false} % Student: True; Instructor: False

% -------------------
% Content
% -------------------
\begin{document}

\homework{13: Due 03/25}{That's the thing. I don't think I believe in deep down. I kinda think that all you are is just the things that you do.}{Diane Nguyen, BoJack Horseman}

% Problem 1
\problem{10} Psychologists are examining how students cope with failure. They find estimates that 20\% of individuals will fail at least one course during their lifetime. Suppose that you take a simple random sample of nine students. 
	\begin{enumerate}[(a)]
	\item What is the probability that exactly 4 of them have failed a course?
	\item What is the probability that at least one of them has failed a course?
	\item What is the probability that less than 5 of them failed a course? 
	\item What is the probability that exactly 8 of them failed a course?
	\end{enumerate} \pspace

\sol Because this is a simple random sample, the samples are independent. There are a fixed number of students sampled. Each student either passed or failed a course with a fixed probability. Therefore, the number of persons in the sample that failed a course is given by a binomial distribution $B(n, p)= B(9, 0.20)$. 

\begin{enumerate}[(a)]
\item 
	\[
	P(X= 4)= 0.0661
	\] \pspace

\item 
	\[
	P(X \geq 1)= 1 - P(X= 0)= 1 - 0.1342= 0.8658
	\] \pspace

\item 
	\[
	\begin{aligned}
	P(X < 5)&= P(X= 0) + P(X= 1) + P(X= 2) + P(X= 3) + P(X= 4) \\[0.3cm]
	&= 0.1342 + 0.3020 + 0.3020 + 0.1762 + 0.0661\\[0.3cm]
	&= 0.9805
	\end{aligned}
	\] \pspace

\item 
	\[
	P(X= 8) \approx 0.
	\]
\end{enumerate}



\newpage



% Problem 2
\problem{10} A state senate race has grown increasing tense as scandal has struck the incumbents office, resulting a huge drop in their support. It is currently estimated that only 10\% of voters currently support the incumbent. Suppose pollsters take a simple random sample of 15 potential voters. 
	\begin{enumerate}[(a)]
	\item What is the probability that more than 8 of them support the incumbent? 
	\item What is the probability that less than 4 of them support the incumbent? 
	\item What is the probability that at least one of them support the incumbent?
	\end{enumerate} \pspace

\sol Because the sample is a simple random sample, the samples are independent. There is a fixed number of people surveyed. Each person surveyed either supports the candidate or not with a fixed probability. Therefore, the number of people supporting the candidate is given by a binomial distribution $B(n, p)= B(15, 0.10)$. 

\begin{enumerate}[(a)]
\item 
	\[
	P(X > 8) \approx 0.0
	\] \pspace

\item 
	\[
	P(X < 4)= P(X= 0) + P(X= 1) + P(X= 2) + P(X= 3)= 0.2059 + 0.3432 + 0.2669 + 0.1285= 0.9445
	\] \pspace
	
\item 
	\[
	P(X \geq 1)= 1 - P(X= 0)= 1 - 0.2059= 0.7941
	\]
\end{enumerate}


\end{document}