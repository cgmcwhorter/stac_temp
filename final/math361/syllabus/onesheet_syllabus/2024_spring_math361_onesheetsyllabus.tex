\documentclass[11pt,letterpaper]{article}
\usepackage[lmargin=0.8in,rmargin=0.8in,bmargin=0.8in,tmargin=0.8in]{geometry}
\usepackage{style}

\pagenumbering{gobble}


% -------------------
% Content
% -------------------
\begin{document}

% TItle
\begin{center} 
\bfseries
\color{stacred}
\LARGE Syllabus Quick Facts \par\vspace{0.2\baselineskip}
\Large MATH 361: Numerical Analysis --- Spring 2024
\end{center} \pspace


% Course Information
\mysection{0.27}{Course Information}{course_info}
\hspace{0.53cm} {\itshape Instructor Email}: \href{mailto:cmcwhort@stac.edu}{cmcwhort@stac.edu} \par
\hspace{0.53cm} {\itshape Course Webpage}: \href{https://coffeeintotheorems.com/courses/2024-2/spring/math-361/}{https://coffeeintotheorems.com/courses/2024-2/spring/math-361/} \par
\hspace{0.53cm} {\itshape Office Hours}: 	\par \vspace{-0.3cm}
	\begin{table}[!ht]
	\centering
	\begin{tabular}{l || l}
	Mon. & 11:30~am -- 12:30~pm \\
	Tues. & 11:30~am -- 12:30~pm \\
	Wed. & 11:30~am -- 12:30~pm \\
	Thurs. & 11:30~am -- 12:30~pm
	\end{tabular}
	\end{table}


% Grading Components
\mysection{0.27}{Grading Components}{grade_comp}
Course grades are determined by the following components: \par \vspace{-0.3cm}
	\begin{table}[!ht]
        \begin{tabular}{clr}
        & Quizzes & 15\% \\
        & Project & 15\% \\
        & Homework & 40\% \\
        & Midterm & 15\% \\
        & Final & 15\%
        \end{tabular} 
        \end{table}


% Attendance 
\mysection{0.27}{Attendance}{attendance}
Attend each lecture and show up on time. Anticipated absences should be addressed with the instructor in advance of the absence. Address any absences---anticipated or otherwise---with the instructor. If you miss a lecture, you are responsible for any material covered, any work assigned, any course changes made, etc. during the class. Four or more unexcused absences from lectures could result in receiving a grade penalty per additional absence or an `F' in the course. Furthermore, excessive lateness will also count as an absence. \pspace


% Quizzes 
\mysection{0.27}{Quizzes}{quizzes}
There will be a quiz \textit{every} class, typically at the start of class. Because solutions will often then be immediately discussed, no make-up quizzes will be given (except under extraordinary circumstances). \pspace


% Homeworks 
\mysection{0.27}{Homeworks}{homeworks}
There will typically be a homework assigned each class, due the next class. Homework is a large portion of your grade, so your best work should be put into them. Your solutions should be neat, organized, display effort and clear mathematical thinking, and they should be submitted using the homework packets. Homeworks will also entail the use of the software Mathematica. Students will be required to having a working copy of Mathematica. These portions may require a fair amount of independence on the part of the student. However, there are resources available to help you with these problems. Should you have difficulty with these problems, ask your instructor for help! Be aware that many of your fellow students may be more technologically literate and ask them for help as well! However, you should \textit{not} simply copy someone's code or take it with only minor changes. Your code and method of solution should be your own! \pspace

The written and software portions of homework will be weighted equally---each worth 20\% of the course grade. Assignments should be started as soon as possible; it is easier to keep up than it is to catch up. You may request extensions on homework assignments (possibly incurring a grade penalty). Requests for extensions should be submitted to the instructor in a timely fashion---do not delay! However, do not simply assume that you will be able to receive extra time on an assignment and plan your schedule carefully. You are encouraged to work with others on homeworks; however, be sure to carefully abide by the academic integrity standards excepted by the college and instructor. \pspace



% Exams 
\mysection{0.27}{Exams}{exams}
There will be a total of 2 exams that are each worth 15\% of the course grade for a total of 30\%. Each exam covers course material, up until the exam preceding it. While the exams are not cumulative, topics from previous exams can appear in an exam if the material is relevant---but it will not be the focus of the exam. You should be present, seated, and prepared for a scheduled exam before the exam begins. If you are late, you should not expect extra exam time. There are no make-up exams except under extraordinary circumstances. The exams may be take-home exams. The exact guidelines may differ between the exams. However, the content covered by an exam, the exam procedures, and the exam due date will be announced the week of the exam. In the case of a take-home exam, students will be given at least 24~hours to do the exam. \pspace



% Project 
\mysection{0.27}{Project}{project}
There will be a project in this course that will be given in Mathematica. The project will be selected by the instructor and will be a blend of both the theoretical and programming aspects of the course. This project will use topics examined in the course but also things students have not seen before. This will test whether you have developed stronger mathematical cognitive skills and programming skills that you can take to `real-world' problems. The project will be given out in-class at least a month before the project is due. However, do not delay in starting the project! The project should be worked on independently and students should not consult each other or any outside resources without the permission of the instructor. \pspace



% Course Schedule 
\mysection{0.27}{Course Schedule}{schedule}
The following is a \emph{tentative} schedule for the course and is subject to change. 
	\begin{table}[!ht]
        \centering
        \scalebox{1}{%
        \begin{tabular}{ll || ll}
        Date & Topic(s) & Date & Topic(s) \\ \hline 
	01/23 & Calculus Review & 03/14 & Spring Break \\
	01/25 & Calculus Review & 03/19 & Newton \& Lagrange Interpolation \\
	01/30 & Mathematica & 03/21 & Divided Differences \\
	02/01 & Mathematica & 03/26 & Hermite's Theorem \\
	02/06 & Floating Point Numbers & 03/28 & Chevshev's Theorem \\
	02/08 & Error, Significance, \& Stability & 04/02 & Numerical Differentiation \\
	02/13 & Bisection Method & 04/04 & Numerical Differentiation \\
	02/15 & Newton's Method & 04/09 & Numerical Integration \\
	02/20 & Secant Method & 04/11 & Numerical Integration \\
	02/22 & Polynomial Roots & 04/16 & Numerical Integration \\
	02/27 & Linear Regression & 04/18 & Quadrature \\
	02/29 & General Curve Fitting & 04/23 & Ordinary Differential Equations \\
	03/05 & Gradient Descent & 04/25 & Euler's Method \\
	03/07 & Newton \& Lagrange Interpolation & 04/30 & Runge-Kutta \\
	03/12 & Spring Break & 05/02 & Additional Topics
        \end{tabular}
        }
        \end{table}


\end{document}