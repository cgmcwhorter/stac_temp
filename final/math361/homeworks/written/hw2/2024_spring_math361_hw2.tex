\documentclass[11pt,letterpaper]{article}
\usepackage[lmargin=1in,rmargin=1in,tmargin=1in,bmargin=1in]{geometry}
\usepackage{../../style/homework}
\setbool{quotetype}{true} % True: Side; False: Under
\setbool{hideans}{true} % Student: True; Instructor: False

% -------------------
% Content
% -------------------
\begin{document}

\homework{2: Due 02/01}{You say impossible, but all I hear is, `I'm possible.'\,}{Ted Lasso, Ted Lasso}

% Problem 1
\problem{10} Showing all your work and fully justifying your reasoning, compute the following:
	\begin{2enumerate}
	\item $\ds \lim_{x \to 0} \dfrac{\sin(3x)}{\cos(5x)}$
	\item $\dfrac{d}{dx}\, \left( \dfrac{x e^x}{x^2 + 1} \right)$
	\item $\dfrac{d^2}{dx^2} (x^2 + 5)^{10}$
	\item $\ds \int x e^x \; dx$
	\end{2enumerate}



\newpage 



% Problem 2
\problem{10} One of the first `non-trivial' approximation techniques one learns is the process of linearization. Recall that if $f(x)$ is differentiable at $c$, the linearization of $f(x)$ at $c$, denoted $L(x)$, is the tangent line of $f(x)$ at $x= c$. But then for $x \approx c$, we have $f(x) \approx L(x)$. Consider the function $f(x)= \sqrt{x}$. 
	\begin{enumerate}[(a)]
	\item Find the linearization of $f(x)$ at $x= 144$. 
	\item Use (a) to approximation $\sqrt{150}$. What is the error for your approximation?
	\item Is this generally a useful method for computing $f(x)= \sqrt{x}$? Explain. 
	\end{enumerate}



\newpage



% Problem 3
\problem{10} Another of the first `non-trivial' approximation techniques one learns is Taylor series. The Taylor series of a function can be used to approximate values of the function. In fact, the (infinite) Taylor series can be exactly equal to the function. Consider the polynomial $f(x)= x^3 - 5x^2 + 7$. 
	\begin{enumerate}[(a)]
	\item Find the Taylor Series for $f(x)$ at $x= 1$. 
	\item Show your Taylor Series in (a) is exactly $f(x)$. 
	\item Assuming that $(x - 1)^n$ is `negligible' whenever $n > 1$ and $x \approx 1$, use (b) to approximate $f(1.01)$. What is the error for this approximation? 
	\end{enumerate}



\newpage



% Problem 4
\problem{10} Taylor series can also be used to approximate integrals that are not exactly computable. For instance, to find the percentage of values within one standard deviation of the mean for a normal distribution one would need to compute\dots
	\[
	\dfrac{1}{\sqrt{2\pi}} \int_{-1}^1 e^{-x^2/2} \;dx
	\]
However, the integral $\ds \int e^{-x^2/2} \; dx$ has no elementary antiderivative. Therefore, approximation must be used. Recall the Maclaurin series for $e^x$ is $\sum_{n=0}^\infty \frac{x^n}{n!}$ and that this series has an infinite radius of convergence. 
	\begin{enumerate}[(a)]
	\item Find the Maclaurin series for $e^{-x^2/2}$. Show that this series converges to $e^{-x^2/2}$ everywhere. 
	\item Let $T_3(x)$ denote the first three nonzero terms from your series in (a). Approximate the integral above by using the fact that $e^{-x^2/2} \approx T_3(x)$ on $[-1, 1]$. 
	\item It is a well-known fact in Statistics that approximately 68\% of values in a normal distribution are within one standard deviation of the mean. Does your answer in (b) agree with this fact? 
	\end{enumerate}


\end{document}