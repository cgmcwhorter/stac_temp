\documentclass[11pt,letterpaper]{article}
\usepackage[lmargin=1in,rmargin=1in,tmargin=1in,bmargin=1in]{geometry}
\usepackage{../style/homework}
\setbool{quotetype}{true} % True: Side; False: Under
\setbool{hideans}{true} % Student: True; Instructor: False

% -------------------
% Content
% -------------------
\begin{document}

\homework{3: Due ????}{}{}

% Problem 1
\problem{10} bisection


% Problem 2
\problem{10} newton


% Problem 3
\problem{10} secantmethod


\newpage



% Problem 4
\problem{10} If $\phi \colon \mathbb{R} \to \mathbb{R}$ is a function, we say that $p$ is a fixed point for $\phi$ if $f(p)= p$.
	\begin{enumerate}
	\item[(a)] If $\phi(x)$ has a fixed point, explain why the graph of $\phi(x)$ must intersect the line $y= x$.
	\item[(b)] Must all the fixed points for $\phi(x)$ lie along the line $y= x$? Explain why or why not. 
	\item[(c)] Find all the fixed points for $\phi(x)= x^2 + 5x - 12$. 
	\end{enumerate}
One method of finding a root for a function $f(x)$ is known as fixed point iteration. In fixed point iteration, one begins with some initial value $x_0$ and defines a recursive sequence $\{ x_n \}$ via $x_{n+1} = f(x_n)$. Assuming this iterative process converges (it may not), one terminates the algorithm when $|x_{n+1} - x_n| < \epsilon$, where $\epsilon > 0$ is some constant. Suppose one wants to find a root of a function $F(x)$.
	\begin{enumerate}[(a)]
	\item[(d)] Show that if $F(x)$ is continuous and $\displaystyle \lim_{n \to \inty} x_n$ converges to a limit $x^*$, then $x^*$ is a fixed point for $fFx)$. 
	\item[(e)] Show that if $r$ is a root of $F(x)$, then $r$ is a fixed point for $G(x):= x - F(x)$. [There are many functions $G$ with $r$ as a fixed point.]
	\item[(f)] Show that if a function $G(x)$ has a fixed point $p$, then $H(x):= x - G(x)$ has a root at $p$. 
	\end{enumerate}
Let $f(x)$ be a differentiable function and define $g(x):= x - \frac{f(x)}{f'(x)}$. 
	\begin{enumerate}
	\item[(g)] Explain why Newton's method is a special example of fixed point iteration.
	\item[(h)] Show that if fixed point iteration converges to a fixed point for $g(x)$, then this fixed point is a root for $f(x)$. 
	\item[(i)] Approximate a root for the function 
	Perform three steps of fixed point iteration 
	\end{enumerate}

	\item[(g)] Take $g(x) = x - \frac{f(x)}{f'(x)}$
	
	Show that Newton's method applied to a differentiable $f(x)$ is fixed point iteration for some function $g(x)$. Furthermore, show that if the fixed point iteration for your function $g(x)$ converges to a fixed point for $g$, then this fixed point is a root 
	
	Show that Newton's method is a type of fixed point iteration and that if Newton's method converges to a root for some function $f(x)$, it converges to a fixed point 
	\end{enumerate}
x^3+4x^2-10

%Suppose that we want to find a root $r$ of some differentiable function $f(x)$; that is, we want to find a solution $r$ equation $f(x)= 0$, where $f(x)$ is a differentiable function. One approach we saw was Newton's method. This began with some initial value $x_0$ and then created recursive approximations to $r$ via $x_{n+1} = x_n - \frac{f(x_n)}{f'(x_n)}$. \pspace
%
%If one appropriately generalizes the notion of differentiability, one can use this to solve simultaneous equations. First, one rewrites the system of equations to have the following form:
%	\[
%	\begin{cases}
%	f_1(x_1, x_2, \ldots, x_n)= 0 \\
%	f_2(x_1, x_2, \ldots, x_n)= 0 \\
%	\hfill \vdots \hfill \\
%	f_n(x_1, x_2, \ldots, x_n)= 0 \\
%	\end{cases}
%	\]
%That is, one reduces the problem to find a common root for the equations. Let $\mathbf{f}$ be the function $\mathbf{f}(\mathbf{x})= \big(f_1(\mathbf{x}), f_2(\mathbf{x}), \ldots, f_n(\mathbf{x}) \big)$. One then defines the multivariable Newton's method to begin with some initial value $\mathbf{x}_0= (x_{1,0}, x_{2,0}, \ldots, x_{n,0})$ and create recursive approximations to the root via
%	\[
%	\mathbf{x}_{n+1} = \mathbf{x} - D\mathbf{f}(\mathbf{x}_n)^{-1} \, \mathbf{f}(\mathbf{x}_n)
%	\]
%where $D\mathbf{f}(\mathbf{x})^{-1}$ is the inverse (assuming it exists) of the Jacobian matrix
%	\[
%	D\mathbf{f}(x_1, x_2, \ldots, x_n)= 
%	\begin{pmatrix}
%	\dfrac{\partial f_1}{\partial x_1} & \dfrac{\partial f_1}{\partial x_2} & \cdots & \dfrac{\partial f_1}{\partial x_n} \\
%	\dfrac{\partial f_2}{\partial x_1} & \dfrac{\partial f_2}{\partial x_2} & \cdots & \dfrac{\partial f_2}{\partial x_n} \\
%	\vdots & \vdots & \ddots & \vdots \\
%	\dfrac{\partial f_n}{\partial x_1} & \dfrac{\partial f_n}{\partial x_2} & \cdots & \dfrac{\partial f_n}{\partial x_n} 
%	\end{pmatrix}
%	\]
%Consider the intersection of the circle $x^2 + y^2= 10$ and the hyperbola $x^2 - 4y^2= 4$, shown below.
%	\begin{enumerate}[(a)]
%	\item Perform two iterations of the multivariable Newton's method with $\mathbf{x}_0= (x, y)= (2, 3)$. 
%	\item Find the exact solutions to this equation. 
%	\end{enumerate}
%	
%	\[
%	\fbox{
%	\begin{tikzpicture}[scale=1,every node/.style={scale=0.5}]
%	\begin{axis}[
%	grid=both,
%	axis lines=middle,
%	ticklabel style={fill=blue!5!white},
%	xmin= -4.5, xmax=4.5,
%	ymin= -4.5, ymax=4.5,
%	xtick={-4,-3,...,4},
%	ytick={-4,-3,...,4},
%	minor tick = {-5,-4.5,...,5},
%	xlabel=\(x\),ylabel=\(y\),
%	]
%	\addplot[line width= 0.02cm,samples=100,domain= 0:6.4] ({sqrt(10)*cos(deg(x))}, {sqrt(10)*sin(deg(x))});
%
%	\addplot[line width= 0.02cm,samples=50,domain= -0.9:0] ({-(x^2 + 1)/(x^2 - 1)}, {2*x/(x^2 - 1)});
%	\addplot[line width= 0.02cm,samples=50,domain= 0:0.9] ({-(x^2 + 1)/(x^2 - 1)}, {2*x/(x^2 - 1)});	
%	\addplot[line width= 0.02cm,samples=50,domain= -0.9:0] ({(x^2 + 1)/(x^2 - 1)}, {2*x/(x^2 - 1)});
%	\addplot[line width= 0.02cm,samples=50,domain= 0:0.9] ({(x^2 + 1)/(x^2 - 1)}, {2*x/(x^2 - 1)});
%	\end{axis}
%	\end{tikzpicture}
%	}
%	\]
	


\newpage {\itshape \hfill ( --- Problem 4 Continued --- ) \hfill} \newpage



% Problem 5
\problem{10} Watch 3Blue1Brown's video \href{https://www.youtube.com/watch?v=-RdOwhmqP5s}{From Newton’s method to Newton’s fractal (which Newton knew nothing about)} on YouTube and then being as detailed as possible, respond to the following:
	\begin{enumerate}[(a)]
	\item What application of finding roots for polynomials did the video discuss to motivate Newton's method?
	\item How did the domain for the functions $P(z)$ the video considers differ from what we did in class?
	\item What is the connection between Newton's method and fractals?
	\item What was interesting or surprising to you in the video?
	\end{enumerate}


\end{document}