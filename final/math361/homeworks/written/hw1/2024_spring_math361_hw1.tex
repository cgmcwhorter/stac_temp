\documentclass[11pt,letterpaper]{article}
\usepackage[lmargin=1in,rmargin=1in,tmargin=1in,bmargin=1in]{geometry}
\usepackage{../style/homework}
\setbool{quotetype}{true} % True: Side; False: Under
\setbool{hideans}{true} % Student: True; Instructor: False

% -------------------
\begin{document}

\homework{1: Due 02/01}{And I knew exactly what to do\dots but in a much more real sense, I had no idea what to do.}{Michael Scott, The Office}

% Problem 1
\problem{10} Showing all your work and fully justifying your reasoning, compute the following:
	\begin{2enumerate}
	\item $\ds \lim_{x \to -6} \dfrac{x^2 + 3x - 18}{x^2 + 5x - 6}$
	\item $\ds \lim_{n \to \infty} \dfrac{2n^2 - 35n + 17}{6n^2 + 19n - 49}$
	\item $\dfrac{d}{dx} \, \ln \left(x \cos x \right)$
	\item $\ds \int_0^1 \dfrac{x}{x + 1} \;dx$
	\end{2enumerate}



\newpage



% Problem 2
\problem{10} Recall that a sequence $\{ a_n \}$ is increasing if $a_{n+1} \geq a_n$ for all $n$ and the sequence is decreasing if $a_{n+1} \leq a_n$ for all $n$. A sequence $\{ a_n \}$ is called bounded above (below) if there exists $M \in \mathbb{R}$ such that $a_n \leq M$ ($a_n \geq M$) for all $n$. The \textit{Monotone Convergence Theorem} states the following: if $\{ a_n \}$ is either increasing or decreasing, i.e. is `monotone', and bounded above or below, respectively, then $\{ a_n \}$ converges. Now consider the sequence with $a_0= 2$ and given recursively via\dots
	\[
	a_{n+1}= \dfrac{1}{2} \left( a_n + \dfrac{5}{a_n} \right)
	\]

\begin{enumerate}[(a)]
\item Compute $a_1, a_2, a_3$. 
\item Compare your values in (a) to $\sqrt{5}$. What might you conjecture?
\item Explain why the Monotone Convergence Theorem implies that $\{ a_n \}$ has a limit. 
\item By (c), we know $L:= \ds \lim_{n \to \infty} a_n$ exists. Taking the limit in both sides of the recursive definition for $\{ a_n \}$, show that $L= \sqrt{5}$. 
\end{enumerate}



\newpage



% Problem 3
\problem{10} The Intermediate Value Theorem states the following: if $f(x)$ is continuous on $[a, b]$ and $f(a) < c < f(b)$, then there exists an $x_0 \in (a, b)$ such that $f(x_0)= c$. Consider the function $f(x)= x^2 - 3x + 4$ on the interval $[-1, 5]$. 
	\begin{enumerate}[(a)]
	\item Give a sketch of $f(x)$ on the interval $[-1, 6]$. 
	\item Explain why $f(x)$ is continuous. 
	\item Explain why there is a $x_0 \in [-1, 6]$ such that $f(x_0)= 14$. 
	\item Find the $x_0 \in [-1, 6]$ such that $f(x_0)= 14$. 
	\end{enumerate}



\newpage



% Problem 4
\problem{10} The Mean Value Theorem states the following: if $f(x)$ is continuous on $[a, b]$ and differentiable on $(a, b)$, then there exists $c \in (a, b)$ such that $f(b) - f(a)= f'(c) \big( b - a \big)$. Consider the function $f(x)= x^3 + x^2 - 4x - 5$. Find the values $c \in [-1, 4]$ that satisfy the Mean Value Theorem. 


\end{document}