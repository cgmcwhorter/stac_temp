\documentclass[11pt,letterpaper]{article}
\usepackage[lmargin=1in,rmargin=1in,bmargin=1in,tmargin=1in]{geometry}
\usepackage{quiz}


% -------------------
% Content
% -------------------
\begin{document}
\thispagestyle{title}

% Quiz 1
\quizsol \textit{True/False}: There must be a solution to the equation $x \left( e^{2x} - 2 \right)= 15$.  \pspace

\sol The statement is \textit{true}. The equation $x \left( e^{2x} - 2 \right)= 15$ has a solution if and only if the equation $x \left( e^{2x} - 2 \right) - 15= 0$ has a solution. But the equation $x \left( e^{2x} - 2 \right) - 15= 0$ has a solution if and only if the function $f(x)= x \left( e^{2x} - 2 \right) - 15$ has a root. The function $f(x)$ is a continuous function---being the composition, sum/difference, and product of continuous functions. We know that $f(0)= 0 (1 - 2) - 15= -15 < 0$. Furthermore, there are clearly values for which $f(x)$ is positive because $\ds \lim_{x \to \infty} f(x)= \infty$. But then by the Intermediate Value Theorem, there must be a value $x_0 \in [-15, \infty)$ such that $f(x_0)= 0$. This proves that $f(x)$ has a root; equivalently, that $x \left( e^{2x} - 2 \right)= 15$ has a solution. 




















\end{document}