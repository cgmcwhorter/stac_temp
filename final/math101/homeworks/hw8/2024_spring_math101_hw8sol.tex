\documentclass[11pt,letterpaper]{article}
\usepackage[lmargin=1in,rmargin=1in,tmargin=1in,bmargin=1in]{geometry}
\usepackage{../style/homework}
\setbool{quotetype}{true} % True: Side; False: Under
\setbool{hideans}{false} % Student: True; Instructor: False

% -------------------
% Content
% -------------------
\begin{document}

\homework{8: Due 02/19}{It's always helpful to learn from your mistakes because then your mistakes seem worthwhile.}{Gary Marshall}

% Problem 1
\problem{10} Showing all your work, compute the following:
	\begin{enumerate}[(a)]
	\item The perimeter of a rectangle that is $2 \times 8$.
	\item The circumference of a circle with diameter 7.
	\item The perimeter of a square with side measure 6.
	\item The perimeter of a 3-4-5 right triangle. 
	\end{enumerate} \pspace

\sol 
\begin{enumerate}[(a)]
\item 
	\[
	P= 2\ell + 2w= 2(2) + 2(8)= 4 + 16= 20 
	\] \pspace

\item 
	\[
	C= 2 \pi r= 2 \pi \cdot \dfrac{7}{2}= 7\pi \approx 21.9911
	\] \pspace

\item 
	\[
	P= 4s= 4(6)= 24 
	\] \pspace

\item 
	\[
	P= a + b + c= 3 + 4 + 5= 12
	\]
\end{enumerate}



\newpage



% Problem 2
\problem{10} Showing all your work, compute the following:
	\begin{enumerate}[(a)]
	\item The area of a rectangle that is $2 \times 8$.
	\item The area of a circle with diameter 7.
	\item The area of a square with side measure 6.
	\item The area of a 3-4-5 right triangle. 
	\end{enumerate} \pspace

\sol 
\begin{enumerate}[(a)]
\item 
	\[
	A= \ell w= 2 \cdot 8= 16
	\] \pspace

\item 
	\[
	A= \pi r^2= \pi \left( \dfrac{7}{2} \right)^2= \pi \cdot \dfrac{49}{4}= \dfrac{49\pi}{4} \approx 38.4845
	\] \pspace

\item 
	\[
	A= s^2= 6^2= 36
	\] \pspace

\item 
	\[
	A= \frac{1}{2} \,bh= \frac{1}{2} \cdot 3 \cdot 4= 6
	\]
\end{enumerate}



\newpage



% Problem 3
\problem{10} Showing all your work, compute the following:
	\begin{enumerate}[(a)]
	\item The volume of a rectangular box that is $5 \times 4 \times 10$
	\item The surface area of a rectangular box that is $5 \times 4 \times 10$.
	\item The volume of a sphere with radius 4.
	\end{enumerate} \pspace

\sol 
\begin{enumerate}[(a)]
\item 
	\[
	V= \ell w h= 5 \cdot 4 \cdot 10= 200
	\] \pspace

\item 
	\[
	S= 2 \ell w + 2 \ell h + 2 wh= 2(5)4 + 2(5)10 + 2(4)10= 40 + 100 + 80= 220
	\] \pspace

\item 
	\[
	V= \dfrac{4}{3}\, \pi r^3= \dfrac{4}{3} \, \pi \cdot 4^3= \dfrac{256 \pi}{3} \approx 268.083
	\]
\end{enumerate}


\end{document}