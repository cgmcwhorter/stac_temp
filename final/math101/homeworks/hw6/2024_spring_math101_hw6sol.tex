\documentclass[11pt,letterpaper]{article}
\usepackage[lmargin=1in,rmargin=1in,tmargin=1in,bmargin=1in]{geometry}
\usepackage{../style/homework}
\setbool{quotetype}{true} % True: Side; False: Under
\setbool{hideans}{false} % Student: True; Instructor: False

% -------------------
% Content
% -------------------
\begin{document}

\homework{6: Due 02/12}{I don't know, Marge. Trying is the first step towards failure.}{Homer Simpson, The Simpsons}

% Problem 1
\problem{10} Showing all your work, compute the following:
	\begin{enumerate}[(a)]
	\item 86\% of 920
	\item 4\% of 77
	\item 180\% of 9
	\end{enumerate} \pspace

\sol We use the fact that to find a \% of a number $N$, ones computes $N \cdot \%_d$, where $\%_d$ is the percentage written as a decimal. 
 
\begin{enumerate}[(a)]
\item 
	\[
	\text{86\% of 920}= 920(0.86)= 791.2
	\]

\item 
	\[
	\text{4\% of 77}= 77(0.04)= 3.08
	\] \pspace

\item 
	\[
	\text{180\% of 9}= 9(1.80)= 16.2
	\]
\end{enumerate}



\newpage



% Problem 2
\problem{10} Showing all your work, compute the following:
	\begin{enumerate}[(a)]
	\item 600 decreased by 45\%
	\item 88 increased by 34\%
	\item 1,450 increased by 111\%
	\end{enumerate} \pspace

\sol We use the fact that to increase or decrease a number $N$ by a percentage \%, ones computes $N(1 \pm \%_d)$, where $\%_d$ is the percentage as a decimal and one chooses `$+$' if a percentage increase and `$-$' if a percentage decrease. \pspace

\begin{enumerate}[(a)]
\item 
	\[
	\text{600 decreased by 45\%}= 600 (1 - 0.45)= 600(0.55)= 330
	\] \pspace

\item 
	\[
	\text{88 increased by 34\%}= 88 (1 + 0.34)= 88(1.34)= 117.92
	\] \pspace

\item 
	\[
	\text{1,450 increased by 111\%}= 1,\!450 (1 + 1.11)= 1450(2.11)= 3,\!059.5
	\] 
\end{enumerate}



\newpage



% Problem 3
\problem{10} Your biology class course grade is determined by the following components: \par
	\begin{table}[h]
	\centering
	\begin{tabular}{lr}
	Homeworks & 60\% \\
	Quizzes & 5\% \\
	Midterm & 15\% \\
	Final & 20\% \\
	\end{tabular}
	\end{table} \par
Suppose that your homework average was 91\%, your quiz average was 81\%, your midterm average was 79\%, and your final exam average was 86\%.
	\begin{enumerate}[(a)]
	\item Compute your course average.
	\item If the final exam had not yet occurred, i.e. you had not yet received the 86\%, but all the other course grades were as listed above, then what is your current course average?
	\end{enumerate} \pspace

\sol 
\begin{enumerate}[(a)]
\item We have\dots
	\[
	\begin{aligned}
	\text{Overall Course Average}&= \dfrac{\sum w_i x_i}{\sum w_i} \\[0.3cm]
	&= \dfrac{0.60 \cdot 0.91 + 0.05 \cdot 0.81 + 0.15 \cdot 0.79 + 0.20 \cdot 0.86}{0.60 + 0.05 + 0.15 + 0.20} \\[0.3cm]
	&= \dfrac{0.546 + 0.0405 + 0.1185 + 0.172}{0.60 + 0.05 + 0.15 + 0.20} \\[0.3cm]
	&= \dfrac{0.877}{1} \\[0.3cm]
	&= 0.877 \\[0.3cm]
	&= 87.7\%
	\end{aligned}
	\] \pspace

\item We have\dots
	\[
	\begin{aligned}
	\text{Current Course Average}&= \dfrac{\sum w_i x_i}{\sum w_i} \\[0.3cm]
	&= \dfrac{0.60 \cdot 0.91 + 0.05 \cdot 0.81 + 0.15 \cdot 0.79}{0.60 + 0.05 + 0.15 + 0.20} \\[0.3cm]
	&= \dfrac{0.546 + 0.0405 + 0.1185}{0.60 + 0.05 + 0.15} \\[0.3cm]
	&= \dfrac{0.705}{0.80} \\[0.3cm]
	&= 0.88125 \\[0.3cm]
	&= 88.125\%
	\end{aligned}
	\]  
\end{enumerate}


\end{document}