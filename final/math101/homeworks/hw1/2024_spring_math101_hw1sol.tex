\documentclass[11pt,letterpaper]{article}
\usepackage[lmargin=1in,rmargin=1in,tmargin=1in,bmargin=1in]{geometry}
\usepackage{../style/homework}
\setbool{quotetype}{true} % True: Side; False: Under
\setbool{hideans}{false} % Student: True; Instructor: False

\DeclareMathOperator{\lcm}{lcm}

% -------------------
% Content
% -------------------
\begin{document}

\homework{1: Due 01/24}{When I was young I observed that nine out of every ten things I did were fails, so I did ten times more work.}{George Bernard Shaw}

% Problem 1
\problem{10} Complete the following:
	\begin{enumerate}[(a)]
	\item List all the divisors of 64.
	\item List all the nonnegative multiples of 14 less than 130. 
	\end{enumerate} \pspace

\sol 
\begin{enumerate}[(a)]
\item We have $64= 2^6$. Therefore, the divisors of 64 are all the powers of 2 from 0 to 6. The divisors of 64 are then 1, 2, 4, 8, 16, 32, 64. \pspace

\item The multiples of 14 are the integers of the form $14k$, where $k$ is an integer. If the multiples are to be nonnegative, we need $k \geq 0$. So we want the integers of the form $14k$, where $k \geq 0$ and $14k \leq 130$, i.e. $k \leq \frac{130}{14} \approx 9.29$---which implies we need $k= 0, 1, \ldots, 9$. The nonnegative multiples of 14 that are less than 130 are then 14, 28, 42, 56, 70, 84, 98, 112, 126. 
\end{enumerate}



\newpage



% Problem 2
\problem{10} Showing all your work, find the prime factorizations of the following:
	\begin{enumerate}[(a)]
	\item $30$
	\item $52$
	\item $61$
	\item $110$
	\item $315$
	\end{enumerate} \pspace

\sol 
\begin{enumerate}[(a)]
\item $30= 2 \cdot 3 \cdot 5$
	\[
	\begin{tikzpicture}
	\node at (0,0) (n) {30};
	\node at (-1.5,-1) (a) {3};
	\node at (1.5,-1) (b) {10};
	\node at (1,-2) (e) {2};
	\node at (2,-2) (f) {5};
	
	\draw[line width=0.03cm] (n) -- (a);
	\draw[line width=0.03cm] (n) -- (b);
	\draw[line width=0.03cm] (b) -- (e);
	\draw[line width=0.03cm] (b) -- (f);
	\end{tikzpicture}
	\]

\item $52= 2^2 \cdot 13$
	\[
	\begin{tikzpicture}
	\node at (0,0) (n) {52};
	\node at (-1.5,-1) (a) {4};
	\node at (1.5,-1) (b) {13};
	\node at (-2,-2) (c) {2};
	\node at (-1,-2) (d) {2};
	
	\draw[line width=0.03cm] (n) -- (a);
	\draw[line width=0.03cm] (n) -- (b);
	\draw[line width=0.03cm] (a) -- (c);
	\draw[line width=0.03cm] (a) -- (d);
	\end{tikzpicture}
	\] \pspace

\item The integer 61 is prime. Therefore, $61= 61^1$. \pspace

\item $110= 2 \cdot 5 \cdot 11$
	\[
	\begin{tikzpicture}
	\node at (0,0) (n) {110};
	\node at (-1.5,-1) (a) {11};
	\node at (1.5,-1) (b) {10};
	\node at (1,-2) (e) {2};
	\node at (2,-2) (f) {5};
	
	\draw[line width=0.03cm] (n) -- (a);
	\draw[line width=0.03cm] (n) -- (b);
	\draw[line width=0.03cm] (b) -- (e);
	\draw[line width=0.03cm] (b) -- (f);
	\end{tikzpicture}
	\]

\item $315= 3^2 \cdot 5 \cdot 7$
	\[
	\begin{tikzpicture}
	\node at (0,0) (n) {315};
	\node at (-1.5,-1) (a) {5};
	\node at (1.5,-1) (b) {63};
	\node at (1,-2) (e) {9};
	\node at (2,-2) (f) {7};
	\node at (0.5,-3) (g) {3};
	\node at (1.5,-3) (h) {3};
	
	\draw[line width=0.03cm] (n) -- (a);
	\draw[line width=0.03cm] (n) -- (b);
	\draw[line width=0.03cm] (b) -- (e);
	\draw[line width=0.03cm] (b) -- (f);
	\draw[line width=0.03cm] (e) -- (g);
	\draw[line width=0.03cm] (e) -- (h);
	\end{tikzpicture}
	\]
\end{enumerate}



\newpage



% Problem 3
\problem{10} Using divisibility criterion, answer the following:
	\begin{enumerate}[(a)]
	\item Does 2 divide 1749? Explain.
	\item Does 3 divide 444444? Explain.
	\item Does 4 divide 793621? Explain.
	\item Does 5 divide 202122? Explain.
	\item Does 9 divide 444444? Explain. 
	\end{enumerate} \pspace

\sol 
\begin{enumerate}[(a)]
\item We know that an integer is divisible by 2 if and only if it is even. Because 1749 is not even, it is not divisible by 2. \pspace

\item We know that an integer is divisible by 3 if and only if the sum of the digits is divisible by 3. We know $4 + 4 + 4 + 4 + 4 + 4= 6 \cdot 4= 24$, which is divisible by 3. Therefore, 444444 is divisible by 3. \pspace

\item We know that an integer is divisible by 4 if and only if the last two digits are divisible by 4. The last two digits of 793621 are 21. Because 21 is not divisible by 4, 793621 is not divisible by 4. \pspace

\item We know that an integer is divisible by 5 if and only if ends in a 0 or 5. Because 202122 ends in a 2, it is not divisible by 5. \pspace

\item We know that an integer is divisible by 3 if and only if the sum of the digits is divisible by 9. We know $4 + 4 + 4 + 4 + 4 + 4= 6 \cdot 4= 24$, which is not divisible by 9. Therefore, 444444 is not divisible by 9. 
\end{enumerate}



\newpage



% Problem 4
\problem{10} Showing all your work, compute the following:
	\begin{enumerate}[(a)]
	\item $\gcd(12, 66)$
	\item $\lcm(12, 66)$
	\item $\gcd(2^{30} \cdot 3^{15} \cdot 7^{10} \cdot 11^3,\, 2^{60} \cdot 3^{5} \cdot 5^{14} \cdot 13^2)$
	\item $\lcm(2^{30} \cdot 3^{15} \cdot 7^{10} \cdot 11^3,\, 2^{60} \cdot 3^{5} \cdot 5^{14} \cdot 13^2)$
	\end{enumerate} \pspace

\sol We use the fact that the gcd of a collection of numbers is the product of the smallest possible power of the primes found in their prime factorizations and the lcm is the product of the largest possible power of the primes found in their prime factorizations.  

\begin{enumerate}[(a)]
\item 
	\[
	\gcd(12, 66)= \gcd(2^2 \cdot 3, 2 \cdot 3 \cdot 11)= 2 \cdot 3= 6
	\] \pspace

\item 
	\[
	\lcm(12, 66)= \lcm(2^2 \cdot 3, 2 \cdot 3 \cdot 11)= 2^2 \cdot 3 \cdot 11= 132
	\] \pspace
	
\item 
	\[
	\gcd(2^{30} \cdot 3^{15} \cdot 7^{10} \cdot 11^3,\, 2^{60} \cdot 3^{5} \cdot 5^{14} \cdot 13^2)= 2^{30} \cdot 3^5= 260,\!919,\!263,\!232
	\] \pspace

\item 
	\[
	\begin{gathered}
	\lcm(2^{30} \cdot 3^{15} \cdot 7^{10} \cdot 11^3,\, 2^{60} \cdot 3^{5} \cdot 5^{14} \cdot 13^2)= 2^{60} \cdot 3^{15} \cdot 5^{14} \cdot 7^{10} \cdot 11^3 \cdot 13^2= \\
	6,\!415,\!696,\!063,\!883,\!373,\!935,\!607,\!978,\!763,\!537,\!612,\!800,\!000,\!000,\!000,\!000
	\end{gathered}
	\]
\end{enumerate}


\end{document}