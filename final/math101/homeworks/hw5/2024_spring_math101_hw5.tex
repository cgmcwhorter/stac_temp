\documentclass[11pt,letterpaper]{article}
\usepackage[lmargin=1in,rmargin=1in,tmargin=1in,bmargin=1in]{geometry}
\usepackage{../style/homework}
\setbool{quotetype}{true} % True: Side; False: Under
\setbool{hideans}{true} % Student: True; Instructor: False

% -------------------
% Content
% -------------------
\begin{document}

\homework{5: Due 02/07}{I don't wanna have to bring this up\dots But it's my turn to take a selfish.}{David Rose, Schitt's Creek}

% Problem 1
\problem{10} Express each of the following decimal numbers as a rational number in simplest form and express each of the rational numbers as a decimal number:
	\begin{enumerate}[(a)]
	\item $\frac{1}{11}$
	\item $1.12$
	\item $\frac{71}{5}$
	\end{enumerate}



\newpage



% Problem 2
\problem{10} Showing all your work, express the number $0.\overline{123}$ as a rational number. 



\newpage



% Problem 3
\problem{10} Perform the following operations in $\mathbb{C}$:
	\begin{enumerate}[(a)]
	\item $(6 - 8i) + (4 + 2i)$
	\item $(13 - i) - (15 - 8i)$
	\item $(5 + i)(6 - 2i)$
	\item $\dfrac{1 + 2i}{3 + i}$
	\end{enumerate}



\newpage



% Problem 4
\problem{10} Every quadratic equation $ax^2 + bx + c= 0$ has exactly two (not necessarily distinct) solutions when the solutions are allowed to be complex numbers. For instance, the equation $2x^2 - 20x + 68= 0$ has as its solutions $5 \pm 3i$. Verify that $5 - 3i$ is a solution to this equation. 


\end{document}