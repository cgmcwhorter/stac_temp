\documentclass[11pt,letterpaper]{article}
\usepackage[lmargin=1in,rmargin=1in,tmargin=1in,bmargin=1in]{geometry}
\usepackage{../style/homework}
\setbool{quotetype}{true} % True: Side; False: Under
\setbool{hideans}{false} % Student: True; Instructor: False

% -------------------
% Content
% -------------------
\begin{document}

\homework{11: Due 03/06}{Wouldst thou like to live deliciously?}{Satan, The Witch}

% Problem 1
\problem{10} A recipe calls for 2,280~g of flour to bake five loaves of bread. 
	\begin{enumerate}[(a)]
	\item Using a proportion, find how many grams are required to bake 9 loaves of bread.
	\item How many grams of flour are required per loaf of break---on average?
	\item Use (b) to answer (a). 
	\end{enumerate} \pspace

\sol 
\begin{enumerate}[(a)]
\item Let $x$ be the number of grams required to make 9~loaves of bread. We have\dots
	\[
	\begin{gathered}
	\dfrac{2,\!280 \text{ g}}{5 \text{ loaves}}= \dfrac{x}{9 \text{ loaves}} \\[0.3cm]
	5x= 20,\!520 \\[0.3cm]
	x= 4,\!104 \text{ g}
	\end{gathered}
	\] \pspace

\item 
	\[
	\dfrac{2,\!280 \text{ g}}{5 \text{ loaves}}= 456 \text{ g/loaf}
	\] \pspace

\item 
	\[
	\text{Grams Flour Required}= \text{Grams per loaf} \cdot \text{\# Loaves}= 456 \text{ g/loaf} \cdot 9= 4,\!104 \text{ g}
	\]
\end{enumerate}



\newpage



% Problem 2
\problem{10} Showing all your work, answer the following:
	\begin{enumerate}[(a)]
	\item A team wins 60 games and loses 20 games, what is the win-loss ratio?
	\item If eight cases of a food costs \$55, what should a dozen cases cost?
	\item If it takes 1.5~gallons of paint to cover 580~sq ft, how much paint would be required to cover 2,000~sq ft?
	\end{enumerate} \pspace

\sol 
\begin{enumerate}[(a)]
\item The win-loss ratio is $60 : 20$, this is $\frac{60}{20}= 3= \frac{3}{1}$. Therefore, the win-loss ratio is $3 : 1$. \pspace

\item Let $x$ be the cost of a dozen cases of food. Then\dots
	\[
	\begin{gathered}
	\dfrac{8 \text{ cases}}{\$55}= \dfrac{12 \text{ cases}}{x} \\[0.3cm]
	8x= 660 \\[0.3cm]
	x= \$82.50
	\end{gathered}
	\]
Alternatively, the average cost per case is $\frac{\$55}{8}= \$6.875 \text{/case}$. But then the cost of a dozen cases should be $\$6.875 \text{/case} \cdot 12 \text{ cases}= \$82.50$. \pspace

\item Let $x$ be the amount of paint required to cover 2,000 sq ft. Then\dots
	\[
	\begin{gathered}
	\dfrac{1.5 \text{ gallons}}{580 \text{ sq ft}}= \dfrac{x}{2,\!000 \text{ sq ft}} \\[0.3cm]
	580x= 3000 \\[0.3cm]
	x= 5.17 \text{ gallons}
	\end{gathered}
	\]
Alternatively, the average amount of area covered per gallon is $\frac{580 \text{ sq ft}}{1.5 \text{ gal}}= 386.667 \text{ sq ft/gal}$. But then we want the amount of paint, say $x$, such that $386.667x= 2,\!000$. But then $x \approx 5.17 \text{ gallons}$. 
\end{enumerate}



\newpage



% Problem 3
\problem{10} Alexis can prepare 45 candy bags in an hour for a children's party while Casey can prepare 50 candy bags per hour. Working together, how long should it take them to prepare 560 candy bags by working together? \pspace

\sol Let $t$ be the amount of time it takes Alexis and Casey to prepare 560~candy bags working together. Let $r_A$, $r_C$, and $r_T$ be the number of candy bags that can be prepared by Alexis, Casey, and Alexis \& Casey, respectively. We then have\dots
	\[
	\begin{gathered}
	\text{Total Bags}= \text{Rate} \cdot \text{Time} \\[0.3cm]
	560= r_T \cdot t \\[0.3cm]
	560= (r_A + r_C) t \\[0.3cm]
	560= (45 + 60)t \\[0.3cm]
	560= 105t \\[0.3cm]
	t= 5.33
	\end{gathered}
	\]
Therefore, working together, it will take Alexis and Casey 5.33~hours to prepare 560~candy bags, i.e. 5~hours and 20~minutes. 


\end{document}