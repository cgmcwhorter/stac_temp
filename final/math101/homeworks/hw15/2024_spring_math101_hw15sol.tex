\documentclass[11pt,letterpaper]{article}
\usepackage[lmargin=1in,rmargin=1in,tmargin=1in,bmargin=1in]{geometry}
\usepackage{../style/homework}
\setbool{quotetype}{false} % True: Side; False: Under
\setbool{hideans}{false} % Student: True; Instructor: False

% -------------------
% Content
% -------------------
\begin{document}

\homework{15: Due 04/08}{Sometimes I think I have felt everything I'm ever gonna feel. And from here on out, I'm not gonna feel anything new. Just lesser versions of what I've already felt.}{Theodore Twombly, Her}

% Problem 1
\problem{10} Solve the following system of equations:
	\[
	\begin{aligned}
	2x + y&= 1 \\
	3x - y&= -11
	\end{aligned}
	\] \pspace

\sol There are two elementary approaches: \pspace

\noindent {\itshape Substitution.} Solving for $y$ in the first equation, we have $y= 1 - 2x$. Using this in the second equation, we have\dots
	\[
	\begin{gathered}
	3x - y= -11 \\[0.3cm]
	3x - (1 - 2x)= -11 \\[0.3cm]
	3x - 1 + 2x= -11 \\[0.3cm]
	5x= -10 \\[0.3cm]
	x= -2
	\end{gathered}
	\]
But then we have $y= 1 - 2x= 1 - 2(-2)= 1 + 4= 5$. Therefore, the lines intersect at the point $(x, y)= (-2, 5)$. \pspace

\noindent {\itshape Elimination.} We add the two equations to eliminate $y$. This obtains $5x= -10$, which immediately implies $x= -2$. Using this in the first equation, we have\dots
	\[
	\begin{gathered}
	2x + y= 1 \\[0.3cm]
	2(-2) + y= 1 \\[0.3cm]
	-4 + y= 1 \\[0.3cm]
	y= 5
	\end{gathered}	
	\]
Therefore, the lines intersect at $(x, y)= (-2, 5)$. 



\newpage



% Problem 2
\problem{10} Solve the following system of equations:
	\[
	\begin{cases}
	4x + 6y= 6 \\
	5x + 2y= 13
	\end{cases}
	\]

\sol There are two elementary approaches: \pspace

\noindent {\itshape Substitution.} Solving for $x$ in the first equation, we have\dots
	\[
	\begin{aligned}
	4x + 6y= 6 \\
	4x= 6 - 6y \\
	x= \dfrac{6 - 6y}{4} \\
	x= \dfrac{3 - 3y}{2}
	\end{aligned}
	\] 
Using this in the second equation, we have\dots
	\[
	\begin{gathered}
	5x + 2y= 13 \\
	5 \cdot \dfrac{3 - 3y}{2} + 2y= 13 \\
	\dfrac{15 - 15y}{2} + 2y= 13 \\
	2 \left( \dfrac{15 - 15y}{2} + 2y \right)= 2 \cdot 13 \\
	15 - 15y + 4y= 26 \\
	15 - 11y= 26 \\
	-11y= 11 \\
	y= -1
	\end{gathered}
	\]
But then we have $x= \frac{3 - 3(-1)}{2}= \frac{3 + 3}{2}= \frac{6}{2}= 3$. Therefore, the lines intersect at the point $(x, y)= (3, -1)$. \pspace

\noindent {\itshape Elimination.} We multiply the second equation by $-3$. This gives us the following system of equations:
	\[
	\begin{cases}
	4x + 6y= 6 \\
	-15x - 6y= -39
	\end{cases}
	\]
Adding these equations, we obtain $-11x= -33$. But then $x= \frac{-33}{-11}= 3$. Using this in the first equation, we have\dots
	\[
	\begin{gathered}
	4x + 6y= 6 \\
	4(3) + 6y= 6 \\
	12 + 6y= 6 \\
	6y= -6 \\
	y= -1
	\end{gathered}
	\]
Therefore, the lines intersect at $(x, y)= (3, -1)$. 


\end{document}