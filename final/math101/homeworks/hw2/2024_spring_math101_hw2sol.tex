\documentclass[11pt,letterpaper]{article}
\usepackage[lmargin=1in,rmargin=1in,tmargin=1in,bmargin=1in]{geometry}
\usepackage{../style/homework}
\setbool{quotetype}{false} % True: Side; False: Under
\setbool{hideans}{false} % Student: True; Instructor: False

\usepackage{cancel} % Use Cancel

% -------------------
% Content
% -------------------
\begin{document}

\homework{2: Due 01/29}{All opinions are not equal. Some are a very great deal more robust, sophisticated and well supported in logic and argument than others.}{Douglas Adams}

% Problem 1
\problem{10} Showing all your work, reduce the following rational numbers:
	\begin{enumerate}[(a)]
	\item $\frac{36}{20}$
	\item $\frac{165}{44}$
	\item $\frac{23}{5}$
	\item $\frac{16}{80}$
	\item $\frac{70}{105}$
	\end{enumerate} \pspace

\sol 
\begin{enumerate}[(a)]
\item 
	\[
	\dfrac{36}{20}= \dfrac{9 \cdot 4}{5 \cdot 4}= \dfrac{9 \cdot \cancel{4}}{5 \cdot \cancel{4}}= \dfrac{9}{5}
	\] \pspace

\item 
	\[
	\dfrac{165}{44}= \dfrac{11 \cdot 15}{11 \cdot 4}= \dfrac{\cancel{11} \cdot 15}{\cancel{11} \cdot 4}= \dfrac{15}{4}
	\] \pspace
 
\item 
	\[
	\dfrac{23}{5}= \dfrac{23}{5}
	\] \pspace

\item 
	\[
	\dfrac{16}{80}= \dfrac{16}{5 \cdot 16}= \dfrac{\cancel{16}}{5 \cdot \cancel{16}}= \dfrac{1}{5}
	\] \pspace

\item 
	\[
	\dfrac{70}{105}= \dfrac{35 \cdot 2}{35 \cdot 3}= \dfrac{\cancel{35} \cdot 2}{\cancel{35} \cdot 3}= \dfrac{2}{3}
	\]
\end{enumerate}



\newpage



% Problem 2
\problem{10} Showing all your work and simplifying as much as possible, compute the following:
	\begin{enumerate}[(a)]
	\item $\frac{3}{7} + \frac{5}{2}$
	\item $\frac{11}{3}  - \frac{5}{33}$
	\item $\frac{12}{25} + \frac{7}{10}$
	\item $\frac{18}{5} - \frac{10}{11}$
	\item $\frac{5}{3} + \frac{11}{2} - \frac{1}{7}$
	\end{enumerate} \pspace

\sol 
\begin{enumerate}[(a)]
\item 
	\[
	\dfrac{3}{7} + \dfrac{5}{2}= \dfrac{3}{7} \cdot \dfrac{2}{2}+ \dfrac{5}{2} \cdot \dfrac{7}{7}= \dfrac{6}{14} + \dfrac{35}{14}= \dfrac{6 + 35}{14}= \dfrac{41}{14}
	\] \pspace

\item 
	\[
	\dfrac{11}{3}  - \dfrac{5}{33}= \dfrac{11}{3}  \cdot \dfrac{11}{11} - \dfrac{5}{33}= \dfrac{121}{33} - \dfrac{5}{33}= \dfrac{121 - 5}{33}= \dfrac{116}{33}
	\] \pspace
 
\item 
	\[
	\dfrac{12}{25} + \dfrac{7}{10}= \dfrac{12}{25} \cdot \dfrac{2}{2} + \dfrac{7}{10} \cdot \dfrac{5}{5}= \dfrac{24}{50} + \dfrac{35}{50}= \dfrac{24 + 35}{50}= \dfrac{59}{50}
	\] \pspace

\item 
	\[
	\dfrac{18}{5} - \dfrac{10}{11}= \dfrac{18}{5} \cdot \dfrac{11}{11} - \dfrac{10}{11} \cdot \dfrac{5}{5}= \dfrac{198}{55} - \dfrac{50}{55}= \dfrac{198 - 50}{55}= \dfrac{148}{55}
	\] \pspace

\item 
	\[
	\dfrac{5}{3} + \dfrac{11}{2} - \dfrac{1}{7}= \dfrac{5}{3} \cdot \dfrac{14}{14} + \dfrac{11}{2} \cdot \dfrac{21}{21} - \dfrac{1}{7} \cdot \dfrac{6}{6}= \dfrac{70}{42} + \dfrac{231}{42} - \dfrac{6}{42}= \dfrac{70 + 231 - 6}{42}= \dfrac{295}{42}
	\] 
\end{enumerate}



\newpage



% Problem 3
\problem{10} Showing all your work and simplifying as much as possible, compute the following:
	\begin{enumerate}[(a)]
	\item $\frac{6}{55} \cdot \frac{44}{21}$
	\item $\dfrac{\frac{49}{12}}{\frac{7}{20}}$
	\item $\dfrac{\frac{11}{5}}{\frac{3}{26}}$
	\item $\frac{30}{18} \cdot \frac{27}{70}$
	\item $\dfrac{\frac{180}{175}}{\frac{30}{98}}$
	\end{enumerate} \pspace

\sol 
\begin{enumerate}[(a)]
\item 
	\[
	\dfrac{6}{55} \cdot \dfrac{44}{21}= \dfrac{\cancel{6}^{\,2}}{\cancel{55}^{\,5}} \cdot \dfrac{\cancel{44}^{\,4}}{\cancel{21}^{\,7}}= \dfrac{2}{5} \cdot \dfrac{4}{7}= \dfrac{8}{35}
	\] \pspace

\item 
	\[
	\dfrac{\;\;\dfrac{49}{12}\;\;}{\dfrac{7}{20}}= \dfrac{49}{12} \cdot \dfrac{20}{7}= \dfrac{\cancel{49}^{\,7}}{\cancel{12}^{\,3}} \cdot \dfrac{\cancel{20}^{\,5}}{\cancel{7}^{\,1}}= \dfrac{7}{3} \cdot \dfrac{5}{1}= \dfrac{35}{3}
	\] \pspace

\item 
	\[
	\dfrac{\;\;\dfrac{11}{5}\;\;}{\dfrac{3}{26}}= \dfrac{11}{5} \cdot \dfrac{26}{3}= \dfrac{286}{15}
	\] \pspace

\item 
	\[
	\dfrac{30}{18} \cdot \dfrac{27}{70}= \dfrac{\cancel{30}^{\,3}}{\cancel{18}^{\,2}} \cdot \dfrac{\cancel{27}^{\,3}}{\cancel{70}^{\,7}}= \dfrac{3}{2} \cdot \dfrac{3}{7}= \dfrac{9}{14}
	\] \pspace

\item 
	\[
	\dfrac{\;\;\dfrac{180}{175}\;\;}{\dfrac{30}{98}}= \dfrac{180}{175} \cdot \dfrac{98}{30}= \dfrac{\cancel{180}^{\,6}}{\cancel{175}^{\,25}} \cdot \dfrac{\cancel{98}^{\,14}}{\cancel{30}^{\,1}}= \dfrac{6}{25} \cdot \dfrac{14}{1}= \dfrac{84}{25}
	\] 
\end{enumerate}



\newpage



% Problem 4
\problem{10} Explain whether the following statements are true or false:
	\begin{enumerate}[(a)]
	\item All real numbers are rational. 
	\item All rational numbers have a decimal expansion which terminates. 
	\item There is only one way to express a rational number. 
	\item All rational numbers are numbers between 0 and 1. 
	\end{enumerate} \pspace

\sol 
\begin{enumerate}[(a)]
\item The statement is \textit{false}. For instance, the real numbers $\sqrt{2}$, $\pi$, $e$, $0.1234567891011121314\cdots$ are all irrational. \pspace

\item The statement is \textit{false}. A rational number is a real number that can be written in the form $\frac{a}{b}$, where $a, b$ are integers. A real number is rational if and only if its decimal expansion terminates or repeats. We know that $\frac{1}{3}$ is rational and $\frac{1}{3}= 0.\overline{3}$ has a decimal expansion that does not terminate. \pspace

\item The statement is \textit{false}. For instance, $\frac{5}{10}= \frac{1}{2}= 0.5$ has infinitely many representations. \pspace

\item The statement is \textit{false}. For instance, $\frac{13}{5}$ is rational and $\frac{13}{5}= 2.6 > 1$. 
\end{enumerate}


\end{document}