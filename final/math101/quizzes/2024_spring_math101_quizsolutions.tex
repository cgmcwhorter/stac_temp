\documentclass[11pt,letterpaper]{article}
\usepackage[lmargin=1in,rmargin=1in,bmargin=1in,tmargin=1in]{geometry}
\usepackage{quiz}

\DeclareMathOperator{\disc}{disc}	% Discriminant

% -------------------
% Content
% -------------------
\begin{document}
\thispagestyle{title}

% Quiz 1
\quizsol \textit{True/False}: The integer 131313 is prime. \pspace

\sol The statement is \textit{false}. We know that an integer $N$ is divisible by 3 if and only if the sum of its digits is divisible by 3. We know that $1 + 3 + 1 + 3 + 1 + 3= 12$ is divisible by 3. Therefore, 131313 cannot be prime. In fact, $131313= 3 \cdot 43771= 3 \cdot 7 \cdot 13 \cdot 37$. \pvspace{1.3cm}



% Quiz 2
\quizsol \textit{True/False}: Every rational number can be written as $\frac{a}{b}$, where $\gcd(a, b)= 1$. \pspace

\sol The statement is \textit{true}. By definition, a rational number $r$ is a number of the form $\frac{a}{b}$, where $a, b$ are integers and $b \neq 0$. Therefore, we can clearly write every rational number in the form $\frac{a}{b}$. Now can we impose the restriction that $\gcd(a, b)= 1$? Yes! By cancelling common factors from $a, b$, we can assure that the fraction is reduced, i.e. $\gcd(a, b)= 1$. In fact, we can always divide the numerator and denominator by $\gcd(a, b)$. After, we have $\gcd(a, b)= 1$. For instance, take $\frac{10}{15}$. We have $\gcd(10, 15)= 5$. But then $\frac{10}{15} \cdot \frac{1/5}{1/5}= \frac{2}{3}$ is reduced. \pvspace{1.3cm}



% Quiz 3
\quizsol \textit{True/False}: $\dfrac{(x^2)^3 x^5}{x^4}= x^6$ \pspace

\sol The statement is \textit{false}. Recall that $x^a \cdot x^b= x^{a + b}$, $(x^a)^b= x^{ab}$, and $\frac{x^a}{x^b}= x^{a - b}$. We then have\dots
	\[
	\dfrac{(x^2)^3 x^5}{x^4}= \dfrac{x^6 \cdot x^5}{x^4}= \dfrac{x^{11}}{x^4}= x^7
	\]
The mistake made was adding the powers in $(x^2)^3$ to obtain $x^5$ rather than multiplying the powers to obtain the correct $x^6$. \pvspace{1.3cm}



% Quiz 4
\quizsol \textit{True/False}: $\sqrt{\sqrt[3]{x^2}}= x^{2/5}$ \pspace

\sol The statement is \textit{false}. Recall that $\sqrt[n]{x^m}= x^{m/n}$ and $(x^a)^b= x^{ab}$. We then have\dots
	\[
	\sqrt{\sqrt[3]{x^2}}= \sqrt{x^{2/3}}= (x^{2/3})^{1/2}= x^{\frac{2}{3} \cdot \frac{1}{2}}= x^{1/3}= \sqrt[3]{x}
	\]
The mistake made was adding the denominators rather than multiplying the powers correctly, i.e. $\sqrt{\sqrt[3]{x^2}}= \big( (x^2)^{1/3} \big)^{1/2}= (x^2)^{1/5}= x^{2/5}$, which is incorrect. \pvspace{1.3cm}



% Quiz 5
\quizsol \textit{True/False}: $(1 - 3i)(2 + 5i)= 17 - i$ \pspace

\sol The statement is \textit{true}. Recall that $i^2= -1$. Then we have\dots
	\[
	(1 - 3i)(2 + 5i)= 1(2) + 1(5i) - 3i(2) - 3i(5i)= 2 + 5i - 6i - 15i^2= 2 - i - 15(-1)= 2 - i + 15= 17 - i
	\]





\newpage





% Quiz 6
\quizsol \textit{True/False}: If one increases 76 by 5\% five times sequentially, the result is $76(1 + 0.25)= 76(1.25)= 95$. \pspace

\sol The statement is \textit{false}. If we want to compute $N$ increased or decreased by a \% a total of $n$ times, we compute $N \cdot (1 \pm \%_d)^n$, where $\%_d$ is the percentage written as a decimal, $n$ is the number of times we apply the percentage increase/decrease, and we choose `$+$' if it is a percentage increase and choose `$-$' if it is a percentage decrease. Then to compute $76$ decreased by 5\% consecutively five times, we need take $N= 76$, $\%_d= 0.05$, and choose `$+$'. Therefore, we have\dots
	\[
	N \cdot (1 \pm \%_d)^n= 76 (1 + 0.05)^5= 76(1.05)^5= 76(1.27628)= 96.9974
	\]
From the $76(1.27628)$ portion from the computation above, we can see that increasing a number by 5\% consecutively five times actually results in a 27.628\% increase in the original number's value because $1 + 0.27628= 1.27628$. The mistake made in the quiz is thinking that repeated percentage increases or decreases are additive. An increase of 5\% five times \textit{does not} result in a $5 \cdot 5\%= 25\%$ increase, which was the percentage decrease computed in the quiz statement. \pvspace{1.3cm}



% Quiz 7
\quizsol \textit{True/False}: The real number $0.3 \cdot 10^1$ is in scientific notation. \pspace

\sol The statement is \textit{false}. A real number is in scientific notation if it is expressed as $r \cdot 10^k$, where $1 \leq r < 10$ is a real number and $k$ is an integer. In this case, we have $r= 0.3$ and $k= 1$. While $k= 1$ is an integer, $r \not\geq 1$. Properly writing this number in scientific notation, we have $3.0 \cdot 10^0$. \pvspace{1.3cm}



% Quiz 8
\quizsol \textit{True/False}: Suppose one picks up one end of a slinky, leaving the other end flat on a table, and pulls the end into the air without stretching the slinky greatly or tilting the end (so that it remains parallel to the table). Knowing the diameter of the slinky and the height one pulled it into the air is sufficient to compute the volume contained within the slinky. \pspace

\sol The statement is \textit{true}. Recall that Cavalieri's Principle states that if two figures have the same height and cross sectional areas at every point along their height, they have the same volume. So long as we do not greatly stretch the slinky, once stretched, every cross section of the slinky will have the same area as cross section of the original slinky. Using Cavalieri's Principle, the stretched slinky will then have the same volume as a cylinder with the same diameter as the original slinky and as tall as the stretched slinky. If the diameter is $d$, then $r= \frac{d}{2}$. But then the volume is $V= A_{\text{base}} h= \pi r^2h= \pi \left( \frac{d}{2} \right)^2 h= \frac{\pi d^2 h}{4}$. \pvspace{1.3cm}



% Quiz 9
\quizsol \textit{True/False}: If a relation $f(x)$ has the property that $f(1)= f(3)$, then $f(x)$ cannot be a function. \pspace

\sol The statement is \textit{false}. A function is a relation such that for a given input, there is only a single output associated to that input. The definition of being a function does not preclude two or more inputs having the same outputs. For instance, the function $f(x)= x^2$ has the property that $f(a)= f(-a)$ for all $a \in \mathbb{R}$, e.g. $4= 2^2= f(2)= f(-2)= (-2)^2= 4$. Furthermore, the function $g(x)= (x - 2)^2$ has the property that $f(1)= f(3)$ but is still a function. What matters is that given an input, one knows the (only) output associated to that input. However, there are relations that are not functions but have equal outputs at different inputs. For instance, the relation $f$ given by $\{ (0, 0), (0, 1), (1, 5), (3, 5) \}$ is not a function but not because $f(1)= f(3)$, but because $f(0)$ is not well defined. \pvspace{0.8cm}



% Quiz 10
\quizsol \textit{True/False}: If a relation $f(x)$ has the property that $f(1)= f(3)$, then $f(x)$ may or may not be a function but the inverse function, $f^{-1}(x)$, does not exist. \pspace

\sol The statement is \textit{true}. From the previous quiz, we know that the fact $f(1)= f(3)$ has nothing to do with whether the relation is a function or not. However, the fact that $f(1)= f(3)$ does prevent $f$ from having an inverse function. Recall that an inverse function $f^{-1}$ is a function such that $f^{-1}(y)= x$ if and only if $f(x)= y$. Suppose that $r$ is the value of $f(1)$ and $f(3)$. Because $f(1)= r$, it must be that $f^{-1}(r)= 1$. But because $f(3)= r$, it must also be that $f^{-1}(r)= 3$. But then $f^{-1}(r)$ is not well defined---one cannot tell whether it should be 1 or 3. Alternatively, the inverse function (should it exist for a function) is defined by the fact that $(f \circ f^{-1})(x)= x$ and $(f^{-1} \circ f)(x)= x$. But then using the fact that $f(1)= f(3)$, we have\dots
	\[
	1= f^{-1} \big( f(1) \big)= f^{-1} \big( f(3) \big)= 3,
	\]
which is clearly impossible. Clearly, $f(1)= f(3)$ implies that $f^{-1}$ not exist. However, the preimage always exists. If $f(1)= r = f(3)$, we know that $f^{-1}(r)$ includes the values $1, 3$. \pvspace{0.8cm}



% Quiz 11
\quizsol \textit{True/False}: If a relation $f(x)$ passes the vertical and horizontal line test, then not only does $f^{-1}(x)$ exist but $f(x)$ is a function of \textit{both} $x$ and $y$. \pspace

\sol The statement is \textit{true}. If $f(x)$ passes the vertical line test, we know that $f(x)$ is a function (of $x$). Because $f(x)$ passes the horizontal line test, we know that $f^{-1}(x)$ exists. Now for $f(x)$ to be a function of $y$, we need there to be a unique $x$ associated to any given $y$. But then every horizontal line $y= y_0$ will intersect the graph of $f(x)$ at a unique point, say $(x_0, y_0)$, so that the unique $x$-value associated to $y_0$ is $x_0$. But this occurs precisely when $f(x)$ passes the horizontal line test, which we already know to be the case. Therefore, $f(x)$ can also be considered a function of $y$. In fact, considering $f(x)$ to be a function of $y$ is to precisely consider the function $f^{-1}(y)$. \pvspace{0.8cm}



% Quiz 12
\quizsol \textit{True/False}: The point $(-1, -2)$ is on the graph of the curve given by $y^2= x^3 + x + 6$. \pspace

\sol The statement is \textit{true}. A point is on the graph of a function (given explicitly or implicitly, as in the statement of the quiz) if and only if it satisfies the equation giving the curve. Observe\dots
	\[
	\begin{gathered}
	y^2 = x^3 + x + 6 \\[0.1cm]
	(-2)^2 \stackrel{?}{=} (-1)^3 + (-1) + 6 \\[0.1cm]
	4 \stackrel{?}{=} -1 - 1 + 6 \\[0.1cm]
	4 = 4 \\
	\text{\cmark}
	\end{gathered}
	\]
Therefore, $(-1, -2)$ is on the curve given by $y^2= x^3 + x + 6$. 



% Quiz 13
\quizsol \textit{True/False}: Let $f(x)$ and $g(x)$ be linear functions. If the slope of $f(x)$ is not equal to the slope of $g(x)$, then $f(x)$ is not parallel to $g(x)$. Furthermore, the lines are perpendicular. \pspace

\sol The statement is \textit{false}. Lines which are not parallel must intersect (by definition). But this intersection need not happen perpendicularly. For instance, consider the linear functions $f(x)= x$ and $g(x)= 2x$. The slope of $f(x)$ is $m_f= 1$ and the slope of $g(x)$ is $m_g= 2$. Because $m_f \neq m_g$, we know the lines are not parallel. [Clearly, they intersect at $(0, 0)$.] However, we can see from plotting these lines that this intersection should not occur perpendicularly. Recall that if $f, g$ have well defined slopes and are perpendicular, then $m_f= -\frac{1}{m_g}$. But $m_f= 1 \neq -\frac{1}{2}= -\frac{1}{m_g}$. Therefore, the lines are not perpendicular. \pvspace{1.3cm}



% Quiz 14
\quizsol \textit{True/False}: Every linear function has an inverse. \pspace

\sol The statement is \textit{false}. Consider the linear function $\ell(x)= 5$. We know that $\ell(1)= 5$ and $\ell(-6)= 5$. But then the inverse cannot be well defined at $5$, i.e. $\ell^{-1}(5)$ cannot be well defined. Therefore, $\ell^{-1}$ does not exist. [Although, certainly the preimage always exists.] Observe that $\ell(x)= 5$ is a constant linear function, i.e. a linear function with slope $0$. We can see this from their graphs---every horizontal line (constant linear function) fails the horizontal line test. It is true that all \textit{nonconstant} linear functions have an inverse. \pvspace{1.5cm}



% Quiz 15
\quizsol \textit{True/False}: For any function $f(x)$, if $f^{-1}(x)$ exists, then $f(x)$ passes both the vertical and horizontal line test. Furthermore, $(f \circ f^{-1})(x)= x$. \pspace

\sol The statement is \textit{true}. We know that a relation $f(x)$ is a function if and only if it passes the vertical line test. Furthermore, we know that a function $f(x)$ has an inverse $f^{-1}(x)$ if and only if it passes the horizontal line test. Because $f(x)$ is a function, we know that the graph of $f(x)$ passes the vertical line test. Because $f^{-1}(x)$ exists, we know that it passes the horizontal line test. Finally, we know the inverse of a function $f(x)$ is a function such that $(f \circ f^{-1})(x)= x$ and $(f^{-1} \circ f)(x)= x$. \pvspace{1.5cm}



% Quiz 16
\quizsol \textit{True/False}: If $f(x)= 5(x + 6)^2 - 7$, then this is a quadratic function that opens upwards and has vertex $(6, -7)$. \pspace

\sol The statement is \textit{false}. Recall that the vertex form of a quadratic function $f(x)= ax^2 + bx + c$ is $f(x)= a(x - P)^2 + Q$, where $(P, Q)$ is the vertex of the function. We have $f(x)= 5(x + 6)^2 - 7= 5 \big(x - (-6) \big)^2 + (-7)$. Because $a= 5 > 9$, we know that the quadratic function opens upwards, i.e. is concave up or convex. However, we can see that the vertex is $(-6, -7)$, not $(6, -7)$. Note that the $x$-coordinate of the vertex makes the $a(x - P)^2$ term zero. \pvspace{1.5cm}





\newpage





% Quiz 17
\quizsol \textit{True/False}: The quadratic formula, $x= \dfrac{-b \pm \sqrt{b^2 - 4ac}}{2a}$, solves any quadratic equation $ax^2 + bx + c= 0$. \pspace

\sol The statement is \textit{true}. Suppose we want the solutions to a quadratic equation $ax^2 + bx + c= 0$. Observe by completing the square and using the fact that $a \neq 0$, we have\dots
	\[
	\begin{gathered}
	ax^2 + bx + c= 0 \\
	a \left(x^2 + \dfrac{b}{a}\, x + \dfrac{c}{a} \right)= 0 \\
	a \left(x^2 + \dfrac{b}{a}\, x + \dfrac{b^2}{4a^2} - \dfrac{b^2}{4a^2} + \dfrac{c}{a} \right)= 0 \\
	a \left( \bigg( x^2 + \dfrac{b}{a}\, x + \dfrac{b^2}{4a^2} \bigg) + \bigg( -\dfrac{b^2}{4a^2} + \dfrac{c}{a} \bigg) \right)= 0 \\
	a \left( \bigg(x + \dfrac{b}{2a} \bigg)^2 + \bigg( \dfrac{4ac - b^2}{4a^2} \bigg) \right)= 0 \\
	\left( x + \dfrac{b}{2a} \right)^2 + \left( \dfrac{4ac - b^2}{4a^2} \right)= 0 \\
	\left( x + \dfrac{b}{2a} \right)^2 = - \dfrac{4ac - b^2}{4a^2} \\
	\left( x + \dfrac{b}{2a} \right)^2 = \dfrac{b^2 - 4ac}{4a^2} \\
	x + \dfrac{b}{2a}= \pm \dfrac{\sqrt{b^2 - 4ac}}{\sqrt{4a^2}} \\
	x= -\dfrac{b}{2a} \pm \dfrac{\sqrt{b^2 - 4ac}}{2a} \\
	x= \dfrac{-b \pm \sqrt{b^2 - 4ac}}{2a}
	\end{gathered}
	\]
This solves any quadratic equation. But this is exactly the quadratic formula. \pvspace{1.3cm}
	


% Quiz 18
\quizsol \textit{True/False}: Let $f(x)= (2x - 1)^2$. Because $\disc f= 0^2 - 4(2)(-1)= 8 > 0$, the equation $f(x)= 0$ has two distinct solutions. Furthermore, because $\disc f= 8$ is not a perfect square, the solutions are not `nice.' \pspace

\sol The statement is \textit{false}. If $f(x)= 0$, then $(2x - 1)^2= 0$. This implies that $2x - 1= 0$, so that $x= \frac{1}{2}$. Clearly, there is a unique, `nice' solution. Observe that $f(x)= (2x - 1)^2= (2x - 1)(2x - 1)= 4x^2 - 4x + 1$. This means that $a= 4$, $b= -4$, and $c= 1$. But then $\disc f= b^2 - 4ac= (-4)^2 - 4(4)1= 16 - 16= 0$. Because $\disc f= 0$, we know the quadratic equation $f(x)= 0$ has a distinct, real, `nice' solution, which we saw was $x= \frac{1}{2}$. It is true that if $\disc f > 0$, there are two distinct, real solutions and if $\disc f < 0$ there are two distinct, complex solutions. Furthermore, the solutions to $f(x)= 0$ are `nice' if and only if $|\disc f|$ is a perfect square. 





























\end{document}