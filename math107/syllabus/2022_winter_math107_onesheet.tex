\documentclass[11pt,letterpaper]{article}
\usepackage[lmargin=1in,rmargin=1in,bmargin=1in,tmargin=1in]{geometry}
\usepackage{style}

\pagenumbering{gobble}


% -------------------
% Content
% -------------------
\begin{document}

% TItle
\begin{center} 
\bfseries
\color{stacred}
\LARGE Syllabus Quick Facts \par\vspace{0.2\baselineskip}
\Large MATH 107: Data and Quantitative Literacy --- Winter 2022
\end{center} \pspace


% Course Information
\mysection{0.27}{Course Information}
\hspace{0.53cm} {\itshape Instructor Email}: \href{mailto:cmcwhort@stac.edu}{cmcwhort@stac.edu} \par
\hspace{0.53cm} {\itshape Course Webpage}: \href{https://coffeeintotheorems.com/courses/2022-2/winter/math-107/}{https://coffeeintotheorems.com/courses/2022-2/winter/math-107/} \par
\hspace{0.53cm} {\itshape Office Hours}: By appointment \pspace


% Grading Components
\mysection{0.27}{Grading Components\label{grade_comp}}
Course grades are determined by the following components: \par \vspace{-0.3cm}
	\begin{table}[!ht]
        \begin{tabular}{clr}
	& Quizzes & 5\% \\
	& Biographies & 5\% \\
	& Homework & 45\% \\
	& Exams & 45\%
        \end{tabular} 
        \end{table}


% Quizzes 
\mysection{0.27}{Quizzes}
There will be a quiz corresponding to \textit{every} class. Quizzes are meant to be short and simple. These quizzes serve more as a method of gauging whether you are keeping up with the material. The quizzes can be found on the course webpage as well as in the course GoogleDrive. Quizzes are submitted to the instructor via email. Students should include the quiz number in the email subject line. No work will be required---simply indicate your final quiz solution. Each quiz is due in the instructors email by 10~am on the assigned due date. No make-up quizzes will be given except under extraordinary circumstances determined on a case-by-case basis at the discretion of the instructor. Unless otherwise instructed, notes, or outside assistance of any kind allowed on quizzes. \pspace


% Biographies
\mysection{0.18}{Biographies\label{bios}}
You will write short biographies of various mathematicians. Therefore, there will be an assigned biography corresponding to \textit{every} class---often highlighting a mathematician discussed during lecture or a mathematician in a field related to the lecture topic. These biographies should be no shorter than half a page in length but should not be `much' longer than a full page. Each page should have one-inch margins, size 11~font (`reasonably' chosen), and be single-spaced. Each biography should be similar to a biography found in a playbook, a textbook biography, or the alike, i.e. they highlight the life, work, and accomplishments of the assigned mathematician. Each biography should include the birth and death dates (unless they are living), the field of mathematics that the mathematician specializes in, and at least 3--5 `interesting' facts about the mathematician, and a `reasonably' sized representative photo of the individual. Above all, these biographies should read `smoothly.' These are submitted via your personal course GoogleDrive folder under the `Biographies' section. Each biography is due by 10~am on their assigned due date. All biographies should appear in a single GoogleDoc with each mathematician appearing on their own page, i.e. one mathematician per page, in the order that they are assigned. The assigned biographies along with their due dates can be found on the course webpage as well as in the course GoogleDrive. No make-up biographies nor late submissions will be allowed except under extraordinary circumstances determined on a case-by-case basis at the discretion of the instructor. \pspace


% Homework 
\mysection{0.27}{Homework}
There will be a homework assigned \textit{every} class day. Homeworks may be accessed via the course webpage as well as in the course GoogleDrive. Your homework solutions should include each problem with your solution clearly indicated and all work shown and thoroughly explained. Each problem should appear on its own page, i.e. one problem per page, with your name at the top of each page. You then scan these pages, combine them into a \textit{single} PDF with the problems appearing in the correct order. This PDF is then uploaded to a GoogleDrive shared with you in the `Homeworks' folder. The file should be renamed `Homework~\#', where `\#' is the homework number. Homeworks should be submitted by 10~am on the date that they are due. Anticipate that they may be problems and do not wait until the last minute to do and upload your assignments! Once you have submitted the PDF, be sure to check that the PDF has uploaded correctly, that the file is downloadable, and that you have followed all formatting requirements. You should not delete, re-upload, alter, etc. any homework that you have submitted after the due date or time has passed without instructor permission. Failure to follow all of these guidelines may result in point-deductions or your homework being rejected. Penalties will be determined by the instructor on a student-by-student basis. If there are issues in this process, be sure to email the homework to your instructor right away so to as minimize any possible penalties that may be incurred. \pspace


% Exams 
\mysection{0.27}{Exams}
There will be three exams in this course, each worth 15\% of the total course grade for a total of 45\% of the course grade. The schedule of the exams can be found in the `Course Schedule' section of the syllabus. However, these exam dates are subject to change. Students should not make plans to be unavailable before January 20th on/before that date. Each of the exams covers approximately the third of the course material proceeding the exam date. However, any course topics may appear on any exam. Before each exam, typically the day before, the procedure for that exam will be announced. Typically, exam submission procedures will be the same as for course homeworks, c.f. the `Homework' section of this syllabus. Students who miss an exam should not expect to receive a make-up exam. There will be no make-up exams except under extraordinary circumstances, e.g. in the case of an emergency. However, determinations for make-up exams or other substitutions, with possible grade deductions, are made at the discretion of the instructor on a case-by-case basis. Unless otherwise instructed, no devices or materials other than those provided by the instructor are allowed on any exam. Exams will be found on the course webpage as well as in the course GoogleDrive. \pspace


% Course Schedule 
\mysection{0.27}{Course Schedule}
The following is a \emph{tentative} schedule for the course and is subject to change. 
        \begin{table}[!ht]
        \centering
        \scalebox{1}{%
        \begin{tabular}{ll || ll}
        Date & Topic(s) & Date & Topic(s) \\ \hline 
	01/03 & Logic \& Circuits & 01/12 & Graph Theory \\
	01/04 & Number Theory \& Game Theory & 01/13 & Review \& Exam~2 \\
	01/05 & Financial Mathematics & 01/16 & MLK Day (No Class) \\
	01/06 & Review \& Exam~1 & 01/17 & Probability \\
	01/09 & Linear Algebra & 01/18 & Statistics \\
	01/10 & Linear Algebra & 01/19 & Statistics \\
	01/11 & Linear Algebra \& Graph Theory & 01/20 & Review \& Exam~3
        \end{tabular}
        }
        \end{table}


\end{document}