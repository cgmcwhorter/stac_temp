\documentclass[11pt,letterpaper]{article}
\usepackage[lmargin=1in,rmargin=1in,tmargin=1in,bmargin=1in]{geometry}
\usepackage{../style/homework}
\usepackage{../style/commands}
\setbool{quotetype}{true} % True: Side; False: Under
\setbool{hideans}{false} % Student: True; Instructor: False

% -------------------
% Content
% -------------------
\begin{document}

\homework{20: Due 01/20}{Facts are stubborn things, but statistics are pliable.}{Mark Twain} 

% Problem 1
\problem{10} What assumptions are required for the Central Limit Theorem to apply? \pspace

\sol For the Central Limit Theorem to apply, the distribution needs to have a finite standard deviation and the sample must be a simple random sample. Furthermore, at least one of the following needs to be true:
	\begin{itemize}
	\item the underlying distribution needs to be normal.
	\item the sample size needs to be `large.' 
	\end{itemize}



\newpage



% Problem 2
\problem{10} Suppose a sample, $X$, is drawn from a normal distribution with mean 220 and standard deviation 18. 
	\begin{enumerate}[(a)]
	\item Find $P(X \leq 200)$. 
	\item Find the probability that a sample of size 8 will have an average less than 200. 
	\end{enumerate} \pspace

\sol The underlying distribution is normal with $\mu= 220$ and standard deviation $\sigma= 18$, i.e. the distribution is $N(220, 18)$. 

\begin{enumerate}[(a)]
\item We know $P(X \leq 200)$ is that the probability of the \textit{single} sample, $X$, is at most 200. We can use the $z$-score for 200 in the normal distribution $N(220, 18)$ to compute this. We have\dots
	\[
	z_{200}= \dfrac{x - \mu}{\sigma}= \dfrac{200 - 220}{18}= \dfrac{-20}{18} \approx -1.11 \squiggle 0.1335 
	\]
Therefore, $P(X \leq 200) \approx 0.1335$; that is, there is a 13.35\% chance that $X \leq 200$. \pspace

\item Because we are asked about a probability of a \textit{sample} and not an `individual', we need the sampling distribution. This requires the Central Limit Theorem---so we need check that it applies. We know that the distribution has a finite standard deviation of 18. We assume that the sample is a simple random sample. The sample size $n= 8$ is not `large'; however, the underlying distribution is normal. Therefore, the Central Limit Theorem applies. Therefore, the distribution of averages of samples with size 8 is normal. In fact, the distribution is\dots
	\[
	N\left(\mu, \dfrac{\sigma}{\sqrt{n}} \right)= N\left(220, \dfrac{18}{\sqrt{8}} \right) \approx N\left(220, \dfrac{18}{2.82843} \right) \approx N\left(220, 6.36 \right)
	\]
We want to compute the probability that a sample of size 8 will have an average, $\overline{X}$, less than 200, i.e. $P(\overline{X} < 200)$. We can compute the $z$-score for $200$ in the normal distribution $N(220, 6.36)$ to compute this:
	\[
	z_{200}= \dfrac{x - \mu}{\sigma}= \dfrac{200 - 220}{6.36}= \dfrac{-20}{6.36} \approx -3.14 \squiggle 0.0008
	\]
Therefore, $P(\overline{X} < 200) \approx 0.0008$; that is, there is only a 0.08\% chance that a random sample of size 8 will have a sample average of less than 200. 
\end{enumerate}



\newpage



% Problem 3
\problem{10} SAT scores in 2017 had mean 1060 and standard deviation 195.\footnote{\url{https://nces.ed.gov/programs/digest/d17/tables/dt17_226.40.asp}} If you randomly sampled 40 students, find\dots
	\begin{enumerate}[(a)]
	\item The probability that their average score was less than 1000. 
	\item The probability that their average score was greater than 1100. 
	\item The probability that their average score was between 1000 and 1100. 
	\end{enumerate} \pspace

\sol The underlying distribution has mean $\mu= 1060$ and standard deviation $\sigma= 195$. In each part, we are asked about the probability of a \textit{sample}, not an `individual.' Therefore, we need to know the sampling distribution. This requires the Central Limit Theorem---so we need check that it applies. We know that the distribution has a finite standard deviation of 195. We assume that the sample is a simple random sample. The underlying distribution is not known to be normal. However, the sample size $n= 40$ is `sufficiently large', i.e. $n= 40 \geq 30$. Therefore, the Central Limit Theorem applies. Therefore, the distribution of averages of samples with size 40 is normal. In fact, the distribution is\dots
	\[
	N\left(\mu, \dfrac{\sigma}{\sqrt{n}} \right)= N\left(1060, \dfrac{195}{\sqrt{40}} \right) \approx N\left(1060, \dfrac{195}{6.32456} \right) \approx N\left(1060, 30.83 \right)
	\]

\begin{enumerate}[(a)]
\item We want to compute the probability that their average score was less than 1000, i.e. $P(\overline{X} < 1000)$. To find this, we can compute the $z$-score for 1000 in the normal distribution $N(1060, 30.83)$. We have\dots
	\[
	z_{1000}= \dfrac{x - \mu}{\sigma}= \dfrac{1000 - 1060}{30.83}= \dfrac{-60}{30.83} \approx -1.95 \squiggle 0.0256
	\]
Therefore, $P(\overline{X} < 1000) \approx 0.0256$, i.e. there is a 2.56\% chance that a random sample of 40 students will have an average SAT score of less than 1000. \pspace

\item We want to compute the probability that their average score was greater than 1100, i.e. $P(\overline{X} > 1100)$. To find this, we can compute the $z$-score for 1100 in the normal distribution $N(1060, 30.83)$. We have\dots
	\[
	z_{1100}= \dfrac{x - \mu}{\sigma}= \dfrac{1100 - 1060}{30.83}= \dfrac{40}{30.83} \approx 1.30 \squiggle 0.9032
	\]
Therefore, $P(\overline{X} < 1100) \approx 0.9032$. We complement to find $P(\overline{X} > 1100)$:
	\[
	P(\overline{X} > 1100)= 1 - P(\overline{X} \leq 1100)= 1 - P(\overline{X} < 1100)= 1 - 0.9032= 0.0968
	\]
Therefore, $P(\overline{X} > 1100)= 0.0968$, i.e. there is a 9.68\% chance that a random sample of 40 students will have an average SAT score of greater than 1100. \pspace 

\item We want to compute the probability that their average score was between 1000 and 1100, i.e. $P(1000 < \overline{X} < 1100)$. But we know\dots
	\[
	P(1000 < \overline{X} < 1100)= P(\overline{X} < 1100) - P(1000 < \overline{X})= 0.9032 - 0.0256= 0.8776
	\]
Therefore, $P(1000 < \overline{X} < 1100)= 0.8776$, i.e. there is a 87.76\% chance that a random sample of 40 students will have an average SAT score between 1000 and 1100. 
\end{enumerate}



\newpage



% Problem 4
\problem{10} Suppose that 17\% of new cars will receive some minor repair after 2~years. If you take a simple random sample of 500~cars, find the probability that less than 60 of the cars will need a repair in their first two years. \pspace

\sol We have a fixed number of observations, i.e. $n= 500$---namely, the 500 cars. Each new car will either receive some minor repair their first 2~years or not. The probability that they receive a minor repair is a fixed 17\%, i.e. $p= 0.17$; hence, the probability they do not receive a minor repair in their first 2~years is a fixed 83\%. We assume that whether or not a given new car requires a minor repair in their first 2~years is independent from the other new cars. [This is not likely to necessarily be the case!] Therefore, this is a binomial distribution with $n= 500$ and $p= 0.17$, i.e. $B(500, 0.17)$. \pspace

We are asked to find the probability that less than 60 of the new cars will need a repair in their first two years, i.e. $P(X < 60)$. But then we would need to find\dots
	\[
	P(X < 60)= P(X= 0) + P(X= 1) + P(X= 2) + \cdots + P(X= 58) + P(X= 59) 
	\]
It would be better to see if we can approximate this binomial distribution with a normal distribution for an easier computation. We need to check whether the Central Limit Theorem applies to `allow' for this approximation; that is, we need to check $np \geq 10$ and $n(1 - p) \geq 10$:
	\[
	\begin{aligned}
	np&= 500(0.17)= 85 \geq 10 \\
	n(1 - p)&= 500(1 - 0.17)= 500(0.83)= 415 \geq 10
	\end{aligned}
	\]
Therefore, the Central Limit Theorem applies and we can use the approximation. We know that if the normal approximation applies, then $B(n, p) \approx N \left(np, \sqrt{np (1 - p)} \right)$. We have\dots
	\[
	\begin{aligned}
	B(500, 0.17) &\approx N \left(np, \sqrt{np (1 - p)} \right) \\
	&= N \left(500(0.17), \sqrt{500 \cdot 0.17 \cdot (1 - 0.87)} \right) \\
	&\approx N \left(85, \sqrt{500 \cdot 0.17 \cdot 0.83} \right) \approx N \left(85, \sqrt{70.55} \right) \\
	&\approx N \left(85, 8.40 \right)
	\end{aligned}
	\]
To compute $P(X < 60)$, we compute the $z$-score for 60 in the normal distribution $N(85, 8.40)$. We have\dots
	\[
	z_{60}= \dfrac{x - \mu}{\sigma}= \dfrac{60 - 85}{8.40}= \dfrac{-25}{8.40} \approx -2.98 \squiggle 0.0014
	\]
Therefore, $P(X < 60) \approx 0.0014$, i.e. there is a 0.14\% chance that less than 60 of the cars will need a minor repair in their first two years. 



\newpage



% Problem 5
\problem{10} Watch The New York Times' video, \href{https://www.youtube.com/watch?v=jvoxEYmQHNM&t=1s&pp=ygUfbnl0IGJ1bm5pZXMgYW5kIHdyYW9uZ3MgY2VudHJhbA\%3D\%3D}{``Bunnies, Dragons and the `Normal' World: Central Limit Theorem''} on YouTube. Being as detailed as possible, comment on what you learned and how it relates to the course material. \pspace

\sol \vfill

\begin{center} {\itshape Solutions will vary.} \end{center} \vfill


\end{document}