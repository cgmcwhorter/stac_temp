\documentclass[11pt,letterpaper]{article}
\usepackage[lmargin=1in,rmargin=1in,tmargin=1in,bmargin=1in]{geometry}
\usepackage{../style/homework}
\usepackage{../style/commands}
\setbool{quotetype}{true} % True: Side; False: Under
\setbool{hideans}{true} % Student: True; Instructor: False

% -------------------
% Content
% -------------------
\begin{document}

\homework{9: Due 01/11}{

% Problem 1
\problem{10} Let $M= \begin{pmatrix} 1 & 2 & 0 & -6 & 1 \\ 4 & 5 & -1 & 0 & 9 \end{pmatrix}$.
	\begin{enumerate}[(a)]
	\item What is the dimension of $M$?
	\item What is $m_{25}$?
	\item What is $m_{14}$? 	
	\end{enumerate}



\newpage



% Problem 2
\problem{10} Define the following:
	\[
	A= \begin{pmatrix} 1 & 2 \\ -1 & 5 \end{pmatrix} \qquad
	B= \begin{pmatrix} -1 & 0 \\ 6 & 2 \end{pmatrix} \qquad
	\vec{u}= \begin{pmatrix} 1 \\ -1 \end{pmatrix}
	\]
Showing all your work, compute the following: 
	\begin{enumerate}[(a)]
	\item $-2B$
	\item $A+ B$
	\item $A - B$
	\item $AB$
	\item $A \vec{u}$
	\end{enumerate}



\newpage



% Problem 3
\problem{10} The matrices below can only be multiplied in a particular order. Find which order they can be multiplied and then compute their product.
	\[
	A= \begin{pmatrix} 1 & -1 \\ 0 & 2 \\ 5 & 6 \end{pmatrix} \qquad
	B= \begin{pmatrix} 1 & -5 & 2 \end{pmatrix}
	\]



\newpage



% Problem 4
\problem{10} Define the following:
	\[
	\begin{aligned}
	A&= \begin{pmatrix}
	1 & 0 & -1 & 4 & 7 \\
	6 & 2 & 0 & 5 & 1 \\
	9 & -1 & 1 & 7 & 4 \\
	2 & 5 & -1 & 6 & 1 
	\end{pmatrix} \\
	B&= \begin{pmatrix}
	1 & -1 & 0 \\
	6 & 2 & 5 \\
	7 & -1 & 2 \\
	1 & 0 & 1 \\
	2 & 4 & 6 
	\end{pmatrix}
	\]
Without compute all of $AB$, what is the entry $ab_{32}$?



\newpage



% Problem 5
\problem{10} Watch 3Blue1Brown's \href{https://www.youtube.com/watch?v=kYB8IZa5AuE&pp=ygUiTGluZWFyIFRyYW5zZm9ybWF0aW9uIGFuZCBtYXRyaWNlcw\%3D\%3D}{``Linear Transformation and matrices''} and \href{https://www.youtube.com/watch?v=XkY2DOUCWMU&pp=ygUmTWF0cml4IE11bHRpcGxpY2F0aW9ucyBhcyBjb21wb3NpdGlvbnM\%3D}{``Matrix Multiplications as compositions''} on YouTube. Being as detailed as possible, comment on what you learned and how it relates to the course material. 


\end{document}