\documentclass[11pt,letterpaper]{article}
\usepackage[lmargin=1in,rmargin=1in,tmargin=1in,bmargin=1in]{geometry}
\usepackage{../style/homework}
\usepackage{../style/commands}
\setbool{quotetype}{true} % True: Side; False: Under
\setbool{hideans}{false} % Student: True; Instructor: False

% -------------------
% Content
% -------------------
\begin{document}

\homework{1: Due 01/04}{I know three things will never be believed---the true, the probable, and the logical.}{John Steinbeck}

% Problem 1
\problem{10} Define the following propositions:
	\begin{table}[H]
	\centering
	\begin{tabular}{rl}
	$P$: & Bill took a Math course. \\
	$Q$: & Susan is not a Biology major. \\
	$R$: & Bill is a senior. \\
	$S$: & Susan is a Sophomore. 
	\end{tabular}
	\end{table}
Write the following logical propositions as a complete English sentence:
	\begin{enumerate}[(a)]
	\item $P \wedge Q$
	\item $P \vee R$
	\item $S \wedge \neg Q$
	\item $R \to P$
	\end{enumerate} \pspace

\sol 
\begin{enumerate}[(a)]
\item The proposition $P \wedge Q$ is the statement, ``Bill took a Math course and Susan is not a Biology major.'' \pspace

\item The proposition $P \vee R$ is the statement, ``Bill took a Math course or Bill is a senior.'' \pspace

\item We know that $\neg Q$ is the statement, ``Susan is a Biology major.'' Therefore, the proposition $S \wedge \neg Q$ is the statement, ``Susan is a Sophomore and Susan is a Biology major.'' \pspace

\item We know that $R \to P$ is the statement, ``If Bill is a Senior, then Bill took a Math course.'' 
\end{enumerate}



\newpage



% Problem 2
\problem{10} Define the following logical statements:
	\begin{table}[H]
	\centering
	\begin{tabular}{rl}
	$P$: & The plant receives sunlight. \\
	$Q$: & The plant lives. 
	\end{tabular}
	\end{table}
Write the following as complete English sentences: 
	\begin{enumerate}[(a)]
	\item $P \to Q$
	\item The inverse of $P \to Q$
	\item The converse of $P \to Q$
	\item The contrapositive of $P \to Q$
	\end{enumerate} \pspace

\sol 
\begin{enumerate}[(a)]
\item The proposition $P \to Q$ is the statement, ``If the plant receives sunlight, then the plant lives.'' \pspace

\item Given an implication $A \to B$, the inverse of this implication is $\neg A \to \neg B$. Then the inverse of $P \to Q$ is $\neg P \to \neg Q$. We know that $\neg P$ is the statement, ``The plant does not receive sunlight,'' and $\neg Q$ is the statement, ``The plant does not live,'' i.e. the statement, ``The plant dies.'' Therefore, the inverse of $P \to Q$ is the statement, ``If the plant does not receive sunlight, then the plant dies.'' \pspace

\item Given an implication $A \to B$, the converse of this implication is $B \to A$. Then the converse of $P \to Q$ is $Q \to P$. But then the converse of $P \to Q$ is the statement, ``If the plant lives, then the plant receives sunlight.'' \pspace

\item Given an implication $A \to B$, the contrapositive of this implication is $\neg B \to \neg A$. Then the contrapositive of $P \to Q$ is $\neg Q \to \neg P$. We know that $\neg Q$ is the  statement, ``The plant does not live,'' i.e. the statement, ``The plant dies,'' and $\neg P$ is the statement, ``The plant does not receive sunlight.'' Therefore, the contrapositive of $P \to Q$ is the statement, ``If the plant dies, then the plant is not receiving sunlight.'' 
\end{enumerate}



\newpage



% Problem 3
\problem{10} Construct the truth table for the following:
	\begin{enumerate}[(a)]
	\item $\neg (P \wedge Q) \to P$
	\item $(P \vee \neg R) \wedge (Q \vee P)$
	\end{enumerate} \pspace

\sol 
\begin{enumerate}[(a)]
\item \phantom{.} \par
	\begin{table}[h]
	\centering
	\begin{tabular}{c|c||c|c||c}
	$P$ & $Q$ & $P \wedge Q$ & $\neg(P \wedge Q)$ & $\neg(P \wedge Q) \to P$ \\ \hline
	$T$ & $T$ & $T$ & $F$ & $T$ \\
	$T$ & $F$ & $F$ & $T$ & $T$ \\
	$F$ & $T$ & $F$ & $T$ & $F$ \\
	$F$ & $F$ & $F$ & $T$ & $F$
	\end{tabular}
	\end{table}

\item \phantom{.} \par
	\begin{table}[h]
	\centering
	\begin{tabular}{c|c|c||c|c|c||c}
	$P$ & $Q$ & $R$ & $\neg R$ & $P \vee \neg R$ & $Q \vee P$ & $(P \vee \neg R) \wedge (Q \vee P)$ \\ \hline
	$T$ & $T$ & $T$ & $F$ & $T$ & $T$ & $T$ \\ 
	$T$ & $T$ & $F$ & $T$ & $T$ & $T$ & $T$ \\
	$T$ & $F$ & $T$ & $F$ & $T$ & $T$ & $T$ \\
	$T$ & $F$ & $F$ & $T$ & $T$ & $T$ & $T$ \\
	$F$ & $T$ & $T$ & $F$ & $F$ & $T$ & $F$ \\ 
	$F$ & $T$ & $F$ & $T$ & $T$ & $T$ & $T$ \\ 
	$F$ & $F$ & $T$ & $F$ & $F$ & $T$ & $F$ \\
	$F$ & $F$ & $F$ & $T$ & $T$ & $F$ & $F$
	\end{tabular}
	\end{table}
\end{enumerate}



\newpage



% Problem 4
\problem{10} Show $\neg (P \vee \neg Q)$ is logically equivalent to $Q \wedge \neg P$. \pspace

\sol To show two logical expressions are logically equivalent, we show that they have the same logical outputs for the same inputs, i.e. equivalent columns. We have\dots \par
	\begin{table}[h]
	\centering
	\begin{tabular}{c|c||c|c||c||c|c}
	$P$ & $Q$ & $\neg Q$ & $P \vee \neg Q$ & $\neg (P \vee \neg Q)$ & $\neg P$ & $Q \wedge \neg P$ \\ \hline 
	$T$ & $T$ & $F$ & $T$ & $F$ & $F$ & $F$ \\
	$T$ & $F$ & $T$ & $T$ & $F$ & $F$ & $F$ \\
	$F$ & $T$ & $F$ & $F$ & $T$ & $T$ & $T$ \\
	$F$ & $F$ & $T$ & $T$ & $F$ & $T$ & $F$
	\end{tabular}
	\end{table} \par
Because the fifth and seventh columns corresponding to $\neg (P \vee \neg Q)$ and $Q \wedge \neg P$, respectively, we know that $\neg (P \vee \neg Q)$ is logically equivalent to $Q \wedge \neg P$, i.e. $\neg (P \vee \neg Q) \equiv Q \wedge \neg P$. \pspace

Alternatively, we can use the properties of logic to show that $\neg (P \vee \neg Q)$ is logically equivalent to $Q \wedge \neg P$:
	\[
	\neg (P \vee \neg Q) \equiv \neg P \wedge \neg (\neg Q) \equiv \neg P \wedge Q \equiv Q \wedge \neg P
	\]



\newpage



% Problem 5
\problem{10} Defining appropriate propositions, write the following using the defined propositions and logical connectives: {\itshape ``Jennifer has her license or if she does not have her license, then she is under 18.''} \pspace

\sol Define $P$ to be the statement, ``Jennifer has her license.'' Define $Q$ to be the statement, ``Jennifer is under 18.'' Then the statement, ``Jennifer has her license or if she does not have her license, then she is under 18,'' can be represented by $P \vee (\neg P \to Q)$. 


\end{document}