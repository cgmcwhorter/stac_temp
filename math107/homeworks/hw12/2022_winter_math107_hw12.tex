\documentclass[11pt,letterpaper]{article}
\usepackage[lmargin=1in,rmargin=1in,tmargin=1in,bmargin=1in]{geometry}
\usepackage{../style/homework}
\usepackage{../style/commands}
\setbool{quotetype}{true} % True: Side; False: Under
\setbool{hideans}{true} % Student: True; Instructor: False

% -------------------
% Content
% -------------------
\begin{document}

\homework{12: Due 01/12}{The origins of graph theory are humble, even frivolous.}{Norman L. Biggs}

% Problem 1
\problem{10} Consider the graph below:
	\[
	\begin{tikzpicture}
	
	\end{tikzpicture}
	\]

\begin{enumerate}[(a)]
\item What edges are adjacent to 5?
\item What vertices are adjacent to $d$?
\item Is the graph simple?
\item Are there parallel edges? 
\end{enumerate}



\newpage



% Problem 2
\problem{10} Consider the graph below:
	\[
	\begin{tikzpicture}
	
	\end{tikzpicture}
	\]

\begin{enumerate}[(a)]
\item What is incident to 1?
\item Is this a multigraph? 
\item Are there isolated vertices? 
\end{enumerate}



\newpage



% Problem 3
\problem{10} Explain some applications of Graph Theory to the `real-world.' 



\newpage



% Problem 4
\problem{10} Are the graphs below isomorphic? 
	\[
	\begin{tikzpicture}
	
	\end{tikzpicture}
	\]



\newpage



% Problem 5
\problem{10} Watch PoincareDuality's \href{https://www.youtube.com/watch?v=fEWj93XjON0&pp=ygUSYSB0cmlidXRlIHRvIGV1bGVy}{``A Tribute to Euler''} on YouTube. Being as detailed as possible, comment on what you learned and how it relates to the course material. 


\end{document}