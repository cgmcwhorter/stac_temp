\documentclass[11pt,letterpaper]{article}
\usepackage[lmargin=1in,rmargin=1in,tmargin=1in,bmargin=1in]{geometry}
\usepackage{../style/homework}
\usepackage{../style/commands}
\setbool{quotetype}{false} % True: Side; False: Under
\setbool{hideans}{true} % Student: True; Instructor: False


\usepackage{hieroglf} 		% Egyptian, e.g. \textpmhg{\Hone 234567}
\usepackage{romannum}		% Roman Numerals, e.g. \romannum{10}, \Romannum{10}
\usepackage{babyloniannum}	% Babylonian, requires font install, e.g. \babyloniannum{47}
\usepackage{mathabx} 		% Mayan, e.g. \mayadigit{17}, \maya{2013}
	\newcommand\mathbfont{\usefont{U}{mathb}{m}{n}}	% formaya package

% MUST COMPILE IN XeLaTeX

% -------------------
% Content
% -------------------
\begin{document}
\pagenumbering{arabic}	% Override romannum package

\homework{3: Due 01/05}{Civilization advances by extending the number of important operations which we can perform without thinking of them.}{Alfred North Whitehead}

% Problem 1
\problem{10} Convert the following numbers to our traditional base-10 number system:
	\begin{enumerate}[(a)]
	\item $\Large \textpmhg{655} \overset{\textpmhg{3}}{\scriptsize\textpmhg{3}} \overset{\textpmhg{3}}{\scriptsize\textpmhg{3}} \,\phantom{}^{\textpmhg{\Hone \Hone \Hone}}$
	\item $\Romannum{2342}$
	\item $\huge \babyloniannum{35}$
	\item $\maya{40}$ { \hspace{-1.069cm} \color{white} $\bigg|$ } { \hspace{-0.615cm} \color{white} $\bigg|$ } { \hspace{-0.055cm} \color{white} $\bigg|$ }  { \hspace{-0.61cm} \color{white} $\bigg|$ }
	\end{enumerate}



\newpage



% Problem 2
\problem{10}  Convert the following:
	\begin{enumerate}[(a)]
	\item 1756 to Egyptian 
	\item 444 to Roman 
	\item 24 to Babylonian 
	\item 16 to Mayan
	\end{enumerate}



\newpage



% Problem 3
\problem{10} Showing all your work, convert the following to base-10:
	\begin{enumerate}[(a)]
	\item $201_3$
	\item $6005_7$
	\item $\texttt{abc}4_{16}$
	\item $101011_2$
	\end{enumerate}



\newpage



% Problem 4
\problem{10} Showing all your work, express the following numbers into the indicated base, $b$:
	\begin{enumerate}[(a)]
	\item $76$, $b= 3$
	\item $123$, $b= 9$
	\item $12$, $b= 2$
	\item $954$, $b= 16$
	\end{enumerate}



\newpage



% Problem 5
\problem{10} If you express the number 145,601 in base 16, how many digits would it have?


\end{document}