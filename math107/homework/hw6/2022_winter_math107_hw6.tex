\documentclass[11pt,letterpaper]{article}
\usepackage[lmargin=1in,rmargin=1in,tmargin=1in,bmargin=1in]{geometry}
\usepackage{../style/homework}
\usepackage{../style/commands}
\setbool{quotetype}{true} % True: Side; False: Under
\setbool{hideans}{true} % Student: True; Instructor: False

% -------------------
% Content
% -------------------
\begin{document}

\homework{6: Due 01/06}{

% Problem 1
\problem{10} Paul is looking to finance some minor repairs to his garage. He takes out a simple discount note from a local bank for \$755 at 6.2\% annual interest. The loan will be for 14~months. 
	\begin{enumerate}[(a)]
	\item What is the maturity for this loan?
	\item What is the discount? 
	\item What are the proceeds? 
	\item How much does Paul owe in 14~months? 
	\item How much does Paul pay in total for this loan? 
	\end{enumerate}



\newpage



% Problem 2
\problem{10} You are looking to take out a short-term loan. A bank offers you two simple discount note options: \$36,160 at 5.9\% interest for 8~months or \$23,340 at 5.6\% interest for 13~months. Which option minimizes the total amount of interest that you will pay?



\newpage



% Problem 3
\problem{10} Laurie takes out a car loan for \$12,500 at 7.5\% annual interest, compounded monthly. How much will she owe on the loan after 4~years? How much of that amount is interest? 



\newpage



% Problem 4
\problem{10} Hasheed is trying to save for a down payment on a snowmobile that costs \$6,800. He opens a savings account with a 2.1\% annual interest rate, compounded quarterly. How much should he deposit into the account now so that he has enough for it in one year? 



\newpage



% Problem 5
\problem{10} Elena is saving money to take some extra college courses. She puts \$3,900 into an account that earns 6.4\% annual interest, compounded semiannually. How long until she has saved \$9,000?


\end{document}