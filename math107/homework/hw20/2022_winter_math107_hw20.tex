\documentclass[11pt,letterpaper]{article}
\usepackage[lmargin=1in,rmargin=1in,tmargin=1in,bmargin=1in]{geometry}
\usepackage{../style/homework}
\usepackage{../style/commands}
\setbool{quotetype}{true} % True: Side; False: Under
\setbool{hideans}{true} % Student: True; Instructor: False

% -------------------
% Content
% -------------------
\begin{document}

\homework{20: Due 01/20}{Facts are stubborn things, but statistics are pliable.}{Mark Twain} 

% Problem 1
\problem{10} What assumptions are required for the Central Limit Theorem to apply? 



\newpage



% Problem 2
\problem{10} Suppose a sample, $X$, is drawn from a normal distribution with mean 220 and standard deviation 18. 
	\begin{enumerate}[(a)]
	\item Find $P(X \leq 200)$. 
	\item Find the probability that a sample of size 8 will have an average less than 200. 
	\end{enumerate}



\newpage



% Problem 3
\problem{10} SAT scores in 2017 had mean 1060 and standard deviation 195.\footnote{\url{https://nces.ed.gov/programs/digest/d17/tables/dt17_226.40.asp}} If you randomly sampled 40 students, find\dots
	\begin{enumerate}[(a)]
	\item The probability that their average score was less than 1000. 
	\item The probability that their average score was greater than 1100. 
	\item The probability that their average score was between 1000 and 1100. 
	\end{enumerate}



\newpage



% Problem 4
\problem{10} Suppose that 17\% of new cars will receive some minor repair after 2~years. If you take a simple random sample of 500~cars, find the probability that less than 60 of the cars will need a repair in their first two years. 



\newpage



% Problem 5
\problem{10} Watch The New York Times' video, \href{https://www.youtube.com/watch?v=jvoxEYmQHNM&t=1s&pp=ygUfbnl0IGJ1bm5pZXMgYW5kIHdyYW9uZ3MgY2VudHJhbA\%3D\%3D}{``Bunnies, Dragons and the `Normal' World: Central Limit Theorem''} on YouTube. Being as detailed as possible, comment on what you learned and how it relates to the course material. 


\end{document}