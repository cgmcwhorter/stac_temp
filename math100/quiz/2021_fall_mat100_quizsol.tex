\documentclass[11pt,letterpaper]{article}
\usepackage[lmargin=1in,rmargin=1in,bmargin=1in,tmargin=1in]{geometry}
\usepackage{style/quiz}
\usepackage{style/commands}

% -------------------
% Content
% -------------------
\begin{document}
\thispagestyle{title}

% Quiz 1
\quizsol \textit{True/False}: The number 1 is prime. \pspace

\sol The statement is \textit{false}. A prime number is an integer greater than 1 that can only be factored as the product of one and itself. So for example, the integer 11 is prime because we can only factor 11 as $11= 1 \cdot 11$. However, the integer 12 is not prime because we can write $12= 2 \cdot 6$, neither of which are 1 or 12. \pvspace{1.5cm}



% Quiz 2
\quizsol \textit{True/False}: $\gcd(2^3 \cdot 3 \cdot 5, 2 \cdot 3^2 \cdot 7)= 2^3 \cdot 3^2 \cdot 5 \cdot 7$. \pspace

\sol The statement is \textit{false}. Remember given a prime factorization of the numbers, we find the gcd by choosing the \textit{smallest} powers of each prime that appears in the factorizations. So we should have $\gcd(2^3 \cdot 3 \cdot 5, 2 \cdot 3^2 \cdot 7)= 2 \cdot 3$. Instead, the largest power of each prime that appears in the factorizations was chosen which is how we compute the lcm. Therefore, we have $\lcm(2^3 \cdot 3 \cdot 5, 2 \cdot 3^2 \cdot 7)= 2^3 \cdot 3^2 \cdot 5 \cdot 7$. \pvspace{1.5cm}



% Quiz 3
\quizsol \textit{True/False}: $\sqrt[3]{2^8 \cdot 3^3 \cdot 5^1 \cdot 7^5}= 2^2 \cdot 3^1 \cdot 7 \sqrt[3]{2^2 \cdot 5^1 \cdot 7^2}$ \pspace

\sol The statement is \textit{true}. There are two ways to think about this. First, we should write out the numbers and group them into threes and pull out/leave the terms appropriately:
	\[
	\sqrt[3]{2^8 \cdot 3^3 \cdot 5^1 \cdot 7^5}= \sqrt[3]{\underline{2 \cdot 2 \cdot 2} \cdot \underline{2 \cdot 2 \cdot 2} \cdot 2 \cdot 2 \cdot \underline{3 \cdot 3 \cdot 3} \cdot 5 \cdot \underline{7 \cdot 7 \cdot 7} \cdot 7 \cdot 7}= 2^2 \cdot 3^1 \cdot 7 \sqrt[3]{2^2 \cdot 5 \cdot 7^2}
	\]
Alternatively, we can use division. We know that $8/3$ is 2 with remainder 2, $3/3$ is 1 with remainder 0, $1/3$ is 0 with remainder 1, and $5/3$ is 1 with remainder 2. So we can pull out two 3's with 2 remaining, one 3 with 0 remaining, no 5's with 1 remaining, and two 7's with 2 remaining, which gives:
	\[
	\sqrt[3]{2^8 \cdot 3^3 \cdot 5^1 \cdot 7^5}= 2^2 \cdot 3^1 \cdot 7 \sqrt[3]{2^2 \cdot 5^1 \cdot 7^2}
	\] \pvspace{1.5cm}



% Quiz 4 
\quizsol \textit{True/False}: 68 increased by 119\% is $68(1.19)$. \pspace

\sol The statement is \textit{false}. To find 119\% of 68, we would multiply 68 by the percent written as a decimal. This would be $68(1.19)$. However, to increase or decrease a number by a percentage, we compute the number $\#(1 \pm \%)$, where we add if we are increasing, subtract if we are decreasing, $\#$ is the number, and \% is the percentage written as a decimal. So to increase 68 by 119\%, we need to compute $68(1 + 1.19)= 68(2.19)$. \pvspace{1.5cm}














\end{document}