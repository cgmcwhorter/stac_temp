\documentclass[11pt,letterpaper]{article}
\usepackage[lmargin=1in,rmargin=1in,tmargin=1in,bmargin=1in]{geometry}
\usepackage{../style/homework}
\usepackage{../style/commands}
\setbool{quotetype}{true} % True: Side; False: Under
\setbool{hideans}{false} % Student: True; Instructor: False

% -------------------
% Content
% -------------------
\begin{document}

\homework{12: Due 11/15}{Reality continues to ruin my life.}{Bill Watterson}

% Problem 1
\problem{10} Showing all your work, find the domain, vertical asymptotes, and zeros of the following function:
	\[
	f(x)= \dfrac{x + 6}{x - 1}
	\] \pspace

\sol Observe that the numerator and denominator are already factored. The domain is the set of real numbers where the denominator is not zero. But if $x - 1= 0$, then $x= 1$. Therefore, the domain is the set of real numbers such that $x \neq 1$. This also implies that the only vertical asymptote is the line $x= 1$. The zeros are the set of values such that the numerator is 0. But then $x + 6= 0$. This implies that $x= -6$. Therefore, the only zero is $x= -6$. 
	\[
	\boxed{
	\begin{aligned}
	\text{Domain: }& x \in \mathbb{R}, x \neq 1 \\
	\text{Vertical Asymptotes: }& x= 1 \\
	\text{Zeros: }& x= -6
	\end{aligned}
	}
	\]





\newpage





% Problem 2
\problem{10} Showing all your work, find the domain, vertical asymptotes, and zeros of the following function:
	\[
	g(x)= \dfrac{x^2 - 2x - 8}{x + 3}
	\] \pspace

\sol First, we factor the numerator and the denominator:
	\[
	g(x)= \dfrac{x^2 - 2x - 8}{x + 3}= \dfrac{(x - 4)(x + 2)}{x + 3}
	\]
The domain is the set of real numbers where the denominator is not zero. But if $x + 3= 0$, then $x= -3$. Therefore, the domain is the set of real numbers such that $x \neq -3$. This also implies that the only vertical asymptote is the line $x= -3$. The zeros are the set of values such that the numerator is 0. But then $(x - 4)(x + 2)= 0$. This implies that either $x - 4= 0$, i.e. $x= 4$, or $x + 2= 0$, i.e. $x= -2$. Therefore, the zeros are $x= -2, 4$. 
	\[
	\boxed{
	\begin{aligned}
	\text{Domain: }& x \in \mathbb{R}, x \neq -3 \\
	\text{Vertical Asymptotes: }& x= -3 \\
	\text{Zeros: }& x= -2, 4
	\end{aligned}
	}
	\]





\newpage





% Problem 3
\problem{10} Showing all your work, find the domain, vertical asymptotes, and zeros of the following function:
	\[
	h(x)= \dfrac{x - 5}{x^2 + 8x + 12}
	\] \pspace

\sol First, we factor the numerator and the denominator:
	\[
	h(x)= \dfrac{x - 5}{x^2 + 8x + 12}= \dfrac{x - 5}{(x + 2)(x + 6)}
	\]
The domain is the set of real numbers where the denominator is not zero. But if $(x + 2)(x + 6)= 0$, then either $x + 2= 0$, i.e. $x= -2$, or $x + 6= 0$, i.e. $x= -6$. Therefore, the domain is the set of real numbers such that $x \neq -6, -2$. This also implies that the only vertical asymptotes are the lines $x= -6$ and $x= -2$. The zeros are the set of values such that the numerator is 0. But then $x - 5= 0$, i.e. $x= 5$. Therefore, the only zero is $x= 5$.  
	\[
	\boxed{
	\begin{aligned}
	\text{Domain: }& x \in \mathbb{R}, x \neq -6, -2 \\
	\text{Vertical Asymptotes: }& x= -6, x= -2 \\
	\text{Zeros: }& x= 5
	\end{aligned}
	}
	\]





\newpage





% Problem 4
\problem{10} Showing all your work, find the domain, vertical asymptotes, and zeros of the following function:
	\[
	j(x)= \dfrac{x^2 - x - 12}{x^2 - 3x - 18}
	\] \pspace

\sol First, we factor the numerator and the denominator:
	\[
	j(x)= \dfrac{x^2 - x - 12}{x^2 - 3x - 18}= \dfrac{(x - 4)(x + 3)}{(x - 6)(x + 3)}
	\]
The domain is the set of real numbers where the denominator is not zero. But if $(x - 6)(x + 3)= 0$, then either $x - 6= 0$, i.e. $x= 6$, or $x + 3= 0$, i.e. $x= -3$. Therefore, the domain is the set of real numbers such that $x \neq -3, 6$. Now that the domain has been found and there are terms to cancel, we simplify the expression for $j(x)$. 
	\[
	j(x)= \dfrac{(x - 4)\cancel{(x + 3)}}{(x - 6)\cancel{(x + 3)}}= \dfrac{x - 4}{x - 6}
	\]
The vertical asymptotes are where the denominator is 0. But then $x - 6= 0$, i.e. $x= 6$. Therefore, the only vertical asymptote is $x= 6$. The zeros are the set of values such that the numerator is 0. But then $x - 4= 0$, i.e. $x= 4$. Therefore, the only zero is $x= 4$.  
	\[
	\boxed{
	\begin{aligned}
	\text{Domain: }& x \in \mathbb{R}, x \neq -3, 6 \\
	\text{Vertical Asymptotes: }& x= 6 \\
	\text{Zeros: }& x= 4
	\end{aligned}
	}
	\]


%\printpoints
\end{document}