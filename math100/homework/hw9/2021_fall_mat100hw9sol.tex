\documentclass[11pt,letterpaper]{article}
\usepackage[lmargin=1in,rmargin=1in,tmargin=1in,bmargin=1in]{geometry}
\usepackage{../style/homework}
\usepackage{../style/commands}
\setbool{quotetype}{false} % True: Side; False: Under
\setbool{hideans}{false} % Student: True; Instructor: False

% -------------------
% Content
% -------------------
\begin{document}

\homework{9: Due 10/29}{Laziness is nothing more than the habit of resting before you get tired.}{Jules Renard}

% Problem 1
\problem{10} Find the vertex form of the quadratic function $y= x^2 + 4x + 6$. \pspace

\sol The $x$-coefficient is $4$. We have $(\frac{1}{2} \cdot 4)^2= 2^2= 4$. Then we have\dots
	\[
	\begin{aligned}
	y&= x^2 + 4x + 6 \\[0.3cm]
	y&= x^2 + 4x + (4 - 4) + 6 \\[0.3cm]
	y&= (x^2 + 4x + 4) - 4 + 6 \\[0.3cm]
	y&= (x + 2)^2 + 2
	\end{aligned}
	\]





\newpage





% Problem 2
\problem{10} Find the vertex form of the quadratic function $y= x^2 + 4x - 5$. \pspace

\sol The $x$-coefficient is $4$. We have $(\frac{1}{2} \cdot 4)^2= 2^2= 4$. Then we have\dots
	\[
	\begin{aligned}
	y&= x^2 + 4x - 5 \\[0.3cm]
	y&= x^2 + 4x + (4 - 4) - 5 \\[0.3cm]
	y&= (x^2 + 4x + 4) - 4 - 5 \\[0.3cm]
	y&= (x + 2)^2 - 9
	\end{aligned}
	\]





\newpage





% Problem 3
\problem{10} Find the vertex form of the quadratic function $y= 2x^2 - 4x + 8$. \pspace

\sol The $x$-coefficient is $-4$. We have $(\frac{1}{2} \cdot -4)^2= 2^2= 4$. Then we have\dots
	\[
	\begin{aligned}
	y&= x^2 - 4x + 8 \\[0.3cm]
	y&= x^2 - 4x + (4 - 4) + 8 \\[0.3cm]
	y&= (x^2 - 4x + 4) - 4 + 8 \\[0.3cm]
	y&= (x - 2)^2 + 4
	\end{aligned}
	\]





\newpage





% Problem 4
\problem{10} Consider the quadratic function $f(x)= x^2 - 8x + 12$.
\begin{enumerate}[(a)]
\item Determine if the parabola opens upwards or downwards.
\item Is the parabola convex or concave?
\item Does the parabola have a maximum or minimum? 
\item Find the vertex and axis of symmetry. 
\item Find the maximum/minimum value of $f(x)$. 
\end{enumerate} \pspace

\sol
\begin{enumerate}[(a)]
\item Because $a= 1 > 0$, the parabola opens upwards, i.e. the parabola is convex. \pspace

\item Because the parabola opens downwards, it is convex. \pspace

\item Because the parabola opens upwards, the vertex is a minimum. \pspace

\item The vertex occurs when $x= -\frac{b}{2a}= -\frac{-8}{2(1)}= \frac{8}{2}= 4$. But then the axis of symmetry is $x= 4$. We have
	\[
	y(4)= 4^2 - 8(4) + 12= 16 - 32 + 12= -4
	\]
Therefore, the vertex is $(4, -4)$. Alternatively, putting the parabola in vertex form:
	\[
	\begin{aligned}
	y&= x^2 - 8x + 12 \\[0.3cm]
	y&= x^2 - 8x + 16 - 16 + 12 \\[0.3cm]
	y&= (x - 4)^2 - 4
	\end{aligned}
	\]
we can easily see that the vertex is $(4, -4)$ and that the axis of symmetry is $x= 4$. \pspace

\item Because the parabola opens upwards, the parabola has a minimum. The minimum occurs at the vertex. The vertex is $(4, -4)$. Therefore, the maximum value is $-4$. 
\end{enumerate}





\newpage





% Problem 5
\problem{10} Consider the quadratic function $f(x)= -2x^2 - 4x + 4$.
\begin{enumerate}[(a)]
\item Determine if the parabola opens upwards or downwards.
\item Is the parabola convex or concave?
\item Does the parabola have a maximum or minimum? 
\item Find the vertex and axis of symmetry. 
\item Find the maximum/minimum value of $f(x)$. 
\end{enumerate} \pspace

\sol
\begin{enumerate}[(a)]
\item Because $a= -2 < 0$, the parabola opens downwards, i.e. the parabola is concave. \pspace

\item Because the parabola opens downwards, it is concave. \pspace

\item Because the parabola opens downwards, the vertex is a maximum. \pspace

\item The vertex occurs when $x= -\frac{b}{2a}= -\frac{-4}{2(-2)}= -\frac{4}{4}= -1$. But then the axis of symmetry is $x= 1$. We have
	\[
	y(-1)= -2(-1)^2 - 4(-1) + 4= -2 + 4 + 4= 6
	\]
Therefore, the vertex is $(1, 6)$. Alternatively, putting the parabola in vertex form:
	\[
	\begin{aligned}
	y&= -2x^2 - 4x + 4 \\[0.3cm]
	y&= -2(x^2 + 2x - 2) \\[0.3cm]
	y&= -2(x^2 + 2x + 1 - 1 - 2) \\[0.3cm]
	y&= -2\big( (x + 1)^2  - 3 \big) \\[0.3cm]
	y&= -2(x + 1)^2 + 6
	\end{aligned}
	\]
we can easily see that the vertex is $(1, 6)$ and that the axis of symmetry is $x= 1$. \pspace

\item Because the parabola opens downwards, the parabola has a maximum. The maximum occurs at the vertex. The vertex is $(1, 6)$. Therefore, the maximum value is $6$. 
\end{enumerate}


%\printpoints
\end{document}