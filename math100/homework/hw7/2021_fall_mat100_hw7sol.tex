\documentclass[11pt,letterpaper]{article}
\usepackage[lmargin=1in,rmargin=1in,tmargin=1in,bmargin=1in]{geometry}
\usepackage{../style/homework}
\usepackage{../style/commands}
\setbool{quotetype}{false} % True: Side; False: Under
\setbool{hideans}{false} % Student: True; Instructor: False

% -------------------
% Content
% -------------------
\begin{document}

\homework{7: Due 10/13}{Be nice to your sister, Bill. Some day you'll be sleeping on her couch after your first divorce}{Frank Murphy, F is for Family}


% Problem 1
\problem{10} Find the equation of the line through the point $(3, -1)$ with slope $-2$. \pspace

\sol Because the line is not vertical, it must be of the form $y= mx + b$. Because the slope is $-2$, we have $y= -2x + b$. Using the point $(3, -1)$, i.e. $x= 3$ and $y= -1$, we have
	\[
	\begin{aligned}
	y&= -2x + b \\
	-1&= -2(3) + b \\
	-1&= -6 + b \\
	b&= 5
	\end{aligned}
	\]
Therefore, the equation of the line is $y= -2x + 5= 5 - 2x$. 





\newpage





% Problem 2
\problem{10} Find the equation of the line through $(-2, 2)$ and $(3, 4)$. \pspace

\sol Because these points are not vertically aligned, we know the line is not vertical. Therefore, the line has the form $y= mx + b$. First, we compute the slope:
	\[
	m= \dfrac{2 - 4}{-2 - 3}= \dfrac{-2}{-5}= \dfrac{2}{5}
	\]
Then $y= \frac{2}{5}\,x + b$. Now we use the point $(3, 4)$, i.e. $x= 3$ and $y= 4$, and find that\dots
	\[
	\begin{aligned}
	y&= \dfrac{2}{5}\,x + b \\
	4&= \dfrac{2}{5} \cdot 3 + b \\
	4&= \dfrac{6}{5} + b \\
	b&= 4 - \dfrac{6}{5} \\
	b&= \dfrac{20}{5} - \dfrac{6}{5} \\
	b&= \dfrac{20 - 6}{5} \\
	b&= \dfrac{14}{5}
	\end{aligned}
	\]
Therefore, the equation of the line is $y= \frac{2}{5}\,x + \frac{14}{5}= \dfrac{2x + 14}{5}$. 





\newpage





% Problem 3
\problem{10} Find the equation of the horizontal line through $(0, -2)$. \pspace

\sol Because the line is not vertical, the line has the form $y= mx + b$. But because the line is horizontal, we know the line has slope 0, i.e. $m= 0$. Then we have $y= b$. We use the point $(0, -2)$, i.e. $x= 0$ and $y= -2$. But then $-2= b$. Therefore, $y= -2$. 





\newpage





% Problem 4
\problem{10} Find the equation of the line parallel to the line $y= 6 - x$ containing the point $(-6, 1)$. \pspace

\sol The line $y= 6 - x$ is not vertical. Therefore, a line parallel to it is not vertical. Therefore, our line has the form $y= mx + b$. Because this line is parallel to the line $y= 6 - x$, it must have slope equal to the slope of the line $y= 6 - x$. The slope of $y= 6 - x$ is $-1$. Therefore, we know that $m= -1$, i.e. $y= -x + b$. Now we use the point $(-6, 1)$, i.e. $x= -6$ and $y= 1$. But then\dots
	\[
	\begin{aligned}
	y&= -x + b \\
	1&= -(-6) + b \\
	1&= 6 + b \\
	b&= -5 
	\end{aligned}
	\]
Therefore, the line is $y= -x - 5$. 





\newpage





% Problem 5
\problem{10} Find the equation of the line perpendicular to the line $y= \frac{5}{3} x + 1$ passing through the point $(10, -13)$. \pspace

\sol Because the line $y= \frac{5}{3}\,x + 1$ is not horizontal, a line perpendicular to it is not vertical. Then our line is not vertical so that it has the form $y= mx + b$. Because our line is perpendicular to the line $y= \frac{5}{3} x + 1$, the slope of our line is the negative reciprocal of the slope of the line $y= \frac{5}{3} x + 1$. The slope of the line $y= \frac{5}{3} x + 1$ is $\frac{5}{3}$. Therefore, the slope of our line is $-\frac{3}{5}$, i.e. $m= -\frac{3}{5}$. Therefore, $y= -\frac{3}{5}\,x + b$. Using the point $(10, -13)$, i.e. $x= 10$ and $y= -13$, we have\dots
	\[
	\begin{aligned}
	y&= -\frac{3}{5}\,x + b \\
	-13&= -\frac{3}{5} \cdot 10 + b \\
	-13&= -6 + b \\
	b&= -7
	\end{aligned}
	\]
Therefore, the equation of the line $y= -\frac{3}{5}\,x - 7$. 





\newpage





% Problem 6
\problem{10} Find the equation of the line perpendicular to the line $y= 6$ that contains the point $(-3, 9)$. \pspace

\sol The line $y= 6$ is horizontal. Therefore, a line perpendicular to the line $y= 6$ must be vertical, i.e. of the form $x= \#$. Because our line contains the point $(-3, 9)$, we know that the line must be $x= -3$. 





\newpage





% Problem 7
\problem{10} Find the equation of the line that is perpendicular to the line $y= 7x - 1$ that passes through the $x$-intercept of the line $y= 4x - 8$. \pspace

\sol The line $y= 7x - 1$ is not horizontal. Therefore, a line perpendicular to it is not vertical. Then our line must have the form $y= mx + b$. The line $y= 7x - 1$ has slope 7. Because our line is perpendicular to the line $y= 7x - 1$, the slope must be the negative reciprocal of the slope of the line $y= 7x - 1$. Therefore, we know that $m= -\frac{1}{7}$. Then we know that $y= -\frac{1}{7}x + b$. We know that the line contains the $x$-intercept of the line $y= 4x - 8$. The $x$-intercept of a curve is where the curve passes through the $x$-axis, i.e. where $y= 0$. But then
	\[
	\begin{aligned}
	y&= 4x - 8 \\
	0&= 4x - 8 \\
	4x&= 8 \\
	x&= 2
	\end{aligned}
	\]
Therefore, the $x$-intercept of $y= 4x - 8$ is the point $(2, 0)$. Then our line contains $(2, 0)$, i.e. $x= 2$ and $y= 0$. But then\dots
	\[
	\begin{aligned}
	y&= -\frac{1}{7}\,x + b \\
	0&= -\frac{1}{7} \cdot 2 + b \\
	0&= -\frac{2}{7} + b \\
	b&= \frac{2}{7}
	\end{aligned}
	\]
Therefore, the equation of the line is $y= -\frac{1}{7}\,x + \frac{2}{7}= \dfrac{-x + 2}{7}= \dfrac{2 - x}{7}$. 


%\printpoints
\end{document}