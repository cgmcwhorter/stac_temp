\documentclass[11pt,letterpaper]{article}
\usepackage[lmargin=1in,rmargin=1in,tmargin=1in,bmargin=1in]{geometry}
\usepackage{homework}

% -------------------
% Content
% -------------------
\begin{document}
\homework{}

% Problem 1
\problem{10} Determine if the following functions are injective, surjective, and/or bijective. Which of the functions have an inverse function? [No formal proofs required.]
\begin{enumerate}[(a)]
\item $f: \mathbb{R} \to [0,1]$ defined by $f(x)= \sin^2 x$.
\item $g: [0, \frac{\pi}{2}] \to [-1, 1]$ defined by $g(x)= \cos x$.
\item $h: \mathbb{N} \to \mathbb{Z}$ given by $h(x)= 3^n$.
\item $j: \mathbb{Z} \times \mathbb{Z}$ given by $j(x,y)= (x - y + 3)^2$.
\end{enumerate}





\newpage





% Problem 2
\problem{10} Show that the function $f: \mathbb{R} \setminus \{ -1 \} \to \mathbb{R} \setminus \{ 3 \}$ given by $f(x)= \dfrac{3x - 5}{x + 1}$ is a bijection. Explain why this implies $f$ is invertible and then find the inverse for $f(x)$. 





\newpage





% Problem 3
\problem{10} Let $S \subseteq \mathbb{R}$ and $f, g: S \to \mathbb{R}$ be monotone functions. 
\begin{enumerate}[(a)]
\item Prove that $f + g$ is a monotone function. 
\item If $f$ and $f + g$ are increasing on $S$, then is $g$ necessarily increasing on $S$? Prove or give a counterexample. 
\end{enumerate}





\newpage





% Problem 4
\problem{10} Let $f: X \to Y$ and let $A, B \in \mathcal{P}(X)$. 
\begin{enumerate}[(a)]
\item Prove that $f(A \cup B)= f(A) \cup f(B)$. 
\item Is it true that $f(A \cap B)= f(A) \cap f(B)$? Prove or give a counterexample. 
\end{enumerate}





\newpage





% Problem 5
\problem{10} Let $f: X \to Y$ and $A, B \subseteq Y$. Prove that $f^{-1}(A \cap B)= f^{-1}(A) \cap f^{-1}(B)$. 





\newpage





% Problem 6
\problem{10} For each of the following, find a function $f: \mathbb{N} \to \mathbb{Z}$ with the following properties:
\begin{enumerate}[(a)]
\item $f$ is injective but not surjective
\item $f$ is surjective but not injective
\item $f$ is neither surjective nor injective
\item $f$ is a bijection
\end{enumerate}





\end{document}