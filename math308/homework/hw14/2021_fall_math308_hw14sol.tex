\documentclass[11pt,letterpaper]{article}
\usepackage[lmargin=1in,rmargin=1in,tmargin=1in,bmargin=1in]{geometry}
\usepackage{homework}

% -------------------
% Content
% -------------------
\begin{document}
\homework{Solutions --- Caleb McWhorter}

% Problem 1
\problem{10} Do there exist integers $a, b$ such that $2a + 3b= 5$? Explain. Do there exist integers $x, y$ such that $8x + 12y= 3$? Explain. \pspace

\sol Let $a, b$ be integers and let $d= \gcd(a, b)$. We know there exists integers $x, y$ such that $ax + by= d$. In fact, there exist infinitely many integers $x, y$ such that $ax + by= d$. This can be seen from the fact that if $ax + by= d$, then $d= a(x + k \, \frac{b}{d}) + b(y - k \, \frac{a}{d})$ for any integer $k$ because\dots
	\[
	a \left(x + k \, \frac{b}{d} \right) + b \left(y - k \, \frac{a}{d} \right)= \left( ax + k \, \dfrac{ab}{d} \right) + \left( by - k \, \dfrac{ab}{d} \right)= ax + by= d
	\]
In fact, all the solutions to $ax + by= d$ are of the form $x= x_0 + k \, \frac{b}{d}$, $y= y_0 - k \, \frac{a}{d}$ for some integer $k$, where $x_0, y_0$ are a solution to $ax_0 + by_0= d$. Furthermore, if $N$ is an integer with $ax + by= N$ for some $x, y$, then $d$ divides $N$. \pspace

Observe that $\gcd(2, 3)= 1$ and $1$ divides $5$. Therefore, there exists integers $a, b$ such that $2a + 3b= 5$. In fact, taking $a= b= 1$, we have $2a + 3b= 2(1) + 3(1)= 2 + 3= 5$. \pspace

Observe that $\gcd(8, 12)= 4$ and $3$ is not divisible by $4$. Therefore, the equation $8x + 12y= 3$ has no integer solutions. Of course, we did not need the theory above to prove this. Observe that $8x + 12y= 2(4x + 6y)$ must be even. Because $3$ is not even, there cannot be integers $x, y$ with $8x + 12y= 3$. 



\newpage



% Problem 2
\problem{10} Compute $\gcd(2^8 \cdot 3^5 \cdot 7^{10} \cdot 11 \cdot 19^6, \, 2^5 \cdot 3^8 \cdot 5^3 \cdot 11^2 \cdot 13 \cdot 17^3)$. \pspace





\newpage





% Problem 3
\problem{10} Compute $\lcm(2^8 \cdot 3^5 \cdot 7^{10} \cdot 11 \cdot 19^6, \, 2^5 \cdot 3^8 \cdot 5^3 \cdot 11^2 \cdot 13 \cdot 17^3)$. \pspace





\newpage





% Problem 4
\problem{10} Prove that if $a, b \in \mathbb{Z}$, then $ab= \gcd(a, b) \cdot \lcm(a, b)$. \pspace





\newpage





% Problem 5
\problem{10} Use the Euclidean Algorithm to compute $\gcd(36, 98)$. \pspace

\sol We have\dots
	\[
	\begin{aligned}
	98&= 2(36) + 26 \\
	36&= 1(26) + 10 \\
	26&= 2(10) + 6 \\
	10&= 1(6) + 4 \\
	6&= 1(4) + 2 \\
	4&= 2(2)
	\end{aligned}
	\]
Therefore, $\gcd(36, 98)= 2$. 



\newpage



% Problem 6
\problem{10} Use the Euclidean Algorithm to find integers $x, y$ such that $36x + 98y= \gcd(36, 98)$. \pspace

\sol We use the Extended Euclidean Algorithm. From Problem~5, we have\dots
	\[
	\begin{aligned}
	98&= 2(36) + 26 \\
	36&= 1(26) + 10 \\
	26&= 2(10) + 6 \\
	10&= 1(6) + 4 \\
	6&= 1(4) + 2 \\
	4&= 2(2)
	\end{aligned}
	\]
But then we have\dots
	\[
	\begin{aligned}
	2&= 6 - 1(4) \\
	&= 6 - 1 \big(10 - 1(6) \big) \\
	&= 6 - 1(10) + 1(6) \\
	&= 2(6) - 1(10) \\
	&= 2 \big(26 - 2(10) \big) - 1(10) \\
	&= 2(26) - 4(10) - 1(10) \\
	&= 2(26) - 5(10) \\
	&= 2(26) - 5 \big(36 - 1(26) \big) \\
	&= 2(26) - 5(36) + 5(26) \\
	&= 7(26) - 5(36) \\
	&= 7 \big(98 - 2(36) \big) - 5(36) \\
	&= 7(98) - 14(36) - 5(36) \\
	&= 7(98) - 19(36)
	\end{aligned}
	\]
Therefore, taking $x= -19$ and $y= 7$, we have $36x + 98y= 2= \gcd(36, 98)$. 


\end{document}