\documentclass[11pt,letterpaper]{article}
\usepackage[lmargin=1in,rmargin=1in,tmargin=1in,bmargin=1in]{geometry}
\usepackage{homework}

% -------------------
% Content
% -------------------
\begin{document}
\homework{}

% Problem 1
\problem{10} Let $f: A \to \mathbb{R}$ be defined by $f(x):= x^3 - 9x^2 + 23x - 12$, where $A= \{ 1, 3, 6 \}$. Let $g: B \to \mathbb{R}$ be defined by $g(x)= x^2 - 4x + 6$, where 
	\[
	B= \{ x \in \mathbb{N} \;|\; x \text{ divides }6 \} \setminus \{x \colon x \text{ is an even prime number} \}
	\]
Prove that $f= g$. 





\newpage





% Problem 2
\problem{10} Recall the absolute value function, $f(x)= |x|$, is given by
	\[
	|x|= 
	\begin{cases}
	x, & x \geq 0 \\
	-x, & x < 0 
	\end{cases}
	\]
Considering $f: \mathbb{R} \to \mathbb{R}$, determine the following sets:
\begin{enumerate}[(a)]
\item $f((-2, 1])$
\item $f(\mathbb{Z})$
\item $f^{-1}((-2,1])$
\item $f^{-1}(\{-5\})$
\item $f^{-1}(\mathbb{Z})$
\end{enumerate}





\newpage





% Problem 3
\problem{10} Let $f: \mathbb{Z} \to \mathbb{R}$ be given by $f(x)= 2^n$, and let $g: \mathbb{Z} \to \mathbb{R}$ be given by $g(x)= 100 - 3^n$. 
\begin{enumerate}[(a)]
\item Compute $f(1)$.
\item Compute $g(1)$.
\item Compute $(fg)(1)$.
\item Compute $(f \circ g)(1)$.
\item Find the rule for $(fg)(x)$.
\end{enumerate}





\newpage





% Problem 4
\problem{10} Recall that given a function $f: S \to S$, we say that $x \in S$ is a fixed point of $f$ if $f(x)= x$. Let $S= \mathbb{R}$ and let $f$ be the function given by $x \mapsto x^2 + 4x - 10$. Find the fixed points of $f$. How does the answer change if $S= \mathbb{N}$? 





\newpage





% Problem 5
\problem{10} Recall that the image of a function $f: S \to S$ (also called the range) is the set $\im f= \{ f(s) \colon s \in S \}$. Consider the function $f: \mathbb{R} \to \mathbb{R}$ defined by $f(x)= \dfrac{1}{1 + x^2}$. 
\begin{enumerate}[(a)]
\item Determine the error in the following `proof' that $\im f= \mathbb{R}$: \pspace

{\itshape We need prove that $\im f \subseteq \mathbb{R}$ and $\mathbb{R} \subseteq \im f$. Clearly, $f(x) \in \mathbb{R}$ so that $\im f \subseteq \mathbb{R}$. Now let $y \in \mathbb{R}$. Define $x:= \sqrt{\dfrac{1 - y}{y}}$. Then
	\[
	f(x)= \dfrac{1}{1 + x^2}= \dfrac{1}{1 + \dfrac{1 - y}{y}}= \dfrac{1}{\dfrac{y + 1 - y}{y}}= \dfrac{1}{1/y}= y.
	\]
But then $f(x)= y$ and $x \in \mathbb{R}$. Therefore, $\mathbb{R} \subseteq \im f$. Because $\im f \subseteq \mathbb{R}$ and $\mathbb{R} \subseteq \im f$, $\im f = \mathbb{R}$.}

\item Determine $\im f$ and prove that your answer is correct. 
\end{enumerate}





\end{document}