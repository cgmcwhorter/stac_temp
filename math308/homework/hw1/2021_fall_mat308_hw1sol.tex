\documentclass[11pt,letterpaper]{article}
\usepackage[lmargin=1in,rmargin=1in,tmargin=1in,bmargin=1in]{geometry}
\usepackage{homework}

% -------------------
% Content
% -------------------
\begin{document}
\homework{\textit{Caleb McWhorter --- Solutions}}

% Problem 1
\problem{10} Determine if the following sentences are propositions. If the sentence is a proposition, mark it `T'; otherwise, mark the sentence `F.'
	\begin{enumerate}[(a),topsep=0pt]
	\item \usol{0.65cm}{T}: $|3 - 5| > 10$
	\item \usol{0.65cm}{T}: I just started watching \textit{`The Chair.'}
	\item \usol{0.65cm}{T}: The universe is infinite.
	\item \usol{0.65cm}{F}: $n + 1$ is odd.  
	\item \usol{0.65cm}{F}: Why are you doing this homework?
	\end{enumerate} \pspace

\sol 
\begin{enumerate}[(a)]
\item It is \textit{true} that the sentence is a proposition. To be a proposition, it does not matter whether the sentence is true or false. It only matters that the sentence have an unambiguous truth value---whether that be true or false. The expression $|3 - 5| > 10$ is either true or false. 

\item It is \textit{true} that the sentence is a proposition. To be a proposition, it does not matter whether the sentence is true or false. It only matters that the sentence have an unambiguous truth value---whether that be true or false. The individual either just started watching \textit{`The Chair'} or they did not. 

\item It is \textit{true} that the sentence is a proposition. To be a proposition, it does not matter whether the sentence is true or false. It only matters that the sentence have an unambiguous truth value---whether that be true or false. Also, it does not matter whether we know if the proposition is true or false or even if we can know if the statement is true or false. It only matters that the sentence is unambiguously true or false. We do not know whether the universe is infinite or not and we may never know. But the universe either is or is not infinite. 

\item It is \textit{false} that the sentence is a proposition. This is a predicate statement---a sentence depending on a variable that becomes a proposition once the variable has been substituted. The sentence `$n + 1$ is odd' depends on what $n$ is. So the sentence is not unambiguously true or false. Therefore, the sentence is not a proposition. But once we know $n$, the resulting sentence is either true or false. Therefore, the sentence is a predicate statement. 

\item It is \textit{false} that the sentence is a proposition. This sentence has no unambiguous truth value. Generally, it does not make sense to associate a truth value to an interrogative sentence---even if the answer is either true or false. Hence, interrogative sentences are generally not regarded as propositions; although, some argue otherwise.  
\end{enumerate}



\newpage



% Problem 2
\problem{10} Give an original example of a proposition.

\sol Remember, a proposition is any sentence which is unambiguously true or false. Therefore, any such sentence will work, e.g. `$2 + 2= 4$', `if $x + 2= 3$, then $x= 0$', `Today is Wednesday', `The universe has a beginning and an end.', etc. 



\newpage



% Problem 3
\problem{10} Give an original non-example of a proposition.

\sol Remember, a proposition is any sentence which is unambiguously true or false. Therefore, any sentence not meeting any of these will work will work, e.g. `$x + 1= 3$', `Why did they go to the store?', `Get a job.', etc. 



\newpage



% Problem 4
\problem{10} Determine if the following propositions are true (T) or false (F). 
	\begin{enumerate}[(a),topsep=0pt]
	\item \usol{0.65cm}{T}: If $n$ is an integer, then $2n$ is even.
	\item \usol{0.65cm}{F}: Every prime number is odd.
	\item \usol{0.65cm}{T}: $x^2 + 1 > 0$
	\item \usol{0.65cm}{T}: It will either rain tomorrow or not.
	\item \usol{0.65cm}{F}: If $x^2= 9$, then $x= 3$,
	\end{enumerate}

\sol
\begin{enumerate}[(a)]
\item The proposition is true. This is either because we define an even number to be an integer of the form $2n$, in which case the result is immediate, or we define an even number to be divisible by 2, in which case we have $2n= 2 \cdot n$ and hence $2n$ is divisible by 2. 

\item The proposition is false. We can find a counterexample: the integer 2 is prime but is even. [In fact, this is the only even prime.]

\item The proposition is true. For any real number $x$, we know that $x^2 \geq 0$. But because $1 > 0$, we know that $x^2 + 1 > 0$. 

\item The proposition is true. This is an example of a dichotomy, i.e. something that can be broken down into two opposite things. It will always either rain or not rain on any given day. Similarly, every human is either alive or dead, any number is either equal to 1 or not, every light is either on or off, etc. 

\item The proposition is false. We can find a counterexample: if $x= -3$, then $x^2= 9$ but $x \neq 3$. Notice that despite having a variable, this is not a predicate statement. Although in theory $x$ might vary, if it has the supposed property that $x^2= 9$, then $x$ is either always 3 or not. This is because we have turned a predicate statement into a proposition using an implied `for all.' Notice we could have written the proposition `if $x^2= 9$, then $x= 3$' instead as `for all $x \in \mathbb{R}$, if $x^2= 9$, then $x= 3$' or symbolically $(\forall x \in \mathbb{R}) ( x^2= 9 \Rightarrow x= 3)$.  
\end{enumerate}



\newpage



% Problem 5
\problem{10} Negate the following sentences:
	\begin{enumerate}[(a),topsep=0pt]
	\item $2 \cdot 2= 4$ or $3 \cdot 3= 6$
	\item Everyone in the room has taken a mathematics course.
	\item She speaks German and English. 
	\item $x > 1$ and $x$ is an integer.
	\item If you study for the exam, then you will pass. 
	\end{enumerate}

\sol
\begin{enumerate}[(a)]
\item The negation is, `$2 \cdot 2\neq 4$ and $3 \cdot 3\neq 6$.' Remember, $\neg (P \wedge Q) \equiv (\neg P) \vee (\neg Q)$. If $P$ is the proposition is $2 \cdot 2= 4$, its negation is `not $2 \cdot 2= 4$', which in context is properly said $2 \cdot 2 \neq 4$. We handle `$3 \cdot 3= 6$' mutatis mutandis. Finally, `or logic' negates to `and logic.' 

\item The negation is, `There is at least one person in the room that has not taken a mathematics course.' To see this, for example, assume the proposition were true. Its negation would then be false. But then you would only have to find a single person in the room who has not taken a mathematics course. This is an example of negating with quantifiers: $\neg [\forall x P(x)] \equiv \exists x (\neg P(x))$ and $\neg [\exists x P(x)] \equiv \forall x (\neg P(x))$. Therefore moralistically, the negation of `for all' is `there exists' and the negation of `there exists' is `for all.'   

\item The negation is, `She does not speak German or English.' Keep in mind, this is the `mathematics negation', meaning, she may speak German, English, or neither but she cannot speak both. It may then better be phrased as, `She does not speak German or she does not speak English.' Remember $\neg (P \wedge Q) \equiv (\neg P) \vee (\neg Q)$. In this case, $P$ is the proposition `she speaks German' while $Q$ is the proposition `she speaks English.' 

\item The negation is, `$x \leq 1$ or $x$ is not an integer.' Remember $\neg (P \wedge Q) \equiv (\neg P) \vee (\neg Q)$. In this case, we have $P$ is `$x > 1$.' Then $\neg P$ is $x \leq 1$. Whereas $Q$ is the proposition `$x$ is an integer.' Then $\neg Q$ is `$x$ is not an integer.' 

\item The negation is, `You will study for the exam and not pass.' Remember $\neg (P \to Q) \equiv P \wedge (\neg Q)$. In this example, $P$ is the proposition `you study for the exam' and $Q$ is the proposition `you pass the exam.' To help remember the negation of `if' statements, think of the `if' statement as a promise---if $P$, then we promise $Q$---and consider the case when it is true. Then the negation would be false. But for the promise to be false, $P$ must be true and then $Q$ did not happen, i.e. $\neg Q$. But this is precisely $P \wedge (\neg Q)$. 
\end{enumerate}



\newpage



% Problem 6
\problem{10} Negate each of the following propositional formulas $P$ by finding a formula logically equivalent $P$ in which the negation applies only to individual atoms. 
	\begin{enumerate}[(a),topsep=0pt]
	\item $P \vee (\neg Q)$
	\item $\neg Q \to \neg P$
	\item $(P \vee Q) \wedge (\neg P \vee \neg Q)$
	\item $P \wedge Q \to P \vee Q$
	\item $P \vee (Q \Leftrightarrow R)$
	\end{enumerate}

\sol This is just a matter of `symbol' manipulation:
\begin{enumerate}[(a)]
\item 
	\[
	\neg [P \vee (\neg Q)] \equiv \neg P \wedge \neg (\neg Q) \equiv \neg P \wedge Q
	\]

\item 
	\[
	\neg [\neg Q \to \neg P] \equiv  \neg Q \wedge \neg (\neg P) \equiv \neg Q \wedge P
	\]

\item 
	\[
	\begin{aligned}
	\neg [ (P \vee Q) \wedge (\neg P \vee \neg Q) ] &\equiv \neg (P \vee Q) \vee \neg (\neg P \vee \neg Q) \\
	&\equiv (\neg P \wedge \neg Q) \vee (\neg (\neg P) \wedge \neg (\neg Q)) \\
	&\equiv (\neg P \wedge \neg Q) \vee (P \wedge Q)
	\end{aligned}
	\]
In fact, one can show that $(\neg P \wedge \neg Q) \vee (P \wedge Q) \equiv \neg (P \veebar Q)$. This comes as no surprise as $(P \vee Q) \wedge (\neg P \vee \neg Q) \equiv P \veebar Q$. 

\item 
	\[
	\begin{aligned}
	\neg [(P \wedge Q) \to (P \vee Q)] &\equiv (P \wedge Q) \wedge \neg (P \vee Q) \\
	&\equiv (P \wedge Q) \wedge (\neg P \wedge \neg Q)
	\end{aligned}
	\]
However, this can be further simplified: 
	\[
	\begin{aligned}
	(P \wedge Q) \wedge (\neg P \wedge \neg Q)&\equiv P \wedge Q \wedge (\neg P) \wedge (\neg Q) \\
	&\equiv P \wedge (\neg P) \wedge Q \wedge (\neg Q) \\
	&\equiv F_0 \wedge F_0 \\
	&\equiv F_0
	\end{aligned}
	\]
But then this is a contradiction. This means that the statement $P \wedge Q \to P \vee Q$ was a tautology. 

\item Recall that $Q \Leftrightarrow R$ is true if and only if $Q$ and $R$ are either both true or both false. But then its negation is true if and only if one is true and the other is false, i.e. $\neg (Q \Leftrightarrow R) \equiv (Q \wedge \neg R) \vee (\neg Q \wedge R)$. Therefore,
	\[
	\neg (P \vee (Q \Leftrightarrow R) \equiv \neg P \wedge \neg (Q \Leftrightarrow R) \equiv \neg P \wedge [ (Q \wedge \neg R) \vee (\neg Q \wedge R) ]
	\]
\end{enumerate}



\newpage



% Problem 7
\problem{10} Express the proposition ``$P$ unless $Q$'' in terms of the propositions $P$ and $Q$ and the logical symbols $\neg, \wedge, \vee, \to$. [Unless can mean many things, here it means ``if not.'']

\sol We know that ``$P$ unless $Q$'' means that $P$ occurs so long as $Q$ does not. But this is equivalent to the statement that if $Q$ does not occur, then $P$ occurs. But then we have that ``$P$ unless $Q$'' is equivalent to $\neg Q \to P$. 

However, ``$P$ unless $Q$'' is not the same as $Q \to \neg P$. Consider the example where $P$ is the statement, ``I go for a walk,'' and $Q$ is the statement, ``It is raining tomorrow.'' Then $P$ unless $Q$ is the statement, ``I will go for a walk tomorrow unless it is raining.'' We can only confirm from this that if I did not go for a walk tomorrow, then it had to be raining. If it is raining, I may or may not have gone for a walk. That is, we understand `$P$ unless $Q$' to be $P$ is true is $Q$ is false. However, if $Q$ is true, we do not know whether $P$ is true or false. It is worth noting that this is `logical speak.' Colloquially, at least in some contexts, some mean `unless' to mean `if and only if.' 



\newpage



% Problem 8
\problem{10} Recall that the `exclusive or', denoted $\veebar$, was defined by $P \veebar Q \Leftrightarrow (P \vee Q) \wedge \neg (P \wedge Q)$. Show that $P \veebar Q$ is logically equivalent to $P \leftrightarrow \neg\ Q$.



\sol We simply compute the truth table for both $P \veebar Q \equiv (P \vee Q) \wedge \neg (P \wedge Q)$ and $P \leftrightarrow \neg Q$ to show that they are the same.
	\begin{table}[!ht]
	\centering
	\begin{tabular}{c|c||c|c|c|c||c|c}
	$P$ & $Q$ & $P \vee Q$ & $P \wedge Q$ & $\neg (P \wedge Q)$ & $\neg Q$ & $P \veebar Q$ & $P \leftrightarrow \neg Q$ \\ \hline
	T & T & T & T & F & F & F & F \\
	T & F & T & F & T & T & T & T \\
	F & T & T & F & T & F & T & T \\
	F & F & F & F & T & T & F & F
	\end{tabular}
	\end{table} \par
Note that $A \leftrightarrow B \equiv (A \to B) \wedge (B \to A)$. This result implies that $(A \to B) \wedge (B \to A) \equiv (A \wedge B) \vee (\neg A \wedge \neg B)$. 



\newpage



% Problem 9
\problem{10} Compute the truth tables for the following compound propositions. In each case, indicate whether the propositional formula is a tautology, contradiction, or neither. 
	\begin{enumerate}[(a),topsep=0pt]
	\item $(P \wedge Q) \wedge (R \wedge \neg Q)$
	\item $(P \leftrightarrow Q) \leftrightarrow (P \wedge Q) \vee (\neg P \wedge \neg Q)$
	\item $(P \to T_0) \wedge (F_0 \to Q)$
	\end{enumerate}

\sol
\begin{enumerate}[(a)]
\item \phantom{.}
	\begin{table}[!ht]
	\centering
	\begin{tabular}{c|c|c||c|c|c||c}
	$P$ & $Q$ & $R$ & $P \wedge Q$ & $\neg Q$ & $R \wedge \neg Q$ & $(P \wedge Q) \wedge (R \wedge \neg Q)$ \\ \hline
	T & T & T & T & F & F & F \\
	T & T & F & T & F & F & F \\
	T & F & T & F & T & T & F \\
	T & F & F & F & T & F & F \\
	F & T & T & F & F & F & F \\
	F & T & F & F & F & F & F \\
	F & F & T & F & T & T & F \\
	F & F & F & F & T & F & F
	\end{tabular}
	\end{table} \par
Therefore, $(P \wedge Q) \wedge (R \wedge \neg Q)$ is a contradiction. 

\item \phantom{.}
	\begin{table}[!ht]
	\hspace{-1.1cm}
	\begin{tabular}{c|c||c|c|c|c|c|c||c}
	$P$ & $Q$ & $P \leftrightarrow Q$ & $P \wedge Q$ & $\neg P$ & $\neg Q$ & $\neg P \wedge \neg Q$ & $(P \wedge Q) \vee (\neg P \wedge \neg Q)$ & $(P \leftrightarrow Q) \leftrightarrow (P \wedge Q) \vee (\neg P \wedge \neg Q)$ \\ \hline
	T & T & T & T & F & F & F & T & T \\
	T & F & F & F & F & T & F & F & T \\
	F & T & F & F & T & F & F & F & T \\
	F & F & T & F & T & T & T & T & T 
	\end{tabular}
	\end{table} \par
Therefore, $(P \leftrightarrow Q) \leftrightarrow (P \wedge Q) \vee (\neg P \wedge \neg Q)$ is a tautology. 

\item \phantom{.}
	\begin{table}[!ht]
	\centering
	\begin{tabular}{c|c||c|c|c}
	$P$ & $Q$ & $P \to T_0$ & $F_0 \to Q$ & $(P \to T_0) \wedge (F_0 \to Q)$ \\ \hline
	T & T & T & T & T \\
	T & F & T & T & T \\
	F & T & T & T & T \\
	F & F & T & T & T
	\end{tabular}
	\end{table} \par
Therefore, $(P \to T_0) \wedge (F_0 \to Q)$ is a tautology. 
\end{enumerate}



\newpage



% Problem 10
\problem{10} Determine if the logical symbol $\to$ is associative. Be sure to fully justify your answer. 

\sol The symbol $\to$ is associative if $P \to (Q \to R) \equiv (P \to Q) \to R$. To determine this, we can compute the associated logic table:
	\begin{table}[!ht]
	\centering
	\begin{tabular}{c|c|c||c|c||c|c}
	$P$ & $Q$ & $R$ & $Q \to R$ & $P \to Q$ & $P \to (Q \to R)$ & $(P \to Q) \to R$ \\ \hline
	T & T & T & T & T & T & T \\
	T & T & F & F & T & F & F \\
	T & F & T & T & F & T & T \\
	T & F & F & T & F & T & T \\
	F & T & T & T & T & T & T \\
	F & T & F & F & T & T & F \\
	F & F & T & T & T & T & T \\
	F & F & F & T & T & T & F
	\end{tabular}
	\end{table} \par
Because the last two columns are not identical, we see that the logical symbol $\to$ is not associative. 

We can also find a counterexample. Let $P$ be the proposition $0= 1$ (or generally any false proposition), $Q$ be the proposition $1 > 0$ (or generally any true proposition), and $R$ be the proposition $1= 2$ (or generally any false proposition). Then $Q \to R$ is false (because $Q$ is true and $R$ is false). But $P \to (Q \to R)$ is true because $P$ is false. However, $P \to Q$ is true (because $P$ is false). But then $(P \to Q) \to R$ is false because $P \to Q$ is true while $R$ is false. 



\newpage



% Problem 11
\problem{10} Give the converse and contrapositive of the following statements. 
	\begin{enumerate}[(a),topsep=0pt]
	\item $P \to Q$
	\item If it is snowing outside, then it is cold.
	\end{enumerate}

\sol
\begin{enumerate}[(a)]
\item The converse of $P \to Q$ is $Q \to P$. The contrapositive of $P \to Q$ is $\neg Q \to \neg P$. 

\item The converse of ``if it is snowing outside, then it is cold.'' is the sentence ``if it is cold, then it is snowing outside.'' The contrapositive of ``if it is snowing outside, then it is cold.'' is the sentence ``if it is not cold, then it is not snowing outside.'' 
\end{enumerate}



\newpage



% Problem 12
\problem{10} Determine if the following argument is logical. Explain. 
	\begin{table}[!ht]
	\centering
	\begin{tabular}{c}
	$P \to R$ \\
	$\neg P \to Q$ \\
	$Q \to S$ \\ \hline
	$\therefore \neg R \to S$
	\end{tabular}
	\end{table}

\sol The argument is logical, i.e. it is valid. By assumption, we know that $P \to R$. But this is logically equivalent to its contrapositive which is $\neg R \to \neg P$. But we know that $\neg P \to Q$ and $Q \to S$. But then $\neg R \to S$, as claimed. One can also compute the associated and confirm that whenever $P \to R$, $\neg P \to Q$, and $Q \to S$ are assumed to be true, then $\neg R \to S$ is always true. 
        \begin{table}[!ht]
        \centering
        \begin{tabular}{c|c|c|c||c|c|c||c}
        $P$ & $Q$ & $R$ & $S$ & $P \to R$ & $\neg P \to Q$ & $Q \to S$ & $\neg R \to S$ \\ \hline \hline \rowcolor{lightgray} 
        T & T & T & T & T & T & T & T \\
        T & T & T & F & T & T & F & T \\
        T & T & F & T & F & T & T & T \\
        T & T & F & F & F & T & F & F \\ \rowcolor{lightgray} 
        T & F & T & T & T & T & T & T \\ \rowcolor{lightgray} 
        T & F & T & F & T & T & T & T \\
        T & F & F & T & F & T & T & T \\
        T & F & F & F & F & T & T & F \\ \rowcolor{lightgray} 
        F & T & T & T & T & T & T & T \\
        F & T & T & F & T & T & F & T \\ \rowcolor{lightgray} 
        F & T & F & T & T & T & T & T \\
        F & T & F & F & T & T & F & F \\
        F & F & T & T & T & F & T & T \\
        F & F & T & F & T & F & T & T \\
        F & F & F & T & T & F & T & T \\
        F & F & F & F & T & F & T & F
        \end{tabular}
        \end{table}


\end{document}