\documentclass[11pt,letterpaper]{article}
\usepackage[lmargin=1in,rmargin=1in,tmargin=1in,bmargin=1in]{geometry}
\usepackage{homework}

% -------------------
% Content
% -------------------
\begin{document}
\homework{\textit{Caleb McWhorter --- Solutions}}

%Problem 1
\problem{10} Watch the video \href{https://www.youtube.com/watch?v=UIKGV2cTgqA&ab_channel=JaredKhan}{New Math (Tom Lehrer)}. Who was Tom Lehrer? What other songs is he famous for? How does the video relate to the material? \pspace

\sol Any `solution' here is fine. Although, it would be worth noting that the video does contain the same method of arithmetic in other bases that we discussed in class. 





\newpage





% Problem 2
\problem{10} Convert the following integers to binary:
        \begin{enumerate}[(a)]
        \item 33
        \item 156
        \end{enumerate} \pspace

\sol First, observe
	\[
	\begin{aligned}
	2^0&= 1 &\quad\quad 2^4&= 16 \\
	2^1&= 2 & 2^5&= 32 \\
	2^2&= 4 & 2^6&= 64 \\
	2^3&= 8 & 2^7&= 128
	\end{aligned}
	\]

\begin{enumerate}[(a)]
\item We have\dots
	\[
	33= 32 + 1= 1 \cdot 2^5 + 0 \cdot 2^4 + 0 \cdot 2^3 + 0 \cdot 2^2 + 0 \cdot 2^1 + 1 \cdot 2^0= 100001_2
	\]

\item We have\dots
	\[
	156= 128 + 16 + 8 + 4= 1 \cdot 2^7 + 0 \cdot 2^6 + 0 \cdot 2^5 + 1 \cdot 2^4 + 1 \cdot 2^3 + 1 \cdot 2^2 + 0 \cdot 2^1 + 0 \cdot 2^0= 10011100_2
	\]
\end{enumerate}





\newpage





% Problem 3
\problem{10} Convert the following binary integers to base-10 integers:
        \begin{enumerate}[(a)]
        \item $101101_2$
        \item $10110100_2$
        \end{enumerate} \pspace

\sol First, observe
	\[
	\begin{aligned}
	2^0&= 1 &\quad\quad 2^4&= 16 \\
	2^1&= 2 & 2^5&= 32 \\
	2^2&= 4 & 2^6&= 64 \\
	2^3&= 8 & 2^7&= 128
	\end{aligned}
	\]

\begin{enumerate}[(a)]
\item We have\dots
	\[
	101101_2= 1 \cdot 2^5 + 0 \cdot 2^4 + 1 \cdot 2^3 + 1 \cdot 2^2 + 0 \cdot 2^1 + 1 \cdot 2^0= 32 + 8 + 4 + 1= 45
	\]

\item We have\dots
	\[
	10110100_2= 1 \cdot 2^7 + 0 \cdot 2^6 + 1 \cdot 2^5 + 1 \cdot 2^4 + 0 \cdot 2^3 + 1 \cdot 2^2 + 0 \cdot 2^1 + 0 \cdot 2^0= 128 + 32 + 16 + 4= 180
	\]
\end{enumerate}





\newpage





% Problem 4
\problem{10} Convert the following integers to hexadecimal:
        \begin{enumerate}[(a)]
        \item 59
        \item 200
        \end{enumerate}

\sol First, observe $16^0= 1$, $16^1= 16$, $16^2= 256$ and
	\[
	\begin{aligned}
	16 \cdot 0&= 0 &\quad 16 \cdot 1&= 16 &\quad 16 \cdot 2&= 32 &\quad 16 \cdot 3&= 48 \\
	16 \cdot 4&= 64 & 16 \cdot 5&= 80 & 16 \cdot 6&= 96 & 16 \cdot 7&= 112 \\
	16 \cdot 8&= 128 & 16 \cdot 9&= 144 & 16 \cdot 10&= 160 & 16 \cdot 11&= 176 \\
	16 \cdot 12&= 192 & 16 \cdot 13&= 208 & 16 \cdot 14&= 224 & 16 \cdot 15&= 240
	\end{aligned}
	\]
Also, recall
	\[
	\begin{aligned}
	0&= 0 &\quad 1&= 1 &\quad 2&= 2 &\quad 3&= 3 \\
	4&= 1 & 5&= 5 & 6&= 1 & 7&= 7 \\
	8&= 1 & 9&= 9 & 10&= A & 11&= B \\
	12&= C & 2&= D & 14&= E & 15&= F 
	\end{aligned}
	\]

\begin{enumerate}[(a)]
\item We have\dots
	\[
	59= 48 + 11= 3 \cdot 16^1 + 11 \cdot 16^0= 3\text{B}
	\]

\item  We have\dots
	\[
	200= 192 + 8= 12 \cdot 16^1 + 8 \cdot 16^0= \text{C}8
	\]
\end{enumerate}





\newpage





% Problem 5
\problem{10} Convert the following hexadecimal numbers to base-10:
        \begin{enumerate}[(a)]
        \item F1
        \item 5BF
        \end{enumerate} \pspace

\sol First, observe $16^0= 1$, $16^1= 16$, $16^2= 256$ and
	\[
	\begin{aligned}
	16 \cdot 0&= 0 &\quad 16 \cdot 1&= 16 &\quad 16 \cdot 2&= 32 &\quad 16 \cdot 3&= 48 \\
	16 \cdot 4&= 64 & 16 \cdot 5&= 80 & 16 \cdot 6&= 96 & 16 \cdot 7&= 112 \\
	16 \cdot 8&= 128 & 16 \cdot 9&= 144 & 16 \cdot 10&= 160 & 16 \cdot 11&= 176 \\
	16 \cdot 12&= 192 & 16 \cdot 13&= 208 & 16 \cdot 14&= 224 & 16 \cdot 15&= 240
	\end{aligned}
	\]
Also, recall
	\[
	\begin{aligned}
	0&= 0 &\quad 1&= 1 &\quad 2&= 2 &\quad 3&= 3 \\
	4&= 1 & 5&= 5 & 6&= 1 & 7&= 7 \\
	8&= 1 & 9&= 9 & 10&= A & 11&= B \\
	12&= C & 2&= D & 14&= E & 15&= F 
	\end{aligned}
	\]
\begin{enumerate}[(a)] 
\item We have\dots
	\[
	F1= 15 \cdot 16^1 + 1 \cdot 16^0= 240 + 1= 241
	\]

\item We have\dots
	\[
	5BF= 5 \cdot 16^2 + 11 \cdot 16^1 + 15 \cdot 16^0= 1280 + 176 + 15= 1471
	\]
\end{enumerate}





\newpage





% Problem 6
\problem{10} Perform the following operations in binary:
        \begin{enumerate}[(a)]
        \item $101_2 + 11_2$
        \item $11011_2 + 11101_2$
        \end{enumerate} \pspace

\sol
\begin{enumerate}[(a)]
\item \phantom{.}
	\begin{table}[!ht]
	\centering
	\begin{tabular}{cccc}
	   & $^11$  & $^10$                  & $1$ \\
	   &            & $\phantom{^1}1$ & $1$ \\ \hline
	$1$ & $0$ & $0$                      & $0$
	\end{tabular}
	\end{table}

\item \phantom{.}
	\begin{table}[!ht]
	\centering
	\begin{tabular}{cccccc}
	       & $^11$ & $^11$ & $^10$ & $^11$ & $1$ \\
	       & $\phantom{^1}1$ & $\phantom{^1}1$ & $\phantom{^1}1$ & $\phantom{^1}0$ & $1$ \\ \hline
	$1$ & $1$ & $1$ & $0$ & $0$ & $0$
	\end{tabular}
	\end{table} 
\end{enumerate}





\end{document}