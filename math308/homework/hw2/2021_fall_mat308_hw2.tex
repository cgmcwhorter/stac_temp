\documentclass[11pt,letterpaper]{article}
\usepackage[lmargin=1in,rmargin=1in,tmargin=1in,bmargin=1in]{geometry}
\usepackage{homework}

% -------------------
% Content
% -------------------
\begin{document}
\homework{}

% Problem 1
\problem{10} Determine if the following sentences are predicates. If the sentence is a predicate, mark it `T'; otherwise, mark the sentence `F.' [Let the universal set be $\mathbb{R}$.]
	\begin{enumerate}[(a)]
	\item \uans{1.5cm}: $x$ is odd.
	\item \uans{1.5cm}: $x^2 + x + 1$
	\item \uans{1.5cm}: $P(x) \colon x^2 + 1 < 0$
	\item \uans{1.5cm}: $Q(x) \colon x \text{ is an integer}$
	\item \uans{1.5cm}: $R(x, y) \colon x^2 < y^3$
	\end{enumerate} 
	




\newpage





% Problem 2
\problem{10} Give an original example of a predicate having more than one variable. \pspace





\newpage





% Problem 3
\problem{10} Let $P(x)$ be the predicate $P(x) \colon 1 \leq 2^x \leq 100$. Suppose that the domain is the nonnegative integers. What is the truth set for $P(x)$? What is the truth set if the domain were instead the set of real numbers? \pspace





\newpage





% Problem 4
\problem{10} Defining appropriate propositional functions and variables, write the following English sentences using logical symbols and the functions/variables that you defined. 
        \begin{enumerate}[(a)]
        \item Every cloud has a silver lining.
        \item All that glitters is not gold. 
        \item Every human is guilty of all the good that they did not do.
        \item None but the brave deserve the fair. 
        \end{enumerate}





\newpage





% Problem 5
\problem{10} Define the following predicates:
	\begin{enumerate}[(i)]
	\item $P(x) \colon x > 0$
	\item $Q(x) \colon x \text{ is even}$
	\item $R(x) \colon x \text{ is a perfect square}$
	\item $S(x) \colon x \text{ is divisible by }4$
	\item $T(x) \colon x \text{ is divisible by }5$
	\end{enumerate}
Write the following in symbolic form:
	\begin{enumerate}[(a)]
	\item Any perfect square is positive. 
	\item If an integer is divisible by 4, then the integer is even. 
	\item No even integer is divisible by 5. 
	\end{enumerate}
Write the following in the form of an English sentence:
	\begin{enumerate}
	\item[(d)] $\forall x\, (S(x) \to Q(x))$
	\item[(e)] $\exists x\, (S(x) \wedge \neg R(x))$
	\end{enumerate}





\newpage





% Problem 6
\problem{10} Let $P(x)$ be the predicate $x^2= x$. Determine if the following statements are true or false:
	\begin{enumerate}[(a)]
	\item \uans{1.5cm}: $P(0)$
	\item \uans{1.5cm}: $P(-1)$
	\item \uans{1.5cm}: $\forall x,\, P(x)$
	\item \uans{1.5cm}: $\exists x,\, P(x)$
	\item \uans{1.5cm}: $\exists! x,\, P(x)$
	\end{enumerate}





\newpage





% Problem 7
\problem{12} Let $P(x), Q(x), R(x)$ denote the predicates $1 - 2x= 7$, $x^2= 9$, and $x^2 > 9$, respectively. Determine whether the following propositions are true or false. If the statement is true, explain why. If the statement is false, give a counterexample. 
        \begin{enumerate}[(a)]
        \item $(\forall x)(P(x) \wedge Q(x))$
        \item $(\exists x)(P(x) \wedge Q(x))$
        \item $(\forall x)(P(x) \to Q(x))$
        \item $(\forall x)(P(x) \to R(x))$
        \item $(\exists x)(P(x) \vee R(x))$
        \item $(\exists! x)(P(x) \wedge Q(x))$ 
        \end{enumerate}





\newpage





% Problem 8
\problem{10} What well-known property does the following proposition represent: $\forall x\, \forall y\, \forall z\, [x + (y + z) = (x + y) + z]$. \pspace





\newpage





% Problem 9
\problem{10} Determine if the following statements are true or false:
	\begin{enumerate}[(a)]
	\item \uans{1.5cm}: $\exists x\, \exists y\, (xy = 1)$
	\item \uans{1.5cm}: $\exists x\, \forall y\, (xy= 1)$
	\item \uans{1.5cm}: $\forall x\, \exists y\, (xy= 1)$
	\item \uans{1.5cm}: $\forall x\, \forall y\, (x^2 + y = 1)$
	\item \uans{1.5cm}: $\forall x\, \exists y\, (x^2 + y= 1)$
	\end{enumerate}





\newpage





% Problem 10
\problem{10} Negate the following proposition:
	\[
	\forall x\, \exists y\, \left( P(x,y) \wedge Q(x,y) \to R(x,y) \right)
	\]





\newpage





% Problem 11
\problem{10} One way of stating the definition for a function $f(x)$ to have limit $L$ at $x$, i.e. $\displaystyle \lim_{x \to a} f(x)= L$, is as follows:
	\[
	(\forall \epsilon > 0)(\exists \delta > 0)(\forall x \in \mathbb{R})[ 0 < |x - a| < \delta \Rightarrow |f(x) - L| < \epsilon]
	\]
Give a definition for a function $f(x)$ to not have a limit at $x= a$ by negating the statement above. Your answer should not contain any negations. \pspace





\newpage





% Problem 12
\problem{10} The universal and existential quantifier do not necessarily `distribute' over $\wedge$ and $\vee$. One of the following `equivalences' is not correct:
	\[
	\begin{aligned}
	\forall x\, (P(x) \wedge Q(x)) &\Longleftrightarrow \forall x\, P(x) \wedge \forall x\, Q(x) \\
	\forall x\, (P(x) \vee Q(x)) &\Longleftrightarrow \forall x\, P(x) \vee \forall x\, Q(x)
	\end{aligned}
	\]
Determine which one is always true and state it. For the one that is false, give an example to show that it is false. \pspace





\end{document}