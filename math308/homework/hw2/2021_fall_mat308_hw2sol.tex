\documentclass[11pt,letterpaper]{article}
\usepackage[lmargin=1in,rmargin=1in,tmargin=1in,bmargin=1in]{geometry}
\usepackage{homework}

% -------------------
% Content
% -------------------
\begin{document}

\homework{\textit{Caleb McWhorter --- Solutions}}

% Problem 1
\problem{10} Determine if the following sentences are predicates. If the sentence is a predicate, mark it `T'; otherwise, mark the sentence `F.' [Let the universal set be $\mathbb{R}$.]
	\begin{enumerate}[(a)]
	\item \usol{0.65cm}{T}: $x$ is odd.
	\item \usol{0.65cm}{F}: $x^2 + x + 1$
	\item \usol{0.65cm}{T}: $P(x) \colon x^2 + 1 < 0$
	\item \usol{0.65cm}{T}: $Q(x) \colon x \text{ is an integer}$
	\item \usol{0.65cm}{T}: $R(x, y) \colon x^2 < y^3$
	\end{enumerate} \pspace

\sol Recall that predicate becomes a proposition when the variables are substituted for their values. 
\begin{enumerate}[(a)]
\item Given an $x$, either $x$ is odd or it is not. 

\item Given an $x$, the result is simply an expression and will not hold a true or false value. 

\item Given an $x$, the resulting inequality will either be true or false. 

\item Given an $x$, either $x$ is an integer or not. 

\item Given a $x, y$, either the resulting inequality will either be true or false. 
\end{enumerate}





\newpage




	
% Problem 2
\problem{10} Give an original example of a predicate having more than one variable.

\sol Remember that to be a predicate, once the variables have been substituted, the result should be a proposition, i.e. something which is unambiguously true or false. There are infinitely many examples, e.g. $P(x,y) \colon x = y$, $Q(x,y,z) \colon x^2 + y^2= z^2$, etc. 





\newpage





% Problem 3
\problem{10} Let $P(x)$ be the predicate $P(x) \colon 1 \leq 2^x \leq 100$. Suppose that the domain is the nonnegative integers. What is the truth set for $P(x)$? What is the truth set if the domain were instead the set of real numbers? 

\sol Suppose the domain is the nonnegative integers, i.e. $\mathbb{Z}_{\geq 0}= \{ 0, 1, 2, 3, \ldots \}$. We need $1 \leq 2^x \leq 100$. For the left inequality to hold, we need $1 \leq 2^x$, which implies that $0 \leq x$. For the right inequality to hold, we require $2^x \leq 100$. But this implies that $x \leq \log_2 100 \approx 6.644$. Because $x \in \mathbb{Z}$, this implies that $x \leq 6$. Then $x \geq 0$ and $x \leq 6$, i.e. $0 \leq x \leq 6$. Therefore, the truth set is $\{ 0, 1, 2, 3, 4, 5, 6 \}$. 

Suppose now that the domain is the set of real numbers. For the left inequality to hold, we need $1 \leq 2^x$, which implies that $0 \leq x$. For the right inequality to hold, we require $2^x \leq 100$. But this implies that $x \leq \log_2 100 \approx 6.644$. Then $0 \leq x$ and $x \leq \log_2 100$. Therefore, the truth set is $[0, \log_2 100]$. 





\newpage





% Problem 4
\problem{10} Defining appropriate propositional functions and variables, write the following English sentences using logical symbols and the functions/variables that you defined. 
        \begin{enumerate}[(a)]
        \item Every cloud has a silver lining.
        \item All that glitters is not gold. 
        \item Every human is guilty of all the good that they did not do.
        \item None but the brave deserve the fair. 
        \end{enumerate}

\sol There are many ways to do this\dots
\begin{enumerate}[(a)]
\item Let $P(x)$ be the predicate that $x$ is a cloud and $Q(x)$ denote the predicate that $x$ has a silver lining. Then `Every cloud has a silver lining' could be translated as $\forall x, P(x) \to Q(x)$. 

\item Let $P(x)$ be the predicate that $x$ glitters and let $Q(x)$ denote the predicate that $x$ is gold. Then `All that glitters is not gold' could be translated as $\exists x, P(x) \wedge \neg Q(x)$. 

\item Let $P(x, y)$ denote the predicate that $x$ did $y$, $Q(x)$ denote the predicate that $x$ is good, and $R(x,y)$ denote the predicate that $x$ is guilty of $y$. Then `Every human is guilty of all the good that they did not do' could be translated as $(\forall x)(\forall y) [ Q(y) \wedge \neg P(x,y) \to R(x,y)]$.

\item Let $P(x)$ be the predicate that $x$ is brave and let $Q(x)$ denote the predicate that $x$ deserves the fair. Then `None but the brave deserve the fair' could be translated as $\forall x, Q(x) \to P(x)$. 
\end{enumerate}





\newpage





% Problem 5
\problem{10} Define the following predicates:
	\begin{enumerate}[(i)]
	\item $P(x) \colon x > 0$
	\item $Q(x) \colon x \text{ is even}$
	\item $R(x) \colon x \text{ is a perfect square}$
	\item $S(x) \colon x \text{ is divisible by }4$
	\item $T(x) \colon x \text{ is divisible by }5$
	\end{enumerate}
Write the following in symbolic form:
	\begin{enumerate}[(a)]
	\item Any perfect square is positive. 
	\item If an integer is divisible by 4, then the integer is even. 
	\item No even integer is divisible by 5. 
	\end{enumerate}
Write the following in the form of an English sentence:
	\begin{enumerate}
	\item[(d)] $\forall x\, (S(x) \to Q(x))$
	\item[(e)] $\exists x\, (S(x) \wedge \neg R(x))$
	\end{enumerate} \pspace

\sol
\begin{enumerate}[(a)]
\item $\forall x, R(x) \to P(x)$

\item $\forall x, S(x) \to Q(x)$

\item $\forall x, Q(x) \to \neg T(x)$

\item All integers divisible by 4 are even.

\item There exists an integer divisible by four that is not a perfect square. 
\end{enumerate}





\newpage





% Problem 6
\problem{10} Let $P(x)$ be the predicate $x^2= x$. Determine if the following statements are true or false:
	\begin{enumerate}[(a)]
	\item \usol{0.65cm}{T}: $P(0)$
	\item \usol{0.65cm}{F}: $P(-1)$
	\item \usol{0.65cm}{F}: $\forall x,\, P(x)$
	\item \usol{0.65cm}{T}: $\exists x,\, P(x)$
	\item \usol{0.65cm}{F}: $\exists! x,\, P(x)$
	\end{enumerate} \pspace

\sol
\begin{enumerate}[(a)]
\item $P(0) \colon 0^2= 0$, which is equivalent to $0= 0$, which is true. 

\item $P(-1) \colon (-1)^2= -1$, which is equivalent to $1= -1$, which is false.

\item From (b), we already know that $-1$ is a counterexample to this statement. 

\item From part (a), we already know that $x= 0$ is such an example. 

\item $P(1) \colon 1^2= 1$, which is equivalent to $1= 1$, which is true. Combining this with (a), we know there are at least two values for $x$, namely $0$ and $1$, for which $P(x)$ is true. 
\end{enumerate}





\newpage





% Problem 7
\problem{12} Let $P(x), Q(x), R(x)$ denote the predicates $1 - 2x= 7$, $x^2= 9$, and $x^2 > 9$, respectively. Determine whether the following propositions are true or false. If the statement is true, explain why. If the statement is false, give a counterexample. 
        \begin{enumerate}[(a)]
        \item $(\forall x)(P(x) \wedge Q(x))$
        \item $(\exists x)(P(x) \wedge Q(x))$
        \item $(\forall x)(P(x) \to Q(x))$
        \item $(\forall x)(P(x) \to R(x))$
        \item $(\exists x)(P(x) \vee R(x))$
        \item $(\exists! x)(P(x) \wedge Q(x))$ 
        \end{enumerate} \pspace

\sol
\begin{enumerate}[(a)]
\item The statement is \textit{false}. For the statement to be true, we need both $P(x)$ and $Q(x)$ to be true. However, for `most' $x$ in the domain, both $P(x)$ and $Q(x)$ are false. As a counterexample, if $x= 0$ then both $P(x)$ and $Q(x)$ are false so that $P(x) \wedge Q(x)$ are false. Then the truth set is not the entire domain. 
 
\item The statement is \textit{true}. For the statement to be true, there must be an $x$ such that $P(x)$ and $Q(x)$ are true. If $P(x)$ is true, then $1 - 2x= 7$, which implies $x= -3$. But if $x= -3$, then $(-3)^2= 9$ so that $Q(x)$ is true. But then if $x= -3$, both $P(x)$ and $Q(x)$ are true so that $P(x) \wedge Q(x)$ is true. But then the truth set is non-empty. 

\item The statement is \textit{true}. We know that $P(x)$ is true if and only if $1 - 2x= 7$ is true, which is true if and only if $x= -3$. From (b), we know that if $x= -3$ that $Q(x)$ is true. But then for $x= -3$, $P(x) \to Q(x)$ is true. If $x \neq -3$, then $P(x)$ is false so that $P(x) \to Q(x)$ is true. But then the truth set is the entire domain. 

\item The statement is \textit{false}. From (c), we know that if $x= -3$, then $P(x)$ is true. But if $x= -3$, then $R(x) \colon (-3)^2 > 9$ is false. Then $P(x) \to R(x)$ is false. Therefore, $x= -3$ is a counterexample. Then the truth set is not the entire domain. 

\item The statement is \textit{true}. For $P(x) \vee R(x)$ to be true, either $P(x)$ or $R(x)$ is true. From (b), we know that if $x= -3$, then $P(x)$ is true. But then $P(x) \vee R(x)$ is true. Therefore, the truth set is non-empty. 

\item The statement is \textit{true}. For $P(x) \wedge Q(x)$ to be true, $P(x)$ and $Q(x)$ need to be true. From (c), we know the only $x$ in the domain such that $P(x)$ is true is $x= -3$. But from (c), we know that $Q(x)$ is true if $x= -3$. Therefore, if $x= -3$, then $P(x)$ and $Q(x)$ are true so that $P(x) \cap Q(x)$ is true. But if $x \neq -3$, then $P(x)$ is false so that $P(x) \wedge Q(x)$ is false. But then $x= -3$ is the unique value in the truth set. 
\end{enumerate}





\newpage





% Problem 8
\problem{10} What well-known property does the following proposition represent: $\forall x\, \forall y\, \forall z\, [x + (y + z) = (x + y) + z]$.

\sol This is quantified version of the associative property for addition for the real numbers. 





\newpage





% Problem 9
\problem{10} Determine if the following statements are true or false:
	\begin{enumerate}[(a)]
	\item \usol{0.65cm}{T}: $\exists x\, \exists y\, (xy = 1)$
	\item \usol{0.65cm}{F}: $\exists x\, \forall y\, (xy= 1)$
	\item \usol{0.65cm}{F}: $\forall x\, \exists y\, (xy= 1)$
	\item \usol{0.65cm}{F}: $\forall x\, \forall y\, (x^2 + y = 1)$
	\item \usol{0.65cm}{T}: $\forall x\, \exists y\, (x^2 + y= 1)$
	\end{enumerate}

\sol

\begin{enumerate}[(a)]
\item If $x= y= 1$, then $xy= 1$. 

\item If $y= 0$, then $xy= 0 \neq 1$ for all $x \in \mathbb{R}$. 

\item If $x= 0$, then $xy= 0 \neq 1$ for all $y \in \mathbb{R}$. Note that if we require $x \neq 0$, the statement is true: if $x \neq 0$, define $y:= 1/x$. But then $xy= x \cdot \frac{1}{x}= 1$. 

\item If $x= y= 0$, then $x^2 + y= 0 \neq 1$. 

\item Given $x \in \mathbb{R}$, define $y:= 1 - x^2$. But then $x^2 + y= x^2 + (1 - x^2)= 1$. 
\end{enumerate}





\newpage





% Problem 10
\problem{10} Negate the following proposition:
	\[
	\forall x\, \exists y\, \left( P(x,y) \wedge Q(x,y) \to R(x,y) \right)
	\]

\sol
	\[
	\begin{aligned}
	\neg [\forall x\, \exists y\, \left( P(x,y) \wedge Q(x,y) \to R(x,y) \right)] \\[0.5cm]
	\exists x\, \neg[\exists y\, \left( P(x,y) \wedge Q(x,y) \to R(x,y) \right)] \\[0.5cm]
	\exists x\, \forall y\, \neg\left( P(x,y) \wedge Q(x,y) \to R(x,y) \right) \\[0.5cm]
	\exists x\, \forall y\, \left( P(x,y) \wedge Q(x,y) \wedge  \neg R(x,y) \right)
	\end{aligned}
	\]





\newpage





% Problem 11
\problem{10} One way of stating the definition for a function $f(x)$ to have limit $L$ at $x$, i.e. $\displaystyle \lim_{x \to a} f(x)= L$, is as follows:
	\[
	(\forall \epsilon > 0)(\exists \delta > 0)(\forall x \in \mathbb{R})[ 0 < |x - a| < \delta \Rightarrow |f(x) - L| < \epsilon]
	\]
Give a definition for a function $f(x)$ to not have a limit at $x= a$ by negating the statement above. Your answer should not contain any negations. 

\sol
	\[
	\begin{aligned}
	\neg [(\forall \epsilon > 0)(\exists \delta > 0)(\forall x \in \mathbb{R})[ 0 < |x - a| < \delta \Rightarrow |f(x) - L| < \epsilon]] \\[0.5cm]
	(\exists \epsilon > 0) \neg[(\exists \delta > 0)(\forall x \in \mathbb{R})[ 0 < |x - a| < \delta \Rightarrow |f(x) - L| < \epsilon]] \\[0.5cm]
	(\exists \epsilon > 0)(\forall \delta > 0) \neg[(\forall x \in \mathbb{R})[ 0 < |x - a| < \delta \Rightarrow |f(x) - L| < \epsilon]] \\[0.5cm]
	(\exists \epsilon > 0)(\forall \delta > 0)(\exists x \in \mathbb{R}) \neg[ 0 < |x - a| < \delta \Rightarrow |f(x) - L| < \epsilon] \\[0.5cm]
	(\exists \epsilon > 0)(\forall \delta > 0)(\exists x \in \mathbb{R}) [ 0 < |x - a| < \delta \wedge \neg (|f(x) - L| < \epsilon) ] \\[0.5cm]
	(\exists \epsilon > 0)(\forall \delta > 0)(\exists x \in \mathbb{R}) [ 0 < |x - a| < \delta \wedge  (|f(x) - L| \geq \epsilon) ]
	\end{aligned}
	\]





\newpage





% Problem 12
\problem{10} The universal and existential quantifier do not necessarily `distribute' over $\wedge$ and $\vee$. One of the following `equivalences' is not correct:
	\[
	\begin{aligned}
	\forall x\, (P(x) \wedge Q(x)) &\Longleftrightarrow \forall x\, P(x) \wedge \forall x\, Q(x) \\
	\forall x\, (P(x) \vee Q(x)) &\Longleftrightarrow \forall x\, P(x) \vee \forall x\, Q(x)
	\end{aligned}
	\]
Determine which one is always true and state it. For the one that is false, give an example to show that it is false. 

\sol The proposition $\forall x\, (P(x) \wedge Q(x)) \Longleftrightarrow \forall x\, P(x) \wedge \forall x\, Q(x)$ is true. If the left side is true, then for all $x$ in the domain, $P(x) \wedge Q(x)$ is true, i.e. $P(x)$ is true and $Q(x)$ is true. But then $\forall x\, P(x)$ is true and $\forall x, Q(x)$ is true. But then $\forall x\, P(x) \wedge \forall x\, Q(x)$ is true. If the left side is false, then there is an $x$ in the domain such that $P(x) \wedge Q(x)$ is false, i.e. either $P(x)$ is false or $Q(x)$ is false. Without loss of generality, assume $P(x)$ is false. But then $\forall x\, P(x)$ is false so that $\forall x\, P(x) \wedge \forall x\, Q(x)$ is false. Now suppose the right hand side is true. Then $\forall x\, P(x) \wedge \forall x\, Q(x)$ is true. But then $\forall x\, P(x)$ is true and $\forall x, Q(x)$ is true. But then $P(x) \wedge Q(x)$ is true for all $x$ in the domain so that $\forall x\, (P(x) \wedge Q(x))$ is true. Suppose that the right hand side is false. Then $\forall x\, P(x) \wedge \forall x\, Q(x)$ is false so that either $\forall x, P(x)$ is false or $\forall x, Q(x)$ is false. Without loss of generality, assume that $\forall x, P(x)$ is false. Then there is an $x$ in the domain such that $P(x)$ is false. But then $P(x) \wedge Q(x)$ is false for this $x$ so that $\forall x\, (P(x) \wedge Q(x))$ is false. 

To see that $\forall x\, (P(x) \vee Q(x))$ and $\forall x\, P(x) \vee \forall x\, Q(x)$ are not equivalent, consider predicates the following example: let the domain be the set of chess pieces and define $P(x)$ to be the predicate that $x$ is black and define $Q(x)$ be the predicate that $x$ is white. It is true that for any $x$ in the domain, that either $x$ is black or $x$ is white, i.e. $\forall x\, (P(x) \vee Q(x))$. However, $\forall x\, P(x) \vee \forall x\, Q(x)$ is false because it is not true that every chess piece is black or that every chess piece is white. However, it is true that $\forall x\, P(x) \vee \forall x\, Q(x) \Rightarrow \forall x\, (P(x) \vee Q(x))$. 





\end{document}