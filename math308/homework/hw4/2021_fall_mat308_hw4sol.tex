\documentclass[11pt,letterpaper]{article}
\usepackage[lmargin=1in,rmargin=1in,tmargin=1in,bmargin=1in]{geometry}
\usepackage{homework}

% -------------------
% Content
% -------------------
\begin{document}
\homework{\textit{Caleb McWhorter --- Solutions}}

% Problem 1
\problem{10} Read Keith Conrad's \href{https://kconrad.math.uconn.edu/blurbs/proofs/writingtips.pdf}{``Advice on Mathematical Writing.''} What are some things that you learned about good mathematical exposition that you may have otherwise thought? \pspace

\sol {\itshape Solutions will vary.}





\newpage





% Problem 2
\problem{10} Watch 3Blue1Brown's \href{https://www.youtube.com/watch?v=M64HUIJFTZM}{``The unexpectedly hard windmill question (2011 IMO, Q2)''} and \href{https://www.youtube.com/watch?v=OkmNXy7er84}{``The hardest problem on the hardest test."} What proof strategies do you believe these videos exhibit? \pspace

\sol {\itshape Solutions will vary.}





\newpage





% Problem 3
\problem{10} Prove that for $n \in \mathbb{N}$, if $n^2 + (n + 1)^2= (n + 2)^2$, then $n= 3$. \pspace

\sol Let $n \in \mathbb{N}$ and assume that $n^2 + (n + 1)^2= (n + 2)^2$. But then
	\[
	\begin{aligned}
	n^2 + (n + 1)^2&= (n + 2)^2 \\
	n^2 + (n^2 + 2n + 1)&= n^2 + 4n + 4 \\
	2n^2 + 2n + 1&= n^2 + 4n + 4 \\
	n^2 - 2n - 3&= 0 \\
	(n - 3)(n + 1)&= 0.
	\end{aligned}
	\]
But then $n - 3= 0$ or $n + 1= 0$. This implies that $n= 3$ or $n= -1$. But because $n \in \mathbb{N}$, it must be that $n= 3$. 





\newpage





% Problem 4
\problem{10} Recall that an integer $n$ is called \textit{even} if there is an integer $k$ such that $n= 2k$ and called odd if there is an integer $k$ such that $n= 2k + 1$. Prove that the product of two odd integers is odd. \pspace

\sol Let $n, m \in \mathbb{Z}$ be odd. But then there exist $k_1, k_2 \in \mathbb{Z}$ such that $n= 2k_1 + 1$ and $m= 2k_2 + 1$, respectively. Then
	\[
	nm= (2k_1 + 1)(2k_2 + 1)= 4k_1k_2 + 2k_1 + 2k_2 + 1= 2 (2k_1k_2 + k_1 + k_2) + 1.
	\]
Because $2, k_1, k_2 \in \mathbb{Z}$, we know that $2k_1k_2 + k_1 + k_2 \in \mathbb{Z}$. Therefore, $nm= 2 (2k_1k_2 + k_1 + k_2) + 1$ is odd. 





\newpage





% Problem 5
\problem{10} Rewrite the proof below to be shorter using either ``without loss of generality'' or ``mutatis mutandis'': \pspace

\begin{thm*}
For all $a, b \in \mathbb{R}$, $|ab|= |a| \, |b|$. 
\end{thm*}

\pf

\textit{Case 1} ($a, b \geq 0$): Here $|a|= a$, $|b|= b$, and $ab \geq 0$. But then $|ab|= ab = |a| \, |b|$. 

\textit{Case 2} ($a < 0, b \geq 0$): Here $|a|= -a$ and $|b|= b$. If $b= 0$, then $|b|= 0$ and $ab= 0$. But then $|ab|= |0|= 0 = -ab= |a| \,|b|$. Otherwise, $b > 0$ and then $ab < 0$. Then $|ab|= -ab= |a| \, |b|$. 

\textit{Case 3} ($a \geq 0, b < 0$): Here $|a|= a$ and $|b|= -b$. If $a= 0$, then $|a|= 0$ and $ab= 0$. But then $|ab|= |0|= 0= -ab= |a| \, |b|$. Otherwise, $a > 0$ and then $ab < 0$. Then $|ab|= -ab= |a| \, |b|$. 

\textit{Case 4} ($a, b < 0$): Here $|a|= -a$, $|b|= -b$, and $ab > 0$. Then $|ab|= ab= (-a)(-b)= |a| \, |b|$. \qed \pspace

\sol There are several approaches. For instance, \dots

\pf

\textit{Case 1} (Both Nonnegative): Assume that $a, b \geq 0$. Here $|a|= a$, $|b|= b$, and $ab \geq 0$. But then $|ab|= ab = |a| \, |b|$. 

\textit{Case 2} (Negative \& Nonnegative): By possibly interchanging the roles of $a$ and $b$, without loss of generality, we may assume that $a< 0$ and $b \geq 0$. Here $|a|= -a$ and $|b|= b$. If $b= 0$, then $|b|= 0$ and $ab= 0$. But then $|ab|= |0|= 0 = -ab= |a| \,|b|$. Otherwise, $b > 0$ and then $ab < 0$. Then $|ab|= -ab= |a| \, |b|$. 

\textit{Case 3} (Both Negative): Assume that $a, b < 0$. Here $|a|= -a$, $|b|= -b$, and $ab > 0$. Then $|ab|= ab= (-a)(-b)= |a| \, |b|$. \qed \pspace

\begin{center} {\bfseries OR} \end{center}

\pf

\textit{Case 1} ($a, b \geq 0$): Here $|a|= a$, $|b|= b$, and $ab \geq 0$. But then $|ab|= ab = |a| \, |b|$. 

\textit{Case 2} ($a < 0, b \geq 0$): Here $|a|= -a$ and $|b|= b$. If $b= 0$, then $|b|= 0$ and $ab= 0$. But then $|ab|= |0|= 0 = -ab= |a| \,|b|$. Otherwise, $b > 0$ and then $ab < 0$. Then $|ab|= -ab= |a| \, |b|$. 

\textit{Case 3} ($a \geq 0, b < 0$): This is proved as in Case~2, mutatis mutandis.  

\textit{Case 4} ($a, b < 0$): Here $|a|= -a$, $|b|= -b$, and $ab > 0$. Then $|ab|= ab= (-a)(-b)= |a| \, |b|$. \qed \pspace





\newpage





% Problem 6
\problem{10} By mimicking the proof that $\sqrt{2}$ is irrational, prove that $\sqrt{p}$ is irrational for any prime $p$. \pspace

\sol Suppose that $\sqrt{p}$ were rational. Then there exist $a, b \in \mathbb{Z}$ such that $\sqrt{p}= \frac{a}{b}$. Without loss of generality, we may assume that $\gcd(a, b)= 1$. But then
	\[
	\begin{aligned}
	\sqrt{p}&= \dfrac{a}{b} \\
	p&= \dfrac{a^2}{b^2} \\
	pb^2&= a^2.
	\end{aligned}
	\]
Because $p$ divides the left side of the equality, we know that $p$ divides $a^2$. But then $p$ divides $a$. Therefore, $a= kp$ for some $k \in \mathbb{Z}$. But then
	\[
	\begin{aligned}
	pb^2&= a^2 \\
	pb^2&= (kp)^2 \\
	pb^2&= k^2p^2 \\
	b^2&= k^2p.
	\end{aligned}
	\]
As $p$ divides the right side of the equality, we know that $p$ divides $b^2$. But then $p$ divides $b$. Therefore, $\gcd(a, b) \geq p > 1$, a contradiction. Therefore, $\sqrt{p}$ is irrational. 

\begin{center} {\bfseries OR} \end{center}

Suppose that $\sqrt{p}$ were rational. Then there exist $a, b \in \mathbb{Z}$ such that $\sqrt{p}= \frac{a}{b}$. But then
	\[
	\begin{aligned}
	\sqrt{p}&= \dfrac{a}{b} \\
	p&= \dfrac{a^2}{b^2} \\
	pb^2&= a^2.
	\end{aligned}
	\]
Because $a^2, b^2$ are squares, each prime in the prime factorizations of $a^2$ and $b^2$ occurs an even number of times. But then $p$ occurs in the prime factorization of the left side of the equality an odd number of times. However, $p$ occurs in the prime factorization of the right side of the equality an even number of times. This contradicts the uniqueness of the factorization for $pb^2$ and $a^2$. Therefore, it must be that $\sqrt{p}$ is irrational. 

 



\newpage





% Problem 6
\problem{10} Consider the checkerboard below that has two squares from each corner removed from the board. Prove that this board cannot be covered with the `T-shapes' (or its rotations) shown on the right. \pspace

\drawboard

\sol A normal chessboard consists of 64 squares---32 white and 32 black. However, we have removed 4 squares: a white and black square from each corner. Therefore, this chessboard consists of 60 squares---30 white and 30 black. Each `T-shape' covers 4 squares. Therefore, we must use 15 `T-shapes' to cover the board. 

Now consider the `center' of the `T-shape' piece, i.e. the square adjacent to 3 squares in the `T-shape.' When placed on the board, this `center' square either covers a white or black square. If the `center' square covers a white square, the `T-shape' covers 3 black squares and 1 white square. If the `center' square covers a black square, the `T-shape' covers 3 white squares and 1 black square. Therefore, each `T-shape' placed on the board reduces the number of uncovered white and black squares by an odd number. 

As there are 30 white squares and 30 black squares needing to be covered, an even number of `T-shape' pieces must be used. It is impossible to use both 15 `T-shapes' and an even number of `T-shapes' to cover the board. Therefore, it is impossible to tile this board only using `T-shapes.' 

 





















\end{document}