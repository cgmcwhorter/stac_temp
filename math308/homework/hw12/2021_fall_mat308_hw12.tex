\documentclass[11pt,letterpaper]{article}
\usepackage[lmargin=1in,rmargin=1in,tmargin=1in,bmargin=1in]{geometry}
\usepackage{homework}

% -------------------
% Content
% -------------------
\begin{document}
\homework{}

% Problem 1
\problem{10} Define a relation $\sim$ on $\mathbb{N} \times \mathbb{N}$ via $(x, y) \sim (a, b)$ if and only if $x - y= a - b$.  
        \begin{enumerate}[(a)]
        \item Is $(3, 1) \sim (2, 5)$? Explain.
        \item Is $(7, 3) \sim (5, 1)$? Explain. 
        \item Show that $\sim$ is an equivalence relation on $X$.
        \item Find at least 3 elements in each of the equivalence classes $[(1, 1)]$ and $[(3, 5)]$. 
        \end{enumerate} \pspace





\newpage





% Problem 2
\problem{10}  Define a relation on $\mathbb{R}$ via $x \sim y$ if and only if $x \leq y$. Prove or disprove whether $\sim$ is an equivalence relation on $\mathbb{R}$. \pspace





\newpage





% Problem 3
\problem{10} Define a relation on $\mathbb{R}^2$ via $(x, y) \sim (a, b)$ if and only if $(x, y)$ and $(a, b)$ are the same distance from the origin. 
	\begin{enumerate}[(a)]
	\item Prove that $\sim$ is an equivalence relation.
	\item Describe the equivalence classes graphically. 
	\item Describe graphically how the equivalence classes partition $\mathbb{R}^2$. 
	\end{enumerate}





\newpage





% Problem 4
\problem{10} Define a relation on $\mathbb{Z}$ via $a \sim b$ if and only if $a$ and $b$ have the same parity, i.e. $a$ and $b$ are either both even or they are both odd. 
        \begin{enumerate}[(a)]
        \item Show that $\sim$ is an equivalence relation. 
        \item Describe all the equivalence classes, i.e. determine the set $\mathbb{Z}/\sim$. 
        \end{enumerate} \pspace





\newpage





% Problem 5
\problem{10} Prove that if $X$ is a set and $S$ is a nonempty subset of $X$, then $\{ S, X \setminus S \}$ is a partition of $X$. \pspace





\newpage





% Problem 6
\problem{10} Let $X$ be a nonempty set. Every equivalence relation $\sim$ on $X$ gives rise to a partition on $X$. Moreover, every partition on $X$ gives rise to an equivalence relation $\sim$ on $X$. We proved the first statement in class. Suppose that $\{ X_i \}_{i \in \mathcal{I}}$ is a partition of $X$. Show that this partition induces an equivalence relation $X/\sim$ given by $a \sim b$ if and only if $a, b \in X_i$ for some $i \in \mathcal{I}$. \pspace





\end{document}