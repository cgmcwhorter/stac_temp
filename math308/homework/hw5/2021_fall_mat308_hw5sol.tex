\documentclass[11pt,letterpaper]{article}
\usepackage[lmargin=1in,rmargin=1in,tmargin=1in,bmargin=1in]{geometry}
\usepackage{homework}

% -------------------
% Content
% -------------------
\begin{document}
\homework{\textit{Caleb McWhorter --- Solutions}}

% Problem 1
\problem{10} List at least 3 elements from each of the following sets:
        \begin{enumerate}[(a)]
        \item $\{ n \in \mathbb{N} \colon \exists k \in \mathbb{N}, n= 6k \}$
        \item $\{ x \in \mathbb{R} \colon \exists y \in \mathbb{R}, x= y^2 \}$
        \item $\{ m \in \mathbb{N} \colon \sqrt[3]{m} \in \mathbb{N} \}$
        \item $\{ q \in \mathbb{Q} \colon 4q + 1 \in \mathbb{N} \}$
        \item $\{ a \in \mathbb{N} \colon \exists b\, \exists c,\, b, c \in \mathbb{N}, a^2 + b^2= c^2 \}$
        \end{enumerate}





\newpage





% Problem 2
\problem{10} Use the set-builder notation to give a set equal to each of the following sets:
        \begin{enumerate}[(a)]
        \item $\{ 1, 4, 9, 16, 25, 36, 49, 64, \ldots \}$
        \item $\{ 0, \pm 3, \pm 6, \pm 9, \pm 12, \pm 15, \ldots \}$
        \item The set of rational numbers between 0 and 1.  
        \item The set of functions passing through the point $(6, 5)$. 
        \item The set of differentiable functions with a horizontal tangent line at $x= 1$.
        \end{enumerate}





\newpage





% Problem 3
\problem{10} Let $\mathscr{U}= \{ 1, 2, 3, \{1\}, \{2\}, \{1,2\} \}$. Let $A= \{ 2, 1, 2 \}$ and $B= \{ 1 \}$. 
	\begin{enumerate}[(a)]
	\item Is $A \in \mathscr{U}$? Explain.
	\item Is $A \subseteq \mathscr{U}$? Explain.
	\item Is $B \in \mathscr{U}$? Explain. 
	\item Is $B \subseteq \mathscr{U}$? Explain. 
	\end{enumerate}





\newpage





% Problem 4
\problem{20} Define the following sets:
	\[
	\begin{aligned}
	A&= \{ 1, 2, 3, 4, 5, 6, 7, 8, 9, 10 \} \\
	B&= \{ 1, 3, 5, 7, 9 \} \\
	C&= \{ 2, 4, 6, 8, 10 \} \\
	D&= \{ 2, 3, 5, 7 \} \\
	E&= \{ 4, 8, 9 \} \\
	F&= \{ 1, 2, \{3\} \} 
	\end{aligned}
	\]
Compute the following sets:
	\begin{enumerate}[(a)] \itemsep=0.2ex
	\item $A \cap B$
	\item $C \cup D$
	\item $D \cap E$
	\item $D \setminus B$
	\item $B \setminus A$
	\item $B \times C$
	\item $(D \cap F) \cup (B \cap E)$
	\end{enumerate}
In addition, answer the following:
	\begin{enumerate} \itemsep=0.2ex
	\item[(h)] Is $F \subseteq A$? Explain. 
	\item[(i)] Is $B \cap F= \{ 1, 3 \}$? Explain. 
	\item[(j)] Is $A$ a universal set for $B, C, D, E, F$? If it is, compute $D^c$. If not, explain why. 
	\end{enumerate}





\newpage





% Problem 5
\problem{10} Compute each of the following sets:
        \begin{enumerate}[(a)]
        \item $\mathscr{P}(\emptyset)$
        \item $\mathscr{P}(\{ 1, \{ 1 \} \})$
        \item $\mathscr{P}(\{ 1, e, \pi \})$
        \item $\mathscr{P}( \{ 1 \} \times \{ a, b \})$
        \end{enumerate}





\newpage





% Problem 6
\problem{10} Suppose $A, B$ are sets with a common universal set $\mathscr{U}$. Denote each of the following sets with a Venn diagram: 
	\begin{enumerate}[(a)]
	\item $A \cap B^c$
	\item $(A \cup B)^c$
	\item $(A \cup B) \setminus (A \cap B)$
	\end{enumerate}





\newpage





% Problem 7
\problem{10} Suppose $A, B, C$ are sets with a common universal set $\mathscr{U}$. For each of the Venn diagrams, write down the shaded sets.  
	\begin{enumerate}[(a)]
	\item \diagramOne
	\item \diagramTwo
	\item \diagramThree
	\end{enumerate}





\newpage





% Problem 8
\problem{10} Let $A= \{ b, c \}$. Suppose that $A \cup B= \{ a, b, c, e \}$ and $B \cup C= \{ a, c, d, e, f \}$. From this information can we determine the sets $A, B, C$? Explain. If not, what is the minimal additional information (in terms of unions and intersections of the sets alone) would uniquely determine the three sets? \pspace





\end{document}