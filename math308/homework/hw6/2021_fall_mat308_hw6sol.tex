\documentclass[11pt,letterpaper]{article}
\usepackage[lmargin=1in,rmargin=1in,tmargin=1in,bmargin=1in]{geometry}
\usepackage{homework}

% -------------------
% Content
% -------------------
\begin{document}
\homework{\textit{Caleb McWhorter --- Solutions}}

% Problem 1
\problem{20} Describe all sets (if any) with\dots
        \begin{enumerate}[(a)] 
        \item no proper subsets. 
        \item one proper subset. 
        \item two proper subsets. 
        \end{enumerate} \pspace

\sol
\begin{enumerate}[(a)]
\item Because $\emptyset$ is a subset of every set, every nonempty set $S$ has a proper subset. If $S= \emptyset$, then $\emptyset \subseteq S$ but $S= \emptyset$ so that $\emptyset$ is not a proper subset of $S$. Therefore, there are no sets without at least one proper subset. 

\item We know that $\emptyset$ has no proper subsets. If $S$ is a nonempty set with at least two elements, say $a, b \in S$, then $\{ a \} \subseteq S$ and $\{ b \} \subseteq S$ so that $S$ has at least two proper subsets. So suppose $S$ is a singleton set, i.e. $S= \{ a \}$. Then $\emptyset \subseteq S$ and $S \neq \emptyset$. Therefore, $S$ has exactly one proper subset. But then the only sets with exactly one proper subset are singleton sets. 

\item We know that $\emptyset$ has no proper subsets. From (b), we know that singleton sets have exactly one proper subset. Suppose that $S$ has at least two elements, say $a, b \in S$. But then $\emptyset \subseteq S$, $\{ a \} \subseteq S$, and $\{ b \} \subseteq S$ are all proper subsets of $S$ so that $S$ has at least three proper subsets. Therefore, there are no sets with exactly two proper subsets. 
\end{enumerate}

{\itshape Remark.} If a set $S$ is infinite, then $S$ has an infinite amount of proper subsets: for all $s \in S$, $\{ s \} \subseteq S$ is a proper subset. We claim that if $S$ is a finite set with $n$ elements, then $S$ has $2^n - 1$ proper subsets. 

{\bfseries Proposition.} {\itshape If $S$ is a finite set with $n$ elements, then $S$ has $2^n - 1$ proper subsets.} 

{\noindent\itshape Proof. } Suppose that $|S|= 0$. But then $S= \emptyset$. From (a) above, we know that $S= \emptyset$ has no proper subsets. Furthermore, $2^0 - 1= 1 - 1= 0$. Now let $S$ be a singleton set, say $S= \{ s \}$. From (b), we know that $S$ has one proper subset---namely, $\emptyset$. Observe that $|S|= 1$ and $2^1 - 1 = 2 - 1= 1$. Now assume that for any finite set $S$ with $|S|= k$ that $S$ has $2^k - 1$ proper subsets. 

Now let $S$ be a set with $|S|= k + 1$. Choose an element $s \in S$. Consider all the proper subsets of $S$ that do not contain $S$. But each such subset is a subset of $S \setminus \{ s \}$. Conversely, every proper subset of $S \setminus \{ s \} \subsetneq S$ is a subset of $S$ that does not contain $S$. Therefore, the number of proper subsets of $S$ not containing $s$ is the number of proper subsets of $S \setminus \{ s \}$. We know that $|S \setminus \{ s \}|= k$. By the induction hypothesis, the number of proper subsets of $S \setminus \{ s \}$ is $2^k - 1$. 

Now consider the proper subsets of $S$ containing $s$. Suppose that $A \subsetneq S$ is a proper subset of $S$ with $s \in A$. Then $A \setminus \{ s \} \subseteq A \subsetneq S$ is a proper subset of $S$ not containing $s$. Conversely, if $B \subsetneq S$ is a proper subset of $S$ not containing $s$, then $B \cup \{ s \} \subseteq S$ is a proper subset of $S$ containing $S$, unless $B= S \setminus \{ s \}$ in which case $B \cup \{ s \}= S$ is not proper. Therefore, the proper subsets of $S$ not containing $s$, except for $S \setminus \{ s \}$, are in one-to-one correspondence with the proper subsets of $S$ containing $s$. Then there are $(2^k - 1) + (2^k - 1) + 1= 2 \cdot 2^k - 1= 2^{k+1} - 1$ proper subsets of $S$. Therefore, by induction, the number of proper subsets of a set $S$ with $|S|= n$ is $2^n - 1$. 

{\noindent\itshape Remark.} If one knows that the number of proper subsets of a finite set $S$ with $|S|= n$ is $2^n$, then an immediate corollary is the number of proper subsets of $S$ is $2^n - 1$: if $S$ is empty the result is clear, and if $|S|= n > 0$, then the only non-proper subset of $S$ is $S$ itself, making $2^n - 1$ proper subsets. 

This makes the problem simple. If $S$ is infinite, it is clear that $S$ cannot have exactly none, one, or two proper subsets. If $S$ is finite with $|S|= n$, then $S$ has $2^n - 1$ proper subsets. But for all $n \in \mathbb{Z}_{\geq 0}$, $2^n - 1 \notin \{ 0, 2 \}$. We only have $2^n - 1= 1$ if $n= 1$, but then $S$ is a singleton set. 





\newpage





% Problem 2
\problem{20} The symmetric difference of two sets $A$ and $B$, denoted $A \Delta B$, is defined by $A \Delta B:= (A \setminus B) \cup (B \setminus A)$. 
	\begin{enumerate}[(a)]
	\item Describe $A \Delta B$ in words. 
	\item Show that $A \Delta B= (A \cup B) - (A \cap B)$.
	\item Prove that the symmetric difference is commutative. 
	\item Prove that if $A \Delta B= \emptyset$, then $A= B$. Is the converse true? 
	\end{enumerate} \pspace

\sol
\begin{enumerate}[(a)]
\item The set $A \setminus B$ is the set of elements that are in $A$ but not in $B$. The set $B \setminus A$ is the set of elements that are in $B$ but not in $A$. Therefore, $A \Delta B$ is the set of elements that are only in $A$ or only in $B$. 

\item Let $x \in A \Delta B:= (A \setminus B) \cup (B \setminus A)$. Then $x \in A \setminus B$ or $x \in B \setminus A$. Assume that $x \in A \setminus B$. Then $x \in A$ and $x \notin B$. Because $x \in A$, we know that $x \in A \cup B$. Because $x \in A$ and $x \notin B$, we know that $x \notin A \cap B$. But then $x \in (A \cup B) - (A \cap B)$. Now assume that $x \in B \setminus A$. Then $x \in B$ and $x \notin A$. Because $x \in B$, we know that $x \in A \cup B$. But because $x \in B$ and $x \notin A$, we know $x \notin A \cap B$. Therefore, $x \in (A \cup B) - (A \cap B)$. Therefore, if $x \in A \Delta B$, then $x \in (A \cup B) - (A \cap B)$ so that  $A \Delta B \subseteq (A \cup B) - (A \cap B)$. 

Now let $x \in (A \cup B) - (A \cap B)$. Then $x \in A$ and $x \notin A \cap B$ or $x \in B$ and $x \notin A \cap B$. Assume that $x \in A$ and $x \notin A \cap B$. But then $x \in A$ and $x \notin B$. Therefore, $x \in A \setminus B$ so that $x \in A \Delta B= (A \setminus B) \cup (B \setminus A)$. Now assume that $x \in B$ and $x \notin A \cap B$. But then $x \in B$ and $x \notin A$. Therefore, $x \in B \setminus A$ so that $x \in A \Delta B= (A \setminus B) \cup (B \setminus A)$. But then if $x \in (A \cup B) - (A \cap B)$, then $x \in A \Delta B$ so that $(A \cup B) - (A \cap B) \subseteq A \Delta B$. Therefore, $A \Delta B= (A \cup B) - (A \cap B)$. 

\begin{center} {\bfseries OR} \end{center}

	\[
	\begin{aligned}
	x \in A \Delta B&\Longleftrightarrow x \in (A \setminus B) \cup (B \setminus A) \\
	&\Longleftrightarrow (x \in A \setminus B) \vee (x \in B \setminus A) \\
	&\Longleftrightarrow (x \in A \wedge x \notin B) \vee (x \in B \wedge x \notin A) \\
	&\Longleftrightarrow [(x \in A \wedge x \notin B) \vee x \in B] \wedge [(x \in A \wedge x \notin B) \vee x \notin A] \\
	&\Longleftrightarrow [(x \in A \vee x \in B) \wedge (x \notin B \vee x \in B)] \wedge [(x \in A \vee x \notin A) \wedge (x \notin B \vee x \notin A)] \\
	&\Longleftrightarrow [(x \in A \vee x \in B) \wedge T_0] \wedge [T_0 \wedge (x \notin B \vee x \notin A)] \\
	&\Longleftrightarrow (x \in A \vee x \in B) \wedge (x \notin B \vee x \notin A) \\
	&\Longleftrightarrow (x \in A \vee x \in B) \wedge (x \notin A \vee x \notin B) \\
	&\Longleftrightarrow (x \in A \cup B) \wedge (x \notin A \cap B) \\
	&\Longleftrightarrow x \in (A \cup B) - (A \cap B)
	\end{aligned}
	\] \pspace

\item Using the commutative of unions, observe that\dots
	\[
	A \Delta B:= (A \setminus B) \cup (B \setminus A)= (B \setminus A) \cup (A \setminus B) =: B \Delta A
	\]

\item Suppose that $A \Delta B= \emptyset$. We then have $(A \setminus B) \cup (B \setminus A)= \emptyset$. Therefore, $A \setminus B= \emptyset$ and $B \setminus A= \emptyset$. Because $A \setminus B= \emptyset$, if $x \in A$, we must have $x \in B$. Because $B \setminus A= \emptyset$, if $x \in B$, then $x \in A$. But then $x \in A$ if and only if $x \in B$. Therefore, $A= B$. 

The converse is also true. Suppose that $A= B$. Then $A \setminus B= \emptyset$ and $B \setminus A= \emptyset$ (because $x \in A$ if and only if $x \in B$). Therefore, $A \Delta B= (A \setminus B) \cup (B \setminus A)= \emptyset \cup \emptyset= \emptyset$. Then $A \Delta B= \emptyset$ if and only if $A= B$. 

\begin{center} {\bfseries OR} \end{center}

{\noindent\itshape Lemma.} If $A$ and $B$ are sets, then $A \cap B^c= \emptyset$ if and only if $A= B$. 

{\noindent\itshape Proof.} Assume $A \cap B^c= \emptyset$ and suppose that $A \neq B$. Then there exists $a \in A$ such that $a \notin B$. Because $a \notin B$, we know that $a \in B^c$. But then $a \in A$ and $a \in B^c$ so that $a \in A \cap B^c= \emptyset$, a contradiction. Therefore, $A= B$. Now assume that $A= B$. But then $A \cap B^c= A \cap A^c= \emptyset$. 

We know use this lemma as follows (in the fourth if and only if):

	\[
	\begin{aligned}
	A \Delta B= \emptyset &\Longleftrightarrow (A \setminus B) \cup (B \setminus A)= \emptyset \\
	&\Longleftrightarrow (A \setminus B= \emptyset) \wedge (B \setminus A= \emptyset) \\
	&\Longleftrightarrow (A \cap B^c= \emptyset) \wedge (B \cap A^c= \emptyset) \\
	&\Longleftrightarrow (A= B) \wedge (B= A) \\
	&\Longleftrightarrow A= B
	\end{aligned}
	\]

\end{enumerate}





\newpage





% Problem 3
\problem{20} Let $A, B$ be sets with a common universal set $\mathscr{U}$. Prove the following:
	\begin{enumerate}[(a)]
	\item $A - (A - B)= A \cap B$
	\item $A \subseteq B$ if and only if $A^c \supseteq B^c$
	\end{enumerate} \pspace

\sol
\begin{enumerate}[(a)]
\item Let $x \in A - (A - B)$. Then $x \in A$ and $x \notin A - B$. By definition, $A - B= A \cap B^c$. Therefore, $x \notin A - B$ implies that $x \notin A \cap B^c$. But then $x \in (A \cap B^c)^c$. Now $(A \cap B^c)^c= A^c \cup B$ so that $x \in A^c \cup B$. Therefore, $x \in A^c$ or $x \in B$. But $x \in A$ so that $x \notin A^c$. Therefore, $x \in B$. But then $x \in A$ and $x \in B$ so that $x \in A \cap B$. This proves that $A - (A - B) \subseteq A \cap B$. 

Now assume that $x \in A \cap B$. This implies that $x \in A$ and $x \in B$. Suppose that $x \notin A - (A - B)$. From the work above, we know that $A - B= A \cap B^c$. But then $A - (A - B)= A - (A \cap B^c)$. By definition, $A - (A \cap B^c)$ is the set $A \cap (A \cap B^c)^c$. But $(A \cap B^c)^c= A^c \cup B$ so that $A \cap (A \cap B^c)^c= A \cap (A^c \cup B)$. Now the set $A \cap (A^c \cup B)$ is $(A \cap A^c) \cup (A \cap B)= \emptyset \cup (A \cap B)= A \cap B$. As $x \notin A - (A - B)$, this implies $x \notin A \cap B$, a contradiction. Therefore, $x \in A - (A - B)$. But then $A \cap B \subseteq A - (A - B)$. Therefore, $A - (A - B)= A \cap B$. 

\begin{center} {\bfseries OR} \end{center}

	\[
	\begin{aligned}
	x \in A \setminus (A \setminus B)&\Longleftrightarrow x \in A \setminus (A \cap B^c) \\
	&\Longleftrightarrow x \in \big( A \cap (A \cap B^c)^c \big) \\
	&\Longleftrightarrow x \in \big( A \cap (A^c \cup B) \big) \\
	&\Longleftrightarrow x \in \big( (A \cap A^c) \cup (A \cap B) \big) \\
	&\Longleftrightarrow x \in \big( \emptyset \cup (A \cap B) \big) \\
	&\Longleftrightarrow x \in A \cap B
	\end{aligned}
	\]

\item  Assume that $A \subseteq B$. We want to show that $B^c \subseteq A^c$. Let $x \in B^c$. Now $x \in B^c$ implies that $x \notin B$. Because $A \subseteq B$, it must be that $x \notin A$; otherwise, $x \in A$ and $x \notin B$, contradicting the fact that $A \subseteq B$. But then $x \in B^c$ implies that $x \in A^c$ so that $B^c \subseteq A^c$. 

Now assume that $A^c \supseteq B^c$. We want to show that $A \subseteq B$. Let $x \in A$. Because $x \in A$, we know that $x \notin A^c$. But if $x \notin A^c$, we know that $x \notin B^c$; otherwise, $x \notin A^c$ and $x \in B^c$ contradicts the fact that $A^c \supseteq B^c$. Therefore, if $x \in A$, then $x \in B$. But then $A \subseteq B$. 

\begin{center} {\bfseries OR} \end{center}

	\[
	\begin{aligned}
	A \subseteq B&\Longleftrightarrow (\forall x)( x \in A \Rightarrow x \in B) \\
	&\Longleftrightarrow (\forall x) \big( \neg(x \in B) \Rightarrow \neg(x \in A) \big) \\
	&\Longleftrightarrow (\forall x) ( x \notin B \Rightarrow x \notin A) \\
	&\Longleftrightarrow (\forall x) (x \in B^c \Rightarrow x \in A^c) \\
	&\Longleftrightarrow B^c \subseteq A^c
	\end{aligned}
	\]
\end{enumerate}





\newpage





% Problem 4
\problem{10} If $A \subseteq U$ and $B \subseteq V$, is $A \times B \subseteq U \times V$? Justify your answer. \pspace

\sol Yes. Suppose that $A \subseteq U$ and $B \subseteq V$. If either $A$ or $B$ are empty, then $A \times B$ is empty. Clearly, $\emptyset \subseteq U \times V$. So suppose that $A$ and $B$ are nonempty. Let $(x, y) \in A \times B$. Then by definition, $x \in A$ and $y \in B$. Because $A \subseteq U$ and $B \subseteq V$, this implies that $x \in U$ and $y \in V$, respectively. But then $(x, y) \in U \times V$. Therefore, $A \times B \subseteq U \times V$. 





\newpage





% Problem 5
\problem{10} Suppose that $X$ and $Y$ are sets with a common universal set $\mathscr{U}$. Show that $X= Y$ if and only if $(X \cap Y^c) \cup (X^c \cap Y)= \emptyset$. \pspace

\sol Suppose that $X= Y$. Then $Y^c= X^c$ so that $X \cap Y^c= X \cap X^c= \emptyset$. Similarly, $X^c= Y^c$ so that $X^c \cap Y= Y^c \cap Y= \emptyset$. But then $(X \cap Y^c) \cup (X^c \cap Y)= \emptyset \cup \emptyset= \emptyset$. 

Now assume that $(X \cap Y^c) \cup (X^c \cap Y)= \emptyset$. This implies that $X \cap Y^c= \emptyset$ and $X^c \cap Y= \emptyset$. But we already proved in Problem~2 (see the lemma below) that $X \cap Y^c= \emptyset$ implies that $X= Y$. Mutatis mutandis, $X^c \cap Y= \emptyset$ implies that $Y= X$. But then we know that $X= Y$. 

\vfill

{\noindent\itshape Lemma.} If $A$ and $B$ are sets, then $A \cap B^c= \emptyset$ if and only if $A= B$. 

{\noindent\itshape Proof.} Assume $A \cap B^c= \emptyset$ and suppose that $A \neq B$. Then there exists $a \in A$ such that $a \notin B$. Because $a \notin B$, we know that $a \in B^c$. But then $a \in A$ and $a \in B^c$ so that $a \in A \cap B^c= \emptyset$, a contradiction. Therefore, $A= B$. Now assume that $A= B$. But then $A \cap B^c= A \cap A^c= \emptyset$.





\newpage





% Problem 6
\problem{20} Prove or disprove:
	\begin{enumerate}[(a)]
	\item $(A \cup B) \setminus B= A$
	\item $A \cap (B \setminus C)= (A \cap B) \setminus (A \cap C)$
	\item $A \cap (B \setminus C)= (A \cap B) \setminus C$
	\item $A \setminus (B \cap C)= (A \setminus B) \cup (A \setminus C)$
	\end{enumerate} 

\sol
\begin{enumerate}[(a)]
\item The statement is \textit{false}. Let $A= \{ 1, 2, 3 \}$ and $B= \{ 3, 4 \}$. Then $A \cup B= \{ 1, 2, 3, 4 \}$ and $(A \cup B) \setminus B= \{ 1, 2, 3, 4 \} \setminus \{ 3, 4 \}= \{ 1, 2 \} \neq \{ 1, 2, 3 \}= A$. 

\item The statement is \textit{true}. Observe that\dots
	\[
	\begin{aligned}
	A \cap (B \setminus C)&= A \cap (B \cap C^c) \\
	&= (A \cap B) \cap C^c \\
	&= (A \cap B) \cap ( \emptyset \cap C^c) \\
	&= (A \cap B) \cap \big( (A \cap A^c) \cap C^c \big) \\
	&= (A \cap B) \cap \big( A \cap (A^c \cap C^c) \big) \\
	&= \big( (A \cap B) \cap A \big) \cap (A^c \cap C^c) \\
	&= \big( A \cap (A \cap B) \big) \cap (A^c \cap C^c) \\
	&= \big( (A \cap A) \cap B \big) \cap (A^c \cap C^c) \\
	&= (A \cap B) \cap (A^c \cup C^c)\\
	&= (A \cap B) \cap (A \cap C)^c \\
	&= (A \cap B) \setminus (A \cap C)
	\end{aligned}
	\]

\item The statement is \textit{true}. Observe that\dots
	\[
	A \cap (B \setminus C)= A \cap (B \cap C^c)= (A \cap B) \cap C^c= (A \cap B) \setminus C
	\]

\item The statement is \textit{true}. Observe that\dots
	\[
	A \setminus (B \cap C)= A \cap (B \cap C)^c= A \cap (B^c \cup C^c)= (A \cap B^c) \cup (A \cap C^c)= (A \setminus B) \cup (A \setminus C)
	\]
\end{enumerate}





\newpage





% Problem 7
\problem{20} Express the following sets as an interval, collection of intervals, or well known set (prove your answer): 
	\begin{enumerate}[(a)]
	\item $\ds\bigcap_{n \geq 1} \left[0, 1 + \frac{1}{n} \right)$
	\item $\ds\bigcup_{n \geq 1} \left[0, 1 + \frac{1}{n} \right)$
	\item $\ds\bigcup_{n \in \mathbb{Z}}\; \bigcap_{m \geq 1} \left( n - \frac{1}{m}, n + \frac{1}{m} \right)$
	\end{enumerate} \pspace

\sol
\begin{enumerate}[(a)]
\item We claim that\dots
	\[
	\bigcap_{n \geq 1} \left[0, 1 + \frac{1}{n} \right)= [0, 1]
	\]
If $x < 0$, then $x \notin [0, 2)= [0, 1 + 1/1)$ so that $x \notin \bigcap_{n \geq 1} \left[0, 1 + \frac{1}{n} \right)$. If $x \in [0, 1]$, then clearly $x \in [0, 1 + 1/n)$ for all $n \geq 1$, so that $x \in \bigcap_{n \geq 1} \left[0, 1 + \frac{1}{n} \right)$. Suppose that $x > 1$, i.e. $x - 1 > 0$. It is clear that $\frac{1}{x - 1} \in \mathbb{R}$ and $\frac{1}{x - 1} > 0$. Choose $n_0 \in \mathbb{N}$ such that $n_0 > \frac{1}{x - 1}$. But then $\frac{1}{n_0} < x - 1$ so that $1 + \frac{1}{n_0} < x$. Clearly, this implies that $x \notin [0, 1 + 1/n_0)$. But then $x \notin \bigcap_{n \geq 1} \left[0, 1 + \frac{1}{n} \right)$. Therefore, $x \in \bigcap_{n \geq 1} \left[0, 1 + \frac{1}{n} \right)$ if and only if $x \in [0, 1]$, as desired. 

\item We claim that\dots
	\[
	\bigcup_{n \geq 1} \left[0, 1 + \frac{1}{n} \right)= [0, 2)
	\]
Clearly, if $x \in [0, 2)= [0, 1+1/1)$, then $x \in \bigcup_{n \geq 1} \left[0, 1 + \frac{1}{n} \right)$. But if $x \in \bigcup_{n \geq 1} \left[0, 1 + \frac{1}{n} \right)$, then $x \in [0, 1 + 1/n_0)$ for some $n_0 \in \mathbb{N}$. But $n_0 \geq 1$ so that $1/n_0 \leq 1$. Then we have $x \in [0, 1 + 1/n_0) \subseteq [0, 1 + 1/1)= [0, 2)$. Therefore, $x \in \bigcup_{n \geq 1} \left[0, 1 + \frac{1}{n} \right)$ if and only if $x \in [0, 2)$, as desired. 

\item We claim that\dots
	\[
	\bigcup_{n \in \mathbb{Z}}\; \bigcap_{m \geq 1} \left( n - \frac{1}{m}, n + \frac{1}{m} \right)= \mathbb{Z}
	\]
Fix $N \in \mathbb{Z}$. Then $N \in (N - \frac{1}{m}, N + \frac{1}{m})$ for all $m \geq 1$. But then $N \in \bigcup_{n \in \mathbb{Z}}\; \bigcap_{m \geq 1} \left( n - \frac{1}{m}, n + \frac{1}{m} \right)$. 

Now suppose that $x \in \bigcup_{n \in \mathbb{Z}}\; \bigcap_{m \geq 1} \left( n - \frac{1}{m}, n + \frac{1}{m} \right)$ and that $x$ is not an integer. Because $x \in \bigcup_{n \in \mathbb{Z}}\; \bigcap_{m \geq 1} \left( n - \frac{1}{m}, n + \frac{1}{m} \right)$, there exists $N_0 \in \mathbb{Z}$ such that $x \in \bigcap_{m \geq 1} \left( N_0 - \frac{1}{m}, N_0 + \frac{1}{m} \right)$. We claim that this $N_0 \in \mathbb{Z}$ is unique. 

Suppose that $x \in \left(N - \frac{1}{m}, N + \frac{1}{m} \right)$ for some $N \in \mathbb{Z}, m \in \mathbb{N}$ with $N \neq N_0$. Either $N > N_0$ or $N < N_0$. Suppose that $N > N_0$. Because $N \in \mathbb{Z}$, we know that $N \geq N_0 + 1$. But as $x \in \bigcap_{m \geq 1} \left( N_0 - \frac{1}{m}, N_0 + \frac{1}{m} \right)$, $x \in (N_0 - 1/2, N_0 + 1/2)$. Therefore, $x < N_0 + 1/2$. But because $x \in \bigcap_{m \geq 1} \left( N - \frac{1}{m}, N + \frac{1}{m} \right)$, we know that $x \in (N - 1/2, N + 1/2)$. Therefore, $x > N - 1/2$. But then 
	\[
	x > N - \frac{1}{2} \geq N_0 + 1 - \frac{1}{2} = N_0 + \frac{1}{2},
	\]
a contradiction. Suppose then that $N < N_0$. Because $N \in \mathbb{Z}$, we know that $N \leq N_0 - 1$. But as $x \in \bigcap_{m \geq 1} \left( N_0 - \frac{1}{m}, N_0 + \frac{1}{m} \right)$, $x \in (N_0 - 1/2, N_0 + 1/2)$. Therefore, $N_0 - 1/2 < x$. But because $x \in \bigcap_{m \geq 1} \left( N - \frac{1}{m}, N + \frac{1}{m} \right)$, we know that $x \in (N - 1/2, N + 1/2)$. Therefore, $x < N + 1/2$. But then 
	\[
	x < N + \frac{1}{2} \leq N_0 - 1 + \frac{1}{2}= N_0 - \frac{1}{2},
	\]
a contradiction. 

Then there is a unique $N_0 \in \mathbb{Z}$ such that $x \in \bigcap_{m \geq 1} \left( N_0 - \frac{1}{m}, N_0 + \frac{1}{m} \right)$. Because $x \in \bigcap_{m \geq 1} \left( N_0 - \frac{1}{m}, N_0 + \frac{1}{m} \right)$, we know that $x \in (N_0 - 1/1, N_0 + 1/1)= (N_0 - 1, N_0 + 1)$. As $x$ is not an integer, we know that $x \neq N_0$. But then $|x - N_0| > 0$. We know also that $|x - N_0| \in \mathbb{R}$. Choose $m_0 \in \mathbb{N}$ such that $m_0 > \frac{1}{|x - N_0|}$. But then $\frac{1}{m_0} < | x - N_0|$. This implies that either $\frac{1}{m_0} < x - N_0$ or $\frac{1}{m_0} < -(x - N_0)$. If $\frac{1}{m_0} < x - N_0$, then $N_0 + \frac{1}{m_0} < x$, contradicting the fact that $x \in \bigcap_{m \geq 1} \left( N_0 - \frac{1}{m}, N_0 + \frac{1}{m} \right)$. If $\frac{1}{m_0} < -(x - N_0)$, then $-\frac{1}{m_0} > x - N_0$, so that $N_0 - \frac{1}{m_0} > x$, contradicting the fact that $x \in \bigcap_{m \geq 1} \left( N_0 - \frac{1}{m}, N_0 + \frac{1}{m} \right)$. Therefore, $x \notin \bigcap_{m \geq 1} \left( N_0 - \frac{1}{m}, N_0 + \frac{1}{m} \right)$. As this was the only $N \in \mathbb{Z}$ such that $x \in \bigcap_{m \geq 1} \left( N - \frac{1}{m}, N + \frac{1}{m} \right)$, it must be that $x \notin \bigcup_{n \in \mathbb{Z}}\; \bigcap_{m \geq 1} \left( n - \frac{1}{m}, n + \frac{1}{m} \right)$. But then $x \in \bigcup_{n \in \mathbb{Z}}\; \bigcap_{m \geq 1} \left( n - \frac{1}{m}, n + \frac{1}{m} \right)$ if and only if $x \in \mathbb{Z}$, as desired. 
\end{enumerate}


\end{document}