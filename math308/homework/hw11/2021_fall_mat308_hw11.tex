\documentclass[11pt,letterpaper]{article}
\usepackage[lmargin=1in,rmargin=1in,tmargin=1in,bmargin=1in]{geometry}
\usepackage{homework}

% -------------------
% Content
% -------------------
\begin{document}
\homework{}

% Problem 1
\problem{10} Show that the following sets have the same cardinality by finding a bijection between them (you need not prove that your function is bijective): 
\begin{enumerate}[(a)]
\item $A= (-2, 2)$, $B= (5, 6)$
\item $A= \{ 0, 1 \} \times \mathbb{N}$, $B= \mathbb{N}$
\item $A= [0, 1]$, $B= (0, 1)$
\end{enumerate}





\newpage





% Problem 2
\problem{10} Show that $\mathbb{N}$ and $\mathbb{Z} \times \mathbb{Z}$ have the same cardinality using the Schr\"oder-Cantor-Bernstein Theorem. 





\newpage





% Problem 3
\problem{10} In class, we discussed that given a set $S$, the cardinality of the power set of $S$, $\mathcal{P}(S)$, is strictly larger. Thus, we can construct sets with arbitrarily large cardinality. Prove this fact, i.e. prove the following:
\begin{enumerate}[(a)]
\item $|S| \leq |\mathcal{P}(S)|$, i.e. find an injection $f: S \to \mathcal{P}(S)$.
\item there does not exist a bijection $\phi: S \to \mathcal{P}(S)$. [Hint: Show there is not a surjection by considering the set $A:= \{ s \in S \colon s \notin \phi(s) \} \subseteq S$.]
\item Explain why this implies there can be no `set of all sets.' 
\end{enumerate}





\newpage





% Problem 4
\problem{10} Determine if the following sets are countable or uncountable (give a brief explanation; however, a formal proof is not necessary):
\begin{enumerate}[(a)]
\item $A= \{ \log n \colon n \in \mathbb{N} \}$.
\item $B=$ set of perfect squares. 
\item $C= \{ (m, n) \in \mathbb{N} \times \mathbb{N} \colon 2 \leq m \leq n^2 \}$.
\item $D=$ set of all irrational numbers. 
\item $E=$ set of linear functions $f: \mathbb{R} \to \mathbb{R}$.
\item $F=$ set of all finite binary strings. 
\item $G=$ set of {\itshape all} binary strings. 
\item $H=$ set of all functions $f: \{ 0 , 1 \} \to \mathbb{N}$. 
\item $I=$ set of all functions $f: \mathbb{N} \to \{ 0, 1 \}$. 
\item $J=$ set of all possible dictionary `words.' 
\item $K=$ set of all subsets of $\mathbb{N}$.
\end{enumerate}





\newpage





% Problem 5
\problem{10} Mimic Cantor's proof that the set $\mathbb{R}$ is uncountable to prove that the set of all real numbers without a 7 in their decimal expansion is uncountable. 






\end{document}