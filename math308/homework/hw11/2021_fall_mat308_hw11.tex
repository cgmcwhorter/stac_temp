\documentclass[11pt,letterpaper]{article}
\usepackage[lmargin=1in,rmargin=1in,tmargin=1in,bmargin=1in]{geometry}
\usepackage{homework}

% -------------------
% Content
% -------------------
\begin{document}
\homework{}

% Problem 1
\problem{10} Show that the following sets have the same cardinality by finding a bijection between them (you need not prove that your function is bijective): 
\begin{enumerate}[(a)]
\item $A= (-2, 2)$, $B= (5, 6)$
\item $A= \{ 0, 1 \} \times \mathbb{N}$, $B= \mathbb{N}$
\item $A= [0, 1]$, $B= (0, 1)$
\end{enumerate}





\newpage





% Problem 2
\problem{10} Show that $\mathbb{N}$ and $\mathbb{N} \times \mathbb{N}$ have the same cardinality using the Schr\"oder-Cantor-Bernstein Theorem.





\newpage





% Problem 3
\problem{10} We discussed in class that if $S$ is a set, then the cardinality of $\mathcal{P}(S)$ is strictly larger than the cardinality of $S$. Therefore, there is no largest cardinality because we can always construct sets with larger cardinality by using power sets. We shall now prove these facts. 
        \begin{enumerate}[(a)]
        \item If $S$ is a finite set, explain why we already know that $|\mathcal{P}(S)| > |S|$. 
        \item Show that $|S| \leq |\mathcal{P}(S)|$ by finding an injection $f: S \to \mathcal{P}(S)$. 
        \item Show that $|S| \neq |\mathcal{P}(S)|$ by showing that there is no bijection $\phi: S \to \mathcal{P}(S)$. [Hint: Show there is no such surjection by considering the set $A:= \{ s \in S \colon s \notin \phi(s) \} \subseteq S$.]
        \item Explain how the previous parts imply that there can be no `set of all sets.'        
        \end{enumerate}





\newpage





% Problem 4
\problem{10} Determine if the following sets are countable or uncountable (give a brief explanation; however, a formal proof is not necessary):
\begin{enumerate}[(a)]
\item $A= \{ \log n \colon n \in \mathbb{N} \}$.
\item $B=$ set of perfect squares. 
\item $C= \{ (m, n) \in \mathbb{N} \times \mathbb{N} \colon 2 \leq m \leq n^2 \}$.
\item $D=$ set of all irrational numbers. 
\item $E=$ set of linear functions $f: \mathbb{R} \to \mathbb{R}$.
\item $F=$ set of all finite binary strings. 
\item $G=$ set of {\itshape all} binary strings. 
\item $H=$ set of all functions $f: \{ 0 , 1 \} \to \mathbb{N}$. 
\item $I=$ set of all functions $f: \mathbb{N} \to \{ 0, 1 \}$. 
\item $J=$ set of all possible dictionary `words.' 
\item $K=$ set of all subsets of $\mathbb{N}$.
\end{enumerate}





\newpage





% Problem 5
\problem{10} Mimic Cantor's proof that the set $\mathbb{R}$ is uncountable to prove that the set of all real numbers without a 7 in their decimal expansion is uncountable. 






\end{document}