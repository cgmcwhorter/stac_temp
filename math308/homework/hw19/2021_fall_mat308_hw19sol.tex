\documentclass[11pt,letterpaper]{article}
\usepackage[lmargin=1in,rmargin=1in,tmargin=1in,bmargin=1in]{geometry}
\usepackage{homework}

% -------------------
% Content
% -------------------
\begin{document}
\homework{\textit{Caleb McWhorter --- Solutions}}

% Problem 1
\problem{10} Consider the following graph:
	\[
	\graphOne
	\]

\begin{enumerate}[(a)]
\item Is the graph directed or undirected?
\item Give the vertex set and the edge set for the graph.
\item Give the adjacency matrix of the graph.
\item Is the graph simple?
\item Are there any isolated vertices?
\item List all pairs of parallel edges.
\item Compute the degree of the graph.  
\item Are the vertices $v_1$ and $v_6$ connected? What about the vertices $v_1$ and $v_4$? 
\item Does the graph $G \setminus \{ v_6 \}$ have an Eulerian circuit? Find one or explain why none exists.  
\item Does the graph $G \setminus \{ v_6 \}$ have an Hamiltonian circuit? Find one or explain why none exists.  
\end{enumerate} \pspace

\sol
\begin{enumerate}[(a)]
\item This graph is undirected. 

\item We have\dots
	\[
	\begin{aligned}
	V(G)&= \{ v_1, v_2, v_3, v_4, v_5, v_6 \} \\[0.3cm]
	E(G)&= \{ e_1, e_2, e_3, e_4, e_5, e_6, e_7, e_8 \} \\
	&= \{ \{v_1, v_3\}, \{v_1, v_2\}, \{v_2, v_3\}, \{v_2, v_3\}, \{v_2, v_4\}, \{v_3, v_4\}, \{v_3, v_5\}, \{v_4, v_5\} \}
	\end{aligned}
	\]

\item 
	\[
	A= 
	\begin{pmatrix}
	0 & 1 & 1 & 0 & 0 & 0 \\
	1 & 0 & 2 & 1 & 0 & 0 \\
	1 & 2 & 0 & 1 & 1 & 0 \\
	0 & 1 & 1 & 0 & 1 & 0 \\
	0 & 0 & 1 & 1 & 0 & 0 \\
	0 & 0 & 0 & 0 & 0 & 0 
	\end{pmatrix}
	\]

\item There are multiple edges as there are two edges between $v_2$ and $v_3$. Therefore, the graph is not simple. 

\item The vertex $v_6$ is isolated. 

\item The only pair of parallel edges are $e_3$ and $e_4$. 

\item For each $i$, we compute the degree of vertex $v_i$, denoted $\deg_i$. 
	\[
	\begin{aligned}
	v_1&= 2 & v_4&= 3 \\
	v_2&= 4 & v_5&= 2 \\
	v_3&= 5 & v_6&= 0
	\end{aligned}
	\]
Then the degree of $G$ is\dots
	\[
	\deg G= \sum \deg_i= 2 + 4 + 5 + 3 + 2 + 0= 16
	\]

\item There is no walk from $v_1$ to $v_6$; therefore, $v_1$ and $v_6$ are not connected. However, there is a walk from $v_1$ to $v_4$, e.g. $e_2, e_5$ or $e_1, e_3, e_2, e_1, e_7, e_8$; therefore, $v_1$ and $v_4$ are connected. 

\item Recall that if a graph $G$ is connected and the degree of every vertex of $G$ is a positive even integer, then $G$ has an Euler circuit. Therefore, if $G$ does not have an Eulerian circuit then either $G$ is not connected or not every vertex of $G$ has positive even degree.  The graph $G \setminus \{ v_6 \}$ is connected but not every vertex has positive even degree. For instance, the vertex $v_4$ has degree 3. Therefore, $G$ does not have an Eulerian circuit. 

\item The graph $G \setminus \{ v_6 \}$ has a Hamiltonian circuit. For instance, beginning at $v_1$, the walk $e_1, e_7, e_8, e_5, e_2$ forms a Hamiltonian circuit. 
\end{enumerate}





\newpage





% Problem 2
\problem{10} Give vertex and edge set for a graph (directed) and tell if....
	\[
	\graphTwo
	\]

\begin{enumerate}[(a)]
\item Is the graph directed or undirected?
\item Give the vertex set and the edge set for the graph.
\item Give the adjacency matrix of the graph.
\item Is the graph simple?
\item Are there any isolated vertices?
\item Compute the in and out degree of each vertex. 
\item Compute the degree of the graph.
\item Are the vertices $v_1$ and $v_6$ connected? What about the vertices $v_6$ and $v_1$? 
\item Does the graph have an Eulerian circuit? Find one or explain why none exists.  
\item Does the graph have an Hamiltonian circuit? Find one or explain why none exists.  
\end{enumerate} \pspace

\sol
\begin{enumerate}[(a)]
\item This graph is directed. 

\item We have\dots
	\[
	\begin{aligned}
	V(G)&= \{ v_1, v_2, v_3, v_4, v_5, v_6 \} \\[0.3cm]
	E(G)&= \{ e_1, e_2, e_3, e_4, e_5, e_6, e_7, e_8 \} \\
	&= \{ (v_1, v_2), (v_3, v_2), (v_4, v_1), (v_2, v_4), (v_2, v_5), (v_3, v_5), (v_4, v_6), (v_5, v_8) \}
	\end{aligned}
	\]

\item 
	\[
	A=
	\begin{pmatrix}
	0 & 1 & 0 & 0 & 0 & 0 \\
	0 & 0 & 0 & 1 & 1 & 0 \\
	0 & 1 & 0 & 0 & 1 & 0 \\
	1 & 0 & 0 & 0 & 0 & 1 \\
	0 & 0 & 0 & 0 & 0 & 1 \\
	0 & 0 & 0 & 0 & 0 & 0 \\
	\end{pmatrix}
	\]

\item There are no multiple edges or loops in $G$. Therefore, $G$ is a simple graph. 

\item 
\item 
\item 
\item 
\item 
\item 
\end{enumerate}





\newpage





% Problem 3
\problem{10} Draw a graph that has the adjacency matrix given below. How can you tell from this matrix if the graph is undirected or directed? 
	\[
	\begin{pmatrix}
	1 & 1 & 1 & 0 & 0 & 0 \\
	1 & 0 & 1 & 1 & 0 & 0 \\
	1 & 1 & 0 & 1 & 1 & 0 \\
	0 & 1 & 1 & 0 & 1 & 0 \\
	0 & 0 & 1 & 1 & 0 & 2 \\
	0 & 0 & 0 & 0 & 2 & 0 \\
	\end{pmatrix}
	\] \pspace





\newpage





% Problem 4
\problem{10} Determine which of the following graphs are isomorphic. In each case, explain the isomorphism or explains why an isomorphism cannot exist. \pspace
	\[
	\graphThree \quad \graphFour \qquad\qquad \graphFive
	\]





\newpage





% Problem 5
\problem{10} Given an undirected graph $G$, the \textit{degree sequence} of $G$ is a monotonic non-increasing sequence of the vertex degrees of $G$. For each of the following degree sequences for a simple graph $G$, either give an example of a graph $G$ with the given degree sequence or prove that one cannot exist.
	\begin{enumerate}[(a)]
	\item $(4, 3, 2, 2, 2)$
	\item $(2, 2, 2, 0)$
	\item $(5, 4, 3, 2, 1, 0)$ 
	\item $(4, 3, 3, 3, 3)$
	\item $(4, 3, 2, 1)$
	\end{enumerate} \pspace





\newpage





% Problem 6
\problem{10} Prove that every nontrivial simple graph has two vertices of the same degree. [Hint: Pigeonhole Principle] \pspace





\newpage





% Problem 7
\problem{10} Prove that in a tree, $T$, every distinct pair of vertices $u$ and $v$ has a unique path connecting them. \pspace





\newpage





% Problem 8
\problem{10} Suppose the ground floor plan of a building is given below. Is it possible to walk through every door on the first floor exactly once, ending up in your starting room? Explain. Is it possible to visit every room exactly once, ending up in your starting room? Explain. \pspace 
	\[
	\graphSix
	\] \pspace





\newpage





% Problem 9
\problem{10} Define the following matrices:
	\[
	A= 
	\begin{pmatrix}
	1 & 0 & -2 \\
	3 & 1 & 5 
	\end{pmatrix}, \quad
	B= 
	\begin{pmatrix}
	3 & 1 & -1 \\
	0 & -2 & 4
	\end{pmatrix}, \quad
	C=
	\begin{pmatrix}
	1 & -1 \\
	0 & 3 \\
	-2 & 1
	\end{pmatrix}, \quad
	v= 
	\begin{pmatrix}
	3 \\
	0 \\
	-1
	\end{pmatrix}
	\]
For each of the following operations, either compute the given expression or explain why it is undefined. 

\begin{enumerate}[(a)]
\item $2A - B$
\item $AB$
\item $A + C$
\item $Av$
\item $AC$
\item $CA$
\end{enumerate} \pspace

\sol
\begin{enumerate}[(a)]
\item 
\item 
\item 
\item 
\item 
\item  
\end{enumerate}





\newpage





% Problem 10
\problem{10} Consider the graph $G$ given below: 
	\[
	\graphSeven
	\] 

\begin{enumerate}[(a)]
\item Compute the adjacency matrix, $A$.
\item Using a computer system, compute $A$, $A^2$, $A^3$, and $A^4$. 
\item Compute the number of walks from $v_1$ to $v_3$ of lengths one, two, three, and four, respectively. 
\item Compare your answers from (b) and (c). Make a conjecture on what the $a_{ij}$ entry of $A^k$ represents. 
\end{enumerate} \pspace

\sol
\begin{enumerate}[(a)]
\item 
\item 
\item 
\item 
\end{enumerate}





\end{document}