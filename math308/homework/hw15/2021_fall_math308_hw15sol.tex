\documentclass[11pt,letterpaper]{article}
\usepackage[lmargin=1in,rmargin=1in,tmargin=1in,bmargin=1in]{geometry}
\usepackage{homework}

% -------------------
% Content
% -------------------
\begin{document}
\homework{Solutions --- Caleb McWhorter}

% Problem 1
\problem{10} Perform the following computations modulo 3:
\begin{enumerate}[(a)]
\item $1234 + 2345$
\item $1784 \cdot 2021$ 
\item $1996^{1997}$
\item $2^{2000}$
\end{enumerate} \pspace

\sol Let $a, b \in \mathbb{Z}$ and $n \in \mathbb{N}$. Recall that $a \equiv b \mod n$ if and only if $a - b$ is divisibly by $n$. This happens if and only if $kn= a - b$ for some $k \in \mathbb{Z}$. But then $a= kn + b$. But then given $a \in \mathbb{Z}$, we can choose a $b \in \{ 0, 1, \ldots, n - 1 \}$ such that $a \equiv b \mod n$. Use the division algorithm to find $q \in \mathbb{Z}$ and $r \in \{ 0, 1, \ldots, n - 1 \}$ such that $a= qn + r$. But then $a - r= qn$ so that $a - r$ is divisible by $n$. Therefore, $a \equiv r \mod n$. That is, every integer is equivalent to its remainder from the division algorithm modulo $n$. 

\begin{enumerate}[(a)]
\item Because we have $1234= 411(3) + 1$ and $2345= 781(3) + 2$, we know that $1234 \equiv 1 \mod 3$ and $2345 \equiv 2 \mod 3$. But then\dots
	\[
	(1234 + 2345) \equiv (1 + 2)= 3 = \big(3(1) + 0 \big) \equiv 0 \mod 3
	\]
Alternatively, we have $1234 + 2345= 3579$ and $3579= 1193(3) + 0$. But then $(1234 + 2345)= 3579 \equiv 0 \mod 3$. \pspace

\item We have $1784= 594(3) + 2$ and $2021= 673(3) + 2$, so that $1784 \equiv 2 \mod 3$ and $2021 \equiv 2 \mod 3$. But then\dots
	\[
	1784 \cdot 2021 \equiv 2 \cdot 2 = 4 = 1(3) + 1 \equiv 1 \mod 3
	\]
Alternatively, we have $1784 \cdot 2021= 3,\!605,\!464$ and $3,\!605,\!464= 1201821(3) + 1$. Therefore, we have\dots
	\[
	1784 \cdot 2021= 3,\!605,\!464 = \big(1201821(3) + 1 \big) \equiv 1 \mod 3
	\] \pspace

\item Observe that we have $1996= 665(3) + 1$ so that $1996 \equiv 1 \mod 3$. But then\dots
	\[
	1996^{1997} \equiv 1^{1997} = 1 \mod 3
	\] \pspace

\item Observe that we have $2= 3 - 1$. But then $2 \equiv -1 \mod 3$ because $2 + 1= 3$. But then\dots
	\[
	2^{2000} \equiv (-1)^{2000}= 1 \mod 3
	\]
\end{enumerate}



\newpage



% Problem 2
\problem{10} Prove that an integer $N$ is divisible by 3 if and only if its the sum of its digits is divisible by 3. \pspace 

\sol Let $N$ be an integer. Express $N$ in base-10 as $a_na_{n-1}\cdots a_1a_0$, i.e. write $N= 10^n a_n + 10^{n-1} a_{n-1} + \cdots + 10a_1 + 1a_0$. We know an integer $k$ is divisible by a positive integer $n$ if and only if $k \equiv 0 \bmod n$. Therefore, it suffices to prove that $N \equiv 0 \bmod 3$ if and only if the sum of its digits is divisible by 3. We have $10= 3(3) + 1$ so that $10 \equiv 1 \bmod 3$. But then\dots
	\[
	N= 10^n a_n + 10^{n-1} a_{n-1} + \cdots + 10a_1 + 1a_0 \equiv 1^n a_n + 1^{n-1} a_{n-1} + \cdots + 1a_1 + 1a_0 \equiv a_n + a_{n-1} + \cdots + a_1 + a_0
	\]
Therefore, $N \equiv 0 \mod 3$ if and only if $a_n + a_{n-1} + \cdots + a_1 + a_0$; that is, $N$ is divisible by $3$ if and only if the sum of its digits is divisible by $3$. 



\newpage



% Problem 3
\problem{10} Prove that for all $n, m \in \mathbb{Z}_{\geq 0}$ that $101^n - 77^m$ is divisible by 4. \pspace

\sol We know an integer $k$ is divisible by a positive integer $n$ if and only if $k \equiv 0 \bmod n$. It then suffices to prove that $101^n - 77^m \equiv 0 \bmod 4$ for all $n, m \in \mathbb{Z}_{\geq 0}$. Observe that $101= 25(4) + 1$ and $77= 19(4) + 1$, so that $101 \equiv 1 \bmod 4$ and $77 \equiv 1 \bmod 4$. But then\dots
	\[
	101^n - 77^m \equiv 1^n - 1^m = 1 - 1= 0 \mod 4
	\]
Therefore, $101^n - 77^m$ is divisible by $4$ for all $n, m \in \mathbb{Z}_{\geq 0}$.\footnote{This theory of modularity that we have developed only works for \textit{integers}. The condition that $n, m \in \mathbb{Z}_{\geq 0}$ is to that $101^n$ and $77^m$ are integers, respectively.} \pspace



\newpage



% Problem 4
\problem{10} Find the ones digit of $2^{98}$ and the tens digit of $7^{100}$. \pspace

\sol Let $N$ be an integer. Express $N$ in base-10 as $a_na_{n-1}\cdots a_1a_0$, i.e. write $N= 10^n a_n + 10^{n-1} a_{n-1} + \cdots + 10a_1 + 1a_0$. We know the ones digit of $N$ is $a_0$. But observe $N= 10^n a_n + 10^{n-1} a_{n-1} + \cdots + 10a_1 + 1a_0 \equiv 0^n a_n + 0^{n-1} a_{n-1} + \cdots + 0a_1 + 1a_0 \equiv a_0 \bmod 10$. But then the ones digit of $N$ is the value of $N \bmod 10$. Similarly, we know for $n \geq 2$, $10^n \equiv 0 \bmod 100$. But then\dots
	\[
	N= 10^n a_n + 10^{n-1} a_{n-1} + \cdots + 10^2a_2 + 10a_1 + 1a_0 \equiv 0^n a_n + 0^{n-1} a_{n-1} + \cdots + 0^2a_2 + 10a_1 + 1a_0= [a_1a_0] \mod 100
	\]
where $[a_1a_0]$ represents the base-10 number with digits $a_1, a_0$. We know the tens digit of $N$ is $a_1$, which can be easily determined from the value of $N \bmod 100$. \pspace

Now observe\dots
	\[
	\begin{aligned}
	&2^1= 2 \equiv 2 \mod 10 \qquad& &2^4= 2^3 \cdot 2 \equiv 8 \cdot 2= 16 \equiv 6 \mod 10 \\
	&2^2= 4 \equiv 4 \mod 10 		   & &2^5= 2^4 \cdot 2 \equiv 6 \cdot 2= 12 \equiv 2 \mod 10 \\
	&2^3= 8 \equiv 8 \mod 10 
	\end{aligned}
	\]
From the work above, it is clear that the value of $2^k \bmod 10$ is cyclic with values $2, 4, 8, 6, 2, 4, 8, 6, \ldots$ beginning at $k= 1$. Therefore, the value of $2^k$ only depends on the value of $98$ modulo $4$ (the length of the repeating cycle). We have\dots
	\[
	2^{98}= 2^{24(4) + 2}= 2^{24(4)} \cdot 2^2= (2^4)^{24} \cdot 2^2= (2^4)^{6 \cdot 4} \cdot 2^2 \equiv 6 \cdot 4= 24 \equiv 4 \mod 10
	\]
Alternatively, observe $2^1= 2 \bmod 10$, $2^2= 4 \bmod 10$, $2^4= (2^2)^2 \equiv 4^2= 16 \equiv 6 \bmod 10$, $2^8= (2^4)^2 \equiv 6^2= 36 \equiv 6 \bmod 10$, $2^{16}= (2^8)^2 \equiv 6^2= 36 \equiv 6 \bmod 10$, $2^{32}= (2^{16})^2 \equiv 6^2= 36 \equiv 6 \bmod 10$, and $2^{64}= (2^{32})^2 \equiv 6^2= 36 \equiv 6 \bmod 10$. But then\dots
	\[
	2^{98}= 2^{64 + 32 + 2}= 2^{64} \cdot 2^{32} \cdot 2^2 \equiv 6 \cdot 6 \cdot 4= 144 \equiv 4 \mod 10
	\]
Therefore, the ones digit of $2^{98}$ is $4$. \pspace

Now observe\dots
	\[
	\begin{aligned}
	&7^1= 7 \equiv 7 \mod 100 \\
	&7^2= 49 \equiv 49 \mod 100 \\
	&7^3= 343= 3(100) + 43 \equiv 43 \mod 100 \\
	&7^4= 7^3 \cdot 7 \equiv 43 \cdot 7= 301= 3(100) + 1 \equiv 1 \mod 100
	\end{aligned}
	\]
From the work above, it is clear that $7^{4k} \equiv 1 \bmod 100$ for all integers $k \geq 0$. But $100= 25(4)$ so that\dots
	\[
	7^{100} = 7^{25(4)} = (7^4)^{25} = 1^{25}= 1 = [01] \mod 100
	\]
where $[01]$ is the base-10 integer with the given digits. Therefore, the tens digit of $7^{100}$ is $0$. 



\newpage



% Problem 5
\problem{10} For the following congruences, find a solution or explains why none exists.
\begin{enumerate}[(a)]
\item $2x \equiv 3 \mod 7$
\item $6x \equiv 5 \mod 8$
\item $4x \equiv 8 \mod 22$
\end{enumerate} \pspace

\sol Let $a, b, x$ be integers, $n$ be a positive integer, and $d= \gcd(a, n)$. Recall that a linear congruence $ax \equiv b \bmod n$ has a solution if and only if $d$ divides $b$. If $d \mid b$, there are infinitely many solutions and they are all of the form $\frac{sb}{d} + \frac{n}{d}\,z$, where $z \in \mathbb{Z}$ and $s$ is such that for some $y$, $d= sx + ny$. When $d= 1$, we can express this simply using the inverse: the solutions modulo $n$ are $x \equiv a^{-1}b$, where $a^{-1}$ is the inverse of $a$ modulo $n$ (which exists because $\gcd(a, n)= 1$). Let $s$ be the integer $1 \leq s \leq n$ such that $s \equiv a^{-1}b$ modulo $n$. Then general solutions are $x= s + zn$, where $z \in \mathbb{Z}$. 

\begin{enumerate}[(a)]
\item Observe that $\gcd(2,7)= 1$ and $3$ is divisible by $1$. Therefore, there is a solution. We can use inverses to solve this congruence. Observe that $2 \cdot 4 = 8 \equiv 1 \bmod 7$. Therefore, $2^{-1} = 4 \bmod 7$. Then we have\dots
	\[
	\begin{gathered}
	2x \equiv 3 \mod 7 \\
	2^{-1} \cdot 2x \equiv 2^{-1} \cdot 3 \mod 7 \\
	(2^{-1}2)x \equiv 4 \cdot 3 \mod 7 \\
	1x \equiv 12 \mod 7 \\
	x \equiv 5 \mod 7
	\end{gathered}
	\]
Therefore, the solutions are the integers equivalent to $5$ modulo $7$, i.e. $\ldots, -9, -2, 5, 12, 19, \ldots$. \pspace

\item Observe that $\gcd(6, 8)= 2$ and $5$ is not divisible by $2$. Therefore, there are no solutions to this congruence. One can verify this by checking all the possible solutions modulo $8$
	\[
	\begin{aligned}
	&x \equiv 0: 6(0)= 0 \not\equiv 5 \mod 8 \hspace{2cm}& &x \equiv 4: 6(4)= 24 \equiv 0 \not\equiv 5 \mod 8 \\
	&x \equiv 1: 6(1)= 6 \not\equiv 5 \mod 8 & &x \equiv 5: 6(5)= 30 \equiv 6 \not\equiv 5 \mod 8 \\
	&x \equiv 2: 6(2)= 12 \equiv 4 \not\equiv 5 \mod 8 & &x \equiv 6: 6(6)= 36 \equiv 4 \not\equiv 5 \mod 8 \\
	&x \equiv 3: 6(3)= 18 \equiv 2 \not\equiv 5 \mod 8 & &x \equiv 7: 6(7)= 42 \equiv 2 \not\equiv 5 \mod 8
	\end{aligned}
	\] \pspace

\item Observe that $\gcd(4, 22)= 2$ and $8$ is divisible by $2$. Therefore, there is a solution (infinitely many in fact). However, because $\gcd(4, 22) \neq 1$, we know that $4^{-1}$ does not exist. Therefore, inverses cannot be used to solve this congruence. But using the comments above, we know the solutions have the form $\frac{xb}{d} + \frac{n}{d}\,z$, where $z \in \mathbb{Z}$ and $x$ is such that for some $y$, $d= ax + ny$. Using the Euclidean algorithm, we write $2= 4(-5) + 22(1)$. So we know $x= -5$. Therefore, the solutions are the integers of the form\dots
	\[
	\dfrac{xb}{d}+ \dfrac{n}{d} \, z = \dfrac{-5(8)}{2} + \dfrac{22}{2} \, z = -20 + 11z = 11z - 20
	\]
where $z$ is an integer. For example, $-31, -20, -9, 2, 13, 24$ are solutions resulting from the choices $z= -1, 0, 1, 2, 3, 4$, respectively. We can also see this directly:
	\[
	4x \equiv 4(11z - 20)= 44z - 80= 2(22z) + (-4 \cdot 22 + 8) \equiv 0 + (0 + 8) = 8 \mod 22
	\]
\end{enumerate}



\newpage



% Problem 6
\problem{10} Use the Chinese Remainder Theorem to find the solutions modulo 60 to\dots
	\[
	\begin{aligned}
	x &\equiv 3 \mod 4 \\
	x&\equiv 2 \mod 3 \\
	x&\equiv 4 \mod 5
	\end{aligned}
	\] \pspace

\sol Suppose we have a system of congruences $x \equiv a_1 \bmod n_1$, $x \equiv a_2 \bmod n_2$, \dots, $x \equiv a_k \bmod n_k$. The Chinese Remainder Theorem (also known as Sunzi's Theorem), states that if the $n_i$ are pairwise coprime, i.e. $\gcd(n_i, n_j)= 1$ for all $i, j$ with $i \neq j$, there is a solution and this solution is unique modulo $N= n_1n_2 \cdots n_k$. Furthermore, the theorem states that the solution is $x= \sum_i a_i M_i N_i$, where $N_i= \frac{N}{n_i}$ and $M_i= N_i^{-1} \bmod n_i$. \pspace

The given system of congruences has a solution because $\gcd(4, 3, 5)= 1$. We have $a_1= 3$, $a_2= 2$, and $a_3= 4$. Now we have\dots
	\[
	\begin{aligned}
	N&= n_1n_2n_3= 4 \cdot 3 \cdot 5= 60 \\
	N_1&= \dfrac{N}{n_1}= \dfrac{60}{4}= 15 \\ 
	N_2&= \dfrac{N}{n_2}= \dfrac{60}{3}= 20 \\
	N_3&= \dfrac{N}{n_3}= \dfrac{60}{5}= 12 
	\end{aligned}
	\]
Now we need find the inverses of $15, 20, 12$ with respect to $4, 3, 5$, respectively. First, we reduce these:
	\[
	\begin{aligned}
	N_1&= 15= 3(4) + 3 \equiv 3 \mod 4 \\
	N_2&= 20= 6(3) + 2 \equiv 2 \mod 3 \\
	N_3&= 12= 2(5) + 2 \equiv 2 \mod 5
	\end{aligned}
	\]
Now observe that $3(3)= 9= 2(4) + 1 \equiv 1 \bmod 4$ so that $3^{-1}= 3 \bmod 4$. Furthermore, observe that $2(2)= 4= 3(1) + 1 \equiv 1 \bmod 3$ so that $2^{-1}= 2 \bmod 3$. Finally, observe that $3(2)= 6= 1(5) + 1 \equiv 1 \bmod 5$ so that $2^{-1}= 3 \bmod 5$. Therefore, we have\dots
	\[
	\begin{aligned}
	M_1&= 3 \\
	M_2&= 2 \\
	M_3&= 3
	\end{aligned}
	\]
But then a solution the given congruence is\dots
	\[
	x= \sum_i a_i M_i N_i= a_1 M_1 N_1 + a_2 M_2 N_2 + a_3 M_3 N_3= 3(3)15 + 2(2)20 + 4(3)12= 135 + 80+ 144= 359
	\]
Reducing $x= 359$ modulo $N= 60$, we have $x= 359= 5(60) + 59 \equiv 59 \bmod 60$. Therefore, the solutions to this system of congruences are $x \equiv 59 \bmod 60$, i.e. the integers of the form $60k + 59$. For example, choosing $k= -2, -1, 0, 1, 2$, we obtain solutions $-85, -25, 35, 95, 155$, respectively. We can also verify these solutions:
	\[
	\begin{aligned}
	x= 60k + 59= 15(4)k + \big( 14(4) + 3 \big) \equiv 0 + 3= 3 \equiv 3 \mod 4 \\
	x= 60k + 59= 20(3)k + \big( 19(3) + 2 \big) \equiv 0 + 2= 2 \equiv 2 \mod 3 \\
	x= 60k + 59= 12(5)k + \big( 11(5) + 4 \big) \equiv 0 + 4= 4 \equiv 4 \mod 5
	\end{aligned}
	\]


\end{document}