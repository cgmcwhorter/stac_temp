\documentclass[11pt,letterpaper]{article}
\usepackage[lmargin=1in,rmargin=1in,bmargin=1in,tmargin=1in]{geometry}
\usepackage{style/quiz}
\usepackage{style/commands}

% -------------------
% Content
% -------------------
\begin{document}
\thispagestyle{title}

% Quiz 1
\quizsol \textit{True/False}: The following is a truth table for $P \to Q$:
	\begin{table}[!ht]
	\centering
	\begin{tabular}{c|c||c}
	$P$ & $Q$ & $P \to Q$ \\ \hline
	T & T& T \\
	T & F & F \\
	F & T & F \\
	F & F & F
	\end{tabular}
	\end{table}

\sol The statement is \textit{false}. The correct truth table should be\dots
	\begin{table}[!ht]
	\centering
	\begin{tabular}{c|c||c}
	$P$ & $Q$ & $P \to Q$ \\ \hline
	T & T& T \\
	T & F & F \\
	F & T & T \\
	F & F & T
	\end{tabular}
	\end{table} \par
One way to think about this is as follows: imagine $P$ is a guarantee. Namely, we promise that if $P$ happens, $Q$ must happen. For instance, $P$ could represent the statement, ``You do not tamper with your hardware,'' and $Q$ could be the statement, ``I will replace your broken computer.'' So $P \to Q$ is then the statement, ``If you do not tamper with your hardware, then I will replace your broken computer.'' If both $P$ and $Q$ are true, then this should be true---because I promised to replace the computer if you left it alone. If $P$ is true and $Q$ is false, then the statement should be false because I broke my promise. However, my promise holds true whenever $P$ is false. Why? Because you broke our agreement by tampering with the hardware. So while I may or may not replace the computer, my promise has not been broken in either case, i.e. it remains true. In an implication $P \to Q$, if $P$ is false, then the statement $P \to Q$ is \textit{always} true. \pvspace{1.5cm}



% Quiz 2
\quizsol \textit{True/False}: $\forall x, \exists y, \, x^2 + y= 4$ \pspace

\sol The statement is \textit{true}. The statement says that for all $x$ there is a $y$ such that $x^2 + y= 4$. If this is true (which it is), we need to prove it. Fix an $x$, say $x_0$. We need to find a $y$ such that $x_0^2 + y = 4$. Define $y_0:= 4 - x_0^2$. But then we have 
	\[
	x_0^2 + y_0= x_0^2 + (4 - x_0^2)= 4,
	\]
as desired. \pvspace{1.5cm}



% Quiz 3
\quizsol \textit{True/False}: $\neg \left( \forall x, \exists y, P(x,y) \vee \neg Q(x,y) \right)= \exists x, \forall y, \neg P(x,y) \wedge Q(x,y)$ \pspace

\sol The statement is \textit{true}. We can simply compute the negation step-by-step:
	\[
	\begin{aligned}
	\neg \left( \forall x, \exists y, P(x,y) \vee \neg Q(x,y) \right)&\equiv \exists x, \neg( \exists y, P(x,y) \vee \neg Q(x,y) ) \\
	&\equiv \exists x, \forall y, \neg ( P(x,y) \vee \neg Q(x,y) ) \\
	&\equiv \exists x, \forall y, \neg P(x,y) \wedge \neg (\neg Q(x,y)) \\
	&\equiv \exists x, \forall y, \neg P(x,y) \wedge Q(x,y)
	\end{aligned}
	\]



% Quiz 4
\quizsol \textit{True/False}: To prove $P \Rightarrow Q$, you can prove $Q \Rightarrow P$. \pspace

\sol The statement is \textit{false}. The converse of $P \Rightarrow Q$ is $Q \Rightarrow P$. The converse of a logical statement is not necessarily logically equivalent to the original statement. So proving the converse does not necessarily prove the original statement. However, the contrapositive of $P \Rightarrow Q$, which is $\neg Q \Rightarrow \neg P$, is logically equivalent to $P \Rightarrow Q$. Therefore, to prove $P \Rightarrow Q$, one only need prove $\neg Q \Rightarrow \neg P$. This is called proof by contrapositive. \pvspace{1.5cm}
















\end{document}