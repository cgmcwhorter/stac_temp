\documentclass[11pt,letterpaper]{article}
\usepackage[lmargin=1in,rmargin=1in,bmargin=1in,tmargin=1in]{geometry}
\usepackage{style/quiz}
\usepackage{style/commands}

% -------------------
% Content
% -------------------
\begin{document}
\thispagestyle{title}

% Quiz 1
\quizsol \textit{True/False}: The following is a truth table for $P \to Q$:
	\begin{table}[!ht]
	\centering
	\begin{tabular}{c|c||c}
	$P$ & $Q$ & $P \to Q$ \\ \hline
	T & T& T \\
	T & F & F \\
	F & T & F \\
	F & F & F
	\end{tabular}
	\end{table}

\sol The statement is \textit{false}. The correct truth table should be\dots
	\begin{table}[!ht]
	\centering
	\begin{tabular}{c|c||c}
	$P$ & $Q$ & $P \to Q$ \\ \hline
	T & T& T \\
	T & F & F \\
	F & T & T \\
	F & F & T
	\end{tabular}
	\end{table} \par
One way to think about this is as follows: imagine $P$ is a guarantee. Namely, we promise that if $P$ happens, $Q$ must happen. For instance, $P$ could represent the statement, ``You do not tamper with your hardware,'' and $Q$ could be the statement, ``I will replace your broken computer.'' So $P \to Q$ is then the statement, ``If you do not tamper with your hardware, then I will replace your broken computer.'' If both $P$ and $Q$ are true, then this should be true---because I promised to replace the computer if you left it alone. If $P$ is true and $Q$ is false, then the statement should be false because I broke my promise. However, my promise holds true whenever $P$ is false. Why? Because you broke our agreement by tampering with the hardware. So while I may or may not replace the computer, my promise has not been broken in either case, i.e. it remains true. In an implication $P \to Q$, if $P$ is false, then the statement $P \to Q$ is \textit{always} true. \pvspace{1.5cm}



% Quiz 2
\quizsol \textit{True/False}: $\forall x, \exists y, \, x^2 + y= 4$ \pspace

\sol The statement is \textit{true}. The statement says that for all $x$ there is a $y$ such that $x^2 + y= 4$. If this is true (which it is), we need to prove it. Fix an $x$, say $x_0$. We need to find a $y$ such that $x_0^2 + y = 4$. Define $y_0:= 4 - x_0^2$. But then we have 
	\[
	x_0^2 + y_0= x_0^2 + (4 - x_0^2)= 4,
	\]
as desired. \pvspace{1.5cm}



% Quiz 3
\quizsol \textit{True/False}: $\neg \left( \forall x, \exists y, P(x,y) \vee \neg Q(x,y) \right)= \exists x, \forall y, \neg P(x,y) \wedge Q(x,y)$ \pspace

\sol The statement is \textit{true}. We can simply compute the negation step-by-step:
	\[
	\begin{aligned}
	\neg \left( \forall x, \exists y, P(x,y) \vee \neg Q(x,y) \right)&\equiv \exists x, \neg( \exists y, P(x,y) \vee \neg Q(x,y) ) \\
	&\equiv \exists x, \forall y, \neg ( P(x,y) \vee \neg Q(x,y) ) \\
	&\equiv \exists x, \forall y, \neg P(x,y) \wedge \neg (\neg Q(x,y)) \\
	&\equiv \exists x, \forall y, \neg P(x,y) \wedge Q(x,y)
	\end{aligned}
	\]



% Quiz 4
\quizsol \textit{True/False}: To prove $P \Rightarrow Q$, you can prove $Q \Rightarrow P$. \pspace

\sol The statement is \textit{false}. The converse of $P \Rightarrow Q$ is $Q \Rightarrow P$. The converse of a logical statement is not necessarily logically equivalent to the original statement. So proving the converse does not necessarily prove the original statement. However, the contrapositive of $P \Rightarrow Q$, which is $\neg Q \Rightarrow \neg P$, is logically equivalent to $P \Rightarrow Q$. Therefore, to prove $P \Rightarrow Q$, one only need prove $\neg Q \Rightarrow \neg P$. This is called proof by contrapositive. \pvspace{1.5cm}



% Quiz 5
\quizsol \textit{True/False}: Let $A= \{ 1 \}$ and $B= \{ 3, \{ 1 \} \}$. Then $A \subseteq B$. \pspace

\sol The statement is \textit{false}. Recall that $A \subseteq B$ if every element of $A$ is an element of $B$. The only element of $A$ is the element $1$. However, $1 \notin B$, but rather $\{ 1 \} \in B$, i.e. $1$ is not in $B$ but the set consisting of only the element of $1$ is in $B$. However, note that $A \in B$ because $A= \{ 1 \}$ and $\{ 1 \} \in B$. \pvspace{1.5cm}



% Quiz 6
\quizsol \textit{True/False}: Take the universal set to be the integers. Then the following two sets are equal:
	\[
	\begin{aligned}
	A&= \{ n \colon n \text{ odd} \} \\
	B&= \{ m \colon m \text{ prime and } m > 2 \}
	\end{aligned}
	\]

\sol The statement is \textit{false}. We know that $9 \in A$ because $9$ is odd. But $9 \notin B$ because $9= 3 \cdot 3$ is not prime. Therefore, $A \not\subseteq B$ so that $A \neq B$. \pvspace{1.5cm}



% Quiz 7
\quizsol \textit{True/False}: The sets $A \times B \times C$ and $(A \times B) \times C$ are not the same. \pspace

\sol The statement is \textit{true}. Elements in $A \times B \times C$ `look like' $(a, b, c)$, where $a \in A$, $b \in B$, and $c \in C$. Whereas elements in $(A \times B) \times C$ `look like' $((a, b), c)$, where $a \in A$, $b \in B$, and $c \in C$. Because elements in these sets are not of the same form, they cannot be the same. As an explicit example, take $A= \{ 1 \}$, $B= \{ 2, 3 \}$, and $C= \{ 4 \}$. Then 
	\[
	\begin{aligned}
	A \times B \times C&= \{ (1, 2, 4), (1, 3, 4) \} \\
	(A \times B) \times C&= \{ ((1, 2), 4), ((1, 3), 4) \}
	\end{aligned}
	\]
Then $A \times B \times C \neq (A \times B) \times C$. 



\newpage



% Quiz 8
\quizsol \textit{True/False}: There is a set $S$ such that $\mathcal{P}(S)$ has 3 elements. \pspace

\sol The statement is \textit{false}. If $S$ is an infinite set, then clearly there is a subset for each element $s \in S$, i.e. the subset $\{ s \}$. Clearly, if there is such a set, it cannot be infinite. Now if $S$ had 3 or more elements---having a subset for each element of $S$---we know that $\mathcal{P}(S)$ would have more than 3 subsets. Therefore, $S$ must have 0, 1, or 2 elements. If $S= \emptyset$, then $\mathcal{P}(S)= \{ \emptyset \}$. If $S= \{ s_1 \}$, then $\mathcal{P}(S)= \{ \emptyset, \{ s_1 \} \}$. Finally, if $S= \{ s_1, s_2 \}$, then $\mathcal{P}(S)= \{ \emptyset, \{ s_1 \}, \{ s_2 \}, S \}$. Therefore, there cannot be such a set $S$. \pvspace{1.5cm}



% Quiz 9
\quizsol \textit{True/False}: The Principle of Induction is logically equivalent to the Well-Ordering Principle. \pspace

\sol The statement is \textit{true}. We saw in class that the Well-Ordering Principle implied the Principle of Induction. From the homework, we know that the Principle of Induction implies the Well-Ordering Principle. \pvspace{1.5cm}



%% Quiz 10
%\quizsol \textit{True/False}: If $P(n)$ is a proposition for each $n \in \mathbb{N}$ and $P(1), P(2), P(3), \ldots, P(k)$ are all true, then $P(n)$ is true for all $n \geq 1$. \pspace
%
%\sol The statement is \textit{false}. These are only base cases. For induction to imply that $P(n)$ is true for all $n \in \mathbb{N}$, we need $P(k)$ being true to imply $P(k+1)$ is true. A statement can be true for \textit{many} $n$ and not be true for all $n$. For instance, the polynomial $p(n)= n^2 - n + 41$ is prime for $n= 1, 2, \ldots, 40$ but not for $n= 41$. In fact, a statement can be true for all but one $n$!  \pvspace{1.5cm}
%
%
%
%% Quiz 11
%\quizsol \textit{True/False}: If $f: A \to \mathbb{R}$ is positive and $g: A \to \mathbb{R}$ is nonnegative, then $fg: A \to \mathbb{R}$ is positive. \pspace
%
%\sol The statement is \textit{false}. It is possible. For instance, $f: \mathbb{R} \to \mathbb{R}$ given by $f(x):= x^2 + 1$ and $g: \mathbb{R} \to \mathbb{R}$ given by $g(x)= |x| + 1$ so that $fg= (x^2 + 1)(|x| + 1)$. However, because $g$ is only nonnegative, it can take on the value zero. But then for these values, $fg$ is zero and hence not positive. For instance, let $f: \mathbb{R} \to \mathbb{R}$ be defined by $f(x):= x^2 + 1$ and $g: \mathbb{R} \to \mathbb{R}$ be given by $g(x):= |x|$. Then $(fg)(0)= (0^2 + 1)(|0|)= 0 \not> 0$ so that $fg$ is not positive. \pvspace{1.5cm}
%
%
%
%% Quiz 12
%\quizsol \textit{True/False}: The sets $\mathbb{Z}$ and $\mathbb{Q}$ have the same cardinality. \pspace
%
%\sol The statement is \textit{true}. We say this via the diagonalization argument given in class. To prove this more concretely, one need show that there is a bijection between $\mathbb{Z}$ and $\mathbb{Q}$. By the Cantor-Schr\"oder-Bernstein Theorem, it suffices to prove there are injections $f: \mathbb{Z} \to \mathbb{Q}$ and $g: \mathbb{Q} \to \mathbb{Z}$. Let $f: \mathbb{Z} \to \mathbb{Q}$ be given by $f(x):= x$, i.e. taking advantage of the fact that $\mathbb{Z} \subseteq \mathbb{Q}$. Clearly, $f$ is injective: if $x= f(x) = f(y) = y$, then $x= y$. Now define $g: \mathbb{Q} \to \mathbb{Z}$ be given as follows: if $q \in \mathbb{Q}$, write $q= a/b$ for some $a, b \in \mathbb{Z}$. Without loss of generality, assume that $\gcd(a, b)= 1$ and either $a, b \geq 0$ or $a < 0$ and $b \geq 0$; that is, assume $a, b$ are relatively prime and that if $q \geq 0$, then $a_1, b_1$ are chosen to be nonnegative and if $q < 0$, then $a$ is chosen to be negative while $b$ is chosen to be nonnegative. Then define $g: \mathbb{Q} \to \mathbb{Z}$ via
%	\[
%	g(q)= 
%	\begin{cases}
%	2^a 3^b, & q \geq 0 \\
%	-2^{-a} 3^b, & q < 0 \\
%	\end{cases}
%	\]
%It is clear that if $q \geq 0$, then $g(q) \in \mathbb{Z}$. If $q= a/b < 0$, then $a < 0$ so that $-a > 0$. But then $-2^{-a} 3^b \in \mathbb{Z}$ so that $g(q) \in \mathbb{Z}$. Note that $g(q) \notin \{ \pm 1 \}$ because this would require $a= b= 0$, but because $q= a/b$, we know $b \neq 0$. \pspace
%
%We claim that $g$ is injective. Suppose that $g(q_1)= g(q_2)$, where $q_1, q_2 \in \mathbb{Q}$ with $q_1= a_1/b_1$, $q_2= a_2/b_2$ and $a_1, b_1, a_2, b_2 \in \mathbb{Z}$ are chosen as above. Obviously, $g(q_1)$ and $g(q_2)$ must have the same sign. By cancelling negatives, we may assume without loss of generality that $q_1, q_2 \geq 0$. But then $g(q_1)= 2^{a_1} 3^{b_1} = 2^{a_2} 3^{b_2}= g(q_2)$. By the uniqueness of factorization for integers, the number of factors of 2 and 3 on the left and right side of the equality must be the same, respectively. But then $a_1= a_2$ and $b_1= b_2$. But then $q_1= a_1/b_1= a_2/b_2= q_2$ so that $g$ is injective. \pvspace{1.5cm}
%
%
%
%% Quiz 13
%\quizsol \textit{True/False}: 
%The relation on $\mathbb{N}$ given by $x \sim y$ if and only if $xy$ is even is an equivalence relation. \pspace
%
%\sol The statement is \textit{false}. For $\sim$ to be an equivalence relation, $\sim$ must be reflexive, i.e. $n \sim n$ for all $n \in \mathbb{N}$. Take $n= 1$. Then $1(1)= 1$ is odd so that $1 \not\sim 1$. But then $\sim$ is not reflexive. 








\end{document}