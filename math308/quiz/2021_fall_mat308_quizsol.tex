\documentclass[11pt,letterpaper]{article}
\usepackage[lmargin=1in,rmargin=1in,bmargin=1in,tmargin=1in]{geometry}
\usepackage{style/quiz}
\usepackage{style/commands}

% -------------------
% Content
% -------------------
\begin{document}
\thispagestyle{title}

\quizsol \textit{True/False}: The following is a truth table for $P \to Q$:
	\begin{table}[!ht]
	\centering
	\begin{tabular}{c|c||c}
	$P$ & $Q$ & $P \to Q$ \\ \hline
	T & T& T \\
	T & F & F \\
	F & T & F \\
	F & F & F
	\end{tabular}
	\end{table}

\sol The statement is \textit{false}. The correct truth table should be\dots
	\begin{table}[!ht]
	\centering
	\begin{tabular}{c|c||c}
	$P$ & $Q$ & $P \to Q$ \\ \hline
	T & T& T \\
	T & F & F \\
	F & T & T \\
	F & F & T
	\end{tabular}
	\end{table} \par
One way to think about this is as follows: imagine $P$ is a guarantee. Namely, we promise that if $P$ happens, $Q$ must happen. For instance, $P$ could represent the statement, ``You do not tamper with your hardware,'' and $Q$ could be the statement, ``I will replace your broken computer.'' So $P \to Q$ is then the statement, ``If you do not tamper with your hardware, then I will replace your broken computer.'' If both $P$ and $Q$ are true, then this should be true---because I promised to replace the computer if you left it alone. If $P$ is true and $Q$ is false, then the statement should be false because I broke my promise. However, my promise holds true whenever $P$ is false. Why? Because you broke our agreement by tampering with the hardware. So while I may or may not replace the computer, my promise has not been broken in either case, i.e. it remains true. In an implication $P \to Q$, if $P$ is false, then the statement $P \to Q$ is \textit{always} true. 


























\end{document}