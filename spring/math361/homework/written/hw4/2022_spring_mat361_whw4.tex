\documentclass[11pt,letterpaper]{article}
\usepackage[lmargin=1in,rmargin=1in,tmargin=1in,bmargin=1in]{geometry}
\usepackage{../style/homework}
\usepackage{../style/commands}
\setbool{quotetype}{true} % True: Side; False: Under
\setbool{hideans}{true} % Student: True; Instructor: False

% -------------------
% Content
% -------------------
\begin{document}

\homework{4: Due 05/12}{Film is one of the three universal languages, the other two: mathematics and music.}{Frank Cappa}


% Problem 1.
\problem{Uniqueness of Interpolation} Given a set of $n + 1$ points $\{(x_i, y_i)\}_{i=0}^n$, we constructed---using various methods---a polynomial interpolating these points. Regardless of the method, especially in appearance, the polynomial was unique. We show this is the case: let $f(x)$ be a polynomial of degree at most $n$ and let $\{(x_i, y_i)\}_{i=0}^n$ be the points given by $y_i= f(x_i)$. 
	\begin{enumerate}[(a)]
	\item Let $p_n(x)$ be an interpolating polynomial for these points. Show that $f(x)= p_n(x)$. 
	\item Suppose that $L(x)$ was a Lagrange polynomial interpolating the points $\{(x_i, y_i)\}_{i=0}^n$. Use the error term associated to $L(x)$ to show that $f(x)= L(x)$. 
	\item Consider the set of points $\{ (-1, -18), (0, 2), (1, 4), (2, 6) \}$. Each of the following polynomials interpolate these points:
		\[
		\begin{aligned}
		f(x)&= 2x + x(x - 1) + 3x(x - 1)(x - 2) - (x + 1)(x - 2) \\
		g(x)&= 18x - 7x(x + 1) + 4(x + 1)(x - 1) + 3(x + 1)(x - 1)(x - 2)
		\end{aligned}
		\]
	Does this violate the uniqueness of interpolating polynomials? Explain why or why not.
	\end{enumerate}



\newpage



% Problem 2.
\problem{Lagrange \& Newton Polynomials} We have seen how to construct Lagrange and Newton interpolating polynomials. Here, we practice the procedures involved in creating these polynomials. Consider the function $f(x)= \sin(\ln x)$. Suppose we want to interpolate this function on the interval $[1, 4]$. We have the following data:
	\begin{table}[!ht]
	\centering
	\begin{tabular}{r|rrrrr}
	$x$ & $1$ & $2\phantom{000}$ & $3\phantom{000}$ & $4\phantom{000}$ \\ \hline
	$f(x)$ & $0$ & $0.638961$ & $0.890577$ & $0.983028$  
	\end{tabular}
	\end{table}

\begin{enumerate}[(a)]
\item Construct the Lagrange polynomial interpolating the function $f(x)$ on the interval $[1, 4]$ using the data above. 
\item Use a divided difference table to construct the Newton polynomial interpolating the function $f(x)$ on the interval $[1, 4]$ using the data above. 
\item Use (a) or (b) to approximate $f(2.5)$. Does it matter if you use the answer from (a) or (b)? Explain.
\item What is the maximum possible error in your approximation from (c)? Justify your answer. 
\item If one were to want to add another point $(x, f(x))$ to the interpolation, should one use the polynomial in (a) or (b)? Explain. 
\end{enumerate}



\newpage



% Problem 3.
\problem{Hermite Interpolation} We have seen that given a set of nodes, $x_0, \ldots, x_n$, we can find a polynomial that values $f_0, \ldots, f_n$ at these nodes. However, we know that we can also find polynomials with not only specified values but specified derivatives as well. Suppose you were a Calculus instructor writing an in-class assignment for students practicing derivatives and their related theorems. Find a polynomial $f(x)$ such that $f(1)= 4$, $f(2)= -3$, and that has a maximum or minimum value at $x= 1$. Show that your function satisfies these properties. 



\newpage



% Problem 3.
\problem{Chebyshev Nodes} When interpolating a function, one must first make a selection of nodes $\{ x_i \}_{i=0}^n$ and then construct the corresponding interpolating polynomial using  a method of choice. This interpolating polynomial comes with an associated error term, which we have seen. However, not all choice of nodes `are created equal.' The Chebyshev nodes minimized the error in interpolating a function. This selection of nodes came from finding roots to the Chebyshev polynomials. Recall the Chebyshev polynomials were generated by choosing $T_0(x)= 1$, $T_1(x)= x$, and using the recurrence relation $T_k(x)= 2x T_{k-1}(x) - T_{k-2}(x)$. 
	\begin{enumerate}[(a)]
	\item Find the first four Chebyshev polynomials. 
	\item Use the property that $T_n(x)= \cos(n \arccos(x))$ for $-1 \leq x \leq 1$ to show that the roots of $T_n(x)$ are 
		\[
		x_k= \cos\left( \dfrac{(2k + 1)\pi}{2n} \right), \quad k= 0, 1, \ldots, n - 1
		\]
	\item Using four Chebyshev nodes, construct an interpolating polynomial for $f(x)= e^{-x}$ on $[-1, 1]$. 
	\item What nodes should be chosen to interpolate $f(x)= e^{-x}$ on $[-2, 6]$? Explain and construct the interpolating polynomial. 
	\end{enumerate}



\newpage



% Evaluation
\noindent{\bfseries Evaluation.} \pvspace{0.3cm}

Complete the following survey by rating each problem. Each area will be rated on a scale of 1 to 5. For interest, 1 is ``mind-numbing'' while a 5 is ``mind-blowing.'' For difficulty, 1 is ``trivial/routine'' while 5 is ``brutal.'' For learning, 1 means ``nothing new'' while 5 means ''profound awakening.'' Then you to estimate the amount of time you spent on each problem (in minutes). 

\vspace{0.25cm}
\begin{center}
\begin{tabular}{c||c|c|c|c|}
 & Interest & Difficulty & Learning & Time Spent \\ \hline \hline
Problem 1 &  &  &  &  \\ \hline
Problem 2 &  &  &  &  \\ \hline
Problem 3 &  &  &  &  \\ \hline
Problem 4 &  &  &  &  
\end{tabular}
\end{center}
\vspace{0.25cm}

Finally, indicate whether you believe lectures were useful in completing this assignment and whether you believe the problems were useful enough/interesting enough to assign again to future students by checking the appropriate space.

\vspace{0.25cm}
\begin{center}
\begin{tabular}{c||c|c|c|c|}
  & \multicolumn{2}{c|}{Lectures} &  \multicolumn{2}{c|}{Assign Again} \\ \cline{2-5}
   & Yes & No & Yes & No \\ \hline \hline
  Problem 1 &  &  &  &  \\ \hline 
  Problem 2 &  &  &  &  \\ \hline 
  Problem 3 &  &  &  &  \\ \hline 
  Problem 4 &  &  &  &  
\end{tabular}
\end{center}

\end{document}