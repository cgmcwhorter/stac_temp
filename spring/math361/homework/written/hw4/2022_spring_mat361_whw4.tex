\documentclass[11pt,letterpaper]{article}
\usepackage[lmargin=1in,rmargin=1in,tmargin=1in,bmargin=1in]{geometry}
\usepackage{../style/homework}
\usepackage{../style/commands}
\setbool{quotetype}{false} % True: Side; False: Under
\setbool{hideans}{true} % Student: True; Instructor: False

% -------------------
% Content
% -------------------
\begin{document}

\homework{1: Due 02/15}{Well, that's no reason to cry. One cries because one is sad. For example, I cry because others are stupid and it makes me sad.}{Sheldon Cooper, The Big Bang Theory}


% Problem 1.
\problem{Horner's Method} We have seen that Horner's Method is an extremely efficient method for evaluating polynomials. Here, we will compare the efficiency of several methods of evaluating polynomials. Let $x_0 \in \mathbb{R}$, $f(x)= x^4 - 9x^2 - 4x + 12$, and $g(x)= a_nx^n + a_{n-1}x^{n-1} + \cdots + a_1x + a_0$
	\begin{enumerate}[(a)]
	\item Evaluate $f(2)$ using the traditional `by-hand way.' How many flops does this evaluation require? How many flops does evaluating $g(x_0)$ require using the traditional `by-hand way'?
	\item Evaluate $f(2)$ by computing the powers of 2 using fast exponentiation. How many flops does this evaluation require? How many flops does evaluating $g(x_0)$ require using fast exponentiation to compute the powers of $x_0$ require?
	\item Evaluate $f(2)$ using Horner's Method. How many flops does this evaluation require? How many flops does evaluating $g(x_0)$ require using Horner's Method require?
	\item Using the fact that $f(x)$ has roots $x= -2, -2, 1, 3$ and leading coefficient 1, we can write $f(x)= 1 \cdot (x + 2)^2 (x - 1)(x - 3)$. Compute $f(2)$ as written. How many flops does this require? If $g(x)$ has roots $r_1, r_2, \ldots, r_n$, we can write
		\[
		g(x)= a_n \prod_{i=1}^n (x - r_i)
		\]
	How many flops does evaluating $g(x)$ written as above require? Comment on any possible issues evaluating $g(x)$ in this fashion and how one might resolve them. 
	\item Of all the methods used above, which requires the least amount of flops to evaluate $g(x_0)$?
	\end{enumerate}



\newpage



% Problem 2. 
\problem{Bisection Method} We have seen that the bisection method always converges to a solution of $f(x)= 0$ on $[a, b]$ if $f(x)$ is continuous and $f(a)f(b) < 0$. Let's examine the bisection method a bit more closely:
	\begin{enumerate}[(a)]
	\item Practice implementing the bisection method by using three steps of the bisection method to find a solution to $x^3= x^2 + 5$ using an initial interval of $[0, 3]$. Be sure to verify that the bisection method will converge to a solution.
	\item How many steps of the bisection method would one have to perform using the initial interval in (a) to approximate the solution to five digits of accuracy?
	\item 
	\item 
	\end{enumerate}

% Problem 3.
\problem{} 
bisection by hand
fail (x-1)^2
can you get all the roots?


% Problem 4.
\problem{}
newton method by hand
newton fail
(0 problem)
diverge problem
loop problem
not quadratic

% Problem 5.
\problem{} 
fixed point


pn= harmonic
show diverges 
show error tends to 0

taylor series to show
e_{k+1} approx |f''(x_k)/(2f'(x_k)) e_k| |e_k|
so if close to enough to make term < 1 conv fast

prob 23 matlab book
prob 18 matlab book
prob 22 matlab book

\newpage



% Evaluation
\noindent{\bfseries Evaluation.} \pvspace{0.3cm}

Complete the following survey by rating each problem. Each area will be rated on a scale of 1 to 5. For interest, 1 is ``mind-numbing'' while a 5 is ``mind-blowing.'' For difficulty, 1 is ``trivial/routine'' while 5 is ``brutal.'' For learning, 1 means ``nothing new'' while 5 means ''profound awakening.'' Then you to estimate the amount of time you spent on each problem (in minutes). 

\vspace{0.25cm}
\begin{center}
\begin{tabular}{c||c|c|c|c|}
 & Interest & Difficulty & Learning & Time Spent \\ \hline \hline
Problem 1 &  &  &  &  \\ \hline
Problem 2 &  &  &  &  \\ \hline
Problem 3 &  &  &  &  \\ \hline
Problem 4 &  &  &  &  \\ \hline
Problem 5 &  &  &  &  \\ \hline
\end{tabular}
\end{center}
\vspace{0.25cm}

Finally, indicate whether you believe lectures were useful in completing this assignment and whether you believe the problems were useful enough/interesting enough to assign again to future students by checking the appropriate space.

\vspace{0.25cm}
\begin{center}
\begin{tabular}{c||c|c|c|c|}
  & \multicolumn{2}{c|}{Lectures} &  \multicolumn{2}{c|}{Assign Again} \\ \cline{2-5}
   & Yes & No & Yes & No \\ \hline \hline
  Problem 1 &  &  &  &  \\ \hline 
  Problem 2 &  &  &  &  \\ \hline 
  Problem 3 &  &  &  &  \\ \hline 
  Problem 4 &  &  &  &  \\ \hline 
  Problem 5 &  &  &  &  \\ \hline 
\end{tabular}
\end{center}

\end{document}