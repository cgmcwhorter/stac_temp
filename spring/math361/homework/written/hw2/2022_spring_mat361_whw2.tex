\documentclass[11pt,letterpaper]{article}
\usepackage[lmargin=1in,rmargin=1in,tmargin=1in,bmargin=1in]{geometry}
\usepackage{../style/homework}
\usepackage{../style/commands}
\setbool{quotetype}{false} % True: Side; False: Under
\setbool{hideans}{true} % Student: True; Instructor: False

% -------------------
% Content
% -------------------
\begin{document}

\homework{2: Due 03/04}{Mathematics is not about numbers, equations, computations, or algorithms: it is about understanding.}{William (Bill) Paul Thurston}


% Problem 1. Horner's Method
\problem{Horner's Method} We have seen that Horner's Method is an extremely efficient method for evaluating polynomials. Here, we will compare the efficiency of several methods of evaluating polynomials. Let $x_0 \in \mathbb{R}$, $f(x)= x^4 - 5x^3 + 20x - 16$, and $g(x)= a_nx^n + a_{n-1}x^{n-1} + \cdots + a_1x + a_0$
	\begin{enumerate}[(a)]
	\item Evaluate $f(2)$ using the traditional `by-hand way.' How many flops does this evaluation require? How many flops does evaluating $g(x_0)$ require using the traditional `by-hand way'?
	\item Evaluate $f(2)$ by computing the powers of 2 using fast exponentiation. How many flops does this evaluation require? How many flops does evaluating $g(x_0)$ require using fast exponentiation to compute the powers of $x_0$ require?
	\item Evaluate $f(2)$ using Horner's Method. How many flops does this evaluation require? How many flops does evaluating $g(x_0)$ require using Horner's Method require?
	\item Using the fact that $f(x)$ has roots $x= -2, 1, 2, 4$ and leading coefficient 1, we can write $f(x)= 1 \cdot (x - 4)(x - 2)(x - 1)(x + 2)$. Compute $f(2)$ as written. How many flops does this require? If $g(x)$ has roots $r_1, r_2, \ldots, r_n$, we can write
		\[
		g(x)= a_n \prod_{i=1}^n (x - r_i)
		\]
	How many flops does evaluating $g(x)$ written as above require? Comment on any possible issues evaluating $g(x)$ in this fashion and how one might resolve them. 
	\item Of all the methods used above, which requires the least amount of flops to evaluate $g(x_0)$?
	\end{enumerate}



\newpage



% Problem 2. Bisection Method
\problem{Bisection Method} If $f(x)$ is continuous and $f(a)f(b) < 0$, we have seen that the bisection method always converges to a solution of $f(x)= 0$ on $[a, b]$. Let's examine the bisection method a bit more closely:
	\begin{enumerate}[(a)]
	\item Practice implementing the bisection method by using three steps of the bisection method to find a solution to $x^3= x^2 + 5$ using an initial interval of $[0, 3]$. Be sure to verify that the bisection method will converge to a solution.
	\item How many steps of the bisection method would one have to perform using the initial interval in (a) to approximate the solution to five decimal digits of accuracy?
	\item Show that the polynomial $p(x)= x^2 - 2x + 1$ has a root on the interval $[-2, 2]$. Will the bisection method find this root? Explain. What if the interval were $[0, 2]$?
	\item Consider the polynomial $p(x)= (x - 1)(x - 2)(x - 3)$. How many roots does $p(x)$ have on the interval $[0, 5]$? Does the bisection method apply to this function on this interval? What root will the bisection method find? Can the bisection method be used to find all the roots? Explain. 
	\end{enumerate}



\newpage



% Problem 3. Newton's Method
\problem{Newton's Method} Recall that if $f \in C^2[a, b]$ and $r$ is a simple root of $f(x)$, i.e. $f(r)= 0$ and $f'(r) \neq 0$, there exists $\delta > 0$ such that for $x_0 \in [r - \delta, r + \delta]$, Newton's Method with initial value $x_0$ will converge to $r$. We have seen that Newton's method is a powerful technique for finding solutions to the equation $f(x)= 0$ when $f(x)$ is differentiable because the algorithm has quadratic convergence. However, Newton's Method does have some issues which we will explore here.
	\begin{enumerate}[(a)]
	\item Use Newton's Method `by hand' to approximate $\sqrt[3]{5}$ to at least two decimal digits of accuracy using an appropriate initial value. 
	\item Consider the function $f(x)= \sqrt[3]{x}$. Does $f(x)$ have a root? Can Newton's Method be applied to find the root using any initial value $x_0 \neq 0$? Your explanation should include a computational justification. Does this violate the theorem in the problem statement? 
	\item Consider the polynomial $p(x)= -3x^3 + 5x^2 + x - 1$. Give a sketch of the function.  Using this sketch, approximately what are the roots of $p(x)$? Apply Newton's Method `by hand' using an initial input of $x_0= 1$. Explain what happens (including a sketch of the method for visualization) and reconcile this with the theorem in the problem statement. 
	\end{enumerate}



\newpage



% Problem 4. Newton's Method for Repeated Roots
\problem{Newton's Method for Repeated Roots} We say that a function has a root of multiplicity $n$ at $x_0$ if $f(x_0)= f'(x_0)= \cdots= f^{(n)}(x_0)= 0$ and $f^{(n+1)}(x_0) \neq 0$. A root of multiplicity one is called a simple root. We have seen that for simple roots of a function, given `appropriate' initial values, Newton's Method converges quadratically. However, if a function has a repeated root, Newton's Method will only converge linearly. However, this convergence rate can be improved.
	\begin{enumerate}[(a)]
	\item Using Taylor series, explain why a sufficiently smooth function $f(x)$ with a root $r$ of multiplicity $n$ can be written as $(x - r)^n q(x)$, where $q(r) \neq 0$, for values of $x$ `close to $r$
	\item Show that the function $h(x):= f(x)/f'(x)$ has a simple root at $r$. 
	\item Show that applying Newton's Method to $h(x)$ yields
		\[
		x_{n+1}= x_n - \dfrac{f(x_n) f'(x_n)}{(f'(x_n))^2 - f(x_n) f''(x_n)}
		\]
	\item Explain why for `appropriate' values for $x_0$, the sequence $\{ x_n \}$ from (c) converges quadratically to $r$. 
	\end{enumerate}



\newpage



% Evaluation
\noindent{\bfseries Evaluation.} \pvspace{0.3cm}

Complete the following survey by rating each problem. Each area will be rated on a scale of 1 to 5. For interest, 1 is ``mind-numbing'' while a 5 is ``mind-blowing.'' For difficulty, 1 is ``trivial/routine'' while 5 is ``brutal.'' For learning, 1 means ``nothing new'' while 5 means ''profound awakening.'' Then you to estimate the amount of time you spent on each problem (in minutes). 

\vspace{0.25cm}
\begin{center}
\begin{tabular}{c||c|c|c|c|}
 & Interest & Difficulty & Learning & Time Spent \\ \hline \hline
Problem 1 &  &  &  &  \\ \hline
Problem 2 &  &  &  &  \\ \hline
Problem 3 &  &  &  &  \\ \hline
Problem 4 &  &  &  &  \\ \hline
%Problem 5 &  &  &  &  \\ \hline
\end{tabular}
\end{center}
\vspace{0.25cm}

Finally, indicate whether you believe lectures were useful in completing this assignment and whether you believe the problems were useful enough/interesting enough to assign again to future students by checking the appropriate space.

\vspace{0.25cm}
\begin{center}
\begin{tabular}{c||c|c|c|c|}
  & \multicolumn{2}{c|}{Lectures} &  \multicolumn{2}{c|}{Assign Again} \\ \cline{2-5}
   & Yes & No & Yes & No \\ \hline \hline
  Problem 1 &  &  &  &  \\ \hline 
  Problem 2 &  &  &  &  \\ \hline 
  Problem 3 &  &  &  &  \\ \hline 
  Problem 4 &  &  &  &  \\ \hline 
%  Problem 5 &  &  &  &  \\ \hline 
\end{tabular}
\end{center}

\end{document}