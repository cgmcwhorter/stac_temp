\documentclass[11pt,letterpaper]{article}
\usepackage[lmargin=1in,rmargin=1in,tmargin=1in,bmargin=1in]{geometry}
\usepackage{../style/homework}
\usepackage{../style/commands}
\setbool{quotetype}{false} % True: Side; False: Under
\setbool{hideans}{true} % Student: True; Instructor: False

% -------------------
% Content
% -------------------
\begin{document}

\homework{1: Due 02/15}{Well, that's no reason to cry. One cries because one is sad. For example, I cry because others are stupid and it makes me sad.}{Sheldon Cooper, The Big Bang Theory}

% Problem 1. Big Oh
\problem{Big $\cO$} The following problem will explore $\cO$, $\smallO$, and their relationship. It will also explore the relationship between $\cO$ and Taylor series. 
	\begin{enumerate}[(a)]
	\item For each of the following sequences $\{ x_n \}$, determine if $x_n= \cO(f_n)$ and $x_n = \smallO(f_n)$ as $x \to \infty$:
		\begin{enumerate}[(i)]
		\item $x_n= 5n + 7$, $f_n= n$
		\item $x_n= 2n^2 + 3n + 1$, $f_n= n^3$
		\item $x_n= \sqrt{n + 5}$, $f_n= n$
		\end{enumerate}
	\item Show that if $x_n= \cO(y_n)$, then $cx_n= \cO(cy_n)$, where $c \in \R$. Is the same true for $\smallO$?
	\item Determine the `best' integer $k$ such that $e^{2x}= 1 + 2x + \cO(x^k)$ as $x \to 0$. 
	\item Explain why every smooth function can be approximated on an interval of length $h$ by a polynomial of degree $n$ with an error that is $\cO(h^{n+1})$ as $h \to 0$. [Hint: Appeal to Taylor's Theorem.]
	\end{enumerate}



\newpage



% Problem 2. IEEE Floating Point Numbers
\problem{IEEE Floating Point Numbers} This problem will work through an explicit example of how IEEE floating point numbers work with a `nice' real number. Define $n:= 2^5 + 2^{-17} + 2^{-21}$.
	\begin{enumerate}[(a)]
	\item Write $n$ in binary. 
	\item Express $n$ as a IEEE single precision number.
	\item Determine $\fl(n)$. 
	\item Determine the absolute error $|n - \fl(n)|$ and relative error $\dfrac{|n - \fl(n)|}{|n|}$. 
	\item What must the relative error be bounded by? Does your answer in (d) agree with this bound?
	\end{enumerate}



\newpage



% Problem 3. Numerical Stability
\problem{Numerical Stability} The following problem will help develop `hands-on' experience recognizing and avoiding loss of significance. For each of the following functions, describe when a loss of significance may occur and suggest---with justification---ways to avoid the loss. 
	\begin{enumerate}[(a)]
	\item $\sqrt{x + 4} - 2$
	\item $\ln x - 1$ [Do \textit{not} use Taylor Series]
	\item $e^x - e^{-2x}$ [Hint: Taylor Series]
	\item $2\cos^2 x - 1$
	\end{enumerate}



\newpage



% Problem 4 Condition Number
\problem{Condition Number} The following problem will help develop `hands-on' computing condition numbers and recognizing where a computation might be `unstable.' For the following functions, compute the condition number and describe the $x$ values where the condition number is large:
	\begin{enumerate}[(a)]
	\item $x^5$
	\item $(x + 2)^3$
	\item $\ln x$
	\item $\sin x$
	\end{enumerate}



\newpage



% Problem 4 Algorithm
\problem{Algorithm Analysis} The following problem will help develop `hands-on' experience reading and improving algorithms. Consider the following pseudocode that takes as an input an integer $n$:
	\begin{verbatim}
	                              for i = 1 to n:
	                                sum = 0;
	                                  for j = 1 to i:
	                                  sum += j
	                                print(sum)
	\end{verbatim}

\begin{enumerate}[(a)]
\item Describe what this algorithm is computing. 
\item How many flops are used in this algorithm? Find $x_n$ such that this algorithm is $\cO(x_n)$. 
\item Explain why this algorithm is computationally inefficient. 
\item Rewrite this pseudocode so that it is computationally less expensive. For your algorithm, count the number of flops required. 
\item Is there an even shorter way avoiding loops entirely?
\end{enumerate}



\newpage



% Evaluation
\noindent{\bfseries Evaluation.} \pvspace{0.3cm}

Complete the following survey by rating each problem. Each area will be rated on a scale of 1 to 5. For interest, 1 is ``mind-numbing'' while a 5 is ``mind-blowing.'' For difficulty, 1 is ``trivial/routine'' while 5 is ``brutal.'' For learning, 1 means ``nothing new'' while 5 means ''profound awakening.'' Then you to estimate the amount of time you spent on each problem (in minutes). 

\vspace{0.25cm}
\begin{center}
\begin{tabular}{c||c|c|c|c|}
 & Interest & Difficulty & Learning & Time Spent \\ \hline \hline
Problem 1 &  &  &  &  \\ \hline
Problem 2 &  &  &  &  \\ \hline
Problem 3 &  &  &  &  \\ \hline
Problem 4 &  &  &  &  \\ \hline
Problem 5 &  &  &  &  \\ \hline
\end{tabular}
\end{center}
\vspace{0.25cm}

Finally, indicate whether you believe lectures were useful in completing this assignment and whether you believe the problems were useful enough/interesting enough to assign again to future students by checking the appropriate space.

\vspace{0.25cm}
\begin{center}
\begin{tabular}{c||c|c|c|c|}
  & \multicolumn{2}{c|}{Lectures} &  \multicolumn{2}{c|}{Assign Again} \\ \cline{2-5}
   & Yes & No & Yes & No \\ \hline \hline
  Problem 1 &  &  &  &  \\ \hline 
  Problem 2 &  &  &  &  \\ \hline 
  Problem 3 &  &  &  &  \\ \hline 
  Problem 4 &  &  &  &  \\ \hline 
  Problem 5 &  &  &  &  \\ \hline 
\end{tabular}
\end{center}

\end{document}