\documentclass[11pt,letterpaper]{article}
\usepackage[lmargin=1in,rmargin=1in,bmargin=1in,tmargin=1in]{geometry}
\usepackage{style/quiz}
\usepackage{style/commands}

% -------------------
% Content
% -------------------
\begin{document}
\thispagestyle{title}


% Quiz 1
\quizsol \textit{True/False}: Let $f(x)= x^3 - 7x + 5$. Observe that $f(0)= 5$ and $f(1)= -1$. Therefore, $f(x)$ has a root between 0 and 1. \pspace

\sol T


% Quiz 2
\quizsol \textit{True/False}: It is impossible to know the error in approximating a solution to a problem without knowing the solution. \pspace

\sol F

% Quiz 3
\quizsol \textit{True/False}: The condition number for a problem measures the responsiveness of a solution to variation in the inputed data used to compute the solution. \pspace

\sol T


% Quiz 4
\quizsol \textit{True/False}: The number of flops in computing $f(x)= 2x^3 + 4x^2 - 5x - 6$ using Horner's Method is 10. \pspace

\sol F


% Quiz 5
\quizsol \textit{True/False}: Suppose $f(x)$ is a continuous function on $[0, 3]$ with $f(0)f(3) < 0$. To find a root of $f(x)$ to one decimal point of accuracy using the bisection method requires three iterations. \pspace

\sol F



% Quiz 6
\quizsol \textit{True/False}: To numerically compute $\sqrt{a}$ using Newton's Method, one chooses an initial `guess' $x_0$ and recursively evaluates $x_{n+1}= \dfrac{1}{2} \left( x_n + \dfrac{a}{x_n} \right)$.

T


% Quiz 7
\quizsol \textit{True/False}: Given $n + 1$ points with distinct $x$-values, the polynomial of minimal degree passing through all the given points has degree $n$.

F



% Quiz 8
\quizsol \textit{True/False}: If you interpolate a distinct set of points $(x_0, y_0)$, $(x_1, y_1)$, \ldots, $(x_n, y_n)$ with the Newton and Lagrange methods, you \textit{must} obtain the same polynomial. \pspace

T 



% Quiz 9
\quizsol \textit{True/False}: Given a function $f(x)$ and an interval $[a, b]$ to which the bisection method applies, if $x_0 \in [a, b]$, then both the bisection method and Newton's method (with initial value $x_0$) converge to the same root of $f(x)$ on the interval $[a, b]$---the only difference being that Newton's method converges quadratically, whereas the bisection method converges linearly. \pspace

F


% Quiz 10
\quizsol \textit{True/False}: Suppose that you wanted to approximate a function $f(x)$ on the interval $[a, b]$ using Lagrange or Newton interpolation. Then the more points you use, the more accurate the approximation, i.e. the less the error $|f(x) - p(x)|$ on the interval $[a, b]$. 

F



% Quiz 11
\quizsol \textit{True/False}: To approximate $f(x)= \sin(3x)$ on the interval $[0, 1]$ with an interpolating polynomial $p(x)$ such that we have $|f(x) - p(x)| < 0.01$ for all $x \in [0, 1]$, it is sufficient to choose \textit{any} five nodes to create your interpolating polynomial. 

F


% Quiz 12
\quizsol \textit{True/False}: If $f(x) \in C^1[a, b]$ with bounded derivative and has a fixed point $p \in [a, b]$, then given $p_0 \in [a, b]$, the fixed point iteration $p_{n+1}= f(p_n)$ converges to $p$. 

F


% Quiz 13
\quizsol \textit{True/False}: The approximation $f'(x) \approx \dfrac{f(x + h) - f(x - h)}{2h}$ is exact for linear and quadratic functions. 
T



% Quiz 14
\quizsol \textit{True/False}: The composite Simpson's rule, 
	\[
	\dfrac{h}{3} \left[ f(x_0) + 2 \sum_{i=2}^{n/2} f(x_{2i-2}) + 4 \sum_{i=1}^{n/2} f(x_{2i-1}) + f(x_n) \right]
	\]
is exact for polynomials $f \in \prod_4$. 



% Quiz 15
\quizsol \textit{True/False}: Using Newton-Cotes to approximate integrals, the nodes are fixed while in quadrature, both the nodes and weights are unknown. 



% Quiz 16
\quizsol \textit{True/False}: To estimate the integral $\displaystyle \int_{-1}^5 \sin(x^2) \;dx$, one can use the Gaussian two-point quadrature\dots
	\[
	f\left( -\dfrac{1}{\sqrt{3}} \right) + f\left( \dfrac{1}{\sqrt{3}} \right)
	\]



% Quiz 17
\quizsol \textit{True/False}: Given a differentiable function $f(\mathbf{x})$ and initial value $\mathbf{x}_0$, the method of gradient descent is given recursively by $\mathbf{x}_{n+1}= \mathbf{x}_n - \nabla f(\mathbf{x}_n)$ and always converges to a global minimum---even if it tends to do so rather slowly. 




\end{document}