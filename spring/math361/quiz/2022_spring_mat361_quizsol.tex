\documentclass[11pt,letterpaper]{article}
\usepackage[lmargin=1in,rmargin=1in,bmargin=1in,tmargin=1in]{geometry}
\usepackage{style/quiz}
\usepackage{style/commands}

% -------------------
% Content
% -------------------
\begin{document}
\thispagestyle{title}


% Quiz 1
\quizsol \textit{True/False}: Let $f(x)= x^3 - 7x + 5$. Observe that $f(0)= 5$ and $f(1)= -1$. Therefore, $f(x)$ has a root between 0 and 1. \pspace

\sol T


% Quiz 2
\quizsol \textit{True/False}: It is impossible to know the error in approximating a solution to a problem without knowing the solution. \pspace

\sol F

% Quiz 3
\quizsol \textit{True/False}: The condition number for a problem measures the responsiveness of a solution to variation in the inputed data used to compute the solution. \pspace

\sol T


% Quiz 4
\quizsol \textit{True/False}: The number of flops in computing $f(x)= 2x^3 + 4x^2 - 5x - 6$ using Horner's Method is 10. \pspace

\sol F


% Quiz 5
\quizsol \textit{True/False}: Suppose $f(x)$ is a continuous function on $[0, 3]$ with $f(0)f(3) < 0$. To find a root of $f(x)$ to one decimal point of accuracy using the bisection method requires three iterations. \pspace

\sol F



% Quiz 6
\quizsol \textit{True/False}: To numerically compute $\sqrt{a}$ using Newton's Method, one chooses an initial `guess' $x_0$ and recursively evaluates $x_{n+1}= \dfrac{1}{2} \left( x_n + \dfrac{a}{x_n} \right)$.

T


% Quiz 7
\quizsol \textit{True/False}: Given $n + 1$ points with distinct $x$-values, the polynomial of minimal degree passing through all the given points has degree $n$.

F
\end{document}