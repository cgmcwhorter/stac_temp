\documentclass[11pt,letterpaper]{article}
\usepackage[lmargin=1in,rmargin=1in,tmargin=1in,bmargin=1in]{geometry}
\usepackage{../style/homework}
\usepackage{../style/commands}
\setbool{quotetype}{true} % True: Side; False: Under
\setbool{hideans}{false} % Student: True; Instructor: False

% -------------------
% Content
% -------------------
\begin{document}

\homework{4: Due 02/17}{Not everyone can become a great artist, but a great artist can come from anywhere.}{Anton Ego, Coco}

% Problem 1
\problem{10} Write the following numbers in scientific notation:
	\begin{enumerate}[(a)]
	\item 126
	\item 5
	\item 0.0000034
	\item 163000000
	\end{enumerate} \pspace

\sol
\begin{enumerate}[(a)]
\item 
	\[
	126= 1.26 \cdot 10^2
	\] \pspace

\item 
	\[
	5= 5.0 \cdot 10^0
	\] \pspace

\item 
	\[
	0.0000034= 3.4 \cdot 10^{-6}
	\] \pspace

\item 
	\[
	163000000= 1.63 \cdot 10^8
	\]
\end{enumerate}



\newpage



% Problem 2
\problem{10} Write the following numbers in decimal notation:
	\begin{enumerate}[(a)]
	\item $1.7 \cdot 10^3$
	\item $9.3 \cdot 10^0$
	\item $1.32 \cdot 10^8$
	\item $4.8 \cdot 10^{-5}$
	\end{enumerate} \pspace

\sol
\begin{enumerate}[(a)]
\item 
	\[
	1.7 \cdot 10^3= 1700
	\] \pspace

\item 
	\[
	9.3 \cdot 10^0= 9.3
	\] \pspace

\item 
	\[
	1.32 \cdot 10^8= 132\ 000\ 000
	\] \pspace

\item 
	\[
	4.8 \cdot 10^{-5}= 0.000\ 048
	\]
\end{enumerate}



\newpage



% Problem 3
\problem{10} Suppose a course grade consists of the following weights:
	\begin{table}[!ht]
	\centering
	\begin{tabular}{rl}
	Homework & 40\% \\
	Quizzes & 10\% \\
	Exam 1 & 20\% \\
	Exam 2 & 20\% \\
	Project & 10\%
	\end{tabular}
	\end{table} \par
Suppose a student had a 81\% homework average, 70\% quiz average, 85\% on exam 1, 74\% on exam 2, and 93\% on the project. Compute the student's course average. \pspace

\sol 
	\[
	\begin{aligned}
	\text{Course Average}&= 40(0.81) + 10(0.70) + 20(0.85) + 20(0.74) + 10(0.93) \\[0.3cm]
	&= 32.4 + 7 + 17 + 14.8 + 9.3 \\[0.3cm]
	&= 80.5
	\end{aligned}
	\]



\newpage



% Problem 4
\problem{10} Suppose a GPA consists of the following weights:
	\begin{table}[!ht]
	\centering
	\begin{tabular}{lr|lr}
	A & 4.0 & C+ & 2.3 \\
	A$-$ & 3.7 & C & 2.0 \\
	B+ & 3.3 & C$-$ & 1.7 \\
	B & 3.0 & D & 1.0 \\
	B$-$ & 2.7 & F & 0.0
	\end{tabular}
	\end{table} \par
Suppose a student had the following grades on their courses:
	\begin{table}[!ht]
	\centering
	\begin{tabular}{lcl}
	Course & Credits & Grade \\ \hline
	Calculus II & 4 & B+ \\
	Sophomore Seminar & 1 & A \\
	Chemistry II & 4 & B$-$ \\
	Women in Music & 3 & B+ \\
	German Philosophy Pre-1950 & 3 & C+ \\
	American Poets & 3 & D
	\end{tabular}
	\end{table}
Compute this student's GPA. \pspace

\sol 
	\[
	\begin{aligned}
	\text{GPA}&= \dfrac{ \text{Sum Credit} \cdot \text{Credit}}{\text{Course Weight}} \\[0.3cm]
	&= \dfrac{4(3.3) + 1(4.0) + 4(2.7) + 3(3.3) + 3(2.3) + 3(1.0)}{4 + 1 + 4 + 3 + 3 + 3} \\[0.3cm]
	&= \dfrac{13.2 + 4.0 + 10.8 + 9.9 + 6.9 + 3.0}{18} \\[0.3cm]
	&= \dfrac{47.8}{18} \\[0.3cm]
	&= 2.656
	\end{aligned}
	\]



\newpage



% Problem 5
\problem{10} Compute the following:
	\begin{enumerate}[(a)]
	\item $(4 - i) - (6 - 10i)$
	\item $(1 - 3i)(2 + 4i)$
	\item $(2i)^3$
	\item $\dfrac{5 + i}{1 - 2i}$
	\end{enumerate} \pspace

\sol
\begin{enumerate}[(a)]
\item 
	\[
	(4 - i) - (6 - 10i)= 4 - i - 6 + 10i= -2 + 9i
	\] \pspace

\item 
	\[
	(1 - 3i)(2 + 4i)= 2 + 4i - 6i - 12i^2= 2 + 4i - 6i - 12(-1)= 2 + 4i - 6i + 12= 14 - 2i
	\] \pspace

\item 
	\[
	(2i)^3= 2i \cdot 2i \cdot 2i= 4i^2 \cdot 2i= 4(-1) \cdot 2i= -4 \cdot 2i= -8i
	\] \pspace

\item 
	\[
	\begin{aligned}
	\dfrac{5 + i}{1 - 2i}&= \dfrac{5 + i}{1 - 2i} \cdot \dfrac{1 + 2i}{1 + 2i} \\[0.3cm]
	&= \dfrac{5 + 10i + i + 2i^2}{1 + 2i - 2i - 4i^2} \\[0.3cm]
	&= \dfrac{5 + 11i + 2(-1)}{1 - 4(-1)} \\[0.3cm]
	&= \dfrac{5 + 11i - 2}{1 + 4} \\[0.3cm]
	&= \dfrac{3 + 11i}{5} \\[0.3cm]
	&= \frac{3}{5} + \frac{11}{5}\,i
	\end{aligned}
	\]
\end{enumerate}


\end{document}