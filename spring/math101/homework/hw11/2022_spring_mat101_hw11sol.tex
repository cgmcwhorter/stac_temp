\documentclass[11pt,letterpaper]{article}
\usepackage[lmargin=1in,rmargin=1in,tmargin=1in,bmargin=1in]{geometry}
\usepackage{../style/homework}
\usepackage{../style/commands}
\setbool{quotetype}{true} % True: Side; False: Under
\setbool{hideans}{false} % Student: True; Instructor: False

% -------------------
% Content
% -------------------
\begin{document}

\homework{11: Due 04/12}{When you are dissatisfied and would like to go back to youth, think of Algebra}{Will Rogers}

% Problem 1
\problem{10} Determine whether $(x, y)= (5, -2)$ is a solution to the following system of equations:
	\[
	\begin{aligned}
	2x + 6y&= -2 \\
	3x - 2y&= 11
	\end{aligned}
	\] \pspace

\sol The point $(5, -2)$ is a solution to the system of equations if and only if it satisfies both of the equations. We check this:
	\[
	\begin{aligned}
	2x + 6y&= -2 \\
	2(5) + 6(-2)&\stackrel{?}{=} -2 \\
	10 - 12&\stackrel{?}{=} -2 \\
	-2&= -2 \\
	&\;\text{\cmark}
	\end{aligned}
	\]
and
	\[
	\begin{aligned}
	3x - 2y&= 11 \\
	3(5) - 2(-2)&\stackrel{?}{=} 11 \\
	15 + 4&\stackrel{?}{=} 11 \\
	19&= 11 \\
	&\;\text{\xmark}
	\end{aligned}
	\]
Therefore, $(5, -2)$ is \textit{not} a solution to the system of equations. 



\newpage



% Problem 2
\problem{10} Use substitution to solve the following system of equations:
	\[
	\begin{aligned}
	3x - 5y&= -29 \\
	2x - y&= -10
	\end{aligned}
	\] \pspace

\sol We solve for $y$ in the second equation:
	\[
	\begin{aligned}
	2x - y&= -10 \\[0.3cm]
	-y&= -2x - 10 \\[0.3cm]
	y&= 2x + 10
	\end{aligned}
	\]
Using this in the first equation, we have\dots
	\[
	\begin{aligned}
	3x - 5y&= -29 \\[0.3cm]
	3x - 5(2x + 10)&= -29 \\[0.3cm]
	3x - 10x - 50&= -29 \\[0.3cm]
	-7x&= 21 \\[0.3cm]
	x&= -3
	\end{aligned}
	\]
But then $y= 2x + 10= 2(-3) + 10= -6 + 10= 4$. Therefore, the solution is $(x, y)= (-3, 4)$. 



\newpage



% Problem 3
\problem{10} Use elimination to solve the following system of equations:
	\[
	\begin{aligned}
	5x + 6y&= 4 \\
	4x - 3y&= 11
	\end{aligned}
	\] \pspace

\sol Multiplying the second equation by 2, we have\dots
	\[
	\begin{aligned}
	5x + 6y&= 4 \\
	8x - 6y&= 22
	\end{aligned}
	\] 
Adding these equations, we have\dots
	\[
	\begin{aligned}
	13x&= 26 \\[0.3cm]
	x&= 2
	\end{aligned}
	\] 
Using this in the first equation, we have\dots
	\[
	\begin{aligned}
	5x + 6y&= 4 \\[0.3cm]
	5(2) + 6y&= 4 \\[0.3cm]
	10 + 6y&= 4 \\[0.3cm]
	6y&= -6 \\[0.3cm]
	y&= -1
	\end{aligned}
	\] 
Therefore, we have solution $(x, y)= (2, -1)$. 



\newpage



% Problem 4
\problem{10} Determine whether the following system of equations has a solution:
	\[
	\begin{aligned}
	-2x + 3y&= 5 \\
	4x - 6y&= -30
	\end{aligned}
	\] \pspace

\sol The system of linear equations has a solution if and only if the lines are not parallel. We solve for $y$ in both equations to determine the slopes. We have
	\[
	\begin{aligned}
	-2x + 3y&= 5 \\[0.3cm]
	3y&= 2x + 5 \\[0.3cm]
	y&= \dfrac{2}{3}\,x + \dfrac{5}{3}
	\end{aligned}
	\]
This line has slope $m_1= \dfrac{2}{3}$. We also have\dots
	\[
	\begin{aligned}
	4x - 6y&= -30 \\[0.3cm]
	-6y&= -4x - 30 \\[0.3cm]
	y&= \dfrac{2}{3}\,x + 5
	\end{aligned}
	\]
This lines has slope $m_2= \dfrac{2}{3}$. Because $m_1= m_2$, the lines are parallel. [Observe also that the lines are distinct.] Therefore, the system of equations has no solution.


\end{document}