\documentclass[11pt,letterpaper]{article}
\usepackage[lmargin=1in,rmargin=1in,tmargin=1in,bmargin=1in]{geometry}
\usepackage{../style/homework}
\usepackage{../style/commands}
\setbool{quotetype}{true} % True: Side; False: Under
\setbool{hideans}{false} % Student: True; Instructor: False

% -------------------
% Content
% -------------------
\begin{document}

\homework{12: Due 04/28}{You must do the things you think you cannot do.}{Eleanor Roosevelt}

% Problem 1
\problem{10} Consider the function $f(x)= -5 \left( \dfrac{4}{9} \right)^x$. 
	\begin{enumerate}[(a)]
	\item Is this function exponential? Explain. If it is exponential, find $A$, $b$, and $c$.
	\item Find $f(-2)$. 
	\item Find the $x$ and $y$-intercepts for $f(x)$. If there are none, state so. 
	\end{enumerate} \pspace

\sol
\begin{enumerate}[(a)]
\item The function $f(x)$ has the form $y= Ab^{cx}$ with $A= -5$, $b= \frac{4}{9}$, and $c= 1$. Therefore, $f(x)$ is an exponential function. 

\item We have\dots
	\[
	f(-2)= -5 \left( \dfrac{4}{9} \right)^{-2}= -5 \left( \dfrac{9}{4} \right)^2= -5 \left( \dfrac{81}{16} \right)= -\dfrac{405}{16}
	\]

\item The $x$-intercept(s) occurs when $f(x)= 0$. But then 
	\[
	\begin{aligned}
	-5 \left( \dfrac{4}{9} \right)^x&= 0 \\
	\left( \dfrac{4}{9} \right)^x&= 0
	\end{aligned}
	\]
But $\left( \frac{4}{9} \right)^x > 0$ for all $x$. Therefore, there are no $x$-intercepts. The $y$-intercept occurs when $x= 0$. But then we have\dots
	\[
	f(0)= -5 \left( \dfrac{4}{9} \right)^0= -5 \cdot 1= -5 
	\]
Therefore, the $y$-intercept is $-5$, i.e. the point $(0, -5)$. 
\end{enumerate}



\newpage



% Problem 2
\problem{10} Determine whether the following exponential functions are increasing or decreasing. Explain your answer for each.
	\begin{enumerate}[(a)]
	\item $y= 5(0.3)^x$
	\item $f(x)= -6(7^x)$
	\item $r= 9 \left(\dfrac{3}{2} \right)^{-2t}$
	\item $g(x)= -7 \left( \dfrac{12}{11} \right)^{x/2}$
	\end{enumerate} \pspace

\sol
\begin{enumerate}[(a)]
\item We have $A= 5 > 0$, $b= 0.3 < 1$, and $c= 1 > 0$. Therefore, $y$ is decreasing. 

\item We have $A= -6 < 0$, $b= 7 > 1$, and $c= 1 > 0$. Therefore, $f(x)$ is decreasing. 

\item We have $A= 9 > 0$, $b= \frac{3}{2} > 1$, and $c= -2 < 0$. Therefore, $r$ is decreasing.

\item We have $A= -7 < 0$, $b= \frac{12}{11} > 1$, and $c= \frac{1}{2} > 0$. Therefore, $g(x)$ is decreasing. 
\end{enumerate}



\newpage



% Problem 3
\problem{10} Write the following functions in the form $y= Ab^x$:
	\begin{enumerate}[(a)]
	\item $y= 11(2^{3x})$
	\item $y= -8 \left( \dfrac{7}{3} \right)^{-x}$
	\item $y= 6 (7^{2x + 1})$
	\end{enumerate} \pspace

\sol
\begin{enumerate}[(a)]
\item 
	\[
	y= 11(2^{3x})= 11 \big( (2^3)^x \big)= 11 (8^x)
	\] \pspace

\item 
	\[
	y= -8 \left( \dfrac{7}{3} \right)^{-x}= y= -8 \left( \left( \dfrac{7}{3} \right)^{-1} \right)^x= -8 \left( \dfrac{3}{7} \right)^x
	\] \pspace

\item 
	\[
	y= 6(7^{2x + 1})= 6(7^{2x} \cdot 7^1)= 42(7^{2x})= 42 (7^2)^x= 42(49^x)
	\]
\end{enumerate}



\newpage



% Problem 4
\problem{10} Solve the following exponential equations:
	\begin{enumerate}[(a)]
	\item $4^{3-x}= \dfrac{1}{64}$
	\item $5(3^x) + 7= 52$
	\item $16^{2x}= 4^{8x - 1}$
	\end{enumerate} \pspace

\sol
\begin{enumerate}[(a)]
\item 
	\[
	\begin{aligned}
	4^{3-x}&= \dfrac{1}{64} \\[0.3cm]
	4^{3-x}&= \dfrac{1}{4^3} \\[0.3cm]
	4^{3-x}&= 4^{-3} \\[0.3cm]
	3 - x&= -3 \\[0.3cm]
	x&= 6
	\end{aligned}
	\] \pspace

\item 
	\[
	\begin{aligned}
	5(3^x) + 7&= 52 \\[0.3cm]
	5(3^x)&= 45 \\[0.3cm]
	3^x&= 9 \\[0.3cm]
	3^x&= 3^2 \\[0.3cm]
	x&= 2
	\end{aligned}
	\] \pspace

\item 
	\[
	\begin{aligned}
	16^{2x}&= 4^{8x - 1} \\[0.3cm]
	(4^2)^{2x}&= 4^{8x - 1} \\[0.3cm]
	4^{4x}&= 4^{8x - 1} \\[0.3cm]
	4x&= 8x - 1 \\[0.3cm]
	4x&= 1 \\[0.3cm]
	x&= \dfrac{1}{4}
	\end{aligned}
	\]
\end{enumerate}


\end{document}