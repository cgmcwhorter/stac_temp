\documentclass[11pt,letterpaper]{article}
\usepackage[lmargin=1in,rmargin=1in,bmargin=1in,tmargin=1in]{geometry}
\usepackage{style/quiz}
\usepackage{style/commands}

% -------------------
% Content
% -------------------
\begin{document}
\thispagestyle{title}

% Quiz 1
\quizsol \textit{True/False}: A real number is `any' number expressible as a decimal. \pspace

\sol The statement is \textit{true}. A real number is `any' number that can be expressed as a decimal---even if it is not immediately given as such, e.g. $1= 1.0$, $-5= -5.0$, $\frac{1}{2}= 0.5$, $\sqrt{2} \approx 1.41421\ldots$, $\pi \approx 3.14159\ldots$, etc. \pvspace{1.5cm}



% Quiz 2
\quizsol \textit{True/False}: $\gcd(2^3 \cdot 3^1, 2 \cdot 3^2 \cdot 5^2)= 2^3 \cdot 3^2 \cdot 5^2$ and $\lcm(2^3 \cdot 3^1, 2 \cdot 3^2 \cdot 5^2)= 2 \cdot 3$. \pspace

\sol The statement is \textit{false}. It is `obvious' that these cannot be correct because the gcd is greater than the given integers and the lcm is smaller than the given integers---both of which is impossible as we know that $\gcd(a, b) \leq \min\{a, b\}$ and $\lcm(a, b) \geq \max\{a, b\}$. Remember given a prime factorization of the numbers, we find the gcd by choosing the \textit{smallest} powers of each prime that appears in \textit{both} of the factorizations. We find the lcm by choosing the \textit{largest} powers of each prime that appears in the factorizations. Here, the gcd and lcm were computed reversing these rules. We should have $\gcd(2^3 \cdot 3^1, 2 \cdot 3^2 \cdot 5^2)= 2 \cdot 3$ and $\lcm(2^3 \cdot 3^1, 2 \cdot 3^2 \cdot 5^2)= 2^3 \cdot 3^2 \cdot 5^2$. \pvspace{1.5cm}



% Quiz 3
\quizsol \textit{True/False}: $\dfrac{7}{5}$ divided by $\dfrac{21}{10}$ is $\dfrac{2}{3}$. \pspace

\sol The statement is \textit{true}. Note that division by a nonzero number is the same as multiplying by its reciprocal. So we have
	\[
	\dfrac{\phantom{-}\dfrac{3}{10}\phantom{-}}{\dfrac{12}{5}}= \dfrac{3}{10} \cdot \dfrac{5}{12}= \dfrac{\cancel{3}^1}{\cancel{10}^{\,2}} \cdot \dfrac{\cancel{5}^1}{\cancel{12}^{\,4}}= \dfrac{1}{8}
	\]
One can also rewrite the problem as\dots 
	\[
	\dfrac{\phantom{-}\dfrac{3}{10}\phantom{-}}{\dfrac{12}{5}}= \dfrac{3}{10} \div \dfrac{12}{5}
	\]
But then to divide, we multiply by the reciprocal and proceed as in the solution above. \pvspace{1.5cm}



% Quiz 4
\quizsol \textit{True/False}: $\left( \dfrac{x^8}{y^3} \right)^{-1/2}= \dfrac{\sqrt{y^3}}{x^4}$ \pspace

\sol The statement is \textit{true}. Observe that\dots
	\[
	\left( \dfrac{x^8}{y^3} \right)^{-1/2}= \left( \dfrac{y^3}{x^8} \right)^{1/2}= \dfrac{y^{3/2}}{x^{8/2}}= \dfrac{\sqrt{y^3}}{x^4}
	\]



\newpage



% Quiz 5
\quizsol \textit{True/False}: The decimal number 14.7 in scientific notation is $1.47 \cdot 10^{-1}$. \pspace

\sol The statement is \textit{false}. We can simply compute $1.47 \cdot 10^{-1}$: $1.47 \cdot 10^{-1}= 1.47 \cdot \frac{1}{10}= 0.147 \neq 14.7$. Alternatively, to write 14.7 in scientific notation, we need to place the decimal between the 1 and 4. This would result in 1.47. To shift the decimal back to its proper place, we would have to move the decimal place one spot to the right, i.e. multiply by 10. Therefore, 14.7 in scientific notation is $1.47 \cdot 10^1$. \pvspace{1.5cm}



% Quiz 6
\quizsol \textit{True/False}: To increase 45.6 by 162\%, you compute $45.6(1.62)$. \pspace

\sol The statement is \textit{false}. The statement is \textit{false}. To find 162\% of 45.6, we would multiply 45.6 by the percent written as a decimal. This would be $45.6(1.62)$. However, to increase or decrease a number by a percentage, we compute the number $\#(1 \pm \%)$, where we add if we are increasing, subtract if we are decreasing, $\#$ is the number, and \% is the percentage written as a decimal. So to increase 45.6 by 162\%, we need to compute $45.6(1 + 1.62)= 45.6(2.62)$. \pvspace{1.5cm}



% Quiz 7
\quizsol \textit{True/False}: If $f(x)= 2x + 3$ and $g(x)= 1 - x$, then $(f \circ g)(1)= 0$. \pspace

\sol The statement is \textit{false}. We know that $(f \circ g)(1)= f(g(1))$. First, we need to compute $g(1)$: $g(1)= 1 - 1= 0$. But then we have $f(g(1))= f(0)$. So now we compute $f(0)$: $f(0)= 2(0) + 3= 0 + 3= 3$. Therefore, $(f \circ g)(1)= 3$. All together:
	\[
	(f \circ g)(1)= f(g(1))= f(0)= 3
	\] \pvspace{1.5cm}



% Quiz 8
\quizsol \textit{True/False}: The function $y= 5 - x$ is linear and has slope 1. \pspace

\sol The statement is \textit{false}. A linear function has the form $y= mx + b$, where $m$ is the slope and $b$ is the $y$-intercept (really $(0, b)$). This function has this form with $m= -1$ and $b= 5$. But then the slope is $-1$ and not 1. \pvspace{1.3cm}



% Quiz 9
\quizsol \textit{True/False}: `Any' function which changes at a constant rate is a linear function. \pspace

\sol The statement is \textit{true}. Suppose the rate of change were 5 and the current value is 2. After one step in time, the value is $2 + 1(5)= 2 + 5= 7$. After another step in time, the value is $7 + 5= 12$, or $2 + 2(5)= 2 + 10= 12$. Generally, after $n$ steps, the value is $2 + n \cdot 5= 5n + 2$, which is a linear function. Generally, if we start with initial value $y_0$ and have a constant rate of change $m$, after $x$ steps, we have $y= y_0 + x \cdot m= mx + y_0$. This is a linear function with $y= y$, $x= x$, $m= m$, and $b= y_0$. But then we see that `any' function which changes at a constant rate is a linear function. We know that a linear function $y= mx + b$ has a constant rate of change---the slope $m$. Therefore, a function is linear if and only if it has a constant rate of change. 



\newpage



% Quiz 10
\quizsol \textit{True/False}: Suppose Lolita's bank account currently has \$5576.28. If she makes \$22.50 per hour, works 40~hours per week, and deposits her paycheck to her bank account each week, then the amount of money in her account, $A(t)$, in $t$ weeks is given by $A(t)= 22.50t + 5576.28$. \pspace

\sol The statement is \textit{false}. This is an issue of units. Because money is flowing into the account at a constant rate, we know that $A(t)$ is linear. Therefore, $A(t)= mt + b$ for some $m, b$. Because the account has \$5576.28 at the start, we know that $A(0)= 5576.28$ so that $5576.28= A(0)= m(0) + b= b$. We then know that $A(t)= mt + 5576.28$. Furthermore, $m$ is the rate of change of the total money in the account. Because $t$ is in weeks, this is Lolita's weekly income. This is $22.50 \cdot 40= 900$. Therefore, $A(t)= 900t + 5576.28$. The given would be $A(t)$ if $t$ was given in days---not weeks. \pvspace{1.3cm}



% Quiz 11
\quizsol \textit{True/False}: The parabola $y= 17 - (x + 2)^2$ has vertex $(2, 17)$. \pspace

\sol The statement is \textit{false}. The vertex form of a quadratic function is $f(x)= a(x - p)^2 + q$, where $(p, q)$ is the location of the vertex. We have $y= 17 - (x + 2)^2= -(x + 2)^2 + 17= -(x - (-2))^2 + 17$ so that $a= -1$, $p= -2$, and $q= 17$. Therefore, the vertex of the quadratic function is the point $(p, q)$. Generally, suppose a quadratic function is written in the form $y= a(x \pm p)^2 + q$. The $y$-coordinate of the vertex is the constant term and the $x$-coordinate of the vertex is the $x$-value for which the quadratic term vanishes. So for $y= 17 - (x+ 2)^2$, the constant term is 17 and the $x$-value that makes the quadratic term vanish, i.e. equal 0, is $x= -2$. Therefore, the vertex is $(-2, 17)$. \pvspace{1.3cm}



% Quiz 12
\quizsol \textit{True/False}: To solve $x^2= 48 - 2x$, we write $x^2 + 2x - 48= 0$ and factor the left side as $(x + 8)(x - 6)$ so that $x= -8, 6$. \pspace

\sol The statement is \textit{true}. We have\dots
	\[
	\begin{aligned}
	x^2&= 48 - 2x \\
	x^2 + 2x - 48&= 0 \\
	(x + 8)(x - 6)&= 0 
	\end{aligned}
	\]
But then either $x + 8= 0$, so that $x= -8$, or $x - 6=0$, so that $x= 6$. Therefore, the solutions are $x= -8, 6$. 








\end{document}