\documentclass[12pt,letterpaper]{exam}
\usepackage[lmargin=1in,rmargin=1in,tmargin=1in,bmargin=1in]{geometry}
\usepackage{../style/exams}

% -------------------
% Course & Exam Information
% -------------------
\newcommand{\course}{MAT 108: Exam 1}
\newcommand{\term}{Spring -- 2021}
\newcommand{\examdate}{02/23/2022}
\newcommand{\timelimit}{85 Minutes}

\setbool{hideans}{true} % Student: True; Instructor: False

% -------------------
% Content
% -------------------
\begin{document}

\examtitle
\instructions{Write your name on the appropriate line on the exam cover sheet. This exam contains \numpages\ pages (including this cover page) and \numquestions\ questions. Check that you have every page of the exam. Answer the questions in the spaces provided on the question sheets. Be sure to answer every part of each question and show all your work. If you run out of room for an answer, continue on the back of the page --- being sure to indicate the problem number.} 
\scores
\bottomline
\newpage

% ---------
% Questions
% ---------
\begin{questions}

% Question 1
\newpage
\question Showing all your work, answer the following: \pspace
\begin{parts}
\part[5] A. Pearson goes to a restaurant with her friends for her 26th birthday. Her portion of the bill comes to \$27.60. She plans on leaving a 26\% tip (in honor of her birthday). What is her total bill? \vfill

\part[5] Count Dyss offers tours of his historic landmark castle for \$76.50 per person. For the Halloween weekend, he plans on offering a 25\% discount on tours. What is the amount he will charge per person for the Halloween weekend? \vfill
\end{parts}



% Question 2
\newpage
\question Professor Oak is a dendrologist---a scientist who studies woody plants, i.e. trees. He finds that for a certain type of tree, the age of the tree can be closely modeled by the number of rings, $r$, the tree has. He determines that the age of the tree, $A(r)$, in years is approximately given by $A(r)= 15.3r - 6.0$. \pspace
	\begin{parts}
	\part[2] Is $A(r)$ linear? Explain. \pvspace{3cm}
	\part[3] Determine the slope of $A(r)$. Interpret the slope in context. \vfill
	\part[3] Determine the $y$-intercept of $A(r)$. Does the $y$-intercept have an interpretation in context? Explain why or why not. \vfill
	\part[2] For this species of tree, approximate the age of a tree with 11~rings. \vfill
	\end{parts}



% Question 3
\newpage
\question Florist Gump is a floral shop famous for its bouquets. The shop purchases white roses in bulk from their distributor at a price of \$1.42 per rose. The delivery fee for the roses is \$52.50. The shop sells the roses for \$4.32 per rose. \pspace
	\begin{parts}
	\part[2] Find the revenue function. \vfill
	\part[2] Find the cost function. \vfill
	\part[3] Find the profit function. \vfill
	\part[3] Determine minimal number of white roses the store needs to sell in order to make a profit on their white rose bouquets. \pvspace{1cm}\vfill
	\end{parts}



% Question 4
\newpage
\question A company sells a certain product with associated revenue function $R(x)= 0.14x^2$ and cost function $C(x)= 16x + 530$. \pspace
	\begin{parts}
	\part[3] Find the fixed costs for this product. \vfill
	\part[4] Find revenue and cost associated to selling and producing 10 units. Is the company making a profit at this level of sales/production? Explain. \vfill
	\part[3] Find the marginal revenue at a production level of 10~units. \vfill
	\end{parts}



% Question 5
\newpage
\question[10] Consider the following system of equations:
	\[
	\begin{aligned}
	2x - y + 2z&= 28 \\
	-x + 5y + 4z&= -5 \\
	x + 6y + 4z&= -4 \\
	7x + 2y - 2z&= 7
	\end{aligned}
	\]
When the augmented matrix for this system of equations is placed in `reduced-row echelon form', we obtain the following matrix:
	\[
	\begin{pmatrix}
	1 & 0 & 0 & 3 \\
	0 & 1 & 0 & -5 \\
	0 & 0 & 1 & 6 \\
	0 & 0 & 0 & 1
	\end{pmatrix}	
	\]
Are there any solutions to the given system of equations? If there are no solutions, explain why. If there are solutions, determine all the solutions and if there is more than one solution, give at least one explicit solution to the system of equations.  



% Question 6
\newpage
\question[10] Consider the following system of equations:
	\[
	\begin{aligned}
	3x_1 + 2x_2 - 5x_3 + 2x_4&= -17 \\
	x_1 - 2x_2 + 3x_3 + 2x_4&= 27 \\
	3x_1 - 2x_2 + x_3 + 8x_4&= 36 \\
	-5x_1 + 4x_2 + 2x_3 + 4x_4&= 5
	\end{aligned}
	\]
When the augmented matrix for this system of equations is placed in reduced-row echelon form, we obtain the following matrix:
	\[
	\begin{pmatrix}
	1 & 0 & 0 & 0 & 3 \\
	0 & 1 & 0 & 0 & -\frac{1}{2} \\
	0 & 0 & 1 & 0 & 6 \\
	0 & 0 & 0 & 1 & \frac{5}{2}
	\end{pmatrix}	
	\]
Are there any solutions to the given system of equations? If there are no solutions, explain why. If there are solutions, determine all the solutions and if there is more than one solution, give at least one explicit solution to the system of equations.  



% Question 7
\newpage
\question[10] Consider the following system of equations:
	\[
	\begin{aligned}
	4x - 8y - 5z + 12w&= 15 \\
	x - 2y - z + 3w&= 5 \\
	2x - 4y - z + 6w&= 15
	\end{aligned}
	\]
When the augmented matrix for this system of equations is placed in reduced-row echelon form, we obtain the following matrix:
	\[
	\begin{pmatrix}
	1 & -2 & 0 & 3 & 10 \\
	0 & 0 & 1 & 0 & 5 \\
	0 & 0 & 0 & 0 & 0 
	\end{pmatrix}
	\]
Are there any solutions to the given system of equations? If there are no solutions, explain why. If there are solutions, determine all the solutions and if there is more than one solution, give at least one explicit solution to the system of equations.  



% Question 8
\newpage
\question[10] G. Jordan is studying for his MATH~108 exam. He is putting an augmented matrix in reduced-row echelon form. He is one step away from putting the matrix in row-echelon form. Currently, he has the following matrix:
	\[
	\begin{pmatrix}
	1 & -2 & 1 & 0 \\
	0 & 1 & 3 & -2 \\
	0 & 3 & 10 & -7
	\end{pmatrix}
	\]
Put the augmented matrix in row-echelon form by performing the next step in placing the matrix in reduced-row echelon form. Then show Jordan a faster way of getting to the solution by using the `shortcut method' of determining the solution from the row-echelon form. 



% Question 9
\newpage
\question Compute the following:
	\begin{parts}
	\part[5]
		\[
		\begin{pmatrix} 2 & -3 \\ 5 & 1 \end{pmatrix} - 2 \begin{pmatrix} -2 & 0 \\ 1 & -3 \end{pmatrix}
		\] \vfill
	
	\part[5]
		\[
		\begin{pmatrix}
		1 & 0 \\
		0 & -1 \\
		2 & 2 \\
		0 & 3
		\end{pmatrix}
		\begin{pmatrix}
		5 & 1 & 2 \\
		-1 & 0 & 1 
		\end{pmatrix}
		\] \vspace{3cm}\vfill
	\end{parts}



% Question 10
\newpage
\question Consider the following system of equations:
	\[
	\begin{aligned}
	2x - 7y&= 3 \\
	x + 3y&= 1
	\end{aligned}
	\]

\begin{parts}
\part[1] Find the coefficient matrix. \pvspace{3cm}
\part[3] Show that the coefficient matrix is invertible, i.e. has an inverse. \pvspace{4cm}
\part[3] Find the inverse of the coefficient matrix. \pvspace{4cm}
\part[3] Writing the system of equations in vector form, find the solution to the given system of equations using the inverse matrix from (b). \vfill
\end{parts}


\end{questions}
\end{document}