\documentclass[11pt,letterpaper]{article}
\usepackage[lmargin=1in,rmargin=1in,bmargin=1in,tmargin=1in]{geometry}
\usepackage{style/quiz}
\usepackage{style/commands}

% -------------------
% Content
% -------------------
\begin{document}
\thispagestyle{title}


% Quiz 1
\quizsol \textit{True/False}: If you had a bill of \$25.77 and were going to pay a tip of 20\%, the total amount you would pay could be computed by finding $25.77(1.20)$. \pspace

\sol The statement is \textit{true}. Recall to calculate a percentage of a number $N$, we compute $N \cdot \%$, where $N$ is the number and $\%$ is the percentage (written as a decimal). For instance, to compute 57\% of 23, we compute $23(0.57)= 13.11$. To compute 172\% of 150, we compute $150(1.72)= 258$. However, to compute a $\%$ percent increase or decrease of a number $N$, we compute $N(1 \pm \%)$, where $N$ is the number, $\%$ is the percentage as a decimal, and we choose plus for increase and negative for decrease. For instance, to compute a 75\% decrease of 13, we compute $13(1 - 0.75)= 13(0.25)= 3.25$. To compute a 115\% increase of 120, we compute $120(1 + 1.15)= 120(2.15)= 258$. Here, we are increasing 25.77 by 20\%, so we compute $25.77(1 + 0.20)= 25.77(1.20)$. \pvspace{1.5cm}



% Quiz 2
\quizsol \textit{True/False}: The amount of concrete in tons, $C$, used to repair $r$ roads remaining in a storage facility is given by $C(r)= 450.7 - 16.3r$. Because this function is linear, we can interpret the slope of $C(r)$ as saying that each road uses approximately $16.3$~tons of concrete to repair. \pspace

\sol The statement is \textit{true}. The slope of the linear function $C(r)= 450.7 - 16.3r$ is\dots
	\[
	m= -16.3= -\dfrac{16.3}{1}= \dfrac{-16.3}{1}
	\]
Thinking of this slope as $\frac{\Delta \text{output}}{\Delta \text{input}}$, we can see that for each one increase in $r$, i.e. one additional road, there is a decrease by $16.3$~tons in the amount of concrete remaining. Therefore, we can summarize this as that each road requires approximately $16.3$~tons of concrete to repair. \pvspace{1.5cm}



% Quiz 3
\quizsol \textit{True/False}: A company sells a product for $\$5.75$ per item. Each item costs approximately $\$1.37$ to manufacture and is produced in a machine that costs $\$87.50$ to operate. Given this data, we have $R(x)= 5.75$ and $C(x)= (1.37 + 87.50)x= 88.88x$. \pspace

\sol The statement is \textit{false}. If one sells $x$~items, the revenue is $R(x)= 5.75 \cdot 7= 5.75x$. Therefore, $R(x)$ is correct. However, we know that $C(x)= \text{VC} + \text{FC}$. The fixed costs are the machine operation costs, i.e. $\text{FC}= \$87.50$. The variable costs are the $\$1.37$ cost per item. If $x$ items are produced, then the manufacture costs are $\text{VC}= 1.37 \cdot x= 1.37x$. Therefore, $C(x)= \text{VC} + \text{FC}= 1.37x + 87.50$. 



% Quiz 4
\quizsol \textit{True/False}: If the following matrix represents an augmented matrix in RREF, then the corresponding system has solution $x_1= 5$, $x_2= -3$, and $x_3= 7$.
	\[
	\begin{pmatrix}
	1 & 0 & 0 & 5 \\
	0 & 1 & 0 & -3 \\
	0 & 0 & 1 & 7 \\
	0 & 0 & 0 & 1
	\end{pmatrix}
	\]



%% Quiz 
%\quizsol \textit{True/False}:



\end{document}