\documentclass[11pt,letterpaper]{article}
\usepackage[lmargin=1in,rmargin=1in,bmargin=1in,tmargin=1in]{geometry}
\usepackage{style/quiz}
\usepackage{style/commands}

% -------------------
% Content
% -------------------
\begin{document}
\thispagestyle{title}


% Quiz 1
\quizsol \textit{True/False}: If you had a bill of \$25.77 and were going to pay a tip of 20\%, the total amount you would pay could be computed by finding $25.77(1.20)$. \pspace

\sol The statement is \textit{true}. Recall to calculate a percentage of a number $N$, we compute $N \cdot \%$, where $N$ is the number and $\%$ is the percentage (written as a decimal). For instance, to compute 57\% of 23, we compute $23(0.57)= 13.11$. To compute 172\% of 150, we compute $150(1.72)= 258$. However, to compute a $\%$ percent increase or decrease of a number $N$, we compute $N(1 \pm \%)$, where $N$ is the number, $\%$ is the percentage as a decimal, and we choose plus for increase and negative for decrease. For instance, to compute a 75\% decrease of 13, we compute $13(1 - 0.75)= 13(0.25)= 3.25$. To compute a 115\% increase of 120, we compute $120(1 + 1.15)= 120(2.15)= 258$. Here, we are increasing 25.77 by 20\%, so we compute $25.77(1 + 0.20)= 25.77(1.20)$. \pvspace{0.5cm}



% Quiz 2
\quizsol \textit{True/False}: The amount of concrete in tons, $C$, used to repair $r$ roads remaining in a storage facility is given by $C(r)= 450.7 - 16.3r$. Because this function is linear, we can interpret the slope of $C(r)$ as saying that each road uses approximately $16.3$~tons of concrete to repair. \pspace

\sol The statement is \textit{true}. The slope of the linear function $C(r)= 450.7 - 16.3r$ is\dots
	\[
	m= -16.3= -\dfrac{16.3}{1}= \dfrac{-16.3}{1}
	\]
Thinking of this slope as $\frac{\Delta \text{output}}{\Delta \text{input}}$, we can see that for each one increase in $r$, i.e. one additional road, there is a decrease by $16.3$~tons in the amount of concrete remaining. Therefore, we can summarize this as that each road requires approximately $16.3$~tons of concrete to repair. \pvspace{0.5cm}



% Quiz 3
\quizsol \textit{True/False}: A company sells a product for $\$5.75$ per item. Each item costs approximately $\$1.37$ to manufacture and is produced in a machine that costs $\$87.50$ to operate. Given this data, we have $R(x)= 5.75$ and $C(x)= (1.37 + 87.50)x= 88.88x$. \pspace

\sol The statement is \textit{false}. If one sells $x$~items, the revenue is $R(x)= 5.75 \cdot 7= 5.75x$. Therefore, $R(x)$ is correct. However, we know that $C(x)= \text{VC} + \text{FC}$. The fixed costs are the machine operation costs, i.e. $\text{FC}= \$87.50$. The variable costs are the $\$1.37$ cost per item. If $x$ items are produced, then the manufacture costs are $\text{VC}= 1.37 \cdot x= 1.37x$. Therefore, $C(x)= \text{VC} + \text{FC}= 1.37x + 87.50$. \pvspace{0.5cm}



% Quiz 4
\quizsol \textit{True/False}: If the following matrix represents an augmented matrix in RREF, then the corresponding system has solution $x_1= 5$, $x_2= -3$, and $x_3= 7$.
	\[
	\begin{pmatrix}
	1 & 0 & 0 & 5 \\
	0 & 1 & 0 & -3 \\
	0 & 0 & 1 & 7 \\
	0 & 0 & 0 & 1
	\end{pmatrix}
	\]

\sol The statement is \textit{false}. Examining the equation corresponding to the last row, we see that $0= 1$, which is impossible. Therefore, the original system of equations was inconsistent. But then the original system of equations has no solution. 



\newpage



% Quiz 5
\quizsol \textit{True/False}: You can perform the following multiplication:
	\[
	\begin{pmatrix}
	1 & -1 & 0 & 5 & 3 \\
	0 & 4 & -2 & 6 & 1
	\end{pmatrix}
	\begin{pmatrix}
	3 & -2 \\
	3 & 8 \\
	4 & 0 \\
	2 & -1 \\
	0 & 5
	\end{pmatrix}
	\] \pspace

\sol The statement is \textit{true}. Recall that you can multiply a $m \times n$ matrix with a $p \times q$ matrix if $n= p$. If so, you obtain a $m \times q$ matrix. The first matrix is $2 \times 5$ while the second matrix is $5 \times 2$. But because $5= 5$, we can multiply these matrix to obtain a $2 \times 2$ matrix. One can check that the product is\dots
	\[
	\begin{pmatrix}
	10 & 0 \\
	16 & 31
	\end{pmatrix}
	\] \pvspace{1.5cm}



% Quiz 6 
\quizsol \textit{True/False}: The matrix $\begin{pmatrix} -2 & 8 \\ -2 & 6 \end{pmatrix}$ has an inverse. \pspace

\sol The statement is \textit{true}. Recall that a matrix has an inverse if and only if the determinant of the matrix is \textit{not} zero. We have\dots
	\[
	\begin{pmatrix} -2 & 8 \\ -2 & 6 \end{pmatrix}= -2(6) - 8(-2)= -12 + 16= 4 \neq 0
	\]
Therefore, the matrix is invertible. Recalling that if $A$ is a $2 \times 2$ matrix (given below) that is invertible, we have\dots
	\[
	\begin{aligned}
	A&= 
	\begin{pmatrix}
	a & b \\ 
	c & d
	\end{pmatrix} \\
	A^{-1}&= \dfrac{1}{\det A}
	\begin{pmatrix}
	d & -b \\
	-c & a 
	\end{pmatrix}
	\end{aligned}
	\]
Therefore, 
	\[
	\begin{pmatrix} -2 & 8 \\ -2 & 6 \end{pmatrix}^{-1}= \dfrac{1}{4} \begin{pmatrix} 6 & -8 \\ 2 & -2 \end{pmatrix}= \begin{pmatrix} \dfrac{3}{2} & -2 \\[0.3cm] \dfrac{1}{2} & -\dfrac{1}{2} \end{pmatrix}
	\]



\newpage



% Quiz 7
\quizsol \textit{True/False}: The point $(1, -3)$ satisfies the following system of inequalities:
	\[
	\begin{aligned}
	x + y&\leq 0 \\
	x - 2y&\leq 5
	\end{aligned}
	\]

\sol The statement is \textit{false}. If a point satisfies a system of inequalities, it satisfies each of the inequalities individually---which we can check:
	\[
	\begin{aligned}
	x + y&\leq 0 && \qquad & x - 2y&\stackrel{?}{\leq} 5 \\
	1 + (-3)&\stackrel{?}{\leq} 0 && \qquad & 1 - 2(-3)&\stackrel{?}{\leq} 5 \\
	-2&\leq 0 \text{ \cmark} && \qquad & 1 + 6&\stackrel{?}{\leq} 5 \\
	&				    && \qquad & 7 &\not\leq 5 \text{ \xmark}
	\end{aligned}
	\]
Because $(1, -3)$ does not satisfy all the inequalities, $(1, -3)$ does not satisfy the system of inequalities. \pvspace{1.5cm}





To maximize $z= 5x + 6y$ subject to $2x + 3y \leq 6$, $-6x + y \leq 20$, and $x, y \leq 0$, the initial simplex tableau is\dots
	\begin{center}
	\begin{tabular}{rrrr|r}
	$2$ & $3$ & $1$ & $0$ & $6$ \\
	$-6$ & $1$ & $0$ & $1$ & $20$ \\ \hline
	$-5$ & $-6$ & $0$ & $0$ & $0$
	\end{tabular}
	\end{center}


Given the following minimization problem:
	\[
	\begin{aligned}
	\min w= x_1 + 2x_2 + 3x_3 \\
	x_1 + x_2 + x_3 \geq 4 \\
	x_1 - x_2 + x_3 \geq 6 \\
	-x_1 + x_2 - x_3 \geq 8
	\end{aligned}
	\]
the dual problem is\dots
	\[
	\begin{pmatrix}
	1 & 1 & -1 & 1 \\
	1 & -1 & 1 & 2 \\
	1 & 1 & -1 & 3 \\
	4 & 6 & 8 & 0 
	\end{pmatrix}
	\]

If \$4,000 is placed into an account that earns 5\% interest, compounded quarterly, then the amount of money in the account after 8~years is\dots
	\[
	4000 \left(1 + \dfrac{0.05}{12} \right)^8
	\]
\end{document}