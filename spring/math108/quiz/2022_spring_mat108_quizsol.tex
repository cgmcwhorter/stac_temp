\documentclass[11pt,letterpaper]{article}
\usepackage[lmargin=1in,rmargin=1in,bmargin=1in,tmargin=1in]{geometry}
\usepackage{style/quiz}
\usepackage{style/commands}

% -------------------
% Content
% -------------------
\begin{document}
\thispagestyle{title}


% Quiz 1
\quizsol \textit{True/False}: If you had a bill of \$25.77 and were going to pay a tip of 20\%, the total amount you would pay could be computed by finding $25.77(1.20)$. \pspace

\sol The statement is \textit{true}. Recall to calculate a percentage of a number $N$, we compute $N \cdot \%$, where $N$ is the number and $\%$ is the percentage (written as a decimal). For instance, to compute 57\% of 23, we compute $23(0.57)= 13.11$. To compute 172\% of 150, we compute $150(1.72)= 258$. However, to compute a $\%$ percent increase or decrease of a number $N$, we compute $N(1 \pm \%)$, where $N$ is the number, $\%$ is the percentage as a decimal, and we choose plus for increase and negative for decrease. For instance, to compute a 75\% decrease of 13, we compute $13(1 - 0.75)= 13(0.25)= 3.25$. To compute a 115\% increase of 120, we compute $120(1 + 1.15)= 120(2.15)= 258$. Here, we are increasing 25.77 by 20\%, so we compute $25.77(1 + 0.20)= 25.77(1.20)$. \pvspace{0.5cm}



% Quiz 2
\quizsol \textit{True/False}: The amount of concrete in tons, $C$, used to repair $r$ roads remaining in a storage facility is given by $C(r)= 450.7 - 16.3r$. Because this function is linear, we can interpret the slope of $C(r)$ as saying that each road uses approximately $16.3$~tons of concrete to repair. \pspace

\sol The statement is \textit{true}. The slope of the linear function $C(r)= 450.7 - 16.3r$ is\dots
	\[
	m= -16.3= -\dfrac{16.3}{1}= \dfrac{-16.3}{1}
	\]
Thinking of this slope as $\frac{\Delta \text{output}}{\Delta \text{input}}$, we can see that for each one increase in $r$, i.e. one additional road, there is a decrease by $16.3$~tons in the amount of concrete remaining. Therefore, we can summarize this as that each road requires approximately $16.3$~tons of concrete to repair. \pvspace{0.5cm}



% Quiz 3
\quizsol \textit{True/False}: A company sells a product for $\$5.75$ per item. Each item costs approximately $\$1.37$ to manufacture and is produced in a machine that costs $\$87.50$ to operate. Given this data, we have $R(x)= 5.75$ and $C(x)= (1.37 + 87.50)x= 88.88x$. \pspace

\sol The statement is \textit{false}. If one sells $x$~items, the revenue is $R(x)= 5.75 \cdot 7= 5.75x$. Therefore, $R(x)$ is correct. However, we know that $C(x)= \text{VC} + \text{FC}$. The fixed costs are the machine operation costs, i.e. $\text{FC}= \$87.50$. The variable costs are the $\$1.37$ cost per item. If $x$ items are produced, then the manufacture costs are $\text{VC}= 1.37 \cdot x= 1.37x$. Therefore, $C(x)= \text{VC} + \text{FC}= 1.37x + 87.50$. \pvspace{0.5cm}



% Quiz 4
\quizsol \textit{True/False}: If the following matrix represents an augmented matrix in RREF, then the corresponding system has solution $x_1= 5$, $x_2= -3$, and $x_3= 7$.
	\[
	\begin{pmatrix}
	1 & 0 & 0 & 5 \\
	0 & 1 & 0 & -3 \\
	0 & 0 & 1 & 7 \\
	0 & 0 & 0 & 1
	\end{pmatrix}
	\]

\sol The statement is \textit{false}. Examining the equation corresponding to the last row, we see that $0= 1$, which is impossible. Therefore, the original system of equations was inconsistent. But then the original system of equations has no solution. 



\newpage



% Quiz 5
\quizsol \textit{True/False}: You can perform the following multiplication:
	\[
	\begin{pmatrix}
	1 & -1 & 0 & 5 & 3 \\
	0 & 4 & -2 & 6 & 1
	\end{pmatrix}
	\begin{pmatrix}
	3 & -2 \\
	3 & 8 \\
	4 & 0 \\
	2 & -1 \\
	0 & 5
	\end{pmatrix}
	\] \pspace

\sol The statement is \textit{true}. Recall that you can multiply a $m \times n$ matrix with a $p \times q$ matrix if $n= p$. If so, you obtain a $m \times q$ matrix. The first matrix is $2 \times 5$ while the second matrix is $5 \times 2$. But because $5= 5$, we can multiply these matrix to obtain a $2 \times 2$ matrix. One can check that the product is\dots
	\[
	\begin{pmatrix}
	10 & 0 \\
	16 & 31
	\end{pmatrix}
	\] \pvspace{1.5cm}



% Quiz 6 
\quizsol \textit{True/False}: The matrix $\begin{pmatrix} -2 & 8 \\ -2 & 6 \end{pmatrix}$ has an inverse. \pspace

\sol The statement is \textit{true}. Recall that a matrix has an inverse if and only if the determinant of the matrix is \textit{not} zero. We have\dots
	\[
	\begin{pmatrix} -2 & 8 \\ -2 & 6 \end{pmatrix}= -2(6) - 8(-2)= -12 + 16= 4 \neq 0
	\]
Therefore, the matrix is invertible. Recalling that if $A$ is a $2 \times 2$ matrix (given below) that is invertible, we have\dots
	\[
	\begin{aligned}
	A&= 
	\begin{pmatrix}
	a & b \\ 
	c & d
	\end{pmatrix} \\
	A^{-1}&= \dfrac{1}{\det A}
	\begin{pmatrix}
	d & -b \\
	-c & a 
	\end{pmatrix}
	\end{aligned}
	\]
Therefore, 
	\[
	\begin{pmatrix} -2 & 8 \\ -2 & 6 \end{pmatrix}^{-1}= \dfrac{1}{4} \begin{pmatrix} 6 & -8 \\ 2 & -2 \end{pmatrix}= \begin{pmatrix} \dfrac{3}{2} & -2 \\[0.3cm] \dfrac{1}{2} & -\dfrac{1}{2} \end{pmatrix}
	\]



\newpage



% Quiz 7
\quizsol \textit{True/False}: The point $(1, -3)$ satisfies the following system of inequalities:
	\[
	\begin{aligned}
	x + y&\leq 0 \\
	x - 2y&\leq 5
	\end{aligned}
	\]

\sol The statement is \textit{false}. If a point satisfies a system of inequalities, it satisfies each of the inequalities individually---which we can check:
	\[
	\begin{aligned}
	x + y&\leq 0 && \qquad & x - 2y&\stackrel{?}{\leq} 5 \\
	1 + (-3)&\stackrel{?}{\leq} 0 && \qquad & 1 - 2(-3)&\stackrel{?}{\leq} 5 \\
	-2&\leq 0 \text{ \cmark} && \qquad & 1 + 6&\stackrel{?}{\leq} 5 \\
	&				    && \qquad & 7 &\not\leq 5 \text{ \xmark}
	\end{aligned}
	\]
Because $(1, -3)$ does not satisfy all the inequalities, $(1, -3)$ does not satisfy the system of inequalities. \pvspace{1.5cm}




% Quiz 8
\quizsol \textit{True/False}:
To maximize $z= 5x + 6y$ subject to $2x + 3y \leq 6$, $-6x + y \leq 20$, and $x, y \geq 0$, the initial simplex tableau is\dots
	\begin{center}
	\begin{tabular}{rrrr|r}
	$2$ & $3$ & $1$ & $0$ & $6$ \\
	$-6$ & $1$ & $0$ & $1$ & $20$ \\ \hline
	$-5$ & $-6$ & $0$ & $0$ & $0$
	\end{tabular}
	\end{center} \pspace

\sol The statement is \textit{true}. First, note that the problem is in standard form. For each inequality, we introduce a slack variable so that we have\dots
	\[
	\begin{aligned}
	2x + 3y + s_1 \phantom{+ s_2}&= 6 \\
	-6x + y \phantom{+ s_1} + s_2&= 20
	\end{aligned}
	\]
Moving all the variables to the left side in $z= 5x + 6y$, we have $z - 5x - 6y= 0$. Aligning the equations, we have\dots
	\[
	\begin{aligned}
	\phantom{z -} 2x + 3y + s_1 \phantom{+ s_2}&= 6 \\
	\phantom{z } -6x + y \phantom{+ s_1} + s_2&= 20 \\
	z - 5x -6y \phantom{+ s_1 + s_2}&= 0
	\end{aligned}
	\]
This gives us the initial simplex tableau\dots
	\begin{center}
	\begin{tabular}{rrrr|r}
	$2$ & $3$ & $1$ & $0$ & $6$ \\
	$-6$ & $1$ & $0$ & $1$ & $20$ \\ \hline
	$-5$ & $-6$ & $0$ & $0$ & $0$
	\end{tabular}
	\end{center}



\newpage



% Quiz 9
\quizsol \textit{True/False}:
Given the following minimization problem:
	\[
	\begin{aligned}
	\min w&= x_1 + 2x_2 + 3x_3 \\
	x_1 &+ x_2 + x_3 \geq 4 \\
	x_1 &- x_2 + x_3 \geq 6 \\
	-x_1 &+ x_2 - x_3 \geq 8
	\end{aligned}
	\]
the dual problem is\dots
	\[
	\begin{pmatrix}
	1 & 1 & -1 & 1 \\
	1 & -1 & 1 & 2 \\
	1 & 1 & -1 & 3 \\
	4 & 6 & 8 & 0 
	\end{pmatrix}
	\] \pspace

\sol The statement is \textit{false}. First, observe that the problem is in standard form. Given a minimization problem in standard form, we first align all the inequalities with the function as the bottom equation:
	\[
	\begin{aligned}
	x_1 + x_2 + x_3&\geq 4 \\
	x_1 - x_2 + x_3&\geq 6 \\
	-x_1 + x_2 - x_3&\geq 8 \\
	x_1 + 2x_3 + 3x_3&= 0 
	\end{aligned}
	\]
From this, we form the matrix\dots
	\[
	\begin{pmatrix}
	1 & 1 & 1 & 4 \\
	1 & -1 & 1 & 6 \\
	-1 & 1 & -1 & 8 \\
	1 & 2 & 3 & 0 
	\end{pmatrix}
	\]
We then take the transpose of this matrix\dots
	\[
	\begin{pmatrix}
	1 & 1 & 1 & 4 \\
	1 & -1 & 1 & 6 \\
	-1 & 1 & -1 & 8 \\
	1 & 2 & 3 & 0 
	\end{pmatrix}^T= 
	\begin{pmatrix}
	1 & 1 & -1 & 1 \\
	1 & -1 & 1 & 2 \\
	1 & 1 & -1 & 3 \\
	4 & 6 & 8 & 0 
	\end{pmatrix}
	\]
But then the corresponding maximization problem (the dual problem) is\dots
	\[
	\begin{aligned}
	\max z&= 4y_1 + 6y_2 + 8y_3 \\
	y_1 &+ y_2 - y_3 \leq 1 \\
	y_1 &- y_2 + y_3 \leq 2 \\
	y_1 &+ y_2 - y_3 \leq 3
	\end{aligned}
	\]



\newpage



% Quiz 10
\quizsol \textit{True/False}: If \$4,000 is placed into an account that earns 5\% interest, compounded quarterly, then the amount of money in the account after 8~years is\dots
	\[
	4000 \left(1 + \dfrac{0.05}{4} \right)^{32}
	\]

\sol The statement is \textit{true}. We know that if $P$ dollars is placed into an account earning an annual interest rate $r$, compounded $k$ times per year, then the amount of money in the account after $t$ years, $F$, is given by\dots
	\[
	F= P \left(1 + \dfrac{r}{k} \right)^{kt}
	\]
We have $P= 4000$, $r= 0.05$, $k= 4$, and $t= 8$ so that we have\dots
	\[
	F= 4000 \left(1 + \dfrac{0.05}{4} \right)^{4 \cdot 8}= 4000 \left(1 + \dfrac{0.05}{4} \right)^{32}
	\] \pvspace{1.3cm}



% Quiz 11
\quizsol \textit{True/False}: If you take out a loan for \$6,500 at a 5.3\% annual interest rate, compounded continuously, then the amount of money you owe after 3~years is $6500 e^{0.053 \cdot 3} \approx \$7,620.20$. \pspace

\sol The statement is \textit{true}. We know that if $P$ dollars is taken out as a loan at an annual interest rate of $r$, the amount owed after $t$ years, $F$, is given by $F= Pe^{rt}$. Here we have $P= 6500$, $r= 0.053$, and $t= 3$. Then we have\dots
	\[
	F= Pe^{rt}= 6500 e^{0.053 \cdot 3}= 6500(1.1723379) \approx \$7,620.20
	\] \pvspace{1.3cm}



% Quiz 12
\quizsol \textit{True/False}: An ordinary annuity is a series of equal payments, paid at equal intervals of time with payments occurring at the start of the payment period. \pspace

\sol The statement is \textit{false}. In an ordinary (simple) annuity, payments are made at the \textit{end} of each payment period and the payment periods are the same as the interest periods (otherwise, it is a general ordinary annuity). If the payments are made at the \textit{start} of each payment period and the payment periods are the same as the interest periods, then it is a (simple) annuity due (otherwise, it is a general annuity due). We can summarize this in the following chart:
	\begin{table}[!ht]
	\centering
	\begin{tabular}{lcc}
	 & \multicolumn{1}{l}{\begin{tabular}[c]{@{}l@{}}Payment at \\ Start of Period\end{tabular}} & \multicolumn{1}{l}{\begin{tabular}[c]{@{}l@{}}Payment at\\ End of Period\end{tabular}} \\ \cline{2-3} 
	\multicolumn{1}{l|}{\begin{tabular}[c]{@{}l@{}}Payment Period $=$\\ Compounding Period\end{tabular}} & \multicolumn{1}{c|}{Ordinary Annuity Due} & \multicolumn{1}{c|}{\begin{tabular}[c]{@{}c@{}}Ordinary (Simple)\\ Annuity\end{tabular}} \\ \cline{2-3} 
	\multicolumn{1}{l|}{\begin{tabular}[c]{@{}l@{}}Payment Period $\neq$\\ Compounding Period\end{tabular}} & \multicolumn{1}{c|}{\begin{tabular}[c]{@{}c@{}}General Annuity\\ Due\end{tabular}} & \multicolumn{1}{c|}{General Annuity} \\ \cline{2-3} 
	\end{tabular}
	\end{table} \pvspace{1.3cm}



\newpage



% Quiz 13
\quizsol \textit{True/False}: If you have a \$5,000 loan at 5.4\% annual interest, compounded monthly, that you pay over 5~years using a series of monthly payments of \$95.28, then the amount you still owe on the loan after 2~years is\dots
	\[
	P= 5000\, a_{\actuarialangle{36}0.0045}= 3160.11
	\] \pspace

\sol The statement is \textit{false}. We know that the amount due on an (ordinary) amortized loan is given by $P= R\, a_{\actuarialangle{n - m}i}$. We have $R= 95.28$, $i_p= r/k= 0.054/12= 0.0045$, $n= kt= 12 \cdot 5= 60$, and $m= kt_0= 12 \cdot 2= 24$. Then we have\dots
	\[
	P= 95.28\, a_{\actuarialangle{36}0.0045}= 3160.11
	\] \pvspace{1.3cm}



% Quiz 14
\quizsol \textit{True/False}: In an amortized loan, a portion of each payment goes towards paying off the principal, with a portion of the payment going towards the interest. Over time, the portion of the payment going towards the interest decreases. \pspace

\sol The statement is \textit{true}. Because the amount owed on the loan is at first `large', the amount of the interest is `large' at first. Therefore, more of each payment is going towards the interest than later in the loan when the interest is less because the interest is being applied to a smaller amount (because portions of the loan have been paid off). 


% Quiz 15
\quizsol \textit{True/False}: A simple linear regression is a linear model which minimizes the sum $\sum (y_i - \widehat{y}_i)^2$. 


% Quiz 16
\quizsol \textit{True/False}: If $P(A)= 0.40$ and $P(B)= 0.50$, then $P(A \text{ and } B)= P(A) P(B)= 0.40 \cdot 0.50= 0.20$. 

% Quiz 17
\quizsol \textit{True/False}: For events $A$ and $B$, we have $P(A \text{ or } B)= P(A) + P(B) - P(A \text{ and } B)$ and $P(A \text{ and } B)= P(A) P(B \;|\; A)$. 


% Quiz 18
\quizsol \textit{True/False}: The expected value for a random variable is the amount one should expect `on average.' 

% Quiz 19
\quizsol \textit{True/False}: If Alice receives a 75 on an exam distributed as $N(82, 2.4)$ and Bob receives a 71 on an exam distributed as $N(85, 8.0)$, then Bob did relatively worse on his exam compared to Alice. 




% Quiz 20
\quizsol \textit{True/False}: The binomial distribution can be used any time one is interested in the probability that a certain number of events occurs over some time period. 










\end{document}