\documentclass[11pt,letterpaper]{article}
\usepackage[lmargin=1in,rmargin=1in,bmargin=1in,tmargin=1in]{geometry}
\usepackage{style/quiz}
\usepackage{style/commands}

% -------------------
% Content
% -------------------
\begin{document}
\thispagestyle{title}


% Quiz 1
\quizsol \textit{True/False}: If you had a bill of \$25.77 and were going to pay a tip of 20\%, the total amount you would pay could be computed by finding $25.77(1.20)$. \pspace

\sol The statement is \textit{true}. Recall to calculate a percentage of a number $N$, we compute $N \cdot \%$, where $N$ is the number and $\%$ is the percentage (written as a decimal). For instance, to compute 57\% of 23, we compute $23(0.57)= 13.11$. To compute 172\% of 150, we compute $150(1.72)= 258$. However, to compute a $\%$ percent increase or decrease of a number $N$, we compute $N(1 \pm \%)$, where $N$ is the number, $\%$ is the percentage as a decimal, and we choose plus for increase and negative for decrease. For instance, to compute a 75\% decrease of 13, we compute $13(1 - 0.75)= 13(0.25)= 3.25$. To compute a 115\% increase of 120, we compute $120(1 + 1.15)= 120(2.15)= 258$. Here, we are increasing 25.77 by 20\%, so we compute $25.77(1 + 0.20)= 25.77(1.20)$. 





\end{document}