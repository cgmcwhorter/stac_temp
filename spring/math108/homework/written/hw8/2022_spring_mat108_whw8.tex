\documentclass[11pt,letterpaper]{article}
\usepackage[lmargin=1in,rmargin=1in,tmargin=1in,bmargin=1in]{geometry}
\usepackage{../style/homework}
\usepackage{../style/commands}
\setbool{quotetype}{true} % True: Side; False: Under
\setbool{hideans}{true} % Student: True; Instructor: False

% -------------------
% Content
% -------------------
\begin{document}

\homework{8: Due 04/11}{Surplus wealth is a sacred trust which its possessor is bound to administer in his lifetime for the good of the community.}{Andrew Carnegie}

% Problem 1
\problem{10} Anne Morale takes out a loan for \$2,300 for 10~months that is discounted at 9.5\% annual simple interest.
	\begin{enumerate}[(a)]
	\item What is the discount for this loan?
	\item What is the maturity? What are the proceeds?
	\item How much interest is paid on this loan? How much is paid in total?
	\item Find the effective annual interest rate for this loan.
	\end{enumerate}



\newpage



% Problem 2
\problem{10} Joe King invests \$6,000 in a savings account that earns 3.7\% annual interest, compounded semiannually. 
	\begin{enumerate}[(a)]
	\item How much is in the account after 5~years?
	\item How long until the account has \$8,000?
	\item What is the effective annual interest for the account?
	\end{enumerate}



\newpage



% Problem 3
\problem{10} Amanda Lynn takes out a loan for \$18,000 at an annual interest rate of 6.1\%, compounded continuously. 
	\begin{enumerate}[(a)]
	\item How much is owed after 3~years?
	\item How long until the loan amount is \$20,000?
	\item If she was going to receive \$25,000 in 3~years, what is the maximum amount she could have taken out for the loan?
	\end{enumerate}



\newpage



% Problem 4
\problem{10} Barry D. Hatchett is going to set up a savings account for his daughter. He has two option: one account earns 4.2\% annual interest, compounded monthly. The other account earns 3.8\% annual interest, compounded continuously. Which account should he take? Justify your answer.


\end{document}