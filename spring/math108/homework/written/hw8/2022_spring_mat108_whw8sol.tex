\documentclass[11pt,letterpaper]{article}
\usepackage[lmargin=1in,rmargin=1in,tmargin=1in,bmargin=1in]{geometry}
\usepackage{../style/homework}
\usepackage{../style/commands}
\setbool{quotetype}{false} % True: Side; False: Under
\setbool{hideans}{false} % Student: True; Instructor: False

% -------------------
% Content
% -------------------
\begin{document}

\homework{8: Due 04/11}{Surplus wealth is a sacred trust which its possessor is bound to administer in his lifetime for the good of the community.}{Andrew Carnegie}

% Problem 1
\problem{10} Anne Morale takes out a loan for \$2,300 for 10~months that is discounted at 9.5\% annual simple interest.
	\begin{enumerate}[(a)]
	\item What is the discount for this loan?
	\item What is the maturity? What are the proceeds?
	\item How much interest is paid on this loan? How much is paid in total?
	\item Find the effective annual interest rate for this loan.
	\end{enumerate} \pspace

\sol
\begin{enumerate}[(a)]
\item We know that\dots
	\[
	D= Mrt= 2300 \cdot 0.095 \cdot \dfrac{10}{12}= \$182.083 \approx \$182.08
	\] \pspace

\item The maturity is the original full loan amount, which is \$2,300. The proceeds are the loan amount discounted by the interest, which is \$182.08. Therefore, the proceeds are $2300 - 182.08= \$2,117.92$. \pspace

\item The interest is the discount, which we know from (a) is \$182.08. After 10~months, the original loan amount of \$2,300 must be paid. One has already paid the interest of \$182.08 up-front. Therefore, the total amount paid is $2300 + 182.08= \$2,482.08$. \pspace

\item We know that\dots
	\[
	r_{\text{eff}}= \dfrac{r}{1 - rt}= \dfrac{0.095}{1 - 0.095 \cdot \dfrac{10}{12}}= 0.103167 \approx 10.32\%
	\]
\end{enumerate}



\newpage



% Problem 2
\problem{10} Joe King invests \$6,000 in a savings account that earns 3.7\% annual interest, compounded semiannually. 
	\begin{enumerate}[(a)]
	\item How much is in the account after 5~years?
	\item How long until the account has \$8,000?
	\item What is the effective annual interest for the account?
	\end{enumerate} \pspace

\sol
\begin{enumerate}[(a)]
\item We know that\dots
	\[
	F= P \left(1 + \dfrac{r}{k} \right)^{kt}= 6000 \left(1 + \dfrac{0.037}{2} \right)^{2 \cdot 5}= 6000 (1.201186)= 7207.1172 \approx \$7,207.12
	\] \pspace

\item We know that\dots
	\[
	t= \dfrac{\ln(F/P)}{k \ln \left( 1 + \dfrac{r}{k} \right)}= \dfrac{\ln(8000/6000)}{2 \ln \left( 1 + \dfrac{0.037}{2} \right)}= \dfrac{0.287682}{0.0366619}= 7.84689 \approx 7.85 \text{ years}
	\] \pspace

\item We know that\dots
	\[
	r_{\text{eff}}= \left( 1 + \dfrac{r}{k} \right)^k - 1= \left( 1 + \dfrac{0.037}{2} \right)^2 - 1= 0.03734 \approx 3.73\%
	\]
\end{enumerate}



\newpage



% Problem 3
\problem{10} Amanda Lynn takes out a loan for \$18,000 at an annual interest rate of 6.1\%, compounded continuously. 
	\begin{enumerate}[(a)]
	\item How much is owed after 3~years?
	\item How long until the loan amount is \$20,000?
	\item If she was going to receive \$25,000 in 3~years, what is the maximum amount she could have taken out for the loan?
	\end{enumerate} \pspace

\sol
\begin{enumerate}[(a)]
\item We know that\dots
	\[
	F= P e^{rt}= 18000 e^{0.061 \cdot 3}= 18000 (1.2008144)= 21614.6593 \approx \$21,614.66
	\] \pspace

\item We know that\dots
	\[
	t= \dfrac{\ln(F/P)}{r}= \dfrac{\ln(20000/18000)}{0.061}= \dfrac{0.105361}{0.061}= 1.7272 \approx 1.72 \text{ years}
	\] \pspace

\item We know that\dots
	\[
	P= \dfrac{F}{e^{rt}}= \dfrac{25000}{e^{0.061 \cdot 3}}= \dfrac{25000}{1.200814}= 20819.2038 \approx \$20,819.20
	\]
\end{enumerate}



\newpage



% Problem 4
\problem{10} Barry D. Hatchett is going to set up a savings account for his daughter. He has two option: one account earns 4.2\% annual interest, compounded monthly. The other account earns 3.8\% annual interest, compounded continuously. Which account should he take? Justify your answer. \pspace

\sol Suppose we compare using doubling time. For the first account, we know that\dots
	\[
	t_D= \dfrac{\ln(2)}{k \ln \left(1 + \dfrac{r}{k} \right)}= \dfrac{\ln(2)}{12 \ln \left(1 + \dfrac{0.042}{12} \right)}= \dfrac{0.693147}{0.0419267}= 16.5324 \approx 16.53 \text{ years}
	\]
For the second account, we have\dots
	\[
	t_D= \dfrac{\ln(2)}{r}= \dfrac{0.693147}{0.038}= 18.2407 \approx 18.24 \text{ years}
	\]
Because the first account doubles the savings in less time, the first account is better. \pspace

Suppose we compare using effective interest. For the first account, we know that\dots
	\[
	r_{\text{eff}}= \left(1 + \dfrac{r}{k} \right)^k - 1= \left(1 + \dfrac{0.042}{12} \right)^{12} - 1= 0.042818 \approx 4.28\%
	\]
For the second account, we have\dots
	\[
	r_{\text{eff}}= e^r - 1= e^{0.038} - 1= 0.03873 \approx 3.87\%
	\]
Because the first account has a higher effective interest rate, it is the better account. 


\end{document}