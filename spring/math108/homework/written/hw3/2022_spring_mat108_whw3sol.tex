\documentclass[11pt,letterpaper]{article}
\usepackage[lmargin=1in,rmargin=1in,tmargin=1in,bmargin=1in]{geometry}
\usepackage{../style/homework}
\usepackage{../style/commands}
\setbool{quotetype}{true} % True: Side; False: Under
\setbool{hideans}{false} % Student: True; Instructor: False

% -------------------
% Content
% -------------------
\begin{document}

\homework{3: Due 02/16}{This is not a dictatorship. This is America. Give me liberty, or give me meth.}{Frank Gallagher, Shameless}

% Problem 1
\problem{10} Determine if the point $(x, y)= (-1, 3)$ is a solution to the system of equations below. Be sure to fully justify your answer.
	\[
	\begin{aligned}
	x^2 + xy + y&= 1 \\
	x^3 - y^3&= -26
	\end{aligned}
	\] \pspace

\sol If $(x, y)= (-1, 3)$ is a solution to the system of equations, then $x= -1$ and $y= 3$ satisfy both of the given equations---which we check:
	\[
	\begin{aligned}
	x^2 + xy + y&= 1 \\
	(-1)^2 + (-1)3 + 3&\stackrel{?}{=} 1 \\
	1 - 3 + 3&\stackrel{?}{=} 1 \\
	1&= 1
	\end{aligned}
	\] \pspace
and \pspace
	\[
	\begin{aligned}
	x^3 - y^3&= -26 \\
	(-1)^3 - 3^3&\stackrel{?}{=} -26 \\
	-1 - 27&\stackrel{?}{=} -26 \\
	-28&\neq -26
	\end{aligned}
	\]
Because $(x, y)= (-1, 3)$ does \textit{not} satisfy both of the equations, $(x, y)= (-1, 3)$ is not a solution to the given system of equations. 



\newpage



% Problem 2
\problem{10} Determine if the linear system of equations below has none, one, or infinitely many solutions. Be sure to fully justify your answer. 
	\[
	\begin{aligned}
	2x - y&= -2 \\
	3x + 5y&= 10
	\end{aligned}
	\] \pspace

\sol Observe that both of the systems are linear equations. Therefore, it suffices to determine if the given pair of lines are the same, parallel, or intersection. We solve for $y$ in both equations:
	\[
	\begin{aligned}
	2x - y&= -2 \\
	-y&= -2x - 2 \\
	y&= 2x + 2
	\end{aligned}
	\] \pspace
and \pspace
	\[
	\begin{aligned}
	3x + 5y&= 10 \\
	5y&= -3x + 10 \\
	y&= -\frac{3}{5}\,x + 2
	\end{aligned}
	\] 
Clearly, these lines are distinct. The first line has slope $m_1= 2$ and the second line has slope $m_2= -\frac{3}{5}$. Because $m_1 \neq m_2$, we know that the lines are not parallel; therefore, the lines intersect. But then there must be a solution to the given system of equations. In fact, one can verify that $(x, y)= (0, 2)$ is a solution to the system of equations. 



\newpage



% Problem 3
\problem{10} Find the coefficient matrix, solution vector, and augmented matrix associated with the system of equations below. 
	\[
	\begin{aligned}
	5x_1 + x_2 - 6x_3&= 19 \\
	3x_2 - 2x_3&= -6 \\
	9x_1 + 8x_3&= 5
	\end{aligned}
	\] \pspace

\sol We have\dots
	\[
	\begin{aligned}
	\text{Coefficient Matrix:}& \begin{pmatrix} 5 & 1 & -6 \\ 0 & 3 & -2 \\ 9 & 0 & 8 \end{pmatrix} \\[0.3cm]
	\text{Solution Vector:}& \begin{pmatrix} 19 \\ -6 \\ 5 \end{pmatrix} \\[0.3cm]
	\text{Augmented Matrix:}&	
	\left(
	\begin{array}{rrr:r}
	5 & 1 & -6 & 19 \\
	0 & 3 & -2 & -6 \\
	9 & 0 & 8 & 5
	\end{array} 
	\right)
	\end{aligned}
	\]



\newpage



% Problem 4
\problem{10} Write the system of equations associated to the augmented matrix below. 
	\[
	\begin{pmatrix}
	6 & 1 & -5 & -7 \\
	4 & 0 & -1 & 9 \\
	1 & 1 & 1 & 4 \\
	\end{pmatrix}
	\] \pspace

\sol The last column corresponds to the solutions. The remaining columns---three columns---must correspond to the three variables. Using variables $x, y, z$, we have\dots \pspace
	\[
	\begin{aligned}
	\begin{array}{rrrrrrr}
	6x & + & y & - & 5z & = & -7 \\
	4x &  &  & - & z & = & 9 \\
	x & + & y & + & z & = & 4
	\end{array} 
	\end{aligned}
	\] \pspace
Using variables $x_1, x_2, x_3$, we have\dots \pspace
	\[
	\begin{aligned}
	\begin{array}{rrrrrrr}
	6x_1 & + & x_2 & - & 5x_3 & = & -7 \\
	4x_1 &  &  & - & x_3 & = & 9 \\
	x_! & + & x_2 & + & x_3 & = & 4
	\end{array} 
	\end{aligned}
	\]



\newpage



% Problem 5
\problem{10} Find all the pivot positions in the augmented matrix below. Also, determine if the system of equations is consistent or not. 
	\[
	\begin{aligned}
	\begin{pmatrix}
	1 & 4 & 6 & -2 & 5 \\
	0 & 0 & -1 & 7 & 12 \\
	0 & 0 & 0 & -9 & 5 \\
	0 & 0 & 0 & 0 & 1 
	\end{pmatrix}
	\end{aligned}
	\] \pspace

\sol We circle the pivot positions in the augmented matrix above:
	\[
	\begin{aligned}
	\begin{pmatrix}
	{\Large \textcircled{\normalsize $1$}} & 4 & 6 & -2 & 5 \\
	0 & 0 & {\Large \textcircled{\small \!\!$-1$}} & 7 & 12 \\
	0 & 0 & 0 & {\Large \textcircled{\small \!\!$-9$}} & 5 \\
	0 & 0 & 0 & 0 &  {\Large \textcircled{\small $1$}}
	\end{pmatrix}
	\end{aligned}
	\] \pspace
Finally, observe that the original matrix is in RREF: \pspace
	\[
	\begin{aligned}
	\begin{pmatrix}
	1 & 4 & 6 & -2 & 5 \\
	0 & 0 & -1 & 7 & 12 \\
	0 & 0 & 0 & -9 & 5 \\
	0 & 0 & 0 & 0 & 1 
	\end{pmatrix}
	\end{aligned}
	\] \pspace
Because this is an augmented matrix, the last row corresponds to the equation $0= 1$, which is impossible. Therefore, the original system of equations is inconsistent, i.e. there are no solutions to the system of equations. 



\newpage



% Problem 4
\problem{10} The matrix below represents a reduced-row echelon form of augmented matrix for a system of equations. Determine the solutions to this original system of equations. 
	\[
	\begin{aligned}
	\begin{pmatrix}
	1 & 0 & 0 & -5 \\
	0 & 1 & 0 & 3 \\
	0 & 0 & 1 & 4
	\end{pmatrix}
	\end{aligned}
	\] \pspace

\sol Using variables $x, y, z$, we see from the equations corresponding to the rows of this matrix that\dots \pspace
	\[
	\begin{cases}
	x= -5 \\
	y= 3 \\
	z= 4
	\end{cases}
	\] \pspace
Using variables $x_1, x_2, x_3$, we see that\dots \pspace
	\[
	\begin{cases}
	x_1= -5 \\
	x_2= 3 \\
	x_3= 4
	\end{cases}
	\]



\newpage



% Problem 5
\problem{10} Solve the following system of equations using elimination. Then solve the system of equations again by creating an augmented matrix and find its reduced-row echelon form.
	\[
	\begin{aligned}
	x - 3y&= -9 \\
	-2x + y&= 8
	\end{aligned}
	\] \pspace

\sol Using ordinary elimination, we first add twice the first row to the second row, this gives us\dots
	\[
	\begin{aligned}
	x - 3y&= -9 \\
	0x - 5y&= -10
	\end{aligned}
	\]
Now we divide the second equation by $-5$ and obtain\dots
	\[
	\begin{aligned}
	x - 3y&= -9 \\
	0x + y&= 2
	\end{aligned}
	\]
Now we add three times the third row to the first row to obtain:
	\[
	\begin{aligned}
	x - 0y&= -3 \\
	0x + y&= 2
	\end{aligned}
	\]
Therefore, the solution to the system of equations is $(x, y)= (-3, 2)$. Using an augmented matrix to solve the system of equations, we have\dots

\begin{minipage}[t]{0.49\textwidth}
	\[
	\left(
	\begin{array}{rr:r}
	1 & -3 & -9 \\
	-2 & 1 & 8
	\end{array} 
	\right)
	\] \pspace
	\[
	\left(
	\begin{array}{rr:r}
	1 & -3 & -9 \\
	0 & -5 & -10
	\end{array} 
	\right)
	\] \pspace
	\[
	\left(
	\begin{array}{rr:r}
	1 & -3 & -9 \\
	0 & 1 & 2
	\end{array} 
	\right)
	\] \pspace
	\[
	\left(
	\begin{array}{rr:r}
	1 & 0 & -3 \\
	0 & 1 & 2
	\end{array} 
	\right)
	\]
\end{minipage}%
\begin{minipage}[t]{0.49\textwidth}
\pvspace{0.5cm}
$2R_1 + R_2 \to R_2$ \pvspace{1.1cm}
$-\dfrac{1}{5} R_2 \to R_2$ \pvspace{1.1cm}
$3R_2 + R_1 \to R_1$ \pvspace{1.1cm}
\end{minipage} \pspace

Therefore, the solution is $(x, y)= (-3, 2)$, i.e. $x= -3$ and $y= 2$:
	\[
	\left\{
	\begin{aligned}
	x&= -3 \\
	y&= 2 
	\end{aligned}
	\right.
	\]



\newpage



% Problem 5
\problem{10} Use \href{https://www.wolframalpha.com/}{WolframAlpha's} \texttt{RowReduce} to find the solution to the following system of equations: 
	\[
	\begin{aligned}
	x_1 + x_2 + x_3 + x_4&= 1 \\
	x_1 - 2x_2 + 3x_3 - 4x_4&= 2 \\
	10x_1 + 3x_2 - 5x_3 - 2x_4&= 3 \\
	-2x_1 - 4x_2 + 6x_3 + 8x_4&= 4
	\end{aligned}
	\] \pspace

\sol The associated matrix is\dots
	\[
	\left(
	\begin{array}{rrrr:r}
	1 & 1 & 1 & 1 & 1 \\
	1 & -2 & 3 & -4 & 2 \\
	10 & 3 & -5 & -2 & 3 \\
	-2 & -4 & 6 & 8 & 4
	\end{array} 
	\right)
	\] 
Using WolframAlpha's \texttt{RowReduce} function, we obtain\dots
	\[
	\left(
	\begin{array}{rrrr:r}
	1 & 0 & 0 & 0 & 128/195 \\
	0 & 1 & 0 & 0 & -19/65 \\
	0 & 0 & 1 & 0 & 92/195 \\
	0 & 0 & 0 & 1 & 32/195 \\
	\end{array} 
	\right)
	\] 
Therefore, the solution is\dots
	\[
	(x_1, x_2, x_3, x_4)= (128/195, -19/65, 92/195, 32/195) \approx (0.65641, -0.292308, 0.471795, 0.164103)
	\] 
i.e. $x_1\approx 0.65641$, $x_2 \approx -0.292308$, $x_3 \approx 0.471795$, and $x_4 \approx 0.164103$:
	\[
	\begin{cases}
	x_1= \frac{128}{195} \\
	x_2= -\frac{19}{65} \\
	x_3= \frac{92}{195} \\
	x_4= \frac{32}{195} \\
	\end{cases}
	\]


\end{document}