\documentclass[11pt,letterpaper]{article}
\usepackage[lmargin=1in,rmargin=1in,tmargin=1in,bmargin=1in]{geometry}
\usepackage{../style/homework}
\usepackage{../style/commands}
\setbool{quotetype}{false} % True: Side; False: Under
\setbool{hideans}{false} % Student: True; Instructor: False

% -------------------
% Content
% -------------------
\begin{document}

\homework{1: Due 02/07}{Leslie, I typed your symptoms into the thing up here and it says you could have network connectivity problems.}{Andy Dwyer, Parks and Recreation}

% Problem 1
\problem{10} Compute the following:
        \begin{enumerate}[(a)]
        \item 36\% of 657.30
        \item 97\% of 450
        \item 154\% of 78.56
        \item 220\% of 11.2
        \end{enumerate} \pspace

\sol
\begin{enumerate}[(a)]
\item 
	\[
	657.30(0.36)= 236.628
	\] \pspace

\item 
	\[
	450(0.97)= 436.50
	\] \pspace

\item 
	\[
	78.56(1.54)= 120.982
	\] \pspace

\item 
	\[
	11.2(2.20)= 24.64
	\]
\end{enumerate}



\newpage



% Problem 2
\problem{10} Compute the following:
        \begin{enumerate}[(a)]
        \item 54 increased by 75\%
        \item 1640 decreased by 22\%
        \item 81 increased by 280\%
        \item 771 decreased by 95\%
        \end{enumerate} \pspace

\sol
\begin{enumerate}[(a)]
\item 
	\[
	54(1 + 0.75)= 54(1.75)= 94.50
	\] \pspace

\item 
	\[
	1640(1 - 0.22)= 1640(0.78)= 1279.20 
	\] \pspace

\item 
	\[
	81(1 + 2.80)= 81(3.80)= 307.80
	\] \pspace

\item 
	\[
	771(1 - 0.95)= 771(0.05)= 38.55
	\]
\end{enumerate}



\newpage



% Problem 3
\problem{10} Convert the following:
	\begin{enumerate}[(a)]
	\item \texteuro120 to USD [\$1~USD = \texteuro 0.88]
	\item 50~km/h to miles per second [1~km = 0.621371~mi]
	\item \texteuro5/m$^2$ to USD/ft$^2$ [\$1~USD = \texteuro 0.88; 1~m = 3.28084~ft]
	\end{enumerate} \pspace

\sol
\begin{enumerate}[(a)]
\item \phantom{.}
	\begin{table}[!ht]
	\centering
	\begin{tabular}{c|c}
	\texteuro120 & \$1 \\ \hline
	& \texteuro0.88
	\end{tabular}
	= \$136.3636
	\end{table}

\item \phantom{.}
	\begin{table}[!ht]
	\centering
	\begin{tabular}{c|c|c|c}
	50~km & 0.621371~mi & 1~hr & 1~min \\ \hline
	1~hr & 1~km & 60~min & 60~s
	\end{tabular}
	= 0.008630~mps
	\end{table}

\item \phantom{.}
	\begin{table}[!ht]
	\centering
	\begin{tabular}{c|c|c|c}
	\texteuro5 & 1~m & 1~m & \$1 \\ \hline
	1~m$^2$ & 3.28084~ft & 3.28084~ft & \texteuro 0.88
	\end{tabular}
	= \$0.5279/ft$^2$
	\end{table}
\end{enumerate}



\newpage



% Problem 4
\problem{10} Given the following tables, do $f(x)$ and $g(x)$ represent functions? Explain. 
	\begin{table}[!ht]
	\centering \setlength\arrayrulewidth{0.02cm}
	\begin{tabular}{c|ccc|c} 
	$x$ & $f(x)$ & \hspace{2cm} & $x$ & $g(x)$ \\ \cline{1-2} \cline{4-5}
	$1$ & $2$ && $3$ & $3$ \\
	$2$ & $4$ && $4$ & $0$ \\
	$3$ & $6$ && $6$ & $4$ \\
	$4$ & $8$ && $7$ & $5$ \\
	$1$ & $10$ && $8$ & $6$  
	\end{tabular}
	\end{table} \pspace

\sol The relation $f(x)$ does \textit{not} represent a function. Observe that the input $x= 1$ has two possible outputs. Therefore, $f(x)$ cannot be a function because not all possible inputs have only one possible output. On the other hand, the relation $g(x)$ is a function---for each possible input, there is only one possible output. 
	\[
	\boxed{%
	\begin{aligned}
	f(x) \text{ is \textit{not} a function.} \\
	g(x) \text{ is a function.}
	\end{aligned}
	}
	\]



\newpage



% Problem 5
\problem{10} Is the relation $f(x)= 576.10 - 14.39x$ a function of $x$? Explain. \pspace

\sol Yes, the relation $f(x)= 576.10 - 14.39x$ is a function of $x$ because for each possible input $x$, there is only one possible output---namely the one obtained by evaluating $f(x)$ at $x$ and following order of operations. 



\newpage



% Problem 6
\problem{10} Is the relation $f(x, y, z)= 45.1x - 36.0y + 1.2z$ a function of $x, y, z$? Explain. \pspace

\sol Yes, the relation $f(x, y, z)= 45.1x - 36.0y + 1.2z$ is a function of $x, y, z$ because for each possible input $x, y, z$, there is only one possible output---namely the one obtained by evaluating $f(x)$ at $x, y, z$ and following order of operations. 



\newpage



% Problem 7 
\problem{10} For each of the following, indicate whether the function is linear (T), or not (F). 
	\begin{enumerate}[(a)]
	\item \usol{0.7cm}{T}: $y= 4.4x + 50.9$
	\item \usol{0.7cm}{F}: $f(x)= x^2 - 2x + 1$
	\item \usol{0.7cm}{T}: $w= \frac{5}{6}p + 14$
	\item \usol{0.7cm}{F}: $g(t)= \dfrac{t}{t + 1}$
	\item \usol{0.7cm}{T}: $r= 16.8(b + 8.3)$
	\item \usol{0.7cm}{F}: $h(x)= 6.8x(2.2x + 4.8)$
	\end{enumerate} \pspace

{\itshape Remark.} It should be routine to see that (a) and (c) are linear and that (b) and (d) are not linear. To see why (e) is linear, observe $r= 16.8(b + 8.3)= 16.8b + 139.44$, which is routinely verified to be linear. To see why (f) is not linear, observe that $h(x)= 6.8x(2.2x + 4.8)= 14.96x^2 + 32.64x$, which is routinely verified not to be linear. 



\newpage



% Problem 8
\problem{10} For each of the following, indicate whether the function is linear (L), affine linear (A), or neither (N). 
	\begin{enumerate}[(a)]
	\item \usol{0.7cm}{L}: $f(x, y, z)= 99.15x + 67.45y - 1.44z$
	\item \usol{0.7cm}{N}: $g(x, y)= 45.34x^2 + 34.1y^2 + 16.1x - 96.0y$
	\item \usol{0.7cm}{A}: $h(x_1, x_2, x_3)= 4.5x_1 + 6.1x_2 - 8.1x_3 + 8.9$
	\end{enumerate}



\newpage



% Problem 8
\problem{10} Assume the numbers below represent the slope for some linear function. For each of the given slopes, indicate whether the function is increasing or decreasing and interpret the given slope in at least two different ways:
	\begin{enumerate}[(a)]
	\item $m= 5$
	\item $m= -3$
	\item $m= \dfrac{2}{3}$
	\item $m= -\dfrac{5}{6}$
	\item $m= 4.67$
	\end{enumerate} \pspace

\sol
\begin{enumerate}[(a)]
\item Because $m= 5 > 0$, the function is increasing. We have $m= 5= \frac{5}{1}= \frac{-5}{-1}$. We can interpret this as every increase of 1 in the input results in an increase of 5 in the output or that every decrease of 1 in the input results in a decrease of 5 in the output. 
 
\item Because $m= -3 < 0$, the function is decreasing. We have $m= -3= -\frac{3}{1}= \frac{-3}{1}= \frac{3}{-1}$. We can interpret this as every increase of 1 in the input results in a decrease of 3 in the output or that every decrease of 1 in the input results in an increase of 3 in the output. 
 
\item Because $m= \frac{2}{3} > 0$, the function is increasing. We have $m= \frac{2}{3}= \frac{-2}{-3}$. We can interpret this as every increase of 3 in the input results in an increase of 2 in the output or that every decrease of 3 in the input results in a decrease of 2 in the output. Alternatively, using the fact that $m= \frac{2}{3} \approx 0.6667= \frac{0.6667}{1}= \frac{-0.6667}{-1}$, we can interpret this as saying every increase of 1 in the input results in an increase of 0.6667 in the output or that every decrease of 1 in the input results in a decrease of 0.6667 in the output. 
 
\item Because $m= -\frac{5}{6} < 0$, the function is decrease. We have $m= -\frac{5}{6}= \frac{-5}{6}= \frac{5}{-6}$. We can interpret this as every increase of 6 in the input results in a decrease of 5 in the output or that every decrease of 6 in the input results in an increase of 5 in the output. Alternatively, using the fact that $m= -\frac{5}{6} \approx -0.8333= \frac{-0.8333}{1}= \frac{0.8333}{-1}$, we can interpret this as saying every increase of 1 in the input results in a decrease of 0.8333 in the output or that every decrease of 1 in the input results in an increase of 0.8333 in the output. 
 
\item Because $m= 4.67 > 0$, the function is increasing. We have $m= 4.67= \frac{4.67}{1}= \frac{-4.67}{-1}$. We can interpret this as every increase of 1 in the input results in an increase of 4.67 in the output or that every decrease of 1 in the input results in a decrease of 4.67 in the output.  
\end{enumerate}



\newpage



% Problem 9
\problem{10} Jon is paid a base salary of \$56,000 each year. However, he also earns a commission of 2\% of the total amount of sales he makes each year. 
        \begin{enumerate}[(a)]
        \item Explain why Jon's yearly income is a linear function of his sales.
        \item Find a function, $I(s)$, that gives Jon's yearly income, $I$, in terms of his total sales, $s$.
        \item What is the $y$-intercept for this function? What does it represent?
        \item What is the slope for this function? What does it represent? 
        \end{enumerate} \pspace

\sol
\begin{enumerate}[(a)]
\item Because Jon's salary has a constant rate of change, his yearly income is a linear function of his sales. \pspace

\item We know Jon earns \$56,000, regardless of his sales. For each $s$~dollars in sales he makes, he receives 2\% of this. But this amount is $0.02s$. Therefore, we have
	\[
	I(s)= 0.02s + 56000
	\] \pspace

\item We know the $y$-intercept occurs when the input(s) are zero. But then we have\dots
	\[
	I(0)= 0.02(0) + 56000= 56000
	\]
Therefore, the $y$-input is $(0, 56000)$ or 56000. This is the amount Jon earns when he makes $s= 0$ in sales. Therefore, the $y$-intercept is represents Jon's base yearly salary. \pspace

\item We know that $I(s)= 0.02s + 56000$. This function is linear with slope $m= 0.02$. Interpreting this as $\frac{\Delta \text{Output}}{\Delta \text{Input}}$, we see that for every dollar in sales Jon makes, his yearly income increases by 0.02. Therefore, the slope represents Jon's commission rate. 
\end{enumerate}



\newpage



% Problem 10
\problem{10} Aiyana is a statistician. She models that the number of traffic accidents at a particular city intersection can be modeled by $A(c)= 0.002c - 1.3$, where $A$ is the number of accidents and $c$ is the number of cars that pass through the intersection each month. 
	\begin{enumerate}[(a)]
	\item Is the model $A(c)$ linear? Explain.
	\item Find the $y$-intercept for this function. If possible, interpret the intercept in context. 
	\item Find the slope of $A(c)$. If possible, interpret this slope in context. 
	\end{enumerate} \pspace

\sol
\begin{enumerate}[(a)]
\item Yes, the function $A(c)$ is linear because $A(c)$ has the form $y= mx + b$ with $x= c$, $m= 0.002$, and $b= -1.3$, we know that $A(c)$ is a linear function. \pspace

\item We know the $y$-intercept occurs when the input(s) are zero. But then we have\dots
	\[
	A(0)= 0.002(0) - 1.3= -1.3
	\]
Therefore, the $y$-intercept is $(0, -1.3)$ or $-1.3$. This is the number of accidents that occur when zero cars pass through the intersection that month. But because the number of accidents occurring at the intersection must be nonnegative and $-1.3 < 0$, this $y$-intercept does not have an (obvious) contextual interpretation. \pspace

\item Because $A(c)$ has the form $y= mx + b$ with $m= 0.002$, we know that the slope is 0.002. We interpret this as $\frac{\Delta \text{Output}}{\Delta \text{Input}}$. Using the fact that $0.002= \frac{0.002}{1}$ and scaling by $1000$, i.e. $\frac{0.002}{1} \cdot \frac{1000}{1000}= \frac{2}{1000}$, we see that for every 1000~cars passing through the intersection, there are 2 accidents. 
\end{enumerate}



\newpage



% Problem 12
\problem{10} Consider the linear function $\ell(x, y)= 56.4x - 5.6y$. 
\begin{enumerate}[(a)]
\item Explain why this function is linear. 
\item Find $\ell(10.3, 7.1)$.
\item What is the slope `in the $x$-direction'? Interpret this slope and indicate whether $\ell$ is increasing or decreasing with respect to $x$. 
\item What is the slope `in the $y$-direction'? Interpret this slope and indicate whether $\ell$ is increasing or decreasing with respect to $x$. 
\end{enumerate} \pspace

\sol
\begin{enumerate}[(a)]
\item This function is linear because it has the form $f(x_1, x_2)= a_1 x_1 + a_2 x_2$ with $x= x_1$, $y= x_2$, $a_1= 56.4$, and $a_2= -5.6$, i.e. it is linear in each variable. \pspace

\item We have\dots
	\[
	\ell(10.3, 7.1)= 56.4(10.3) - 5.6(7.1)= 580.92 - 39.76= 541.16
	\] \pspace

\item The `slope in the $x$-direction' is $a_1= 56.4$. Because $a_1 > 0$, the function is increasing in $x$. Interpreting this as $56.4= \frac{56.4}{1}$, we see that for every increase of 1 in $x$, $\ell(x, y)$ increases by 56.4. \pspace

\item The `slope in the $y$-direction' is $a_2= -5.6$. Because $a_2 < 0$, the function is decreasing in $y$. Interpreting this as $-5.6= \frac{-5.6}{1}$, we see that for every increase of 1 in $x$, $\ell(x, y)$ decreases by 5.6.
\end{enumerate}


\end{document}