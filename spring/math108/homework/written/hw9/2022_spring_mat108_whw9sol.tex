\documentclass[11pt,letterpaper]{article}
\usepackage[lmargin=1in,rmargin=1in,tmargin=1in,bmargin=1in]{geometry}
\usepackage{../style/homework}
\usepackage{../style/commands}
\setbool{quotetype}{true} % True: Side; False: Under
\setbool{hideans}{false} % Student: True; Instructor: False

% -------------------
% Content
% -------------------
\begin{document}

\homework{9: Due 04/11}{Economics has never been a science---and it is even less now than a few years ago.}{Paul A. Samuelson}

% Problem 1
\problem{10} Bennie Factor is setting up a college fund for his children. He will deposit \$1,500 every 6~months into an account that earns 4.6\% annual interest, compounded semiannually. How much is in the account after 18~years? \pspace

\sol This is an annuity. We have six-month payments of $R= 1500$, an annual interest rate of $r= 0.046$ (so $i_p= r/k= 0.023$), and number of compounds $n= 18 \cdot 2= 36$. We then have\dots
	\[
	s_{\actuarialangle{n}\,i_p}= s_{\actuarialangle{36}\,0.023}= \dfrac{(1 + 0.023)^{36} - 1}{0.023}= 55.1031
	\]
so that we have\dots
	\[
	F= R\, s_{\actuarialangle{36}\,0.023}= 1500 (55.1031)= 82654.7074 \approx \$82,654.71
	\]



\newpage



% Problem 2
\problem{10} Holly Wood is saving money for her senior project in film school. To produce her film, she needs \$6,000 in 18~months. If she makes equal monthly payments into an account earning 2.3\% annual interest, compounded monthly, how much money does she need to deposit into the account each month? \pspace

\sol This is an annuity. We have future value $F= 6000$, annual interest $r= 0.023$ (so that $i_p= r/k= 0.00191667$), and number of payments $n= 18$. We then have\dots
	\[
	s_{\actuarialangle{n}\,i_p}= s_{\actuarialangle{18}\,0.00191667}= \dfrac{(1 + 0.00191667)^{18} - 1}{0.00191667}= 18.2963
	\]
so that we have\dots
	\[
	R= \dfrac{F}{s_{\actuarialangle{18}\,0.00191667}}= \dfrac{6000}{18.2963}= 327.936 \approx \$327.94
	\]



\newpage



% Problem 3
\problem{10} Jack Pott wins a \$125,000 lottery. To make the money last as long as possible, he deposits this money into an account that earns 8\% annual interest, compounded quarterly. If he wants to withdraw money from this account monthly so that the money lasts 10~years, what should he withdraw from the account each month? \pspace

\sol This is an annuity. We have present value $P= 125000$, annual interest $r= 0.08$ (so that $i_p= r/k= 0.02$), and number of compounds $n= 10 \cdot 12= 120$. We then have\dots
	\[
	a_{\actuarialangle{n}\,i_p}= a_{\actuarialangle{120}\,0.02}= \dfrac{1 - (1 + 0.02)^{-120}}{0.02}= 45.3554
	\]
so that we have\dots
	\[
	R= \dfrac{P}{a_{\actuarialangle{120}\,0.02}}= \dfrac{125000}{45.3554}= 2756.0121 \approx \$2,756.01
	\]



\newpage



% Problem 4
\problem{10} To purchase a new car, Horace Cope takes out a loan for \$9,000 at a rate of 8.5\% annual interest, compounded monthly. He will make equal monthly payments on this loan for 2~years. What are the monthly payments? \pspace

\sol This is an amortization. We have principal $P= 9000$, annual interest $r= 0.085$ (so that $i_p= r/k= 0.00708333$), and number of compounds $n= 2 \cdot 12= 24$. We then have\dots
	\[
	a_{\actuarialangle{n}\,i_p}= a_{\actuarialangle{24}\,0.00708333}= \dfrac{1 - (1 + 0.00708333)^{-24}}{0.00708333}= 21.9995
	\]
so that we have\dots
	\[
	R= \dfrac{P}{a_{\actuarialangle{24}\,0.00708333}}= \dfrac{9000}{21.9995}= 409.101 \approx \$409.10
	\]


\end{document}