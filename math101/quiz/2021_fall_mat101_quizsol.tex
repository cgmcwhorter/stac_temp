\documentclass[11pt,letterpaper]{article}
\usepackage[lmargin=1in,rmargin=1in,bmargin=1in,tmargin=1in]{geometry}
\usepackage{style/quiz}
\usepackage{style/commands}

% -------------------
% Content
% -------------------
\begin{document}
\thispagestyle{title}

% Quiz 1
\quizsol \textit{True/False}: $9/3 + 2(3^2 + 10) - 8 + 4 \cdot 3= 45$ \pspace

\sol The statement is \textit{false}. To see this, we can simply follow the order of operations---using PEMDAS as a guide:
	\[
	\begin{aligned}
	9/3 + 2(3^2 + 10) - 8 + 4 \cdot 3&\stackrel{?}{=} 55 \\
	9/3 + 2(9 + 10) - 8 + 4 \cdot 3&\stackrel{?}{=} 55 \\
	9/3 + 2(19) - 8 + 4 \cdot 3&\stackrel{?}{=} 55 \\
	3 + 2(19) - 8 + 4 \cdot 3&\stackrel{?}{=} 55 \\ 
	3 + 38 - 8 + 4 \cdot 3&\stackrel{?}{=} 55 \\ 
	3 + 38 - 8 + 12&\stackrel{?}{=} 55 \\ 
	41 - 8 + 12&\stackrel{?}{=} 55 \\
	33 + 12&\stackrel{?}{=} 55 \\
	45&\neq 55 \\
	\end{aligned}
	\] \pvspace{1.5cm}



% Quiz 2
\quizsol \textit{True/False}: $\gcd(2^3 \cdot 3 \cdot 5, 2 \cdot 3^2 \cdot 7)= 2^3 \cdot 3^2 \cdot 5 \cdot 7$. \pspace

\sol The statement is \textit{false}. Remember given a prime factorization of the numbers, we find the gcd by choosing the \textit{smallest} powers of each prime that appears in the factorizations. So we should have $\gcd(2^3 \cdot 3 \cdot 5, 2 \cdot 3^2 \cdot 7)= 2 \cdot 3$. Instead, the largest power of each prime that appears in the factorizations was chosen which is how we compute the lcm. Therefore, we have $\lcm(2^3 \cdot 3 \cdot 5, 2 \cdot 3^2 \cdot 7)= 2^3 \cdot 3^2 \cdot 5 \cdot 7$. \pvspace{1.2cm}



% Quiz 3
\quizsol \textit{True/False}: $\dfrac{\dfrac{3}{10}}{\dfrac{12}{5}}= \dfrac{1}{8}$ \pspace

\sol The statement is \textit{true}. Note that division by a nonzero number is the same as multiplying by its reciprocal. So we have
	\[
	\dfrac{\phantom{-}\dfrac{3}{10}\phantom{-}}{\dfrac{12}{5}}= \dfrac{3}{10} \cdot \dfrac{5}{12}= \dfrac{\cancel{3}^1}{\cancel{10}^{\,2}} \cdot \dfrac{\cancel{5}^1}{\cancel{12}^{\,4}}= \dfrac{1}{8}
	\]
One can also rewrite the problem as\dots 
	\[
	\dfrac{\phantom{-}\dfrac{3}{10}\phantom{-}}{\dfrac{12}{5}}= \dfrac{3}{10} \div \dfrac{12}{5}
	\]
But then to divide, we multiply by the reciprocal and proceed as in the solution above. \pvspace{1.5cm}



% Quiz 4
\quizsol \textit{True/False}: The number $0.\overline{19}$ is rational. \pspace

\sol The statement is \textit{true}. Any real number with a decimal expansion that either terminates or repeats is a rational and hence can be expressed as $a/b$, where $a$ and $b$ are integers and $b \neq 0$. Moreover, every rational number, i.e. the $a/b$'s, have a decimal expansion that either terminates or repeats. We can even find a rational expression for $0.\overline{19}$:
	\begin{table}[!ht]
	\centering
	\begin{tabular}{rccc}
	 & $100r$ & $=$ & $19.191919191919\ldots$ \\ 
	$-$& $r$ & $=$ & $\phantom{0}0.191919191919\ldots$ \\ \hline
	& $99r$ & $=$ & $19$ 
	\end{tabular}
	\end{table} \par
But then $r= 0.\overline{19}= \dfrac{19}{99}$. \pvspace{1.5cm}

 

% Quiz 5
\quizsol \textit{True/False}: $\sqrt[3]{2^8 \cdot 3^3 \cdot 5^1 \cdot 7^5}= 2^2 \cdot 3^1 \cdot 7 \sqrt[3]{2^2 \cdot 5^1 \cdot 7^2}$ \pspace

\sol The statement is \textit{true}. There are two ways to think about this. First, we should write out the numbers and group them into threes and pull out/leave the terms appropriately:
	\[
	\sqrt[3]{2^8 \cdot 3^3 \cdot 5^1 \cdot 7^5}= \sqrt[3]{\underline{2 \cdot 2 \cdot 2} \cdot \underline{2 \cdot 2 \cdot 2} \cdot 2 \cdot 2 \cdot \underline{3 \cdot 3 \cdot 3} \cdot 5 \cdot \underline{7 \cdot 7 \cdot 7} \cdot 7 \cdot 7}= 2^2 \cdot 3^1 \cdot 7 \sqrt[3]{2^2 \cdot 5 \cdot 7^2}
	\]
Alternatively, we can use division. We know that $8/3$ is 2 with remainder 2, $3/3$ is 1 with remainder 0, $1/3$ is 0 with remainder 1, and $5/3$ is 1 with remainder 2. So we can pull out two 3's with 2 remaining, one 3 with 0 remaining, no 5's with 1 remaining, and two 7's with 2 remaining, which gives:
	\[
	\sqrt[3]{2^8 \cdot 3^3 \cdot 5^1 \cdot 7^5}= 2^2 \cdot 3^1 \cdot 7 \sqrt[3]{2^2 \cdot 5^1 \cdot 7^2}
	\] \pvspace{1.5cm}



% Quiz 6
\quizsol \textit{True/False}: 57 increased by 127\% is $57(1.27)$. \pspace

\sol The statement is \textit{false}. To find 127\% of 57, we would multiply 57 by the percent written as a decimal. This would be $57(1.27)$. However, to increase or decrease a number by a percentage, we compute the number $\#(1 \pm \%)$, where we add if we are increasing, subtract if we are decreasing, $\#$ is the number, and \% is the percentage written as a decimal. So to increase 57 by 127\%, we need to compute $57(1 + 1.27)= 57(2.27)$. \pvspace{1.5cm}



% Quiz 7
\quizsol \textit{True/False}: If $f(x)= x + 1$ and $g(x)= x^2$, then $(f \circ g)(2)= 9$. \pspace

\sol The statement is \textit{false}. Recall that $(f \circ g)(2)= f(g(2))$. First, we compute $g(2)$. We have $g(2)= 2^2= 4$. Then we have $f(g(2))= f(4)$ and we compute $f(4)$: $f(4)= 4 +1= 5$. \pvspace{1.5cm}



\newpage



% Quiz 8
\quizsol \textit{True/False}: The point $(1, 3)$ is on the graph of $f(x)= 2x - 5$. \pspace

\sol The statement is \textit{false}. We have the point $(x, y)= (1, 3)$. If this point is on the graph of $f(x)$, then these $x$ and $y$ satisfy the equation for $f(x)$. We can check this:
	\[
	\begin{aligned}
	f(x)&= 2x - 5 \\
	3&= 2(2) - 5 \\
	3&= 4 - 5 \\
	3&\neq -1
	\end{aligned}
	\]
Therefore, the point $(1, 3)$ is not on the graph of $f(x)$. Alternatively, if $x= 1$, then the corresponding point on the graph of $f(x)$ would have $y$-value $f(1)= 2(1) - 5= 2 - 5= -3$. Then the point $(1, -3)$ is on the graph of $f(x)$. But then $(1, 3)$ is not on the graph of $f(x)$. \pvspace{1.5cm}



% Quiz 9
\quizsol \textit{True/False}: If $f^{-1}(3)= 9$, then $f(3)= 9$. \pspace

\sol The statement is \textit{false}. Recall that $f^{-1}(y)= x$ if and only if $f(x)= y$; that is, $f^{-1}$ asks the question, `what do I plug into $f$ to get this number.' So if $f^{-1}(3)= 9$, this means we should be able to plug in $9$ into $f(x)$ and obtain $3$, i.e. $f(9)= 3$. But then $f(3)= 9$ is not necessarily true. \pvspace{1.5cm}



% Quiz 10
\quizsol \textit{True/False}: To find the $x$-intercept, you find $f(0)$. \pspace

\sol The statement is \textit{false}. Recall that an $x$-intercept is where a function intersects the $x$-axis. But then the $y$-value must be zero. But then because $y= f(x)$, we have $f(x)= 0$. Whereas if we wanted to find a $y$-intercept, we would recall that along the $x$-axis, $x= 0$ so that we would need to find $f(0)$. So finding $x$-intercepts involves solving $f(x)= 0$, whereas finding $y$-intercepts involves evaluating $f(0)$. \pvspace{1.5cm} 



% Quiz 11
\quizsol \textit{True/False}: The lines $y= \frac{2}{3}x + 5$ and $3x + 2y= -6$ are perpendicular. \pspace

\sol The statement is \textit{true}. The line $y= \frac{2}{3}x + 5$ has slope $m= \frac{2}{3}$. Solving for $y$ in the second line, we have $y= -3 - \frac{3}{2} x$. This line has slope $m= -\frac{3}{2}$. The negative reciprocal of $\frac{2}{3}$ is $-\frac{3}{2}$. Therefore, the lines are perpendicular. \pvspace{1.5cm}



% Quiz 12
\quizsol All lines perpendicular to $y= 4$ are of the form $x= \#$. \pspace

\sol The statement is \textit{true}. The line $y= 4$ is horizontal. For a line to be perpendicular to a horizontal line, the line must be vertical. But all vertical lines are of the form $x= \#$. 



\newpage



% Quiz 13
\quizsol \textit{True/False}: Any line with slope 0 must be of the form $y= \#$. \pspace

\sol The statement is \textit{true}. All vertical lines `look like' $y= mx + b$ for some $m, b$. If the slope is 0, then $m= 0$. But then $y= \#$. \pvspace{1.5cm}



% Quiz 14
\quizsol \textit{True/False}: All functions have inverses. \pspace

\sol The statement is \textit{false}. All constant functions, i.e. $f(x)= \#$, do not have inverses. Constant functions are functions---every input has exactly one output (even if they all happen to be the same). However, you cannot `tell' what $x$ gave you \#. Alternatively, $f(x)= \#$ fails the horizontal line test. [Recall that a function has an inverse if and only if it passes the horizontal line test.] \pvspace{1.5cm}



% Quiz 15
\quizsol \textit{True/False}: The quadratic function $y= 5x + 3 - x^2$ opens downwards, is concave, and has a maximum. \pspace

\sol The statement is \textit{true}. Writing the quadratic function in standard form, i.e. $y= ax^2 + bx + c$, we have $y= -x^2 + 5x + 3$. Therefore, for this quadratic function, $a= -1$, $b= 5$, and $c= 3$. Because $a= -1 < 0$, the quadratic function opens downwards, i.e. is concave (down), and has a maximum. \pvspace{1.5cm} 



% Quiz 16
\quizsol \textit{True/False}: The quadratic function $f(x)= 2(x + 2)^2 + 4$ has vertex $(2, 4)$. \pspace

\sol The statement is \textit{false}. The $x$-coordinate of the vertex is the $x$-value that makes the square term zero. In this case, $x= -2$ would make $2(x + 2)^2$ zero. Then we would be left with $y= 4$, which is the $y$-coordinate of the vertex. Therefore, the vertex is $(-2, 4)$. Alternatively, the `proper' vertex form of a quadratic function is $y= A(x - B) + C$. The vertex is $(B, C)$. Writing the `proper' vertex form of the quadratic function $y= 2(x + 2)^2 + 4$, we have $y= 2(x - \;-2)^2 + 4$. Therefore, the vertex form is $(-2, 4)$. Finally, one could expand this out: $y= 2(x + 2)^2 + 4= 2(x^2 + 4x + 4) + 4= 2x^2 + 8x + 8 + 4= 2x^2 + 8x + 12$. The $x$-coordinate of the vertex is $x= \frac{-b}{2a}= \frac{-8}{2(2)}= -2$. Then the $y$-coordinate of the vertex is $y(-2)= 2(-2)^2 + 8(-2) + 12= 8 - 16 + 12= 4$. Therefore, the vertex is $(-2, 4)$. \pvspace{1.3cm}



% Quiz 17
\quizsol \textit{True/False}: The quadratic function $y= x^2 - 4x - 12$ factors as $(x - 6)(x + 2)$. \pspace

\sol \sol The statement is \textit{true}. One way of seeing this would be to expand $(x - 6)(x + 2)$,
	\[
	(x - 6)(x + 2)= x^2 + 2x - 6x - 12= x^2 - 4x - 12.
	\]
Alternatively, we can factor the polynomial $x^2 - 4x - 12$. First, we find the factors of $12$, which are only $1, 12$, and $2, 6$, and $3, 4$. Because the 12 is negative, the factors must have opposite signs. 
	\[
	\begin{aligned}
	1, -12&\colon \,-11 \\
	-1, 12&\colon \phantom{-..}11 \\
	2, -6&\colon \;\;-\!4 \\
	-2, 6&\colon \;\phantom{-..}4 \\
	3, -4&\colon \;\;-\!1 \\
	-3, 4&\colon \;\phantom{-..}1
	\end{aligned}
	\]
We want these signed factors to add to $-4$. Therefore, we want `factors' $2, -6$. Therefore, 
	\[
	x^2 - 4x - 12= (x + 2)(x - 6)
	\] \pvspace{1.3cm}



% Quiz 18
\quizsol \textit{True/False}: Let $D$ be the discriminant of a quadratic polynomial. If $D= -4$, then the polynomial factors `nicely.' \pspace

\sol The statement is \textit{false}. Recall that a quadratic polynomial factors `nicely' if and only if its discriminant is a perfect square. However, $D= -4$ is \textit{not} a perfect square. The number $4$ is a perfect square because $2^2= 4$. But there is no real number whose square is $-4$. Therefore, the quadratic polynomial must not factor `nicely.' Note that if $D < 0$, then the quadratic polynomial factors over $\mathbb{C}$. \pvspace{1.3cm}



% Quiz 19
\quizsol \textit{True/False}: $x^2 - 2x - 1= \big(x - (1 + \sqrt{2}) \big)\big(x - (1 - \sqrt{2}) \big)$ \pspace

\sol The statement is \textit{true}. For the quadratic function $x^2 - 2x - 1$, we have $a= 1$, $b= -2$, and $c= -1$. We can compute the discriminant to find $D= b^2 - 4ac= (-2)^2 - 4(1)(-1)= 4 + 4= 8$. Because $D= 8$ is not a perfect square, the quadratic polynomial $x^2 - 2x - 1$ does not factor `nicely.' However, \text{all} quadratic functions are factorable. To find the factorization, we find the roots of $x^2 - 2x - 1$, i.e. the solutions to $x^2 - 2x - 1= 0$, using the quadratic formula. We have\dots
	\[
	\begin{aligned}
	x&= \dfrac{-b \pm \sqrt{b^2 - 4ac}}{2a} \\
	&= \dfrac{-(-2) \pm \sqrt{(-2)^2 - 4(1)(-1)}}{2(1)} \\
	&= \dfrac{2 \pm \sqrt{4 + 4}}{2} \\
	&= \dfrac{2 \pm \sqrt{8}}{2} \\
	&= \dfrac{2 \pm \sqrt{4 \cdot 2}}{2} \\
	&= \dfrac{2 \pm 2 \sqrt{2}}{2} \\
	&= 1 \pm \sqrt{2}
	\end{aligned}
	\]
Then we have roots $r_1= 1 + \sqrt{2}$ and $r_2= 1 - \sqrt{2}$. Therefore, the factorization is
	\[
	x^2 - 2x - 1= a (x - r_1)(x - r_2)= \big(x - (1 + \sqrt{2}) \big)\big(x - (1 - \sqrt{2}) \big)
	\] \pvspace{1.3cm}



% Quiz 20
\quizsol \textit{True/False}: The quadratic formula can be used to solve $4x - 5x + 1= 0$. \pspace

\sol The statement is \textit{false}. The quadratic formula can be used to solve \textit{quadratic} equations. The equation $4x - 5x + 1= 0$ is linear. However, if the equation were $4x^2 - 5x + 1= 0$, then this quadratic formula could be used to solve this equation. We would have $a= 4, b= -5, c= 1$. Then\dots
	\[
	\begin{aligned}
	x&= \dfrac{-b \pm \sqrt{b^2 - 4ac}}{2a} \\
	&= \dfrac{-(-5) \pm \sqrt{(-5)^2 - 4(4)1}}{2(4)} \\
	&= \dfrac{5 \pm \sqrt{25 - 16}}{8} \\
	&= \dfrac{5 \pm \sqrt{9}}{8} \\
	&= \dfrac{5 \pm 3}{8} 
	\end{aligned}
	\]
But then either $x= (5 + 3)/8= 8/8= 1$ or $x= (5 - 3)/8= 2/8= 1/4$. 







\end{document}