\documentclass[11pt,letterpaper]{article}
\usepackage[lmargin=1in,rmargin=1in,bmargin=1in,tmargin=1in]{geometry}
\usepackage{style/quiz}
\usepackage{style/commands}

% -------------------
% Content
% -------------------
\begin{document}
\thispagestyle{title}

% Quiz 1
\quizsol \textit{True/False}: $9/3 + 2(3^2 + 10) - 8 + 4 \cdot 3= 45$ \pspace

\sol The statement is \textit{false}. To see this, we can simply follow the order of operations---using PEMDAS as a guide:
	\[
	\begin{aligned}
	9/3 + 2(3^2 + 10) - 8 + 4 \cdot 3&\stackrel{?}{=} 55 \\
	9/3 + 2(9 + 10) - 8 + 4 \cdot 3&\stackrel{?}{=} 55 \\
	9/3 + 2(19) - 8 + 4 \cdot 3&\stackrel{?}{=} 55 \\
	3 + 2(19) - 8 + 4 \cdot 3&\stackrel{?}{=} 55 \\ 
	3 + 38 - 8 + 4 \cdot 3&\stackrel{?}{=} 55 \\ 
	3 + 38 - 8 + 12&\stackrel{?}{=} 55 \\ 
	41 - 8 + 12&\stackrel{?}{=} 55 \\
	33 + 12&\stackrel{?}{=} 55 \\
	45&\neq 55 \\
	\end{aligned}
	\] \pvspace{1.5cm}



% Quiz 2
\quizsol \textit{True/False}: $\gcd(2^3 \cdot 3 \cdot 5, 2 \cdot 3^2 \cdot 7)= 2^3 \cdot 3^2 \cdot 5 \cdot 7$. \pspace

\sol The statement is \textit{false}. Remember given a prime factorization of the numbers, we find the gcd by choosing the \textit{smallest} powers of each prime that appears in the factorizations. So we should have $\gcd(2^3 \cdot 3 \cdot 5, 2 \cdot 3^2 \cdot 7)= 2 \cdot 3$. Instead, the largest power of each prime that appears in the factorizations was chosen which is how we compute the lcm. Therefore, we have $\lcm(2^3 \cdot 3 \cdot 5, 2 \cdot 3^2 \cdot 7)= 2^3 \cdot 3^2 \cdot 5 \cdot 7$. \pvspace{1.2cm}



% Quiz 3
\quizsol \textit{True/False}: $\dfrac{\dfrac{3}{10}}{\dfrac{12}{5}}= \dfrac{1}{8}$ \pspace

\sol The statement is \textit{true}. Note that division by a nonzero number is the same as multiplying by its reciprocal. So we have
	\[
	\dfrac{\phantom{-}\dfrac{3}{10}\phantom{-}}{\dfrac{12}{5}}= \dfrac{3}{10} \cdot \dfrac{5}{12}= \dfrac{\cancel{3}^1}{\cancel{10}^{\,2}} \cdot \dfrac{\cancel{5}^1}{\cancel{12}^{\,4}}= \dfrac{1}{8}
	\]
One can also rewrite the problem as\dots 
	\[
	\dfrac{\phantom{-}\dfrac{3}{10}\phantom{-}}{\dfrac{12}{5}}= \dfrac{3}{10} \div \dfrac{12}{5}
	\]
But then to divide, we multiply by the reciprocal and proceed as in the solution above. \pvspace{1.5cm}



% Quiz 4
\quizsol \textit{True/False}: The number $0.\overline{19}$ is rational. \pspace

\sol The statement is \textit{true}. Any real number with a decimal expansion that either terminates or repeats is a rational and hence can be expressed as $a/b$, where $a$ and $b$ are integers and $b \neq 0$. Moreover, every rational number, i.e. the $a/b$'s, have a decimal expansion that either terminates or repeats. We can even find a rational expression for $0.\overline{19}$:
	\begin{table}[!ht]
	\centering
	\begin{tabular}{rccc}
	 & $100r$ & $=$ & $19.191919191919\ldots$ \\ 
	$-$& $r$ & $=$ & $\phantom{0}0.191919191919\ldots$ \\ \hline
	& $99r$ & $=$ & $19$ 
	\end{tabular}
	\end{table} \par
But then $r= 0.\overline{19}= \dfrac{19}{99}$. \pvspace{1.5cm}

 

% Quiz 5
\quizsol \textit{True/False}: $\sqrt[3]{2^8 \cdot 3^3 \cdot 5^1 \cdot 7^5}= 2^2 \cdot 3^1 \cdot 7 \sqrt[3]{2^2 \cdot 5^1 \cdot 7^2}$ \pspace

\sol The statement is \textit{true}. There are two ways to think about this. First, we should write out the numbers and group them into threes and pull out/leave the terms appropriately:
	\[
	\sqrt[3]{2^8 \cdot 3^3 \cdot 5^1 \cdot 7^5}= \sqrt[3]{\underline{2 \cdot 2 \cdot 2} \cdot \underline{2 \cdot 2 \cdot 2} \cdot 2 \cdot 2 \cdot \underline{3 \cdot 3 \cdot 3} \cdot 5 \cdot \underline{7 \cdot 7 \cdot 7} \cdot 7 \cdot 7}= 2^2 \cdot 3^1 \cdot 7 \sqrt[3]{2^2 \cdot 5 \cdot 7^2}
	\]
Alternatively, we can use division. We know that $8/3$ is 2 with remainder 2, $3/3$ is 1 with remainder 0, $1/3$ is 0 with remainder 1, and $5/3$ is 1 with remainder 2. So we can pull out two 3's with 2 remaining, one 3 with 0 remaining, no 5's with 1 remaining, and two 7's with 2 remaining, which gives:
	\[
	\sqrt[3]{2^8 \cdot 3^3 \cdot 5^1 \cdot 7^5}= 2^2 \cdot 3^1 \cdot 7 \sqrt[3]{2^2 \cdot 5^1 \cdot 7^2}
	\]


















\end{document}