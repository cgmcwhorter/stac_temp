\documentclass[11pt,letterpaper]{article}
\usepackage[lmargin=1in,rmargin=1in,bmargin=1in,tmargin=1in]{geometry}
\usepackage{style/quiz}
\usepackage{style/commands}

% -------------------
% Content
% -------------------
\begin{document}
\thispagestyle{title}

% Quiz 1
\quizsol \textit{True/False}: $9/3 + 2(3^2 + 10) - 8 + 4 \cdot 3= 45$

\sol The statement is \textit{false}. To see this, we can simply follow the order of operations---using PEMDAS as a guide:
	\[
	\begin{aligned}
	9/3 + 2(3^2 + 10) - 8 + 4 \cdot 3&\stackrel{?}{=} 55 \\
	9/3 + 2(9 + 10) - 8 + 4 \cdot 3&\stackrel{?}{=} 55 \\
	9/3 + 2(19) - 8 + 4 \cdot 3&\stackrel{?}{=} 55 \\
	3 + 2(19) - 8 + 4 \cdot 3&\stackrel{?}{=} 55 \\ 
	3 + 38 - 8 + 4 \cdot 3&\stackrel{?}{=} 55 \\ 
	3 + 38 - 8 + 12&\stackrel{?}{=} 55 \\ 
	41 - 8 + 12&\stackrel{?}{=} 55 \\
	33 + 12&\stackrel{?}{=} 55 \\
	45&\neq 55 \\
	\end{aligned}
	\] \pspace



% Quiz 2
\quizsol \textit{True/False}: $\gcd(2^3 \cdot 3 \cdot 5, 2 \cdot 3^2 \cdot 7)= 2^3 \cdot 3^2 \cdot 5 \cdot 7$. \pspace

\sol The statement is \textit{false}. Remember given a prime factorization of the numbers, we find the gcd by choosing the \textit{smallest} powers of each prime that appears in the factorizations. So we should have $\gcd(2^3 \cdot 3 \cdot 5, 2 \cdot 3^2 \cdot 7)= 2 \cdot 3$. Instead, the largest power of each prime that appears in the factorizations was chosen which is how we compute the lcm. Therefore, we have $\lcm(2^3 \cdot 3 \cdot 5, 2 \cdot 3^2 \cdot 7)= 2^3 \cdot 3^2 \cdot 5 \cdot 7$.






















\end{document}