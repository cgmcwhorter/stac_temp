\documentclass[11pt,letterpaper]{article}
\usepackage[lmargin=1in,rmargin=1in,tmargin=1in,bmargin=1in]{geometry}
\usepackage{../style/homework}
\usepackage{../style/commands}
\setbool{quotetype}{true} % True: Side; False: Under
\setbool{hideans}{false} % Student: True; Instructor: False

% -------------------
% Content
% -------------------
\begin{document}

\homework{2: Due 09/24}{Time is money; Money is power; Power is pizza; Pizza is knowledge. Let's go!}{April Ludgate, Parks and Recreation}

% Problem 1
\problem{5} Give the definition of a rational number. \pvspace{0.8cm}

{\itshape A rational number is a real number of the form $\frac{a}{b}$, where $a$ and $b$ are integers and $b \neq 0$.} \pvspace{1.3cm}





% Problem 2
\problem{15} Find the prime factorization for each of the following integers: 
\begin{enumerate}[(a)]
\item $105= 3 \cdot 5 \cdot 7$ \vfill
\item $137= 137$ \vfill
\item $138= 2 \cdot 3 \cdot 23$ \vfill
\item $525= 3 \cdot 5^2 \cdot 7$ \vfill
\item $1320= 2^3 \cdot 3 \cdot 5 \cdot 11$ \vfill
\end{enumerate}





\newpage





% Problem 3
\problem{10} 
\begin{enumerate}[(a)]
\item By listing out all of the divisors of $24$ and $60$, find $\gcd(24, 60)$. \pvspace{1cm}

{\itshape
	\begin{table}[!ht]
	\centering
	\begin{tabular}{r||l}
	$24$ & $1, 2, 3, 4, 6, 8, \mathbf{12}, 24$ \\ \hline
	$60$ & $1, 2, 3, 4, 5, 6, 10, \mathbf{12}, 15, 20, 30, 60$
	\end{tabular}
	\end{table} \par
Therefore, $\gcd(24,60)= 12$.} \pvspace{1.7cm}

\item By listing out at least the first 10 multiples of both 8 and 15, find $\lcm(8, 15)$. \pvspace{1cm}
{\itshape
	\begin{table}[!ht]
	\centering
	\begin{tabular}{r||l}
	$8$ & $8, 16, 24, 32, 40, 48, 56, 64, 72, 80, 88, 96, 104, 112, \mathbf{120}, 128$ \\ \hline
	$15$ & $15, 30, 45, 60, 75, 90, 105, \mathbf{120}, 135, 150, 165, 180, 195, 210, 225, 240$
	\end{tabular}
	\end{table} \par
Therefore, $\lcm(8,15)= 120$.} \pvspace{1.7cm}
\end{enumerate}





% Problem 4
\problem{12} Use the `prime factorization method' to find the following: \pspace
\begin{enumerate}[(a)]
\item $\gcd(48, 80)= \gcd(2^4 \cdot 3, 2^4 \cdot 5)= 2^4= 16$ \vfill
\item $\gcd(10050, 483000)= \gcd(2 \cdot 3 \cdot 5^2 \cdot 67, 2^3 \cdot 3 \cdot 5^3 \cdot 7 \cdot 23)= 2 \cdot 3 \cdot 5^2= 150$ \vfill
\item $\gcd(30, 132, 245)= \gcd(2 \cdot 3 \cdot 5, 2^2 \cdot 3 \cdot 11, 5 \cdot 7^2)= 1$ \vfill
\item $\lcm(18,30)= \lcm(2 \cdot 3^2, 2 \cdot 3 \cdot 5)= 2 \cdot 3^2 \cdot 5= 90$ \vfill
\item $\lcm(462,4200)= \lcm(2 \cdot 3 \cdot 7 \cdot 11, 2^3 \cdot 3 \cdot 5^2 \cdot 7)= 2^3 \cdot 3 \cdot 5^2 \cdot 7 \cdot 11= 46200$ \vfill
\item $\lcm(30, 105, 44)= \lcm(2 \cdot 3 \cdot 5, 3 \cdot 5 \cdot 7, 2^2 \cdot 11)= 2^2 \cdot 3 \cdot 5 \cdot 7 \cdot 11= 4620$ \vfill
\end{enumerate}





\newpage





% Problem 5
\problem{4} Use \href{https://www.wolframalpha.com/}{WolframAlpha} to find the following: \pspace
\begin{enumerate}[(a)] 
\item $\gcd(1050, 29398)= 2$ \vspace{0.5cm}
\item $\gcd(6\,300\,020, 9\,119\,019)= 1$ \vspace{0.5cm}
\item $\lcm(4918, 24287)= 119\, 443\, 466$ \vspace{0.5cm}
\item $\lcm(705\,299\,330, 342\,828\,558)= 120\, 898\, 376\, 131\, 133\, 070$ \vspace{0.5cm}
\end{enumerate}



\vfill



% Problem 6
\problem{8} For each of the following rational numbers, if it is a mixed number, write is as an improper fraction, and if the number is an improper fraction, write it as a mixed number. \pspace
\begin{enumerate}[(a)] \itemsep=2ex
\item $1\frac{3}{5}= \frac{1(5) + 3}{5}= \dfrac{8}{5}$
\item $\frac{29}{4}= \frac{28 + 1}{4}= \frac{4(7) + 1}{7}= 7\frac{1}{4}$
\item $-2\frac{3}{7}= -\frac{2(7) + 3}{7}= -\frac{14 + 3}{7}= -\frac{17}{7}$
\item $\frac{101}{7}= \frac{98 + 3}{7}= \frac{14(7) + 3}{7}= 14\frac{3}{7}$
\end{enumerate} \pspace





\vfill





% Problem 7
\problem{5} Fully reduce the following rationals: \pspace
\begin{enumerate}[(a)] \itemsep=2ex
\item $\dfrac{18}{24}= \dfrac{9}{12}= \dfrac{3}{4}$
\item $\dfrac{80}{72}= \dfrac{40}{36}= \dfrac{20}{18}= \dfrac{10}{9}$
\item $\dfrac{18}{9}= \dfrac{2}{1}= 2$
\item $\dfrac{23}{52}= \dfrac{23}{52}$
\item $\dfrac{35}{120}= \dfrac{7}{24}$
\end{enumerate} \pspace





\vfill
\newpage





% Problem 8
\problem{10} Compute the following, being sure to simplify the answer completely: \pspace
\begin{enumerate}[(a)] \itemsep=2ex
\item $\dfrac{3}{4} + \dfrac{2}{5}= \dfrac{15}{20} + \dfrac{8}{20}= \dfrac{23}{20}$
\item $\dfrac{5}{6} - \dfrac{7}{4}= \dfrac{10}{12} - \dfrac{21}{12}= -\dfrac{11}{12}$
\item $\dfrac{1}{50} + \dfrac{201}{450}= \dfrac{9}{450} + \dfrac{201}{450}= \dfrac{210}{450}= \dfrac{21}{45}= \dfrac{7}{15}$
\item $-\dfrac{5}{6} + \dfrac{17}{9}= -\dfrac{15}{18} + \dfrac{34}{18}= \dfrac{19}{18}$
\item $\dfrac{3}{4} - \dfrac{5}{2} + \dfrac{7}{12}= \dfrac{9}{12} - \dfrac{30}{12} + \dfrac{7}{12}= -\dfrac{14}{12}= -\dfrac{7}{6}$
\end{enumerate}





\vfill





% Problem 9
\problem{10} Compute the following, being sure to simplify the answer completely:
\begin{enumerate}[(a)] \itemsep=2ex
\item $\dfrac{2}{3} \cdot -\dfrac{7}{5}= -\dfrac{14}{15}$
\item $\dfrac{5}{2} \cdot \dfrac{11}{20}= \dfrac{1}{2} \cdot \dfrac{11}{4}= \dfrac{11}{8}$
\item $\dfrac{7}{10} \cdot \dfrac{2}{21}= \dfrac{1}{5} \cdot \dfrac{1}{3}= \dfrac{1}{15}$
\item $\dfrac{\frac{2}{11}}{\frac{9}{10}}= \dfrac{2}{11} \cdot \dfrac{10}{9}= \dfrac{20}{99}$
\item $\dfrac{4}{15} \div \dfrac{3}{5}= \dfrac{4}{15} \cdot \dfrac{5}{3}= \dfrac{4}{3} \cdot \dfrac{1}{3}= \dfrac{4}{9}$
\end{enumerate}





\vfill





% Problem 10
\problem{10} Simplify the following expressions as much as possible:
\begin{enumerate}[(a)] \itemsep=2ex
\item $\sqrt{90}= \sqrt{2 \cdot 3^2 \cdot 5}= 3 \sqrt{2 \cdot 5}= 3 \sqrt{10}$
\item $\sqrt{200}= \sqrt{100 \cdot 2}= 10 \sqrt{2}$
\item $\sqrt[3]{360}= \sqrt[3]{2^3 \cdot 3^2 \cdot 5}= 2 \sqrt[3]{3^2 \cdot 5}= 2 \sqrt[3]{45}$
\item $\sqrt{\dfrac{45}{4}}= \sqrt{\dfrac{3^2 \cdot 5}{2^2}}= \dfrac{3 \sqrt{5}}{2}$
\item $\sqrt[4]{2^8 \cdot 3^5 \cdot 5 \cdot 7}= 2^2 \cdot 3 \sqrt[4]{3 \cdot 5 \cdot 7}= 12 \sqrt[4]{105}$
\end{enumerate}





\vfill
\newpage





% Problem 11
\problem{8} Rationalize the following: \pspace
\begin{enumerate}[(a)] \itemsep=2ex
\item $-\dfrac{6}{\sqrt{3}}= -\dfrac{6}{\sqrt{3}} \cdot \dfrac{\sqrt{3}}{\sqrt{3}}= -\dfrac{6 \sqrt{3}}{3}= -2\sqrt{3}$
\item $\dfrac{1}{\sqrt{7}}= \dfrac{1}{\sqrt{7}} \cdot \dfrac{\sqrt{7}}{\sqrt{7}}= \dfrac{\sqrt{7}}{7}$
\item $\sqrt{\dfrac{2}{3}}= \dfrac{\sqrt{2}}{\sqrt{3}}= \dfrac{\sqrt{2}}{\sqrt{3}} \cdot \dfrac{\sqrt{3}}{\sqrt{3}}= \dfrac{\sqrt{6}}{3}$
\item $\sqrt[3]{\dfrac{4}{8}}= \dfrac{\sqrt[3]{4}}{\sqrt[3]{8}}= \dfrac{\sqrt[3]{4}}{2}$
\end{enumerate} 





\vfill





% Problem 12
\problem{20} Determine if the following numbers are a natural number, integer, rational, irrational, or real. Place a check in the each spot that applies. 
        \begin{table}[!ht]
        \centering
        \begin{tabular}{rccccc}
         & Natural & Integer & Rational & Irrational & Real \\[0.3cm]
        $6$ & \usol{0.5cm}{\cmark} & \usol{0.5cm}{\cmark} & \usol{0.5cm}{\cmark} & \uans{1.3cm} & \usol{0.5cm}{\cmark} \\[0.3cm]
        $-5$ & \uans{1.3cm} & \usol{0.5cm}{\cmark} & \usol{0.5cm}{\cmark} & \uans{1.3cm} & \usol{0.5cm}{\cmark} \\[0.3cm]
        $0$ & \uans{1.3cm} & \usol{0.5cm}{\cmark} & \usol{0.5cm}{\cmark} & \uans{1.3cm} & \usol{0.5cm}{\cmark} \\[0.3cm]
        $3/2$ & \uans{1.3cm} & \uans{1.3cm} & \usol{0.5cm}{\cmark} & \uans{1.3cm} & \usol{0.5cm}{\cmark} \\[0.3cm]
	$24/8$ & \usol{0.5cm}{\cmark} & \usol{0.5cm}{\cmark} & \usol{0.5cm}{\cmark} & \uans{1.3cm} & \usol{0.5cm}{\cmark} \\[0.3cm]
        $0.\overline{174}$ & \uans{1.3cm} & \uans{1.3cm} & \usol{0.5cm}{\cmark} & \uans{1.3cm} & \usol{0.5cm}{\cmark} \\[0.3cm]
        $\sqrt{25}$ & \usol{0.5cm}{\cmark} & \usol{0.5cm}{\cmark} & \usol{0.5cm}{\cmark} & \uans{1.3cm} & \usol{0.5cm}{\cmark} \\[0.3cm]
        $1.65$ & \uans{1.3cm} & \uans{1.3cm} & \usol{0.5cm}{\cmark} & \uans{1.3cm} & \usol{0.5cm}{\cmark} \\[0.3cm]
        $\sqrt{2}$ & \uans{1.3cm} & \uans{1.3cm} & \uans{1.3cm} & \usol{0.5cm}{\cmark} & \usol{0.5cm}{\cmark} \\[0.3cm]
        $\pi$ & \uans{1.3cm} & \uans{1.3cm} & \uans{1.3cm} & \usol{0.5cm}{\cmark} & \usol{0.5cm}{\cmark} 
        \end{tabular}
        \end{table}





\vfill
\newpage





% Problem 13
\problem{8} Use \href{https://www.wolframalpha.com/}{WolframAlpha} to complete the following parts: 
\begin{enumerate}[(a)]
\item Find the prime factorization of $8533654013159899073434698$. \pspace

{\itshape Entering this integer into WolframAlpha, we find
	\[
	8533654013159899073434698= 2 \cdot 3 \cdot 7 \cdot 103 \cdot 563 \cdot 367457 \cdot 592531 \cdot 16092463
	\]
} \pvspace{0.8cm}


\item Determine whether the number $37124508045065437$ is prime. \pspace

{\itshape Entering this integer into WolframAlpha, we find that 37124508045065437 is prime.} \pvspace{2.5cm}


\item Recall that if $n$ is \textit{not} a prime number, then it must have at least one prime factor $p$ with $p \leq \sqrt{n}$. Therefore, if $n$ is prime, you only need to check divisibility for numbers up to $\sqrt{n}$. If you have not found a factor by then, the integer $n$ must be prime. For instance, if you are looking for prime factors of $137$, you only need check the numbers up to $\sqrt{137} \approx 11.705$, i.e. up to 11. If you have not found a prime factor by then, the number 137 is prime. If you were checking whether the integer from (b) was prime by hand, up to what integer do you need to check? \pspace

{\itshape We have\dots
	\[
	\sqrt{37124508045065437} \approx 192677212.05 
	\]
Therefore, one would have to check all the primes up to 192,677,212!
} \pvspace{1.3cm}


\item  Suppose you could check whether one integer was a factor of another in one second, regardless of the sizes of the number. How long would it then take to show that the number in (b) was prime by hand? [Hint: You can enter `XXXX seconds' in WolframAlpha and it will tell you how long this is in more reasonable units.] \pspace

{\itshape Checking all the integers from 2 to 192,677,212 (a total of 192,677,211 integers), this would take 192,677,211~seconds! Entering this into WolframAlpha, this is 53,521~hours, i.e. 2,230~days or 6.11~years!}
\end{enumerate} \vfill


%\printpoints
\end{document}