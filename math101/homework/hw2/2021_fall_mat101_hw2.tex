\documentclass[11pt,letterpaper]{article}
\usepackage[lmargin=1in,rmargin=1in,tmargin=1in,bmargin=1in]{geometry}
\usepackage{../style/homework}
\usepackage{../style/commands}
\setbool{quotetype}{true} % True: Side; False: Under
\setbool{hideans}{true} % Student: True; Instructor: False

% -------------------
% Content
% -------------------
\begin{document}

\homework{2: Due 09/24}{Time is money; Money is power; Power is pizza; Pizza is knowledge. Let's go!}{April Ludgate, Parks and Recreation}

% Problem 1
\problem{5} Give the definition of a rational number. 



\vfill



% Problem 2
\problem{15} Find the prime factorization for each of the following integers: 
\begin{enumerate}[(a)]
\item $105$ \vfill
\item $137$ \vfill
\item $137$ \vfill
\item $525$ \vfill
\item $1320$ \vfill
\end{enumerate}



\newpage



% Problem 3
\problem{10} 
\begin{enumerate}[(a)]
\item By listing out all of the divisors of $24$ and $60$, find $\gcd(24, 60)$. \vspace{5cm} 

\item By listing out at least the first 10 multiples of both 8 and 15, find $\lcm(8, 15)$. 
\end{enumerate} \vspace{5cm}





% Problem 4
\problem{12} Use the `prime factorization method' to find the following: \pspace
\begin{enumerate}[(a)]
\item $\gcd(48, 80)=$ \vfill
\item $\gcd(10050, 483000)= \gcd(2 \cdot 3 \cdot 5^2 \cdot 67, 2^3 \cdot 3 \cdot 5^3 \cdot 7 \cdot 23)=$ \vfill
\item $\gcd(30, 132, 245)= \gcd(2 \cdot 3 \cdot 5, 2^2 \cdot 3 \cdot 11, 5 \cdot 7^2)=$ \vfill
\item $\lcm(18,30)=$ \vfill
\item $\lcm(462,4200)= \lcm(2 \cdot 3 \cdot 7 \cdot 11, 2^3 \cdot 3 \cdot 5^2 \cdot 7)=$ \vfill
\item $\lcm(30, 105, 44)= \lcm(2 \cdot 3 \cdot 5, 3 \cdot 5 \cdot 7, 2^2 \cdot 11)=$ \vfill
\end{enumerate}



\newpage



% Problem 5
\problem{4} Use \href{https://www.wolframalpha.com/}{WolframAlpha} to find the following: \pspace
\begin{enumerate}[(a)] 
\item $\gcd(1050, 29398)=$ \vspace{0.5cm}
\item $\gcd(6\,300\,020, 9\,119\,019)=$ \vspace{0.5cm}
\item $\lcm(4918, 24287)=$ \vspace{0.5cm}
\item $\lcm(705\,299\,330, 342\,828\,558)=$ \vspace{0.5cm}
\end{enumerate}



\vfill



% Problem 6
\problem{8} For each of the following rational numbers, if it is a mixed number, write is as an improper fraction, and if the number is an improper fraction, write it as a mixed number. \pspace
\begin{enumerate}[(a)] \itemsep=2ex
\item $1\frac{3}{5}=$
\item $\frac{29}{4}=$
\item $-2\frac{3}{7}=$
\item $\frac{101}{7}=$
\end{enumerate} \pspace



\vfill



% Problem 7
\problem{5} Fully reduce the following rationals: \pspace
\begin{enumerate}[(a)] \itemsep=2ex
\item $\dfrac{18}{24}=$
\item $\dfrac{80}{72}=$
\item $\dfrac{18}{9}=$
\item $\dfrac{23}{52}=$
\item $\dfrac{35}{120}=$
\end{enumerate} \pspace



\vfill
\newpage



% Problem 8
\problem{10} Compute the following, being sure to simplify the answer completely: \pspace
\begin{enumerate}[(a)] \itemsep=2ex
\item $\dfrac{3}{4} + \dfrac{2}{5}=$
\item $\dfrac{5}{6} - \dfrac{7}{4}=$
\item $\dfrac{1}{50} + \dfrac{201}{450}=$
\item $-\dfrac{5}{6} + \dfrac{17}{9}=$
\item $\dfrac{3}{4} - \dfrac{5}{2} + \dfrac{7}{12}=$
\end{enumerate}




\vfill



% Problem 9
\problem{10} Compute the following, being sure to simplify the answer completely:
\begin{enumerate}[(a)] \itemsep=2ex
\item $\dfrac{2}{3} \cdot -\dfrac{7}{5}=$
\item $\dfrac{5}{2} \cdot \dfrac{11}{20}=$
\item $\dfrac{7}{10} \cdot \dfrac{2}{21}=$
\item $\dfrac{\frac{2}{11}}{\frac{9}{10}}=$
\item $\dfrac{4}{15} \div \dfrac{3}{5}=$
\end{enumerate}



\vfill



% Problem 10
\problem{10} Simplify the following expressions as much as possible:
\begin{enumerate}[(a)] \itemsep=2ex
\item $\sqrt{90}=$
\item $\sqrt{200}=$
\item $\sqrt[3]{360}=$
\item $\sqrt{\dfrac{45}{4}}=$
\item $\sqrt[4]{2^8 \cdot 3^5 \cdot 5 \cdot 7}=$
\end{enumerate}



\vfill
\newpage



% Problem 11
\problem{8} Rationalize the following: \pspace
\begin{enumerate}[(a)] \itemsep=2ex
\item $-\dfrac{6}{\sqrt{3}}=$
\item $\dfrac{1}{\sqrt{7}}=$
\item $\sqrt{\dfrac{2}{3}}=$
\item $\sqrt[3]{\dfrac{4}{8}}=$
\end{enumerate} 



\vfill



% Problem 12
\problem{20} Determine if the following numbers are a natural number, integer, rational, irrational, or real. Place a check in the each spot that applies. 
        \begin{table}[!ht]
        \centering
        \begin{tabular}{rccccc}
         & Natural & Integer & Rational & Irrational & Real \\[0.3cm]
        $6$ & \uans{1.3cm} & \uans{1.3cm} & \uans{1.3cm} & \uans{1.3cm} & \uans{1.3cm} \\[0.3cm]
        $-5$ & \uans{1.3cm} & \uans{1.3cm} & \uans{1.3cm} & \uans{1.3cm} & \uans{1.3cm} \\[0.3cm]
        $0$ & \uans{1.3cm} & \uans{1.3cm} & \uans{1.3cm} & \uans{1.3cm} & \uans{1.3cm} \\[0.3cm]
        $3/2$ & \uans{1.3cm} & \uans{1.3cm} & \uans{1.3cm} & \uans{1.3cm} & \uans{1.3cm} \\[0.3cm]
	$24/8$ & \uans{1.3cm} & \uans{1.3cm} & \uans{1.3cm} & \uans{1.3cm} & \uans{1.3cm} \\[0.3cm]
        $0.\overline{174}$ & \uans{1.3cm} & \uans{1.3cm} & \uans{1.3cm} & \uans{1.3cm} & \uans{1.3cm} \\[0.3cm]
        $\sqrt{25}$ & \uans{1.3cm} & \uans{1.3cm} & \uans{1.3cm} & \uans{1.3cm} & \uans{1.3cm} \\[0.3cm]
        $1.65$ & \uans{1.3cm} & \uans{1.3cm} & \uans{1.3cm} & \uans{1.3cm} & \uans{1.3cm} \\[0.3cm]
        $\sqrt{2}$ & \uans{1.3cm} & \uans{1.3cm} & \uans{1.3cm} & \uans{1.3cm} & \uans{1.3cm} \\[0.3cm]
        $\pi$ & \uans{1.3cm} & \uans{1.3cm} & \uans{1.3cm} & \uans{1.3cm} & \uans{1.3cm}     
        \end{tabular}
        \end{table}



\vfill
\newpage



% Problem 13
\problem{8} Use \href{https://www.wolframalpha.com/}{WolframAlpha} to complete the following parts: 
\begin{enumerate}[(a)]
\item Find the prime factorization of $8533654013159899073434698$. \vfill
\item Determine whether the number $37124508045065437$ is prime. \vfill
\item Recall that if $n$ is \textit{not} a prime number, then it must have at least one prime factor $p$ with $p \leq \sqrt{n}$. Therefore, if $n$ is prime, you only need to check divisibility for numbers up to $\sqrt{n}$. If you have not found a factor by then, the integer $n$ must be prime. For instance, if you are looking for prime factors of $137$, you only need check the numbers up to $\sqrt{137} \approx 11.705$, i.e. up to 11. If you have not found a prime factor by then, the number 137 is prime. If you were checking whether the integer from (b) was prime by hand, up to what integer do you need to check? \vfill
\item  Suppose you could check whether one integer was a factor of another in one second, regardless of the sizes of the number. How long would it then take to show that the number in (b) was prime by hand? [Hint: You can enter `XXXX seconds' in WolframAlpha and it will tell you how long this is in more reasonable units.]
\end{enumerate} \vfill



%\printpoints
\end{document}