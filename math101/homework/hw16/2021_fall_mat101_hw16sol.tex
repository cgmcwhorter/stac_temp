\documentclass[11pt,letterpaper]{article}
\usepackage[lmargin=1in,rmargin=1in,tmargin=1in,bmargin=1in]{geometry}
\usepackage{../style/homework}
\usepackage{../style/commands}
\setbool{quotetype}{true} % True: Side; False: Under
\setbool{hideans}{false} % Student: True; Instructor: False

% -------------------
% Content
% -------------------
\begin{document}

\homework{15: Due 11/16}{I wanted to buy a candle holder, but the store didn't have one. So I got a cake.}{Mitch Hedberg}

% Problem 1
\problem{10} Write the function $f(x)= -5(2^{x-1})$ in the form $y= Ab^x$ for some $A$ and $b$. Show all your work. \pspace

\sol 
	\[
	f(x)= -5(2^{x-1})= -5(2^{-1} \cdot 2^x)= -5 \cdot \dfrac{1}{2} \cdot 2^x= -\frac{5}{2}\; (2^x)
	\] \pspace
Therefore, $f(x)= -\frac{5}{2} (2^x)$, where here $A= -\frac{5}{2}$ and $b= 2$. 





\newpage





% Problem 2
\problem{10} Write the function $f(x)= 6(3^{2x+1})$ in the form $y= Ab^x$ for some $A$ and $b$. Show all your work. \pspace

\sol 
	\[
	f(x)= 6(3^{2x+1})= 6(3^1 \cdot 3^{2x})= 6 \cdot 3 \cdot 3^{2x}= 18 (3^2)^x= 18(9^x)
	\] \pspace
Therefore, $f(x)= 18(9^x)$, where here $A= 18$ and $b= 9$. 





\newpage





% Problem 3
\problem{10} Solve the equation $4^{x+1}= \dfrac{1}{16}$. Show all your work. \pspace

\sol
	\[
	\begin{aligned}
	4^{x + 1}&= \dfrac{1}{16} \\[0.3cm]
	4^{x + 1}&= \dfrac{1}{4^2} \\[0.3cm] 
	4^{x + 1}&= 4^{-2}
	\end{aligned}
	\] \pspace
Because the bases on both sides are equal, we must have $x + 1= -2$. But then $x= -3$. 





\newpage





% Problem 4
\problem{10} Solve the equation $25^{1-x} + 3= 4$. Show all your work. \pspace

\sol
	\[
	\begin{aligned}
	25^{1-x} + 3&= 4 \\[0.3cm]
	25^{1 - x}&= 1 \\[0.3cm]
	25^{1 - x}&= 25^0
	\end{aligned}
	\] \pspace
Because the bases on both sides are equal, we must have $1 - x= 0$. But then $x= 1$. 


%\printpoints
\end{document}