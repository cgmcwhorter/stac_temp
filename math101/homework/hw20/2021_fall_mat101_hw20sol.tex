\documentclass[11pt,letterpaper]{article}
\usepackage[lmargin=1in,rmargin=1in,tmargin=1in,bmargin=1in]{geometry}
\usepackage{../style/homework}
\usepackage{../style/commands}
\setbool{quotetype}{true} % True: Side; False: Under
\setbool{hideans}{false} % Student: True; Instructor: False

% -------------------
% Content
% -------------------
\begin{document}

\homework{20: Due 12/10}{Give me some of your tots!}{Napoleon Dynamite, \par Napoleon Dynamite}

% Problem 1
\problem{10} Determine if the system of equations below has a solution. If it does, find it; if not, explain why. \pspace
	\[
	\begin{aligned}
	x + y&= 5 \\
	x - y&= 9
	\end{aligned}
	\] \pspace

\sol This a system of linear equations. The system will have a solution if and only if the lines intersect. But this will only happen if they are not parallel. So we find the slopes of each line.
	\[
	\begin{aligned}
	x + y&= 5 &\quad\quad x - y&= 9 \\
	y&= -x + 5 & y&= x - 9
	\end{aligned}
	\]
The slope of the first line is $m_1= -1$ while the slope of the second line is $m_2= 1$. Because $m_1 \neq m_2$, the lines are not parallel. But then the lines intersect so that there is a solution to the system of equations. Now we find the solution by using both substitution and elimination. 

If we use substitution, we can solve for $y$ in the first equation. This yields $y= 5 - x$. Using this in the second equation, we have\dots
	\[
	\begin{aligned}
	x - y&= 9 \\
	x - (5 - x)&= 9 \\
	x - 5 + x&= 9 \\
	2x - 5&= 9 \\
	2x&= 14 \\
	x&= 7
	\end{aligned}
	\]
But then we have $y= 5 - 7= -2$. Therefore, the solution is $(7, -2)$. \pspace

Using elimination, suppose we eliminate $y$. Adding the equations, we find\dots
	\[
	\begin{aligned}
	x + y&= 5 \\
	x - y&= 9 \\ \hline
	2x&= 14 \\
	x&= 7
	\end{aligned}
	\] 
Using this in the first equation, we find
	\[
	\begin{aligned}
	x + y&= 5 \\
	7 + y&= 5 \\
	y&= -2
	\end{aligned}
	\]
Therefore, the solution is $(7, -2)$. 





\newpage





% Problem 2
\problem{10} Determine if the system of equations below has a solution. If it does, find it; if not, explain why. \pspace
	\[
	\begin{aligned}
	15x - 6y&= 10 \\
	-5x + 2y&= -8
	\end{aligned}
	\] \pspace

\sol This a system of linear equations. The system will have a solution if and only if the lines intersect. But this will only happen if they are not parallel. So we find the slopes of each line.
	\[
	\begin{aligned}
	15x - 6y&= 10 &\quad\quad -5x + 2y&= -8 \\
	-6y&= -15x + 10 & 2y&= 5x - 8 \\
	y&= \frac{5}{2}\,x + \frac{5}{3} & y&= \frac{5}{2}\,x - 4
	\end{aligned}
	\]
The slope of the first line is $m_1= \frac{5}{2}$ while the slope of the second line is $m_2= \frac{5}{2}$. Because $m_1= m_2$, the lines are parallel. But then the lines do not intersect so that there is no solution to the system of equations. 





\newpage





% Problem 3
\problem{10} Determine if the system of equations below has a solution. If it does, find it; if not, explain why. 
	\[
	\begin{aligned}
	5x + 3y&= 7 \\
	3x - 2y&= -11
	\end{aligned}
	\] 

\sol This a system of linear equations. The system will have a solution if and only if the lines intersect. But this will only happen if they are not parallel. So we find the slopes of each line.
	\[
	\begin{aligned}
	5x + 3y&= 7 &\quad\quad 3x - 2y&= -11 \\
	3y&= -5x + 7 & -2y&= -3x - 11 \\
	y&= -\frac{5}{3}\,x + \frac{7}{3} & y&= \frac{3}{2}\,x + \frac{11}{2}
	\end{aligned}
	\]
The slope of the first line is $m_1= -\frac{5}{3}$ while the slope of the second line is $m_2= \frac{3}{2}$. Because $m_1 \neq m_2$, the lines are not parallel. But then the lines intersect so that there is a solution to the system of equations. Now we find the solution by using both substitution and elimination. 

If we use substitution, we can solve for $y$ in the first equation. This yields $y= -\frac{5}{3}\,x + \frac{7}{3}$. Using this in the second equation, we have\dots
	\[
	\begin{aligned}
	3x - 2y&= -11 \\
	3x - 2 \left( -\frac{5}{3}\,x + \frac{7}{3} \right)&= -11 \\
	3x + \frac{10}{3}\, x - \frac{14}{3}&= -11 \\
	3 \left( 3x + \frac{10}{3}\, x - \frac{14}{3} \right)&= -11 \cdot 3 \\
	9x + 10x - 14&= -33 \\
	19x - 14&= -33 \\
	19x&= -19 \\
	x&= -1
	\end{aligned}
	\]
But then we have $y= -\frac{5}{3} \cdot -1 + \frac{7}{3}= \frac{5}{3} + \frac{7}{3}= \frac{12}{3}= 4$. Therefore, the solution is $(-1, 4)$. \pspace

Using elimination, suppose we eliminate $y$. Multiplying the first equation by $2$ and the second equation by $3$ and adding, we find
	\[
	\begin{aligned}
	10x + 6y&= 14 \\
	9x - 6y&= -33 \\ \hline
	19x&= -19 \\
	x&= -1
	\end{aligned}
	\] 
Using this in the first equation, we find
	\[
	\begin{aligned}
	5x + 3y&= 7 \\
	5(-1) + 3y&= 7 \\
	3y - 5&= 7 \\
	3y&= 12 \\
	y&= 4
	\end{aligned}
	\]
Therefore, the solution is $(-1, 4)$. 





\newpage





% Problem 4
\problem{10} Determine if the system of equations below has a solution. If it does, find it; if not, explain why. 
	\[
	\begin{aligned}
	5x - 6y&= 3 \\
	2x + 3y&= 3
	\end{aligned}
	\] 

\sol This a system of linear equations. The system will have a solution if and only if the lines intersect. But this will only happen if they are not parallel. So we find the slopes of each line.
	\[
	\begin{aligned}
	5x - 6y&= 3 &\quad\quad 2x + 3y&= 3 \\
	-6y&= -5x + 3 & 3y&= -2x + 3 \\
	y&= \frac{5}{6}\,x - \frac{1}{2} & y&= -\frac{2}{3}\,x + 1
	\end{aligned}
	\]
The slope of the first line is $m_1= \frac{5}{6}$ while the slope of the second line is $m_2= \frac{2}{3}$. Because $m_1 \neq m_2$, the lines are not parallel. But then the lines intersect so that there is a solution to the system of equations. Now we find the solution by using both substitution and elimination. 

If we use substitution, we can solve for $y$ in the first equation. This yields $y= \frac{5}{6}\,x - \frac{1}{2}$. Using this in the second equation, we have\dots
	\[
	\begin{aligned}
	2x + 3y&= 3 \\
	2x + 3 \left( \frac{5}{6}\,x - \frac{1}{2} \right)&= 3 \\
	2x + \frac{5}{2}\,x - \frac{3}{2}&= 3 \\
	2 \left( 2x + \frac{5}{2}\,x - \frac{3}{2} \right)&= 3 \cdot 2 \\
	4x + 5x - 3&= 6 \\
	9x - 3&= 6 \\
	9x&= 9 \\
	x&= 1
	\end{aligned}
	\]
But then we have $y= \frac{5}{6} \cdot 1 - \frac{1}{2}= \frac{5}{6} - \frac{1}{2}= \frac{5}{6} - \frac{3}{6}= \frac{2}{6}= \frac{1}{3}$. Therefore, the solution is $(1, \frac{1}{3})$. \pspace

Using elimination, suppose we eliminate $y$. Multiplying the second equation by $2$ and adding, we find
	\[
	\begin{aligned}
	5x - 6y&= 3 \\
	4x + 6y&= 6 \\ \hline
	9x&= 9 \\
	x&= 1
	\end{aligned}
	\] 
Using this in the first equation, we find
	\[
	\begin{aligned}
	5x - 6y&= 3 \\
	5(1) - 6y&= 3 \\
	5 - 6y&= 3 \\
	-6y&= -2 \\
	y&= \frac{1}{3}
	\end{aligned}
	\]
Therefore, the solution is $(1, \frac{1}{3})$. 


%\printpoints
\end{document}