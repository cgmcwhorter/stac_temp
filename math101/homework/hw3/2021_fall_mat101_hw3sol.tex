\documentclass[11pt,letterpaper]{article}
\usepackage[lmargin=1in,rmargin=1in,tmargin=1in,bmargin=1in]{geometry}
\usepackage{../style/homework}
\usepackage{../style/commands}
\setbool{quotetype}{true} % True: Side; False: Under
\setbool{hideans}{false} % Student: True; Instructor: False

% -------------------
% Content
% -------------------
\begin{document}

\homework{3: Due 09/24}{People make fun of the guy who stays home every night doing nothing. But the truth is that guy is a genius.}{Ted Mosby, How I Met Your Mother}

% Problem 1
\problem{2} What are the approximate values of $\pi$ and $e$? \pvspace{1cm}

{\itshape We have $\pi \approx 3.141592654$ and $e \approx 2.718281828$} \pvspace{1.5cm}





% Problem 2
\problem{2} You may have recalled in school that $\pi \approx \frac{22}{7}$. Is it possible to find integers $a$ and $b$ such that $\pi= \frac{a}{b}$? Explain. \pvspace{1.4cm}

{\itshape No, it is not possible to find integers $a, b$ so that $\pi= \frac{a}{b}$. The number $\pi$ is irrational, which means it cannot be expressed as a ratio of integers.} \pvspace{1.7cm}





% Problem 3
\problem{8} Express the following rational numbers as a decimal. Show all your work. \pspace

\begin{minipage}[t]{0.49\textwidth}
\begin{enumerate}
\item[(a)] $\dfrac{5}{4}: \longdivision{5}{4}$. Then $\frac{5}{4}= 1.25$. \vfill
\item[(b)] $-\dfrac{1}{8}: \longdivision{1}{8}$. Then $-\frac{1}{8}= 0.125$. \vfill
\end{enumerate}
\end{minipage}
\begin{minipage}[t]{0.49\textwidth}
\begin{enumerate}
\item[(c)] $\dfrac{170}{9}: \longdivision{170}{9}$. Then $\frac{170}{9}= 18.\overline{8}$. \vfill
\item[(d)] $\dfrac{13}{99}: \longdivision{13}{99}$. Then $\frac{13}{99}= 0.\overline{13}$. \vfill
\end{enumerate}
\end{minipage}





\newpage





% Problem 4
\problem{8} Express the following decimals as rational numbers, reducing your rational expression as much as possible and showing all your work: \pspace
\begin{enumerate}[(a)]
\item $3.0= \dfrac{3}{1}$ \pvspace{1cm}
\item $-1.5= -1\frac{1}{2}= -\frac{3}{2}$ \pvspace{1cm}
\item $1.25= 1 + 0.25= \frac{4}{4} + \frac{1}{4}= \frac{5}{4}$ \pvspace{1cm}
\item $-0.94= - \frac{94}{100}= -\frac{47}{50}$ \pvspace{1cm}
\end{enumerate}





% Problem 5
\problem{9} Express the following decimals as a rational number, reducing your rational expression as much as possible and showing all your work:
{\itshape\small
\begin{enumerate}[(a)]
\item $0.7777\overline{7}$ 
	\begin{table}[!ht]
	\centering\small
	\begin{tabular}{ccc}
	$10N$ & $=$ & $7.7777\overline{7}$ \\ 
	$N$ & $=$ & $0.7777\overline{7}$ \\ \hline
	$9N$ & $=$ & $7$ \\
	& $N= \frac{7}{9}$ & 
	\end{tabular}
	\end{table}
	

\item $0.212121\overline{21}$ 
	\begin{table}[!ht]
	\centering\small
	\begin{tabular}{ccc}
	$100N$ & $=$ & $21.212121\overline{21}$ \\ 
	$N$ & $=$ & $0.212121\overline{21}$ \\ \hline
	$99N$ & $=$ & $21$ \\
	& $N= \frac{21}{99}$ & \\[0.1cm]
	& $N= \frac{7}{33}$ & 
	\end{tabular}
	\end{table}

\item $0.25555\overline{5}$ \pspace

Note that $0.25555\overline{5}= 0.2 + 0.05555\overline{5}= \frac{2}{10} + 0.05555\overline{5}= \frac{1}{5} + 0.05555\overline{5}$. 
	\begin{table}[!ht]
	\centering\small
	\begin{tabular}{ccc}
	$100N$ & $=$ & $5.55555\overline{5}$ \\ 
	$N$ & $=$ & $0.05555\overline{5}$ \\ \hline
	$99N$ & $=$ & $5.5$ \\
	$99N$ & $=$ & $\frac{55}{10}$ \\
	$99N$ & $=$ & $\frac{11}{2}$ \\
	& $N= \frac{11}{99(2)}$ & \\[0.1cm]
	& $N= \frac{1}{18}$ & 
	\end{tabular}
	\end{table} \par
Then $0.25555\overline{5}= \frac{1}{5} + \frac{1}{18}= \frac{18}{90} + \frac{5}{90}= \frac{23}{90}$.
\end{enumerate}
}





\newpage





% Problem 6
\problem{3} Suppose two paints have to be mixed in a $5 : 6$ ratio. If you want to use all of the second paint you have in the mix and you have 5.3~gallons of the paint left, how many gallons of the first paint should you add to the mix? \pspace

{\itshape Suppose $x$ is the number of gallons needed to be used. Then we have
	\[
	\begin{aligned}
	\dfrac{5}{6}&= \dfrac{x}{5.3} \\
	x&= \dfrac{5}{6} \cdot 5.3 \\
	x&= 4.42
	\end{aligned}	
	\]
Therefore, you should use $4.42$~gallons of the first paint. 
} \pvspace{0.5cm}





% Problem 7
\problem{8} Find the following:
\begin{enumerate}[(a)]
\item 40\% of 60 $= 60(0.40)= 24$ \pvspace{1cm}
\item 17\% of 55 $= 55(0.17)= 9.35$\pvspace{1cm}
\item 120 increased by 12\% $= 120(1+0.12)= 120(1.12)= 134.4$ \pvspace{1cm}
\item 89 decreased by 5\% $= 89(1- 0.05)= 89(0.95)= 84.55$ \pvspace{1cm}
\end{enumerate} \pspace





% Problem 8
\problem{6} Water is flowing into a tank at a rate of 12~gallons per minute. The tank can hold 1500~gallons and currently contains 86~gallons of water. 
\begin{enumerate}[(a)]
\item How many minutes until the tank overflows? \pspace

{\itshape There are currently 86~gallons of water in the tank. Then there is $1500 - 86= 1414$~gallons left in the tank. But then it will take $\frac{1414}{12}= 117.83$~minutes until the tank overflows, i.e. approximately 1~hour and 58~minutes.} \pvspace{0.6cm}

\item If the time is currently 5:00~pm, at what time will the tank overflow? \pspace

{\itshape It will take 117.83~minutes for the tank to overflow. Then this is $\frac{117.83}{60}= 1.96$~hours, i.e. 1~hour and another 0.96~hours. We know 0.96~hours is $60(0.96)= 57.6 \approx 58$~minutes. Then it will take 1~hour and 58~minutes for the tank to overflow. But then the tank will overflow at 6:58~pm.
}
\end{enumerate}





\newpage





% Problem 9
\problem{6} The travel distance between Sparkill, NY and Boston, MA is 203~miles. The speed limit on this highway is 65~mph. 
\begin{enumerate}[(a)]
\item How long will it take to travel from Sparkill, NY to Boston, MA using this highway? 

{\itshape Let $x$ be the time you spend driving at 65~mph. Then
	\[
	\begin{aligned}
	65x&= 203 \\
	x&= \dfrac{203}{65} \text{ hours} \\
	x&= 3.12 \text{ hours}
	\end{aligned}
	\]
Therefore, it will take 3.12~hours, i.e. 3~hours and 7~minutes to make the drive at this speed.}


\item Is it possible to make the drive in 3~hours or less without speeding? Explain. \pspace

{\itshape No. From (a), we know that driving at this speed it will take over 3~hours to make the drive. To make it in less time, one would have to go over 65~mph, i.e. one would have to speed.}
\end{enumerate} \pvspace{0.6cm}





% Problem 10
\problem{8} Convert the following:
\begin{enumerate}[(a)]
\item 165~lbs to kg [1~kg $=$ 2.205~lb]
	\begin{table}[!ht]
	\centering
	\begin{tabular}{r|r}
	165~lb & 1~kg  \\ \hline
	& 2.205~lb 
	\end{tabular}
	= 74.83~kg
	\end{table} \vfill

\item 16~km to miles [1~mi $=$ 1.609~km] 
	\begin{table}[!ht]
	\centering
	\begin{tabular}{r|r}
	16~km & 1~mi \\ \hline
	& 1.609~km
	\end{tabular}
	= 9.94~mi
	\end{table} \vfill

\item 3.6~ft/s to mph [5280~ft $=$ 1~mi] 
	\begin{table}[!ht]
	\centering
	\begin{tabular}{r|r|r|r}
	3.6~ft & 1~mi & 60~s & 60~min\\ \hline
	1~s & 5280~ft & 1~min & 1~hr
	\end{tabular}
	= 2.45~mph
	\end{table} \vfill

\item 9.8~m/s$^2$ to ft/hr$^2$ [1~m $=$ 3.2808~ft] 
	\begin{table}[!ht]
	\centering
	\begin{tabular}{r|r|r|r|r}
	9.8~km & 1000~m & 3.2808~ft & (60~s)$^2$ & (60~min)$^2$ \\ \hline
	1~s$^2$ & 1~km & 1~m &  (1~min)$^2$ & (1~hr)$^2$
	\end{tabular}
	= 416687846400~ft/hr
	\end{table} \vfill
\end{enumerate}





\newpage





% Problem 11
\problem{8} Write the following numbers as complex numbers:
\begin{enumerate}[(a)]
\item $5= 5 + 0i$ \vfill
\item $\sqrt{-4}= \sqrt{4}i= 2i= 0 + 2i$ \vfill
\item $\sqrt{16}= 4= 4 + 0i$ \vfill
\item $1 - \sqrt{-24}= 1 - \sqrt{24}\,i= 1 - \sqrt{4 \cdot 6}\,i= 1 - 2\sqrt{6}\,i$ \vfill
\end{enumerate}





% Problem 12
\problem{12} Compute the following:
\begin{enumerate}[(a)]
\item $(1 + 3i) + (-6 + 7i)= (1 - 6) + (3 + 7)i= -5 + 10i$ \vfill
\item $2i - (8 - 3i)= 2i - 8 + 3i= -8 + 5i$ \vfill
\item $(1 - i)(6 + i)= 6 + i - 6i - i^2= 6 - 5i - (-1)= 6 - 5i + 1= 7 - 5i$ \vfill
\item $(7 + 9i)^2= (7 + 9i)(7 + 9i)= 49 + 63i + 63i + 81i^2= 49 + 126i - 81= -32 + 126i$ \vfill
\item $\dfrac{10 + i}{1 - i}= \dfrac{10 + i}{1 - i} \cdot \dfrac{1 + i}{1 + i}= \dfrac{(10 + i)(1 + i)}{(1 - i)(1 + i)}= \dfrac{10+10i + i + i^2}{1 + i - i - i^2}= \dfrac{9 + 11i}{1 - (-1)}= \dfrac{9 +11i}{2}= \frac{9}{2} + \frac{11}{2}\,i$ \vfill
\item $\dfrac{16 - 12i}{(2i)^2}= \dfrac{16 - 12i}{4i^2}= \dfrac{16 - 12i}{4(-1)}= \dfrac{16 - 12i}{-4}= -4 + 3i$ \vfill
\end{enumerate}


%\printpoints
\end{document}