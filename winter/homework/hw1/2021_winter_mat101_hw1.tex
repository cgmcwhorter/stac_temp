\documentclass[11pt,letterpaper]{article}
\usepackage[lmargin=1in,rmargin=1in,tmargin=1in,bmargin=1in]{geometry}
\usepackage{../style/homework}
\usepackage{../style/commands}
\setbool{quotetype}{false} % True: Side; False: Under
\setbool{hideans}{true} % Student: True; Instructor: False

% -------------------
% Content
% -------------------
\begin{document}

\homework{1: Due 01/04}{I knew exactly what to do. But in a much more real sense, I had no idea what to do.}{Michael Scott, The Office}

% Problem 1
\problem{10} Showing all your work, compute each of the following:
\begin{enumerate}[(a)]
\item $20 - 5(3 - 6)$ 
\item $30/3 - 3 + (-1)^3$
\item $5(1 - 3)^2/4$
\item $4 - 2(3^2 - 10)$
\item $\dfrac{10 - 2^2}{5 - 6} + 10/2 - 5$
\end{enumerate}



\newpage



% Problem 2
\problem{10} Showing all your work, compute each of the following:
\begin{enumerate}[(a)]
\item $6/2(1 + 2)$
\item $6/(2(1+2))$
\item $\dfrac{10^2/5 - 6 + 4 \cdot 2}{(5 + 1)(3 - 8)}$
\item $\dfrac{10 - 6}{-2^2} + 12$
\item $5(-1)^3 - 4(-1)^3 + 3 \cdot 16/4$
\end{enumerate}



\newpage



% Problem 3 
\problem{10} Showing all your work, find the prime factorizations of the following integers:
\begin{enumerate}[(a)]
\item $60$
\item $61$
\item $132$
\item $125$
\item $147$
\end{enumerate}



\newpage



% Problem 4
\problem{10} Using the relationship between factors of integers and their square root, explain why the integer 79 is prime. \pspace



\newpage



% Problem 5
\problem{10} Compute each of the following by finding the divisors/multiples of the given integers:
\begin{enumerate}[(a)]
\item $\gcd(12, 20)$
\item $\gcd(18, 27)$
\item $\lcm(12, 30)$
\item $\lcm(8, 15)$
\end{enumerate}



\newpage



% Problem 6
\problem{10} Use the prime factorizations of the given integers to compute each of the following:
\begin{enumerate}[(a)]
\item $\gcd(124, 144)$
\item $\lcm(128, 146)$
\item $\gcd(2^3 \cdot 3 \cdot 7^8 \cdot 17^6, 2^2 \cdot 3^5 \cdot 5^2 \cdot 11^9)$
\item $\lcm(2^3 \cdot 3 \cdot 7^8 \cdot 17^6, 2^2 \cdot 3^5 \cdot 5^2 \cdot 11^9)$
\end{enumerate}



\newpage



% Problem 7
\problem{10} Lena is making gift baskets for an event. Each basket will contain at least one jam jar, candy box, and gift card. Suppose there are 120 jam jars, 280 candy boxes, and 360 gift cards with which to make the baskets. 
\begin{enumerate}[(a)]
\item If she is trying to make the most number of gift baskets, how many can she make? Explain. 
\item If she is trying to put the most number of items in each basket with each basket having the same number of each item, how many baskets can she make? Explain. 
\item If she were making gift baskets that each contained 4 jam jars, 1 candy box, and 6 gift cards each, what is the fewest number of each item she would have to purchase if she was going to purchase an equal number of each item.
\end{enumerate} \pspace



\newpage



% Problem 8 
\problem{10} Showing all your work and being sure to reduce as much as possible, compute the following:
\begin{enumerate}[(a)]
\item $\dfrac{4}{5} + \dfrac{9}{7}$
\item $\dfrac{9}{4} - \dfrac{7}{12}$
\item $\dfrac{6}{35} + \dfrac{4}{15}$
\item $\dfrac{17}{24} - \dfrac{13}{40}$
\item $4\frac{1}{2} + 3\frac{1}{3}$
\end{enumerate}



\newpage



% Problem 9
\problem{10} Showing all your work and being sure to reduce as much as possible, compute the following:
\begin{enumerate}[(a)]
\item $\dfrac{9}{10} \cdot \dfrac{-6}{33}$
\item $\dfrac{12}{35} \cdot \dfrac{25}{2}$
\item $\dfrac{100}{3} \cdot \dfrac{27}{25}$
\item $\dfrac{-11}{68} \cdot \dfrac{4}{55}$
\item $2 \frac{2}{3} \cdot -4 \frac{5}{8}$
\end{enumerate}



\newpage



% Problem 10 
\problem{10} Showing all your work and being sure to reduce as much as possible, compute the following:
\begin{enumerate}[(a)]
\item $\dfrac{\dfrac{10}{21}}{\dfrac{6}{7}}$
\item $\dfrac{\dfrac{5}{4}}{\dfrac{3}{10}}$
\item $\dfrac{-\dfrac{9}{44}}{\dfrac{84}{11}}$
\item $\dfrac{\dfrac{3}{5}}{\dfrac{10}{9}}$
\item $\dfrac{12\frac{1}{3}}{3\frac{1}{2}}$
\end{enumerate}


\end{document}