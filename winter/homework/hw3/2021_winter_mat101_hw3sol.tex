\documentclass[11pt,letterpaper]{article}
\usepackage[lmargin=1in,rmargin=1in,tmargin=1in,bmargin=1in]{geometry}
\usepackage{../style/homework}
\usepackage{../style/commands}
\setbool{quotetype}{false} % True: Side; False: Under
\setbool{hideans}{false} % Student: True; Instructor: False

% -------------------
% Content
% -------------------
\begin{document}

\homework{3: Due 01/06}{Who's the one who didn't bring lice into the office? Meredith. Sure, I gave everybody pink eye once, and my ex keyed a few of their cars, and yeah, I BM'ed in the shredder on New Years. But I didn't bring the lice in. That was all Pam.}{Meredith Palmer, The Office}

% Problem 1
\problem{10} Compute the following:
\begin{enumerate}[(a)]
\item 60\% of 77
\item 32\% of 1230
\item 89\% of 151
\item 140\% of 290
\item 225\% of 45
\end{enumerate} \pspace

\sol
\begin{enumerate}[(a)]
\item 
	\[
	77(0.60)= 46.2
	\] \pspace

\item 
	\[
	1230(0.32)= 393.6
	\] \pspace

\item 
	\[
	151(0.89)= 134.39
	\] \pspace

\item 
	\[
	290(1.40)= 406
	\] \pspace

\item 
	\[
	45(2.25)= 101.25
	\]
\end{enumerate}



\newpage



% Problem 2
\problem{10} Compute the following:
\begin{enumerate}[(a)]
\item 57 increased by 15\%
\item 630 decreased by 40\%
\item 485 decreased by 96\%
\item 110 increased by 120\%
\item 78 increased by 230\%
\end{enumerate} \pspace

\sol
\begin{enumerate}[(a)]
\item 
	\[
	57(1 + 0.15)= 57(1.15)= 65.55
	\] \pspace

\item 
	\[
	630(1 - 0.40)= 630(0.60)= 378
	\] \pspace

\item 
	\[
	485(1 - 0.96)= 485(0.04)= 19.4
	\] \pspace

\item 
	\[
	110(1 + 1.20)= 110(2.20)= 242
	\] \pspace

\item 
	\[
	78(1 + 2.30)= 78(3.30)= 257.4
	\]
\end{enumerate}



\newpage



% Problem 3
\problem{10} Suppose you invest \$5,600 in an account that earns 6\% annual interest compounded quarterly. 
\begin{enumerate}[(a)]
\item Write a function which gives the amount of money in the account after $t$ years.
\item Find the amount of money in the account after 7 years. 
\item Find the amount of money in the account after 27~months. 
\end{enumerate} \pspace

\sol
\begin{enumerate}[(a)]
\item Let $M(t)$ be the amount of money in the account after $t$ years. Because the interest is compounded discretely, we know that $M(t)= P(1 + \frac{r}{k})^{kt}$, where $P$ is the principal, $r$ is the yearly interest (as a decimal), and $k$ is the compounding rate per year. We know also that $P= 5600$, $r= 0.06$, and $k= 4$ (because the compounding is quarterly, i.e. 4 times per year). Therefore, 
	\[
	M(t)= P \left(1 + \dfrac{r}{k} \right)^{kt}= 5600 \left( 1 + \dfrac{0.06}{4} \right)^{4t}= 5600(1.015)^{4t}
	\] \pspace

\item Because we are looking for the amount of money after 7~years, we know that $t= 7$.
	\[
	M(7)= 5600(1.015)^{4 \cdot 7}= 5600(1.015)^{28}= 5600(1.51722)= \$8496.44
	\] \pspace

\item We need the time, $t$, to be in years. We know that 27~months is $\frac{27}{12}= 2.25$~years. But then we have\dots
	\[
	M(2.25)= 5600(1.015)^{4 \cdot 2.25}= 5600(1.015)^9= 5600(1.14339)= \$6402.98
	\]
\end{enumerate}



\newpage



% Problem 4
\problem{10} Suppose you invested money in an account which compounds interest discretely. The amount of money in the account after $t$ years is given by $M(t)= 683(1.0175)^{2t}$.
\begin{enumerate}[(a)]
\item How much was initially invested in the account?
\item How often is the interest compounded?
\item What is the interest rate on the account?
\end{enumerate} \pspace

\sol
\begin{enumerate}[(a)]
\item Because the interest is compounded discretely, we know that $M(t)= P(1 + \frac{r}{k})^{kt}$, where $P$ is the principal, $r$ is the yearly interest (as a decimal), and $k$ is the compounding rate per year. Because we know that $M(t)= 683(1.0175)^{2t}$, we must have\dots
	\[
	\begin{aligned}
	M(t)= P \left( 1 + \frac{r}{k} \right)^{kt}&= 683(1.0175)^{2t} \\
	P&= 683 \\
	1 + \dfrac{r}{k}&= 1.0175 \\
	kt&= 2t
	\end{aligned}
	\]
So we immediately know that $P= 683$, i.e. the initial amount invested in the account was \$683. From the last equation, we know that $kt= 2t$ for all $t$, i.e. $k= 2$. Therefore, the interest is being compounded twice per year, i.e. semiannually. But then we have\dots
	\[
	\begin{aligned}
	1 + \dfrac{r}{k}&= 1.0175 \\[0.3cm]
	1 + \dfrac{r}{2}&= 1.0175 \\[0.3cm]
	\dfrac{r}{2}&= 0.0175 \\[0.3cm]
	r&= 0.0175 \cdot 2 \\[0.3cm]
	r&= 0.035
	\end{aligned}
	\]
Therefore, the yearly interest rate is 3.5\%. \pspace

\item From the work in part (a), we found that the interest is compounded twice per year, i.e. semiannually. \pspace

\item From the work in part (a), we know that the interest rate is 3.5\%. 
\end{enumerate}



\newpage



% Problem 5
\problem{10} A carton containing a dozen eggs costs \$3.26.
\begin{enumerate}[(a)]
\item What is the cost per egg?
\item Approximately how much should 75 eggs cost?
\item How many eggs could one purchase for \$27.30?
\end{enumerate} \pspace

\sol
\begin{enumerate}[(a)]
\item 
	\[
	\dfrac{\$3.26}{12 \text{ eggs}}= \dfrac{3.26}{12} \text{ \$/egg}= 0.271667 \text{ \$/egg}
	\] \pspace

\item 
	\[
	\begin{aligned}
	\dfrac{0.271667\; \$}{1 \text{ egg}}&= \dfrac{x}{75 \text{ eggs}} \\[0.3cm]
	x&= 75 \text{ eggs} \cdot (0.271667 \text{ \$/egg }) \\[0.3cm]
	x&= \$20.375 \approx \$20.38
	\end{aligned}
	\]
	
	\begin{center} {\bfseries OR} \end{center}
	\[
	\begin{aligned}
	0.271667 \text{ \$/egg} \cdot 75 \text{ eggs}= \$20.375 \approx \$20.38
	\end{aligned}
	\] \pspace

\item 
	\[
	\begin{aligned}
	\dfrac{0.271667\; \$}{1 \text{ egg}}&= \dfrac{\$27.30}{x \text{ eggs}} \\[0.3cm]
	0.271667 \text{ \$/egg } x&= \$27.30 \\[0.3cm]
	x&= \dfrac{\$27.30}{0.271667 \text{ \$/egg}} \\[0.3cm]
	x&= 100.491 \text{ eggs} \approx 100 \text{ eggs}
	\end{aligned}
	\]
	
	\begin{center} {\bfseries OR} \end{center}
	\[
	\begin{aligned}
	\dfrac{\$27.30}{0.271667 \text{ \$/egg}}= 100.491 \text{ eggs} \approx 100 \text{ eggs}
	\end{aligned}
	\] \pspace
\end{enumerate}



\newpage



% Problem 6
\problem{10} Assume you have been driving on a highway for 3~hours and have traveled 191~miles.
\begin{enumerate}[(a)]
\item What is your average rate of speed?
\item Assuming you continue at this rate of speed, how far will you have traveled 4.5~hours from now?
\item Continuing at this speed, how long would it take to travel an additional 500~miles?
\end{enumerate} \pspace

\sol
\begin{enumerate}[(a)]
\item 
	\[
	\dfrac{191 \text{ miles}}{3 \text{ hours}}= \dfrac{191}{3} \text{ mph}= 63.67 \text{ mph}
	\] \pspace

\item 
	\[
	\begin{aligned}
	d&= vt \\[0.3cm]
	d&= 63.67 \text{ mph} \cdot 4.5 \text{ hours} \\[0.3cm]
	d&= 286.515 \text{ miles}
	\end{aligned}
	\] \pspace

\item 
	\[
	\begin{aligned}
	d&= vt \\[0.3cm]
	500 \text{ miles}&= 63.67 \text{ mph} \cdot t \\[0.3cm]
	t&= \dfrac{500 \text{ miles}}{63.67 \text{ mph}} \\[0.3cm]
	t&= 7.85 \text{ hours}
	\end{aligned}
	\]
\end{enumerate}



\newpage



% Problem 7
\problem{10} The bones on a certain species of bird are approximately proportional to its wingspan. For one of its subspecies, a bird with a wing bone length of 1.5~ft has a wing span of 8.3~ft. 
\begin{enumerate}[(a)]
\item If you find the remains of another bird of this subspecies with a bone length of 2.2~ft, how much would you estimate its wingspan was?
\item If a bird of this subspecies has a wingspan of 7.9~ft, how long would you estimate the bone in its wing to be?
\item Suppose you find a different species of bird with a wingspan of 3.8~ft and bone length of 0.79~ft. Do these two species of birds have approximately the same proportion of wingspan to bone length?
\end{enumerate} \pspace

\sol
\begin{enumerate}[(a)]
\item 
	\[
	\begin{aligned}
	\dfrac{1.5 \text{ ft}}{8.3 \text{ ft}}&= \dfrac{2.2 \text{ ft}}{x} \\[0.3cm]
	1.5x&= 8.3 \cdot 2.2 \\[0.3cm]
	1.5x&= 18.26 \\[0.3cm]
	x&= 12.1733 \text{ ft}
	\end{aligned}
	\] \pspace

\item 
	\[
	\begin{aligned}
	\dfrac{1.5 \text{ ft}}{8.3 \text{ ft}}&= \dfrac{x}{7.9 \text{ ft}} \\[0.3cm]
	x&= 7.9 \cdot \dfrac{1.5}{8.3} \\[0.3cm]
	x&= 7.9 \cdot 0.180723 \\[0.3cm]
	x&= 1.42771 \text{ ft}
	\end{aligned}
	\] \pspace
 
\item 
	\[
	\begin{aligned}
	\dfrac{1.5 \text{ ft}}{8.3 \text{ ft}} &\stackrel{?}{\approx} \dfrac{0.79 \text{ ft}}{3.8 \text{ ft}} \\[0.3cm]
	0.180723 &\stackrel{?}{\approx} 0.207895
	\end{aligned}
	\] 
Because these are not `approximately' equal (0.207895 is an approximately 15\% increase from the 0.180723), these birds to not have `approximately' the same proportions. 
\end{enumerate}



\newpage



% Problem 8
\problem{10} Convert the following:
\begin{enumerate}[(a)]
\item 15~ft to m [1~ft $= 0.3048$~m]
\item 15~ft to km [1000~m $=$ 1~km]
\item 6400~ft to km
\end{enumerate} \pspace

\sol
\begin{enumerate}[(a)]
\item \phantom{.} \par
	\begin{table}[!ht]
	\centering
	\begin{tabular}{r|r}
	15~ft & 0.3048~m \\ \hline
		 & 1~ft  
	\end{tabular}
	= 4.572~m
	\end{table}
 
\item \phantom{.} \par
	\begin{table}[!ht]
	\centering
	\begin{tabular}{r|r|r}
	15~ft & 0.3048~m & 1~km \\ \hline
		& 1~ft 	   & 1000~m
	\end{tabular}
	= 0.004572~km
	\end{table} 

\item \phantom{.} \par
	\begin{table}[!ht]
	\centering
	\begin{tabular}{r|r|r}
	6400~ft & 0.3048~m & 1~km \\ \hline
		     & 1~ft 	     & 1000~m
	\end{tabular}
	= 1.95072~km
	\end{table} 
\end{enumerate}



\newpage



% Problem 9
\problem{10} Convert the following:
\begin{enumerate}[(a)]
\item 17~hours to days [24~hours $=$ 1~day]
\item 27~oz to tons [16~oz $=$ 1~lb; 2000~lb $=$ 1~ton]
\item 2.3~mi to in [1~mi $=$ 5280~ft; 1~ft $=$ 12~in]
\end{enumerate} \pspace

\sol
\begin{enumerate}[(a)]
\item \phantom{.} \par
	\begin{table}[!ht]
	\centering
	\begin{tabular}{r|r}
	17~hours & 1~day \\ \hline
		        & 24~hours 
	\end{tabular}
	= 0.708333~days
	\end{table}

\item \phantom{.} \par
	\begin{table}[!ht]
	\centering
	\begin{tabular}{r|r|r}
	27~oz & 1~lb & 1~ton \\ \hline
		   & 16~oz & 2000~lb
	\end{tabular}
	= 0.00084375~tons= $8.4375 \cdot 10^{-4}$~tons
	\end{table}

\item  \phantom{.} \par
	\begin{table}[!ht]
	\centering
	\begin{tabular}{r|r|r}
	2.3~mi & 5280~ft & 12~in \\ \hline
		    & 1~mi     & 1~ft
	\end{tabular}
	= 145728~in
	\end{table}
\end{enumerate}



\newpage



% Problem 10
\problem{10} Convert the following:
\begin{enumerate}[(a)]
\item 400~ft$^2$ to in$^2$ [12~in $=$ 1~ft]
\item 60~mph to ft per second [1~mi $=$ 5280~ft; 1~hr $=$ 3600~s]
\item 9.8~m/s$^2$ to ft/hr$^2$ [1~m $=$ 3.28084~ft; 3600~s $=$ 1~hr]
\end{enumerate} \pspace

\sol
\begin{enumerate}[(a)]
\item \phantom{.} \par
	\begin{table}[!ht]
	\centering
	\begin{tabular}{r|r|r}
	400~ft$^2$ & 12~in & 12~in \\ \hline
			  & 1~ft    & 1~ft
	\end{tabular}
	= 57600~in$^2$
	\end{table}

\item \phantom{.} \par
	\begin{table}[!ht]
	\centering
	\begin{tabular}{r|r|r}
	60~mi     & 5280~ft & 1~hr \\ \hline
	1~hr	       & 1~mi     & 3600~s
	\end{tabular}
	= 88~ft/s
	\end{table}

\item \phantom{.} \par
	\begin{table}[!ht]
	\centering
	\begin{tabular}{r|r|r|r}
	9.8~m     & 3.28084~ft & 3600~s & 3600~s \\ \hline
	1~s$^2$ & 1~m		   & 1~hr      & 1~hr
	\end{tabular}
	= 416692926.72~ft/s$^2 \approx$ $4.16693 \cdot 10^8$~ft/s$^2$
	\end{table}
\end{enumerate}


\end{document}