\documentclass[12pt,letterpaper]{exam}
\usepackage[lmargin=1in,rmargin=1in,tmargin=1in,bmargin=1in]{geometry}
\usepackage{../style/exams}

% -------------------
% Course & Exam Information
% -------------------
\newcommand{\course}{MAT 101: Exam 1}
\newcommand{\term}{Winter -- 2021}
\newcommand{\examdate}{01/07/2021}
\newcommand{\timelimit}{85 Minutes}

\setbool{hideans}{false} % Student: True; Instructor: False

% -------------------
% Content
% -------------------
\begin{document}

\examtitle
\instructions{Write your name on the appropriate line on the exam cover sheet. This exam contains \numpages\ pages (including this cover page) and \numquestions\ questions. Check that you have every page of the exam. Answer the questions in the spaces provided on the question sheets. Be sure to answer every part of each question and show all your work.} 
\scores
\bottomline
\newpage

% ---------
% Questions
% ---------
\begin{questions}

% Question 1
\newpage
\question[6] Showing all your work, compute the following: \pspace
\begin{parts}
\part $84 + 84 - 16 \cdot 0 + 12= 84 + 84 - 0 + 12= 168 - 0 + 12 = 168 + 12= 180$ \vfill
\part $12/6 - 3(4 - 6)^3= 12/6 - 3(-2)^3= 12/6 - 3(-8)= 2 - 3(-8)= 2 + 24= 26$ \vfill
\part $6 + 10/5 \cdot 2 - 7= 6 + 2 \cdot 2 - 7= 6 + 4 - 7= 10 - 7= 3$ \vfill
\end{parts}





% Question 2
\newpage
\question[6] Showing all your work, find the prime factorization of each of the following integers: \pspace
\begin{parts}
\part $120= 12 \cdot 10= (4 \cdot 3) \cdot (2 \cdot 5)= (2 \cdot 2 \cdot 3) \cdot (2 \cdot 5)= 2^3 \cdot 3 \cdot 5$ \vfill
\part $17= 17$ \vfill
\part $51= 3 \cdot 17$ \vfill
\end{parts}





% Question 3
\newpage
\question[8] Showing all your work, find the following: \pspace
\begin{parts}
\part $\gcd(10, 15)= \gcd(2 \cdot 5, 3 \cdot 5)= 5$ \vfill
\part $\lcm(10, 15)= \lcm(2 \cdot 5, 3 \cdot 5)= 2 \cdot 3 \cdot 5= 30$ \vfill
\part $\gcd(2^3 \cdot 3^2 \cdot 5,\, 2^2 \cdot 3 \cdot 7)= 2^2 \cdot 3= 12$ \vfill
\part $\lcm(2^3 \cdot 3^2 \cdot 5,\, 2^2 \cdot 3 \cdot 7)= 2^3 \cdot 3^2 \cdot 5 \cdot 7= 2520$ \vfill
\end{parts}





% Question 4
\newpage
\question[8] Showing all your work and being sure to simplify as much as possible, compute the following: \pspace
\begin{parts} 
\part $\dfrac{5}{6} - \dfrac{3}{8}= \dfrac{20}{24} - \dfrac{9}{24}= \dfrac{20 - 9}{24}= \dfrac{11}{24}$ \vfill
\part $\dfrac{3}{5} - \dfrac{7}{2}= \dfrac{6}{10} - \dfrac{35}{10}= \dfrac{6 - 35}{10}= -\dfrac{29}{10}$ \vfill
\part $\dfrac{10}{18} \cdot \dfrac{6}{25}= \dfrac{2 \cdot 5}{2 \cdot 3^2} \cdot \dfrac{2 \cdot 3}{5 \cdot 5}= \dfrac{\cancel{2} \cdot \cancel{5}}{\cancel{2} \cdot 3^{\cancel{2}}} \cdot \dfrac{2 \cdot \cancel{3}}{\cancel{5} \cdot 5}= \dfrac{2}{3 \cdot 5}= \dfrac{2}{15}$ \vfill
\part $\dfrac{\phantom{-}\frac{14}{15}\phantom{-}}{\frac{21}{10}}= \dfrac{14}{15} \cdot \dfrac{10}{21}= \dfrac{2 \cdot 7}{3 \cdot 5} \cdot \dfrac{2 \cdot 5}{3 \cdot 7}= \dfrac{2 \cdot \cancel{7}}{3 \cdot \cancel{5}} \cdot \dfrac{2 \cdot \cancel{5}}{3 \cdot \cancel{7}}= \dfrac{2 \cdot 2}{3 \cdot 3}= \dfrac{4}{9}$ \vfill
\end{parts}





% Question 5
\newpage
\question[8] Simplify the following as much as possible, being sure to have no negative exponents in your expression: \pspace
\begin{parts} 
\part $\dfrac{x^3 y^3 z}{x y^6 z^{-2}}= \dfrac{x^3 y^3 z z^2}{xy^6}= \dfrac{x^2 z^3}{y^3}$ \vfill
\part $\left( \dfrac{(xy^2)^3}{x^5y} \right)^{-1}= \dfrac{x^5y}{(xy^2)^3}= \dfrac{x^5y}{x^3 y^6}= \dfrac{x^2}{y^5}$ \vfill
\part $\left( \dfrac{x^{-2}}{y^{-3}} \right)^{5}= \dfrac{x^{-10}}{y^{-15}}= \dfrac{y^{15}}{x^{10}}$ \vfill
\part $\dfrac{15 x^{-3} y^2}{5xy}= \dfrac{15 y^2}{5xy x^3}= \dfrac{3y}{x^4}$ \vfill
\end{parts}





% Question 6
\newpage
\question[6] Simplify the following as much as possible: \pspace
\begin{parts}
\part $\sqrt{45}= \sqrt{9 \cdot 5}= 3 \sqrt{5}$ \vfill
\part $\sqrt[3]{24}= \sqrt[3]{8 \cdot 3}= 2 \sqrt[3]{3}$ \vfill
\part $\sqrt[4]{2^8 \cdot 3^9 \cdot 5^4 \cdot 7}= 2^2 \cdot 3^2 \cdot 5 \sqrt[4]{3 \cdot 7}= 180 \sqrt[4]{21}$ \vfill
\end{parts}





% Question 7 & 8 
\newpage
\question[6] Convert the following numbers from scientific to decimal notation: \pspace
\begin{parts}
\part $1.5 \cdot 10^0= 1.5$ \vfill
\part $4.35 \cdot 10^{-3}= 0.00435$ \vfill
\part $6.7 \cdot 10^5= 670000$ \vfill
\end{parts}

\vfill

\question[6] Convert the following numbers from decimal to scientific notation: \pspace
\begin{parts}
\part $0.0004= 4.0 \cdot 10^{-4}$ \vfill
\part $5.4= 5.4 \cdot 10^0$ \vfill
\part $1540000= 1.54 \cdot 10^6$ \vfill
\end{parts} \vfill





% Question 9
\newpage
\question[6] Convert $0.181818\overline{18}$ from a decimal to a fraction. \pspace

	\begin{table}[!ht]
	\centering\small
	\begin{tabular}{ccc}
	$100N$ & $=$ & $18.181818\overline{18}$ \\[0.3cm] 
	$N$ & $=$ & $0.181818\overline{18}$ \\[0.3cm] \hline
	$99N$ & $=$ & $18$ \\[0.3cm]
	& $N= \dfrac{18}{99}$ & \\[0.3cm]
	& $N= \dfrac{2}{11}$ & 
	\end{tabular}
	\end{table} 





% Question 10 & 11
\newpage
\question[6] Showing all your work, compute the following: \pspace
\begin{parts}
\part 35\% of 68 \hfill $68(0.35)= 23.8$ \hfill\phantom{.} \vfill
\part 50\% of 19.4 \hfill $19.4(0.50)= 9.7$ \hfill\phantom{.} \vfill
\part 131\% of 46 \hfill $46(1.31)= 60.26$ \hfill\phantom{.} \vfill
\end{parts}

\vfill

\question[6] Showing all your work, compute the following: \pspace
\begin{parts}
\part 650 increased by 40\% \hfill $650(1 + 0.40)= 650(1.40)= 910$ \hfill\phantom{.} \vfill
\part 84 decreased by 45\% \hfill $84(1 - 0.45)= 84(0.55)= 46.2$ \hfill\phantom{.} \vfill
\part 93 increased by 160\% \hfill $93(1 + 1.60)= 93(2.60)= 241.8$ \hfill\phantom{.} \vfill
\end{parts} \vfill





% Question 12
\newpage
\question[6] A gas pump takes 1.6~minutes to pump 12~gallons of gas. Assuming the rate at which the pump works remains constant, find how long it will take to pump 26~gallons of gas. \pspace

	\[
	\begin{aligned}
	\dfrac{1.6 \text{ min}}{12 \text{ gal}}&= \dfrac{x}{26 \text{ gal}} \\[0.3cm]
	x&= 26 \text{ gal} \cdot \dfrac{1.6 \text{ min}}{12 \text{ gal}} \\[0.3cm]
	x&= 3.47 \text{ min}
	\end{aligned}
	\]





% Question 13
\newpage
\question[6] Convert 23~gallons per square foot to liters per square inches. [1~gallon $=$ 3.79 liters; 12~inches $=$ 1 foot.] \pspace

	\begin{table}[!ht]
	\centering
	\begin{tabular}{r|r|r|r}
	23~gal  & 3.79~L & 1~ft    & 1~ft \\ \hline
	1 ft$^2$ & 1~gal  & 12~in & 12~in
	\end{tabular}
	= 0.605347~gal/ft$^2$
	\end{table}





% Question 14
\newpage
\question[6] Suppose you invest \$970 in an account which earns 5\% annual interest compounded monthly. Find the amount of money in the account after 6~years. Be sure to give the formula for the interest at time $t$ and to show all your work. \pspace

	\[
	\begin{aligned}
	M(t)&= P \left(1 + \dfrac{r}{k} \right)^{kt} \\[0.3cm]
	M(t)&= 970 \left(1 + \dfrac{0.05}{12} \right)^{12t} \\[0.3cm]
	M(t)&= 970 (1.00417)^{12t} \\[0.3cm]
	\\ \\
	M(6)&= 970 (1.00417)^{12 \cdot 6} \\[0.3cm]
	M(6)&= 970 (1.00417)^{72} \\[0.3cm]
	M(6)&= 970(1.34934) \\[0.3cm]
	M(6)&= \$1308.86
	\end{aligned}
	\]





% Question 15
\newpage
\question[10] Suppose $f(x)$ and $g(x)$ are functions whose values are given in the table below.
	\begin{table}[!ht]
	\centering
	\begin{tabular}{|r||c|c|c|c|c|c|} \hline
	$x$ & $-4$ & $-1$ & $\phantom{-}0$ & $\phantom{-}1$ & $\phantom{-}3$ & $\phantom{-}5$ \\ \hline
	$f(x)$ & $\phantom{-}0$ & $\phantom{-}4$ & $\phantom{-}8$ & $-1$ & $\phantom{-}7$ & $\phantom{-}4$ \\ \hline
	$g(x)$ & $-2$ & $-6$ & $-3$ & $\phantom{-}7$ & $\phantom{-}2$ & $\phantom{-}3$ \\ \hline
	\end{tabular}
	\end{table} \par
Compute the following: \pspace
        \begin{parts}
        \part $g(1)= 7$ \vfill
        \part $(f + g)(-1)= f(-1) + g(-1)= 4 + (-6)= -2$ \vfill
        \part $(fg)(5)= f(5) \cdot g(5)= 4 \cdot 3= 12$ \vfill
        \part $(f \circ g)(5)= f(g(5))= f(3)= 7$ \vfill
        \part $(g \circ f)(1)= g(f(1))= g(-1)= -6$ \vfill
        \end{parts}


\end{questions}
\end{document}