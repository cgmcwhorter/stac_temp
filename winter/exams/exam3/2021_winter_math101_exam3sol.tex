\documentclass[12pt,letterpaper]{exam}
\usepackage[lmargin=1in,rmargin=1in,tmargin=1in,bmargin=1in]{geometry}
\usepackage{../style/exams}

% -------------------
% Course & Exam Information
% -------------------
\newcommand{\course}{MAT 101: Exam 3}
\newcommand{\term}{Winter -- 2021}
\newcommand{\examdate}{01/21/2021}
\newcommand{\timelimit}{80 Minutes}

\setbool{hideans}{false} % Student: True; Instructor: False

% -------------------
% Content
% -------------------
\begin{document}

\examtitle
\instructions{Write your name on the appropriate line on the exam cover sheet. This exam contains \numpages\ pages (including this cover page) and \numquestions\ questions. Check that you have every page of the exam. Answer the questions in the spaces provided on the question sheets. Be sure to answer every part of each question and show all your work.} 
\scores
\bottomline
\newpage

% ---------
% Questions
% ---------
\begin{questions}

% Question 1
\newpage
\question[6] Determine if the following system of equations has a solution. Be sure to fully justify your answer.
	\[
	\left\{
	\begin{aligned}
	y= 2x &- 5 \\
	6x - 3y&= -12
	\end{aligned}\right.
	\] \pspace

{\itshape Solving for $y$ in the second equation, we find\dots
	\[
	\begin{aligned}
	6x - 3y&= -12 \\[0.3cm]
	-3y&= -6x - 12 \\[0.3cm]
	y&= 2x + 4 	
	\end{aligned}
	\]
Then the given system of equations is equivalent to\dots
	\[
	\left\{
	\begin{aligned}
	y&= 2x - 5 \\
	y&= 2x + 4
	\end{aligned}\right.
	\]
The slope of the first line is $m_1= 2$ and the slope of the second line is $m_2= 2$. Because $m_1= m_2$, the two given lines are parallel. Therefore, the lines do not intersect. But then there cannot be a solution to the given system of equations. 

\begin{center} {\bfseries OR} \end{center}

If there is a solution, say $(x, y)$, then $x$ and $y$ satisfy both the equations. In particular, from the first equation, we know that $y= 2x - 5$. Using this in the second equation, we have\dots
	\[
	\begin{aligned}
	6x - 3y&= -12 \\[0.3cm]
	6x - 3(2x - 5)&= -12 \\[0.3cm]
	6x - 6x + 15&= -12 \\[0.3cm]
	15&= -12
	\end{aligned}
	\]
But as this is impossible, we have a contradiction. Therefore, there cannot be a solution to this system of equations. 
}



% Question 2
\newpage
\question[6] Being sure to fully justify your answer, determine if $(-5, 4)$ is a solution to the following system of equations:
	\[
	\begin{aligned}
	2x + 5y&= 10 \\
	-3x - 6y&= -9
	\end{aligned}
	\] \pspace

{\itshape If $(-5, 4)$ is a solution to the system of equations, then it satisfies the given equations, which we can check:
	\[
	\begin{aligned}
	2x + 5y&= 10 &\qquad\qquad -3x - 6y&= -9 \\[0.3cm]
	2(-5) + 5(4)&\stackrel{?}{=} 10 &\qquad\qquad -3(-5) - 6(4)&\stackrel{?}{=} -9 \\[0.3cm]
	-10 + 20&\stackrel{?}{=} 10 &\qquad\qquad 15 - 24&\stackrel{?}{=} -9 \\[0.3cm]
	10&\stackrel{\text{\cmark}}{=} 10 &\qquad\qquad -9&\stackrel{\text{\cmark}}{=} -9 \\[0.3cm]
	\end{aligned}
	\]
Therefore, $(-5, 4)$ satisfies both of the equations and is then a solution to the given system of equations. \pspace

Note that because the system is a system of linear equations, there can be at most one solution. Therefore, $(-5, 4)$ is the only solution. 
}



% Question 3
\newpage
\question[6] Showing all your work, solve the following system of equations:
	\[
	\left\{
	\begin{aligned}
	x - 2y&= 12 \\
	3x + y&= 15
	\end{aligned}\right.
	\] \pspace

{\itshape Using substitution, we solve for $x$ in the first equation:
	\[
	\begin{aligned}
	x - 2y&= 12 \\
	x&= 2y + 12
	\end{aligned}
	\] 
Using this in the second equation, we find\dots
	\[
	\begin{aligned}
	3x + y&= 15 \\
	3(2y + 12) + y&= 15 \\
	6y + 36 + y&= 15 \\
	7y + 36&= 15 \\
	7y&= -21 \\
	y&= -3
	\end{aligned}
	\] 
But then we know that $x= 2y + 12= 2(-3) + 12= -6 + 12= 6$. Therefore, the solution is $(x, y)= (6, -3)$. 
	
\begin{center} {\bfseries OR} \end{center}

Using elimination, we eliminate $y$ by multiplying the second equation by 2 and adding the equations:
	\[
	\begin{aligned}
	x - 2y&= 12 \\
	6x + 2y&= 30 \\ \hline
	7x&= 42 \\
	x&= 6
	\end{aligned}
	\] 
Using the second equation, we have\dots
	\[
	\begin{aligned}
	3x + y&= 15 \\
	3(6) + y&= 15 \\
	18 + y&= 15 \\
	y&= -3
	\end{aligned}
	\] 
Therefore, the solution to the system of equations is $(x, y)= (6, -3)$. 
}



% Question 4
\newpage
\question[7] Showing all your work, solve the following system of equations:
	\[
	\begin{aligned}
	3x + 2y&= -2 \\
	3x - 4y&= -5
	\end{aligned}
	\] \pspace

{\itshape Using substitution, we solve for $y$ in the first equation:
	\[
	\begin{aligned}
	3x + 2y&= -2 \\
	2y&= -3x - 2 \\
	y&= -\frac{3}{2}\,x - 1
	\end{aligned}
	\] 
Using this in the second equation, we find\dots
	\[
	\begin{aligned}
	3x - 4y&= -5 \\
	3x - 4 \left( -\frac{3}{2}\,x - 1 \right)&= -5 \\
	3x + 6x + 4&= -5 \\
	9x + 4&= -5 \\
	9x&= -9 \\
	x&= -1
	\end{aligned}
	\] 
But then we know that $y= -\frac{3}{2}\,x - 1= -\frac{3}{2} \cdot -1 - 1= \frac{3}{2} - 1= \frac{3}{2} - \frac{2}{2}= \frac{1}{2}$. Therefore, the solution is $(x, y)= (-1, \frac{1}{2})$. 
	
\begin{center} {\bfseries OR} \end{center}

Using elimination, we eliminate $x$ by multiplying the second equation by $-1$ and adding the equations:
	\[
	\begin{aligned}
	3x + 2y&= -2 \\
	-3x + 4y&= 5 \\ \hline
	6y&= 3 \\
	y&= \frac{1}{2}
	\end{aligned}
	\] 
Using the first equation, we have\dots
	\[
	\begin{aligned}
	3x + 2y&= -2 \\
	3x + 2 \cdot \frac{1}{2}&= -2 \\
	3x + 1&= -2 \\
	3x&= -3 \\
	x&= -1
	\end{aligned}
	\] 
Therefore, the solution to the system of equations is $(x, y)= (-1, \frac{1}{2})$. 
}



% Question 5
\newpage
\question[6] Compute the following, being sure to simplify as much as possible:
	\[
	\dfrac{2x - 1}{x + 2} - \dfrac{x - 4}{x - 3}
	\] \pspace

	\[
	\begin{aligned}
	\dfrac{2x - 1}{x + 2} - \dfrac{x - 4}{x - 3}&= \dfrac{(2x - 1)(x - 3)}{(x + 2)(x - 3)} - \dfrac{(x - 4)(x + 2)}{(x - 3)(x + 2)} \\[0.3cm]
	&= \dfrac{2x^2 - 6x - x + 3}{(x + 2)(x - 3)} - \dfrac{x^2 + 2x - 4x - 8}{(x - 3)(x + 2)} \\[0.3cm]
	&= \dfrac{2x^2 - 7x + 3}{(x + 2)(x - 3)} - \dfrac{x^2 - 2x - 8}{(x - 3)(x + 2)} \\[0.3cm]
	&= \dfrac{(2x^2 - 7x + 3) - (x^2 - 2x - 8)}{(x - 3)(x + 2)} \\[0.3cm]
	&= \dfrac{2x^2 - 7x + 3 - x^2 + 2x + 8}{(x - 3)(x + 2)} \\[0.3cm]
	&= \dfrac{x^2 - 5x + 11}{(x - 3)(x + 2)} \\[0.3cm]
	\end{aligned}
	\]



% Question 6
\newpage
\question[6] Compute the following, being sure to simplify as much as possible:
	\[
	\dfrac{x^2 - 1}{x^2 - x - 6} + \dfrac{x + 1}{x^2 - 9}
	\]
	
	\[
	\begin{aligned}
	\dfrac{x^2 - 1}{x^2 - x - 6} + \dfrac{x + 1}{x^2 - 9}&= \dfrac{x^2 - 1}{(x - 3)(x + 2)} + \dfrac{x + 1}{(x - 3)(x + 3)} \\[0.3cm]
	&= \dfrac{(x^2 - 1)(x + 3)}{(x - 3)(x + 2)(x + 3)} + \dfrac{(x + 1)(x + 2)}{(x - 3)(x + 3)(x + 2)} \\[0.3cm]
	&= \dfrac{x^3 + 3x^2 - x - 3}{(x - 3)(x + 2)(x + 3)} + \dfrac{x^2 + 2x + x + 2}{(x - 3)(x + 3)(x + 2)} \\[0.3cm]
	&= \dfrac{x^3 + 3x^2 - x - 3}{(x - 3)(x + 2)(x + 3)} + \dfrac{x^2 + 3x + 2}{(x - 3)(x + 3)(x + 2)} \\[0.3cm]
	&= \dfrac{x^3 + 3x^2 - x - 3 + x^2 + 3x + 2}{(x - 3)(x + 2)(x + 3)} \\[0.3cm]
	&= \dfrac{x^3 + 4x^2 + 2x - 1}{(x - 3)(x + 2)(x + 3)} \\[0.3cm]
	\end{aligned}
	\]



% Question 7
\newpage
\question[6] Compute the following, being sure to simplify as much as possible:
	\[
	\dfrac{x^2 + 14x + 24}{x^2 - 5x} \cdot \dfrac{x^2 + x}{x^2 - 4x - 12}
	\] \pspace
	
	\[
	\begin{aligned}
	\dfrac{x^2 + 14x + 24}{x^2 - 5x} \cdot \dfrac{x^2 + x}{x^2 - 4x - 12}&= \dfrac{(x + 2)(x + 12)}{x(x - 5)} \cdot \dfrac{x(x + 1)}{(x - 6)(x + 2)} \\[0.3cm]
	&= \dfrac{\cancel{(x + 2)}(x + 12)}{\cancel{x}(x - 5)} \cdot \dfrac{\cancel{x}(x + 1)}{(x - 6)\cancel{(x + 2)}} \\[0.3cm]
	&= \dfrac{(x + 1)(x + 12)}{(x - 6)(x - 5)}
	\end{aligned}
	\]



% Question 8
\newpage
\question[7] Compute the following, being sure to simplify as much as possible:
	\[
	\dfrac{\phantom{.}\;\;\;\dfrac{x^2 - 16}{x^2 + 10x + 25}\;\;\;\phantom{.}}{\dfrac{x^2 + 11x + 28}{x^2 + 2x - 15}}
	\] \pspace
	
	\[
	\begin{aligned}
	\dfrac{\phantom{.}\;\;\;\dfrac{x^2 - 16}{x^2 + 10x + 25}\;\;\;\phantom{.}}{\dfrac{x^2 + 11x + 28}{x^2 + 2x - 15}}&= \dfrac{x^2 - 16}{x^2 + 10x + 25} \cdot \dfrac{x^2 + 2x - 15}{x^2 + 11x + 28} \\[0.3cm]
	&= \dfrac{(x - 4)(x + 4)}{(x + 5)(x + 5)} \cdot \dfrac{(x - 3)(x + 5)}{(x + 4)(x + 7)} \\[0.3cm]
	&= \dfrac{(x - 4)\cancel{(x + 4)}}{(x + 5)\cancel{(x + 5)}} \cdot \dfrac{(x - 3)\cancel{(x + 5)}}{\cancel{(x + 4)}(x + 7)} \\[0.3cm]
	&= \dfrac{(x - 4)(x - 3)}{(x + 5)(x + 7)}
	\end{aligned}
	\]



% Question 9
\newpage
\question[6] Write the function $y= -4(3^{1 - 2x})$ in the form $y= Ab^x$ and determine whether $y$ is an increasing or decreasing function. Be sure to fully justify your answer. \pspace

{\itshape We have\dots
	\[
	\begin{aligned}
	y&= -4(3^{1 - 2x}) \\[0.3cm]
	&= -4 \cdot 3^1 \cdot 3^{-2x} \\[0.3cm]
	&= -4 \cdot 3 \cdot (3^{-2})^x \\[0.3cm]
	&= -4 \cdot 3 \cdot \left( \dfrac{1}{3^2} \right)^x \\[0.3cm]
	&= -12 \left( \dfrac{1}{9} \right)^x
	\end{aligned}
	\] \pspace
Because this is a function of the form $y= Ab^{cx}$ with $A= -12 < 0$, $b= \frac{1}{9} < 1$, and $c= 1 > 0$, the function is increasing. 
}



% Question 10
\newpage
\question[6] Solve the following equation without the use of logarithms:
	\[
	4^{3 - x}= \dfrac{1}{2}
	\] \pspace

{\itshape We have\dots
	\[
	\begin{aligned}
	4^{3 - x}&= \dfrac{1}{2} \\[0.3cm]
	(2^2)^{3 - x}&= 2^{-1} \\[0.3cm]
	2^{6 - 2x}&= 2^{-1}
	\end{aligned}
	\]
Comparing powers, this implies that $6 - 2x= -1$. Then we have\dots
	\[
	\begin{aligned}
	6 - 2x&= -1 \\[0.3cm]
	2x&= 7 \\[0.3cm]
	x&= \dfrac{7}{2}
	\end{aligned}
	\]
}



% Question 11
\newpage
\question[6] Solve the following equation:
	\[
	6(9^x) + 10= 12
	\] \pspace

{\itshape We have\dots
	\[
	\begin{aligned}
	6(9^x) + 10&= 12 \\
	6(9^x)&= 2 \\
	9^x&= \dfrac{1}{3} \\
	(3^2)^x&= 3^{-1} \\
	3^{2x}&= 3^{-1}
	\end{aligned}
	\]
Comparing powers, we have $2x= -1$, which implies $x= -\dfrac{1}{2}$.

\begin{center} {\bfseries OR} \end{center}

	\[
	\begin{aligned}
	6(9^x) + 10&= 12 \\
	6(9^x)&= 2 \\
	9^x&= \dfrac{1}{3} \\
	\log_9(9^x)&= \log_9 \left( \dfrac{1}{3} \right) \\
	x&= \log_9 \left( \dfrac{1}{3} \right)
	\end{aligned}
	\]

\begin{center} {\bfseries OR} \end{center}

	\[
	\begin{aligned}
	6(9^x) + 10&= 12 \\
	6(9^x)&= 2 \\
	9^x&= \dfrac{1}{3} \\
	\ln(9^x)&= \ln\left( \frac{1}{3} \right) \\
	x \ln(9)&= \ln\left( \frac{1}{3} \right) \\
	x&= \dfrac{\ln\left( \frac{1}{3} \right)}{\ln(9)}
	\end{aligned}
	\]
}



% Question 12
\newpage
\question[7] Solve the following equation:
	\[
	3e^{2x} - 5= 22
	\] \pspace
	
{\itshape 
	\[
	\begin{aligned}
	3e^{2x} - 5&= 22 \\[0.3cm]
	3e^{2x}&= 27 \\[0.3cm]
	e^{2x}&= 9 \\[0.3cm]
	\ln(e^{2x})&= \ln(9) \\[0.3cm]
	2x&= \ln(9) \\[0.3cm]
	x&= \dfrac{\ln(9)}{2}
	\end{aligned}
	\]

\begin{center} {\bfseries OR} \end{center}

	\[
	\begin{aligned}
	3e^{2x} - 5&= 22 \\[0.3cm]
	3e^{2x}&= 27 \\[0.3cm]
	e^{2x}&= 9 \\[0.3cm]
	\ln(e^{2x})&= \ln(9) \\[0.3cm]
	2x&= \ln(3^2) \\[0.3cm]
	2x&= 2\ln(3) \\[0.3cm]
	\end{aligned}
	\]
	
Note: These solutions are equivalent as\dots
	\[
	\dfrac{\ln(9)}{2}= \frac{1}{2}\, \ln(9)= \ln(9^{1/2})= \ln(\sqrt{9})= \ln(3)	
	\]
}



% Question 13
\newpage
\question[6] Evaluate the following:
	\begin{enumerate}[(a)]
	\item $\log_6(1)$
	\item $\log_{17}(17)$
	\item $\ln(1)$
	\item $\log_8(64)$
	\item $\log_2(\frac{1}{16})$
	\item $\log_9(\frac{1}{3})$
	\end{enumerate} \pspace

\noindent{\bfseries Solution.}
\begin{enumerate}[(a)]
\item $\log_6(1)= \log_6(6^0)= 0$ \pspace

\item $\log_{17}(17)= \log_{17}(17^1)= 1$ \pspace

\item $\ln(1)= \ln(e^0)= 0$ \pspace

\item $\log_8(64)= \log_8(8^2)= 2$ \pspace

\item $\log_2(\frac{1}{16})= \log_2(16^{-1})= \log_2\left( (2^4)^{-1} \right)= \log_2(2^{-4})= -4$ \pspace

\item $\log_9(\frac{1}{3})= \log_9(3^{-1})= \log_9\left( (9^{1/2})^{-1} \right)= \log_9(9^{-1/2})= -\dfrac{1}{2}$
\end{enumerate}



% Question 14
\newpage
\question[6] Write the following expression using a single logarithm and no negative powers:
	\[
	2 \ln(x) - 3 \ln(y^2) + \dfrac{1}{4}\,\ln(z^{-1})
	\] \pspace
	
	\[
	\begin{aligned}
	2 \ln(x) - 3 \ln(y^2) + \dfrac{1}{4}\,\ln(z^{-1})&= \ln(x^2) - \ln\left( (y^2)^3 \right) + \ln\left( (z^{-1})^{1/4} \right) \\[0.3cm]
	&= \ln(x^2) - \ln(y^6) + \ln(z^{-1/4}) \\[0.3cm]
	&= \ln\left( \dfrac{x^2}{y^6} \right) + \ln(z^{-1/4}) \\[0.3cm]
	&= \ln\left( \dfrac{x^2 z^{-1/4}}{y^6} \right) \\[0.3cm]
	&= \ln\left( \dfrac{x^2}{y^6 z^{1/4}} \right) \\[0.3cm]
	&= \ln\left( \dfrac{x^2}{y^6 \sqrt[4]{z}} \right) \\[0.3cm]
	\end{aligned}
	\]



% Question 15
\newpage
\question[6] Showing all your work, solve the following:
	\[
	\log_5(2x - 3) + 6= 8
	\] \pspace
	
	\[
	\begin{aligned}
	\log_5(2x - 3) + 6&= 8 \\[0.3cm]
	\log_5(2x - 3)&= 2 \\[0.3cm]
	5^{\log_5(2x - 3)}&= 5^2 \\[0.3cm]
	2x - 3&= 25 \\[0.3cm]
	2x&= 28 \\[0.3cm]
	x&= 14
	\end{aligned}
	\]



% Question 16
\newpage
\question[7] Showing all your work, solve the following:
	\[
	10 - 2\ln x= 5
	\] \pspace
	
	\[
	\begin{aligned}
	10 - 2\ln x&= 5 \\[0.3cm]
	2\ln x&= 5 \\
	\ln x&= \dfrac{5}{2} \\[0.3cm]
	e^{\ln x}&= e^{5/2} \\[0.3cm]
	x&= e^{5/2}
	\end{aligned}
	\]


\end{questions}
\end{document}