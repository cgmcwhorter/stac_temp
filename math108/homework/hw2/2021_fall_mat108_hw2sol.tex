\documentclass[11pt,letterpaper]{article}
\usepackage[lmargin=1in,rmargin=1in,tmargin=1in,bmargin=1in]{geometry}
\usepackage{../style/homework}
\usepackage{../style/commands}
\setbool{quotetype}{false} % True: Side; False: Under
\setbool{hideans}{false} % Student: True; Instructor: False

% -------------------
% Content
% -------------------
\begin{document}

\homework{2: Due 09/28}{Listen, I'm not the nicest guy in the universe, because I’m the smartest, and being nice is something stupid people do to hedge their bets.}{Rick Sanchez, Rick \& Morty}

% Problem 1
\problem{10} Solve the following system of equations:
	\[
	\begin{aligned}
	2x - y&= 5 \\
	x + y&= 1
	\end{aligned}
	\] \pspace

\sol {\itshape Adding the equations, we have\dots
	\[
	\begin{aligned}
	3x&= 6 \\
	x&= 2
	\end{aligned}
	\]
But then using $x= 2$ in the second equation, we have
	\[
	\begin{aligned}
	2 + y&= 1 \\
	y&= -1
	\end{aligned}
	\]
Therefore, the solution is $(x, y)= (2, -1)$, i.e. $x= 2$ and $y= -1$.}



\newpage



% Problem 2
\problem{10} Solve the following system of equations:
	\[
	\begin{aligned}
	4x - 3y&= 10 \\
	6x + 5y&= -4
	\end{aligned}
	\] \pspace

\sol {\itshape First, we multiply the first equation by 3 and the second equation by $-2$ to obtain\dots
	\[
	\begin{aligned}
	12x - 9y&= 30 \\
	-12x - 10y&= 8
	\end{aligned}
	\]
Adding these equations, we get
	\[
	\begin{aligned}
	-19y&= 38 \\
	y&= -2
	\end{aligned}
	\]
Using $y= -2$ in the first equation, we have\dots
	\[
	\begin{aligned}
	4x - 3(-2)&= 10 \\
	4x + 6&= 10 \\
	4x&= 4 \\
	x&= 1
	\end{aligned}
	\]
Therefore, the solution is $(x, y)= (1, -2)$, i.e. $x= 1$ and $y= -2$.}


\newpage



% Problem 3
\problem{10} Solve the following system of equations:
	\[
	\begin{aligned}
	\frac{1}{2}x - \frac{1}{6}y&= -1 \\
	x - \frac{1}{3}y&= -2 
	\end{aligned}
	\] \pspace

\sol {\itshape To make the equations simpler to solve, we clear denominators by multiplying the first equation by 6 and the second by 3. Then we have
	\[
	\begin{aligned}
	3x - y&= -6 \\
	3x - y&= -6
	\end{aligned}
	\]
But then these are the same line! Therefore, any point satisfying one equation will satisfy the other. So the set of solutions will be the every point on the line $3x - y= -6$, i.e $y= 3x + 6$. So the set of solutions is $(x, y)= (x, 3x + 6)$ for any $x \in \mathbb{R}$.}



\newpage



% Problem 4
\problem{10} Solve the following system of equations:
	\[
	\begin{aligned}
	6x - 5y&= 1 \\
	3x + 10y&= 3
	\end{aligned}
	\] \pspace

\sol {\itshape We multiply the first equation by 2 to obtain\dots
	\[
	\begin{aligned}
	12x - 10y&= 2 \\
	3x + 10y&= 3
	\end{aligned}
	\]
Adding the equations, we have\dots
	\[
	\begin{aligned}
	15x&= 5 \\
	x&= \dfrac{1}{3}
	\end{aligned}
	\]
Using $x= 1/3$ in the first equation, we have
	\[
	\begin{aligned}
	6\left(\dfrac{1}{3}\right) - 5y&= 1 \\
	2 - 5y&= 1 \\
	5y&= 1 \\
	y&= \dfrac{1}{5}
	\end{aligned}
	\]
Therefore, the solution is $(x, y)= (\frac{1}{3}, \frac{1}{5})$, i.e. $x= \frac{1}{3}$ and $y= \frac{1}{5}$.}



\newpage



% Problem 5
\problem{10} Plot the solution set to the system of inequalities: 
	\[
	\begin{aligned}
	4x + y&\leq 1 \\
	x - 2y&\leq -20
	\end{aligned}
	\] \pspace

\sol Solving for $y$ in each, we have
	\[
	\begin{aligned}
	y&\leq 1 - 4x \\
	\frac{1}{2}\, x &+ 10 \leq y
	\end{aligned}
	\]
Notice each line is solid. For the first line, we shade beneath the line. For the second line, we shade above the line. 
	\[
	\fbox{
	\begin{tikzpicture}[scale=2,every node/.style={scale=0.5}]
	\begin{axis}[
	grid=both,
	axis lines=middle,
	ticklabel style={fill=blue!5!white},
	xmin= -15, xmax=15,
	ymin= -15, ymax=15,
	xtick={-14,-12,-10,-8,-6,-4,-2,0,2,4,6,8,10,12,14},
	ytick={-14,-12,...,12,14},
	minor tick = {-15,-13,...,15},
	xlabel=\(x\),ylabel=\(y\),
	]
	\draw[draw=none,pattern=north west lines, pattern color=blue!40,opacity=0.3] (-15,2.5) -- (-15,15) -- (10,15)--cycle;
	\draw[draw=none,pattern=north east lines, pattern color=red!40,opacity=0.3] (-15,15) -- (-15,-15) -- (4,-15) -- (-7/2,15) --cycle;
	\draw[draw=none,fill=gray,opacity=0.3] (-15,15) -- (-15,2.5) -- (-2,9) -- (-7/2,15) -- cycle;
	\addplot[thick, domain= -15:15] {1 - 4*x};
	\addplot[thick, domain= -15:15] {1/2*x+10};
	\node at (-9.5,10.5) {$\mathbf{\mathcal{R}}$};
	\end{axis}
	\end{tikzpicture}
	}
	\]



\newpage



% Problem 6
\problem{10} Plot the solution set to the system of inequalities: 
	\[
	\begin{aligned}
	x&\geq 1 \\
	-3x + y&\geq -4
	\end{aligned}
	\] \pspace

\sol Solving for $y$ in the second, we have
	\[
	\begin{aligned}
	x& \geq 1 \\
	y&\geq 3x - 4
	\end{aligned}
	\]
Notice each line is solid. For the first line, we shade to the right of the line. For the second line, we shade above the line. 
	\[
	\fbox{
	\begin{tikzpicture}[scale=2,every node/.style={scale=0.5}]
	\begin{axis}[
	grid=both,
	axis lines=middle,
	ticklabel style={fill=blue!5!white},
	xmin= -10, xmax=10,
	ymin= -10, ymax=10,
	xtick={-10,-8,...,10},
	ytick={-10,-8,...,10},
	minor tick = {-11,...,11},
	xlabel=\(x\),ylabel=\(y\),
	]
	\draw[draw=none,pattern=north west lines, pattern color=blue!40,opacity=0.3]  (1,10) -- (10,10) -- (10,-10) -- (1,-10) -- cycle;
	\draw[draw=none,pattern=north east lines, pattern color=red!40,opacity=0.3] (-10,10) -- (-10,-10) -- (-2,-10) -- (14/3,10) -- cycle;
	\draw[draw=none,fill=gray,opacity=0.3] (1,10) -- (1,-1) -- (14/3,10) -- cycle;
	\draw[thick] (1,-10) -- (1,10);
	\addplot[thick, domain= -10:10] {3*x - 4};
	\node at (2.4,7.3) {$\mathbf{\mathcal{R}}$};
	\end{axis}
	\end{tikzpicture}
	}
	\]


\end{document}