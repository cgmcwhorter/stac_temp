\documentclass[11pt,letterpaper]{article}
\usepackage[lmargin=1in,rmargin=1in,tmargin=1in,bmargin=1in]{geometry}
\usepackage{../style/homework}
\usepackage{../style/commands}
\setbool{quotetype}{true} % True: Side; False: Under
\setbool{hideans}{false} % Student: True; Instructor: False

% -------------------
% Content
% -------------------
\begin{document}

\homework{9: Due 11/16}{What I hear when I'm being yelled at is people caring loudly at me.}{Leslie Knopp, Parks and Recreation}

% Problem 1
\problem{10} Otto Graf borrows \$8,000 from a neighbor to help fund his startup. He promises to pay back the neighbor, plus 15\% interest per year on the borrowed amount. 
\begin{enumerate}[(a)]
\item What is the interest owed at the end of 9~months?
\item How much does Otto owe the neighbor if he pays them back after a year and a half? How much of this payment was interest?
\item Suppose Otto wants to be sure he does not owe the neighbor more than \$10,000. If he plans on repaying them in 2~years, what is the largest sum of money that he can borrow from them? 
\end{enumerate} \pspace

\sol This is a simple interest problem.
\begin{enumerate}[(a)]
\item 
	\[
	\begin{aligned}
	I&= Prt \\
	I&= 8000(0.15)(9/12) \\
	I&= \$900
	\end{aligned}
	\]
Therefore, Otto owes \$900 in interest after 9~months. \pspace

\item 
	\[
	\begin{aligned}
	I&= Prt \\
	I&= 8000(0.15)(18/12) \\
	I&= \$1800
	\end{aligned}
	\]
Therefore, Otto owes them $\$8000 + \$1800= \$9800$. Clearly, $\$1800$ of this is interest. Alternatively, we can find the future value\dots
	\[
	\begin{aligned}
	F&= P(1 + rt) \\
	F&= 8000(1 + 0.15 \cdot 18/12) \\
	F&= \$9800
	\end{aligned}
	\]
The interest was calculated above. \pspace

\item 
	\[
	\begin{aligned}
	P&= \dfrac{F}{1 + rt} \\
	P&= \dfrac{10000}{1 + 0.15 \cdot 2} \\
	P&= \$7692.31
	\end{aligned}
	\]
Therefore, the most Otto can borrow is \$7692.31. 
\end{enumerate}





\newpage





% Problem 2
\problem{10} Mary A. Richman takes out a \$600 loan from the bank. Because of her credit history, she is charged a 6\% discount on a six month note. 
\begin{enumerate}[(a)]
\item What are the nominal and effective interest rates for this loan?
\item What is the discount of this loan?
\item How much money does Mary receive from the bank?
\item How much does Mary owe the bank after six months?
\end{enumerate} \pspace

\sol This is a simple discount note problem.
\begin{enumerate}[(a)]
\item 
	\[
	\begin{aligned}
	r_{\text{eff}}&= \dfrac{r}{1 - rt} \\
	r_{\text{eff}}&= \dfrac{0.06}{1 - 0.06 \cdot 6/12} \\
	r_{\text{eff}}&= 0.062
	\end{aligned}
	\]
Therefore, the effective interest rate is 6.2\%. The nominal rate is the advertised rate of 6\%. \pspace

\item 
	\[
	\begin{aligned}
	D&= Mrt \\
	D&= 600(0.06)(6/12) \\
	D&= \$18
	\end{aligned}
	\]
Therefore, the discount is \$18. \pspace

\item 
	\[
	\begin{aligned}
	P&= M - D \\
	P&= 600 - 18 \\
	P&= \$582
	\end{aligned}
	\]
Therefore, the proceeds are \$582, i.e. Mary receives \$582. \pspace

\item 
	\[
	\begin{aligned}
	F&= P(1 + rt) \\
	F&= 600(1 + 0.06 \cdot 6/12) \\
	F&= \$618
	\end{aligned}
	\]
Alternatively, 
	\[
	\begin{aligned}
	I&= Prt \\
	I&= 600(0.06)(6/12) \\
	I&= \$18
	\end{aligned}
	\]
Mary then owes the bank the original \$600 plus the \$18 in interest. Therefore, at the end of the six months, Mary owes the bank \$618. 
\end{enumerate}





\newpage





% Problem 3
\problem{10} Paige Turner invests \$3,500 in a social media marketing company that promises a 4.1\% annual return on the investment, with interest paid to the investors semiannually. 
\begin{enumerate}[(a)]
\item What is the nominal and effective interest rate for this investment?
\item How much money will the investment be worth after 2~years?
\item How long until the investment is worth \$5,000?
\item How much should Paige have placed in the company if she wanted \$5,000 at the end of 2 years? 
\end{enumerate} \pspace

\sol This is a discrete compound interest problem.
\begin{enumerate}[(a)]
\item 
	\[
	\begin{aligned}
	r_{\text{eff}}&= \left(1 + \dfrac{r}{k} \right)^k - 1 \\
	r_{\text{eff}}&= \left(1 + \dfrac{0.041}{2} \right)^2 - 1 \\
	r_{\text{eff}}&= 0.0414
	\end{aligned}
	\]
Therefore, the effective interest rate is 4.14\%. The nominal rate is the advertised rate of 4.1\%.

\item 
	\[
	\begin{aligned}
	F&= P \left(1 + \dfrac{r}{k} \right)^{kt} \\
	F&= 3500 \left(1 + \dfrac{0.041}{2} \right)^{2 \cdot 2} \\
	F&= \$3795.95
	\end{aligned}
	\] \pspace

\item 
	\[
	\begin{aligned}
	n&= \dfrac{\log(F/P)}{\log\left(1 + \dfrac{r}{k} \right)} \\
	n&= \dfrac{\log(5000/3500)}{\log(1 + 0.041/2)} \\
	n&= 17.5765
	\end{aligned}
	\]
The investment will grow to \$5000 after 17.5765 compoundings. Therefore, the investment will grow to \$5000 after $17.5765/2= 8.79$ years, i.e. 8 years and 9.5 months. \pspace

\item 
	\[
	\begin{aligned}
	P&= \dfrac{F}{\left(1 + \dfrac{r}{k} \right)^{kt}} \\
	P&= \dfrac{5000}{\left(1 + \dfrac{0.041}{2} \right)^{2 \cdot 2}} \\
	P&= \$4610.18
	\end{aligned}
	\]
\end{enumerate}





\newpage





% Problem 4
\problem{10} Adam Baum deposits \$18,500 into a savings account that pays 2.2\% annual interest, compounded continuously. 
\begin{enumerate}[(a)]
\item What is the nominal and effective interest rate for this account?
\item How much money will be in the account after 10~years?
\item How long until the account has \$20,000? 
\item How much should Mr. Baum have placed in the account if he wanted \$20,000 at the end of 5 years? 
\end{enumerate} \pspace

\sol This is a continuous compounding interest problem.
\begin{enumerate}[(a)]
\item 
	\[
	\begin{aligned}
	r_{\text{eff}}&= e^r - 1 \\
	r_{\text{eff}}&= e^{0.022} - 1 \\
	r_{\text{eff}}&= 0.0222
	\end{aligned}
	\]
Therefore, the effective interest rate is 2.22\%. The nominal interest rate is the advertised 2.2\%. \pspace

\item 
	\[
	\begin{aligned}
	F&= Pe^{rt} \\
	F&= 18500 e^{0.022 \cdot 10} \\
	F&= \$23052.40
	\end{aligned}
	\]
Therefore, the account will have \$23052.40 after 10~years. \pspace

\item 
	\[
	\begin{aligned}
	F&= Pe^{rt} \\
	20000&= 18500 e^{0.022 t} \\
	e^{0.022t}&= 1.08108 \\
	0.022t&= \ln(1.08108) \\
	t&= \dfrac{\ln(1.08108}{0.022} \\
	t&= 3.54
	\end{aligned}
	\]
Therefore, there will be \$20000 in the account after 3.54~years, i.e. 3 years and 6~months. \pspace

\item 
	\[
	\begin{aligned}
	P&= Fe^{-rt} \\
	P&= 20000 e^{-0.022 \cdot 5} \\
	P&= \$17916.70
	\end{aligned}
	\]
Therefore, Mr. Baum should put \$17916.70 in the account. 
\end{enumerate}


\end{document}