\documentclass[11pt,letterpaper]{article}
\usepackage[lmargin=1in,rmargin=1in,tmargin=1in,bmargin=1in]{geometry}
\usepackage{../style/homework}
\usepackage{../style/commands}
\setbool{quotetype}{true} % True: Side; False: Under
\setbool{hideans}{false} % Student: True; Instructor: False

% -------------------
% Content
% -------------------
\begin{document}

\homework{5: Due 10/05}{I go. You stay. No following.}{Iron Giant, Iron Giant}

% Problem 1
\problem{10} Watch the following three videos by 3Blue1Brown (Grant Sanderson):
	\begin{enumerate}[(i)]
	\item \href{https://www.youtube.com/watch?v=kYB8IZa5AuE&list=PL0-GT3co4r2y2YErbmuJw2L5tW4Ew2O5B&index=4&ab_channel=3Blue1Brown}{Linear transformations and matrices}
	\item \href{https://www.youtube.com/watch?v=XkY2DOUCWMU&list=PL0-GT3co4r2y2YErbmuJw2L5tW4Ew2O5B&index=5&ab_channel=3Blue1Brown}{Matrix multiplication as composition}
	\item \href{https://www.youtube.com/watch?v=Ip3X9LOh2dk&list=PL0-GT3co4r2y2YErbmuJw2L5tW4Ew2O5B&index=7&ab_channel=3Blue1Brown}{The determinant}
	\end{enumerate}
What did you learn from these videos? \vfill


\begin{center} Answers will vary. \end{center} \vfill





\newpage





% Problem 2
\problem{10} Suppose the reduced-row echelon form for an augmented matrix is the following:
	\[
	\begin{pmatrix}
	1 & -4 & 0 & 1 & 0 & 5 \\
	0 & 0 & 1 & 0 & 0 & -1 \\
	0 & 0 & 0 & 0 & 1 & 3
	\end{pmatrix}
	\]
Using this, find all the solutions to the system of equations. \pspace

\sol The last row of the matrix tells us that $x_5= 3$. The second row of the matrix tells us that $x_3= -1$. The first row of the matrix tells us that $x_1 - 4x_2 + x_4= 5$. Fixing any two of $x_1, x_2, x_4$, we can solve for the third. Therefore, we have two free variables. For instance, choosing $x_2$ and $x_4$ to be free, we have solutions
	\[
	\begin{cases}
	x_1= 4x_2 - x_4 + 5 \\
	x_2: \text{free} \\
	x_3= -1 \\
	x_4: \text{free} \\
	x_5= 3
	\end{cases}
	\]





\newpage





% Problem 3
\problem{10} Can you compute the following product of matrices? If you can, compute the product. If you can not, explain why. 
	\[
	\begin{pmatrix}
	1 & -1 & 8 \\
	2 & 3 & 5 
	\end{pmatrix}
	\begin{pmatrix}
	1 & 4 \\
	0 & -6 \\
	7 & 7 \\
	-8 & 0
	\end{pmatrix}
	\] \pspace

\sol The first matrix is $2 \times 3$. The second matrix is $4 \times 2$. For matrix multiplication to be defined, we need the number of columns of the first to equal the number of rows of the second. But $3 \neq 4$ so that this matrix multiplication is not defined, i.e. we cannot compute this product. 





\newpage





% Problem 4
\problem{10} Showing all your work, compute the following:
	\[
	\begin{pmatrix}
	5 & 0 & 1 \\
	-2 & -3 & 4 \\
	1 & -1 & 1
	\end{pmatrix}
	\begin{pmatrix}
	2 \\
	-1 \\
	1
	\end{pmatrix}
	\] \pspace

\sol 
	\[
	\begin{aligned}
	\begin{pmatrix}
	5 & 0 & 1 \\
	-2 & -3 & 4 \\
	1 & -1 & 1
	\end{pmatrix}
	\begin{pmatrix}
	2 \\
	-1 \\
	1
	\end{pmatrix}&=
	\begin{pmatrix}
	5(2) + 0(-1) + 1(1) \\
	-2(2) + (-3)(-1) + 4(1) \\
	1(2) + (-1)(-1) + 1(1) 
	\end{pmatrix} \\[0.3cm]
	&= 
	\begin{pmatrix}
	10 + 0 + 1 \\
	-4 + 3 + 4 \\
	2 + 1 + 1 
	\end{pmatrix} \\[0.3cm]
	&=
	\begin{pmatrix}
	11 \\
	3 \\
	4
	\end{pmatrix}
	\end{aligned}
	\]





\newpage





% Problem 5
\problem{10} Showing all your work, compute the following:
	\[
	\begin{pmatrix}
	1 & -1 & 2 \\
	-3 & 6 & 0
	\end{pmatrix}
	\begin{pmatrix}
	0 & 4 \\
	-2 & 3 \\
	3 & -2
	\end{pmatrix}
	\] \pspace

\sol
	\[
	\begin{aligned}
	\begin{pmatrix}
	1 & -1 & 2 \\
	-3 & 6 & 0
	\end{pmatrix}
	\begin{pmatrix}
	0 & 4 \\
	-2 & 3 \\
	3 & -2
	\end{pmatrix}&= 
	\begin{pmatrix}
	1(0) + (-1)(-2) + 2(3) & 1(4) + (-1)3 + 2(-2) \\
	-3(0) + 6(-2) + 0(3) & -3(4) + 6(3) + 0(-2) 
	\end{pmatrix} \\[0.3cm]
	&= 	
	\begin{pmatrix}
	0 + 2 + 6 & 4 - 3 - 4 \\
	0 - 12 + 0 & -12 + 18 + 0 
	\end{pmatrix} \\[0.3cm]
	&= 
	\begin{pmatrix}
	8 & -3 \\
	-12 & 6
	\end{pmatrix}	
	\end{aligned}
	\]





\newpage





% Problem 6
\problem{10} Compute the determinant of the following matrix:
	\[
	\begin{pmatrix}
	2 & 1 & 5 \\
	-3 & 0 & 3 \\
	7 & 2 & 7 
	\end{pmatrix}
	\]
Is this matrix invertible? Explain. \pspace

\sol
	\[
	\begin{aligned}
	\det 	\begin{pmatrix}
	2 & 1 & 5 \\
	-3 & 0 & 3 \\
	7 & 2 & 7 
	\end{pmatrix}&= -(-3) \begin{vmatrix} 1 & 5 \\ 2 & 7 \end{vmatrix} + 0 \begin{vmatrix} 2 & 5 \\ 7 & 7 \end{vmatrix} - 3 \begin{vmatrix} 2 & 1 \\ 7 & 2 \end{vmatrix} \\[0.3cm]
	&= 3 \big(1(7) - 2(5) \big) + 0 - 3 \big( 2(2) - 7(1) \big) \\[0.3cm]
	&= 3 \big( 7 - 10 \big) - 3 \big( 4 - 7 \big) \\[0.3cm]
	&= 3(-3) - 3(-3) \\[0.3cm]
	&= -9 - (-9) \\[0.3cm]
	&= -9 + 9 \\[0.3cm]
	&= 0 
	\end{aligned}
	\] \pspace

Because the determinant of the matrix is 0, the matrix cannot be invertible. 





\newpage





% Problem 7
\problem{10} Find the inverse of the following matrix:
	\[
	\begin{pmatrix}
	2 & -1 \\
	-3 & 4
	\end{pmatrix}
	\] \pspace

\sol \par
\begin{minipage}[t]{0.49\textwidth}
	\[
	\left(
	\begin{array}{rr:rr}
	2 & -1 & 1 & 0 \\
	-3 & 4 & 0 & 1 
	\end{array} 
	\right)
	\]
	\[
	\left(
	\begin{array}{rr:rr}
	2 & -1 & 1 & 0 \\
	0 & 5 & 3 & 2 
	\end{array} 
	\right)
	\]
	\[
	\left(
	\begin{array}{rr:rr}
	2 & -1 & 1 & 0 \\
	0 & 1 & \frac{3}{5} & \frac{2}{5}
	\end{array} 
	\right)
	\]
	\[
	\left(
	\begin{array}{rr:rr}
	2 & 0 & \frac{8}{5} & \frac{2}{5} \\
	0 & 1 & \frac{3}{5} & \frac{2}{5}
	\end{array} 
	\right)
	\]
	\[
	\left(
	\begin{array}{rr:rr}
	1 & 0 & \frac{4}{5} & \frac{1}{5} \\
	0 & 1 & \frac{3}{5} & \frac{2}{5}
	\end{array} 
	\right)
	\]
\end{minipage}%
\begin{minipage}[t]{0.49\textwidth}
\pvspace{0.3cm}
$3R_1 + 2R_2 \to R_2$ \pvspace{0.6cm}
$\dfrac{1}{5}R_2 \to R_2$ \pvspace{0.7cm}
$R_2 + R_1 \to R_1$ \pvspace{0.6cm}
$\dfrac{1}{2}R_1 \to R_1$
\end{minipage} \pspace

Therefore, the inverse is\dots
	\[
	\begin{pmatrix}
	\frac{4}{5} & \frac{1}{5} \\[0.2cm]
	\frac{3}{5} & \frac{2}{5}
	\end{pmatrix}
	\]
Alternatively, because the matrix is $2 \times 2$, we can use the shortcut method:
	\[
	\det 	
	\begin{pmatrix}
	2 & -1 \\
	-3 & 4
	\end{pmatrix}
	= 2(4) - (-3)(-1)= 8 - 3= 5
	\]
Then we have
	\[
	\begin{pmatrix}
	2 & -1 \\
	-3 & 4
	\end{pmatrix}^{-1}=
	\dfrac{1}{5}
	\begin{pmatrix}
	4 & 1 \\
	3 & 2 
	\end{pmatrix}=
	\begin{pmatrix}
	\frac{4}{5} & \frac{1}{5} \\[0.2cm]
	\frac{3}{5} & \frac{2}{5}
	\end{pmatrix}	
	\]


\end{document}