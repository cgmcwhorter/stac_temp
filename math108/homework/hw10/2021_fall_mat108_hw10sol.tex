\documentclass[11pt,letterpaper]{article}
\usepackage[lmargin=1in,rmargin=1in,tmargin=1in,bmargin=1in]{geometry}
\usepackage{../style/homework}
\usepackage{../style/commands}
\setbool{quotetype}{true} % True: Side; False: Under
\setbool{hideans}{false} % Student: True; Instructor: False

% -------------------
% Content
% -------------------
\begin{document}

\homework{10: Due 11/16}{If you rearrange the letters of Peru, you could spell Europe.}{Andy Dwyer, Parks and Recreation}

% Problem 1
\problem{10} Dr. Anne Graham runs a clinic in her hometown. To be prepared for variations in tax rates, she sets aside \$800 per month from the business' profits into an account that earns 3.2\% yearly interest, compounded monthly. 
\begin{enumerate}[(a)]
\item Assuming the money has not been used, how much is in the account after 3~years?
\item If she wants to be prepared for a sudden \$10,000 increase in yearly taxes, how much should she be setting aside each month from the business' profits? [Assume the account will start with \$0.]
\end{enumerate} \pspace

\sol This is an annuity problem. 
\begin{enumerate}[(a)]
\item 
	\[
	\begin{aligned}
	F&= R \; \dfrac{\left(1 + \dfrac{r}{k} \right)^n - 1}{r/k} \\
	F&= 800\; \dfrac{\left(1 + \dfrac{0.032}{12} \right)^{12 \cdot 3}}{0.032/12} \\
	F&= \$30185.50
	\end{aligned}
	\]
Therefore, there will be $\$30,185.50$ in the account. \pspace

\item Dr. Graham wants to have enough set aside after each year for a tax increase, so we have\dots
	\[
	\begin{aligned}
	R&= \dfrac{F}{\dfrac{\left(1 + \dfrac{r}{k} \right)^n - 1}{r/k}} \\
	R&= \dfrac{10000}{\dfrac{\left(1 + \dfrac{0.032}{12} \right)^{12 \cdot 1} - 1}{0.032/12}} \\
	R&= \$821.18
	\end{aligned}
	\]
Therefore, she should set aside $\$821.18$ per month. 
\end{enumerate}





\newpage





% Problem 2
\problem{10} Rebecca Law and her husband Marshall are saving money for their first child's college fund. They are going to make four yearly payments, equally spaced, into a savings account that earns 2.8\% interest, compounded quarterly. 
\begin{enumerate}[(a)]
\item If they want to have \$30,000 saved by the time their newborn turns 18, what should these payments be?
\item How long until the account has \$20,000?
\end{enumerate} \pspace

\sol This is an annuity problem.
\begin{enumerate}[(a)]
\item 
	\[
	\begin{aligned}
	R&= \dfrac{F}{\dfrac{\left(1 + \dfrac{r}{k} \right)^n - 1}{r/k}} \\
	R&= \dfrac{30000}{\dfrac{\left(1 + \dfrac{0.028}{4} \right)^{18 \cdot 4} - 1}{0.028/4}} \\
	R&= \$321.88
	\end{aligned}
	\] \pspace

\item 
	\[
	\begin{aligned}
	F&= R \; \dfrac{\left(1 + \dfrac{r}{k} \right)^n - 1}{r/k} \\
	20000&= 321.88 \; \dfrac{\left(1 + \dfrac{0.028}{4} \right)^n - 1}{0.028/4} \\
	20000 \cdot \dfrac{0.028}{4}&= 321.88 \left( \left(1 + \dfrac{0.028}{4} \right)^n - 1 \right) \\
	140&= 321.88 \left( \left(1 + \dfrac{0.028}{4} \right)^n - 1 \right) \\
	0.434945&= \left(1 + \dfrac{0.028}{4} \right)^n - 1 \\
	\left(1 + \dfrac{0.028}{4} \right)^n&= 1.43494 \\
	n \log\left(1 + \dfrac{0.028}{4} \right)&= \log(1.43494) \\
	0.00697561n&= 0.361123 \\
	n&= 51.7694
	\end{aligned}
	\]
Therefore, the account will have $\$20,000$ after 52~payments, i.e. after $52/4= 13$~years.
\end{enumerate}





\newpage





% Problem 3
\problem{10} Richard O'Shea has a credit card with \$4,800 already on the card. Suppose the yearly interest rate on the card is 18.24\%, compounded monthly. 
\begin{enumerate}[(a)]
\item Suppose the credit card company computes the monthly minimum payment by supposing that you are going to pay off the amount in even amounts over 10~years. How is Rick's current monthly minimum payment?
\item If Rick did not spend any additional money and paid off the credit card in 10~years, how much interest would he have paid assuming he paid the fixed monthly minimum from (a).
\item Suppose the monthly minimum was fixed, after 18~months, how much of the monthly minimum payment is going towards the debt remaining and how much is being used to pay interest?
\item Rick plans on paying off the card sooner than by using the monthly minimums. If he wants to pay the card off in 2~years, how much should he be paying per month?
\end{enumerate} 

\sol This is an amortization problem.
\begin{enumerate}[(a)]
\item 
	\[
	\begin{aligned}
	R&= \dfrac{P}{\dfrac{1 - (1 + r/k)^{-n}}{r/k}} \\
	R&= \dfrac{4800}{\dfrac{1 - (1 + 0.1824/12)^{-10 \cdot 12}}{0.1824/12}} \\
	R&= \$87.23
	\end{aligned}
	\]
\item Rick pays a total of $\$87.23 \cdot 10 \cdot 12= \$10467.60$. Because the amount originally owed was \$4500, the rest must be interest. Therefore, Rick paid $\$10467.60 - \$4800= \$5,667.60$ in interest. 

\item 
	\[
	\begin{aligned}
	\text{PAP}&= R \left( \dfrac{1 - (1 + r/k)^{-(n - m + 1)}}{r/k} - \dfrac{1 - (1 + r/k)^{-(n - m)}}{r/k} \right) \\
	\text{PAP}&= 87.23 \left( \dfrac{1 - (1 + 0.1824/12)^{-(120 - 18 + 1)}}{0.1824/k} - \dfrac{1 - (1 + 0.1824/12)^{-(120 - 18)}}{0.1824/12} \right) \\
	\text{PAP}&= \$18.44
	\end{aligned}
	\]
Therefore, $\$18.44$ is going towards paying off the loan while the remaining $\$87.23 - \$18.44= \$68.79$ is going towards the interest. 

\item 
	\[
	\begin{aligned}
	R&= \dfrac{P}{\dfrac{1 - (1 + r/k)^{-n}}{r/k}} \\
	R&= \dfrac{4800}{\dfrac{1 - (1 + 0.1824/12)^{-2 \cdot 12}}{0.1824/12}} \\
	R&= \$240.19
	\end{aligned}
	\]
Therefore, Rick should make monthly payments of $\$240.19$. 
\end{enumerate}





\newpage





% Problem 4
\problem{10} Kurt C. Hose and his wife Gutta bought a home for \$86,000 that is amortized over 25~years, after a 15\% down payment. The interest on the house is 13\%, compounded monthly. 
\begin{enumerate}[(a)]
\item What are the monthly mortgage payments?
\item How much has been paid to date after 10~years?
\item How much of the loan remains to be paid after 10~years?
\end{enumerate} \pspace

\sol This is an amortization problem.
\begin{enumerate}[(a)]
\item The family first pays the down payment, which is $\$86000(0.15)= \$12900$. Then loan is on the $\$86000 - \$12900= \$73100$ (or $\$86000(0.85)= \$73100$) remaining to be paid. Then\dots
	\[
	\begin{aligned}
	R&= \dfrac{P}{\dfrac{1 - (1 + r/k)^{-n}}{r/k}} \\
	R&= \dfrac{73100}{\dfrac{1 - (1 + 0.13/12)^{-12 \cdot 25}}{0.13/12}} \\
	R&= \$824.45
	\end{aligned}
	\] \pspace

\item After 10~years, $(10 \cdot 12) \cdot \$824.45= \$98,934$ has been paid. \pspace

\item 
	\[
	\begin{aligned}
	P&= R\; \dfrac{1 - (1 + r/k)^{-(n - m)}}{r/k} \\
	P&= 824.45\; \dfrac{1 - (1 + 0.13/12)^{-(300 - 120)}}{0.13/12} \\
	P&= \$65161.40
	\end{aligned}
	\]
Therefore, there is still an outstanding balance of $\$65,161.40$. 
\end{enumerate}


\end{document}