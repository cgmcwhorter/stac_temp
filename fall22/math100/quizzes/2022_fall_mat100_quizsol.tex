\documentclass[11pt,letterpaper]{article}
\usepackage[lmargin=1in,rmargin=1in,bmargin=1in,tmargin=1in]{geometry}
\usepackage{style/quiz}
\usepackage{style/commands}

% -------------------
% Content
% -------------------
\begin{document}
\thispagestyle{title}


% Quiz 1
\quizsol \textit{True/False}: If $x \in A$ but $x \notin B$, then $x \notin A \cup B$. \pspace

\sol The statement is \textit{false}. Recall that if $S$ is a set, then $x \in S$ means that $x$ is an element of $S$. For example, suppose $S= \{ 1, 2, 3, 4, 5\}$. If $x= 1$, then $x \in S$. However, if $x= 9$ then $x$ is not in $S$, i.e. $x \notin S$. Recall also that $A \cup B$ is the set of elements that are either in $A$ or in $B$---including the possibility that it might be in both! Because $x \in A$, even though $x \notin B$, we know that $x \in A \cup B$ because $x$ is in $A$ or $B$---after all, it's in $A$! As a concrete example, take $A= \{ 1, 2, 3 \}$ and $B= \{ 2, 3, 4 \}$. Then we know that $A \cup B= \{ 1, 2, 3, 4 \}$. Now if $x= 1$, then $x \in A$ and $x \notin B$. But we can see that $x \in A \cup B$. \pvspace{1.5cm}



% Quiz 2
\quizsol \textit{True/False}: A person's salary at their first job is a function of their number of years of schooling. \pspace

\sol The statement is \textit{false}. Recall that a relation is a function if for each input, there is only one possible output, i.e. $f(x)$ is a function if given each $x$, there is one and only one possible output $f(x)$. In this example, we are wondering whether a person's income, $I$, is a function of their number of years of school, $n$. So is $I(n)$ a function? Then there would be one and only one possible output, i.e. salary, given some number of years of schooling. For instance, if $n= 4$~years (of high school, college, etc.), then the person's salary, $I(4)$, would have only one possible value. But we know that there are many people with the same number of years of schooling that have the same salary! For instance, there will be many people that graduate together (most having the same number of years of schooling) and will have widely varying salaries. Therefore, $I(n)$ cannot be a function, i.e. we cannot exactly predict someone's income from their number of years of school. However, one could try to perform statistical analysis on this problem, e.g. what does the \textit{average} person make if they have $n$ years of schooling. \pvspace{1.5cm}



% Quiz 3
\quizsol \textit{True/False}: A company is bulk ordering parts for their production line. The order is for $\$256,478.33$ and the processing company charges a $1$\% surcharge on an order. Therefore, the total they will be charged (before tax) for the goods is $\$256478.33(1.10) \approx \$282126.16$. \pspace

\sol The statement is \textit{false}. Because the company is charging a surcharge, the price is going up. We know the final bill will then be 1\%, i.e. we need to compute $\$256,478.33$ increased by 1\%. To compute a percentage increase/decrease of a number $N$ by $P\%$, we compute $N( 1 \pm P_d)$, where $N$ is the number, we choose `$+$' if we are computing an increase and `$-$' if we are computing a decrease, and $P_d$ is the percentage written as a decimal. In our case, writing $1$\% as a decimal, we have $0.01$. But then we have total $\$256\,478.33(1 + 0.01)= 256\,478.33(1.01) \approx \$259,043.11$.





\newpage





% Quiz 4
\quizsol \textit{True/False}: If there are three exams in a class. You received an 85\% on the first, 91\% on the second, and 78\% on the last. The exams are weighted such that the last is worth three times as much as the other two. Then your exam average is given by $85 \left( \frac{1}{5} \right) + 91 \left( \frac{1}{5} \right) + 78 \left( \frac{3}{5} \right)= 82\%$. \pspace

\sol The statement is \textit{true}. If this were an ordinary average, we could simply add up the grades and divide by the number of grades: $\frac{85\% + 91\% + 78\%}{3}= \frac{254\%}{3} \approx 84.7\%$. We can view this as a weighted average by algebraic manipulation and see that the weight is then one over the number of grades: $\frac{85\% + 91\% + 78\%}{3}= 85\% (\frac{1}{3}) + 91\% (\frac{1}{3}) + 78\% (\frac{1}{3}) \approx 28.33\% + 30.33\% + 26.0\% \approx 84.7\%$. In a weighted average, we add up each of the grades times their weight. For an ordinary average, this weight is simply $\frac{1}{n}$, where $n$ is the number of objects. Here, Exam~3 is worth three times as much as the other two. If we split the grade into five parts, the weights of Exam~1, Exam~2, and Exam~3 are $1/5$, $1/5$, and $3/5$, respectively. Then the exam average is $85\% (\frac{1}{5}) + 91\% (\frac{1}{5}) + 78\% (\frac{3}{5})= 17.0\% + 18.2\%.+ 46.8\%= 82\%$. 


% If~1 meter is 39.3701~inches, then to convert 5~m$^2$ to in$^2$, one computes $5 \cdot 39.3701 \approx 196.85$~in$^2$. 
















\end{document}