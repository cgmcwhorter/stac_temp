\documentclass[11pt,letterpaper]{article}
\usepackage[lmargin=1in,rmargin=1in,bmargin=1in,tmargin=1in]{geometry}
\usepackage{style/quiz}
\usepackage{style/commands}

% -------------------
% Content
% -------------------
\begin{document}
\thispagestyle{title}


% Quiz 1
\quizsol \textit{True/False}: If $x \in A$ but $x \notin B$, then $x \notin A \cup B$. \pspace

\sol The statement is \textit{false}. Recall that if $S$ is a set, then $x \in S$ means that $x$ is an element of $S$. For example, suppose $S= \{ 1, 2, 3, 4, 5\}$. If $x= 1$, then $x \in S$. However, if $x= 9$ then $x$ is not in $S$, i.e. $x \notin S$. Recall also that $A \cup B$ is the set of elements that are either in $A$ or in $B$---including the possibility that it might be in both! Because $x \in A$, even though $x \notin B$, we know that $x \in A \cup B$ because $x$ is in $A$ or $B$---after all, it's in $A$! As a concrete example, take $A= \{ 1, 2, 3 \}$ and $B= \{ 2, 3, 4 \}$. Then we know that $A \cup B= \{ 1, 2, 3, 4 \}$. Now if $x= 1$, then $x \in A$ and $x \notin B$. But we can see that $x \in A \cup B$. \pvspace{1.5cm}



% Quiz 2
\quizsol \textit{True/False}: A person's salary at their first job is a function of their number of years of schooling. \pspace

\sol The statement is \textit{false}. Recall that a relation is a function if for each input, there is only one possible output, i.e. $f(x)$ is a function if given each $x$, there is one and only one possible output $f(x)$. In this example, we are wondering whether a person's income, $I$, is a function of their number of years of school, $n$. So is $I(n)$ a function? Then there would be one and only one possible output, i.e. salary, given some number of years of schooling. For instance, if $n= 4$~years (of high school, college, etc.), then the person's salary, $I(4)$, would have only one possible value. But we know that there are many people with the same number of years of schooling that have the same salary! For instance, there will be many people that graduate together (most having the same number of years of schooling) and will have widely varying salaries. Therefore, $I(n)$ cannot be a function, i.e. we cannot exactly predict someone's income from their number of years of school. However, one could try to perform statistical analysis on this problem, e.g. what does the \textit{average} person make if they have $n$ years of schooling. \pvspace{1.5cm}









\end{document}