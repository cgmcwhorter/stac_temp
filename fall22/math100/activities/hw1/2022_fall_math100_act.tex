\documentclass[11pt,letterpaper]{article}
\usepackage[lmargin=1in,rmargin=1in,tmargin=1in,bmargin=1in]{geometry}
\usepackage{../style/homework}
\usepackage{../style/commands}
\setbool{quotetype}{true} % True: Side; False: Under
\setbool{hideans}{true} % Student: True; Instructor: False

% -------------------
% Content
% -------------------
\begin{document}

\homework{Markup/down, Taxes, \& CPI}

% Problem 1
\problem Robin buys a cellphone for \$899.99. The tax on the phone is 7\%. Find the amount she is charged in tax and the total amount she pays for the phone. \pspace

% Problem 2
\problem Robert goes shopping and buys 10~lemons for \$1.25 per lemon, strawberries for \$6.99, two bags of sugar for \$4.99, and a jug for \$15.99. How much is the total cost of the goods? If he pays 5\% in taxes, what is the total amount that he spends? \pspace

% Problem 3
\problem Alice goes to buy jeans. The jeans cost \$49.99 but are on sale. The sale sign indicates that they are 30\% off. If she pays 7.5\% in sales tax, what is the total cost per jean? How much does she pay in total buying six of these jeans on sale? \pspace

% Problem 4
\problem Suppose that you are filing your taxes. You are a single filer taking the standard deduction of \$2,700. You made \$82,000 last year. Use the table below to find the amount you pay in federal taxes. \par
	\begin{table}[!ht]
	\centering
	\begin{tabular}{|l|l|} \hline
	Taxable Income & Tax Owed \\ \hline \hline
	\$0--\$10,275 & 10\% of taxable income \\ \hline
	\$10,276--\$41,775 & \$1,027.50 + 12\% amount over \$10,275 \\ \hline
	\$41,776--\$89,075 & \$4,807.50 + 22\% amount over \$41,775 \\ \hline
	\$89,076--\$170,050 & \$15,213.50 + 24\% amount over \$89,075 \\ \hline
	\$170,051--\$215,950 & \$34,647.50 + 32\% amount over \$170,050 \\ \hline
	\$215,951--\$539,900 & \$49,335.50 + 35\% amount over \$215,950 \\ \hline
	$\geq$ \$539,901 & \$162,718 + 37\% amount over \$539,900 \\ \hline
	\end{tabular}
	\end{table}


% Problem 5
\problem Suppose that you are filing your taxes. You are a single filer taking the standard deduction of \$1,900. You made \$66,000 last year. Use the table below to find the amount you pay in federal taxes. \par
	\begin{table}[!ht]
	\centering
	\begin{tabular}{|l|l|} \hline
	Tax Rate & Taxable Income \\ \hline \hline
	10\% & Up to \$10,275 \\ \hline
	12\% & \$10,276--\$41,775 \\ \hline
	22\% & \$41,776--\$89,075 \\ \hline
	24\% & \$89,076--\$170,050 \\ \hline
	32\% & \$170,051--\$215,950 \\ \hline
	35\% & \$215,951--\$539,900 \\ \hline
	37\% & $\geq$ \$539,901 \\ \hline
	\end{tabular}
	\end{table}

% Problem 6
\problem Suppose that the CPI last year was \$255.67 and this year it is \$268.21. Find the inflation rate from last year to this year. If a good cost \$26.50 last year, what do you estimate that it will cost this year? \pspace



\newpage



% Problem 7
\problem Suppose that the CPI last year was \$157.33 and this year it is \$161.22. Find the inflation rate from last year to this year. If this rate of inflation continues, find how much more goods will cost 6~years from now compared to this year. 


\end{document}