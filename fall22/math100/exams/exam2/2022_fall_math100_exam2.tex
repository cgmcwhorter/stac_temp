\documentclass[12pt,letterpaper]{exam}
\usepackage[lmargin=1in,rmargin=1in,tmargin=1in,bmargin=1in]{geometry}
\usepackage{../style/exams}

% -------------------
% Course & Exam Information
% -------------------
\newcommand{\course}{MAT 100: Exam 2}
\newcommand{\term}{Fall -- 2022}
\newcommand{\examdate}{11/14/2022}
\newcommand{\timelimit}{85 Minutes}

\setbool{hideans}{true} % Student: True; Instructor: False

% -------------------
% Content
% -------------------
\begin{document}

\examtitle
\instructions{Write your name on the appropriate line on the exam cover sheet. This exam contains \numpages\ pages (including this cover page) and \numquestions\ questions. Check that you have every page of the exam. Answer the questions in the spaces provided on the question sheets. Be sure to answer every part of each question and show all your work.} 
\scores
%\bottomline
\newpage

% ---------
% Questions
% ---------
\begin{questions}

% Question 1
\newpage
\question[10] Suppose that the milage of a car, $M$, after $t$~years is modeled by $M(t)= 8500t + 48000$. 
	\begin{enumerate}[(a)]
	\item Find the number of miles on the car after 4~years.
	\item How long until the car's milage is 100,000~miles?
	\end{enumerate}



% Question 2
\newpage
\question[10] Suppose that a worker at a local warehouse is paid an hourly wage of \$20/hour. Explain why the worker's net salary is a linear function. 



% Question 3
\newpage
\question[10] An oil company is selling off oil in one of their reserves. The amount of oil in the tank in gallons, $O$, after $d$ days is given by $O(d)= 180000 - 19000d$.
	\begin{enumerate}[(a)]
	\item Find and interpret the slope of $O(d)$ in the context of the problem. 
	\item Find and interpret the $y$-intercept of $O(d)$ in the context of the problem. 
	\end{enumerate}



% Question 4
\newpage
\question[10] If the CPI was \$284.581 last year and this year it is \$301.779, find the inflation rate from last year to this year. 



% Question 5
\newpage
\question[10] Suppose you make \$72,000 in a year and take a standard deductible of \$13,200. Find your federal income tax. \par
	\begin{table}[!ht]
	\centering
	\begin{tabular}{|l|l|} \hline
	Tax Rate & Taxable Income \\ \hline \hline
	10\% & Up to \$10,275 \\ \hline
	12\% & \$10,276 -- \$41,775 \\ \hline
	22\% & \$41,776 -- \$89,075 \\ \hline
	24\% & \$89,076 -- \$170,050 \\ \hline
	32\% & \$170,051 -- \$215,950 \\ \hline
	35\% & \$215,951 -- \$539,900 \\ \hline
	37\% & $\geq$ \$539,901 \\ \hline
	\end{tabular}
	\end{table}



% Question 6
\newpage
\question[10] If you placed \$500 into an account which earns 1.2\% annual interest, compounded semiannually, how long until there is \$600 in the account? 



% Question 7
\newpage
\question[10] Suppose that you take out a loan for \$13,000 at 6.5\% annual interest, compounded monthly. How much do you owe on the loan after 3~years?



% Question 8
\newpage
\question[10] If you invest \$5,400 in an account which earns 2.3\% annual interest, compounded quarterly, how much interest has been earned from this investment after 6~years?



% Question 9
\newpage
\question[10] Researchers at a think tank work in a variety of fields and have a wide range of ages. Below is a summary of the workers at the facility. \par
	\begin{table}[!ht]
	\centering
	\begin{tabular}{|c||c|c|c|c||c|} \hline
	& Biology & Chemistry & Physics & Computer Science & Total \\ \hline
	18 -- 30 & 26 & 25 & 18 & 10 & 79 \\ \hline
	30 -- 40 & 21 & 19 & 13 & 17 & 70 \\ \hline 
	40 -- 60 & 14 & 18 & 19 & 28 & 79 \\ \hline
	60+ & 13 & 19 & 22 & 22 & 76 \\ \hline \hline
	Total & 74 & 81 & 72 & 77 & 304 \\ \hline
	\end{tabular}
	\end{table} \par

\begin{enumerate}[(a)]
\item Find the probability that a randomly selected worker is 30--40 or researches Physics.
\item Find the probability that a randomly selected worker is over 60 and researchers Biology.
\item Find the probability that Computer Science researcher is 18--30. 
\end{enumerate}



% Question 10
\newpage
\question[10] Researchers are investigating people's movie preferences. Below is a summary of their data broken down by gender. \par
	\begin{table}[!ht]
	\centering
	\begin{tabular}{|l||c|c|c|c||c|} \hline
	& Action & Horror & Comedy & Drama & Total \\ \hline
	Male & 60 & 45 & 70 & 53 & 228 \\ \hline
	Female & 51 & 38 & 65 & 67 & 221 \\ \hline 
	Unspecified & 40 & 30 & 20 & 15 & 105 \\ \hline \hline
	Total & 151 & 113 & 155 & 135 & 554 \\ \hline
	\end{tabular}
	\end{table} \par

\begin{enumerate}[(a)]
\item Find the percentage of people that were female or preferred horror. 
\item Find the percentage of people that were male and preferred drama.
\item Assuming a person preferred action, what was the probability that their gender was unspecified? 
\end{enumerate}



% Question 11
\newpage
\question[10] At a local college with 1,540 students, 431 students have a minor in a STEM field, 687 students have a minor in the Humanities, and 84 students have a minor in both. Complete the Venn diagram below and find the percentage of students that do not have a minor in STEM nor the Humanities. 
	\[
	\begin{tikzpicture}
	\draw (0,0) rectangle (9,6);
	\draw (3.5,3) circle (2);
	\draw (5.5,3) circle (2);
	
	\node at (3.2,5.3) {};
	\node at (5.7,5.3) {}; 
	
	\node at (2.5,3) {};
	\node at (4.5,3) {};
	\node at (6.5,3) {};
	\node at (8,1) {};
	\end{tikzpicture}
	\]



% Question 12
\newpage
\question[10] Suppose that if a student studies for an exam, there is an 85\% chance that they pass the exam. If a student does not study for an exam, there is a 80\% chance that they fail the exam. A school estimates that 70\% of their students study for their exams. Complete the tree diagram below and find the percentage of students at this school that fail their exam. 
	\[
	\begin{tikzpicture}[scale= 1.0]
	\def\FirstUpLabel{}
	\def\FirstDownLabel{}
	\def\SecondUpLabel{}
	\def\SecondDownLabel{}
	\def\Up{$$}
	\def\Down{$$}
	\def\UpUp{$$}
	\def\UpDown{$$}
	\def\DownUp{$$}
	\def\DownDown{$$}
	\def\first{$$}
	\def\second{$$}
	\def\third{$$}
	\def\fourth{$$}
		
	\node at (1,1) {\FirstUpLabel};	
	\node at (1,-1) {\FirstDownLabel};	
	\node at (1.8,0.6) {\Up};
	\node at (1.8,-0.6) {\Down};
	\draw[thick] (0,0) -- (2.5,1.5);
	\draw[thick] (0,0) -- (2.5,-1.5);
		
	\node at (3.75,2.3) {\SecondUpLabel};
	\node at (3.75,0.7) {\SecondDownLabel};
	\node at (4,1.8) {\UpUp};
	\node at (4.1,1.2) {\UpDown};
	\node at (5.75,2.5) {\first};
	\node at (5.75,0.5) {\second};
	\draw[thick] (2.5,1.5) -- (5,2.5);
	\draw[thick] (2.5,1.5) -- (5,0.5);

	\node at (3.75,-0.7) {\SecondUpLabel};
	\node at (3.75,-2.3) {\SecondDownLabel};
	\node at (4.1,-1.2) {\DownUp};
	\node at (4.1,-1.8) {\DownDown};
	\node at (5.75,-0.5) {\third};	
	\node at (5.75,-2.5) {\fourth};	
	\draw[thick] (2.5,-1.5) -- (5,-0.5);
	\draw[thick] (2.5,-1.5) -- (5,-2.5);
	\end{tikzpicture}
	\] \pspace



% Question 13
\newpage
\question[10] The probability of a car over 10~years old having a critical issue is 45\%. Researchers estimate that 13\% of cars are Ford brand cars. The same researchers then estimate that $0.45 \cdot 0.13= 0.0585 \squiggle 5.85\%$ of cars that are over 10~years old having a critical issue are Ford brand. Explain what is wrong with the researchers mathematical calculation. 


\end{questions}
\end{document}