\documentclass[12pt,letterpaper]{exam}
\usepackage[lmargin=1in,rmargin=1in,tmargin=1in,bmargin=1in]{geometry}
\usepackage{../style/exams}

% -------------------
% Course & Exam Information
% -------------------
\newcommand{\course}{MAT 100: Exam 3}
\newcommand{\term}{Fall -- 2022}
\newcommand{\examdate}{12/14/2022}
\newcommand{\timelimit}{85 Minutes}

\setbool{hideans}{true} % Student: True; Instructor: False

% -------------------
% Content
% -------------------
\begin{document}

\examtitle
\instructions{Write your name on the appropriate line on the exam cover sheet. This exam contains \numpages\ pages (including this cover page) and \numquestions\ questions. Check that you have every page of the exam. Answer the questions in the spaces provided on the question sheets. Be sure to answer every part of each question and show all your work.} 
\scores
%\bottomline
\newpage

% ---------
% Questions
% ---------
\begin{questions}

% Question 1
\newpage
\question[10] Mark the following statements as True (T) or False (F): \pspace
\begin{enumerate}[(a)]
\item \underline{\hspace{1.5cm}}: A notion of center is robust if it is resistant to outliers. \vfill
\item \underline{\hspace{1.5cm}}: The mean of a dataset is always larger than its median. \vfill
\item \underline{\hspace{1.5cm}}: The standard deviation of a dataset measures its `spread.' \vfill
\item \underline{\hspace{1.5cm}}: The IQR measures the spread about the mean. \vfill
\item \underline{\hspace{1.5cm}}: The $z$-score of a value measures how far it is from the mean. \vfill
\item \underline{\hspace{1.5cm}}: The larger the magnitude of a $z$-score, the more `unusual' the corresponding value. \vfill
\item \underline{\hspace{1.5cm}}: For any dataset, $P_{50}$ will always be the median. \vfill
\item \underline{\hspace{1.5cm}}: The larger the value of $z_{x_0}$, the larger $P(X < x_0)$ will be. \vfill
\item \underline{\hspace{1.5cm}}: By the Central Limit Theorem, if one samples from a distribution with mean $\mu$ and standard deviation $\sigma$, then the distribution of means of groups, $\overline{X}$, of size $n$ is given by $N(\mu, \sigma/\sqrt{n})$. \vfill
\item \underline{\hspace{1.5cm}}: For fixed $p$, if $n$ is `large', $B(n, p) \approx N\big(np, \sqrt{np(1 - p)} \big)$. \vfill
\end{enumerate}



% Question 2
\newpage
\question[10] Being as detailed and accurate as possible, answer the following:
	\begin{enumerate}[(a)]
	\item Define what it means for a measurement of center to be robust. \vfill
	\item Given a value $x$ from a normal distribution, what does $z_x$ measure? \vfill
	\item For a finite collection of numbers, what is the IQR of the dataset and what does it measure? \vfill
	\item For a finite collection of numbers, what does the standard deviation measure and how is it defined? \vfill
	\item What assumption is required for events in a binomial distribution? \vfill
	\end{enumerate}



% Question 3
\newpage
\question[10] Suppose that you have the following dataset:
	\[
	-2 \qquad 0 \qquad 1 \qquad 3 \qquad 8
	\]
Showing all your work, compute the mean and standard deviation for this dataset. 



% Question 4
\newpage
\question[10] Suppose you have the following dataset:
	\[
	4 \qquad 13 \qquad 24 \qquad 30 \qquad 6 \qquad 10 \qquad 14 \qquad 1 \qquad 8 \qquad 3 \qquad 28
	\]
Showing all your work, compute the 5-number summary for this dataset.



% Question 5
\newpage
\question[10] Suppose you have the following dataset:
	\[
	4 \qquad 13 \qquad 24 \qquad 30 \qquad 6 \qquad 10 \qquad 14 \qquad 1 \qquad 8 \qquad 3 \qquad 28
	\]
Showing all your work, compute $P_{20}$ and $P_{60}$. 



% Question 6
\newpage
\question[10] Suppose you have the following dataset:
	\[
	4 \qquad 13 \qquad 24 \qquad 30 \qquad 6 \qquad 10 \qquad 14 \qquad 1 \qquad 8 \qquad 3 \qquad 28
	\]
Give a box plot for this dataset. 



% Question 7
\newpage
\question[10] Below are four different normal distributions, labeled I -- IV:
	\newcommand\gauss[2]{1/(#2*sqrt(2*pi))*exp(-((x-#1)^2)/(2*#2^2))}
	\pgfplotsset{every tick label/.append style={font=\small}}
	\[
	\begin{tikzpicture}[scale=1]
	\begin{axis}[
	xmin= -10.5, xmax= 10.5,
	ymin= 0, ymax = 1.1,
	xtick= {-10,-8,...,10},
	axis x line= middle,
	axis y line= none,
	]
	% I 
	\addplot[line width=0.03cm, samples=150, domain= -10:8] {\gauss{8}{0.5}} node[above,pos=1] {I};
	\addplot[line width=0.03cm, samples=150, domain= 8:10] {\gauss{8}{0.5}};
	% II
	\addplot[line width=0.03cm, samples=150, domain= -10:-2] {\gauss{-2}{1}} node[above,pos=1] {II};
	\addplot[line width=0.03cm, samples=150, domain= -2:10] {\gauss{-2}{1}};	
	% III
	\addplot[line width=0.03cm, samples=150, domain= -10:-5] {\gauss{-5}{3}} node[above,pos=1] {III};
	\addplot[line width=0.03cm, samples=150, domain= -5:10] {\gauss{-5}{3}};
	% IV 
	\addplot[line width=0.03cm, samples=150, domain= -10:3] {\gauss{3}{2}} node[above,pos=1] {IV};
	\addplot[line width=0.03cm, samples=150, domain= 3:10] {\gauss{3}{2}};
	\end{axis}
	\end{tikzpicture}
	\]
\begin{enumerate}[(a)]
\item Arrange these normal distributions by the size of their means.
\item Arrange these normal distributions by the size of their standard deviations. 
\end{enumerate}



% Question 8
\newpage
\question[10] Dwight and Danny are salesmen at two different paper companies. Dwight earns \$20,000 a year in commissions while Danny earns \$15,200. Commissions at Dwight's company are normally distributed with mean $\$15,\!620$ and standard deviation $\$2,\!100$. Danny works at a company whose sales commissions have distribution $N(\$12350, \$1300)$. Relative to other workers at their companies, who is the better salesman? Be sure to fully justify your answer. 



% Question 9
\newpage
\question[10] Suppose that professional YouTube channels earn a yearly salary that is approximately normally distributed with mean \$46,293 with standard deviation \$8,527. What would your `professional' YouTube channel have to earn to be in the top 6\% of YouTube channels? 



% Question 10
\newpage
\question[10] Suppose that you have a normal distribution with mean 140.3 and standard deviation 16.4. Showing all your work, find the following:
	\begin{enumerate}[(a)]
	\item $P(X= 140.3$)
	\item $P(X \leq 130)$
	\item $P(X \geq 130)$
	\item $P(X \leq 180)$
	\item $P(130 \leq X \leq 180)$
	\end{enumerate}



% Question 11
\newpage
\question[10] Suppose you have a normal distribution with mean 140.3 and standard deviation 16.4. You take a simple random sample from this distribution of size 8 and take the sample mean, $\overline{X}$. Showing all your work, find the following:
	\begin{enumerate}[(a)]
	\item $P(\overline{X}= 140.3$)
	\item $P(\overline{X} \leq 130)$
	\item $P(\overline{X} \geq 130)$
	\item $P(\overline{X} \leq 180)$
	\item $P(130 \leq \overline{X} \leq 180)$
	\end{enumerate}



% Question 12
\newpage
\question[10] Suppose that only 20\% of professors at a college use TokTik. You take a simple random sample of 11 professors from this college.
	\begin{enumerate}[(a)]
	\item Find the probability that exactly 3 of the professors use TokTik.
	\item Find the probability that at most 3 of the professors use TokTik.
	\item Find the probability that less than 3 of the professors use TokTik.
	\item Find the probability that at least 8 of the professors use TokTik.
	\item Find the probability that at least 1 of the professors use TokTik.
	\end{enumerate}



% Question 13
\newpage
\question[10] In a large lecture class, there are 263 students. From historical academic records, it is estimated that 16\% of the students will fail the course. 
	\begin{enumerate}[(a)]
	\item How approximately how many students do you estimate will likely fail this course?
	\item What is the probability that less than 30 students will fail this course?
	\end{enumerate}



% Question 14
\newpage
\question[10] You are taking a history class. The professor states that the class average is approximately normally distributed with mean 72.6 with standard deviation 8.3. Suppose you take a sample of 5 students from the course. What is the probably that the average of these five student's averages is less than 65?


\end{questions}
\end{document}