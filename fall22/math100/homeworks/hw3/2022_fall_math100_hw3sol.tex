\documentclass[11pt,letterpaper]{article}
\usepackage[lmargin=1in,rmargin=1in,tmargin=1in,bmargin=1in]{geometry}
\usepackage{../style/homework}
\usepackage{../style/commands}
\setbool{quotetype}{false} % True: Side; False: Under
\setbool{hideans}{false} % Student: True; Instructor: False

% -------------------
% Content
% -------------------
\begin{document}

\homework{3: Due 09/21}{I think the big mistake in schools is trying to teach children anything, and by using fear as the basic motivation. Fear of getting failing grades, fear of not staying motivated with your class, etc. Interest can produce learning on a scale compared to fear as nuclear explosion to firecracker.}{Stanley Kubrick}


% Problem 1
\problem{10} Alfonso and Lydia are debating over the rise in prices of consumer goods. Economists state that over the past 3~years, prices have risen an average of 4\% per year. Alfonso then argues that then over the past 3~years, prices have risen a total of 12\% from 3~years ago. Lydia argues that this is not true and that the actual percentage is approximately a 12.49\% rise in prices compared to 3~years ago. Who is correct? Explain your reasoning by computing the rise in price now for a good that cost \$5,000 three years ago. What does this have to do with a mathematical error explained in the latter half of this video: \href{https://www.youtube.com/watch?v=aokNwKx7gM8&ab_channel=Stand-upMaths}{Do these scatter plots reveal fraudulent vote-switching in Michigan?} \pspace

\sol Lydia is correct. To see why, let us apply both Alfonso's and Lydia's logic. According to Alfonso, the prices have gone up 4\% each year for a total of $4\% + 4\% + 4\%= 12\%$ compared to 3~years ago. Therefore, a good that costs \$5,000 three years ago will now cost $\$5000(1 + 0.12)= \$5000(1.12)= \$5600$. According to Lydia, the costs of goods have gone up by 12.49\% compared to 3~years ago. Therefore, an item that cost \$5,000 three years ago will now cost $\$5000(1 + 0.1249)= \$5000(1.1249)= \$5624.50$. \pspace

Now let us compute the cost of the good directly. Three years ago, the good cost \$5,000. After one year (so two years ago), the cost goes up by 4\%, so that it then costs $\$5000(1 + 0.04)= \$5000(1.04)= \$5200$. After another year (one year ago), the cost goes up by 4\%, so that it then costs $\$5200(1 + 0.04)= \$5200(1.04)= \$5408$. After another year (bringing us to the current year), the cost goes up by 4\%, so that it then costs $\$5408(1 + 0.04)= \$5408(1.04)= \$5624.32$. \pspace

Clearly, Lydia's value is closer to reality than Alfonso's value. The reason for this is that percentages do not add in this way. Percentages only add if they are applied to a static (unchanging) value. Whereas here, the 4\% increase is being applied to a new value each year. If the price is $P$~dollars now and goes up by \% each year, then the cost after $t$ years will be $P(1 + \%_d)^t$, where $\%_d$ is the percentage written as a decimal. Applying this to the current scenario, we have\dots \pspace
	\[
	P(1 + 0.04)^3= P(1.04)^3= P(1.12486).
	\] \pspace
We can recognize this product as representing a 12.486\% increase to $P$. Because $12.486\% \approx 12.49\%$, we can see that Lydia's interpretation is correct. \pspace

While the answers for the video may vary, one of the errors discussed in the video is precisely the idea that percentages typically do not add in the way that many people, e.g. Alfonso, believe that they do. 



\newpage



% Problem 2
\problem{10} Barb E. Dahl is taking a Sociology course that has the following grade components: \par
	\begin{table}[!ht]
	\centering
	\begin{tabular}{lrclr}
	Participation & 10\% & &  Quizzes & 5\% \\
	Project & 25\% & & Midterm & 10\% \\
	Homework & 35\% & & Final & 15\% \\
	\end{tabular}
	\end{table} \par
She has a 100\% participation average, 85\% homework average, 72\% quiz average, and she received a 78\% on the midterm. There are two weeks left in the course, with the project and final exam the last day of the semester. Barb is confident she can get at least an 80\% on the project. What is the minimum grade she then needs to get on the final exam to receive an 80\% in the course? Can she get a 92\% or higher? Explain. Also, what are some of the assumptions that go into these grade estimations? \pspace

\sol We can compute Barb's course average by computing her weighted average for the course. Recall that a weighted average is given by\dots 
	\[
	\sum \text{value} \cdot \text{weight},
	\] 
where $\sum$ stands for the sum of the values value~$\cdot$~weight. For instance, if one value is the 10\% project grade and one has a 100\% on the project, one receives $10\%(1.00)= 10\%$ on that portion of the course grade. If one receives a 50\% on the project, one receives $10\%(0.50)= 5\%$ on that portion of the course grade. One finds each of these values for each portion of the course grade and adds them up. So we compute Barb's weighted course average, assuming she receives an 80\% on the project, that her other grades are fixed, and that she receives an average of $G$ on her final exam: 
	\[
	\begin{aligned}
	\text{Course Average}&= \sum \text{value} \cdot \text{weight} \\[0.3cm]
	&= 10\% (1.00) + 25\% (0.80) + 35\% (0.85) + 5\% (0.72) + 10\% (0.78) + 15\% \cdot G \\[0.3cm]
	&= 10.0 + 20.0 + 29.75 + 3.6 + 7.8 + 15G \\[0.3cm]
	&= 15A + 71.15.
	\end{aligned}
	\] 
To receive an `A' in the course, she must receive at least a 92 in the course. But then we have\dots
	\[
	\begin{aligned}
	15G + 71.15&\geq 92 \\[0.3cm]
	15G + 71.15 - 71.15&\geq 92 - 71.15 \\[0.3cm]
	15G&\geq 20.15 \\[0.3cm]
	\dfrac{\cancel{15}G}{\cancel{15}}&\geq \dfrac{18.15}{15} \\[0.3cm]
	G&\geq 1.343
	\end{aligned}
	\]
But then Barb must receive at least a 134.3\% on the final exam. As this is impossible, she cannot receive a 92\% or higher in the course. However, this assume she only gets an 80\% on the project and that her homework and quiz average cannot go up. So while it is not likely, it is not necessarily impossible, depending on how many grades remain in these components. 




\newpage



% Problem 3
\problem{10} Jerry Atrick is finishing his undergraduate at Scatter. Thus far, he has taken 114~credits and has a 3.552 GPA. While he is currently cum laude (GPA at least 3.5), he would like to finish magna cum laude (a GPA at least 3.65) or even summa cum laude (GPA 3.8 or higher). The courses he is taking in his final semester are shown below.  [The college's letter grade scheme is show above on the right.] \par
	\begin{table}[!ht]
	\centering
	\begin{tabular}{lr}
	Course & Credits \\ \hline
	Honors Algebra & 4 \\
	Connections in Advanced Mathematics & 3 \\
	Mathematics Capstone II & 1 \\
	Women in Music & 3 \\
	Advanced Physics Research & 2 \\
	Argument & 3
	\end{tabular} \hspace{1cm}
        \begin{tabular}{|l||c|l||c|} \hline
        A & 4.0 & C+ & 2.3 \\ \hline
        A-- & 3.7 & C & 2.0 \\ \hline
        B+ & 3.3 & C-- & 1.7 \\ \hline
        B & 3.0 & D & 1.0 \\ \hline
        B-- & 2.7 & F & 0.0 \\ \hline
        \end{tabular}
	\end{table} \par
What is the lowest and the highest GPA that Jerry can earn? What do you think his `likely' GPA will be, i.e. give a sample prediction of what his final GPA will be if he does `reasonably' this semester. Does it seem likely that Jerry will achieve his goal? \pspace

\sol We can compute GPA using: \pspace
	\[
	\dfrac{\sum \text{value} \cdot \text{credits}}{\text{total credits}}.
	\] \pspace
If we are computing a GPA `mid-career', we can compute one's new GPA from the old one using: \pspace
	\[ 
	\text{New GPA}= \dfrac{\text{old GPA} \cdot \text{old credits} + \text{current GPA} \cdot \text{current credits}}{\text{old credits} + \text{current credits}}.
	\] \pspace
That is, the new GPA is the weighted average of the old GPA with the current GPA. Jerry's highest possible GPA will be if he receives a 4.0, i.e. all A's, this semester on the $4 + 3 + 1 + 3 + 2 + 3= 16$~credits that he took. Then his new (highest possible) GPA would be\dots \pspace
	\[
	\text{Highest GPA}= \dfrac{3.552(114) + 4.0(16)}{114 + 16}= \dfrac{404.928 + 64.000}{130}= \dfrac{468.928}{130} \approx 3.607.
	\] \pspace
Jerry's lowest possible GPA will be if he receives a 0.0, i.e. all F's, this semester on the 16~credits that he took. Then his new (lowest possible) GPA would be\dots \pspace
	\[
	\text{Lowest GPA}= \dfrac{3.552(114) + 0.0(16)}{114 + 16}= \dfrac{404.928 + 0.000}{130}= \dfrac{404.928}{130} \approx 3.115.
	\] \pspace
Examining these cases, it is not possible for Jerry to graduate magna cum laude or summa cum laude. However, if Jerry continues performing at the level he has been, he may graduate cum laude. 





\newpage



% Problem 4
\problem{10} Natural gas traders often do not measure the average price per gallon of gas in a region. Instead, they measure the average gas price per gallon weighted by the volumes purchased from various regions. Suppose one natural gas trader is measuring the average gas price per gallon in a small local community. Over the course of a week, the price and volumes of gas purchased at the three local gas stations are given below: \par
	\begin{table}[!ht]
	\centering
        \begin{tabular}{lrr} \hline
	Station & Price (\$) & Volume (Gallons) \\ \hline
	Gas N' Go & \$3.76 & 1,600 \\
	QuickieMart & \$3.79 & 1,200 \\
	Sunny Plains & \$3.72 & 5,200
        \end{tabular}
	\end{table} \par
Find the average gas price per gallon `normally' and then the average gas price per gallon weighted by volume. Explain the differences between the two. \pspace

\sol In a normal average, we add all the possible values and divide by the total number of values. So the `normal' average is\dots 
	\[
	\text{Average}= \dfrac{\$3.76 + \$3.79 + \$3.72}{3}= \dfrac{\$11.27}{3} \approx \$3.76.
	\]
Recall that a weighted average is given by\dots 
	\[
	\sum \text{value} \cdot \text{weight},
	\] 
where $\sum$ stands for the sum of the values value~$\cdot$~weight. The values here are the prices and the weight factor is the volume sold at that price level divided by the total volume of gas sold. There was a total of 8,000~gallons of gas sold. Then the weighted average is\dots
	\[
	\begin{aligned}
	\text{Weighted Average}&= \sum \text{value} \cdot \text{weight} \\[0.3cm]
	&= \$3.76 \cdot \dfrac{1600}{8000} + \$3.79 \cdot \dfrac{1200}{8000} + \$3.72 \cdot \dfrac{5200}{8000} \\[0.3cm]
	&= \$3.76 (0.20) + \$3.79 (0.15) + \$3.72 (0.65) \\[0.3cm]
	&= \$0.752 + \$0.5685 + \$2.418 \\[0.3cm]
	&= \$3.7385 \\[0.3cm]
	&\approx \$3.74
	\end{aligned}
	\]
The difference between a `normal' average and a weighted average is that a weighted average considers what are the `typical' values, whereas a `normal' average weights everything equally. For instance, if 99\% of people buy a product for \$100 and 1\% of people buy the same product for \$10, a `normal' average would say the product costs $(\$100 + \$10)/2= \$55$. However, no one pays a price near that value. In fact, the overwhelming majority of people pay far more than that. Whereas the weighted average considers what is `typical': $\$100(0.99) + \$10(0.01)= \$99.10$. `Normal' averages are `good' when all the values occur essentially equal frequency; otherwise, one should consider using a weighted average. Notice here, because `most' people are paying \$3.72 for gas and not the maximum price of \$3.76, the weighted average is a better measure of `average' gas price for this town. 




\end{document}