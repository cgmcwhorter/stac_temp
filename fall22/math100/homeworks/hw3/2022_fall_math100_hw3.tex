\documentclass[11pt,letterpaper]{article}
\usepackage[lmargin=1in,rmargin=1in,tmargin=1in,bmargin=1in]{geometry}
\usepackage{../style/homework}
\usepackage{../style/commands}
\setbool{quotetype}{false} % True: Side; False: Under
\setbool{hideans}{true} % Student: True; Instructor: False

% -------------------
% Content
% -------------------
\begin{document}

\homework{3: Due 09/21}{I think the big mistake in schools is trying to teach children anything, and by using fear as the basic motivation. Fear of getting failing grades, fear of not staying motivated with your class, etc. Interest can produce learning on a scale compared to fear as nuclear explosion to firecracker.}{Stanley Kubrick}


% Problem 1
\problem{10} Alfonso and Lydia are debating over the rise in prices of consumer goods. Economists state that over the past 3~years, prices have risen an average of 4\% per year. Alfonso then argues that then over the past 3~years, prices have risen a total of 12\% from 3~years ago. Lydia argues that this is not true and that the actual percentage is approximately a 12.49\% rise in prices compared to 3~years ago. Who is correct? Explain your reasoning by computing the rise in price now for a good that cost \$5,000 three years ago. What does this have to do with a mathematical error explained in the latter half of this video: \href{https://www.youtube.com/watch?v=aokNwKx7gM8&ab_channel=Stand-upMaths}{Do these scatter plots reveal fraudulent vote-switching in Michigan?} \pspace



\newpage



% Problem 2
\problem{10} Barb E. Dahl is taking a Sociology course that has the following grade components: \par
	\begin{table}[!ht]
	\centering
	\begin{tabular}{lrclr}
	Participation & 10\% & &  Quizzes & 5\% \\
	Project & 25\% & & Midterm & 10\% \\
	Homework & 35\% & & Final & 15\% \\
	\end{tabular}
	\end{table} \par
She has a 100\% participation average, 85\% homework average, 72\% quiz average, and she received a 78\% on the midterm. There are two weeks left in the course, with the project and final exam the last day of the semester. Barb is confident she can get at least an 80\% on the project. What is the minimum grade she then needs to get on the final exam to receive an 80\% in the course? Can she get a 92\% or higher? Explain. Also, what are some of the assumptions that go into these grade estimations?




\newpage



% Problem 3
\problem{10} Jerry Atrick is finishing his undergraduate at Scatter. Thus far, he has taken 114~credits and has a 3.552 GPA. While he is currently cum laude (GPA at least 3.5), he would like to finish magna cum laude (a GPA at least 3.65) or even summa cum laude (GPA 3.8 or higher). The courses he is taking in his final semester are shown below.  [The college's letter grade scheme is show above on the right.] \par
	\begin{table}[!ht]
	\centering
	\begin{tabular}{lr}
	Course & Credits \\ \hline
	Honors Algebra & 4 \\
	Connections in Advanced Mathematics & 3 \\
	Mathematics Capstone II & 1 \\
	Women in Music & 3 \\
	Advanced Physics Research & 2 \\
	Argument & 3
	\end{tabular} \hspace{1cm}
        \begin{tabular}{|l||c|l||c|} \hline
        A & 4.0 & C+ & 2.3 \\ \hline
        A-- & 3.7 & C & 2.0 \\ \hline
        B+ & 3.3 & C-- & 1.7 \\ \hline
        B & 3.0 & D & 1.0 \\ \hline
        B-- & 2.7 & F & 0.0 \\ \hline
        \end{tabular}
	\end{table} \par
What is the lowest and the highest GPA that Jerry can earn? What do you think his `likely' GPA will be, i.e. give a sample prediction of what his final GPA will be if he does `reasonably' this semester. Does it seem likely that Jerry will achieve his goal? 



\newpage



% Problem 4
\problem{10} Natural gas traders often do not measure the average price per gallon of gas in a region. Instead, they measure the average gas price per gallon weighted by the volumes purchased from various regions. Suppose one natural gas trader is measuring the average gas price per gallon in a small local community. Over the course of a week, the price and volumes of gas purchased at the three local gas stations are given below: \par
	\begin{table}[!ht]
	\centering
        \begin{tabular}{lrr} \hline
	Station & Price (\$) & Volume (Gallons) \\ \hline
	Gas N' Go & \$3.76 & 1,600 \\
	QuickieMart & \$3.79 & 1,200 \\
	Sunny Plains & \$3.72 & 5,200
        \end{tabular}
	\end{table} \par
Find the average gas price per gallon `normally' and then the average gas price per gallon weighted by volume. Explain the differences between the two. 


\end{document}