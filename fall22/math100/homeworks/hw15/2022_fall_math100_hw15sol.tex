\documentclass[11pt,letterpaper]{article}
\usepackage[lmargin=1in,rmargin=1in,tmargin=1in,bmargin=1in]{geometry}
\usepackage{../style/homework}
\usepackage{../style/commands}
\setbool{quotetype}{true} % True: Side; False: Under
\setbool{hideans}{false} % Student: True; Instructor: False

% -------------------
% Content
% -------------------
\begin{document}

\homework{15: Due 11/07}{Life is a school of probability.}{Walter Bagehot}

% Problem 1
\problem{10} Researchers surveyed 320~people on which superpower from comics they found most interesting/best. A table of their findings are found below.
	\begin{table}[!ht]
	\centering
	\begin{tabular}{|c||c|c||c|} \hline
	& Child & Adult & Total \\ \hline \hline
	Flight & 30 & 36 & 66 \\ \hline
	Mind Reading & 15 & 60 & 75 \\ \hline
	Super Strength & 40 & 14 & 54 \\ \hline \hline
	Total & 85 & 110 & 195 \\ \hline
	\end{tabular}
	\end{table}

\begin{enumerate}[(a)]
\item Find the probability that a person chose flight as the best superpower.
\item Find the probability that a person surveyed was a child.
\item Find the probability that a person was a child and chose super strength as the most interesting super power.
\item Find the probability that a person chose mind reading or was an adult.
\item Find the probability that a person that chose mind reading was an adult. 
\end{enumerate} \pspace

\sol 
\begin{enumerate}[(a)]
\item 
	\[
	P(\text{flight})= \dfrac{66}{195} \approx 0.3385
	\]

\item 
	\[
	P(\text{child})= \dfrac{85}{195} \approx 0.4359
	\]

\item 
	\[
	P(\text{child and super strength})= \dfrac{40}{195} \approx 0.2051
	\]

\item 
	\[
	P(\text{mind reading or adult})= \dfrac{75 + 110 - 60}{195}= \dfrac{125}{195} \approx 0.6410
	\]

\item 
	\[
	P(\text{adult} \;|\; \text{mind reading})= \dfrac{60}{75} \approx 0.80
	\]
\end{enumerate}


\newpage



% Problem 2
\problem{10} Consider the breakdown of students majors at a college by gender.
	\begin{table}[!ht]
	\centering
	\begin{tabular}{|c||c|c|c|c|} \hline
	& School of Business & School of Education & Arts \& Social Sciences & STEM \\ \hline \hline
	Male & 80 & 85 & 75 & 92 \\ \hline
	Female & 75 & 80 & 90 & 94 \\ \hline
	Unspecified & 8 & 15 & 22 & 1 \\ \hline
	\end{tabular}
	\end{table}

\begin{enumerate}[(a)]
\item Find the probability that a randomly selected person from this school was in education. 
\item Find the probability that a randomly selected person from this school was male.
\item Find the probability that a randomly selected person from this school was a female and in the School of Business.
\item Find the probability that a randomly selected person from this school was in STEM or had unspecified gender.
\item Find the probability that a person was in the school of Arts \& Social Sciences given that their gender was unspecified. 
\end{enumerate} \pspace

\sol First, we should find the totals in each row and column: 
	\begin{table}[!ht]
	\centering
	\begin{tabular}{|c||c|c|c|c||c|} \hline
	& School of Business & School of Education & Arts \& Social Sciences & STEM & Total \\ \hline \hline
	Male & 80 & 85 & 75 & 92 & 332 \\ \hline
	Female & 75 & 80 & 90 & 94 & 339 \\ \hline
	Unspecified & 8 & 15 & 22 & 1 & 46 \\ \hline \hline
	Total & 163 & 180 & 187 & 187 & 717 \\ \hline
	\end{tabular}
	\end{table}

\begin{enumerate}[(a)]
\item 
	\[
	P(\text{education})= \dfrac{180}{717} \approx 0.2510
	\]

\item 
	\[
	P(\text{male})= \dfrac{332}{717} \approx 0.4630
	\]

\item 
	\[
	P(\text{female and business})= \dfrac{75}{717} \approx 0.1046
	\]

\item 
	\[
	P(\text{STEM or unspecified})= \dfrac{187 + 46 - 1}{717}= \dfrac{232}{717} \approx 0.3236
	\]

\item 
	\[
	P(\text{Arts \& Social} \;|\; \text{unspecified})= \dfrac{22}{46} \approx 0.4783
	\]
\end{enumerate}



\newpage



% Problem 3
\problem{10} Is it always true that for events $A$ and $B$, $P(A \text{ and } B)= P(A) \cdot P(B)$? Explain using an original example. \pspace

\sol No. It is true that if $A$ and $B$ are independent events that $P(A \text{ and } B)= P(A) \cdot P(B)$. However, this is not generally true for arbitrary events $A$ and $B$. Generally, we have $P(A \text{ and } B)= P(A) P(B \;|\; A)$ or $P(A \text{ and } B)= P(B) P(A \;|\; B)$. To see why $P(A \text{ and } B)$ may be different than $P(A) \cdot P(B)$, consider the events $A$ and $B$, where $A$ is passing a course and $B$ is failing a course. Clearly, you cannot pass and fail a course at the same time. So $P(A \text{ and } B)= 0$. However, unless $P(A)= 0$ or $P(B)= 0$, then $P(A) \cdot P(B) \neq 0$. Because $A$ and $B$ are disjoint events (they cannot happen at the same time), they cannot be independent. So we should not have expected that $P(A \text{ and } B)= P(A) \cdot P(B)$. 


\end{document}