\documentclass[11pt,letterpaper]{article}
\usepackage[lmargin=1in,rmargin=1in,tmargin=1in,bmargin=1in]{geometry}
\usepackage{../style/homework}
\usepackage{../style/commands}
\setbool{quotetype}{false} % True: Side; False: Under
\setbool{hideans}{true} % Student: True; Instructor: False

% -------------------
% Content
% -------------------
\begin{document}

\homework{11: Due 10/26}{The tax code is a monstrosity and there's only one thing to do with it: scrap it, kill it, drive a stake through its heart, bury it, and hope it never rises again to terrorize the American people.}{Steve Forbes}

% Problem 1
\problem{10} Wyatt is filing his federal taxes. He is filing as a single individual taking the standard deduction of \$2,400. Last year working as a CPA, he made \$72,000. Find the amount he pays in federal taxes. 
	\begin{table}[!ht]
	\centering
	\begin{tabular}{|l|l|} \hline
	Taxable Income & Tax Owed \\ \hline \hline
	\$0--\$10,275 & 10\% of taxable income \\ \hline
	\$10,276--\$41,775 & \$1,027.50 + 12\% amount over \$10,275 \\ \hline
	\$41,776--\$89,075 & \$4,807.50 + 22\% amount over \$41,775 \\ \hline
	\$89,076--\$170,050 & \$15,213.50 + 24\% amount over \$89,075 \\ \hline
	\$170,051--\$215,950 & \$34,647.50 + 32\% amount over \$170,050 \\ \hline
	\$215,951--\$539,900 & \$49,335.50 + 35\% amount over \$215,950 \\ \hline
	$\geq$ \$539,901 & \$162,718 + 37\% amount over \$539,900 \\ \hline
	\end{tabular}
	\end{table}



\newpage



% Problem 2
\problem{10} Suppose that the CPI last year was \$296.808. This year, the CPI is \$301.21. What was the inflation rate from last year to this year? If a good cost \$95.99 last year, what do you predict that it will cost this year?


\end{document}