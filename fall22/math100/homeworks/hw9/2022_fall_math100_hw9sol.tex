\documentclass[11pt,letterpaper]{article}
\usepackage[lmargin=1in,rmargin=1in,tmargin=1in,bmargin=1in]{geometry}
\usepackage{../style/homework}
\usepackage{../style/commands}
\setbool{quotetype}{false} % True: Side; False: Under
\setbool{hideans}{false} % Student: True; Instructor: False

% -------------------
% Content
% -------------------
\begin{document}

\homework{9: Due 10/26}{There's nothing inside you. I invented you inside my head. For 12~years, I've been inventing you---I made you up.}{Sunny Balwani, The Droupout}

% Problem 1
\problem{10} Louise just read \textit{Remembrance of Things Past} by Marcel Proust, which is approximately 3,200~pages long. She kept careful track of the time she was at various pages. She found that she count model the amount of pages she had read, $P$, after $t$ hours by $P(t)= 65t$.
	\begin{enumerate}[(a)]
	\item Find and interpret the slope of $P(t)$.
	\item Find and interpret the $y$-intercept of $P(t)$.
	\item How long did it take her to read this work?
	\end{enumerate} \pspace

\sol 
\begin{enumerate}[(a)]
\item Because the function $P(t)= 65t$ has the form $f(x)= mx + b$ for $m= 65$ and $b= 0$, we know that $P(t)$ is linear and that the slope is $m= 65= \frac{65}{1}$. Interpreting the slope $m= \frac{\Delta P}{\Delta t}$, we can see that every additional hour, Louise increases the page count by 65, i.e. Louise reads 65~pages per hour. \pspace

\item Because the function $P(t)= 65t$ has the form $f(x)= mx + b$ for $m= 65$ and $b= 0$, we know that $P(t)$ is linear and that the $y$-intercept is $b= 0$. That is, at time $t= 0$, there have been zero pages read of the book. Then the $y$-intercept is the fact that at the start, Louise had only just begun reading the book. \pspace

\item She is finished reading the work when she has completed the 3,200~pages, i.e. when $P(t)= 3200$. But then we have\dots
	\[
	\begin{aligned}
	P(t)&= 3200 \\[0.3cm]
	65t&= 3200 \\[0.3cm]
	t&= 49.23
	\end{aligned}
	\]
Therefore, Louise finishes the book after 49.23~hours, i.e. 49~hours, 13~minutes, and 50~seconds. 
\end{enumerate}



\newpage



% Problem 2
\problem{10} Suppose that the number of people, $N$ that have ridden the subway $t$~hours after 8:00~am can be modeled by $N(t)= 8429t - 1008$. 
	\begin{enumerate}[(a)]
	\item Find and interpret the slope of $N(t)$.
	\item Does the $y$-intercept of $N(t)$ have an interpretation in the context of this problem? Explain.
	\item Find the number of people that have ridden the subway by 5~pm. 
	\end{enumerate} \pspace

\sol 
\begin{enumerate}[(a)]
\item Because $N(t)= 8429t - 1008$ has the form $f(x)= mx + b$ with $m= 8429$ and $b= -1008$, we know that $N(t)$ is linear with slope $m= 8429$. Interpreting this as $\frac{\Delta N}{\Delta t}$, we can see that for every additional hour, the number of people that have ridden the subway has increased by 8,429, i.e. 8,429~people ride the subway every hour. \pspace

\item Because $N(t)= 8429t - 1008$ has the form $f(x)= mx + b$ with $m= 8429$ and $b= -1008$, we know that $N(t)$ is linear with $y$-intercept $-1008$, i.e. $(0, -1008)$. That this means at $t= 0$, i.e. 8:00~am, $-1008$ have ridden the subway. As this is impossible, the $y$-intercept does not have an interpretation in the context of the problem. \pspace

\item We know that 5~pm is 9~hours after 8:00~am. But then we have\dots
	\[
	N(9)= 8429(9) - 1008= 75861 - 1008= 74853
	\]
Therefore, 74,853~people have ridden the subway by 5~pm. 
\end{enumerate}


\end{document}