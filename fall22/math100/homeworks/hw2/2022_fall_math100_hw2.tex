\documentclass[11pt,letterpaper]{article}
\usepackage[lmargin=1in,rmargin=1in,tmargin=1in,bmargin=1in]{geometry}
\usepackage{../style/homework}
\usepackage{../style/commands}
\setbool{quotetype}{false} % True: Side; False: Under
\setbool{hideans}{true} % Student: True; Instructor: False

% -------------------
% Content
% -------------------
\begin{document}

\homework{2: Due 09/19}{This is not yours to fix alone. You act like you're all alone out there in the world, but you're not. You're not alone}{Joyce Byers, Stranger Things}

% Problem 1
\problem{10} Showing all your work, compute the following:
	\begin{enumerate}[(a)]
	\item 55\% of 143
	\item 1\% of 3.6
	\item 49\% of 49
	\item 121\% of 4000
	\end{enumerate}



\newpage



% Problem 2
\problem{10} Showing all your work, compute the following:
	\begin{enumerate}[(a)]
	\item 78 increased by 40\%
	\item 94 decreased by 65\%
	\item 166 decreased by 2\%
	\item 1820 increased by 163\%
	\end{enumerate}



\newpage



% Problem 3
\problem{10} Why is 80\% of 485 the same value as the value obtained by reducing 485 by 20\%? Be sure to give an explanation that does not simply involve computing both. Then compute both values as stated. 



\newpage



% Problem 4
\problem{10} Sally Forth is looking to book a vacation trip. She does not want to spend more than \$3,500 on the trip. The prices listed on the travel website she is using to book the trip do not include 7\% sales tax or a \$50 booking surcharge that the website charges (applied to the cost of the trip \textit{before} the tax). What is the highest advertised price on the website that she can book and actually afford? \pspace



\newpage



% Problem 5
\problem{10} Ophelia Pane is taking a Mathematics course. For this Mathematics course, the grading scheme is as follows:
	\begin{table}[!ht]
	\centering
	\begin{tabular}{lrclr}
	Participation & 5\% & \hspace{1.5cm} & Quizzes & 10\% \\
	Activities & 5\% & & Exams & 30\% \\
	Project & 10\% & & Homework & 40\%
	\end{tabular}
	\end{table} \par
Suppose she had a 90\% participation average, 100\% activities average, 85\% project average, 75\% quiz average, and 79\% homework average. She received a 72\% on exam~1, 89\% on exam~2, and 84\% on exam~3, which gave her a 81.67 exam average (because they were weighted equally). What was her final average in the course?



\newpage



% Problem 6
\problem{10} Suppose that Jim Nasium is in his first semester at college. His transcript at the end of the first semester was as follows: \par
	\begin{table}[!ht]
	\centering
	\begin{tabular}{llr}
	Course & Letter Grade & Credits \\ \hline
	First-Year Seminar & A & 1 \\
	Calculus & A$-$ & 4 \\
	Introductory Philosophy & C+ & 3 \\
	German~I & B$-$ & 3 \\
	Writing~I & B+ & 3 \\
	American Poets & D & 3
	\end{tabular} \hspace{1cm}
        \begin{tabular}{|l||c|l||c|} \hline
        A & 4.0 & C+ & 2.3 \\ \hline
        A-- & 3.7 & C & 2.0 \\ \hline
        B+ & 3.3 & C-- & 1.7 \\ \hline
        B & 3.0 & D & 1.0 \\ \hline
        B-- & 2.7 & F & 0.0 \\ \hline
        \end{tabular}
	\end{table} \par
What was Jim's GPA for his first semester? [The college's letter grade scheme is show above on the right.]


\end{document}