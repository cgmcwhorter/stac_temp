\documentclass[11pt,letterpaper]{article}
\usepackage[lmargin=1in,rmargin=1in,tmargin=1in,bmargin=1in]{geometry}
\usepackage{../style/homework}
\usepackage{../style/commands}
\setbool{quotetype}{false} % True: Side; False: Under
\setbool{hideans}{true} % Student: True; Instructor: False

% -------------------
% Content
% -------------------
\begin{document}

\homework{5: Due 09/28}{Guys, Shierley's costume is once again unintentionally ambiguous. I don't know who she's supposed to be, but she's definitely not Miss Piggy. Repeat: not Miss Piggy. You're on your own.}{Annie Edison, Community}

% Problem 1
\problem{10} Water is flowing into a large vat that can contain 288~ft$^3$ of water. Suppose that water is flowing it at a rate of 37.5~gallons per minute. 
	\begin{enumerate}[(a)]
	\item Find the rate at which the water is flowing in ft$^3$ per minute. [1~gallon = 0.134~ft$^3$]
	\item How long does it take to fill the whole tank?
	\item Assuming the tank begins empty, how much of the tank is unfilled after 10~minutes?
	\end{enumerate}



\newpage



% Problem 2
\problem{10} Ann Velope is stuffing envelopes for an upcoming charity event. Counting, she has been able to stuff 116~envelopes in the last 20~minutes. 
	\begin{enumerate}[(a)]
	\item What is her rate in envelopes per hour?
	\item How long will it take for her to fill 1,200 envelopes? 
	\item If her coworker helps her and he can stuff 250~envelopes per hour, how long would it take both of them to stuff 2,000 envelopes? 
	\end{enumerate}



\newpage



% Problem 3
\problem{10} Showing all your work and explaining your logic, estimate the number of times an `average' American inhales during their lifetime. What are sources of error in your estimation? 



\newpage



% Problem 4
\problem{10} Choose an original Fermi estimation problem of your choice and produce an estimate. Be sure to show your work and explain your reasoning. What are sources of error in your estimation?


\end{document}