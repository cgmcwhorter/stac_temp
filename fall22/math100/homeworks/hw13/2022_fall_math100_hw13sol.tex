\documentclass[11pt,letterpaper]{article}
\usepackage[lmargin=1in,rmargin=1in,tmargin=1in,bmargin=1in]{geometry}
\usepackage{../style/homework}
\usepackage{../style/commands}
\setbool{quotetype}{true} % True: Side; False: Under
\setbool{hideans}{false} % Student: True; Instructor: False

% -------------------
% Content
% -------------------
\begin{document}

\homework{13: Due 10/31}{A penny saved is worth two pennies earned\dots after taxes.}{Randy Thurman}

% Problem 1
\problem{10} Consider the 2022--2023 federal income tax brackets shown below. In the chart on the left below, only the brackets and tax levels are found. In the chart on the right below, the tax rate, bracket, and amount of taxes owed are shown. \par
	\begin{table}[!ht]
	\centering
	\scalebox{0.80}{%
	\begin{tabular}{|l|l|} \hline
	Tax Rate & Taxable Income \\ \hline
	10\% & \$0 -- \$10,275 \\ \hline
	12\% & \$10276 -- \$41,775 \\ \hline
	22\% & \$41,776 -- \$89,075 \\ \hline
	24\% & \$89,076 -- \$170,050 \\ \hline
	32\% & \$170,051 -- \$215,950 \\ \hline
	35\% & \$215,951 -- \$539,900 \\ \hline
	37\% & \$539,901 or more \\ \hline
	\end{tabular}} \hfill \scalebox{0.80}{%
	\begin{tabular}{|l|l|l|} \hline
	Tax Rate & Taxable Income & Tax Owed \\ \hline
	10\% & \$0 -- \$10,275 & 10\% of taxable income. \\ \hline
	12\% & \$10276 -- \$41,775 & \$1,027.50 plus 12\% of amount over \$10,275. \\ \hline
	22\% & \$41,776 -- \$89,075 & \$4,807.50 plus 22\% of amount over \$41,775. \\ \hline
	24\% & \$89,076 -- \$170,050 & \$15,213.50 plus 24\% of amount over \$89,075.\\ \hline
	32\% & \$170,051 -- \$215,950 & \$34,647.50 plus 32\% of amount over \$170,050.\\ \hline
	35\% & \$215,951 -- \$539,900 & \$49,335.50 plus 35\% of amount over \$215,950.\\ \hline
	37\% & \$539,901 or more & \$162,718 plus 37\% of amount over \$539,900.\\ \hline
	\end{tabular}}
	\end{table} \par
Explain how the amount of tax owed in the chart on the right is computed by showing how the \$1,027.50 and the \$4,807.50 are obtained. \pspace

\sol Each entry in the `Tax Owed' column on the chart on the right is the tax owed from the previous brackets plus the tax owed in the `current bracket.' For instance, the \$1,027.50 in the second bracket is the tax owed from the initial \$10,275 of taxable income:
	\[
	0.10(\$10275 - \$0)= 0.10(\$10275)= \$1027.50
	\]
If you are in the third tax bracket, then you have paid\dots
	\[
	0.10(\$10275 - \$0) + 0.12(\$41775 - \$10275)= 0.10(\$10275) + 0.12(\$31500)= \$1027.50 + \$3780= \$4807.50
	\]
from the previous two tax brackets. 



\newpage



% Problem 2
\problem{10} Suppose that you place \$8,520 into a bank account that earns 1.2\% annual interest, compounded monthly. If you save this money in this bank account for 4~years, how much money do you have at the end of the 4~years? How much interest have you earned? \pspace

\sol We know that if $P$ dollars accumulating interest at an annual interest rate of $r$, compounded $k$ times year, then the amount after $t$ years is $P \left(1 + \frac{r}{k} \right)^{kt}$. The initial amount of money is $P= \$8520$. The annual interest rate is $r= 0.012$, compounded each month, i.e. $k= 12$ times per year. Then after $t= 4$ years, we have\dots
	\[
	\$8520 \left(1 + \dfrac{0.012}{12} \right)^{12 \cdot 4}= \$8520 (1.001)^{48}= \$8520(1.0491455)= \$8938.72
	\]
Therefore, the amount in the account after 4~years is \$8938.72. Because the account began with \$8,520, any additional money must have come from interest. But then the total interest earned is $\$8938.72 - \$8520= \$418.72$. 


\end{document}