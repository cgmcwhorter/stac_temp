\documentclass[11pt,letterpaper]{article}
\usepackage[lmargin=1in,rmargin=1in,tmargin=1in,bmargin=1in]{geometry}
\usepackage{../style/homework}
\usepackage{../style/commands}
\setbool{quotetype}{false} % True: Side; False: Under
\setbool{hideans}{false} % Student: True; Instructor: False

% -------------------
% Content
% -------------------
\begin{document}

\homework{10: Due 10/26}{Getting arrested, that makes people look guilty, even the innocent ones and innocent people get arrested every day.}{James (Jimmy) McGill, Better Call Saul}

% Problem 1
\problem{10} Jeffrey is writing a term paper. Currently, he has only written 8~pages. He returns from a writing break and then goes back to the paper. After an additional 5~hours of writing, he has written 20~pages. 
	\begin{enumerate}[(a)]
	\item Assuming Jeffrey writes at a constant rate, find the linear function representing the number of pages that he has written after $t$ hours of writing. 
	\item How long after this break will it take him in total to write this 50~page term paper?
	\end{enumerate} \pspace

\sol 
\begin{enumerate}[(a)]
\item Assuming that Jeffrey writes at a constant rate, we know that the number of pages he has written, $P$, after $t$~hours is a linear function. But then we know that $P(t)= mt + b$ for some $m, b$. We know that after 5~hours, he has written 20~pages. But then we have\dots
	\[
	m= \dfrac{\Delta P}{\Delta t}= \dfrac{20}{5}= 4
	\]
But then $P(t)= 4t + b$. At the start, he has written 8~pages. Therefore, $(0, 8)$ is a point on the line $P(t)$. So we know\dots
	\[
	\begin{aligned}
	P(t)&= 4t + b \\[0.3cm]
	P(0)&= 4(0) + b \\[0.3cm]
	8&= 0 + b \\[0.3cm]
	b&= 8
	\end{aligned}
	\]
Therefore, $P(t)= 4t + 8$. \pspace

\item If he needs to write 50~pages, then we have $P(t)= 50$. But then we have\dots
	\[
	\begin{aligned}
	P(t)&= 50 \\[0.3cm]
	4t + 8&= 50 \\[0.3cm]
	4t&= 42 \\[0.3cm]
	t&= 10.5
	\end{aligned}
	\]
Therefore, it will take Jeffrey 10.5~hours after his break to write the term paper. 
\end{enumerate}



\newpage



% Problem 2
\problem{10} Richard is a tailor. He uses an automated sewing machine can create custom labels on jackets. Every 4~hours, it is able to stitch 26~jackets. Richard sets the machine going during the night and when he comes in the next morning it has stitched 80~jackets.
	\begin{enumerate}[(a)]
	\item Assuming the machine works at a constant rate, find the number of jackets, $J$, that the machine has stitched $t$~hours from now.
	\item How many total jackets has the machine stitched 8~hours after opening?
	\end{enumerate} \pspace

\sol 
\begin{enumerate}[(a)]
\item Because the machine works at a constant rate, the number of jackets, $J$, it has produced after $t$~hours is linear, i.e. $J(t)$ can be written in the form $J(t)= mt + b$ for some $m, b$. Because the machine produces 26~additional jackets every 4~hours, we know that the machine produces at a rate of $m= \frac{26}{4}= \frac{13}{2}$ jackets per hour. But then $J(t)= \frac{13}{2}\,t + b$. Using the fact that it currently has 80~jackets produced in the morning, i.e. $t= 0$, we know that $80= J(0)= \frac{13}{2} \cdot 0 + b$ so that $b= 80$. But then $J(t)= \frac{13}{2}\,t + 80$. \pspace

\item We have\dots
	\[
	J(8)= \dfrac{13}{2} \cdot 8 + 80= 52 + 80= 132
	\]
Therefore, 8~hours after opening, the machine will have stitched a total of 132~jackets. 
\end{enumerate}


\end{document}