\documentclass[11pt,letterpaper]{article}
\usepackage[lmargin=1in,rmargin=1in,tmargin=1in,bmargin=1in]{geometry}
\usepackage{../style/homework}
\usepackage{../style/commands}
\setbool{quotetype}{true} % True: Side; False: Under
\setbool{hideans}{true} % Student: True; Instructor: False

% -------------------
% Content
% -------------------
\begin{document}

\homework{19: Due 11/30}{In the first place, god made idiots. That was for practice. Then he made school boards.}{Mark Twain}

% Problem 1
\problem{10} Suppose that we have distribution $N(1010, 80)$. Suppose we find the mean of a simple random sample of size 20, $\overline{X}$, from this distribution. Compute the following:
	\begin{enumerate}[(a)]
	\item $P(\overline{X}= 1010)$
	\item $P(\overline{X} \leq 1000)$
	\item $P(\overline{X} \geq 1000)$
	\item $P(\overline{X} \leq 1100)$
	\item $P(1000 \leq \overline{X} \leq 1100)$
	\end{enumerate}



\newpage



% Problem 2
\problem{10} A group of speedrunning gamers are in a team competition. The competition requires teams of five. Each runner completes the game as quickly as possible. The times of the five runners are then averaged and the team with lowest average completion time wins. Suppose that the times for an \textit{individual} to complete the game are normally distributed with mean 47~minutes and standard deviation 0.50~minutes. What average time would a team of five need to be in the top 1\% of teams?



\newpage


% Problem 3
\problem{10} Recently the average SAT scores fell from an average of 1060 in 2021 to an average value of 1050 in 2022. While the drop in score seems small, this is highly unlikely to have occurred by `random chance' rather than there being some `outside' explanatory variable(s). In 2021, the SAT had an average score of 1060 with standard deviation 217. Assuming this distribution stayed constant, what is the probability that the 1.7~million SAT takes in 2022 would have an average score of 1050? How does this explain how this could not happen by random chance? 


\end{document}