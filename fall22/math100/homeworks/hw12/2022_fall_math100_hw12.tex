\documentclass[11pt,letterpaper]{article}
\usepackage[lmargin=1in,rmargin=1in,tmargin=1in,bmargin=1in]{geometry}
\usepackage{../style/homework}
\usepackage{../style/commands}
\setbool{quotetype}{false} % True: Side; False: Under
\setbool{hideans}{true} % Student: True; Instructor: False

% -------------------
% Content
% -------------------
\begin{document}

\homework{12: Due 10/31}{I'm proud to pay taxes in the United States; the only thing is, I could be just as proud for half the money.}{Arthur Godfrey}

% Problem 1
\problem{10} Justin is filing his 2022 taxes. Over the past year, he made \$98,553. He is a single filer taking the standard deduction of \$12,950. The tax brackets for the 2022~tax year are found below. Compute Justin's federal income tax. How much will he make after taxes?  \par
	\begin{table}[!ht]
	\centering
	\begin{tabular}{|l|l|} \hline
	Tax Rate & Taxable Income \\ \hline
	10\% & \$0 -- \$10,275 \\ \hline
	12\% & \$10276 -- \$41,775 \\ \hline
	22\% & \$41,776 -- \$89,075 \\ \hline
	24\% & \$89,076 -- \$170,050 \\ \hline
	32\% & \$170,051 -- \$215,950 \\ \hline
	35\% & \$215,951 -- \$539,900 \\ \hline
	37\% & \$539,901 or more \\ \hline
	\end{tabular}
	\end{table}



\newpage



% Problem 2
\problem{10} If a good that cost \$15 last year now costs \$20, what was the inflation rate from last year to this year? Suppose that this was the true inflation rate. If the CPI last year was \$287.33, what is the CPI this year?


\end{document}