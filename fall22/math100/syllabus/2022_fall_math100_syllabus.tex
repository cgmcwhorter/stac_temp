\documentclass[11pt,letterpaper]{article}
\usepackage[lmargin=1in,rmargin=1in,bmargin=1in,tmargin=1in]{geometry}
\usepackage{style}

% Course Subject Abbreviation
\newcommand{\coursenumber}{MAT 100}
% Course Title
\newcommand{\coursetitle}{Fundamentals of Algebra}
% Section
\newcommand{\coursesection}{}
% Term
\newcommand{\semester}{Fall}
% Class Dates
\newcommand{\classdates}{September 6 -- December 16}
% Class Time
\newcommand{\classtimes}{MW 1:00~pm -- 2:25~pm}
% Classroom
\newcommand{\classroom}{MAGR G 17}


% Instructor
\newcommand{\instructor}{Dr. Caleb McWhorter}
% Instructor Office
\newcommand{\office}{Maguire 129}
% Instructor Number
\newcommand{\phone}{845.398.4077}
% Instructor Email
\newcommand{\email}{cmcwhort@stac.edu}
% Instructor Website
\newcommand{\website}{http://coffeeintotheorems.com}
% Instructor Office Hours
\newcommand{\officehours}{See \textit{`Mathematics Help'}}


% -------------------
% Content
% -------------------
\begin{document}

% Title
\mytitle



% Table of Contents
\largeheader{0cm}{Table of Contents}

\begin{minipage}[t]{0.45\textwidth}
{\bfseries\color{stacred} Course Information} \dotfill \pageref{course_info} \par
\hspace{0.3cm} Instructor Information \dotfill \pageref{instr_info} \par
\hspace{0.3cm} Class Information \dotfill \pageref{class_info} \par
\hspace{0.3cm} Course Description \dotfill \pageref{course_desc} \par
\hspace{0.3cm} Course Objectives \dotfill \pageref{course_obj} \par
\hspace{0.3cm} Course Materials \dotfill \pageref{course_mat} \par
{\bfseries\color{stacred} Course Policies} \dotfill \pageref{course_polc} \par
\hspace{0.3cm} Grading Components \dotfill \pageref{grade_comp} \par
\hspace{0.3cm} Grading Scale \dotfill \pageref{grade_scale} \par
\hspace{0.3cm} Course Format \dotfill \pageref{course_form} \par
\hspace{0.3cm} Attendance \& Participation \dotfill \pageref{attend} \par
\hspace{0.3cm} Quizzes \dotfill \pageref{quiz} \par
\hspace{0.3cm} Exams \dotfill \pageref{exams} \par
\hspace{0.3cm} Project \dotfill \pageref{project} \par
\hspace{0.3cm} Homework \dotfill \pageref{hw} \par
\hspace{0.3cm} Mathematics Help \dotfill \pageref{help} \par
\hspace{0.3cm} Respect Policy \dotfill \pageref{respect} \par
\hspace{0.3cm} COVID Discussion Policy \dotfill \pageref{covid_disc} \par
\hspace{0.3cm} Email Policy \dotfill \pageref{email_policy} \par
\hspace{0.3cm} Electronic Device Policy \dotfill \pageref{electronic} \par
%
\end{minipage}\hfill\begin{minipage}[t]{0.45\textwidth} \par
\hspace{0.3cm} Mental Health \& Counseling Services \dotfill \pageref{mental_health} \par
\hspace{0.3cm} Faith/Tradition Observances Policy \dotfill \pageref{faith} \par
\hspace{0.3cm} Use of Student Work \dotfill \pageref{std_work} \par
\hspace{0.3cm} Course Materials Policy \dotfill \pageref{copyright} \par
\hspace{0.3cm} Syllabus Policy \dotfill \pageref{syllabus} \par
\hspace{0.3cm} Tips for Success \dotfill \pageref{tips} \par
\hspace{0.3cm} Important Dates \dotfill \pageref{imp_dates} \par
{\bfseries\color{stacred} College Policies} \dotfill \pageref{college_polc} \par
\hspace{0.3cm} Academic Integrity \dotfill \pageref{college_acadint} \par
\hspace{0.3cm} Academic Dishonesty \dotfill \pageref{college_acaddis} \par
\hspace{0.3cm} Electronic Use Policy \dotfill \pageref{college_elecuse} \par
\hspace{0.3cm} Academic Accommodations for Students \par
\hspace{0.6cm} with Disabilities Statement \dotfill \pageref{college_acadacc} \par
\hspace{0.3cm} Sexual Misconduct Policy \dotfill \pageref{college_sexmisconduct} \par
\hspace{0.3cm} COVID-19 Policies/Procedures \dotfill \pageref{college_healthsafety} \par
\hspace{0.3cm} Diversity and Inclusivity Statement \dotfill \pageref{college_inclusive} \par
{\bfseries\color{stacred} Course Schedule} \dotfill \pageref{schd} \par
\hfill {\bfseries\color{stacred} Total Pages:} \pageref*{LastPage}
\end{minipage}
\sectionbreak





% -----
% Course Information
% -----

% Course Information
\largeheader{0.3cm}{Course Information\label{course_info}}



% Instructor Information
\mysection{0.27}{Instructor Information\label{instr_info}}
\textit{Name:} \instructor \par
\textit{Office:} \office \par
\textit{Phone:} \phone \par
\textit{Email:} \href{mailto:\email}{\email} \par
\textit{Office Hours:} \officehours 
\sectionbreak



% Class Information
\mysection{0.27}{Class Information\label{class_info}}
\textit{Dates:} \classdates \par
\textit{Time:} \classtimes \par
\textit{Classroom:} \classroom \par
\textit{Course Webpage:} \href{\website}{\website}
\sectionbreak



% Course Description 
\mysection{0.27}{Course Description\label{course_desc}}
Mathematical skills for students with fewer than two years of high school mathematics preparation or who are otherwise deficient in mathematics. A basic algebra course to prepare students for MATH~101.
\sectionbreak



% Course Objectives
\mysection{0.27}{Course Objectives\label{course_obj}}
After this course, among other things, students should be able to\dots
	\begin{itemize} \itemsep=0.3ex
	\item
	\item
	\item
	\item
	\item
	\item
	\item
	\item
	\item
	\item
	\item
	\item
	\item
	\item
	\item
	\item
	\item
	\item
% 18
%	\item Perform algebraic operations with integers, rationals, and reals. 
%	\item Compute with ratios, proportions, percentages, and rates. 
%	\item Understand the definition of a function.
%	\item Evaluate functions and perform basic function arithmetic. 
%	\item Understand the definition of a sequence. 
%	\item Understand arithmetic series and geometric series as well as their sums.
%	\item Graph linear functions. 
%	\item Identify as well as interpret slopes and intercepts of linear functions.
%	\item Solve linear equations. 
%	\item Apply linear functions to `real world' problems.  
%	\item Understand what a line of best fit is. 
%	\item Evaluate polynomials and perform algebraic operations with polynomials.
%	\item Factor polynomials, especially quadratics, using a variety of methods.
%	\item Solve quadratic equations.
%	\item Perform algebraic operations with rational functions. 
%	\item Simply rational functions. 
%	\item Apply polynomials and rational functions in `real world' problems.
%	\item Solve systems of linear equations. 
	\end{itemize}
Furthermore, students should\dots
	\begin{itemize} \itemsep=0.3ex
	\item  Improve their ability to engage in mathematical thinking, reasoning, communication, and problem solving.
	\item Develop a matured perspective on how to approach mathematical problems and concepts.
	\item Be able to state ways Mathematics applies to real world problems.
	\item Learn to properly utilize technology to explore, expand upon, or answer mathematical questions.
	\item Refine their cognitive skills by improving their ability to learn independently, approach problems imaginatively, solve problems methodically, and communicate solutions intelligibly.
	\end{itemize}
\sectionbreak



% Course Materials
\mysection{0.22}{Course Materials\label{course_mat}}
{\itshape\bfseries\color{stacred}Textbook.} The primary reference for course topics will be lecture notes and related materials provided by the instructor. However, students wishing to have a consistent standard reference are suggested to use the free open source textbook \textit{College Algebra} by Jay Abramson et al. found at \url{https://openstax.org/details/books/college-algebra} and the free open source textbook \textit{Applied Finite Mathematics} by Rupinder Sekhon found at \url{https://cnx.org/contents/8c-1jjEY@5.1:DjaX961v@2/Linear-Equations}. \pspace

{\itshape\bfseries\color{stacred}Calculators.} The course will make use of the computational engine Mathematica via the WolframAlpha website: \url{https://www.wolframalpha.com}. Although WolframAlpha does have a paid account option for additional resources, the course will not make use of these features and students {\itshape will not} be required to setup an account or make any kind of payment. Furthermore, no calculators will be required for this course. Unless otherwise instructed, students may make use of a calculator (physical or digital) and any calculator (particularly graphing calculators) will suffice. 
\sectionbreak





% -----
% Course Policies
% -----

% Course Policies
\largeheader{0.3cm}{Course Policies\label{course_polc}}



% Grading Components
\mysection{0.27}{Grading Components\label{grade_comp}}
Course grades are determined by the following components: \par
	\begin{table}[!ht]
        \begin{tabular}{clr}
	& Participation & 10\% \\
	& Quizzes & 10\% \\
	& Exam 1 & 15\% \\
	& Exam 2 & 15\% \\
	& Project & 20\% \\
	& Homework & 30\% \\
        \end{tabular} 
        \end{table}
\sectionbreak



% Grading Scale
\mysection{0.18}{Grading Scale\label{grade_scale}}
The grade scale is the standard St. Thomas Aquinas College grading scale and is as follows: \par
        \begin{table}[!ht]
        \centering
        \begin{tabular}{|l||c|l||c|} \hline
        A & 95 -- 100 & C+ & 77 -- 79 \\ \hline
        A-- & 90 -- 94 & C & 73 -- 76 \\ \hline
        B+ & 87 -- 89 & C-- & 70 -- 72 \\ \hline
        B & 83 -- 86 & D & 65 -- 69 \\ \hline
        B-- & 80 -- 82 & F & 0 -- 64 \\ \hline
        \end{tabular}
        \end{table}
\sectionbreak



% Course Format
\mysection{0.19}{Course Format\label{course_form}}
The course consists of two lectures per week. Each class will begin with a quiz followed by lecture. Lectures will typically include some time for individual and group problem solving. While the weekly lectures will typically cover all course materials, students may be required to do outside reading. Each student will be required to have a weekly, hour-long group meeting. Students are expected to typically spend approximately 3~hours per credit outside of class on course materials. However, some weeks this may be more or less. The course will have two exams, each covering approximately half of the semester's material. There will also be project for the course, having both a group and individual component, that will be presented in the final week of classes. 
\sectionbreak



% Attendance and Participation
\mysection{0.36}{Attendance and Participation\label{attend}}
{\itshape\bfseries\color{stacred}Attendance.} It is essential to your success in this course that you attend each lecture and participate in class discussions. It is also a federal requirement that students who do not attend or stop attending a class be reported at the time of determination by the faculty that the student never attended or stopped attending the class. Therefore, you are expected to attend each lecture and to show up on time. Address any absence(s), anticipated or unanticipated, with the instructor as soon as possible. Should you need to miss a class for any reason, you are to contact the instructor as soon as possible. Certain absences from lecture(s) may be excused, depending on the reason for the absence. Determinations are made on a case-by-case basis at the discretion of the instructor. The student should discuss the issue with the instructor as soon as possible; however, to excuse an absence, the reason(s) for missing lecture(s) must be documentable and presented, if requested. \pspace

If you miss a lecture, you are responsible for any material covered, any work assigned, any course changes made, etc. during the class. Do not assume or expect the instructor to provide you with anything, particularly lecture notes, from the class(es) missed. {\itshape Four or more unexcused absences from lectures could result in receiving an `F' in the course.} Furthermore, excessive lateness will also count as an absence. If you are dismissed from lecture due to problems during the lecture, e.g. disruptive behavior or unauthorized cell phone use, then this dismissal will be recorded as an absence for the lecture. If you cannot attend a class due to a mandated quarantine, inform your instructor immediately so that arrangements can be made. In this case, the student may be required to participate in lectures virtually and submit assignments online. \pspace

{\itshape\bfseries\color{stacred}Participation.} 
Students are expected to participate in the course---both inside and outside the classroom. Inside the classroom, this means attending class, paying attention, taking notes, asking and answering questions when appropriate, etc. However, course participation does not begin and end at the classroom door. Students are expected to review course material and complete course assignments. Typically, students can expect to spend approximately 3~hours per credit outside of class working for the course---although some weeks this could be more or less. \pspace

Students will be broken up into groups throughout the semester. Part of the participation grade will be meeting weekly with their group for at least one hour. The purpose of these meetings will be to discuss course material, work on appropriate course assignments, and generally support each other through the course. Groups will determine on their own when and where to meet as well as what they will do during the meeting. To present proof that the meeting occurred, students will take a group photo and submit it to the instructor electronically. 
\sectionbreak



% Quizzes
\mysection{0.18}{Quizzes\label{quiz}}
There will be a quiz \textit{every} class. Quizzes are meant to be short and simple. These quizzes serve more as a method of gauging whether you are keeping up with the material. It is important that if you are late that you obtain a copy of the quiz immediately. Quiz solutions will often be discussed following the quiz. Because quiz solutions will often be discussed in class, no make-up quizzes will be given, even in the case of an emergency. However, the lowest four scores on quizzes as well as any missed quizzes due to excused absences will be dropped. Unless otherwise instructed, there are no calculators, computational devices, notes, or outside assistance of any kind allowed on quizzes. 
\sectionbreak



% Exams
\mysection{0.18}{Exams\label{exams}}
There will be two exams in this course. Exam~1 is scheduled for October~20th and Exam~2 is scheduled for December~13th. However, these exam dates are subject to change. Exam~1 focuses on material from the first half of the course while Exam~2 focuses on material from the second half of the course. However, any course topics may appear on any exam. Students should be present, seated, and prepared for a scheduled exam before the exam begins. Students who are late should not expect extra exam time. Furthermore, students who miss an exam should not expect to receive a make-up exam. There will be no make-up exams except under extraordinary circumstances, e.g. in the case of an emergency. However, determinations for make-up exams or other substitutions are made at the discretion of the instructor on a case-by-case basis. Unless otherwise instructed, no devices or materials other than those provided by the instructor are allowed on any exam. 
\sectionbreak



% Project
\mysection{0.18}{Project\label{project}}
There will be a project in this course. This project will involve both an individual and group component. Groups will consist of at least two students but no more than four students. As a group, students will choose a mathematical topic to research. The topic should be something not standardly seen in a mathematics curriculum. While students will be provided a list of suggested topic areas for this project, students may choose any appropriate topic for their project. The project groups will prepare a presentation on their topic that will be given the last week of class. These presentations are anticipated to occur on December~15th, so students should make no plans to leave campus before then or otherwise have conflicts on this date. Exact presentation grading scales and guidelines will be distributed to students the first week in October. \pspace

Additionally, each student will have to write a paper on their group topic. Unlike their presentations, each group member will write their paper individually and independently. Each student's paper topic must be the on the same broad mathematical topic as their group presentation topic. However, the paper need not be and should not be substantially identical to the group presentation. Papers are only required to be between one and two pages long and will be due at the beginning of presentations on December~15th. These papers should allow students to show off their own individual perspective, creativity, and talents. Therefore, students may substitute an approved creative project in place of a paper. This substitution, along with a suggested grading scale, must be approved by the instructor and be of equivalent workload to the paper. Exact paper grading scales and guidelines will be distributed to students the first week in October. 
\sectionbreak
 


% Homework
\mysection{0.18}{Homework\label{hw}}
The only way to learn Mathematics is to do Mathematics! Therefore, there will be weekly homework assignments. It is essential for students to complete all of the assignments for the course. Working on homework is the best way of engaging with course concepts and gauging one's mastery of the material. Moreover, homework is an essential portion of the course grade. Assignments should be started as soon as possible. Do not delay working through homework; it is easier to keep up than it is to catch up. Students may request extensions on homework assignments. Requests for extensions should be submitted to the instructor in a timely fashion---do not delay! However, do not simply assume that you will be able to receive extra time on an assignment and plan your schedule carefully. Except in exceptional circumstances, homework extensions on topics included in an exam will not be granted beyond that exam date. Any extensions, due dates, and grade penalties for late assignments will be determined by the instructor on a student-by-student basis. \pspace

You are encouraged to work with others on homeworks. Mathematics is a social activity! Working on homework assignments is one of the purposes of the weekly group meetings. The purpose of working together on assignments is to engage with course topics, see different perspectives, ask questions, and have others look over your work. However, do not simply use others to do your assignments. You should also not allow other students to use you to complete their assignments. Of course, using online solutions is a violation of the St. Thomas Aquinas College academic integrity policies. If you are unsure of whether a particular resource is appropriate to use on an assignment, consult with your instructor first. Any issues which cannot be properly resolved amongst group members should be brought to the attention of the instructor. 
\sectionbreak



% Mathematics Help
\mysection{0.24}{Mathematics Help\label{help}}
Be proactive about your success in the course! If you need help, there are many resources available to help you. Your first primary contact for help is the instructor. If you are struggling, attend office hours or send an email. The instructors office hours for this semester can be found below:
	\begin{table}[!ht]
	\centering
	\begin{tabular}{l || l}
	Mon. & 10:30~am -- 11:30~am \\
	Tues. & 2:30~pm -- 3:30~pm \\
	Wed. & 3:00~pm -- 4:00~pm \\
	Thurs. & 11:30~am -- 12:30~pm \\
	Fri. & 11:30~am -- 2:00~pm
	\end{tabular}
	\end{table}
Do not wait to bring issues, course related or otherwise, to the attention of the instructor. If you cannot attend office hours, send an email to the instructor to try to make other arrangements. There are also a number of resources available to you at St. Thomas Aquinas College: Center for Student Success, Academic Recovery Program, Writing Center, etc. Students looking for extra mathematics help should consult with the Academic Services Office in Spellman~106, via email at \href{mailto:AcademicServices@stac.edu}{academicservices@stac.edu}, or on the web at \href{https://www.stac.edu/academics/academic-services}{https://www.stac.edu/academics/academic-services}. The Center for Student Success website is \url{https://www.stac.edu/academics/academic-services/center-student-success} and can be found at Spellman~111 or contacted at 845.398.4090.
\sectionbreak



% Respect Policy
\mysection{0.19}{Respect Policy\label{respect}}
Learning requires a healthy academic environment. A key component to this is respecting everyone's time---especially giving everyone time to fail, ask questions, and learn. Therefore, everyone should abide by the following respect policies: 





\newpage





The instructor will respect student's time:
	\begin{itemize}
	\item They will come prepared to help you understand the course material and prepare students for quizzes/exams. 
	\item They will listen to student feedback on how to best help them succeed. 
	\item They will return assignments, respond to emails, and give feedback in a timely fashion. 
	\item They will be patient during the student learning process and will treat all students fairly. 
	\end{itemize} \pspace

Students will respect the instructor's time:
	\begin{itemize}
	\item They will be on time to class. Moreover, they will come prepared and pay attention during class. 
	\item They will ask for help and communicate with the instructor in a timely fashion. 
	\item They will keep track of assignments---completing them on time and to the best of their ability.  
	\item They will read and follow course policies. 
	\end{itemize} \pspace

Students will respect each other's time:
	\begin{itemize}
	\item They will not be disruptive in class. If you need to call or text someone, take it outside of the classroom. 
	\item They will work with each other to find solutions and understand course material. However, they will not simply solve problems. 
	\item They will allow each other to make mistakes, ask questions, and participate in the learning process. 
	\item They will use respectful language when speaking to or about one another. 
	\end{itemize}
\sectionbreak



% COVID Discussion Policy
\mysection{0.31}{COVID Discussion Policy\label{covid_disc}}
At the time of writing, there have been over 38.8~million cases of COVID-19 in the United States with over 638,000 deaths; moreover, there have been over 216~million cases with 4.5~million deaths worldwide. It is an understatement to say that these are trying times not just for students, including their friends and family, but for our broader community. While many of us use humor to cope with difficult situations, we are often able to do so without great offense because we can choose our words and our actions to fit an audience with which we are familiar---be it friends or family. \pspace

However, this luxury may not be available to us in the classroom. You will likely not know all your classmates and their circumstances. It is not unlikely that at least one of your classmates in at least one of your classes has lost an acquaintance, friend, or family member to COVID-19. Worse yet, because of social distancing, they may not have been able to properly mourn them. Even if a classmate has not lost someone, they or someone in their life may be experiencing financial hardships or other crises due to COVID-19. \pspace

All students are expected to respect and protect each other by abiding by the college's vaccination and mask policies. But protecting the health of others is the minimum that one can do during a pandemic. We need to go beyond basic physical health and support our community's mental health. By enrolling in this course, you agree to refrain from making jokes or other trivialization of the COVID-19 pandemic while participating in the course, both online and in-person.
\sectionbreak



% Email Policy
\mysection{0.18}{Email Policy\label{email_policy}}
All email communication in this course should be done using your @stac.edu email account. Similarly, any digital course access and file submissions should be made using your @stac.edu email account. Abiding by federal guidelines, emails coming from a non-STAC email may not receive a response. Emails should be properly written: contain appropriate subject line, possess an opening and closing address, be understandable and contain appropriate language, be grammatically correct, have appropriate font style and size, etc. Emails which do not follow these guidelines may not receive a response.
\sectionbreak



% Electronic Device Policy
\mysection{0.31}{Electronic Device Policy\label{electronic}}
Students are expected to complete the course without the use of calculators or other computational devices on assignments, quizzes, exams, etc., unless otherwise instructed. Any unauthorized use of such devices are considered a violation of the academic integrity policies. During the course, \href{http://www.wolframalpha.com/}{http://www.wolframalpha.com/}, \href{https://www.symbolab.com/}{https://www.symbolab.com/}, and Mathematica will be used to demonstrate concepts give students an opportunity to be able to check work. However, these should only be used as instructed, and never during a quiz or exam. All electronic devices should be turned off and put away during class unless otherwise instructed or given specific permission. Use of such devices can result in dismissal from class.
\sectionbreak



% Mental Health & Counseling Services
\mysection{0.49}{Mental Health and Counseling Services\label{mental_health}}
If at any point during the semester, you feel overwhelmed with your class work, feel thoughts of depression/suicide, experience sexual assault/rape, experience problems with substance abuse or relationship abuse, or have any other struggles with physical/mental health, \underline{\bfseries\itshape please seek help}! The Counseling \& Psychological Services (CAPS) at St. Thomas Aquinas College is a resource offering assistance with any issue you might have. There is \underline{\bfseries\itshape never} any shame in seeking help. If you or someone you know is struggling with any of these issues, {\itshape speak out}! The CAPS website can be found at \href{https://www.stac.edu/student-life/counseling-psychological-services}{https://www.stac.edu/student-life/counseling-psychological-services}. CAPS is located in the upper level of the Romano Student Alumni Center and can be contacted at 845.398.4065. If you or someone you know is having issues with gender or sexual identity issues, CAPS is also there to create a safe space for those with marginalized genders and sexualities or those who might be struggling with these issues. Know that my office is a safe space and should you prefer any gender specific pronoun/name, please be sure to make me aware! Students may also make use of the College Health \& Wellness Services located in the McNelis Commons residence life complex, Apartment~2B which can also be contacted at \href{mailto:stachealth@stac.edu}{stachealth@stac.edu} or 845.398.4242, as well as the Campus Ministry and Volunteer Services, directed by Daniel Cummings, located in the Romano Student Alumni Center and can be contacted at \href{mailto:dcumming@stac.edu}{dcumming@stac.edu} or 845.398.4092.
\sectionbreak



% Faith/Tradition Observances Policy
\mysection{0.44}{Faith/Tradition Observances Policy\label{faith}}
The instructor recognizes the diversity of faiths and traditions represented in the campus community. Students should have the right to observe religious holy days according to their faith and traditions. Accordingly, students may notify their instructor, no later than the end of the second week of classes, of any classes that they will be missing due to religious or traditional observances. Students following this guideline will be excused from these classes. Under this policy, students should have an opportunity to make up any examination, study, or work missed due to these observances or have an equitable and appropriate substitution made. All policy and procedural decisions are made at the discretion of the instructor on a student-by-student basis. 
\sectionbreak



% Use of Student Work
\mysection{0.27}{Use of Student Work\label{std_work}}
In compliance with the federal Family Educational Rights and Privacy Act (FERPA), registration in this class is understood as permission for assignments prepared for this class to be used anonymously in the future for educational purposes.
\sectionbreak



% Course Materials Policy
\mysection{0.30}{Course Materials Policy\label{copyright}}
All course materials (defined to include, but not limited to, course handouts, video/audio lectures, assignments, quizzes, exams, etc.) are the intellectual property of the instructor or St. Thomas Aquinas College, unless the copyright is already explicitly held by some other individual, group, or other entity. Therefore, course materials are protected by United States copyright law, see Title~17~USC. Students in this course are permitted to download some course materials for personal use. \pspace

However, students are not permitted to (in print, digitally, or otherwise) share, distribute, sell, or publish course materials, either in part or in whole, without the instructors explicit written and signed permission along with a personal usage code. Unauthorized reproduction or distribution of course materials is a violation of intellectual property law, and is a violation of the student code of conduct. The instructor, or agent acting on behalf of the instructor with written and signed permission, also reserves the right to delete or disable any link to any course materials. In enrolling in the course, the student agrees to abide by this course materials policy in perpetuity.
\sectionbreak



% Syllabus Policy
\mysection{0.20}{Syllabus Policy\label{syllabus}}
The instructor reserves the right to revise, including substantially revise, the course syllabus at any time---with or without notification. By enrolling in this course, students agree to all the policies found in the syllabus. Wherever applicable, students also agree to follow syllabus policies in perpetuity, e.g. students may not provide unauthorized assistance, materials, etc. to students enrolled in future versions of this course. 
\sectionbreak



% Tips for Success
\mysection{0.21}{Tips for Success\label{tips}}
\begin{itemize} \itemsep=0.3ex
\item Be proactive about your success in the course.
\item Do not procrastinate! Begin your assignments and studying early!
\item Attend every lecture.
\item Address issues immediately. Ask questions during class, recitation, office hours, etc. 
\item Form a study group! Working together will help you and others better understand the course material as you can work through different difficulties and offer each other clarifications on concepts.
\item Do problems! Reading through your notes is not enough. Seek out new problems and work through them carefully. When you are done, check your answer. If you are wrong, examine carefully what misunderstanding occurred and how to avoid it in the future. If you were correct, examine if there was a faster way, check to see if your solution `flowed' and was easy to read, and think over what concepts/computations were used and what `type' of problem was the exercise.
\end{itemize}
\sectionbreak



% Important Dates
\mysection{0.22}{Important Dates\label{imp_dates}}
\begin{itemize} \itemsep=0.3ex
\item 09/14: Academic Add/Drop Deadline
\item 10/22: Mid-semester
\item 11/11: Academic Withdrawal Deadline
\item 12/17: Last day of classes/exams
\end{itemize}
\sectionbreak





% -----
% College Policies
% -----

% College Policies
\largeheader{0.5cm}{College Policies\label{college_polc}}

% Academic Integrity
\mysection{0.25}{Academic Integrity\label{college_acadint}}

Academic integrity is a commitment to honesty, trust, fairness, respect, and responsibility within an academic community. An academic community of integrity advances the quest for truth and knowledge by requiring intellectual and personal honesty in learning, teaching, research, and service. Honesty begins with oneself and extends to others. Such a community also fosters a climate of mutual trust, encourages the free exchange of ideas, and enables all to reach their highest potential. \pspace

A college community of integrity upholds personal accountability and shared responsibility, and ensures fairness in all academic interactions of students, faculty, and administrators. While we recognize the participatory and collaborative nature of the learning process, faculty and students alike must show respect for the work of others by adhering to the clear standards, practices, and procedures contained in the policy described below. \pspace

Academic integrity is essential to St. Thomas Aquinas College’s mission to educate in an atmosphere of mutual understanding, concern, cooperation, and respect. All members of the College community are expected to possess and embrace academic integrity. \sectionbreak



% Academic Dishonesty
\mysection{0.28}{Academic Dishonesty\label{college_acaddis}}

Academic dishonesty is defined as any behavior that violates the principles outlined above. St. Thomas Aquinas College strictly prohibits academic dishonesty. Any violation of academic integrity policies that constitutes academic dishonesty will be subject to harsh penalties, ranging up to and including dismissal from the College.

For all Academic Integrity violations, faculty must file a Student Conduct Academic Dishonesty Report, which will be shared with the Dean of the appropriate School, the Provost, and the student. The student will also have to file a Student Academic Integrity Violation Report. Please view the full policy and the associated forms at \url{https://www.stac.edu/academics/academic-integrity-policy}. \sectionbreak



% Electronic Use Policy
\mysection{0.27}{Electronic Use Policy\label{college_elecuse}}

Faculty members at St. Thomas Aquinas College have the discretion to regulate the use of electronic devices in their classes, and students should not use such devices without the expressed permission of the professor. This policy covers cell phones, tablets, laptop computers, or any other device the use of which might constitute a distraction to the professor or to the other students in the class, as determined by the professor. Students with documented disabilities should discuss the use of laptops and/or other electronic devices with their professor at the beginning of the semester. \pspace

When a professor designates a time during which electronic devices may be used, they are only to be used at the discretion of the faculty member and in accordance with the mission of the college. Professors may develop specific and reasonable penalties to deal with violations of these general policies. For more extreme cases of classroom disruption, refer to the College's Disruptive Student Policy. \pspace

Please note that a browser lockdown system may be implemented in order to prevent cheating during assessments such as exams and quizzes. Faculty are expected to confirm that these systems will work with students’ laptops before requiring their use. \pspace


{\itshape Recording of Lectures:} Class meetings that include course content or identifiable student information are protected by the Family Education Rights and Privacy Act (FERPA), found at \url{https://www2.ed.gov/policy/gen/guid/fpco/ferpa/index.html}. At times throughout the semester, the faculty member may record their lecture. It is a best practice for faculty to notify participants that their session is going to be recorded. This recording \textit{\textbf{CANNOT}} be shared with anyone who is not enrolled in this specific course section. \pspace

Students cannot personally record class sessions and then share them outside of the course, although they can maintain them for personal use. \sectionbreak



% Academic Accommodations for Students with Disabilities Statement
\mysection{0.83}{Academic Accommodations for Students with Disabilities Statement\label{college_acadacc}}

St. Thomas Aquinas College values diverse types of learners and is committed to ensuring that each student is afforded equal access to participate in all learning experiences. If you have a learning difference or a disability---including a mental health, medical, or physical impairment---that would hinder your access to learning in this class, please contact Disability Services. They will confidentially explain the accommodation request process and the type of documentation that may be needed to determine your eligibility for reasonable accommodations. To learn more about academic accommodations for students with disabilities, please contact Anne Schlinck, Director of Disability Services, at \href{mailto:aschlinc@stac.edu}{aschlinc@stac.edu} or call/text 845.398.4087. Disability Services is located in Room~L102 in the lower level of Spellman Hall. \pspace

If you have already been granted academic accommodations at St. Thomas Aquinas College, you have the right to receive the academic accommodations that are listed on your Letter of Accommodation. Please, understand that it is your responsibility as a student registered with Disability Services to provide your Letter of Accommodation to your instructor if you wish to use your accommodations in this course. If you will need to use your testing accommodations, please be sure to review the Disability Services Testing Accommodation Policies---Academic Year 2022--2023 found at \href{https://docs.google.com/document/d/1V5iUtgypiS8kClqhSLPde7AOSZPoLu6CsIDcpiEic2w/edit}{Disability Services Testing Accommodation Policies}. \sectionbreak



% Gender- or Sex-Based Misconduct Policy
\mysection{0.50}{Sexual Misconduct Policy\label{college_sexmisconduct}}

Students should be aware that faculty members are required to report certain information to the STAC’s Title~IX Coordinator. If you inform your instructor about, or that person witnesses, gender- or sex-based misconduct, which includes sexual harassment, sexual assault, intimate partner or domestic violence, stalking, or any gender- or sex-based discrimination, the faculty member will keep the information as private as possible, but must bring it to the attention of STAC’s Title~IX Coordinator. \pspace

Students should also be aware that disclosing such experiences in course assignments does NOT put the College on notice and will NOT begin the process of STAC providing assistance or response to those experiences. If you would like to talk to the Title~IX Coordinator directly, you can contact Mr. Norman Huling (\href{mailto:nhuling@stac.edu}{nhuling@stac.edu}, 845.398.4068). Additionally, you also may report incidents or complaints to Title IX Deputy Coordinators, Ms. Nicole Ryan (\href{mailto:nryan@stac.edu}{nryan@stac.edu}, 845.398.4163) or Dr. Benjamin Wagner (\href{mailto:bwagner@stac.edu}{bwagner@stac.edu}, 845.398.4212), or you can contact the Office of Campus Safety and Security (845.398.4080). You can find more information at \url{www.stac.edu/titleix}. \pspace

Please remember that instances of gender- and sex-based misconduct that occur in virtual/online environments are covered by STAC’s Title~IX, Student Code of Conduct, and Faculty/Employee Conduct policies. If you would like to report a private concern to a confidential counseling resource who is not required to initiate a Title~IX report, you may contact the following people on a confidential basis: \pspace

	\hfill\begin{minipage}[t]{0.49\textwidth}
	{\bfseries Ms. Anne Walsh RN, BSN} \par
	Director, Health and Wellness Services \par
	845.398.4242 \par
	\href{mailto:awalsh@stac.edu}{awalsh@stac.edu}
	\end{minipage}\begin{minipage}[t]{0.49\textwidth}
	{\bfseries Dr. Lou Muggeo} \par
	Director, Counseling \& \par
	Psychological Services \par
	845.398.4174 \par
	\href{mailto:lmuggeo@stac.edu}{lmuggeo@stac.edu}
        \end{minipage} \pspace
        \hfill\begin{minipage}[t]{0.49\textwidth} 
 	{\bfseries Dr. Alexa Gaydos} \par
	Licensed Clinical Psychologist, \par
	Counseling \& Psychological Services \par
	\href{mailto:agaydos@stac.edu}{agaydos@stac.edu}
	\end{minipage}\begin{minipage}[t]{0.49\textwidth} 
	{\bfseries Elysse Sellers, LCSW} \par
	Licensed Clinical Social Worker, \par
	Counseling \& Psychological Services \par
	\href{mailto:esellers@stac.edu}{esellers@stac.edu}
        \end{minipage} \pspace
 
The College also has an affiliation with the following organization, which will provide virtual office hours to STAC students weekly in addition to its other web-based programming: \pspace
 
        \hspace{0.3cm} {\bfseries Center for Safety and Change} \par
        \hspace{0.3cm} \url{http://centerforsafetyandchange.org} \par
        \hspace{0.3cm} 9 Johnsons Lane, New City, NY 10956 \par
        \hspace{0.3cm} 845.634.3344
\sectionbreak



% Classroom Health and Safety Protocols
\mysection{0.49}{COVID-19 Related Policies and Procedures\label{college_healthsafety}}

% Classroom Health and Safety Protocols
\noindent {\bfseries Classroom Health and Safety Protocols}

The health and safety of students, faculty, and staff on our campus is our top priority. In response to the ongoing COVID-19 pandemic, the STAC community will continue to work together to support compliance with recommended health and safety standards to optimize the learning experience while minimizing health risks.  

	\begin{enumerate}[1.]
	\item {\bfseries Follow quarantine and isolation guidelines.} If you feel ill, have recently tested positive for COVID-19, or have come into contact with someone who has tested positive for COVID-19, {\itshape do not} come to campus or leave your residence hall until you have been cleared to do so by STAC Health Services (\href{mailto:stachealth@stac.edu}{stachealth@stac.edu}). It is important that you always contact STAC Health Services in any of these circumstances and follow the quarantine and isolation instructions given. \href{https://timely.md/schools/index.html?school=stac&}{STAC Telehealth} can be used outside of normal business hours. Please also let your professor know if you cannot attend class. 

	\item {\bfseries Mask policy.} Mask-wearing is optional on STAC’s campus except for in the STAC Health and Wellness Center, where masks must be worn; however, you must wear a mask if you are directed to do so under our isolation/quarantine policy. Please note that individual professors may encourage students to wear masks in their classrooms, but masks cannot be required. In the event of an increase in COVID cases on campus, the College may decide to return to a mask requirement. 

	\item {\bfseries Minimize shared equipment.} Individuals should avoid sharing equipment where possible. However, if equipment does need to be shared, please wipe it down with provided disinfecting wipes in between users and maintain physical distancing as much as possible.

	\item {\bfseries Disinfect your classroom space.} Students and faculty are encouraged to disinfect areas within their workspaces by cleaning these at the beginning and end of each class. This includes desk tops, seats, and equipment used during class. Disinfectant supplies will be provided. 

	\item {\bfseries Practice good hand hygiene.} Individuals should wash their hands with soap and water for at least 20~seconds as often as possible or use personal hand sanitizers. Hand sanitizer stations are available throughout the campus.

	\item {\bfseries Respect each other.} Show concern for each other's health and safety, and remember that this is a stressful time for everyone.
	\end{enumerate} \pspace


% COVID-19, Other Illness, and Absences
\noindent {\bfseries COVID-19, Other Illness, and Absences} \par

As stated above, for the health and safety of the campus community, students who are ill {\itshape should not attend classes}. Students in the following situations must contact Health Services as soon as possible: 

	\begin{itemize}
	\item Those who have tested positive for COVID-19 or are exhibiting COVID-19 related symptoms. 
	\item Those who have been instructed to quarantine because of close contact with someone who has tested positive for COVID-19.
	\end{itemize}

If a student cannot attend classes for any of the above reasons, they should:

	\begin{enumerate}[1.]
	\item Communicate this change with their instructor(s) via email. Contact instructors as soon as possible, preferably within 24~hours.
	\item Keep up with coursework and participate in class activities as much as possible. Students are responsible for completing any work that they might miss due to illness, including assignments, quizzes, tests, and exams.
	\item Reach out to the instructor if illness will require late submission or modifications of assignments; work with the instructor to reschedule exams and other critical academic activities before they are due.
	\end{enumerate}
\sectionbreak



% Diversity and Inclusivity Statement
\mysection{0.44}{Diversity and Inclusivity Statement\label{college_inclusive}}

St. Thomas Aquinas College is committed to creating an inclusive environment. Our community actively seeks the inclusion and full participation of individuals from groups that have historically experienced discrimination and prejudice. We are committed to a climate of mutual respect and inclusion, one in which diversity is a source of pride rather than a source of division. We encourage all persons---students, faculty, and staff alike---to reflect on their own experiences to explore the ways in which others' experiences can and do differ; the goal is to use this reflection to learn about different values, cultures, and ways of thinking. Ultimately, a just and equitable society will be easier to realize if we do not exclude those who are different from us and instead practice empathy and inclusivity. \pspace 

To that end, if you experience or are aware of bias, mistreatment, or discrimination based on a person's (or your own) membership in a historically under-privileged or marginalized group, please contact one of the following individuals to share your concerns: \pspace

	\hfill\begin{minipage}[t]{0.33\textwidth}
	{\bfseries Samantha Bazile} \par
	Director of Admissions \& \par
	Chief Diversity Officer \par
	845.398.4104 \par
	\href{mailto:sbazile@stac.edu}{sbazile@stac.edu}
	 \end{minipage}\begin{minipage}[t]{0.33\textwidth}
	{\bfseries Cindy Garvey} \par
	Associate Director of \par
	Financial Aid \par
	845.398.4098 \par
	\href{mailto:cgarvey@stac.edu}{cgarvey@stac.edu}
	\end{minipage} 
	\hfill\begin{minipage}[t]{0.33\textwidth}
	{\bfseries Nicholas Migliorino} \par
	Director of Student Engagement \par
	845.398.4084 \par
	\href{mailto:nmiglior@stac.edu}{nmiglior@stac.edu}
	\end{minipage}\hfill \pspace

Faculty reserve the right to provide open and honest readings and discussions in their classes about personal and institutional biases and prejudices and other topics that may cause discomfort to some. \pspace

More detailed information about the College's expectations and policies related to these matters can be found in the Student Handbook, specifically in the Student Code of Conduct, Section D: Harassment and Abuse, the Anti-Harassment Policy, and Rules and Regulations for Maintenance of Order.





% -----
% Course Schedule
% -----
\largeheader{0.3cm}{Course Schedule\label{schd}}
The following is a \emph{tentative} schedule for the course and is subject to change. 
        \begin{table}[!ht]
        \centering
        \scalebox{1}{%
        \begin{tabular}{ll || ll}
        Date & Topic(s) & Date & Topic(s) \\ \hline 
	09/05 & Labor Day (No Classes) & 10/26 & Mathematics of Finance \\
	09/07 & Course Introduction & 10/31 & Probability \\
	09/12 & Operations, Sets, Functions & 11/02 & Probability \\
	09/14 & Units, Conversions, Magnitudes & 11/07 & Probability \\
	09/19 & (Weighted) Averages & 11/09 & Statistics \\
	09/21 & Length, Area, Volumes, Distances & 11/14 & Statistics \\
	09/26 & Percentages & 11/16 & Statistics \\
	09/28 & Rates \& Fermi Estimation & 11/21 & Statistics \\
	10/03 & Estimathon \& Review & 11/23 & Thanksgiving Break \\
	10/05 & Linear Functions & 11/28 & Statistics \\
	10/10 & Study Day (No Classes) & 11/30 & Statistics \\
	10/12 & Linear Functions & 12/05 & Statistics \\
	10/17 & Linear Functions & 12/07 & Statistics \\
	10/19 & Mathematics of Finance & 12/12 & Data Science Day \\
	10/24 & Mathematics of Finance & 12/14 & Papers \& Final Exam 
        \end{tabular}
        }
        \end{table}

\end{document}