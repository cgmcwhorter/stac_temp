\documentclass[11pt,letterpaper]{article}
\usepackage[lmargin=1in,rmargin=1in,bmargin=1in,tmargin=1in]{geometry}
\usepackage{style}

\pagenumbering{gobble}


% -------------------
% Content
% -------------------
\begin{document}

% TItle
\begin{center} 
\bfseries
\color{stacred}
\LARGE Syllabus Quick Facts \par\vspace{0.2\baselineskip}
\Large MATH 100: Fundamentals of Algebra --- Fall 2022 
\end{center} \pspace


% Course Information
\mysection{0.27}{Course Information}
\hspace{0.53cm} {\itshape Instructor Email}: \href{mailto:cmcwhort@stac.edu}{cmcwhort@stac.edu} \par
\hspace{0.53cm} {\itshape Course Webpage}: \href{https://coffeeintotheorems.com/courses/2022-2/fall/math-100/}{https://coffeeintotheorems.com/courses/2022-2/fall/math-100/} \par
\hspace{0.53cm} {\itshape Office Hours}: 	\par \vspace{-0.3cm}
	\begin{table}[!ht]
	\centering
	\begin{tabular}{l || l}
	Mon. & 11:30~am -- 12:30~pm \\
	Tues. & 4:00~pm -- 5:00~pm \\
	Wed. & 11:30~am -- 12:30~pm \\
	Thurs. & 4:00~pm -- 5:00~pm \\
	Fri. & 11:30~am -- 1:30~pm
	\end{tabular}
	\end{table}


% Grading Components
\mysection{0.27}{Grading Components\label{grade_comp}}
Course grades are determined by the following components: \par \vspace{-0.3cm}
	\begin{table}[!ht]
        \begin{tabular}{clr}
	& Participation & 5\% \\
        & Activities & 5\% \\
	& Project & 10\% \\
	& Quizzes & 10\% \\
	& Exams & 30\% \\
	& Homework & 40\% 
        \end{tabular} 
        \end{table}


% Attendance & Participation
\mysection{0.27}{Attendance \& Participation}
Attend each lecture and show up on time. Anticipated absences should be addressed with the instructor in advance of the absence. Address any absences---anticipated or otherwise---with the instructor. If you miss a lecture, you are responsible for any material covered, any work assigned, any course changes made, etc. during the class. Four or more unexcused absences from lectures could result in receiving a grade penalty per additional absence or an `F' in the course. Furthermore, excessive lateness will also count as an absence. In addition to classroom participation, you are expected to work on additional problems outside of class to practice, engage with the material, and prepare for exams. Additional problem sets will be distributed during lectures as well as on the course webpage. There will be a weekly check whether you are working on these problems. This is simply a check---these problems will not be graded. Students will simply be accessed on whether they have completed and written solutions to at least half these problems. Each such verification must occur by the Friday of that particular class week. Because one of the purposes of this grade component is encourage students to keep pace with the material, no make-ups will be given. Each such verification will be weighted equally in the participation grade. \pspace



% Activities
\mysection{0.27}{Activities}
As a small liberal arts college, we are committed to fostering a student-focused, inclusive, and engaging environment. This course should help to foster community building. You will be required to attend at least 10~different approved college or community events by the end of the semester. These events could include convocation, seminars or other college presentations, sporting events, college social nights, community volunteerism events, etc. If you are unsure whether a particular activity is appropriate, consult with your instructor before assuming that it will be counted. You must submit proof of attendance and participation for each of these events to the instructor. This could be a photo or video of event participation (including the student), a signed form/affidavit, etc. The submission should include the name of the event and the day/time of the event. Each event will be weighted equally. Students are especially encouraged to attend events with others---especially from the course! While the same verification, e.g. the same photo of the students at the event, may be used by multiple students, each student need submit the verification separately. \pspace



% Project
\mysection{0.27}{Project}
There will be a project in this course. This project will involve both an individual and group component. Groups will consist of at least two students but no more than five students. As a group, students will choose a mathematical topic in Data Science to research. The project should discuss a way that Data Science is being used in modern society, especially focusing on the societal impact of Data Science. The project groups will prepare a presentation on their topic that will be given the last week of class. Additionally, you will have to write a paper on their group topic. Unlike their presentations, each group member will write their paper individually and independently. Each student's paper topic must be the on the same broad topic/application of Data Science as their group presentation topic. However, the paper need not be and should not be substantially identical to the group presentation. Papers are only required to be at least two pages long, but no longer than five pages, and will be due at the beginning of presentations. With instructor permission, you may substitute an approved creative project in place of a paper. \pspace



% Quizzes 
\mysection{0.27}{Quizzes}
There will be a quiz \textit{every} class, typically at the start of class. Because solutions will often then be immediately discussed, no make-up quizzes will be given (except under extraordinary circumstances). \pspace


% Homeworks 
\mysection{0.27}{Homeworks}
There will typically be a homework assigned each class, due the next class. Homework is a large portion of your grade, so your best work should be put into them. Your solutions should be neat, organized, display effort and clear mathematical thinking, and they should be submitted using the homework packets. Assignments should be started as soon as possible; it is easier to keep up than it is to catch up. You may request extensions on homework assignments (possibly incurring a grade penalty). Requests for extensions should be submitted to the instructor in a timely fashion---do not delay! However, do not simply assume that you will be able to receive extra time on an assignment and plan your schedule carefully. You are encouraged to work with others on homeworks; however, be sure to carefully abide by the academic integrity standards excepted by the college and instructor. \pspace


% Exams 
\mysection{0.27}{Exams}
There will be a total of 3 exams that are each worth 10\% of the course grade for a total of 30\%. The tentative schedule for these exams can be found below. Each exam covers course material, up until the exam preceding it. While the exams are not cumulative, topics from previous exams can appear in an exam if the material is relevant---but it will not be the focus of the exam. You should be present, seated, and prepared for a scheduled exam before the exam begins. If you are late, you should not expect extra exam time. There are no make-up exams except under extraordinary circumstances. \pspace



% Course Schedule 
\mysection{0.27}{Course Schedule}
The following is a \emph{tentative} schedule for the course and is subject to change. 
        \begin{table}[!ht]
        \centering
        \scalebox{1}{%
        \begin{tabular}{ll || ll}
        Date & Topic(s) & Date & Topic(s) \\ \hline 
	09/05 & Labor Day (No Classes) & 10/26 & Mathematics of Finance \\
	09/07 & Course Introduction & 10/31 & Probability \\
	09/12 & Operations, Sets, Functions & 11/02 & Probability  \& Exam 2 \\
	09/14 & Units, Conversions, Magnitudes & 11/07 & Probability \\
	09/19 & (Weighted) Averages & 11/09 & Statistics \\
	09/21 & Length, Area, Volumes, Distances & 11/14 & Statistics \\
	09/26 & Percentages & 11/16 & Statistics \\
	09/28 & Rates \& Fermi Estimation & 11/21 & Statistics \\
	10/03 & Estimathon \& Exam 1 & 11/23 & Thanksgiving Break \\
	10/05 & Linear Functions & 11/28 & Statistics \\
	10/10 & Study Day (No Classes) & 11/30 & Statistics \\
	10/12 & Linear Functions & 12/05 & Statistics \\
	10/17 & Linear Functions & 12/07 & Statistics \\
	10/19 & Mathematics of Finance & 12/12 & Data Science Day \\
	10/24 & Mathematics of Finance & 12/14 & Papers/Presentations \& Exam 3 
        \end{tabular}
        }
        \end{table}


\end{document}