\documentclass[11pt,letterpaper]{article}
\usepackage[lmargin=1in,rmargin=1in,bmargin=1in,tmargin=1in]{geometry}
\usepackage{style}

\pagenumbering{gobble}


% -------------------
% Content
% -------------------
\begin{document}

% TItle
\begin{center} 
\bfseries
\color{stacred}
\LARGE Syllabus Quick Facts \par\vspace{0.2\baselineskip}
\Large MATH 308: Discrete Mathematics --- Fall 2022 
\end{center} \pspace


% Course Information
\mysection{0.27}{Course Information}
\hspace{0.53cm} {\itshape Instructor Email}: \href{mailto:cmcwhort@stac.edu}{cmcwhort@stac.edu} \par
\hspace{0.53cm} {\itshape Course Webpage}: \href{https://coffeeintotheorems.com/courses/2022-2/fall/math-308/}{https://coffeeintotheorems.com/courses/2022-2/fall/math-308/} \par
\hspace{0.53cm} {\itshape Office Hours}: 	\par \vspace{-0.3cm}
	\begin{table}[!ht]
	\centering
	\begin{tabular}{l || l}
	Mon. & 11:30~am -- 12:30~pm \\
	Tues. & 4:00~pm -- 5:00~pm \\
	Wed. & 11:30~am -- 12:30~pm \\
	Thurs. & 4:00~pm -- 5:00~pm \\
	Fri. & 11:30~am -- 1:30~pm
	\end{tabular}
	\end{table}


% Grading Components
\mysection{0.27}{Grading Components\label{grade_comp}}
Course grades are determined by the following components: \par \vspace{-0.3cm}
	\begin{table}[!ht]
        \begin{tabular}{clr}
        & Discussions & 10\% \\
	& Quizzes & 15\% \\
	& Exams & 30\% \\
	& Homework & 45\% \\
        \end{tabular} 
        \end{table}


% Attendance 
\mysection{0.27}{Attendance}
Attend each lecture and show up on time. Anticipated absences should be addressed with the instructor in advance of the absence. Address any absences---anticipated or otherwise---with the instructor. If you miss a lecture, you are responsible for any material covered, any work assigned, any course changes made, etc. during the class. Four or more unexcused absences from lectures could result in receiving a grade penalty per additional absence or an `F' in the course. Furthermore, excessive lateness will also count as an absence. \pspace


% Discussions 
\mysection{0.27}{Discussions}
You are required to meet with the instructor for a discussion lasting five to ten minutes a minimum of seven times during the semester. These meetings may occur during office hours or during another meeting scheduled with the instructor. During the meeting, you will only discuss a topic from the previous week of course material (unless it relates to the material from the previous week, e.g. notation or definitions learned earlier in the semester). The discussion will consist of one or more of the following: giving an overview of the topic, stating definitions or theorems, providing examples or counterexamples to particular concepts, briefly solving a simple problem relating to the topic, addressing a common misconception, etc. Your grade for the discussion is based on your mastery of the material discussed. Your `Discussion' grade will be graded based on seven of these discussions. You may opt to participate in more than seven of these discussion sessions, in which case the highest seven discussion grades will be taken. There are no make-ups for these discussions. 



\newpage



% Quizzes 
\mysection{0.27}{Quizzes}
There will be a quiz \textit{every} class, typically at the start of class. Because solutions will often then be immediately discussed, no make-up quizzes will be given (except under extraordinary circumstances). \pspace


% Homework 
\mysection{0.27}{Homework}
There will typically be a homework assigned each class, due the next class. Homework is a large portion of your grade, so your best work should be put into them. Your solutions should be neat, organized, display effort and clear mathematical thinking, and they should be submitted using the homework packets. Assignments should be started as soon as possible; it is easier to keep up than it is to catch up. You may request extensions on homework assignments (possibly incurring a grade penalty). Requests for extensions should be submitted to the instructor in a timely fashion---do not delay! However, do not simply assume that you will be able to receive extra time on an assignment and plan your schedule carefully. You are encouraged to work with others on homeworks; however, be sure to carefully abide by the academic integrity standards excepted by the college and instructor. \pspace


% Exams 
\mysection{0.27}{Exams} 
There will be a total of 3 exams that are each worth 10\% of the course grade for a total of 30\%. Each of the three exams will be take-home exams, perhaps except for the last exam which will likely in an in-person exam scheduled the week of the exam. The exam procedures will be announced in advance of the exam and may differ between the exams. Typically, the exam will be released `that course week', e.g. Friday, Saturday, or Sunday. You will finish the exam in the allotted time and submit your exam by the deadline (often outside of class). Once an exam is released, students may not discuss the exam or anything related to the exam until the instructor indicates that the exam period has passed. You must work on these exams independently, using only the permitted resources. \pspace


% Course Schedule 
\mysection{0.27}{Course Schedule } 
The following is a \emph{tentative} schedule for the course and is subject to change. 
        \begin{table}[!ht]
        \centering
        \scalebox{1}{%
        \begin{tabular}{ll || ll}
        Date & Topic(s) & Date & Topic(s) \\ \hline 
	09/06 & Logic & 10/25 & Number Theory \\
	09/08 & Logic \& Circuits & 10/27 & Number Theory \\
	09/13 & Predicates \& Quantifiers & 11/01 & Number Theory \\
	09/15 & Predicates, Quantifiers, \& Proofs & 11/03 & Matrices (Exam 2) \\
	09/20 & Sets & 11/08 & Combinatorics \\
	09/22 & Sets & 11/10 & Combinatorics \\
	09/27 & Sets/Functions & 11/15 & Probability \\
	09/29 & Functions & 11/24 & Thanksgiving Break \\
	10/04 & Functions & 11/29 & Probability \\
	10/06 & Induction/Summations/Products (Exam 1) & 12/01 & Graph Theory \\
	10/11 & No Class (Study Day) & 12/06 & Graph Theory \\
	10/13 & Recursion & 12/08 & Graph Theory \\
	10/18 & Recursion & 12/13 & Algorithms \& Complexity \\
	10/20 & Equivalence Relations & 12/15 & Algorithms \& Complexity (Exam 3) \\
        \end{tabular}
        }
        \end{table}






\end{document}