\documentclass[11pt,letterpaper]{article}
\usepackage[lmargin=1in,rmargin=1in,tmargin=1in,bmargin=1in]{geometry}
\usepackage{../style/homework}
\usepackage{../style/commands}
\setbool{quotetype}{true} % True: Side; False: Under
\setbool{hideans}{true} % Student: True; Instructor: False

% -------------------
% Content
% -------------------
\begin{document}

\homework{22: Due 12/15}{The art of simplicity is a puzzle of complexity.}{Douglas Horton}

% Problem 1
\problem{10} Showing all your work and fully justifying your reasoning, answer the following:
	\begin{enumerate}[(a)]
	\item Show that $f(x)= x^5 + 3x^4 - x^2 + 6$ is $\Omega(x^5)$. Does this imply that $f(x)$ is $\Omega(x^4)$?
	\item Show that $g(x)= x^4 - 3x^2 + 6x^2 - 8$ is $O(x^5)$. Does this imply that $g(x)$ is $O(x^6)$?
	\item Show that $h(x)= x^3 - x + 7$ is $\Theta(x^3)$. For $n \neq 3$, can $h(x)$ be $\Theta(x^n)$? Explain. 
	\end{enumerate}



\newpage



% Problem 2
\problem{10} Show that $\displaystyle \sum_{i= 0}^n (3i + 2)$ is $\Theta(n^2)$. 



\newpage



% Problem 3
\problem{10} Define $f(x)= 2x + \log x$ and $h(x)= x^2 + 2^x + 5$. Show that $f(x)$ is $\Theta(x)$ and $h(x)$ is $\Theta(2^x)$. 



\newpage



% Problem 4
\problem{10} Assume that each addition, subtraction, multiplication, division, and \texttt{print} `costs' one flop while defining/redefining variables `cost' no flops. Suppose you have an algorithm, whose pseudocode is given below. \pspace

\hspace{1cm} \texttt{for} i := 1 to n \par
\hspace{1.5cm} \texttt{for} j := 1 to n \par
\hspace{2cm} a := i\^{}3*j + i - 2 + n \par
\hspace{2cm} \texttt{print}(a) \par
\hspace{1.5cm} \texttt{next} j \par
\hspace{1cm} \texttt{next} i \pspace

\begin{enumerate}[(a)]
\item Find the outputs for this algorithm for $n= 2$.
\item  Find the total number of flops performing this algorithm. What is $\Theta$ for this algorithm? 
\end{enumerate}


\end{document}