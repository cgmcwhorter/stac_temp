\documentclass[11pt,letterpaper]{article}
\usepackage[lmargin=1in,rmargin=1in,tmargin=1in,bmargin=1in]{geometry}
\usepackage{../style/homework}
\usepackage{../style/commands}
\setbool{quotetype}{true} % True: Side; False: Under
\setbool{hideans}{false} % Student: True; Instructor: False

% -------------------
% Content
% -------------------
\begin{document}

\homework{14: Due 11/10}{Mathematics is the queen of the sciences and number theory is the queen of mathematics.}{Carl Friedrich Gauss}

% Problem 1
\problem{10} For each of the following pairs $(a, b)$, determine the quotient $q$ and remainder $r$ from the division algorithm and express $b$ as $b= aq + r$:
	\begin{enumerate}[(a)]
	\item $(a, b)= (4, 17)$
	\item $(a, b)= (3, 117)$
	\item $(a, b)= (-6, 25)$
	\item $(a, b)= (9, -82)$
	\end{enumerate} \pspace

\sol 
\begin{enumerate}[(a)]
\item Because $a > 0$, we have $q= \floor*{\dfrac{17}{4}}= 4$ so that $r= 17 - 4 \cdot 4= 17 - 16= 1$. Therefore, $17= 4(4) + 1$. \pspace

\item Because $a > 0$, we have $q= \floor*{\dfrac{117}{3}}= 39$ so that $r= 117 - 3 \cdot 39= 117 - 117= 0$. Therefore, $117= 3(39) + 0$. \pspace
 
\item Because $a < 0$, we have $q= \ceil*{\dfrac{25}{-6}}= -4$ so that $r= 25 - (-6)(-4)= 25 - 24= 1$. Therefore, $25= (-6)(-4) + 1$. \pspace
 
\item Because $a > 0$, we have $q= \floor*{\dfrac{-82}{9}}= -10$ so that $r= -82 - 9(-10)= -82 + 90= 8$. Therefore, $-82= 9(-10) + 8$. \pspace
 
\end{enumerate}



\newpage



% Problem 2
\problem{10} Showing all your work and explaining all your reasoning, answer the following:
	\begin{enumerate}[(a)]
	\item Use the Euclidean algorithm to find $\gcd(220, 815)$. 
	\item Do there exist integer solutions $x, y$ to the equation $20x - 84y= 25$? Explain.
	\end{enumerate} \pspace

\sol 
\begin{enumerate}[(a)]
\item Using the Euclidean algorithm, we have\dots
	\[
	\begin{aligned}
	815&= 220(3) + 155 \\[0.3cm]
	220&= 155(1) + 65 \\[0.3cm]
	155&= 65(2) + 25 \\[0.3cm]
	65&= 25(2) + 15 \\[0.3cm]
	25&= 15(1) + 10 \\[0.3cm]
	15&= 10(1) + 5 \\[0.3cm]
	10&= 5(2)
	\end{aligned}
	\]
Therefore, $\gcd(200, 815)= 5$. \pspace

\item We know that the gcd of two integers, not both zero, divides any linear combination of the two integers; that is, if $a, b \in \mathbb{Z}$ are not both zero and $ax + by= c$, then we know that $\gcd(a, b)$ divides $c$. If there were integers $x, y$ such that $20x - 84y= 25$, then $\gcd(20, 84)$ divides 25. However, $\gcd(20, 84)= 4$ does not divide 25 (for instance, because 4 is even but 25 is odd). Therefore, there can be no integer solutions $x, y$ to $20x - 84y= 25$. 
\end{enumerate}



\newpage



% Problem 3
\problem{10} Showing all your work, use the extended Euclidean algorithm to express $\gcd(350, 480)$ as a linear combination of 350 and 480. \pspace

\sol Using the Euclidean algorithm, we have\dots
	\[
	\begin{aligned}
	480&= 350(1) + 130 \\[0.3cm]
	350&= 130(2) + 90 \\[0.3cm]
	130&= 90(1) + 40 \\[0.3cm]
	90&= 40(2) + 10 \\[0.3cm]
	40&= 10(4)
	\end{aligned}
	\]
Therefore, $\gcd(350, 480)= 10$. Solving for the remainders, we have\dots
	\[
	\begin{aligned}
	10&= 90 - 40(2) \\
	40&= 130 - 90(1) \\
	90&= 350 - 130(2) \\
	130&= 480 - 350(1) 
	\end{aligned}
	\]
Now extending the Euclidean algorithm, we have\dots
	\[
	\begin{aligned}
	10&= 90 - 40(2) \\[0.3cm]
	&= 90 - 2(130 - 1 \cdot 90)= 90 - 2 \cdot 130 + 2 \cdot 90= 3 \cdot 90 - 2 \cdot 130 \\[0.3cm]
	&= 3 \cdot 90 - 2 \cdot 130= 3(350 - 2 \cdot 130) - 2 \cdot 130= 3 \cdot 350 - 6 \cdot 130 - 2 \cdot 130= 3 \cdot 350 - 8 \cdot 130 \\[0.3cm]
	&= 3 \cdot 350 - 8 \cdot 130= 3 \cdot 350 - 8(480 - 1 \cdot 350)= 3 \cdot 350 - 8 \cdot 480 + 8 \cdot 350= -8 \cdot 480 + 11 \cdot 350
	\end{aligned}
	\]
Therefore, we have\dots
	\[
	-8 \cdot 480 + 11 \cdot 350= 10
	\]



\newpage



% Problem 4
\problem{10} Recall that a rational number is a real number of the form $\frac{a}{b}$, where $a, b \in \mathbb{Z}$ and $b \neq 0$. A real number which is not rational is called irrational. All integers are rational numbers: if $n \in \mathbb{Z}$, we have $n= \frac{n}{1}$. Some real numbers are rational, e.g. $0.26= \frac{26}{100}= \frac{13}{50}$ and $0.\overline{3}= \frac{1}{3}$. However, not all real numbers are rational. Write a proof that $\sqrt{2}$ is not rational by completing the following:
	\begin{enumerate}[(a)]
	\item We know that $\sqrt{2}$ is either rational or irrational. If $\sqrt{2}$ is not irrational, what do we know about $\sqrt{2}$? 
	\item Explain why we can write $\sqrt{2}$ as $\sqrt{2}= \frac{a}{b}$, where $a, b \in \mathbb{Z}$, $b \neq 0$, and $\gcd(a, b)= 1$. 
	\item Show that (b) implies that $a^2= 2b^2$.
	\item Use Euclid's Theorem to show that $2 | a$. 
	\item Explain why (d) implies that $b^2= 2k^2$ for some $k \in \mathbb{Z}$.
	\item Explain why (e) implies that $2 | b$. 
	\item Explain why (f) contradicts (b). What does this imply about $\sqrt{2}$?
	\end{enumerate} \pspace

\sol 
\begin{enumerate}[(a)]
\item Because $\sqrt{2}$ is either rational or irrational, if $\sqrt{2}$ is not irrational, then it must be rational, i.e. $\sqrt{2}= \frac{a}{b}$, where $a, b \in \mathbb{Z}$ and $b \neq 0$. \pspace

\item By (a), if $\sqrt{2}$ were rational, then by definition, we know that $\sqrt{2}= \frac{a}{b}$, where $a, b \in \mathbb{Z}$ and $b \neq 0$. Of course, $\frac{a}{b}$ need not be `reduced', i.e. it could be that $\gcd(a, b) > 1$. By multiplying by $1= \frac{1/\gcd(a, b)}{1/\gcd(a, b)}$, we obtain new integers $a', b'$ with $\frac{a}{b}= \frac{a'}{b'}$ and $\gcd(a', b')= 1$. But we could have simply chosen this representation to begin with, i.e. simply define $a= a'$ and $b= b'$. Therefore, we may assume that we have already done that, i.e. $\sqrt{2}= \frac{a}{b}$, where $a, b \in \mathbb{Z}$, $b \neq 0$, and $\gcd(a, b)= 1$. \pspace

\item If $\sqrt{2}= \frac{a}{b}$, then $a= b \sqrt{2}$. Squaring both sides, we obtain $a^2= (b \sqrt{2})^2= 2b^2$. \pspace

\item Clearly, $2 \mid (2b^2)$. But then 2 divides $a^2$ because $a^2= 2b^2$. But because $a^2= a \cdot a$, by Euclid's Theorem, we know that $2 \mid a$ or $2 \mid a$, i.e. 2 divides $a$. \pspace

\item By (d), we know that $2 \mid a$, i.e. $a$ is a multiple of 2. But then $a= 2k$ for some $k \in \mathbb{Z}$. Then we know that $2b^2= a^2= (2k)^2= 4k^2$, i.e. $2b^2= 4k^2$. Dividing both sides by 2, we obtain $b^2=  2k^2$. \pspace

\item By (e), we know that $b^2= 2k^2$. But because $2 \mid (2k^2)$, we know that 2 divides $b^2$ because $b^2= 2k^2$. Because $b^2= b \cdot b$, by Euclid's Theorem, we know that $2 \mid b$ or $2 \mid b$, i.e. 2 divides $b$. \pspace

\item By (d) and (f), we know that $2 \mid a$ and $2 \mid b$. But then $\gcd(a, b) \geq 2$. This contradicts the assumption in (b) that we have chosen $a, b$ such that $\gcd(a, b)= 1$. Therefore, it cannot be that $\sqrt{2}$ is rational. This shows that $\sqrt{2}$ must be irrational. 
\end{enumerate}


\end{document}