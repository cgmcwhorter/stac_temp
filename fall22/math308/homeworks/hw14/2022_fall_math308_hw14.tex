\documentclass[11pt,letterpaper]{article}
\usepackage[lmargin=1in,rmargin=1in,tmargin=1in,bmargin=1in]{geometry}
\usepackage{../style/homework}
\usepackage{../style/commands}
\setbool{quotetype}{true} % True: Side; False: Under
\setbool{hideans}{true} % Student: True; Instructor: False

% -------------------
% Content
% -------------------
\begin{document}

\homework{14: Due 11/10}{Mathematics is the queen of the sciences and number theory is the queen of mathematics.}{Carl Friedrich Gauss}

% Problem 1
\problem{10} For each of the following pairs $(a, b)$, determine the quotient $q$ and remainder $r$ from the division algorithm and express $b$ as $b= aq + r$:
	\begin{enumerate}[(a)]
	\item $(a, b)= (4, 17)$
	\item $(a, b)= (3, 117)$
	\item $(a, b)= (-6, 25)$
	\item $(a, b)= (9, -82)$
	\end{enumerate}



\newpage



% Problem 2
\problem{10} Showing all your work and explaining all your reasoning, answer the following:
	\begin{enumerate}[(a)]
	\item Use the Euclidean algorithm to find $\gcd(220, 815)$. 
	\item Do there exist integer solutions $x, y$ to the equation $20x - 84y= 25$? Explain.
	\end{enumerate}



\newpage



% Problem 3
\problem{10} Showing all your work, use the extended Euclidean algorithm to express $\gcd(350, 480)$ as a linear combination of 350 and 480. 



\newpage



% Problem 4
\problem{10} Recall that a rational number is a real number of the form $\frac{a}{b}$, where $a, b \in \mathbb{Z}$ and $b \neq 0$. A real number which is not rational is called irrational. All integers are rational numbers: if $n \in \mathbb{Z}$, we have $n= \frac{n}{1}$. Some real numbers are rational, e.g. $0.26= \frac{26}{100}= \frac{13}{50}$ and $0.\overline{3}= \frac{1}{3}$. However, not all real numbers are rational. Write a proof that $\sqrt{2}$ is not rational by completing the following:
	\begin{enumerate}[(a)]
	\item We know that $\sqrt{2}$ is either rational or irrational. If $\sqrt{2}$ is not irrational, what do we know about $\sqrt{2}$? 
	\item Explain why we can write $\sqrt{2}$ as $\sqrt{2}= \frac{a}{b}$, where $a, b \in \mathbb{Z}$, $b \neq 0$, and $\gcd(a, b)= 1$. 
	\item Show that (b) implies that $a^2= 2b^2$.
	\item Use Euclid's Theorem to show that $2 | a$. 
	\item Explain why (d) implies that $b^2= 2k^2$ for some $k \in \mathbb{Z}$.
	\item Explain why (e) implies that $2 | b$. 
	\item Explain why (f) contradicts (b). What does this imply about $\sqrt{2}$?
	\end{enumerate}


\end{document}