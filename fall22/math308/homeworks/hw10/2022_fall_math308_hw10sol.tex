\documentclass[11pt,letterpaper]{article}
\usepackage[lmargin=1in,rmargin=1in,tmargin=1in,bmargin=1in]{geometry}
\usepackage{../style/homework}
\usepackage{../style/commands}
\setbool{quotetype}{true} % True: Side; False: Under
\setbool{hideans}{false} % Student: True; Instructor: False

% -------------------
% Content
% -------------------
\begin{document}

\homework{10: Due 10/13}{I think that some intuition leaks out in every step of an induction proof.}{Jim Propp}

% Problem 1
\problem{10} Let $\{ a_n \}_{n \in \mathbb{N}}$ be the sequence defined by $a_n:= 2^n - 5$ and $\{ b_m \}_{m \in \mathbb{Z}^\times}$ be defined by $b_m:= \dfrac{m + 1}{m}$. Showing all your work, compute the following:
	\begin{2enumerate}
	\item $\displaystyle \sum_{k= 0}^5 a_k$
	\item $\displaystyle \sum_{\substack{j= -3 \\ j \neq 0}}^3 b_m$
	\item $\displaystyle \prod_{k= 1}^3 a_n$
	\item $\displaystyle \sum_{p=0}^0 a_p$
	\item $\displaystyle \sum_{j=2}^4 (a_j + b_j)$
	\item $\displaystyle \prod_{n=1}^{10^{50}} b_n$
	\end{2enumerate} \pspace

\sol 
\begin{enumerate}[(a)]
\item 
	\[
	\sum_{k= 0}^5 a_k= a_0 + a_1 + a_2 + a_3 + a_4 + a_5= -4 + (-3) + (-1) + 3 + 11 + 27= 33
	\]

\item 
	\[
	\sum_{\substack{j= -3 \\ j \neq 0}}^3 b_m= b_{-3} + b_{-2} + b_{-1} + b_1 + b_2 + b_3= \dfrac{2}{3} + \dfrac{1}{2} + 0 + 2 + \dfrac{3}{2} + \dfrac{4}{3}= 6
	\]

\item 
	\[
	\prod_{k= 1}^3 a_n= a_1 \cdot a_2 \cdot a_3= -3 \cdot -1 \cdot 3= 9
	\]

\item 
	\[
	\sum_{p=0}^0 a_p= a_0= -3
	\]

\item 
	\[
	\sum_{j=2}^4 (a_j + b_j)= (a_2 + b_2) + (a_3 + b_3) + (a_4 + b_4)= \left(-1 + \dfrac{3}{2} \right) + \left(3 + \dfrac{4}{3} \right) + \left(11 + \dfrac{5}{4} \right)= \dfrac{1}{2} + \dfrac{13}{3} + \dfrac{49}{4}= \dfrac{205}{12}
	\]

\item 
	\[
	\hspace{-1.1cm} \prod_{n=1}^{10^{50}} b_n= b_1 \cdot b_2 \cdot \cdots \cdot b_{10^{50}}= 2 \cdot \dfrac{3}{2} \cdot \dfrac{4}{3} \cdot \dfrac{5}{4} \cdot \dfrac{6}{5} \cdot\; \cdots\; \cdot \dfrac{10^{50}}{10^{50} - 1} \cdot \dfrac{10^{50} + 1}{10^{50}}= \cancel{2} \cdot \dfrac{\cancel{3}}{\cancel{2}} \cdot \dfrac{\cancel{4}}{\cancel{3}} \cdot \dfrac{\cancel{5}}{\cancel{4}} \cdot \dfrac{\cancel{6}}{\cancel{5}} \cdot\; \cdots\; \cdot \dfrac{\cancel{10^{50}}}{\cancel{10^{50} - 1}} \cdot \dfrac{10^{50} + 1}{\cancel{10^{50}}}= 10^{50} + 1
	\]
\end{enumerate}



\newpage



% Problem 2
\problem{10} Let $a \in \mathbb{R}$. Consider the following sum defined for $n > 7$:
	\[
	\sum_{k=7}^n \left(k + a - 7 \right)^2
	\]

\begin{enumerate}[(a)]
\item Reindex the sum above so that it begins at $k= 0$. 
\item Using the given summation formulas below, find the sum from (a) in terms of $n, a$ alone. 
	\[
	\sum_{k=0}^n 1= n + 1, \qquad \sum_{k=0}^n k= \dfrac{n (n + 1)}{2}, \qquad \sum_{k=0}^n k^2= \dfrac{n (n + 1)(2n + 1)}{6}
	\]
\end{enumerate} 

\sol 
\begin{enumerate}[(a)]
\item We have\dots
	\[
	\sum_{k=7}^n \left(k + a - 7 \right)^2= \sum_{k= 7 - 7}^{n - 7} \big( (k + 7) + a - 7 \big)^2= \sum_{k=0}^{n - 7} (k + a)^2
	\] \pspace

\item First, observe that\dots
	\[
	\begin{aligned}
	\sum_{k=7}^n \left(k + a - 7 \right)^2&= \sum_{k=0}^{n - 7} (k + a)^2 \\
	&= \sum_{k=0}^{n - 7} (k^2 + 2ak + a^2) \\
	&= \sum_{k=0}^{n - 7} k^2 + \sum_{k=0}^{n - 7} 2ak + \sum_{k=0}^{n - 7} a^2 \\
	&= \sum_{k=0}^{n - 7} k^2 + 2a \sum_{k=0}^{n - 7} k + a^2 \sum_{k=0}^{n - 7} 1 
	\end{aligned}
	\]
Recall the following formulas:
	\[
	\sum_{k=0}^n 1= n + 1, \qquad \sum_{k=0}^n k= \dfrac{n(n + 1)}{2}, \qquad \sum_{k=0}^n k^2= \dfrac{n(n + 1)(2n + 1)}{6}
	\]
But then we have\dots
	\[
	\begin{aligned}
	\sum_{k=7}^n \left(k + a - 7 \right)^2&= \sum_{k=0}^{n - 7} k^2 + 2a \sum_{k=0}^{n - 7} k + a^2 \sum_{k=0}^{n - 7} 1 \\
	&= \dfrac{(n - 7)(n - 7 + 1) \big( 2(n - 7) + 1 \big)}{6} + 2a \cdot \dfrac{(n - 7)(n - 7 + 1)}{2} + a^2 \cdot (n - 7 + 1) \\
	&= \dfrac{(n - 7)(n - 6)(2n - 13)}{6} + a(n - 7)(n - 6) + a^2 (n - 6)
	\end{aligned}
	\]
\end{enumerate}



\newpage



% Problem 3
\problem{10} Complete the proof of the given proposition below by filling in the corresponding blanks. \pspace

\noindent {\bfseries Proposition.} For $n \geq 2$, $\displaystyle\prod_{k=2}^n \left(1 - \dfrac{1}{k^2} \right)= \dfrac{n + 1}{2n}$. \pspace

\noindent {\itshape Proof.} We prove this using \uans{1.7cm}{weak induction}. First, we establish a base case. \pspace

\noindent {\itshape Base Case}: Let $n= 2$. Then we have\dots
	\[
	\begin{aligned}
	\uans{0.2cm}{\prod_{k=2}^2 \left(1 - \frac{1}{k^2} \right)}&= \uans{0.3cm}{1 - \frac{1}{2^2}}= \uans{0.3cm}{1 - \frac{1}{4}}= \dfrac{3}{4} \\[0.3cm]
	\dfrac{n + 1}{2n} \bigg|_{n= 2}&= \dfrac{2 + 1}{2(2)}= \dfrac{3}{4}
	\end{aligned}
	\]
But then if $n= 2$, we know that $\displaystyle\prod_{k=2}^n \left(1 - \dfrac{1}{k^2} \right)= \dfrac{n + 1}{2n}$. 

We know establish the induction step. 

\noindent {\itshape Induction Step}: Assume that for $n= N$, $\displaystyle\prod_{k=2}^N \left(1 - \dfrac{1}{k^2} \right)= \dfrac{N + 1}{2N}$. We show that the statement of \pspace 

the proposition is then true for $n=$ \uans{0.5cm}{$N + 1$}. We have\dots \pspace
	\[
	\begin{aligned}
	\prod_{k=2}^{N+1} \left(1 - \dfrac{1}{k^2} \right)&= \uans{0.1cm}{\left(1 - \dfrac{1}{(N + 1)^2} \right)} \cdot \prod_{k=2}^N \left(1 - \dfrac{1}{k^2} \right) \\[0.3cm]
	&= \uans{0cm}{\dfrac{(N + 1)^2}{(N + 1)^2} - \dfrac{1}{(N + 1)^2}} \cdot \uans{0.65cm}{\dfrac{N + 1}{2N}} \\[0.3cm]
	&= \uans{0.35cm}{\dfrac{N^2 + 2N + 1 - 1}{(N + 1)^2}} \cdot \uans{0.65cm}{\dfrac{N + 1}{2N}} \\[0.3cm]
	&= \uans{1.05cm}{\dfrac{N^2 + 2N}{(N + 1)^2}} \cdot \uans{0.65cm}{\dfrac{N + 1}{2N}} \\[0.3cm]
	&= \dfrac{N + 2}{2(N + 1)} \\[0.3cm]
	&= \dfrac{(N + 1) + 1}{2(N + 1)}
	\end{aligned}
	\]
But then we know that $\displaystyle\prod_{k=2}^{N+1} \left(1 - \dfrac{1}{k^2} \right)= \dfrac{(N + 1) + 1}{2(N + 1)}$. 

Therefore, by \uans{1.41cm}{weak induction}, we know that for $n \geq 2$, $\displaystyle\prod_{k=2}^n \left(1 - \dfrac{1}{k^2} \right)= \dfrac{n + 1}{2n}$. \qed



\newpage



% Problem 4
\problem{10} Let $\{ a_n \}_{n \in \mathbb{Z}^{\geq 0}}$ be the recursive sequence given by $a_0= 1$, $a_1= 3$, and $a_n= 2a_{n - 1}  - a_{n - 2}$ for $n \geq 2$. A student observe that $a_0= 1$, $a_1= 3$, $a_2= 5$, $a_3= 7$, and $a_4= 9$. They then predict that $a_n= 2n + 1$ for $n \geq 0$. Below is a proof of this conjecture, with parts of their proof removed. Complete the missing parts. \pspace

\noindent {\bfseries Proposition.} Let $\{ a_n \}_{n \in \mathbb{Z}^{\geq 0}}$ be the recursive sequence given by $a_0= 1$, $a_1= 3$, and $a_n= 2a_{n - 1}  - a_{n - 2}$ for $n \geq 2$. Then for all $n \geq 0$, $a_n= 2n + 1$. \pspace

\noindent {\itshape Proof.} We prove this using \uans{1.55cm}{Strong Induction}. First, we establish a few bases cases. \pspace

\noindent {\itshape Base Case}: If \uans{1cm}{$n= 0$}\,, we have $a_0= 1$ and $2n + 1= 2(0) + 1= 1$. Then if $n= 0$, we have \pspace

$a_n= 2n + 1$. Now if $n=$ \uans{1.37cm}{1}\,, we have \uans{1.45cm}{$a_1= 3$} and \uans{0.53cm}{$a_1= 2(1) + 1= 3$}. \pspace

But then if $n= 1$, we have \uans{0.5cm}{$a_n= 2n + 1$}. \pvspace{1cm}

We now establish the induction case. \pspace

\noindent {\itshape Induction Case}: Now assume that $a_k= 2k + 1$ for all $0 \leq k \leq n$. Now consider the term \pspace

\uans{0.7cm}{$k= n + 1$}. \pspace

We have\dots
	\[
	\begin{aligned}
	a_{n + 1}&= 2a_n - a_{n - 1} \\[0.3cm]
	&= \uans{0.8cm}{2(2n + 1) - (2(n - 1) + 1)} \\[0.3cm]
	&= \uans{1.35cm}{4n + 2 - 2n + 2 - 1} \\[0.3cm]
	&= 2n + 3 \\[0.3cm]
	&= 2(n + 1) + 1
	\end{aligned}
	\] \pspace
But then we know that $a_{n + 1}= 2(n + 1) + 1$. \pspace

Therefore, by \uans{1.59cm}{Strong Induction}\,, we know that $a_n= 2n + 1$ for all $n \geq 0$. \qed


\end{document}