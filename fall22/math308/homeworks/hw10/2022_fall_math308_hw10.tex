\documentclass[11pt,letterpaper]{article}
\usepackage[lmargin=1in,rmargin=1in,tmargin=1in,bmargin=1in]{geometry}
\usepackage{../style/homework}
\usepackage{../style/commands}
\setbool{quotetype}{true} % True: Side; False: Under
\setbool{hideans}{true} % Student: True; Instructor: False

% -------------------
% Content
% -------------------
\begin{document}

\homework{10: Due 10/13}{I think that some intuition leaks out in every step of an induction proof.}{Jim Propp}

% Problem 1
\problem{10} Let $\{ a_n \}_{n \in \mathbb{N}}$ be the sequence defined by $a_n:= 2^n - 5$ and $\{ b_m \}_{m \in \mathbb{Z}^\times}$ be defined by $b_m:= \dfrac{m + 1}{m}$. Showing all your work, compute the following:
	\begin{2enumerate}
	\item $\displaystyle \sum_{k= 0}^5 a_k$
	\item $\displaystyle \sum_{\substack{j= -3 \\ j \neq 0}}^3 b_m$
	\item $\displaystyle \prod_{k= 1}^3 a_n$
	\item $\displaystyle \sum_{p=0}^0 a_p$
	\item $\displaystyle \sum_{j=2}^4 (a_j + b_j)$
	\item $\displaystyle \prod_{n=1}^{10^{50}} b_n$
	\end{2enumerate}



\newpage



% Problem 2
\problem{10} Let $a \in \mathbb{R}$. Consider the following sum defined for $n > 7$:
	\[
	\sum_{k=7}^n \left(k + a - 7 \right)^2
	\]

\begin{enumerate}[(a)]
\item Reindex the sum above so that it begins at $k= 0$. 
\item Using the given summation formulas below, find the sum from (a) in terms of $n, a$ alone. 
	\[
	\sum_{k=0}^n 1= n + 1, \qquad \sum_{k=0}^n k= \dfrac{n (n + 1)}{2}, \qquad \sum_{k=0}^n k^2= \dfrac{n (n + 1)(2n + 1)}{6}
	\]
\end{enumerate}



\newpage



% Problem 3
\problem{10} Complete the proof of the given proposition below by filling in the corresponding blanks. \pspace

\noindent {\bfseries Proposition.} For $n \geq 2$, $\displaystyle\prod_{k=2}^n \left(1 - \dfrac{1}{k^2} \right)= \dfrac{n + 1}{2n}$. \pspace

\noindent {\itshape Proof.} We prove this using \underline{\hspace{6cm}}. First, we establish a base case. \pspace

\noindent {\itshape Base Case}: Let $n= 2$. Then we have\dots
	\[
	\begin{aligned}
	\underline{\hspace{2cm}}&= \underline{\hspace{2cm}}= \underline{\hspace{2cm}}= \dfrac{3}{4} \\[0.3cm]
	\dfrac{n + 1}{2n} \bigg|_{n= 2}&= \dfrac{2 + 1}{2(2)}= \dfrac{3}{4}
	\end{aligned}
	\]
But then if $n= 2$, we know that $\displaystyle\prod_{k=2}^n \left(1 - \dfrac{1}{k^2} \right)= \dfrac{n + 1}{2n}$. \pspace

We know establish the induction step. \pspace

\noindent {\itshape Induction Step}: Assume that for $n= N$, $\displaystyle\prod_{k=2}^N \left(1 - \dfrac{1}{k^2} \right)= \dfrac{N + 1}{2N}$. We show that the statement of \pspace 

the proposition is then true for $n=$ \underline{\hspace{2cm}}. We have\dots \pspace
	\[
	\begin{aligned}
	\prod_{k=2}^{N+1} \left(1 - \dfrac{1}{k^2} \right)&= \underline{\hspace{3cm}} \cdot \prod_{k=2}^N \left(1 - \dfrac{1}{k^2} \right) \\[0.3cm]
	&= \underline{\hspace{3cm}} \cdot \underline{\hspace{3cm}} \\[0.3cm]
	&= \underline{\hspace{3cm}} \cdot \underline{\hspace{3cm}} \\[0.3cm]
	&= \underline{\hspace{3cm}} \cdot \underline{\hspace{3cm}} \\[0.3cm]
	&= \dfrac{N + 2}{2(N + 1)} \\[0.3cm]
	&= \dfrac{(N + 1) + 1}{2(N + 1)}
	\end{aligned}
	\]
But then we know that $\displaystyle\prod_{k=2}^{N+1} \left(1 - \dfrac{1}{k^2} \right)= \dfrac{(N + 1) + 1}{2(N + 1)}$. \pspace

Therefore, by \underline{\hspace{6cm}}, we know that for $n \geq 2$, $\displaystyle\prod_{k=2}^n \left(1 - \dfrac{1}{k^2} \right)= \dfrac{n + 1}{2n}$. \qed



\newpage



% Problem 4
\problem{10} Let $\{ a_n \}_{n \in \mathbb{Z}^{\geq 0}}$ be the recursive sequence given by $a_0= 1$, $a_1= 3$, and $a_n= 2a_{n - 1}  - a_{n - 2}$ for $n \geq 2$. A student observe that $a_0= 1$, $a_1= 3$, $a_2= 5$, $a_3= 7$, and $a_4= 9$. They then predict that $a_n= 2n + 1$ for $n \geq 0$. Below is a proof of this conjecture, with parts of their proof removed. Complete the missing parts. \pspace

\noindent {\bfseries Proposition.} Let $\{ a_n \}_{n \in \mathbb{Z}^{\geq 0}}$ be the recursive sequence given by $a_0= 1$, $a_1= 3$, and $a_n= 2a_{n - 1}  - a_{n - 2}$ for $n \geq 2$. Then for all $n \geq 0$, $a_n= 2n + 1$. \pspace

\noindent {\itshape Proof.} We prove this using \underline{\hspace{6cm}}. First, we establish a few bases cases. \pspace

\noindent {\itshape Base Case}: If \underline{\hspace{3cm}}\,, we have $a_0= 1$ and $2n + 1= 2(0) + 1= 1$. Then if $n= 0$, we have \pspace

$a_n= 2n + 1$. Now if $n=$ \underline{\hspace{3cm}}\,, we have \underline{\hspace{4cm}} and \underline{\hspace{4cm}}. \pspace

But then if $n= 1$, we have \underline{\hspace{3cm}}.  \pvspace{1cm}

We now establish the induction case. \pspace

\noindent {\itshape Induction Case}: Now assume that $a_k= 2k + 1$ for all $0 \leq k \leq n$. Now consider the term \pspace

\underline{\hspace{3cm}}. \pspace

We have\dots
	\[
	\begin{aligned}
	a_{n + 1}&= 2a_n - a_{n - 1} \\[0.3cm]
	&= \underline{\hspace{6cm}} \\[0.3cm]
	&= \underline{\hspace{6cm}} \\[0.3cm]
	&= 2n + 3 \\[0.3cm]
	&= 2(n + 1) + 1
	\end{aligned}
	\] \pspace
But then we know that $a_{n + 1}= 2(n + 1) + 1$. \pspace

Therefore, by \underline{\hspace{6cm}}\,, we know that $a_n= 2n + 1$ for all $n \geq 0$. \qed


\end{document}