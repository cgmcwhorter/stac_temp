\documentclass[11pt,letterpaper]{article}
\usepackage[lmargin=1in,rmargin=1in,tmargin=1in,bmargin=1in]{geometry}
\usepackage{../style/homework}
\usepackage{../style/commands}
\setbool{quotetype}{false} % True: Side; False: Under
\setbool{hideans}{false} % Student: True; Instructor: False

% -------------------
% Content
% -------------------
\begin{document}

\homework{3: Due 09/15}{If people do not believe that mathematics is simple, it is only because they do not realize how complicated life is.}{George Polya}

% Problem 1
\problem{10} Let the universe, $\mathcal{U}$, for $m, n, j, k$ be the set of integers. Define the following predicates:
	\[
	\begin{aligned}
	P(m) &\colon m \text{ is even} \\
	Q(n) &\colon n \text{ is a perfect square} \\
	R(j) &\colon j \text{ is divisible by 3} \\
	S(k) &\colon k \text{ is divisible by 6} \\
	W(m) &\colon 1 < m \leq 8
	\end{aligned}
	\]
Write the open sentences below as complete English sentences as `simply' as possible and then determine whether the statement is true or false. If the statement is true, explain why. If not, give a counterexample. 
	\begin{enumerate}[(a)]
	\item $(\exists! n) \big( Q(n) \wedge W(n) \big)$ 
	\item $(\forall m) \big( P(m) \vee R(m) \big)$ 
	\item $(\forall m) \big( \neg R(m) \to \neg S(m) \big)$ 
	\item $(\forall m) (\exists n) \big( S(m) \to [(m= 2n) \wedge P(n)] \big)$ 
	\end{enumerate} \pspace

\sol
\begin{enumerate}[(a)]
\item Written in words, the proposition $(\exists! n) \big( Q(n) \wedge W(n) \big)$ is the statement, ``There exists a unique integer $n$ such that $n$ is a perfect square and $1 < n \leq 8$.'' The only perfect squares from 1 to 10 are 1, 4, 9. The only one of these greater than 1 and at most 8 is 1. Therefore, there is a unique perfect square greater than 1 and at most 8. Therefore, $(\exists! n) \big( Q(n) \wedge W(n) \big)$ is true. \pspace

\item Written in words, the proposition $(\forall m) \big( P(m) \vee R(m) \big)$ is the statement, ``For all integers $m$, either $m$ is even $m$ is divisible by 3.'' The statement is false. As a counterexample, if $n= 1$, then $n$ is an integer but $n$ is neither even nor divisible by 3. \pspace
 
\item Written in words, the proposition $(\forall m) \big( \neg R(m) \to \neg S(m) \big)$ is the statement, ``For all integers $m$, if $m$ is not divisible by 3, then $m$ is not divisible by 6.'' This statement is true. If $m$ is not divisible by 3, then it does not have 3 as a factor. If $m$ is divisible by 6, then $m$ has a factor of 6. But 6 has 3 as a factor. This would imply that $m$ has 3 as a factor. Therefore, if $m$ is not divisible by 3, it is not divisible by 6. Alternatively, the contrapositive of ``if $m$ is not divisible by 3, then $m$ is not divisible by 6'' is the statement ``if $m$ is divisibly by 6, then $m$ is divisible by 3.'' Because any integer divisible by 6 must be divisible by 3, the contrapositive is true. But then the original statement is also true. \pspace


\item Written in words, the proposition $(\forall m) (\exists n) \big( S(m) \to [(m= 2n) \wedge P(n)] \big)$ is the statement, ``For all integers $m$, there exists an integer $n$ such that if $m$ is divisible by 6, then $m= 2n$ and $n$ is even.'' The statement is false. Take $m= 6$. Then $m$ is divisible by 6. If there were an integer $n$ with $m= 2n$, then we know $n= m/6= 6/6= 1$. But then $n$ is not even. This shows that the statement $m= 2n$ and $n$ is even is false. Therefore, the statement $S(m) \to [(m= 2n) \wedge P(n)]$ is false so that the statement $(\forall m) (\exists n) \big( S(m) \to [(m= 2n) \wedge P(n)] \big)$. 
\end{enumerate}



\newpage



% Problem 2
\problem{10} By defining appropriate universes and predicates, quantify the open sentences below. Indicate whether the resulting statement is true or false. No justification is necessary. 
	\begin{enumerate}[(a)]
	\item For all $m$, there exists $n$ such that $m= n + 1$.
	\item For all integers $n$, if $n$ is divisible by 5 then the 1's digit of $n$ is either 0 or 5.
	\item Nonzero real numbers have a unique multiplicative inverse. 
	\item Given any pair of distinct integers, there is another integer between them.
	\item Everybody has problems. 
	\end{enumerate} \pspace

\sol
\begin{enumerate}[(a)]
\item 
\item 
\item 
\item 
\item 
\end{enumerate}



\newpage



% Problem 3
\problem{10} Being as clear and detailed as possible, explain why $\exists x P(x) \wedge \exists x Q(x)$ does not imply $\exists x [P(x) \wedge Q(x)]$. \pspace

\sol 



\newpage



% Problem 4
\problem{10} Let $P(x)$ be the predicate $R(x)$: $x$ is a rectangle and let $S(x)$ be the predicate $S(x)$: $x$ is a square. 
	\begin{enumerate}[(a)]
	\item Write $\forall x \big( R(x) \to S(x) \big)$ as a complete English sentence.
	\item Write the contrapositive, converse, and negation of the open sentence in (a) as complete English sentences. 
	\end{enumerate} \pspace

\sol
\begin{enumerate}[(a)]
\item 
\item 
\end{enumerate}
















\end{document}