\documentclass[11pt,letterpaper]{article}
\usepackage[lmargin=1in,rmargin=1in,tmargin=1in,bmargin=1in]{geometry}
\usepackage{../style/homework}
\usepackage{../style/commands}
\setbool{quotetype}{false} % True: Side; False: Under
\setbool{hideans}{true} % Student: True; Instructor: False

% -------------------
% Content
% -------------------
\begin{document}

\homework{4: Due 09/20}{Pure Mathematics is the world's best game. It is more absorbing than chess, more of a gamble than poker, and lasts longer than Monopoly. It's free. It can be played anywhere---Archimedes did it in a bathtub.}{Richard J. Trudeau}

% Problem 1
\problem{10} Suppose that $P(x)$ is a predicate. Being sure to justify your answer, explain whether the following statements are true or false.
	\begin{enumerate}[(a)]
	\item There are choices of $x$ for which $P(x)$ is true and choices of $x$ for which $P(x)$ is false.
	\item Once one quantifies $P(x)$ using $\forall x$ or $\exists x$, the resulting statement is always true or always false---but not both.
	\item If $\exists!x\, P(x)$ is true, then $\exists x\, P(x)$ is true. 
	\item The converse of (c) is also true.
	\end{enumerate}



\newpage



% Problem 2
\problem{10} Let the universe for $x$ be the set of real numbers. Let $P(x)$ be the predicate $P(x) \colon 0 < x^2 \leq 50$ and $Q(x)$ be the predicate $Q(x) \colon x^2= 50$.
	\begin{enumerate}[(a)]
	\item Find at least two values for which $P(x)$ is true and two values for which $P(x)$ is false. Do the same for $Q(x)$. 
	\item Find the truth set for $P(x)$, and also for $Q(x)$.
	\item Is it true that there is a unique $x$ in the domain such that $P(x) \wedge Q(x)$ is true? Explain.
	\item How would your answer in (b) change if the universe were instead the set of integers? Explain. 
	\end{enumerate}



\newpage



% Problem 3
\problem{10} Students in their first algebra course may believe that the following rule is true for real numbers: $\sqrt{x + y}= \sqrt{x} + \sqrt{y}$. Write this `rule' as a quantified open statement in English, being as clear and specific as possible. Then prove or disprove the resulting statement.  



\newpage



% Problem 4
\problem{10} A certain computer program has $n$ as an integer variable. Suppose that $A$ is an array of 20~integers values, i.e. $A$ is a `list' of the integer values $A[1], A[2], \ldots, A[20]$. Write the following as quantified open statements using $A[k]$:
	\begin{enumerate}[(a)]
	\item Every entry in the array is nonnegative.
	\item The value $A[1]$ is the smallest value in the array.
	\item The array is sorted in ascending order. 
	\item All the values in the array are distinct. 
	\end{enumerate}



\newpage



% Problem 5
\problem{10} Showing all your work and simplifying your logical expression as much as possible, negate the following quantified open statements:
	\begin{enumerate}[(a)]
	\item $\forall x \big( P(x) \to \neg Q(x) \big)$
	\item $\exists x \big( P(x) \Longleftrightarrow Q(x) \wedge R(x) \big)$
	\item $\forall x\, \exists y\, \big( P(x, y) \vee Q(x, y) \big)$
	\item $\forall x \big( P(x) \to 1 < x < 3 \big)$
	\end{enumerate}



\newpage



% Problem 6
\problem{10} Recall that the definition of a function, $f(x)$, having a limit as $x$ approaches $a$ was as follows: we say that the limit of $f(x)$ as $x$ approaches $a$ is $L$, denoted $\displaystyle \lim_{x \to a} f(x)= L$, if for all $\epsilon > 0$, there exists $\delta > 0$ such that for all $x$, if $|x - a| < \delta$, then $|f(x) - L| < \epsilon$. 
	\begin{enumerate}[(a)]
	\item Write the definition above using logical symbols and quantifiers.
	\item Find the definition of \textit{not} having a limit by negating the logical expression from (a).
	\item Explain why $\displaystyle \lim_{x \to 0} \dfrac{1}{x}$ does not exist using your response from (b) and considering what happens when $x= 1/n$ and $n \to \infty$. 
	\end{enumerate}


\end{document}