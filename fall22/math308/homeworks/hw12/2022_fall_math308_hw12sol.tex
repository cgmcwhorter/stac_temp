\documentclass[11pt,letterpaper]{article}
\usepackage[lmargin=1in,rmargin=1in,tmargin=1in,bmargin=1in]{geometry}
\usepackage{../style/homework}
\usepackage{../style/commands}
\setbool{quotetype}{false} % True: Side; False: Under
\setbool{hideans}{false} % Student: True; Instructor: False

% -------------------
% Content
% -------------------
\begin{document}

\homework{12: Due 11/04}{Algebra is the intellectual instrument which has been created for rendering clear the quantitative aspects of the world.}{Alfred North Whitehead}

% Problem 1
\problem{10} Showing all your work, complete the following:
	\begin{enumerate}[(a)]
	\item Find the last digit of $3^{300}$.
	\item Find the last two digits of $13^{100}$.
	\item Fermat's Little Theorem states that if $p$ is prime, then $a^p \equiv a \mod p$. Verify this claim when $p= 5$ and $a= 3$. 
	\item A generalization of Fermat's Little Theorem states that $a^{\varphi(n)} \equiv 1 \mod n$ if $a$ is coprime to $n$, where $\varphi(n)$ is the Euler Phi function. Verify this claim when $n= 3$ and $a= 8$. 
	\end{enumerate} \pspace

\sol 
\begin{enumerate}[(a)]
\item Given an $n$-digit number $a:= a_{n-1}a_{n-2} \cdots a_3a_2a_1a_0$, we can write $a_{n-1}a_{n-2} \cdots a_2a_1 \cdot 10 + a_0$ so that $a \equiv a_{n-1}a_{n-2} \cdots a_2a_1 \cdot 10 + a_0 \equiv a_0 \mod 10$. For example, $175892 \equiv 175890 + 2 \equiv 17589 \cdot 10 + 2 \equiv 2 \mod 10$. Therefore, the last digit of an integer is its reduction modulo 10. Observe that\dots
	\[
	\begin{aligned}
	3^0&\equiv 1 \mod 10 &\qquad 3^{16}&\equiv 1^2 \equiv 1 \mod 10 \\
	3^1&\equiv 3 \mod 10 & 3^{32}&\equiv 1^2 \equiv 1 \mod 10 \\
	3^2&\equiv 9 \mod 10 & 3^{64}&\equiv 1^2 \equiv 1 \mod 10 \\
	3^4&\equiv 9^2 \equiv 81 \equiv 1 \mod 10 & 3^{128}&\equiv 1^2 \equiv 1 \mod 10 \\
	3^8&\equiv 1^2 \equiv 1 \mod 10 & 3^{256}&\equiv 1^2 \equiv 1 \mod 10
	\end{aligned}
	\]
But then we have\dots
	\[
	3^{300} \equiv 3^{256 + 32 + 8 + 4} \equiv 3^{256} \cdot 3^{32} \cdot 3^8 \cdot 3^4 \equiv 1 \cdot 1 \cdot 1 \cdot 1 \equiv 1 \mod 10
	\]
Therefore, the last digit of $3^{300}$ is $1$. \pspace

\item Given an $n$-digit number $a:= a_{n-1}a_{n-2} \cdots a_3a_2a_1a_0$, we can write $a_{n-1}a_{n-2} \cdots a_2 \cdot 100 + a_1a_0$ so that $a \equiv a_{n-1}a_{n-2} \cdots a_2 \cdot 100 + a_1a_0 \equiv a_1a_0 \mod 100$. For example, $175892 \equiv 175800 + 92 \equiv 1758 \cdot 100 + 92 \equiv 92 \mod 100$. Therefore, the last two digits of an integer are its reduction modulo 100. Observe that\dots
	\[
	\begin{aligned}
	13^0&\equiv 1 \mod 100 &\qquad 13^8&\equiv 61^2 \equiv 3721 \equiv 21 \mod 100 \\
	13^1&\equiv 13 \mod 100 & 13^{16}&\equiv 21^2 \equiv 441 \equiv 41 \mod 100 \\
	13^2&\equiv 13^2 \equiv 169 \equiv 69 \mod 100 & 13^{32}&\equiv 41^2 \equiv 1681 \equiv 81 \mod 100 \\
	13^4&\equiv 69^2 \equiv 4761 \equiv 61 \mod 100 & 13^{64}&\equiv 81^2 \equiv 6561 \equiv 61 \mod 100
	\end{aligned}
	\]
But then we have\dots
	\[
	3^{100} \equiv 3^{64 + 32 + 4} \equiv 3^{64} \cdot 3^{32} \cdot 3^4 \equiv 61 \cdot 81 \cdot 61 \equiv 4941 \cdot 61 \equiv 41 \cdot 61 \equiv 2501 \equiv 1 \mod 100
	\]
Therefore, the last two digits of $13^{100}$ are $01$.  

\item We have\dots
	\[
	3^5 \equiv 3^2 \cdot 3^2 \cdot 3 \equiv 9 \cdot 9 \cdot 3 \equiv 4 \cdot 4 \cdot 3 \equiv 16 \cdot 3 \equiv 1 \cdot 3 \equiv 3 \mod 5
	\] 
But then $3^5 \equiv 3 \mod 5$. \pspace

\item We have $\gcd(a, n)= \gcd(8, 3)= 1$ so that 8 and 3 are coprime. Now $\phi(3)= 3 - 1= 2$. But then we have\dots
	\[
	8^{\phi(3)} \equiv 8^2 \equiv 2^2 \equiv 4 \equiv 1
	\]
But then $8^{\phi(3)} \equiv 1 \mod 3$. 
\end{enumerate}



\newpage



% Problem 2
\problem{10} Showing all your work, compute the following:
	\begin{enumerate}[(a)]
	\item Compute 147 modulo 3. 
	\item Compute 147 modulo 3 by writing $147= 1 \cdot 100 + 4 \cdot 10 + 7 \cdot 1$.
	\item Compute $a_2a_1a_0$ modulo 3 by writing $a_2a_1a_0= a_2 \cdot 100 + a_1 \cdot 10 + a_0 \cdot 1$. When is $a_2a_1a_0$ divisible by 3? Explain.
	\item Using the previous parts, give a necessary and sufficient condition for an integer to be divisible by 3. 
	\end{enumerate} 

\sol 
\begin{enumerate}[(a)]
\item Because we have $147= 3(49) + 0$, we have $147 \equiv 0 \mod 3$. Notice that $147 \equiv 0 \mod 3$ implies that 147 is divisible by 3. \pspace

\item Using the fact that $10 \equiv 1 \mod 3$, we have\dots
	\[
	\begin{aligned}
	147 &\equiv 1 \cdot 100 + 4 \cdot 10 + 7 \cdot 1 \\[0.3cm]
	&\equiv 1 \cdot 10^2 + 4 \cdot 10^1 + 7 \cdot 10^0 \\[0.3cm]
	&\equiv 1 \cdot 1^2 + 4 \cdot 1^1 + 7 \cdot 1^0 \\[0.3cm]
	&\equiv 1 + 4 + 7 \\[0.3cm]
	&\equiv 12 \\[0.3cm]
	&\equiv 0 \mod 3
	\end{aligned}
	\] \pspace

\item Using the fact that $10 \equiv 1 \mod 3$, we have\dots
	\[
	\begin{aligned}
	a_2a_1a_0&\equiv a_2 \cdot 100 + a_1 \cdot 10 + a_0 \cdot 1 \\[0.3cm]
	&\equiv a_2 \cdot 10^2 + a_1 \cdot 10^1 + a_0 \cdot 10^0 \\[0.3cm]
	&\equiv a_2 \cdot 1^2 + a_1 \cdot 1^1 + a_0 \cdot 1^0 \\[0.3cm]
	&\equiv a_2 + a_1 + a_0
	\end{aligned}
	\] 
Because $a_2a_1a_0$ is divisible by 3 if and only if $a_2a_1a_0 \equiv 0 \mod 3$. By the work above, $a_2a_1a_0$ is divisible by 3 if and only if $a_2 + a_1 + a_0 \equiv 0 \mod 3$, i.e. if and only if $a_2 + a_1 + a_0$ is divisible by 3. Therefore, a three digit number is divisible by 3 if and only if the sum of its digits is divisible by 3. 

\item We would predict using (c) that an integer is divisible by 3 if and only if the sum of its digits is divisible by 3. We can confirm this. If we have an $n$ digit number, say $a= \sum_{i=0}^{n-1} a_i \cdot 10^i$, then $a$ is divisible by 3 if and only if $a \equiv 0 \mod 3$. But this is\dots
	\[
	0 \equiv a \equiv \sum_{i=0}^{n-1} a_i \cdot 10^i \equiv \sum_{i=0}^{n-1} a_i \cdot 1^i \equiv \sum_{i=0}^{n-1} a_i= a_{n-1} + a_{n-2} + \cdots + a_1 + a_0
	\]
\end{enumerate}



\newpage



% Problem 3
\problem{10} Use the Chinese Remainder Theorem to solve the following system of linear congruences
	\[
	\begin{aligned}
	2x &\equiv 1 \mod 3 \\
	x - 3&\equiv 0 \mod 4 \\
	3x + 2&\equiv 4 \mod 5
	\end{aligned}
	\] \pspace

\sol First, we put this system of congruences into the form stated in the Chinese Remainder Theorem, i.e. a collection of congruences of the form $x \equiv a_i \mod n_i$. Observe\dots
	\[
	\begin{aligned}
	2x&\equiv 1 \mod 3 &\qquad x - 3&\equiv 0 \mod 4 &\qquad 3x + 2&\equiv 4 \mod 5 \\
	2^{-1} \cdot 2x&\equiv 2^{-1} \cdot 1 \mod 3 & x&\equiv 3 \mod 4 & 3x&\equiv 2 \mod 5 \\
	x&\equiv 2 \cdot 1 \mod 3 & & & 3^{-1} \cdot 3x&\equiv 3^{-1} \cdot 2 \mod 5 \\
	x&\equiv 2 \mod 3 & & & x&\equiv 2 \cdot 2 \mod 5 \\
	& & & & x&\equiv 4 \mod 5
	\end{aligned}
	\]
Because 3, 4, and 5 are coprime, the Chinese Remainder Theorem states that there is a unique solution modulo $M= \prod_i n_i= 3 \cdot 4 \cdot 5= 60$. The Chinese Remainder Theorem also states that an integer solution to this system is $x= \sum a_i N_i M_i$. We have\dots
	\[
	\begin{aligned}
	a_1&= 2 \\
	a_2&= 3 \\
	a_3&= 4 
	\end{aligned}
	\]
Also, we have\dots
	\[
	\begin{aligned}
	M_1&= M/n_1= 60/3= 20, \text{i.e. } M_1= 4 \cdot 5= 20 \\
	M_2&= M/n_2= 60/4= 15, \text{i.e. } M_2= 3 \cdot 5= 15 \\
	M_3&= M/n_3= 60/5= 12, \text{i.e. } M_3= 3 \cdot 4= 12
	\end{aligned}
	\]
Finally, $N_i:= M_i^{-1} \mod n_i$. Now observe\dots
	\[
	\begin{aligned}
	N_1&:= M_1^{-1} \equiv 20^{-1} \equiv 2^{-1} \equiv 2 \mod 3 \\
	N_2&:= M_2^{-1} \equiv 15^{-1} \equiv (-1)^{-1} \equiv -1 \equiv 3 \mod 4 \\
	N_3&:= M_3^{-1} \equiv 12^{-1} \equiv 2^{-1} \equiv 3 \mod 5
	\end{aligned}
	\]
We can check these: $20 \cdot 2 \equiv 2 \cdot 2 \equiv 4 \equiv 1 \mod 3$, $15 \cdot 3 \equiv 3 \cdot 3 \equiv 9 \equiv 1 \mod 4$, and $12 \cdot 3 \equiv 2 \cdot 3 \equiv 6 \equiv 1 \mod 5$. But then the solution to this system of congruences is\dots
	\[
	\begin{aligned}
	\sum_{i=1}^3 a_i N_i M_i &\equiv a_1N_1M_1 + a_2N_2M_2 + a_3N_3M_3 \\
	&\equiv 2 \cdot 20 \cdot 2 + 3 \cdot 15 \cdot 3 + 4 \cdot 12 \cdot 3 \\
	&\equiv 80 + 135 + 144 \\
	&\equiv 20 + 15 + 24 \\
	&\equiv 59 \mod 60
	\end{aligned}
	\]
We can check this solution:
	\[
	\begin{aligned}
	2 \cdot 59 \equiv 2 \cdot 2 \equiv 4 \equiv 1 \mod 3 \\
	59 - 3 \equiv 56 \equiv 0 \mod 4 \\
	3 \cdot 59 + 2 \equiv 3 \cdot 4 + 2 \equiv 12 + 2 \equiv 2 + 2 \equiv 4 \mod 5
	\end{aligned}
	\]



\newpage



% Problem 4
\problem{10} Show that there are no integer solutions to $x^3 + 7y^2= 5$. \pspace

\sol If there is a solution pair, $x, y$, to the equation $x^3 + 7y^2= 5$, then reducing both sides modulo 7, there must be a mod 7 solution pair, $\overline{x}$, $\overline{y}$. But reducing modulo 7, we have\dots
	\[
	5 \equiv \overline{x}^3 + 7 \overline{y}^2 \equiv \overline{x}^3 + 0 \cdot \overline{y}^2 \equiv \overline{x}^3
	\]
But then 5 is a cube modulo 7. However, observe\dots
	\[
	\begin{aligned}
	0^3 \equiv 0 \mod 7 \\
	1^3 \equiv 1 \mod 7 \\
	2^3 \equiv 8 \equiv 1 \mod 7 \\
	3^3 \equiv 3^2 \cdot 3 \equiv 9 \cdot 3 \equiv 2 \cdot 3 \equiv 6 \mod 7 \\
	4^3 \equiv 4^2 \cdot 4 \equiv 16 \cdot 4 \equiv 2 \cdot 4 \equiv 8 \equiv 1 \mod 7 \\
	5^3 \equiv 5^2 \cdot 5 \equiv 25 \cdot 5 \equiv 4 \cdot 5 \equiv 20 \equiv 6 \mod 7 \\
	6^3 \equiv 6^2 \cdot 6 \equiv 36 \cdot 6 \equiv 1 \cdot 6 \equiv 6 \mod 7
	\end{aligned}
	\]
But no cube modulo 7 is 5, i.e. 5 is not a cube modulo 7. Therefore, there is no solution modulo 7 so that there cannot be an integer solution pair $x, y$ to the original equation. 


\end{document}