\documentclass[11pt,letterpaper]{article}
\usepackage[lmargin=1in,rmargin=1in,tmargin=1in,bmargin=1in]{geometry}
\usepackage{../style/homework}
\usepackage{../style/commands}
\setbool{quotetype}{false} % True: Side; False: Under
\setbool{hideans}{false} % Student: True; Instructor: False

% -------------------
% Content
% -------------------
\begin{document}

\homework{12: Due 11/04}{Algebra is the intellectual instrument which has been created for rendering clear the quantitative aspects of the world.}{Alfred North Whitehead}

% Problem 1
\problem{10} Showing all your work, complete the following:
	\begin{enumerate}[(a)]
	\item Find the last digit of $3^{300}$.
	\item Find the last two digits of $13^{100}$.
	\item Fermat's Little Theorem states that if $p$ is prime, then $a^p \equiv a \mod p$. Verify this claim when $p= 5$ and $a= 3$. 
	\item A generalization of Fermat's Little Theorem states that $a^{\varphi(n)} \equiv 1 \mod n$ if $a$ is coprime to $n$, where $\varphi(n)$ is the Euler Phi function. Verify this claim when $p= 3$ and $a= 8$. 
	\end{enumerate}



\newpage



% Problem 2
\problem{10} Showing all your work, compute the following:
	\begin{enumerate}[(a)]
	\item Compute 147 modulo 3. 
	\item Compute 147 modulo 3 by writing $147= 1 \cdot 100 + 4 \cdot 10 + 7 \cdot 1$.
	\item Compute $a_2a_1a_0$ modulo 3 by writing $a_2a_1a_0= a_2 \cdot 100 + a_1 \cdot 10 + a_0 \cdot 1$. When is $a_2a_1a_0$ divisible by 3? Explain.
	\item Using the previous parts, give a necessary and sufficient condition for an integer to be divisible by 3. 
	\end{enumerate} \pspace

\sol 
\begin{enumerate}[(a)]
\item 
\item 
\item 
\item 
\end{enumerate}



\newpage



% Problem 3
\problem{10} Use the Chinese Remainder Theorem to solve the following system of linear congruences
	\[
	\begin{aligned}
	2x &\equiv 1 \mod 3 \\
	x - 3&\equiv 0 \mod 4 \\
	3x + 2&\equiv 4 \mod 5
	\end{aligned}
	\]



\newpage



% Problem 4
\problem{10} Show that there are no integer solutions to $x^3 + 7y^2= 5$. \pspace

\sol If there is a solution pair, $x, y$, to the equation $x^3 + 7y^2= 5$, then reducing both sides modulo 7, there must be a mod 7 solution pair, $\overline{x}$, $\overline{y}$. But reducing modulo 7, we have\dots
	\[
	5 \equiv \overline{x}^3 + 7 \overline{y}^2 \equiv \overline{x}^3 + 0 \cdot \overline{y}^2 \equiv \overline{x}^3
	\]
But then 5 is a cube modulo 7. However, observe\dots
	\[
	\begin{aligned}
	0^3 \equiv 0 \mod 7 \\
	1^3 \equiv 1 \mod 7 \\
	2^3 \equiv 8 \equiv 1 \mod 7 \\
	3^3 \equiv 3^2 \cdot 3 \equiv 9 \cdot 3 \equiv 2 \cdot 3 \equiv 6 \mod 7 \\
	4^3 \equiv 4^2 \cdot 4 \equiv 16 \cdot 4 \equiv 2 \cdot 4 \equiv 8 \equiv 1 \mod 7 \\
	5^3 \equiv 5^2 \cdot 5 \equiv 25 \cdot 5 \equiv 4 \cdot 5 \equiv 20 \equiv 6 \mod 7 \\
	6^3 \equiv 6^2 \cdot 6 \equiv 36 \cdot 6 \equiv 1 \cdot 6 \equiv 6 \mod 7
	\end{aligned}
	\]
But no cube modulo 7 is 5, i.e. 5 is not a cube modulo 7. Therefore, there is no solution modulo 7 so that there cannot be an integer solution pair $x, y$ to the original equation. 


\end{document}