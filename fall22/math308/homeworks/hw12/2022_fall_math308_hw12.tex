\documentclass[11pt,letterpaper]{article}
\usepackage[lmargin=1in,rmargin=1in,tmargin=1in,bmargin=1in]{geometry}
\usepackage{../style/homework}
\usepackage{../style/commands}
\setbool{quotetype}{false} % True: Side; False: Under
\setbool{hideans}{true} % Student: True; Instructor: False

% -------------------
% Content
% -------------------
\begin{document}

\homework{12: Due 11/04}{Algebra is the intellectual instrument which has been created for rendering clear the quantitative aspects of the world.}{Alfred North Whitehead}

% Problem 1
\problem{10} Showing all your work, complete the following:
	\begin{enumerate}[(a)]
	\item Find the last digit of $3^{300}$.
	\item Find the last two digits of $13^{100}$.
	\item Fermat's Little Theorem states that if $p$ is prime, then $a^p \equiv a \mod p$. Verify this claim when $p= 5$ and $a= 3$. 
	\item A generalization of Fermat's Little Theorem states that $a^{\varphi(n)} \equiv 1 \mod n$ if $a$ is coprime to $n$, where $\varphi(n)$ is the Euler Phi function. Verify this claim when $n= 3$ and $a= 8$. 
	\end{enumerate}



\newpage



% Problem 2
\problem{10} Showing all your work, compute the following:
	\begin{enumerate}[(a)]
	\item Compute 147 modulo 3. 
	\item Compute 147 modulo 3 by writing $147= 1 \cdot 100 + 4 \cdot 10 + 7 \cdot 1$.
	\item Compute $a_2a_1a_0$ modulo 3 by writing $a_2a_1a_0= a_2 \cdot 100 + a_1 \cdot 10 + a_0 \cdot 1$. When is $a_2a_1a_0$ divisible by 3? Explain.
	\item Using the previous parts, give a necessary and sufficient condition for an integer to be divisible by 3. 
	\end{enumerate}



\newpage



% Problem 3
\problem{10} Use the Chinese Remainder Theorem to solve the following system of linear congruences
	\[
	\begin{aligned}
	2x &\equiv 1 \mod 3 \\
	x - 3&\equiv 0 \mod 4 \\
	3x + 2&\equiv 4 \mod 5
	\end{aligned}
	\]



\newpage



% Problem 4
\problem{10} Show that there are no integer solutions to $x^3 + 7y^2= 5$. 


\end{document}