\documentclass[11pt,letterpaper]{article}
\usepackage[lmargin=1in,rmargin=1in,tmargin=1in,bmargin=1in]{geometry}
\usepackage{../style/homework}
\usepackage{../style/commands}
\setbool{quotetype}{true} % True: Side; False: Under
\setbool{hideans}{false} % Student: True; Instructor: False

% -------------------
% Content
% -------------------
\begin{document}

\homework{7: Due 09/29}{\,`Obvious' is the most dangerous word in mathematics.}{E.T. Bell}

% Problem 1
\problem{10} Determine whether each of the following relations is a function. If the relation is a function, determine its image.
	\begin{enumerate}[(a)]
	\item $\{ (x, y) \colon x, y \in \mathbb{Z}, y= x^2 + 5 \}$ as a relation from $\mathbb{Z}$ to $\mathbb{Z}$
	\item $\{ (x, y) \colon x, y \in \mathbb{R}, y= x^2 \}$ as a relation from $\mathbb{R}$ to $\mathbb{R}$
	\item $\{ (x, y) \colon x, y \in \mathbb{R}, y^2= x \}$ as a relation from $\mathbb{R}$ to $\mathbb{R}$
	\item $\{ (x, y) \colon x, y \in \mathbb{Z}, y= 2x + 3 \}$ as a relation from $\mathbb{Z}$ to $\mathbb{Z}$
	\item $\{ (x, y) \colon x, y \in \mathbb{R}, x^2 + y^2= 4 \}$ as a relation from $\mathbb{R}$ to $\mathbb{R}$
	\end{enumerate} \pspace

\sol 
\begin{enumerate}[(a)]
\item This relation is a function. For each $x$, there is precisely one associated $y$---namely, the one obtained by evaluating $y= x^2 + 5$ for a given $x$. The image of the function is the set $\{ x^2 + 5 \;|\; x \in \mathbb{Z} \}$. The graph, $\{ (x, y) \colon x, y \in \mathbb{Z}, y= x^2 + 5 \}$, is the set of lattice points on the parabola $y= x^2 + 5$. 

\item This relation is a function. For each $x$, there is precisely one associated $y$---namely, the one obtained by evaluating $y= x^2$ for a given $x$. The image of the function is the set $\{ x^2  \;|\; x \in \mathbb{R} \}$. The graph, $\{ (x, y) \colon x, y \in \mathbb{R}, y= x^2 \}$, is the set of points on the parabola $y= x^2$.  

\item This relation is not a function of $x$. For instance, if $x= 4$, observe that both $(-2)^2= 4$ and $2^2= 4$ so that each $x$ is not associated to a unique $y$. The relation is a function of $y$. For each $y$, there is precisely one associated $x$---namely, the one obtained by evaluating $x= y^2$ for a given $y$. The image of the function is the set $\{ y^2 \;|\; y \in \mathbb{R} \}$. The graph, $\{ (x, y) \colon x, y \in \mathbb{R}, x= y^2 \}$, is the set of points on the `sideways' parabola $x= y^2$.  

\item This relation is a function. For each $x$, there is precisely one associated $y$---namely, the one obtained by evaluating $y= 2x + 3$ for a given $x$. The image of the function is the set $\{ 2x + 3 \;|\; x \in \mathbb{Z} \}$. The graph, $\{ (x, y) \colon x, y \in \mathbb{Z}, y= 2x + 3 \}$, is the set of lattice points on the line $y= 2x + 3$. 

\item This relation is neither a function of $x$ nor $y$. For instance, if $x= 1$, then $1^2 + y^2= 4$ so that $y^2= 3$, i.e. $y= \pm \sqrt{3}$. But then $x= 1$ is associated to both $y= -\sqrt{3}$ and $y= \sqrt{3}$. Similarly,  if $y= 1$, then $x^2 + 1^2= 4$ so that $x^2= 3$, i.e. $x= \pm \sqrt{3}$. But then $y= 1$ is associated to both $x= -\sqrt{3}$ and $x= \sqrt{3}$. Therefore, this relation is neither a function of $x$ nor a function of $y$. The image of $\{ (x, y) \colon x, y \in \mathbb{Z}, x^2 + y^2= 4 \}$ as a subset of $\mathbb{R}^2$ is the circle of radius 2 centered at the origin. 
\end{enumerate}



\newpage



% Problem 2
\problem{10} Define $A= \{ 3, 6, 9 \}$ and $B= \{ 3x \colon x \in \mathbb{Z} \} - \{ x \in \mathbb{Z} \colon x \leq 0, x > 10 \}$. Let $f: A \to \mathbb{Z}$ be given by $f(x)= 2x + 1$ and $g: B \to \mathbb{Z}$ be defined by $g(x)= x^3 - 18x^2 + 101x - 161$. Show that $f= g$. \pspace

\sol We need to show that $f, g$ have the same domain, same codomain, and agree with each other 

First, observe that we have\dots
	\[
	\begin{aligned}
	B&= \{ 3x \colon x \in \mathbb{Z} \} - \{ x \in \mathbb{Z} \colon x \leq 0, x > 10 \} \\[0.3cm]
	&= \{ \ldots, -12, -9, -6, -3, 0, 3, 6, 9, 12, \ldots \}  - \{ \ldots, -5, -4, -3, -2, -1, 0, 11, 12, 13, 14, 15, \ldots \} \\[0.3cm]
	&= \{ 3, \; 6, \; 9 \} 
	\end{aligned}
	\]
Therefore, $A= B$. Trivially, we have $\mathbb{Z}= \mathbb{Z}$. Then $f$ and $g$ have the same domain and codomain. It only remains to show that the agree on every element in their domain. 
	\[
	\begin{aligned}
	f(3)&= 2(3) + 1= 6 + 1= 7 &\qquad g(3)&=  3^3 - 18(3^2) + 101(3) - 161=  27 - 162 - 303 - 161= 7 \\
	f(6)&= 2(6) + 1= 12 + 1= 13 & g(6)&=  6^3 - 18(6^2) + 101(6) - 161=  216 - 648 + 606 - 161= 13 \\
	f(9)&= 2(9) + 1= 18 + 1= 19 & g(9)&=  9^3 - 18(9^2) + 101(9) - 161= 729 - 1458 + 909 - 161= 19 
	\end{aligned}
	\]
Now because $f$ and $g$ have the same domain, same codomain, and agree on every element in their domain, we know that $f= g$. 



\newpage



% Problem 3
\problem{10} Let $f: \mathbb{N} \to \mathbb{R}$ be given by $f(n)= 1 - n$ and $g: \mathbb{N} \to \mathbb{R}$ be given by $g(n)= \frac{n}{n + 1}$. For each of the following, either find a rule for the given function or evaluate the given function:
	\begin{enumerate}[(a)]
	\item $(fg)(1)$
	\item $(f + g)(n)$
	\item $(g \circ f)(5)$
	\item $(6f)(-3)$
	\item $\left( \dfrac{f}{g} \right)(n)$
	\end{enumerate} \pspace

\sol 
\begin{enumerate}[(a)]
\item 
	\[
	(fg)(1)= f(1) \cdot g(1)= (1 - 1) \cdot \dfrac{1}{1 + 1}= 0 \cdot \dfrac{1}{2}= 0
	\] \pspace

\item 
	\[
	(f + g)(n)= f(n) + g(n)= 1 - n + \dfrac{n}{n + 1}= \dfrac{(1 - n)(n + 1)}{n + 1} + \dfrac{n}{n + 1}= \dfrac{1 - n^2}{n + 1} + \dfrac{n}{n + 1}= \dfrac{-n^2 + n - 1}{n + 1}
	\] \pspace

\item 
	\[
	(g \circ f)(5)= g(f(5))= g(1 - 5)= g(-4)= \dfrac{-4}{-4 + 1}= \dfrac{-4}{-3}= \dfrac{4}{3}
	\] \pspace

\item 
	\[
	(6f)(-3)= 6 f(-3)= 6 \cdot \big(1 - (-3) \big)= 6 \cdot (1 + 3)= 6 \cdot 4= 24
	\] \pspace

\item 
	\[
	\left( \dfrac{f}{g} \right)(n)= \dfrac{f(n)}{g(n)}= \dfrac{1 - n}{\frac{n}{n + 1}}= \dfrac{(1 - n)(n + 1)}{n}= \dfrac{1 - n^2}{n}
	\]
\end{enumerate}



\newpage



% Problem 4
\problem{10} Let $f: A \to \mathbb{R}$ be given by $f(x)= |x + 1|$, where $| \cdot |$ denotes the absolute value. For each of the following, find the image of $A$ under $f$---no justification is necessary: 
	\begin{enumerate}[(a)]
	\item $A= [1, 6]$
	\item $A= (-3, 4]$
	\item $A= \mathbb{N}$
	\item $A= \mathbb{Z}$
	\item $A= \mathbb{R}$
	\end{enumerate}

\sol Let $f(x)= |x|$. If $S$ is a set of real numbers, let $\pm |S|:= \{ \pm |s| \colon s \in S \}$. Observe that because $f(P)= \{ f(p) \colon p \in P \}$, if $P$ is a set of nonnegative real numbers, then $f(P)= P$. Moreover, because $f(N)= \{ f(n) \colon n \in N \}$, if $N$ is a set of negative real numbers, we know that $f(N)= \{ f(n) \colon n \in N \}= \{ f(|n|) \colon n \in N \}= f(|N|)= |N|$. But then given a set $S$ of real numbers, we can decompose $S= P \cup N$ into a set of nonnegative numbers, $P$, and negative numbers, $N$, respectively. But then we have $f(S)= f(P \cup N)= f(P) \cup f(N)= P \cup |N|$. 

\begin{enumerate}[(a)]
\item 
	\[
	f \big( [1,6] \big)= [1, 6]
	\] \pspace

\item 
	\[
	f \big( (-3, 4] \big)= f \big( (-3, 0) \cup [0, 4] \big)= f \big( (-3, 0) \big) \cup f \big( [0, 4] \big)= (0, 3) \cup [0, 4]= [0, 4]
	\] \pspace
	
\item 
	\[
	f(\mathbb{N})= \mathbb{N}
	\] \pspace

\item 
	\[
	f(\mathbb{Z})= f(\mathbb{Z}_{< 0} \cup \mathbb{Z}_{\geq 0})= f(\mathbb{Z}_{< 0}) \cup f(\mathbb{Z}_{\geq 0})= \mathbb{Z}_+ \cup \mathbb{Z}_{\geq 0}= \mathbb{Z}_{\geq 0}
	\] \pspace

\item 
	\[
	f(\mathbb{R})= f(\mathbb{R}_{< 0} \cup \mathbb{R}_{\geq 0})= f(\mathbb{R}_{< 0}) \cup f(\mathbb{R}_{\geq 0})= \mathbb{R}_{> 0} \cup \mathbb{R}_{\geq 0}= \mathbb{R}_{\geq 0}
	\]
\end{enumerate}


\end{document}