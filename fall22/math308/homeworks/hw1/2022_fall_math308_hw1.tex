\documentclass[11pt,letterpaper]{article}
\usepackage[lmargin=1in,rmargin=1in,tmargin=1in,bmargin=1in]{geometry}
\usepackage{../style/homework}
\usepackage{../style/commands}
\setbool{quotetype}{true} % True: Side; False: Under
\setbool{hideans}{true} % Student: True; Instructor: False

% -------------------
% Content
% -------------------
\begin{document}

\homework{1: Due 09/08}{I mean not homework. It's not work if you love it.}{Alex Dunphy, Modern Family}

% Problem 1
\problem{10} Determine if each of the following are propositions. If the example is a proposition, state its truth value with a brief justification. If the example is \textit{not} a proposition, briefly explain why:
	\begin{enumerate}[(a)]
	\item Have you been watching `The Rings of Power'?
	\item $|9 - 17| > 10$
	\item $x^2 + x - 2= 0$
	\item The novel \textit{Ulysses} was written by James Joyce. 
	\item The sixth digit of $e$ is 1.
	\end{enumerate}



\newpage



% Problem 2
\problem{10} For each of the following, either define appropriate primitive propositions (using $P$, $Q$, $R$, etc.) and write the `statement' using logical connectives, or give an English sentence for the given primitives and `translate' the logical `sentence' into an English sentence:
	\begin{enumerate}[(a)]
	\item Either he is lying and isn't coming, or we are at the wrong place. 
	\item $(P \wedge \neg Q) \to R$
	\item If you exercise and eat healthy, then you will live a long life.
	\item $P \vee (\neg P \wedge Q)$
	\end{enumerate}



\newpage



% Problem 3
\problem{10} Consider the following compound statement: $(P \vee \neg Q) \to (\neg P \wedge Q) \vee \neg Q$
	\begin{enumerate}[(a)]
	\item Determine whether the given compound statement is a tautology. Be sure to justify your response. 
	\item Using a truth table, show that the \textit{negation} of the given compound statement is logically equivalent to $P \wedge Q$. 
	\item Show that the \textit{negation} of the given compound statement is logically equivalent to $P \wedge Q$ by simplifying the given compound statement.
	\end{enumerate}



\newpage



% Problem 4
\problem{10} Consider the statement, ``if $x= 3$, then $x^2= 9$.''
	\begin{enumerate}[(a)]
	\item Determine the truth value of this statement with an explanation. 
	\item Rewrite the given statement by defining appropriate primitive propositions and logical connectives. 
	\item Find the negation, converse, and contrapositive of your result from (b).
	\item Rewrite your answers from (c) as English sentences. Then determine the truth value, with explanation, of each of the statements. 
	\end{enumerate}


\end{document}