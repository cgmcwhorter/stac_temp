\documentclass[11pt,letterpaper]{article}
\usepackage[lmargin=1in,rmargin=1in,tmargin=1in,bmargin=1in]{geometry}
\usepackage{../style/homework}
\usepackage{../style/commands}
\setbool{quotetype}{true} % True: Side; False: Under
\setbool{hideans}{false} % Student: True; Instructor: False

% -------------------
% Content
% -------------------
\begin{document}

\homework{1: Due 09/08}{I mean not homework. It's not work if you love it.}{Alex Dunphy, Modern Family}

% Problem 1
\problem{10} Determine if each of the following are propositions. If the example is a proposition, state its truth value with a brief justification. If the example is \textit{not} a proposition, briefly explain why:
	\begin{enumerate}[(a)]
	\item Have you been watching `The Rings of Power'?
	\item $|9 - 17| > 10$
	\item $x^2 + x - 2= 0$
	\item The novel \textit{Ulysses} was written by James Joyce. 
	\item The sixth decimal digit of $e$ is 1.
	\end{enumerate} \pspace

\sol 
\begin{enumerate}[(a)]
\item This is \textit{not} a proposition. This is a question, which is neither true nor false. The answer to the question might be `yes' or `no', which will have a definite truth value, e.g. if a person says, `Yes', but is lying then the answer `yes' would be false. However, the question itself is neither true nor false---though the response might be. Therefore, this is not a proposition. \pspace

\item This is a proposition. The inequality is either holds or does not; hence, the inequality is either true or false. This is the definition of a proposition. Now observe that $|9 - 17|= |-8|= 8$ so that $|9 - 17| > 10$ is false. \pspace

\item This is \textit{not} a proposition. Without knowing what $x$ is, we do not know if the left-hand side works out to be the required 0 to be equal to the right-hand side. For instance, if $x= 5$, the left-hand side is $25 + 5 - 2= 28 \neq 0$, whereas if $x= 1$, the left-hand side is $1 + 1 - 2= 0$. Therefore, this is not a proposition. \pspace

\item This is a proposition. The novel \textit{Ulysses} has a definite author(s), which either includes James Joyce or does not. Therefore, the given statement either is true or not. This makes the given statement a proposition. In fact, the statement is true as James Joyce wrote \textit{Ulysses}. \pspace

\item This is a proposition. Every digit of a given number is a definite integer between 0 and 9 (in base-10), inclusively. The sixth digit of $e$ then either is or is not 1---whether or not we know which it is. Therefore, the statement is either true or false and then hence a proposition. In fact, the statement is true because we have $e \approx 2.718281828\ldots$, so that the sixth decimal digit of $e$ is indeed one.
\end{enumerate}



\newpage



% Problem 2
\problem{10} For each of the following, either define appropriate primitive propositions (using $P$, $Q$, $R$, etc.) and write the `statement' using logical connectives, or give an English sentence for the given primitives and `translate' the logical `sentence' into an English sentence:
	\begin{enumerate}[(a)]
	\item Either he is lying and isn't coming, or we are at the wrong place. 
	\item $(P \wedge \neg Q) \to R$
	\item If you exercise and eat healthy, then you will live a long life.
	\item $P \vee (\neg P \wedge Q)$
	\end{enumerate} \pspace

\sol {\itshape Note: There are many possible solutions.} 
\begin{enumerate}[(a)]
\item Let $P$ be the proposition ``he is lying,'' $Q$ be the proposition ``he is coming,'' and $R$ be the proposition ``we are at the wrong place.'' Then we can write the given statement as $(P \wedge \neg Q) \vee R$. \pspace

\item Let $P$ be the proposition ``It is raining,'' $Q$ be the proposition ``I bring an umbrella,'' and $R$ be the proposition ``I get wet.'' Then $(P \wedge \neg Q) \to R$ is the statement, ``If it is raining and I do not bring an umbrella, then I get wet.'' \pspace

\item Let $P$ be the proposition ``you exercise,'' $Q$ be the proposition ``you eat healthy,'' and $R$ be the proposition ``you live a long life.'' Then we can write the given statement as $P \wedge Q \to R$. \pspace

\item Let $P$ be the proposition ``you study for the exam,'' and $Q$ be the proposition ``you fail the exam.'' Then $P \vee (\neg P \wedge Q)$ is the statement, ``You study for the exam, or you do not study for the exam and fail the exam.'' 
\end{enumerate}



\newpage



% Problem 3
\problem{10} Consider the following compound statement: $(P \vee \neg Q) \to (\neg P \wedge Q) \vee \neg Q$
	\begin{enumerate}[(a)]
	\item Determine whether the given compound statement is a tautology. Be sure to justify your response. 
	\item Using a truth table, show that the \textit{negation} of the given compound statement is logically equivalent to $P \wedge Q$. 
	\item Show that the \textit{negation} of the given compound statement is logically equivalent to $P \wedge Q$ by simplifying the given compound statement.
	\end{enumerate}\sol 
\begin{enumerate}[(a)]
\item We can show this using a truth table. We need only show that for every input $P$ and $Q$, the given proposition yields $T_0$: \par
	\begin{table}[!ht]
	\centering
	\begin{tabular}{cc||ccccc|c}
	$P$ & $Q$ & $\neg P$ & $\neg Q$ & $P \vee \neg Q$ & $\neg P \wedge Q$ & $(\neg P \wedge Q) \vee \neg Q$ & $(P \vee \neg Q) \to (\neg P \wedge Q) \vee \neg Q$ \\ \hline 
	$T$ & $T$ & $F$ & $F$ & $T$ & $F$ & $F$ & $F$ \\
	$T$ & $F$ & $F$ & $T$ & $T$ & $F$ & $T$ & $T$ \\
	$F$ & $T$ & $T$ & $F$ & $F$ & $T$ & $T$ & $T$ \\
	$F$ & $F$ & $T$ & $T$ & $T$ & $F$ &  $T$ & $T$
	\end{tabular}
	\end{table} \par
Because when $P \equiv T_0$ and $Q \equiv T_0$ gives $(P \vee \neg Q) \to (\neg P \wedge Q) \vee \neg Q \equiv F_0$, we know that the given statement is not a tautology. 

\item We can use the same truth table as in (a), and we only need to negate the entries for $(P \vee \neg Q) \to (\neg P \wedge Q) \vee \neg Q$ and add those for $P \wedge Q$ to see that the truth values align:
	\begin{table}[!ht]
	\centering
	\begin{tabular}{cc||c|c|c}
	$P$ & $Q$ & $(P \vee \neg Q) \to (\neg P \wedge Q) \vee \neg Q$ & $\neg \big( (P \vee \neg Q) \to (\neg P \wedge Q) \vee \neg Q \big)$ & $P \wedge Q$ \\ \hline 
	$T$ & $T$ & $F$ & $T$ & $T$ \\
	$T$ & $F$ & $T$ & $F$ & $F$ \\
	$F$ & $T$ & $T$ & $F$ & $F$ \\
	$F$ & $F$ & $T$ & $F$ & $F$
	\end{tabular}
	\end{table} \par
Therefore, from the truth table, we can see that $\neg \big( (P \vee \neg Q) \to (\neg P \wedge Q) \vee \neg Q \big)$ is logically equivalent to $P \wedge Q$ because they have the same truth value for every truth value input for $P$ and $Q$. 

\item We have\dots
	\[
	\begin{aligned}
	\neg \big( (P \vee \neg Q) \to (\neg P \wedge Q) \vee \neg Q \big)&\equiv (P \vee \neg Q) \wedge \neg \big( (\neg P \wedge Q) \vee \neg Q \big) \\[0.3cm]
	&\equiv (P \vee \neg Q) \wedge \big( \neg (\neg P \wedge Q) \wedge \neg (\neg Q) \big) \\[0.3cm]
	&\equiv (P \vee \neg Q) \wedge \big( (\neg(\neg P) \vee \neg Q) \wedge Q \big) \\[0.3cm]
	&\equiv (P \vee \neg Q) \wedge \big( ( P \vee \neg Q) \wedge Q \big) \\[0.3cm]
	&\equiv (P \vee \neg Q) \wedge \big( (P \wedge Q) \vee (\neg Q \wedge Q) \big) \\[0.3cm]	
	&\equiv (P \vee \neg Q) \wedge \big( (P \wedge Q) \vee F_0 \big) \\[0.3cm]	
	&\equiv (P \vee \neg Q) \wedge (P \wedge Q) \\[0.3cm]	
	&\equiv \big( P \wedge (P \wedge Q) \big) \vee \big( \neg Q \wedge (P \wedge Q) \big) \\[0.3cm]	
	&\equiv (P \wedge Q) \vee \big( P \wedge Q \wedge \neg Q \big) \\[0.3cm]	
	&\equiv (P \wedge Q) \vee F_0 \\[0.3cm]
	&\equiv P \wedge Q	
	\end{aligned}
	\]
\end{enumerate}



\newpage



% Problem 4
\problem{10} Fix a real number $x$. Consider the statement, ``if $x= 3$, then $x^2= 9$.''
	\begin{enumerate}[(a)]
	\item Determine the truth value of this statement with an explanation. 
	\item Rewrite the given statement by defining appropriate primitive propositions and logical connectives. 
	\item Find the negation, converse, and contrapositive of your result from (b).
	\item Rewrite your answers from (c) as English sentences. Then determine the truth value, with explanation, of each of the statements. 
	\end{enumerate} \pspace

\sol 
\begin{enumerate}[(a)]
\item If $x \neq 3$, then $x= 3$ is false. But then no matter the truth value of $x^2= 9$, we know that the implication is true. Now assume that $x= 3$. Then we know that $x^2= 3^2= 9$. But then we know that we would have $T_0 \to T_0$, which is true. Therefore, the given statement is true. \pspace

\item Let $P$ be the proposition $x= 3$ and $Q$ be the proposition $x^2= 9$. Then the given statement can be written $P \to Q$. \pspace

\item The negation of $P \to Q$ is $\neg (P \to Q) \equiv P \wedge \neg Q$. The converse of the statement $P \to Q$ is $Q \to P$. The contrapositive of $P \to Q$ is $\neg Q \to \neg P$. \pspace

\item The negation was $P \wedge \neg Q$. We can write this as an English sentence as, ``$x= 3$ and $x^2 \neq 9$.'' If $x \neq 3$, then the first part of the `and' statement is false, so that the statement is false. If $x= 3$, then $x= 3$ so that $x^2= 9$ so that the second part of the `and' statement is false. Therefore, the negation is false. The converse was $Q \to P$. We can write this as an English sentence as, ``If $x^2= 9$, then $x= 3$.'' We know that this statement is also false. Consider the counterexample when $x= -3$. Then we have $x^2= (-3)^2= 9$ but $x \neq 3$. The contrapositive was $\neg Q \to \neg P$. We can write this as an English sentence as, ``If $x^2 \neq 9$, then $x \neq 3$.'' We know that this statement has to be true, because in (a) we found that the original statement was true. An implication and its contrapositive always have the same truth value. 
\end{enumerate}


\end{document}