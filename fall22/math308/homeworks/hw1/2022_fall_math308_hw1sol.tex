\documentclass[11pt,letterpaper]{article}
\usepackage[lmargin=1in,rmargin=1in,tmargin=1in,bmargin=1in]{geometry}
\usepackage{../style/homework}
\usepackage{../style/commands}
\setbool{quotetype}{true} % True: Side; False: Under
\setbool{hideans}{false} % Student: True; Instructor: False

% -------------------
% Content
% -------------------
\begin{document}

\homework{1: Due 09/08}{I mean not homework. It's not work if you love it.}{Alex Dunphy, Modern Family}

% Problem 1
\problem{10} Determine if each of the following are propositions. If the example is a proposition, state its truth value with a brief justification. If the example is \textit{not} a proposition, briefly explain why:
	\begin{enumerate}[(a)]
	\item Have you been watching `The Rings of Power'?
	\item $|9 - 17| > 10$
	\item $x^2 + x - 2= 0$
	\item The novel \textit{Ulysses} was written by James Joyce. 
	\item The sixth digit of $e$ is 1.
	\end{enumerate} \pspace

\sol
\begin{enumerate}[(a)]
\item The interrogative sentence, ``Have you been watching `The Rings of Power'?'' is neither true nor false. The answer to this question may either be `yes' or `no' and that yes/no may have a truth value of `true' or `false', but the question itself has neither truth value. The question is asking for an answer whose likely intended answer will have a truth value. But the question itself has no inherent truth value. \pspace

\item The inequality $|9 - 17| > 10$ is either true or false. Hence, this must be a proposition. Because $|9 - 17|= |-8|= 8 \not> 10$, we know that the statement is false. \pspace

\item The mathematical equation $x^2 + x - 2= 0$ is neither true nor false. For instance, if $x= 0$, then this is $-2 = 0$, which is false. If $x= 1$, we have $0= 0$, which is true. If $x=$ blue, then the left side does not even make sense. The equation  $x^2 + x - 2= 0$ will only make sense for some possible inputs for $x$ and of those for only some will the equation be true. Therefore, the equation $x^2 + x - 2= 0$ has no fixed true or false value. \pspace

\item The novel \textit{Ulysses} either was or was not written by James Joyce. Therefore, the declarative statement, ``The novel \textit{Ulysses} was written by James Joyce,'' is either true or false. Therefore, this sentence is a proposition. In fact, \textit{Ulysses} was written by James Joyce. \pspace

\item The sixth digit of $e$ is an integer from 0 to 9. But then the sixth digit of $e$ either is 1 or is not 1. Therefore, the declarative statement, ``The sixth digit of $e$ is 1,'' is either true or false. Therefore, this is a proposition. Because $e \approx 2.718281828\ldots$, so that the sixth digit is 1, the statement is true. 
\end{enumerate}



\newpage



% Problem 2
\problem{10} For each of the following, either define appropriate primitive propositions (using $P$, $Q$, $R$, etc.) and write the `statement' using logical connectives, or give an English sentence for the given primitives and `translate' the logical `sentence' into an English sentence:
	\begin{enumerate}[(a)]
	\item Either he is lying and isn't coming, or we are at the wrong place. 
	\item $(P \wedge \neg Q) \to R$
	\item If you exercise and eat healthy, then you will live a long life.
	\item $P \vee (\neg P \wedge Q)$
	\end{enumerate} \pspace

\sol
\begin{enumerate}[(a)]
\item Let $P$ be the proposition, ``He is lying,'' $Q$ be the proposition, ``He is coming,'' and $R$ be the proposition, ``We are at the right place.'' Then the statement, ``Either he is lying and isn't coming, or we are at the wrong place,'' can be translated into logical symbols as $(P \wedge \neg Q) \vee \neg R$. \pspace

\item Let $P$ be the proposition, ``You hear it raining,'' $Q$ be the proposition, ``The weather indicates it is sunny,'' and $R$ be the proposition, ``You bring an umbrella with you.'' Then $(P \wedge \neg Q) \to R$ is the statement, ``If you hear it raining and the weather indicates that it is not sunny, then you bring an umbrella with you.'' \pspace

\item Let $P$ be the proposition, ``You exercise,'' $Q$ be the proposition, ``You eat healthy,'' and $R$ be the proposition, ``You will live a long life.'' Then the declarative sentence, ``If you exercise and eat healthy, then you will live a long life,'' can be written in logical symbols as $P \wedge Q \to R$. \pspace

\item Let $P$ be the proposition, ``You clean your apartment,'' and $Q$ be the proposition, ``You do your laundry.'' Then the logical expression $P \vee (\neg P \wedge Q)$ can be written, ``You clean your apartment, or you don't clean your apartment and do your laundry.'' 
\end{enumerate}



\newpage



% Problem 3
\problem{10} Consider the following compound statement: $(P \vee \neg Q) \to (\neg P \wedge Q) \vee \neg Q$
	\begin{enumerate}[(a)]
	\item Determine whether the given compound statement is a tautology. Be sure to justify your response. 
	\item Using a truth table, show that the \textit{negation} of the given compound statement is logically equivalent to $P \wedge Q$. 
	\item Show that the \textit{negation} of the given compound statement is logically equivalent to $P \wedge Q$ by simplifying the given compound statement.
	\end{enumerate} 

\sol
\begin{enumerate}[(a)]
\item We can construct a truth table for the logical expression $(P \vee \neg Q) \to (\neg P \wedge Q) \vee \neg Q$: \par
	\begin{table}[!ht]
	\centering
	\begin{tabular}{cc||ccccc|c}
	$P$ & $Q$ & $\neg P$ & $\neg Q$ & $P \vee \neg Q$ & $\neg P \wedge Q$ & $ (\neg P \wedge Q) \vee \neg Q$ & $(P \vee \neg Q) \to (\neg P \wedge Q) \vee \neg Q$ \\ \hline
	T & T & F & F & T & F & F & F \\
	T & F & F & T & T & F & T & T \\
	F & T & T & F & F & T & T & T \\
	F & F & T & T & T & F & T & T
	\end{tabular}
	\end{table} \par
Because $(P \vee \neg Q) \to (\neg P \wedge Q) \vee \neg Q$ is not true for every input (true/false) of $P$ and $Q$, the proposition is not a tautology. \pspace

\item We know the truth value of $(P \vee \neg Q) \to (\neg P \wedge Q) \vee \neg Q$ from the previous part. We can then easily compute its negation and compare with $P \wedge Q$: \par
	\begin{table}[!ht]
	\centering
	\begin{tabular}{c|c||c|c|c}
	$P$ & $Q$ & $(P \vee \neg Q) \to (\neg P \wedge Q) \vee \neg Q$ & $\neg \big( (P \vee \neg Q) \to (\neg P \wedge Q) \vee \neg Q \big)$ & $P \wedge Q$ \\ \hline
	T & T & F & T & T \\
	T & F & T & F & F \\
	F & T & T & F & F \\
	F & F & T & F & F
	\end{tabular}
	\end{table} \par
Because the columns for $\neg \big( (P \vee \neg Q) \to (\neg P \wedge Q) \vee \neg Q \big)$ and $P \wedge Q$ match for each input of $P$ and $Q$, these propositions are logically equivalent. \pspace

\item We have\dots
	\[
	\begin{aligned}
	\neg \big( (P \vee \neg Q) \to (\neg P \wedge Q) \vee \neg Q \big) &\equiv (P \vee \neg Q) \wedge \neg \big( (\neg P \wedge Q) \vee \neg Q \big) \\
	&\equiv (P \vee \neg Q) \wedge \big( \neg( \neg P \wedge Q) \wedge \neg(\neg Q) \big) \\
	&\equiv (P \vee \neg Q) \wedge \big( ( \neg(\neg P) \vee \neg Q) \wedge Q \big) \\
	&\equiv (P \vee \neg Q) \wedge \big( (P \vee \neg Q) \wedge Q \big) \\
	&\equiv (P \vee \neg Q) \wedge \big( (P \wedge Q) \vee (\neg Q \wedge Q) \big) \\
	&\equiv (P \vee \neg Q) \wedge \big( (P \wedge Q) \vee F_0 \big) \\
	&\equiv (P \vee \neg Q) \wedge (P \wedge Q) \\ 
	&\equiv (P \wedge P \wedge Q) \vee (\neg Q \wedge P \wedge Q) \\
	&\equiv (P \wedge Q) \vee F_0 \\
	&\equiv P \wedge Q
	\end{aligned}
	\]
Therefore, the two expressions are logically equivalent. 
\end{enumerate}



\newpage



% Problem 4
\problem{10} Consider the statement, ``if $x= 3$, then $x^2= 9$.''
	\begin{enumerate}[(a)]
	\item Determine the truth value of this statement with an explanation. 
	\item Rewrite the given statement by defining appropriate primitive propositions and logical connectives. 
	\item Find the negation, converse, and contrapositive of your result from (b).
	\item Rewrite your answers from (c) as English sentences. Then determine the truth value, with explanation, of each of the statements. 
	\end{enumerate} \pspace

\sol
\begin{enumerate}[(a)]
\item If $x \neq 3$, then the statement is true regardless because $P \to Q$ is always true if $P$ is false. Now if $x= 3$, then we know that $x^2= 3^2= 9$ so that $x^2= 9$. But then if $x= 3$, the logical expression is equivalent to $T_0 \to T_0$ so that the statement is true. But then the statement ``if $x= 3$, then $x^2= 9$'' is always true. \pspace

\item Fix a real number $x$. Let $P$ be the proposition that $x= 3$ and let $Q$ be the proposition that $x^2= 9$. Then the implication ``if $x= 3$, then $x^2= 9$'' can be expressed as $P \to Q$. \pspace 

\item The negation of $P \to Q$ is $\neg (P \to Q) \equiv P \wedge \neg Q$. The converse of $P \to Q$ is $Q \to P$. The contrapositive of $P \to Q$ is $\neg Q \to \neg P$. \pspace

\item As a sentence, the negation $P \wedge \neg Q$ is written ``$x= 3$ and $x^2 \neq 9$.'' This is clearly false because when $x= 3$, we know $x^2= 9$ so that we cannot have $x^2 \neq 9$. [Notice this implies that the original statement ``if $x= 3$, then $x^2= 9$ is true because its negation is false.] \pspace

As a sentence, the converse $Q \to P$ is written ``if $x^2= 9$, then $x= 3$.'' This statement is false. As a counterexample, let $x= -3$. Then we do have $x^2= 9$ but $x \neq 3$. [Notice the converse being false \textit{does not} imply that the original statement was true---though it is. A statement and its converse have `unrelated' truth values.] \pspace

As a sentence, the contrapositive $\neg Q \to \neg P$ is written ``if $x^2 \neq 9$, then $x \neq 3$.'' Because the original statement is true, we know that this statement is true because a statement and its contrapositive have the same truth value. As a proof of this statement, suppose that $x^2 \neq 9$ but $x= 3$, i.e. assume the statement is false. But this is a contradiction because $x^2= 3^2= 9$. Alternatively, assume $x^2 \neq 9$. Then we have $x^2 - 9 \neq 0$ which is equivalent to $(x - 3)(x + 3) \neq 0$. This implies that $x \neq -3$ and $x \neq 3$. 
\end{enumerate}


\end{document}