\documentclass[11pt,letterpaper]{article}
\usepackage[lmargin=1in,rmargin=1in,tmargin=1in,bmargin=1in]{geometry}
\usepackage{../style/homework}
\usepackage{../style/commands}
\setbool{quotetype}{false} % True: Side; False: Under
\setbool{hideans}{false} % Student: True; Instructor: False

% -------------------
% Content
% -------------------
\begin{document}

\homework{5: Due 09/22}{To choose one sock from each of infinitely many pairs of socks requires the Axiom of Choice, but for shoes the Axiom is not needed.}{Bertrand Russell}

% Problem 1
\problem{10} For each of the sets described below, either give the set by enumerating all its elements (if possible) or give the set using set-builder notation. Also for each set, give an element and non-element of the set. 
	\begin{enumerate}[(a)]
	\item The set of integer multiples of 8. 
	\item The set of negative solutions to $(x - 4)(x + 1)(x + 6)= 0$.
	\item The set of nonnegative rational numbers less than 1. 
	\item The set of real numbers with a real-valued square root. 
	\item The set of integer cubes with absolute value less than 100.
	\end{enumerate} \pspace

\sol
\begin{enumerate}[(a)]
\item 
\item 
\item 
\item 
\item 
\end{enumerate}



\newpage



% Problem 2
\problem{10} For each of the sets given below, describe the sets in words. Also for each set, give an example of an element and non-element of the set.
	\begin{enumerate}[(a)]
	\item $\{ 2, 3, 5, 7, 11, 13, \ldots \}$
	\item $\{ \ldots, \frac{1}{8}, \frac{1}{4}, \frac{1}{2}, 1, 2, 4, 8, 16, \ldots \}$
	\item $\{ n \in \mathbb{N} \colon n^2= 30 - n \}$
	\item $\{ k \in \mathbb{Z} \colon (3k + 1)/5 \in \mathbb{Z} \}$
	\item $\{ n \in \mathbb{N} \colon (\exists k \in \mathbb{N})(n= 3k + 1) \}$
	\end{enumerate} \pspace

\sol
\begin{enumerate}[(a)]
\item This is the set of prime numbers. For instance, we have $2 \in \{ 2, 3, 5, 7, 11, 13, \ldots \}$, $11 \in \{ 2, 3, 5, 7, 11, 13, \ldots \}$, and $92305 \cdot 2^{16998} + 1 \in \{ 2, 3, 5, 7, 11, 13, \ldots \}$, whereas $1 \notin \{ 2, 3, 5, 7, 11, 13, \ldots \}$, $6 \notin \{ 2, 3, 5, 7, 11, 13, \ldots \}$, and $



MULTIPLE OF 9



\item 
\item 
\item 
\item 
\end{enumerate}



\newpage



% Problem 3
\problem{10} Define the following sets:
	\[
	\begin{aligned}
	A&= \{ 1, \; 2, \; 3, \; 4, \; 5, \; 6, \; 7, \; 8, \; 9, \; 10 \} \\
	B&= \{ 2, \; 4, \; 6 , \; 8, \; 10 \} \\
	C&= \{ 1, \; 3, \; 5, \; 7, \; 9 \} \\
	D&= \{ 2, \; 3, \; 5, \; 7 \} \\
	E&= \{ 1, \; 2, \; 4 , \; 8, \; 10 \} \\
	F&= \{ 3, \; 5, \; 8 , \; 9 , \; 10 \}
	\end{aligned}
	\]
Consider each of the sets above as coming from the universal set $\mathcal{U}:= A$. Compute the following:
	\begin{enumerate}[(a)]
	\item $D^c$
	\item $B \cup C$
	\item $C \cup (B \cap D)$
	\item $E \setminus F$
	\item $E \Delta F$
	\item $(B \cup C)^c$
	\end{enumerate} \pspace

\sol
\begin{enumerate}[(a)]
\item 
	\[
	D^c= \{ 1,\; 4,\; 6,\; 8,\; 9,\; 10 \} 
	\]

\item 
	\[
	B \cup C= \{ 1,\; 2,\; 3,\; 4,\; ,\; 5,\; 6,\; 7,\; 8,\; 9,\; 10 \}= A
	\]

\item 
	\[
	\hspace{-0.5cm} C \cup (B \cap D)= \{ 1,\; 3,\; 5,\; 7,\; 9 \} \cup \big( \{ 2,\; 4,\; 6,\; 8,\; 10 \} \cap \{ 2,\; 3,\; 5,\; 7 \} \big)= \{ 1,\; 3,\; 5,\; 7,\; 9 \} \cup \{ 2 \}= \{ 1,\; 2,\; 3,\; 5,\; 7,\; 9 \} 
	\]

\item 
	\[
	E \setminus F= \{ 1,\; 2,\; 4 \} 
	\]

\item 
	\[
	E \Delta F= \{ 1,\; 2,\; 3,\; 4,\; 5,\; 9 \}
	\]

\item  
	\[
	(B \cup C)^c= \big( \{ 2,\; 4,\; 6,\; 8,\; 10 \} \cup \{ 1,\; 3,\; 5,\; 7,\; 9 \} \big)^c= \{ 1,\; 2,\; 3,\; 4,\; 5,\; 6,\; 7,\; 8,\; 9,\; 10 \}^c= \varnothing
	\]
\end{enumerate}



\newpage



% Problem 4
\problem{10} Let the universal set of discourse be the set of integers. Define the following sets:
	\[
	\begin{aligned}
	A&= \text{set of even integers} \\
	B&= \text{set of odd integers} \\
	C&= \text{set of prime integers} \\
	D&= \text{set of square integers} \\
	E&= \text{set of nonnegative integers} \\
	F&= \text{set of positive integers} \\
	G&= \text{set of integers strictly between 0 and 20} \\
	H&= \text{set of integers that are a multiple of 5}
	\end{aligned}
	\]
Compute the sets below. When giving your solution, either enumerate all the elements of the resulting set (if possible), give the set using set-builder notation, or give the set using some `standard' notation. 
	\begin{enumerate}[(a)]
	\item $B^c$
	\item $A \cup B$
	\item $A \cap C$
	\item $B \cap C$
	\item $G - D$	
	\item $E \Delta F$
	\item $C \cap H$
	\item $D \cap E^c$
	\item $D^c$
	\end{enumerate} \pspace

\sol
\begin{enumerate}[(a)]
\item 
\item 
\item 
\item 
\item 
\item 
\item 
\item 
\item 
\end{enumerate}
	

\end{document}