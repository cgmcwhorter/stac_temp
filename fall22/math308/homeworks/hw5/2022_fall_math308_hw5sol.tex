\documentclass[11pt,letterpaper]{article}
\usepackage[lmargin=1in,rmargin=1in,tmargin=1in,bmargin=1in]{geometry}
\usepackage{../style/homework}
\usepackage{../style/commands}
\setbool{quotetype}{false} % True: Side; False: Under
\setbool{hideans}{false} % Student: True; Instructor: False

% -------------------
% Content
% -------------------
\begin{document}

\homework{5: Due 09/22}{To choose one sock from each of infinitely many pairs of socks requires the Axiom of Choice, but for shoes the Axiom is not needed.}{Bertrand Russell}

% Problem 1
\problem{10} For each of the sets described below, either give the set by enumerating all its elements (if possible) or give the set using set-builder notation. Also for each set, give an element and non-element of the set. 
	\begin{enumerate}[(a)]
	\item The set of integer multiples of 8. 
	\item The set of negative solutions to $(x - 4)(x + 1)(x + 6)= 0$.
	\item The set of nonnegative rational numbers less than 1. 
	\item The set of real numbers with a real-valued square root. 
	\item The set of integer cubes with absolute value less than 100.
	\end{enumerate} 

\sol 
\begin{enumerate}[(a)]
\item The integer multiples of 8 can be constructed by\dots
	\[
	\{ n \colon (\exists k \in \mathbb{Z})(n= 8k) \}= \{ 8k \colon k \in \mathbb{Z} \}
	\]

\item The set of negative solutions to $(x - 4)(x + 1)(x + 6)= 0$ can be constructed by 
	\[
	\{ x \in \mathbb{R} \colon (x - 4)(x + 1)(x + 6)= 0, x < 0 \}
	\]
However, we can enumerate this set. If $x$ is a solution to $(x - 4)(x + 1)(x + 6)= 0$, then $x - 4= 0$, $x + 1= 0$, or $x + 6= 0$. But this implies that $x= 4$, $x= -1$, or $x= -6$, respectively, and one can easily verify that each are a solution. Therefore, the set of negative solutions to $(x - 4)(x + 1)(x + 6)= 0$ is\dots
	\[
	\{ -1, -6 \}
	\]

\item The set of nonnegative rational numbers less than 1 can be constructed by\dots
	\[
	\{ q \in \mathbb{Q} \colon 0 \leq q < 1 \}= \{ r \in \mathbb{R} \colon (\exists a)(\exists b)(a, b \in \mathbb{Z} \wedge b \neq 0 \wedge r= a/b) \wedge 0 \leq r < 1 \}
	\]

\item Let $r \in \mathbb{R}$. If $r < 0$, then $\sqrt{r}$ is complex but not real, i.e. $\sqrt{r} \in \mathbb{C} \setminus \mathbb{R}$. However, if $r \geq 0$, then $\sqrt{r} \in \mathbb{R}$. Alternatively, $r \in \mathbb{R}$ has a real-valued square root if there is a real number whose square is $r$. Therefore, the set of real numbers with a real-valued square root can be constructed by\dots
	\[
	\{ r \in \mathbb{R} \colon r \geq 0 \}= \{ r \in \mathbb{R} \colon (\exists s \in \mathbb{R})(r= s^2) \}
	\]

\item We know that $|k^3| < 100$ if and only if $-100 < k^3 < 100$ if and only if $-4.64159 \approx \sqrt[3]{-100} < k < \sqrt[3]{100} \approx 4.64159$. Then set of integer cubes with absolute value less than 100 can be constructed by\dots
	\[
	\hspace{-2.5cm} \{ n \in \mathbb{Z} \colon (\exists k \in \mathbb{Z})(n= k^3 \wedge |n| < 100) \}= \{ n \in \mathbb{Z} \colon (\exists k \in \mathbb{Z})(n= k^3 \wedge -100 < n < 100) \}=  \{ k^3 \colon k \in \mathbb{Z}, \sqrt[3]{-100} < k < \sqrt[3]{100} \}
	\]
However, we can enumerate this set. The only integers with $-4.64159 \approx \sqrt[3]{-100} < k < \sqrt[3]{100} \approx 4.64159$ are $k= -4$, $-3$, $-2$, $-1$, $0$, $1$, $2$, $3$, $4$. The cube of these numbers are $-64$, $-27$, $-8$, $-1$, $0$, $1$, $8$, $27$, $64$. Therefore, the set of integer cubes with absolute value less than 100 is $\{ -64, -27, -8, -1, 0, 1, 8, 27, 64 \}$. 
\end{enumerate}



\newpage



% Problem 2
\problem{10} For each of the sets given below, describe the sets in words. Also for each set, give an example of an element and non-element of the set.
	\begin{enumerate}[(a)]
	\item $\{ 2, 3, 5, 7, 11, 13, \ldots \}$
	\item $\{ \ldots, \frac{1}{8}, \frac{1}{4}, \frac{1}{2}, 1, 2, 4, 8, 16, \ldots \}$
	\item $\{ n \in \mathbb{N} \colon n^2= 30 - n \}$
	\item $\{ k \in \mathbb{Z} \colon (3k + 1)/5 \in \mathbb{Z} \}$
	\item $\{ n \in \mathbb{N} \colon (\exists k \in \mathbb{N})(n= 3k + 1) \}$
	\end{enumerate} \pspace

\sol
\begin{enumerate}[(a)]
\item The set $P:= \{ 2, 3, 5, 7, 11, 13, \ldots \}$ is the set of prime numbers. Observe that $2 \in P$, $3 \in P$, $17 \in P$, $2\,760\,727\,302\,517 \in P$, and $2^{82\,589\,933} - 1 \in P$ but $1 \notin P$, $4 \notin P$, $6 \notin P$, and $493\,949\,595\,303 \notin P$. \pspace

\item The set $T:= \{ \ldots, \frac{1}{8}, \frac{1}{4}, \frac{1}{2}, 1, 2, 4, 8, 16, \ldots \}$ is the set of integer powers of $2$. Observe that $1 \in T$, $2 \in T$, $\frac{1}{2} \in T$, $2^{253\,453} \in T$, and $\frac{1}{2^{642\,443}} \in T$ but $3 \notin T$, $0 \notin T$, $15 \notin T$, $-\frac{1}{2} \notin T$, and $-2 \notin T$. \pspace

\item The set $S:= \{ n \in \mathbb{N} \colon n^2= 30 - n \}$ is the set of natural number solutions to $n^2= 30 - n$. Observe that if $x^2= 30 - x$ then $x^2 + x - 30= 0$. But as $(x + 6)(x - 5)$, this implies that $x= -6$ or $x= 5$. But then we know that $\{ n \in \mathbb{N} \colon n^2= 30 - n \}= \{ 5 \}$. Observe that $5 \in S$ and $0 \notin S$, $18 \notin S$, and $-6 \notin S$. \pspace

\item The set $D:= \{ k \in \mathbb{Z} \colon (3k + 1)/5 \in \mathbb{Z} \}$ is the of integers $k$ such that $(3k + 1)/5$ is also an integer. Observe that $(3 \cdot -7 + 1)/5= -4$, $(3 \cdot -2 + 1)/5= -1$, $(3 \cdot 3 + 1)/5= 2$, and $(3 \cdot 8 + 1)/5= 5$, and also $(3 \cdot 0 + 1)/5= \frac{1}{5}$, $(3 \cdot 7 + 1)/5= \frac{22}{5}$, and $(3 \cdot -10 + 1)/5= -\frac{29}{5}$.  But then we have $-7 \in D$, $-2 \in D$, $3 \in D$, and $8 \in D$, and also $0 \notin D$, $7 \notin D$, and $-10 \notin D$. \pspace

\item The set $M:= \{ n \in \mathbb{N} \colon (\exists k \in \mathbb{N})(n= 3k + 1) \}$ is the set of natural numbers that are one more than a multiple of 3. Observe that $3(1) + 1= 4$, $3(2) + 1= 7$, and $3(5) + 1= 16$. But then $4 \in M$, $7 \in M$, and $16 \in M$, but $1 \notin M$, $5 \notin M$, and $18 \notin M$. 
\end{enumerate}



\newpage



% Problem 3
\problem{10} Define the following sets:
	\[
	\begin{aligned}
	A&= \{ 1, \; 2, \; 3, \; 4, \; 5, \; 6, \; 7, \; 8, \; 9, \; 10 \} \\
	B&= \{ 2, \; 4, \; 6, \; 8, \; 10 \} \\
	C&= \{ 1, \; 3, \; 5, \; 7, \; 9 \} \\
	D&= \{ 2, \; 3, \; 5, \; 7 \} \\
	E&= \{ 1, \; 2, \; 4, \; 8, \; 10 \} \\
	F&= \{ 3, \; 5, \; 8, \; 9, \; 10 \}
	\end{aligned}
	\]
Consider each of the sets above as coming from the universal set $\mathcal{U}:= A$. Compute the following:
	\begin{2enumerate}
	\item $D^c$
	\item $B \cup C$
	\item $C \cup (B \cap D)$
	\item $E \setminus F$
	\item $E \Delta F$
	\item $(B \cup C)^c$
	\end{2enumerate} \pspace

\sol
\begin{enumerate}[(a)]
\item 
	\[
	D^c= \{ 1, \; 4, \; 6, \; 8, \; 9, \; 10 \} 
	\] \pspace

\item 
	\[
	B \cup C=  \{ 2, \; 4, \; 6 , \; 8, \; 10 \} \cup  \{ 1, \; 3, \; 5, \; 7, \; 9 \}= \{ 1, \; 2, \; 3, \; 4, \; 5, \; 6, \; 7, \; 8, \; 9, \; 10 \}= A
	\] \pspace

\item 
	\[
	\begin{aligned}
	C \cup (B \cap D)&=  \{ 1, \; 3, \; 5, \; 7, \; 9 \} \cup \big( \{ 2, \; 4, \; 6 , \; 8, \; 10 \} \cap \{ 2, \; 3, \; 5, \; 7 \} \big) \\[0.3cm]
	&= \{ 1, \; 3, \; 5, \; 7, \; 9 \} \cup \{ 2 \} \\[0.3cm]
	&= \{ 1, \; 2, \; 3, \; 5, \; 7, \; 9 \}
	\end{aligned}
	\] \pspace

\item 
	\[
	E \setminus F= \{ 1, \; 2, \; 4, \; 8, \; 10 \} - \{ 3, \; 5, \; 8, \; 9, \; 10 \}= \{ 1, \; 2, \; 4 \}
	\] \pspace

\item 
	\[
	E \Delta F= \{ 1, \; 2, \; 4, \; 8, \; 10 \} \Delta \{ 3, \; 5, \; 8, \; 9, \; 10 \}= \{ 1, \; 2, \; 3, \; 4, \; 5, \; 9 \} 
	\] \pspace

\item 
	\[
	(B \cup C)^c= \{ 1, \; 2, \; 3, \; 4, \; 5, \; 6, \; 7, \; 8, \; 9, \; 10 \}^c= A^c= \varnothing
	\]
\end{enumerate}



\newpage



% Problem 4
\problem{10} Let the universal set of discourse be the set of integers. Define the following sets:
	\[
	\begin{aligned}
	A&= \text{set of even integers} \\
	B&= \text{set of odd integers} \\
	C&= \text{set of prime integers} \\
	D&= \text{set of square integers} \\
	E&= \text{set of nonnegative integers} \\
	F&= \text{set of positive integers} \\
	G&= \text{set of integers strictly between 0 and 20} \\
	H&= \text{set of integers that are a multiple of 5}
	\end{aligned}
	\]
Compute the sets below. When giving your solution, either enumerate all the elements of the resulting set (if possible), give the set using set-builder notation, or give the set using some `standard' notation. 
	\begin{2enumerate}
	\item $B^c$
	\item $A \cup B$
	\item $A \cap C$
	\item $B \cap C$
	\item $G - D$	
	\item $E \Delta F$
	\item $C \cap H$
	\item $D \cap E^c$
	\item $D^c$
	\end{2enumerate} 

\sol
\begin{enumerate}[(a)]
\item The elements of $B^c$ are the integers that are not in $B$, i.e. not odd. Therefore, the elements of $B^c$ are the even integers. We can give this as a set by\dots
	\[
	B^c= \{ n \colon (\exists k \in \mathbb{Z})(n= 2k) \}= \{ 2k \colon k \in \mathbb{Z} \}
	\]

\item The elements of $A \cup B$ are either even or odd integers. But every integer is either even or odd. Therefore, the union of all even and odd integers is the entire collection of integers, i.e. $A \cup B= \mathbb{Z}$. \pspace

\item The elements of $A \cap C$ are the integers that are both even and prime. However, any even number that is not 2 is divisible by 2 and another integer that is not $\pm 1$---which is not a prime integer. Therefore, the only element of $A \cap C$ is 2, i.e. $A \cap C= \{ 2 \}$. \pspace

\item The elements in $B \cap C$ are the integer which are odd and prime. This can be given in \textit{many} ways, e.g.
	\[
	\begin{aligned}
	B \cap C&= \{ n \in \mathbb{N} \colon \neg(\exists a \in \mathbb{Z})(\exists b \in \mathbb{Z})(a > 1 \wedge b > 1 \wedge n= ab) \wedge \neg(\exists k \in \mathbb{Z})(n= 2k) \} \\
	 &= \{ n \in \mathbb{N} \colon \neg(\exists a \in \mathbb{Z})(\exists b \in \mathbb{Z})(a > 1 \wedge b > 1 \wedge n= ab) \wedge \neg(\exists k \in \mathbb{Z})(n= 2k) \} \\
	 &= \{ n \in \mathbb{Z} \colon (\forall a \in \mathbb{Z})(\forall b \in \mathbb{Z})(n= ab \to a= 1 \vee b= 1) \wedge \neg(\exists k \in \mathbb{Z})(n= 2k) \} \\
	 &= \{ n \in \mathbb{N} \colon \neg(\exists a \in \mathbb{Z})(\exists b \in \mathbb{Z})(a > 1 \wedge b > 1 \wedge n= ab) \wedge (\exists k \in \mathbb{Z})(n= 2k + 1) \} \\
	 &= \vdots \\
	&= \{ p \in \mathbb{N} \colon p \text{ prime},\, p > 2 \} 
	\end{aligned}
	\] \pspace

\item The elements of $G \setminus D$ are the elements of $G$ that are not in $D$, i.e. the set of integers strictly between 0 and 20 that are not also square integers. The integers strictly between 0 and 20 are 1, 2, 3, \ldots, 19. The squares are 0, 1, 4, 9, 16, 25, \ldots. But then the set of integers strictly between 0 and 20 that are not square integers is\dots
	\[
	G - D= \{ 2, \; 3, \; 5, \; 6, \; 7, \; 8, \; 10, \; 11, \; 12, \; 13, \; 14, \; 15, \; 17, \; 18, \; 19 \}
	\] \pspace

\item The elements of $E \Delta F$ are the elements that are only in $E$ or $F$ but not both, i.e. the integers that are either nonnegative or positive but not both. But there are no integers that are both nonnegative and positive. Therefore, we know that $E \Delta F$ are the integers that are nonnegative or positive. But these are just the nonnegative integers, which we can give as a set by\dots
	\[
	\{ z \in \mathbb{Z} \colon z \geq 0 \}
	\] \pspace

\item The elements of $C \cap H$ are the elements that are in both $C$ and $H$, i.e. integers that are both prime and a multiple of 5. The primes are 2, 3, 5, 7, 11, 13, 17, 19, \ldots and the multiples of 5 are $\ldots, -15, -10, -5, 0, 5, 10, 15, \ldots$. But then it is clear that any multiple of 5---other than 5 itself---cannot be prime. Therefore, the only integer that is both prime and a multiple of 5 is 5 itself. Then we know that $C \cap H= \{ 5 \}$. \pspace

\item The elements of $D \cap E^c$ are the elements that are in $D$ and also not in $E$, i.e. the integers that are square but not nonnegative. If an integer is not nonnegative, i.e. $\neg (n \geq 0) \equiv n < 0$, then the integer is negative. However, if an integer is a square, then it is equal to the square of another integer. In particular, the square numbers are nonnegative. Therefore, a number cannot both be a square and be negative. This shows that\dots
	\[
	D \cap E^c= \varnothing
	\] \pspace

\item The elements of $D^c$ are the elements that are not in $D$, i.e. the integers that are not squares. But we can give this set by\dots
	\[
	\{ n \in \mathbb{Z} \colon \neg(\exists k \in \mathbb{Z})(n= k^2) \}= \{ n \in \mathbb{Z} \colon (\forall k \in \mathbb{Z})(n \neq k^2) \}
	\]
\end{enumerate}
	

\end{document}