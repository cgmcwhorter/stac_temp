\documentclass[11pt,letterpaper]{article}
\usepackage[lmargin=1in,rmargin=1in,tmargin=1in,bmargin=1in]{geometry}
\usepackage{../style/homework}
\usepackage{../style/commands}
\setbool{quotetype}{false} % True: Side; False: Under
\setbool{hideans}{true} % Student: True; Instructor: False

% -------------------
% Content
% -------------------
\begin{document}

\homework{9: Due 10/13}{It's fine to work on any problem, so long as it generates interesting mathematics along the way---even if you don’t solve it at the end of the day.}{Andrew Wiles}

% Problem 1
\problem{10} Suppose that you have a function $f: \mathbb{R} \to \mathbb{R}$ that is strictly increasing. 
	\begin{enumerate}[(a)]
	\item Explain why $f$ must be an injective function. 
	\item If $f$ is merely increasing, does $f$ have to be an injection? Explain why or give a counterexample. 
	\item Does $f$ have to be surjective? Explain why or give a counterexample. 
	\end{enumerate}



\newpage



% Problem 2
\problem{10} Consider the function $f: \mathbb{R}^{\geq 2} \to \mathbb{R}$ given by $f(x)= \sqrt{x - 2}$.
	\begin{enumerate}[(a)]
	\item Solve the equation $\sqrt{x - 2}= \sqrt{y - 2}$ for $y$. 
	\item Using your work in (a), explain why this shows that $f(x)$ is injective. 
	\item Is $f(x)$ surjective? If $f(x)$ is surjective, explain why. If $f(x)$ is not surjective, find an element of the codomain not in the image of $f(x)$. 
	\end{enumerate}



\newpage



% Problem 3
\problem{10} Let $A, B$ be nonempty sets. Find a bijective function from $A \times B$ to the set $B \times A$. Be sure to explain why your function is bijective. Does this mean that $A \times B$ and $B \times A$ are the same sets? Explain why or why not. 


\end{document}