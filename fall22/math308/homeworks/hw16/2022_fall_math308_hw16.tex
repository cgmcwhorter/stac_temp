\documentclass[11pt,letterpaper]{article}
\usepackage[lmargin=1in,rmargin=1in,tmargin=1in,bmargin=1in]{geometry}
\usepackage{../style/homework}
\usepackage{../style/commands}
\setbool{quotetype}{true} % True: Side; False: Under
\setbool{hideans}{true} % Student: True; Instructor: False

% -------------------
% Content
% -------------------
\begin{document}

\homework{16: Due 12/06}{Algebra is the metaphysics of arithmetic.}{John Ray}

% Problem 1
\problem{10} Let $\mathbf{u}, \mathbf{v} \in \mathbb{R}^4$ be defined by $\mathbf{u}= \begin{pmatrix} 1 \\ 0 \\ -3 \\ 2 \end{pmatrix}$ and $\mathbf{v}= \begin{pmatrix} 4 \\ -1 \\ 6 \\ -5 \end{pmatrix}$. Showing all your work, compute the following:
	\begin{enumerate}[(a)]
	\item $-6 \mathbf{u}$
	\item $\mathbf{v} - \mathbf{u}$
	\item $\mathbf{u} + 2 \mathbf{v}$
	\item $\mathbf{u} \cdot \mathbf{v}$
	\end{enumerate}



\newpage



% Problem 2
\problem{10} Define matrices $A, B, C$ as follows:
	\[
	A= \begin{pmatrix} 1 & 0 & -4 \\ -2 & 3 & 1 \end{pmatrix}, \qquad B= \begin{pmatrix} 0 & 2 & -2 \\ 5 & 1 & 4 \end{pmatrix}, \qquad C= \begin{pmatrix} 2 & 0 \\ -1 & 6 \\ 5 & 3 \end{pmatrix}
	\]
Showing all your work, compute the following:
	\begin{enumerate}[(a)]
	\item $4A$
	\item $A - B$
	\item $3A + B$
	\item $AC$
	\item $B^T$
	\end{enumerate}



\newpage



% Problem 3
\problem{10} Define matrices $A, B, C$ as follows:
	\[
	A= \begin{pmatrix} 1 & 0 & 2 \\ 0 & -1 & 3 \end{pmatrix}, \qquad B= \begin{pmatrix} 2 & -1 \\ 0 & 3 \end{pmatrix}, \qquad C= \begin{pmatrix} 0 & 1 & 0 \\ 1 & 0 & 0 \\ 0 & 0 & 1 \end{pmatrix}
	\]
Showing all your work and explaining your reasoning, answer the following:
        \begin{enumerate}[(a)]
        \item What is $B^2$?
        \item If $CA$ is defined, compute it. If not, explain why. 
        \item What is $a_{23}$? What is $b_{21}$?
        \item If $M= AC$, without explicitly computing $AC$, what is $m_{23}$?
        \end{enumerate}



\newpage



% Problem 4
\problem{10} If $A, B$ are matrices, is it true $(A + B)^2= A^2 + 2AB + B^2$? If so, explain why. If not, explain why not. 


\end{document}