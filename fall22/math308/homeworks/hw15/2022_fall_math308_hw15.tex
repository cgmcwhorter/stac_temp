\documentclass[11pt,letterpaper]{article}
\usepackage[lmargin=1in,rmargin=1in,tmargin=1in,bmargin=1in]{geometry}
\usepackage{../style/homework}
\usepackage{../style/commands}
\setbool{quotetype}{false} % True: Side; False: Under
\setbool{hideans}{true} % Student: True; Instructor: False

% -------------------
% Content
% -------------------
\begin{document}

\homework{15: Due 11/10}{I memorized the hexadecimal times tables when I was 14 writing machine code, okay? Ask me what 9 times F is. It's fleventy-five.}{Erlich Bachman, Silicon Valley}

% Problem 1
\problem{10} Showing all your work, convert the following numbers to base-10:
	\begin{enumerate}[(a)]
	\item $9_9$
	\item $121_3$
	\item $5F01$
	\item $1001_{17}$
	\end{enumerate}



\newpage



% Problem 2
\problem{10} Showing all your work, convert the following base-10 numbers numbers in the given base $b$:
	\begin{enumerate}[(a)]
	\item 15, $b= 7$
	\item 25, $b= 4$
	\item 88, $b= 2$
	\item 1400, $b= 11$
	\end{enumerate}



\newpage



% Problem 3
\problem{10} Showing all your work and without working in base-10, compute the following:
	\begin{enumerate}[(a)]
	\item $1001_2 + 1011_2$
	\item $101_2 - 11_2$
	\item $32_5 - 14_5$
	\item $1A \cdot 2B$
	\end{enumerate}



\newpage



% Problem 4
\problem{10} Suppose you have integers represented in a computer written using only 4-bit binary with the first bit reserved for the sign (1 representing a negative). Using the 2's complement method to find all the representations of the negative integers, give a table of the possible integer values and their binary pattern. 


\end{document}