\documentclass[11pt,letterpaper]{article}
\usepackage[lmargin=1in,rmargin=1in,bmargin=1in,tmargin=1in]{geometry}
\usepackage{style/quiz}
\usepackage{style/commands}

% -------------------
% Content
% -------------------
\begin{document}
\thispagestyle{title}


% Quiz 1
\quizsol \textit{True/False}: If $P$ is the proposition $6 < 5$ and $Q$ is the proposition, ``Earth is a planet,'' then the logical statement $P \to Q$ is false. \pspace

\sol The statement is \textit{false}. Recall that the truth table for $P \to Q$ is as follows: \par
	\begin{table}[!ht]
	\centering
	\begin{tabular}{c|c|c}
	$P$ & $Q$ & $P \to Q$ \\ \hline
	$T$ & $T$ & $T$ \\
	$T$ & $F$ & $F$ \\
	$F$ & $T$ & $T$ \\
	$F$ & $F$ & $T$
	\end{tabular}
	\end{table} \par
Here, $P$ is the proposition $P: 6 < 5$ and $Q$ is the proposition $Q$: ``Earth is a planet.'' It is clear that $P$ is false and $Q$ is true. But then examining the logic table above, we can see that $P \to Q$ is true. \pvspace{1.5cm}



% Quiz 2
\quizsol \textit{True/False}: $\neg (P \to \neg Q) \equiv P \wedge Q$ \pspace

\sol The statement is \textit{true}. To determine if two propositions are logically equivalent, one can either examine the truth table or apply logical rules to obtain one logical expression from the other. If we construct a truth table, we have\dots \par
	\begin{table}[!ht]
	\centering
	\begin{tabular}{c|c||c|c|c||c}
	$P$ & $Q$ & $\neg Q$ & $P \to \neg Q$ & $\neg (P \to \neg Q)$ & $P \wedge Q$ \\ \hline
	$T$ & $T$ & $F$ & $F$ & $T$ & $T$ \\
	$T$ & $F$ & $T$ & $T$ & $F$ & $F$ \\
	$F$ & $T$ & $F$ & $T$ & $F$ & $F$ \\
	$F$ & $F$ & $T$ & $T$ & $F$ & $F$
	\end{tabular}
	\end{table}
Because for each possible pair of choices for $P$ and $Q$ the outputs for $\neg (P \to \neg Q)$ and $P \wedge Q$ match, $\neg (P \to \neg Q) \equiv P \wedge Q$. Alternatively, we can transform one into the other by applying logical equivalences (recall $P \to Q \equiv \neg P \vee Q$ or $\neg (P \to Q) \equiv P \wedge \neg Q$):
	\[
	\neg (P \to \neg Q) \equiv \neg (\neg P \vee \neg Q) \equiv \neg (\neg P) \wedge \neg (\neg Q) \equiv P \wedge Q.
	\]  \pvspace{0.5cm}



% Quiz 3
\quizsol \textit{True/False}: The logic corresponding to the circuit shown below is the proposition:
	\[
	(\neg P \wedge Q) \vee \neg Q.
	\]
	\begin{figure}[!ht]
	\centering
	\includegraphics[width=0.74\textwidth]{images/circuit}
	\end{figure} \par

\newpage

\sol The statement is \textit{false}. We can trace through the circuit. We see that the current from $P$ passes through a NOT gate and we obtain $\neg P$. This then feeds into an AND gate along with $Q$ so that we obtain $\neg P \wedge Q$. The resulting current is then passed through a NOT gate, obtaining $\neg (\neg P \wedge Q)$. This finally reaches an OR gate---along with $Q$---to obtain $\neg (\neg P \wedge Q) \vee Q$. We can see a diagrammatic explanation below. \par
	\begin{figure}[!ht]
	\centering
	\includegraphics[width=0.74\textwidth]{images/circuit_sol}
	\end{figure} \pvspace{1.5cm}



% Quiz 4
\quizsol \textit{True/False}: Let the universe $\mathcal{U}$ be the set of real numbers and define $P(x)$ to be the predicate $P(x): x^2 + x - 4 \geq 0$. Then $(\forall x) \big(\neg P(x) \big)$ is true. \pspace

\sol The statement is \textit{false}. If $P(x): x^2 + x - 4 \geq 0$, then $\neg P(x): x^2 + x - 4 < 0$. But then $(\forall x) \big(\neg P(x) \big)$ is the statement, ``For all $x$, $x^2 + x - 4 < 0$.'' Now if $x= 1$, we have $\neg P(1) \colon 1^2 + 1 - 4 < 0$, i.e. $-2 < 0$, which is true. If $x= 0$, we have $\neg P(0) \colon 0^2 + 0 - 4 < 0$, i.e. $-4 < 0$, which is true. However, while $(\forall x) \big(\neg P(x) \big)$ is clearly true for \textit{some} (we found at least two), it is not true \textit{for all} $x$. As a counterexample, let $x= 10$. Then $\neg P(10) \colon 10^2 + 10 - 4 < 0$, which is $104 < 0$---clearly false. Therefore, $\neg P(x)$ is not true for all $x$. But then $(\forall x) \big(\neg P(x) \big)$ is false. \pvspace{1.5cm}




% Quiz 5
\quizsol \textit{True/False}: Let the domain of $x, y$ be the integers. Then $(\exists! x)(\forall y)(x + 2y= 5)$. \pspace

\sol The statement is \textit{false}. The logical proposition $(\exists! x)(\forall y)(x + 2y= 5)$ in words states, ``There exists a unique $x$ such that for all $y$, $x + 2y= 5$.'' Suppose that there were such a $x$, say $x_0$. Then we know that $x_0 + 2y= 5$ for all $y$. In particular, $x_0$ satisfies this equality when $y= 0$. But then we know that $x_0= 5$. But also, it must satisfy the equality when $x= 1$. But then $x_0 + 2= 5$ so that $x_0$. Then there is not a unique $x$ that works for all $y$! Therefore, the statement is false. Note that if we reverse the quantifiers, the statement is true: $(\forall y)(\exists! x)(x + 2y= 5)$. In this case, this is the statement, ``For all $y$, there exists a unique $x$ such that $x + 2y= 5$.'' If you were given any $y$, define $x_0:= 5 - 2y$. But then $x + 2y= (5 - 2y) + 2y= 5$. So there exists such an $x$. Is it unique? Well if there were two or more $x$ values that worked for some $y$, say two of them are $x_0$ and $\tilde{x}_0$, then we have $x_0 + 2y= 5= \tilde{x}_0 + 2y$. But then $x_0 + 2y= \tilde{x}_0 + 2y$. Subtracting $2y$, we have $x_0 = \tilde{x}_0$. Therefore, there can only be one such $x$. Because we have found one, we know that the statement that for all $y$, there exists a unique $y$ such that $x + 2y= 5$ is true. 





\newpage





% Quiz 5
\quizsol \textit{True/False}:  $\{ 1, 2 \} \subseteq \{ \varnothing, \{ 1 \}, \{ 2 \}, \{ 1, 2 \} \}$ \pspace

\sol The statement is \textit{false}. We know that $A \subseteq B$ if and only if for all $a \in A$, we have $a \in B$. We test every element of the set $\{ 1, 2 \}$. The first element is 1. However, $1 \notin \{ \varnothing, \{ 1 \}, \{ 2 \}, \{ 1, 2 \} \}$. [Note that $1 \notin \{ \varnothing, \{ 1 \}, \{ 2 \}, \{ 1, 2 \} \}$ but $\{ 1 \} \in \{ \varnothing, \{ 1 \}, \{ 2 \}, \{ 1, 2 \} \}$.] However, we do have $\{ 1, 2 \} \notin \{ \varnothing, \{ 1 \}, \{ 2 \}, \{ 1, 2 \} \}$. \pvspace{1.5cm}


% Quiz 6
\quizsol \textit{True/False}: $\displaystyle\bigcap_{n \in \mathbb{N}} \left( -\frac{1}{n}, \frac{1}{n} \right)= \varnothing$ \pspace

\sol The statement is \textit{false}. For $n= 1$, the set $(-\frac{1}{n}, \frac{1}{n})$ is the interval $(-1, 1)$. For $n= 2$, the set $(-\frac{1}{n}, \frac{1}{n})$ is the interval $(-\frac{1}{2}, \frac{1}{2})$. For $n= 3$, the set $(-\frac{1}{n}, \frac{1}{n})$ is the interval $(-\frac{1}{3}, \frac{1}{3})$. Note that $0$ is an element of all these sets. Generally, we have $0 \in (-\frac{1}{n}, \frac{1}{n})$ for all $n \in \mathbb{N}$. But then we know that $0 \in \displaystyle\bigcap_{n \in \mathbb{N}} \left( -\frac{1}{n}, \frac{1}{n} \right)$. This is sufficient to demonstrate that this is not empty. [Note that it is actually true that $\displaystyle\bigcap_{n \in \mathbb{N}} \left( -\frac{1}{n}, \frac{1}{n} \right)= \{ 0 \}$---though this takes more work to prove.] \pvspace{1.5cm}



% Quiz 7
\quizsol \textit{True/False}: Let $E(n)$ denote the relation from $\mathbb{N}$ to $\mathbb{Z}^{\geq 0}$ given by the rule that $E(n)$ is the number of positive even integers less than or equal to $n$. Then this relation is a function with $E(5)= 2$, i.e. 2 is in the image of 5, and 10 in the preimage of 5. \pspace

\sol The statement is \textit{false}. There are several claims here. First, the claim that $E(n): \mathbb{N} \to \mathbb{Z}^{\geq 0}$ is a function. Given some $n \in \mathbb{N}$, there is a single number of positive even integers $\leq n$. But then for every input for $E(n)$, there is only one possible output. Therefore, $E(n)$ is a function from $\mathbb{N}$ to $\mathbb{Z}^{\geq 0}$. For 2 to be in the image of 5, we need $E(5)= 2$. There are two positive even integers $\leq 5$ (namely, 2 and 4) so that 2 is in the image of 5. For 10 to be in the preimage of 5, we would have to have $E(10)= 5$. Note that there are 5 even integers $\leq 10$ (namely 2, 4, 6, 8, 10). Therefore, 10 is in the preimage of 5. \pvspace{1.5cm}




% Quiz 8
\quizsol \textit{True/False}: Let $f: X \to Y$ be a function. Then $f^{-1}$ will be a function if and only if the preimage set satisfies the following: $(\forall y \in \text{im } f)(\exists x \in X)(f^{-1}(y)= x)$. \pspace

\sol The statement is \textit{false}. Take for example the function $f: \mathbb{R} \to \mathbb{R}^{\geq 0}$ given by $f(x)= x^2$. For all $y \in \mathbb{R}^{\geq 0}$, there exists an $x \in \mathbb{R}$ such that $f(x)= y$, namely $\pm\sqrt{y}$. But if $y > 0$, then there are two possibilities: $+\sqrt{y}$ and $-\sqrt{y}$. But this function $f(x)$ has $f^{-1}$ with the property that $(\forall y \in \text{im }f)(\exists x \in X)(f^{-1}(y)= x)$. If we want $f^{-1}$ to be a function, we require $(\forall y \in \text{im }f)(\exists! x \in X)(f^{-1}(y)= x)$. 





\newpage












\end{document}