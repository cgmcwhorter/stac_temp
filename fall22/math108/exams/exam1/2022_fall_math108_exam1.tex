\documentclass[12pt,letterpaper]{exam}
\usepackage[lmargin=1in,rmargin=1in,tmargin=1in,bmargin=1in]{geometry}
\usepackage{../style/exams}

% -------------------
% Course & Exam Information
% -------------------
\newcommand{\course}{MAT 108: Exam 1}
\renewcommand{\term}{Fall -- 2022}
\newcommand{\examdate}{10/07/2022}
\newcommand{\timelimit}{85 Minutes}

\setbool{hideans}{true} % Student: True; Instructor: False

% -------------------
% Content
% -------------------
\begin{document}

\examtitle
\instructions{Write your name on the appropriate line on the exam cover sheet. This exam contains \numpages\ pages (including this cover page) and \numquestions\ questions. Check that you have every page of the exam. Answer the questions in the spaces provided on the question sheets. Be sure to answer every part of each question and show all your work. If you run out of room for an answer, continue on the back of the page --- being sure to indicate the problem number.} 
\scores
\bottomline
\newpage

% ---------
% Questions
% ---------
\begin{questions}

% Question 1
\newpage
\question Owen Scharck offers quick loans to his `friends' and neighbors. He charges a simple 13.2\% annual interest rate on his quick loans. But to be sure that his clients always pay something, he takes the interest on the loan up front. Anthony S. Muck takes out a \$4,000 loan from Owen for 5~months.
	\begin{parts}
	\part[3] What is the maturity on the loan that Anthony takes?
	\part[3] What is the discount for the loan that Anthony takes?
	\part[3] What are the proceeds on the loan that Anthony takes?
	\part[3] How much does Anthony pay in total for this loan?
	\part[3] If instead, Anthony took a loan out from a bank for this same period of time, what is the highest annual compounded continuously interest rate he can be offered to pay the same or less on the loan he takes from Owen?
	\end{parts}



% Question 2
\newpage
\question[10] Winney Millons has won a lottery jackpot of \$12.6~million (after taxes). She would like this money to last at least the next 35~years. She will deposit this money into a savings account that earns 2.91\% annual interest, compounded quarterly. At the end of each month, she will have money automatically transferred from her savings account into her checking account to spend. What is the most each monthly transfer can be (to the nearest cent)?



% Question 3
\newpage
\question[10] Vera Schrewd is comparing savings accounts at different banks. At Loan Wolves, they offer a savings account with a 2.6\% annual interest, compounded semiannually. Whereas at Hedge Bets, they offer a savings account with a 2.59\% annual interest, compounded continuously. Which bank should Vera use? Be sure to fully justify your answer with numerical support and a written explanation.



% Question 4
\newpage
\question Suppose that this year the CPI was 305.648, up from last years CPI of 296.171. 
	\begin{parts}
	\part[3] What was the inflation rate from last year to this year (to the nearest tenth of a percent)?
	\part[3] If a good cost \$19.99 last year, how would you estimate that it costs this year (to the nearest cent)?
	\part[2] If the inflation rate continues at this level, how much would you estimate a good that costs \$9.99 now will cost in 5~years (to the nearest cent)?
	\part[2] Given your answer in (c), would would the estimated inflation rate be from now to 5~years from now (to the nearest tenth of a percent)?
	\end{parts}



% Question 5
\newpage
\question[15] Suppose that the standard deduction for federal taxable income for a single filer in a certain year was \$2,800 and that the tax brackets for that year are given below. Find the tax for a single filer taking the standard deduction that made \$85,000 that tax year (to the nearest cent). \par
	\begin{table}[!ht]
	\centering
	\begin{tabular}{|l|l|} \hline
	Tax Rate & Taxable Income \\ \hline \hline
	15\% & Up to \$26,250 \\ \hline
	28\% & \$26,251 -- \$63,550 \\ \hline
	31\% & \$63,551 -- \$132,600 \\ \hline
	36\% & \$132,601 -- \$288,350 \\ \hline
	39.6\% & $\geq$ \$288,351 \\ \hline
	\end{tabular}
	\end{table}



% Question 6
\newpage
\question[10] Ivan Oeder has taken out a loan to cover some medical expenses. The loan was for \$7,000 and was approved by a bank that set a 7.5\% annual interest, compounded monthly. If Mr.~Oeder has not paid any amount on the loan in the past 4~years, how much does he currently owe the bank?



% Question 7
\newpage
\question[10] Thomas Katt is purchasing a condo. The condo was listed at \$280,000, but he managed to negotiate a price of \$265,000. He takes out a 30~year mortgage with end of the month payments at a 6.7\% annual interest rate, compounded monthly. The bank requires him to put down at least 20\%, i.e. pay at least \$53,000 upfront. He has \$60,000 to put down but is considering taking \$7,000 and investing it on improvements instead. How much more will be pay in total on the condo (to the nearest dollar) if he puts \$53,000 down instead of \$60,000?



% Question 8
\newpage
\question[10] William Powers has had the same car for the past 7~years and the upkeep costs on the car are starting to add up. So Will begins saving for a used car to replace his car once it `goes.' The car he has in mind has a Blue Book value of \$8,200. He places \$7,500 into an account that earns 2.5\% annual interest rate, compounded continuously. How long until he has enough money to purchase the car?



% Question 9
\newpage
\question[10] Eaton Wright runs a dietary consulting business. He would like to expand his business and has started setting aside profits to save for new office spaces. So at the start of each month, he places \$3,400 into an account that earnest 1.8\% annual interest, compounded quarterly. If he has been saving money for the past 5~years, how much is currently in the account (to the nearest dollar)?


\end{questions}
\end{document}