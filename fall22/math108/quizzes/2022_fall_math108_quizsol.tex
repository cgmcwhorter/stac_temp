\documentclass[11pt,letterpaper]{article}
\usepackage[lmargin=1in,rmargin=1in,bmargin=1in,tmargin=1in]{geometry}
\usepackage{style/quiz}
\usepackage{style/commands}

% -------------------
% Content
% -------------------
\begin{document}
\thispagestyle{title}


% Quiz 1
\quizsol \textit{True/False}: Any function of one-variable which has a constant rate of change can be written in the form $f(x)= mx + b$ for some values $m$ and $b$. \pspace

\sol The statement is \textit{true}. Suppose the rate of change were 5 and the current value is 2. After one step in time, the value is $2 + 1(5)= 2 + 5= 7$. After another step in time, the value is $7 + 5= 12$, or $2 + 2(5)= 2 + 10= 12$. Generally, after $n$ steps, the value is $2 + n \cdot 5= 5n + 2$, which is a linear function. Generally, if we start with initial value $y_0$ and have a constant rate of change $m$, after $x$ steps, we have $y= y_0 + x \cdot m= mx + y_0$. This is a linear function with $f(x)= y$, $x= x$, $m= m$, and $b= y_0$. But then we see that `any' function which changes at a constant rate is a linear function. We know that a linear function $f(x)= mx + b$ has a constant rate of change---the slope $m$. Therefore, a function is linear if and only if it has a constant rate of change.  \pvspace{1.5cm}



% Quiz 2
\quizsol \textit{True/False}: A break-even point is the point where a revenue and cost function curve intersect. \pspace

\sol The statement is \textit{true}. Suppose that $(x, y)$ is a point in the plane where a revenue curve and a cost curve intersect, i.e. $(x, R(x) )$ and $(x, C(x) )$ are points on the revenue and cost curves, respectively. Then at level of production $x$, the revenue is $y$ and the cost is $y$, i.e. $R(x)= y$ and $C(x)= y$. But then at level of production $x$, we have $y= R(x)= C(x)$. Therefore at $x$, the profit is $P(x):= R(x) - C(x)= y - y= 0$. Therefore, $(x, y)$ is a break-even point. \pvspace{1.5cm}



% Quiz 3
\quizsol \textit{True/False}: If $C(q)$ is a \textit{linear} cost function, then $C(0)$ is the fixed costs and the slope of $C(q)$ is the marginal cost, i.e. the production cost per item. \pspace

\sol The statement is \textit{true}. Suppose that $C(q)$ is the cost of producing $q$ items. The fixed costs are the costs that are incurred regardless of the level of production. Because if $q > 0$ some product is produced, the level of production corresponding to no production is $q= 0$. But then the fixed costs (the costs at a level of production of zero) are $C(0)$. The marginal cost at a level of production $q$ is the additional cost of producing one additional item, i.e. $C(q + 1) - C(q)$. Suppose that $C(q)$ were linear; that is, $C(q)= mq + b$ for some $m, b$, where $m$ is the slope. Notice if we use the points $(q, C(q) )$ and $(q + 1, C(q + 1) )$ to compute the slope of $C(q)$, we obtain\dots
	\[
	m= \dfrac{\Delta C}{\Delta q}= \dfrac{C(q + 1) - C(q)}{(q + 1) - q}= \dfrac{C(q + 1) - C(q)}{1}= C(q + 1) - C(q).
	\]
But then the slope of the linear function $C(q)$ is the marginal cost of production. \pvspace{1.5cm}



% Quiz 4
\quizsol \textit{True/False}: If the CPI (Consumer Price Index) was 253.80 last year and it is 284.75 this year, then the inflation rate from last year to this year was 12.91\%. \pspace

\sol The statement is \textit{false}. The given two CPI's (measured from the same baseline), we can find the quotient of the new CPI and the old CPI and recognize it as a percentage increase (or decrease in the case of deflation). This percentage increase/decrease is the inflation/deflation rate. Therefore, we have $284.75/253.80 \approx 1.1219$, i.e. $284.75= 253.80(1.1219)= 253.80(1 + 0.1219)$. We can see that $1.1219= 1 + 0.1219$ represents a percent increase of 12.19\%---not 12.91\%. Note that some will say that the inflation rate should be calculated from the expression $\frac{\text{new CPI} - \text{old CPI}}{\text{old CPI}}$. Using this, we have $\frac{284.75 - 253.80}{284.75}= \frac{30.95}{253.80} \approx 0.1219 \squiggle 12.19\%$. These give equivalent answers in the case of inflation:
	\[
	\frac{\text{new CPI} - \text{old CPI}}{\text{old CPI}}= \frac{\text{new CPI}}{\text{old CPI}} - \frac{\text{old CPI}}{\text{old CPI}}= \frac{\text{new CPI}}{\text{old CPI}} - 1.
	\]
We already computed $\text{new CPI}/\text{old CPI}$ above---the minus one merely makes it simpler to recognize the percentage increase. However, in the case of deflation, one would need take the absolute value of this difference in the case of deflation---a disadvantage from the original method. [Unless one interprets a negative inflation rate as deflation.] \pvspace{1.5cm}



% Quiz 5
\quizsol \textit{True/False}: If someone invests \$1,500 at 5.4\% annual interest, compounded quarterly, then in 5~years the investment is worth $\$1500 \left( 1 + \dfrac{0.054}{4} \right)^5 \approx \$1604.02$. \pspace

\sol The statement is \textit{false}. If we have a principal of $P$ invested at an annual interest rate $r$, compounded discretely $k$ times per year for $t$ years, then the amount that $P$ has grown to after $t$ years is given by\dots
	\[
	P \left( 1 + \dfrac{r}{k} \right)^{kt}
	\]
We know that the principal is $P= 1500$ and that the annual interest rate is $r= 0.054$ . Because the interest is compounded quarterly (four times per year), we know also that $k= 4$. Finally, because $t= 5$, we have
	\[
	P \left( 1 + \dfrac{r}{k} \right)^{kt}= \$1500 \left( 1 + \dfrac{0.054}{4} \right)^{4 \cdot 5} \approx \$1500 (1.0135)^{20} \approx \$1500(1.30760045) \approx \$1961.40
	\]
The power on the given expression should have been 20---the number of times that the amount has been compounded. \pvspace{1.5cm}



% Quiz 6
\quizsol \textit{True/False}: At the start of each month, Carter deposits \$150 into an account that earns 0.3\% yearly annual interest, compounded monthly. To compute how much Carter will have in 8~years, one would need to calculate the future value for general ordinary annuity. \pspace

\sol The statement is \textit{false}. A general annuity is where the yearly payment and compound rates are not the same, while a simple annuity is where they occur at the same rate. Because the deposits and interest occur monthly, this is a simple annuity. An ordinary annuity is where the payments occur at the end of a period, while an annuity due are where payments occur at the start of the month. Carter deposits the month at the start of the month; therefore, this is an annuity due. Because we want to know how much money is accrued at the end of this time period, we are seeking a future value. Therefore, Carter needs to compute a future value for a simple ordinary annuity. 





\newpage





% Quiz 6
\quizsol \textit{True/False}: Suppose Rasheed takes out an amortized loan for \$57,000 at an annual interest rate of 6.7\%, compounded monthly. He makes monthly payments of \$1,752.19 for 3 years. The total interest he pays on this loan is $\$1752.19(36) - \$57000 = \$6078.84$. \pspace

\sol The statement is \textit{true}. If he pays \$1,752.19 per month for 3~years, he makes a total of $3 \cdot 12= 36$~payments. Then the total payment is $36 \cdot \$1752.19= \$63,078.84$. Because this repays both the loan and the interest, we know that the difference between the total amount paid and the loan value must be the interest. But then we have $\$63,078.84 - \$57,000= \$6,078.84$ in interest. \pvspace{1.5cm}




%Given a dataset $\{ (x_i, y_i) \}$, a simple linear regression line $\hat{y}= b_0 + b_1x$, is the line which minimizes the sum of the squares of the errors for this model. 


% Suppose you have a dataset $\{ (x_i, y_i ) \}$, where $x$ is the number of hours spent at the mall $y$ is the amount of money left in someone's wallet. If a linear regression model performed on this dataset and results in $\widehat{y}= 27.1 - 4.5x$. Then the model predicts that $x$ and $y$ are negatively correlated and that, on average, you spend \$4.50 every hour at the mall.





\end{document}