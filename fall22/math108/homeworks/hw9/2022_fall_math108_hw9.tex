\documentclass[11pt,letterpaper]{article}
\usepackage[lmargin=1in,rmargin=1in,tmargin=1in,bmargin=1in]{geometry}
\usepackage{../style/homework}
\usepackage{../style/commands}
\setbool{quotetype}{false} % True: Side; False: Under
\setbool{hideans}{true} % Student: True; Instructor: False

% -------------------
% Content
% -------------------
\begin{document}

\homework{9: Due 10/25}{I have never played the lottery in my life and never will. Voltaire described lotteries as a tax on stupidity. More specifically, I think, on innumeracy.}{Daniel Tammet}

% Problem 1
\problem{10} Let $X$ be a discrete random variable. We know that $P(X= -3)= 0.25$, $P(X= 0)= 0.30$, $P(X= 2)= 0.45$. 
	\begin{enumerate}[(a)]
	\item Given a random event, what is $P(X= -3 \text{ or } X= 2)$?
	\item Given a sequence of two independent random events, what is the probability that $X= 2$ both times?
	\item Find the average value for this random variable, i.e. find the expected value.
	\item Find the standard deviation for this random variable. 
	\end{enumerate}



\newpage



% Problem 2
\problem{10} Suppose you play a game where you roll a tetrahedral die with sides labeled one through four. The probabilities for which are (partially) given below. If you roll a 4, you win \$20. However, if you roll a 3, you win nothing; if you roll a 2, you must pay \$4; if you roll a 1, you must pay \$6. 
	\begin{table}[!ht]
	\centering 
	\begin{tabular}{|c||c|c|c|c|} \hline 
	$n$ & $1$ & $2$ & $3$ & $4$ \\ \hline 
	$P(n)$ & $\dfrac{3\rule{0pt}{2.9ex}}{10\rule[-1.3ex]{0pt}{0pt}}$ & \phantom{$\dfrac{00}{00}$} & $\dfrac{2}{10}$ & $\dfrac{1}{10}$ \\ \hline 
	\end{tabular}
	\end{table}

\begin{enumerate}[(a)]
\item Find $P(2)$. 
\item Find the probability that if you roll the die twice, lose money both times. 
\item Find the average amount you win per game. 
\item Should you play this game? Explain.
\end{enumerate}



\newpage



% Problem 3
\problem{10} Recently, the Mega Millions jackpot was \$1.28~billion. If you won and took the `cash option' (the smarter move), the payout is then \$747.2~million. After a mandatory 24\% federal tax withholding, you would finally walk away with 567.872~million. The odds of hitting the jackpot were 1 in 302~million (specifically, 1 in 302,575,350). A Mega Millions ticket costs \$2. Should you have purchased a ticket?


\end{document}