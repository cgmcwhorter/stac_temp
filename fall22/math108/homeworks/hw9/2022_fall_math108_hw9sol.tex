\documentclass[11pt,letterpaper]{article}
\usepackage[lmargin=1in,rmargin=1in,tmargin=1in,bmargin=1in]{geometry}
\usepackage{../style/homework}
\usepackage{../style/commands}
\setbool{quotetype}{false} % True: Side; False: Under
\setbool{hideans}{false} % Student: True; Instructor: False

% -------------------
% Content
% -------------------
\begin{document}

\homework{9: Due 10/25}{I have never played the lottery in my life and never will. Voltaire described lotteries as a tax on stupidity. More specifically, I think, on innumeracy.}{Daniel Tammet}

% Problem 1
\problem{10} Let $X$ be a discrete random variable. We know that $P(X= -3)= 0.25$, $P(X= 0)= 0.30$, $P(X= 2)= 0.45$. 
	\begin{enumerate}[(a)]
	\item Given a random event, what is $P(X= -3 \text{ or } X= 2)$?
	\item Given a sequence of two independent random events, what is the probability that $X= 2$ both times?
	\item Find the average value for this random variable, i.e. find the expected value.
	\item Find the standard deviation for this random variable. 
	\end{enumerate} \pspace

\sol 
\begin{enumerate}[(a)]
\item 
	\[
	P(X= -3 \text{ or } X= 2)= P(X= -3) + P(X= 2)= 0.25 + 0.45= 0.70
	\] \pspace

\item 
	\[
	P(X= 2 \text{ and } X= 2)= P(X= 2) \cdot P(X= 2)= 0.45 \cdot 0.45= 0.2025
	\] \pspace

\item 
	\[
	EX= \sum x \cdot P(X= x)= -3 \cdot 0.25 + 0 \cdot 0.30 + 2 \cdot 0.45= -0.75 + 0.00 + 0.90= 0.15
	\] \pspace

\item 
	\[
	\begin{aligned}
	\sigma^2&= \sum (x - EX)^2 \cdot P(X= x) \\[0.3cm]
	&= (-3 - 0.2025)^2 \cdot 0.25 + (0 - 0.2025)^2 \cdot 0.30 + (2 - 0.2025)^2 \cdot 0.45 \\[0.3cm]
	&= (-3.2025)^2 \cdot 0.25 + (-0.2025)^2 \cdot 0.30 + (1.7975)^2 \cdot 0.45 \\[0.3cm]
	&= 10.256 \cdot 0.25 + 0.0410063 \cdot 0.30 + 3.23101 \cdot 0.45 \\[0.3cm]
	&= 2.564 + 0.0123019 + 1.45395 \\[0.3cm]
	&= 4.03026
	\end{aligned}
	\]
Therefore, we have $\sigma = \sqrt{\sigma^2}= \sqrt{4.03026}= 2.00755$. 
\end{enumerate}



\newpage



% Problem 2
\problem{10} Suppose you play a game where you roll a tetrahedral die with sides labeled one through four. The probabilities for which are (partially) given below. If you roll a 4, you win \$20. However, if you roll a 3, you win nothing; if you roll a 2, you must pay \$4; if you roll a 1, you must pay \$6. 
	\begin{table}[!ht]
	\centering 
	\begin{tabular}{|c||c|c|c|c|} \hline 
	$n$ & $1$ & $2$ & $3$ & $4$ \\ \hline 
	$P(n)$ & $\dfrac{3\rule{0pt}{2.9ex}}{10\rule[-1.3ex]{0pt}{0pt}}$ & \phantom{$\dfrac{00}{00}$} & $\dfrac{2}{10}$ & $\dfrac{1}{10}$ \\ \hline 
	\end{tabular}
	\end{table}

\begin{enumerate}[(a)]
\item Find $P(2)$. 
\item Find the probability that if you roll the die twice, lose money both times. 
\item Find the average amount you win per game. 
\item Should you play this game? Explain.
\end{enumerate} \pspace

\sol 
\begin{enumerate}[(a)]
\item We know that the sum of the probabilities of the entire sample space must be 1. But then we have\dots
	\[
	\begin{aligned}
	1&= P(X= 1) + P(X= 2) + P(X= 3) + P(X= 4) \\[0.3cm]
	&= \dfrac{3}{10} + P(X= 2) + \dfrac{2}{10} + \dfrac{1}{10} \\[0.3cm]
	&= P(X= 2) + \dfrac{6}{10}
	\end{aligned}
	\]
Therefore, $P(X= 2)= 1 - \frac{6}{10}= \frac{4}{10}$. \pspace

\item First, observe you only lose money if you roll a 1 or a 2. We know that $P(X= 1 \text{ or } X= 2)= P(X= 1) + P(X= 2)= \frac{3}{10} + \frac{4}{10}= \frac{7}{10}$. Second, observe that the dice rolls are independent from each other. Finally, using these two facts, we have\dots
	\[
	P(\text{lose money twice})= \dfrac{7}{10} \cdot \dfrac{7}{10}= \dfrac{49}{100}
	\] \pspace

\item This is the expected value for this game. We have\dots
	\[
	EX= \sum x \cdot P(X= x)= -\$6 \cdot \dfrac{3}{10} + -\$4 \cdot \dfrac{4}{10} + \$0 \cdot \dfrac{2}{10} + \$20 \cdot \dfrac{1}{10}= -\dfrac{14}{10}= -\$1.40
	\] \pspace

\item Because the expected payout is negative, on average, you are losing money playing this game. For instance, playing the game 100 times, one would expect to lose $100 \cdot \$1.40= \$140$. Therefore, one should not play this game. 
\end{enumerate}



\newpage



% Problem 3
\problem{10} Recently, the Mega Millions jackpot was \$1.28~billion. If you won and took the `cash option' (the smarter move), the payout is then \$747.2~million. After a mandatory 24\% federal tax withholding, you would finally walk away with 567.872~million. The odds of hitting the jackpot were 1 in 302~million (specifically, 1 in 302,575,350). A Mega Millions ticket costs \$2. Should you have purchased a ticket? \pspace

\sol There are many factors that can/should be used in deciding whether to play the lottery. At least, on average, one could use the expected value for the lottery. We know that the probability of winning, $P(\text{win})$, is 1 in 302,575,350. Therefore, $P(\text{lose})$ is 302,575,349 in 302,575,350. For this record payout, the amount earned from winning is the end winnings minus the cost of the ticket, i.e. $567,872,000 - 2= 567,871,998$. The amount `won' from losing the lottery is the cost of the ticket, i.e. \$2. Therefore, on average, the expected payout from playing the lottery is\dots
	\[
	\begin{aligned}
	EX&= \sum x \cdot P(X= x) \\[0.3cm]
	&= \$567871998 \cdot \dfrac{1}{302575350} + -\$2 \cdot \dfrac{302575349}{302575350} \\[0.3cm]
	&\approx \$1.8768 - \$2 \\[0.3cm]
	&= -\$0.1232 \\[0.3cm]
	&\approx -\$0.12
	\end{aligned}
	\]
Even for this historically high lottery payout, the expected value is negative. This shows that, on average, even playing a lottery with this high a payout that one loses money. Then for more `reasonable' lottery payouts, the expected value will be even more negative. This shows that, on average, one should not play the lottery. 


\end{document}