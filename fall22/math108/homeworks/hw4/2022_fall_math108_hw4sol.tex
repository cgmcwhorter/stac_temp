\documentclass[11pt,letterpaper]{article}
\usepackage[lmargin=1in,rmargin=1in,tmargin=1in,bmargin=1in]{geometry}
\usepackage{../style/homework}
\usepackage{../style/commands}
\setbool{quotetype}{false} % True: Side; False: Under
\setbool{hideans}{false} % Student: True; Instructor: False

% -------------------
% Content
% -------------------
\begin{document}

\homework{4: Due 09/20}{I don't know if anybody's ever told you that half the time this business comes down to, `I don't like that guy.'\,}{Roger Sterling, Mad Men}

% Problem 1
\problem{10} Levy Tate is tracking the CPI to measure and predict inflation. The CPI last year was 296.17, while this year it 305.86. 
	\begin{enumerate}[(a)]
	\item What was the inflation rate from last year to this year?
	\item If the inflation rate remains constant from this year to next year, what will be the CPI that Levy should hope to predict for next year?
	\item Assuming the inflation rate is constant over the next year, if a certain good cost \$15.99 this year, what should Levy predict that it costs next year according to this data?
	\end{enumerate} \pspace

\sol
\begin{enumerate}[(a)]
\item We have\dots
	\[
	\dfrac{305.86}{296.17} \approx 1.03272= 1 + 0.03271
	\] \pspace
so that the inflation rate was 3.271\%. \pspace

\item If the inflation rate is the same, then by (a), we expect to see a 3.271\% increase in prices next year. We then expect (roughly) that the CPI next year will be 3.271\% higher than this year, i.e. a 3.271\% increase. But that means that a prediction for next year's CPI would be $305.86(1 + 0.03271)= 305.86(1.03271) \approx 315.87$. \pspace

\item If the inflation rate is the same, then by (a), we expect to see a 3.271\% increase in prices next year. We then expect (roughly) that the price next year will be 3.271\% higher than this year, i.e. a 3.271\% increase. But that means that a prediction for next year's price would be $\$15.99(1 + 0.03271)= \$15.99(1.03271) \approx \$16.51$. 
\end{enumerate}



\newpage



% Problem 2
\problem{10} Arty Fischel is attempting to compute his federal income tax for this year. Going online, as he is a single filer, he finds the following two charts to compute his federal income tax: \par
	\begin{table}[!ht]
	\centering
	\scalebox{0.90}{%
	\begin{tabular}{|l|l|} \hline
	Taxable Income & Tax Owed \\ \hline \hline
	\$0--\$10,275 & 10\% of taxable income \\ \hline
	\$10,276--\$41,775 & \$1,027.50 + 12\% amount over \$10,275 \\ \hline
	\$41,776--\$89,075 & \$4,807.50 + 22\% amount over \$41,775 \\ \hline
	\$89,076--\$170,050 & \$15,213.50 + 24\% amount over \$89,075 \\ \hline
	\$170,051--\$215,950 & \$34,647.50 + 32\% amount over \$170,050 \\ \hline
	\$215,951--\$539,900 & \$49,335.50 + 35\% amount over \$215,950 \\ \hline
	$\geq$ \$539,901 & \$162,718 + 37\% amount over \$539,900 \\ \hline
	\end{tabular}} \hfill
	%
	\scalebox{0.90}{%
	\begin{tabular}{|l|l|} \hline
	Tax Rate & Taxable Income \\ \hline \hline
	10\% & Up to \$10,275 \\ \hline
	12\% & \$10,276--\$41,775 \\ \hline
	22\% & \$41,776--\$89,075 \\ \hline
	24\% & \$89,076--\$170,050 \\ \hline
	32\% & \$170,051--\$215,950 \\ \hline
	35\% & \$215,951--\$539,900 \\ \hline
	37\% & $\geq$ \$539,901 \\ \hline
	\end{tabular}}
	\end{table}	

\begin{enumerate}[(a)]
\item Explain how these two charts convey the same information. Your explanation should include an explanation of how to use each chart to compute the federal income tax for a single individual with \$20,000 of taxable income.
\item If Arty made \$365,000 taxable income last year as a software engineer, how much will he pay in federal income tax.
\end{enumerate}

\sol
\begin{enumerate}[(a)]
\item The charts convey the same information because they both give the tax income brackets as well as the rate at which income within those brackets are taxed. The difference is that the first chart computes the tax already charged up to a given bracket, whereas the second one leaves it for you to compute. For instance, the \$1,027.50 in the first chart's second bracket comes from the tax on the first \$10,275 of taxable income, which is charged at 10\%. The tax already charged up to that bracket is then $\$10275(0.10)= \$1,027.50$. The additional term in the third bracket of the first chart comes from the fact that one would have $\$41775 - \$10275= \$31500$ of income in the previous bracket, taxed at 12\%, for a total tax of $\$31500(0.12)= \$3780$ in the previous bracket. But along with the \$1,027.50 charged in the first bracket's income, this gives a total of $\$1027.50 + 3780= \$4807.5$ of tax charged up to that bracket. This logic follows similarly for all the other brackets. Then to compute the tax for \$20,000 of income (falling into the second bracket) using the first chart, we compute the tax of the income at that bracket and add this to the tax from the previous: $\$\$1027.50 + 0.12(\$20000 - \$10275)= \$2194.50$. Using the second chart, we compute the tax at each bracket `by hand': $\$10275(0.10) + 0.12(\$20000 - \$10275)= \$2194.50$. 

\item Using the first chart, we have $\$49335.50 + 0.35(\$365000 - \$215950)= \$101503$. Using the second chart, we have\dots
	\[
	\hspace{-2.75cm}0.10(10275) + 0.12(41775 - 10275) + 0.22(89075 - 41775) + 0.24(170050 - 89075) + 0.32(215950 - 170050) + 0.35(365000 - 215950)
	\]
	\[
	0.10(10275) + 0.12(31500) + 0.22(47300) + 0.24(80975) + 0.32(45900) + 0.35(149050)
	\]
	\[
	1027.50 + 3780.00 + 10406.00 + 19434.00 + 14688.00 + 52167.50
	\]
	\[
	\$101,503
	\]
\end{enumerate}
	


\newpage



% Problem 3
\problem{10} Sue Flay is taking out a small business loan to open her dream bakery. The loan she takes out is for \$85,000 at a 2.89\% annual interest rate, compounded monthly.
	\begin{enumerate}[(a)]
	\item What is the effective interest rate for this loan?
	\item How much does she owe after 2~years?
	\item How long until Sue owes the bank \$150,000?
	\end{enumerate} \pspace

\sol Note that this is a discrete compounded interest problem. We have a principal value of $P= \$85000$, an annual interest rate of $r= 0.0289$, and that the interest is compounded 12~times per year $k= 12$.

\begin{enumerate}[(a)]
\item The nominal interest rate is 2.89\%, while the effective annual interest rate is\dots
	\[
	\hspace{-1cm} r_{\text{eff}}= \left(1 + \dfrac{r}{k} \right)^k - 1= \left(1 + \dfrac{0.0289}{12} \right)^12 - 1= 1.0024083333^{12} - 1= 1.0292858943 - 1= 0.0292858943 \approx 2.93\%
	\] \pspace

\item The amount she owes after 2~years is the future value of $P= 85000$. This is\dots
	\[
	\hspace{-1cm} F= P \left(1 + \dfrac{r}{k} \right)^{kt}= 85000 \left(1 + \dfrac{0.0289}{12} \right)^{12 \cdot 2}= 85000(1.0024083333^{24})= 85000(1.0594294515) \approx \$90,051.50
	\] \pspace

\item We are now wondering how long until she owes the bank \$150,000, i.e. how long until the future value is \$150,000. This is\dots
	\[
	t= \dfrac{\ln(F/P)}{k \ln \left(1 + \dfrac{r}{k} \right)}= \dfrac{\ln(150000/85000)}{12 \ln \left(1 + \dfrac{0.0289}{12} \right)}= \dfrac{\ln(1.76470588235)}{12 \ln(1.0024083333)}= \dfrac{0.5679840376}{0.028865254958} \approx 12.68 \text{ years}
	\]
\end{enumerate}



\newpage



% Problem 4
\problem{10} Brock Lee is open a savings account to have enough money for community college. He places \$2,500 in the account, which earns 0.13\% annual interest, compounded continuously.
	\begin{enumerate}[(a)]
	\item What is the effective interest rate for this account?
	\item How much is in his account after 4~years?
	\item If the cost of a year at the college is \$21,714, how much should have Lee placed in the account to have enough for his first full year at the college in four years?
	\end{enumerate} \pspace

\sol Note that this is a continuous compounding interest problem. We have a principal value of \$2,500 and an annual interest rate of 0.13\%. 

\begin{enumerate}[(a)]
\item The nominal interest rate is 0.13\%, while the effective annual interest rate is\dots
	\[
	r_{\text{eff}}= e^r - 1= e^{0.0013} - 1= 1.0013008454 - 1= 0.0013008454 \approx 0.13008\%
	\] \pspace

\item The amount in his account after 4~years is the future value of $P= 2500$. This is\dots
	\[
	F= Pe^{rt}= 2500 e^{0.0013 \cdot 4}= 2500 e^{0.0052}= 2500(1.005213543) \approx \$2,513.03
	\] \pspace

\item We are wondering how much he should have placed in the account initially, i.e. the principal value, so that after 4~years the amount in the account, $F$, is \$21,714. This is\dots
	\[
	P= \dfrac{F}{e^{rt}}= \dfrac{21714}{e^{0.0013 \cdot 4}}= \dfrac{21714}{e^{0.0052}}= \dfrac{21714}{1.00521354347} \approx \$21,601.38.
	\] 
\end{enumerate}



\newpage



% Problem 5
\problem{10} Ray Gunne is comparing loan rates at two different banks. The first offers loans at a 7.99\% annual interest rate, compounded semiannually, while the other offers a loan at 7.89\% annual interest rate, compounded continuously. Which loan should he take? Justify your answer completely. \pspace

\sol We can compare these different types of accounts either using doubling time or the effective interest. Obviously, because this is a loan, the longer the doubling time, the better the loan. The less the effective interest, the better the loan. We can compute the doubling time for both: \pspace
	\[
	\begin{aligned}
	t_{D, \text{Disc.}}&= \dfrac{\ln(2)}{k \ln(1 + r/k)}= \dfrac{\ln(2)}{2 \ln(1 + 0.0799/2)}= \dfrac{0.69314718056}{0.078345270148}= 8.8473 \text{ years} \\[0.3cm]
	t_{D, \text{Cont.}}&= \dfrac{\ln(2)}{r}= \dfrac{\ln(2)}{0.0789} \approx 8.785 \text{ years}
	\end{aligned}
	\] \pspace
Because the doubling time for the discrete compounded interest loan, at a rate of 7.99\% annual interest, compounded semiannually, is less than that for the continuous compounding loan, the discrete compounded interest loan at 7.99\% is the better loan. Similarly, computing effective interests, we have\dots \pspace
	\[
	\begin{aligned}
	r_{\text{eff}}&= \left(1 + \dfrac{r}{k} \right)^k - 1= \left(1 + \dfrac{0.0799}{2} \right)^2 - 1= 1.03995^2 - 1= 1.0814960025 - 1= 0.0814960025 \approx 8.15\% \\[0.3cm]
	r_{\text{eff}}&= e^r - 1= e^{0.0789} - 1= 1.08209610 - 1= 0.08209610 \approx 8.21\%
	\end{aligned}
	\] \pspace
Because the discrete compounded interest loan, at a rate of 7.99\%, has a lower effective interest rate than the continuous compounding loan, the discrete compounded interest loan, at a rate of 7.99\%, is the better loan. 



\newpage



% Problem 6
\problem{10} Sue Render wants to be able to save \$500 in the next two years by depositing \$400 in a savings account. She can either choose a savings account that compounds interest quarterly or continuously. For both types of savings account, find the interest rate on the savings account she would have to secure to have the \$500 after 2~years. Does her plan seem feasible? \pspace

\sol Suppose she puts the money into an account that compounds interest discretely at a rate of four times per year (quarterly) at an annual interest rate of $r$, in order to save \$500 from a principal value of \$400 in 2~years, we would have\dots
	\[
	\begin{aligned}
	F&= P \left(1 + \dfrac{r}{k} \right)^{kt} \\[0.3cm]
	\$500&= \$400 \left(1 + \dfrac{r}{4} \right)^{4 \cdot 2} \\[0.3cm]
	\$500&= \$400 \left(1 + \dfrac{r}{4} \right)^8 \\[0.3cm] 
	1.25&= \left(1 + \dfrac{r}{4} \right)^8 \\[0.3cm]
	\sqrt[8]{1.25}&= \sqrt[8]{\left(1 + \dfrac{r}{4} \right)^8} \\[0.3cm] 
	1.0282855942979&= 1 + \dfrac{r}{4} \\[0.3cm]
	0.0282855942978&= \dfrac{r}{4} \\[0.3cm]
	r&\approx 0.1131
	\end{aligned}
	\]
which is an interest rate of 11.31\%. If she instead puts the money into an account that compounds interest continuously at an annual interest rate of $r$, then to save \$500 from a principal value of \$400 in 2~years, we would have\dots 
	\[
	\begin{aligned}
	F&= Pe^{rt} \\[0.3cm]
	\$500&= \$400 e^{2r} \\[0.3cm]
	1.25&= e^{2r} \\[0.3cm]
	\ln(1.25)&= \ln(e^{2r}) \\[0.3cm]
	0.2231435513142&= 2r \\[0.3cm]
	r&\approx 0.1116
	\end{aligned}
	\]
which is an interest rate of 11.16\%. While both the 11.31\% interest rate for the discretely compounded interest account and the 11.16\% interest rate for the continuously compounded interest account are \textit{possible}, these are both very high interest rates for a savings account. Therefore, her plan to save \$500 by depositing \$400 into a savings account for 2~years is likely not feasible. 


\end{document}