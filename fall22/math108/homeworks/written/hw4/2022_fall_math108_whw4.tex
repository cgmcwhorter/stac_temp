\documentclass[11pt,letterpaper]{article}
\usepackage[lmargin=1in,rmargin=1in,tmargin=1in,bmargin=1in]{geometry}
\usepackage{../style/homework}
\usepackage{../style/commands}
\setbool{quotetype}{false} % True: Side; False: Under
\setbool{hideans}{true} % Student: True; Instructor: False

% -------------------
% Content
% -------------------
\begin{document}

\homework{4: Due 09/20}{I don't know if anybody's ever told you that half the time this business comes down to, `I don't like that guy.'\,}{Roger Sterling, Mad Men}

% Problem 1
\problem{10} Levy Tate is tracking the CPI to measure and predict inflation. The CPI last year was 296.17, while this year it 305.86. 
	\begin{enumerate}[(a)]
	\item What was the inflation rate from last year to this year?
	\item If the inflation rate remains constant from this year to next year, what will be the CPI that Levy should hope to predict for next year?
	\item Assuming the inflation rate is constant over the next year, if a certain good cost \$15.99 this year, what should Levy predict that it costs next year according to this data?
	\end{enumerate}



\newpage



% Problem 2
\problem{10} Arty Fischel is attempting to compute his federal income tax for this year. Going online, as he is a single filer, he finds the following two charts to compute his federal income tax: \par
	\begin{table}[!ht]
	\centering
	\begin{tabular}{|l|l|} \hline
	Taxable Income & Tax Owed \\ \hline \hline
	\$0--\$10,275 & 10\% of taxable income \\ \hline
	\$10,276--\$41,775 & \$1,027.50 + 12\% amount over \$10,275 \\ \hline
	\$41,776--\$89,075 & \$4,807.50 + 22\% amount over \$41,775 \\ \hline
	\$89,076--\$170,050 & \$15,213.50 + 24\% amount over \$89,075 \\ \hline
	\$170,051--\$215,950 & \$34,647.50 + 32\% amount over \$170,050 \\ \hline
	\$215,951--\$539,900 & \$49,335.50 + 35\% amount over \$215,950 \\ \hline
	$\geq$ \$539,901 & \$162,718 + 37\% amount over \$539,900 \\ \hline
	\end{tabular}
	\end{table}

	\begin{table}[!ht]
	\centering
	\begin{tabular}{|l|l|} \hline
	Tax Rate & Taxable Income \\ \hline \hline
	10\% & Up to \$10,275 \\ \hline
	12\% & \$10,276--\$41,775 \\ \hline
	22\% & \$41,776--\$89,075 \\ \hline
	24\% & \$89,076--\$170,050 \\ \hline
	32\% & \$170,051--\$215,950 \\ \hline
	35\% & \$215,951--\$539,900 \\ \hline
	37\% & $\geq$ \$539,901 \\ \hline
	\end{tabular}
	\end{table}	

\begin{enumerate}[(a)]
\item Explain how these two charts convey the same information. Your explanation should include an explanation of how to use each chart to compute the federal income tax for a single individual with \$20,000 of taxable income.
\item If Arty made \$365,000 last year as a software engineer, how much will he pay in federal income tax.
\end{enumerate}
	


\newpage



% Problem 3
\problem{10} Sue Flay is taking out a small business loan to open her dream bakery. The loan she takes out is for \$85,000 at a 2.89\% annual interest rate, compounded monthly.
	\begin{enumerate}[(a)]
	\item What is the effective interest rate for this loan?
	\item How much does she owe after 2~years?
	\item How long until Sue owes the bank \$150,000?
	\end{enumerate}



\newpage



% Problem 4
\problem{10} Brock Lee is open a savings account to have enough money for community college. He places \$2,500 in the account, which earns 0.13\% annual interest, compounded continuously.
	\begin{enumerate}[(a)]
	\item What is the effective interest rate for this account?
	\item How much is in his account after 4~years?
	\item If the cost of a year at the college is \$21,714, how much should have Lee placed in the account to have enough for his first full year at the college in four years?
	\end{enumerate}



\newpage



% Problem 5
\problem{10} Ray Gunne is comparing loan rates at two different banks. The first offers loans at a 7.99\% annual interest rate, compounded semiannually, while the other offers a loan at 7.89\% annual interest rate, compounded continuously. Which loan should he take? Justify your answer completely.



\newpage



% Problem 6
\problem{10} Sue Render wants to be able to save \$500 in the next two years by depositing \$400 in a savings account. She can either choose a savings account that compounds interest quarterly or continuously. For both types of savings account, find the interest rate on the savings account she would have to secure to have the \$500 after 2~years. Does her plan seem feasible? 


\end{document}