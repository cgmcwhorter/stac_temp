\documentclass[11pt,letterpaper]{article}
\usepackage[lmargin=1in,rmargin=1in,tmargin=1in,bmargin=1in]{geometry}
\usepackage{../style/homework}
\usepackage{../style/commands}
\setbool{quotetype}{true} % True: Side; False: Under
\setbool{hideans}{true} % Student: True; Instructor: False

% -------------------
% Content
% -------------------
\begin{document}

\homework{3: Due 09/15}{Taking on a challenge is a lot like riding a horse, isn't it? If you're comfortable while you're doing it, you're probably doing it wrong.}{Ted Lasso, Ted Lasso}

% Problem 1
\problem{10} Sally Forth takes out a \$960 loan for 7~months that is discounted at an 8.3\% annual simple interest rate.
	\begin{enumerate}[(a)]
	\item What is the discount for this loan?
	\item What is the maturity? What are the proceeds?
	\item How much interest is paid on this loan? How much is paid in total?
	\item Find the nominal and effective interest rates for this loan.
	\end{enumerate}



\newpage



% Problem 2
\problem{10} Recall that the 2022~Federal Income Tax Bracket for single filers (who had a standard deduction \$12,950) was as follows:
	\begin{table}[!ht]
	\centering
	\scalebox{0.75}{%
	\begin{tabular}{|l|c|c|c|c|c|c|c|} \hline 
	Income & \$0--\$10,275 & \$10,275--\$41,775 & \$41,775--\$89,075 & \$89,075--\$170,050 & \$170,050--\$215,950 & \$215,950--\$539,900 & $>$\$539,000 \\ \hline
	Tax Rate & 10\% & 12\% & 22\% & 24\% & 32\% & 35\% & 37\% \\ \hline
	\end{tabular}
	}
	\end{table}

	\begin{enumerate}[(a)]
	\item Find a single filer's federal tax amount if they made \$40,852 and took the standard deduction. 
	\item Find a single filer's federal tax amount if they made \$92,661 and took the standard deduction. 
	\item In what sense is it sensible to say that millionaires (and higher) are taxed at a rate of 37\%? In what sense does that not make sense? Explain.
	\end{enumerate}



\newpage



% Problem 3
\problem{10} If a goods that cost \$97.65 last year now cost \$103.39, what is the inflation rate on these goods from last year to this year? Explain why this might be `harmful' for consumers and why this might not be `harmful' for consumers. 



\newpage



% Problem 4
\problem{10} Suppose the CPI in 2045 was 278.405. By what factors are goods more expensive in 2045 than in 1982? What is the inflation rate over this time period? If the CPI in 2044 were 276.901, what is the inflation rate from 2044 to 2045?


\end{document}