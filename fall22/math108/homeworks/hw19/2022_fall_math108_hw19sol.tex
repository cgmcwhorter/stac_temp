\documentclass[11pt,letterpaper]{article}
\usepackage[lmargin=1in,rmargin=1in,tmargin=1in,bmargin=1in]{geometry}
\usepackage{../style/homework}
\usepackage{../style/commands}
\setbool{quotetype}{false} % True: Side; False: Under
\setbool{hideans}{false} % Student: True; Instructor: False

% -------------------
% Content
% -------------------
\begin{document}

\homework{19: Due 12/08}{Linear programming can be viewed as part of a great revolutionary development which has given mankind the ability to state general goals and to lay out a path of detailed decisions to take in order to `best' achieve its goals when faced with practical situations of great complexity.}{George Dantzig}

% Problem 1
\problem{10} Write down the initial simplex tableau for the following optimization problem:
	\[
	\begin{aligned}
	\max z= 4.6x_1 + \,&3.1x_2 + 7.9x_3 \\
	5.5x_1 - 6x_2 + \,&1.1x_3 \leq 110.3 \\
	-6.7x_1 - \,&8.3 x_3 \leq 220.1 \\
	x_1 - 7.7x_2 + \,&4.5x_3 \leq 662.0 \\
	x_1, \, &x_2, x_3 \geq 0 
	\end{aligned}
	\] \pspace

\sol Introducing slack variables into each inequality (except the last non-negativity inequality) to obtain equalities, we have\dots \par
	\begin{table}[!ht]
	\centering
	\begin{tabular}{rrrrrrrrrrr}
	$5.5x_1$ & $+$ & $-6x_2$ & $+$ & $1.1x_3$ & $+$ & $s_1$ & & & $=$ & $110.3$ \\
	$-6.7x_1$ & $+$ & $0x_2$ & $+$ & $-8.3x_3$ & $+$ & & $s_2$ & & $=$ & $220.1$ \\
	$1x_1$ & $+$ & $-7.7x_2$ & $+$ & $4.5x_3$ & $+$ & & & $s_3$ & $=$ & $662.0$
	\end{tabular}
	\end{table} \par
Moving things to the `$z$'-side of the equality in the function, we have $z - 4.6x_1 - 3.1x_2 - 7.9x_3= 0$. Adding this to the table yields\dots \par
	\begin{table}[!ht]
	\centering
	\begin{tabular}{rrrrrrrrrrrrr}
	&& $5.5x_1$ & $+$ & $-6x_2$ & $+$ & $1.1x_3$ & $+$ & $s_1$ & & & $=$ & $110.3$ \\
	&& $-6.7x_1$ & $+$ & $0x_2$ & $+$ & $-8.3x_3$ & $+$ & & $s_2$ & & $=$ & $220.1$ \\
	&& $1x_1$ & $+$ & $-7.7x_2$ & $+$ & $4.5x_3$ & $+$ & & & $s_3$ & $=$ & $662.0$ \\
	$z$ & $+$ & $-4.6x_1$ & $+$ & $-3.1x_2$ & $+$ & $-7.9x_3$ & & & & & $=$ & $0$
	\end{tabular}
	\end{table} \par
This yields the following initial simplex tableau: \par
	\begin{table}[!ht]
	\centering
	\begin{tabular}{rrrrrr|r}
	$5.5$ & $-6.0$ & $1.1$ & $1$ & $0$ & $0$ & $110.3$ \\ 
	$-6.7$ & $0.0$ & $-8.3$ & $0$ & $1$ & $0$ & $220.1$ \\ 
	$1.0$ & $-7.7$ & $4.5$ & $0$ & $0$ & $1$ & $662.0$ \\ \hline
	$-4.6$ & $-3.1$ & $-7.9$ & $0$ & $0$ & $0$ & $0$ \\ 
	\end{tabular}
	\end{table}
	


\newpage



% Problem 2
\problem{10} Suppose that the initial simplex tableau below was associated to a standard maximization problem. Write down the function being maximized and the corresponding system of constraints. \par
	\begin{table}[!ht]
	\centering
	\begin{tabular}{rrrrrr|r}
	$2$ & $-1$ & $4$ & $1$ & $0$ & $0$ & $100$ \\
	$6$ & $0$ & $2$ & $0$ & $1$ & $0$ & $80$ \\
	$-4$ & $8$ & $3$ & $0$ & $0$ & $1$ & $220$ \\ \hline
	$-3$ & $-1$ & $-5$ & $0$ & $0$ & $0$ & $0$ \\
	\end{tabular}
	\end{table} \pspace

\sol Each row of the tableau `corresponds' to an inequality with the exception of the last row which `corresponds to the function.' But then there were $4 - 1= 3$ inequalities in the original system (ignoring the non-negativity inequality). For each inequality, we introduce a slack variable. Therefore, there were 3 slack variables. Each column of the tableau `corresponds' to a variable in the system with the exception of the last column which `corresponds to the solutions.' Therefore, there were $7 - 1= 6$ variables in the system. Because 3 of the variables are slack variables, we have $6 - 3= 3$ `original' variables in the system of inequalities. Labeling these columns in the tableau, we have\dots \par
	\begin{table}[!ht]
	\centering
	\begin{tabular}{rrrrrr|r}
	{\small $x_1$} & {\small $x_2$} & {\small $x_3$} & {\small $s_1$} & {\small $s_2$} & {\small $s_3$} & \\
	$2$ & $-1$ & $4$ & $1$ & $0$ & $0$ & $100$ \\
	$6$ & $0$ & $2$ & $0$ & $1$ & $0$ & $80$ \\
	$-4$ & $8$ & $3$ & $0$ & $0$ & $1$ & $220$ \\ \hline
	$-3$ & $-1$ & $-5$ & $0$ & $0$ & $0$ & $0$ \\
	\end{tabular}
	\end{table} \par
The last row `corresponds' to the function. But then we have $z - 3x_1 - x_2 - 5x_3= 0$ so that $z= 3x_1 + x_2 + 5x_3$. Writing the equalities corresponding to the first 3 rows, we have\dots
	\[
	\begin{aligned}
	2x_1 - x_2 + 4x_3 + s_1&= 100 \\
	6x_1 + 2x_3 + s_2&= 80 \\
	-4x_1 + 8x_2 + 3x_3 + s_3&= 220 
	\end{aligned}
	\]
Removing the slack variables, we have\dots
	\[
	\begin{aligned}
	2x_1 - x_2 + 4x_3&\leq 100 \\
	6x_1 + 2x_3&\leq 80 \\
	-4x_1 + 8x_2 + 3x_3&\leq 220 
	\end{aligned}
	\]
Therefore, the original minimization problem was\dots
	\[
	\begin{aligned}
	\max z= 3x_1 + x_2 + 5x_3 \\
	2x_1 - x_2 + 4x_3 \leq 100 \\
	6x_1 + 2x_3 \leq 80 \\
	-4x_1 + 8x_2 + 3x_3 \leq 220 \\
	x_1, x_2, x_3 \geq 0 
	\end{aligned}
	\] \pspace



\newpage



% Problem 3
\problem{10} Suppose that the final simplex tableau associated to a maximization problem was the following: \par
	\begin{table}[!ht]
	\centering
	\begin{tabular}{rrrrrrrrrr}
	$1$ & $1.1$ & $2$ & $0$ & $0$ & $0.22$ & $0.067$ & $-0.011$ & $0$ & $140$ \\
	$0$ & $2.1$ & $1.5$ & $1$ & $0$ & $-0.021$ & $0.23$ & $-0.037$ & $0$ & $85$ \\
	$0$ & $-1.1$ & $-0.59$ & $0$ & $1$ & $0.008$ & $-0.088$ & $0.16$ & $0$ & $42$ \\
	$0$ & $-6.4$ & $-12$ & $0$ & $0$ & $-0.55$ & $-0.45$ & $0.54$ & $1$ & $270$ \\
	$0$ & $2.3$ & $2.3$ & $0$ & $0$ & $0.2$ & $0.59$ & $0.72$ & $0$ & $760$ \\
	\end{tabular}
	\end{table}

\begin{enumerate}[(a)]
\item How many inequalities were considered?
\item How many variables were there in the original inequalities?
\item How many slack/surplus variables were introduced?
\item What was the solution to this maximization problem?
\end{enumerate} \pspace

\sol 
\begin{enumerate}[(a)]
\item Each row of the tableau `corresponds' to an inequality with the exception of the last row which `corresponds to the function.' But then there were $5 - 1= 4$ inequalities in the original system (ignoring the non-negativity inequality). \pspace

\item Each column of the tableau `corresponds' to a variable in the system with the exception of the last column which `corresponds to the solutions.' Therefore, there were $10 - 1= 9$ variables in the system. Note by (c), there are 4 slack/surplus variables. Therefore, there were $9 - 4= 5$ `original' variables in the system of inequalities. \pspace

\item Because we introduce a slack/surplus variable for each inequality and by (a) there were 4 inequalities in the original system, there were 4 slack/surplus variables. \pspace

\item By (b) and (c), there were 5 `original' variables and 4 slack/surplus variables. Therefore, we need find the maximum value along with the values of the variables----namely, the values for $(x_1, x_2, x_3, x_4, x_5, s_1, s_2, s_3, s_4)$. Adding `dividers' to the tableau and `naming' the columns, we have\dots \par
	\begin{table}[!ht]
	\centering
	\begin{tabular}{rrrrrrrrr|r}
	{\small $x_1$} & {\small $x_2$} & {\small $x_3$} & {\small $x_4$} & {\small $x_5$} & {\small $s_1$} & {\small $s_2$} & {\small $s_3$} & {\small $s_4$} & \\
	\boxed{$1$} & $1.1$ & $2$ & $0$ & $0$ & $0.22$ & $0.067$ & $-0.011$ & $0$ & $140$ \\
	$0$ & $2.1$ & $1.5$ & \boxed{$1$} & $0$ & $-0.021$ & $0.23$ & $-0.037$ & $0$ & $85$ \\
	$0$ & $-1.1$ & $-0.59$ & $0$ & \boxed{$1$} & $0.008$ & $-0.088$ & $0.16$ & $0$ & $42$ \\
	$0$ & $-6.4$ & $-12$ & $0$ & $0$ & $-0.55$ & $-0.45$ & $0.54$ & \boxed{$1$} & $270$ \\ \hline
	$0$ & $2.3$ & $2.3$ & $0$ & $0$ & $0.2$ & $0.59$ & $0.72$ & $0$ & $760$ \\
	\end{tabular}
	\end{table} \par
We indicate the pivot positions above. This yields $x_1= 140$, $x_4= 85$, $x_5= 42$, and $s_4= 270$. All remaining variables have value 0. The maximum value is $760$. Therefore, the maximum value is $760$ and occurs at $(x_1, x_2, x_3, x_4, x_5, s_1, s_2, s_3, s_4)= (140, 0, 0, 85, 42, 0, 0, 0, 270)$. 
\end{enumerate}


\end{document}