\documentclass[11pt,letterpaper]{article}
\usepackage[lmargin=1in,rmargin=1in,tmargin=1in,bmargin=1in]{geometry}
\usepackage{../style/homework}
\usepackage{../style/commands}
\setbool{quotetype}{true} % True: Side; False: Under
\setbool{hideans}{false} % Student: True; Instructor: False

% -------------------
% Content
% -------------------
\begin{document}

\homework{8: Due 10/17}{I hate Algebra.}{John H. Conway}

% Problem 1
\problem{10} Determine whether the point $(-6, -2)$ is on the graph of $f(x)= 8 - \frac{5}{3}\,x$. Determine also whether the point $(12, -12)$ is on the graph of $f(x)$. For each, explain why or why not. \pspace

\sol If $(-6, -2)$ is a point on the graph of $f(x)$, then we know that $f(-6)= -2$. But we have\dots
	\[
	f(-6)= 8 - \frac{5}{3} \cdot -6= 8 - 5(-2)= 8 + 10= 18
	\]
Because $18 \neq -2$, we know that $(-6, -2)$ is not a point on the graph of $f(x)$. \pspace

If the point $(12, -12)$ is on the graph of $f(x)$, then we know that $f(12)= -12$. But we have\dots
	\[
	f(12)= 8 - \frac{5}{3} \cdot 12= 8 - 5(4)= 8 - 20= -12
	\]
Therefore, $(12, -12)$ is a point on the graph of $f(x)$. 



\newpage



% Problem 2
\problem{10} Suppose $f(x)$ and $g(x)$ are the functions given below. 
        \begin{table}[!ht]
        \centering
        \begin{tabular}{| c || c | c | c | c | c | c | c |} \hline
	$x$ & $-3$ & $-2$ & $-1$ & $\phantom{-}0$ & $\phantom{-}1$ & $\phantom{-}2$ & $\phantom{-}3$ \\ \hline
	$f(x)$ & $\phantom{-1}4$ & $\phantom{-}2$ & $\phantom{-}0$ & $-5$ & $\phantom{-}1$ & $\phantom{-}2$ & $\phantom{-}4$ \\ \hline
	$g(x)$ & $\phantom{-1}2$ & $\phantom{-}1$ & $-1$ & $\phantom{-}1$ & $-2$ & $\phantom{-}3$ & $-3$ \\ \hline
	$h(x)$ & $-12$ & $\phantom{-}4$ & $\phantom{-}10$ & $-2$ & $\phantom{-}4$ & $-4$ & $\phantom{-}0$ \\ \hline
        \end{tabular}
        \end{table}

Compute the following: \pspace
        \begin{enumerate}[(a)]
        \item $(f + h)(-1)= f(-1) + h(-1)= 0 + 10= 10$ \vfill
        \item $(h - g)(2)= h(2) - g(2)= -4 - 3= -7$ \vfill
        \item $(5f)(2)= 5f(2)= 5(2)= 10$ \vfill
        \item $\left(\dfrac{h}{g}\right)(-3)= \dfrac{h(-3)}{g(-3)}= \dfrac{-12}{2}= -6$ \vfill
        \item $f(0)\, h(1)= -5 \cdot 4= -20$ \vfill
        \item $g \big(2 - h(1) \big)= g(2 - 4)= g(-2)= 1$ \vfill
        \item $(f \circ g)(-3)= f \big( g(-3) \big)= f(2)= 2$ \vfill
	\item $(g \circ h)(3)= g \big( h(3) \big)= g(0)= 1$ \vfill
        \item $(h \circ g)(3)= h \big( g(3) \big)= h(-3)= -12$ \vfill
	\item $(f \circ g \circ h)(0)= f \big( g \big( h(0) \big) \big)= f \big( g( -2) \big)= f(1)= 1$ \vfill
        \end{enumerate} 



\newpage



% Problem 3
\problem{10} Suppose $f(x)$ and $g(x)$ are the functions given below. 
	\[
	\begin{aligned}
	f(x)&= 2 - x \\[0.3cm]
	g(x)&= x^2 - 3x + 2
	\end{aligned}
	\]

Compute the following: \pspace
\begin{enumerate}[(a)]
\item $f(-4)= 2 - (-4)= 2 + 4= 6$ \vfill
\item $g(2)= 2^2 - 3(2) + 2= 4 - 6 + 2= 0$ \vfill
\item $2f(1) - g(3)= 2(1) - 2= 2 - 2= 0$ \vfill
\item $f(x) - g(x)= (2 - x) - (x^2 - 3x + 2)= 2 - x - x^2 + 3x - 2= -x^2 + 2x$ \vfill
\item $f(x) \, g(x)= (2 - x)(x^2 - 3x + 2)= 2x^2 - 6x + 4 - x^3 - 3x^2 - 2x= -x^3 + 5x^2 - 8x + 4$ \vfill
\item $\left( \dfrac{f}{g} \right)(x)= \dfrac{f(x)}{g(x)}= \dfrac{2 - x}{x^2 - 3x + 2}$ \vfill
\item $(f \circ g)(0)= f \big( g(0) \big)= f(2)= 0$ \vfill
\item $(g \circ f)(0)= g \big( f(0) \big)= g(2)= 0$ \vfill
\item $(f \circ g)(x)= f \big( g(x) \big)= f( x^2 - 3x + 2)= 2 - (x^2 - 3x + 2)= 2 - x^2 + 3x - 2= -x^2 + 3x$ \vfill
\item $(g \circ f)(x)= g \big( f(x) \big)= g(2 - x)= (2 - x)^2 - 3(2 - x) + 2= (4 - 4x + x^2) + (-6 + 3x) + 2= x^2 - x$ \vfill
\end{enumerate} 



\newpage



% Problem 4
\problem{10} Suppose $f(x)$ and $g(x)$ are functions. 
	\begin{enumerate}[(a)]
	\item Explain what it means for $f(2)= g(2)$ graphically. 
	\item Explain what $f(x)$ and $g(x)$ intersecting at the point $(-1, 7)$ means algebraically. 
	\end{enumerate} \pspace

\sol 
\begin{enumerate}[(a)]
\item We know that $\big(x, f(x) \big)$ and $\big(x, g(x) \big)$ are points on the graph of $f(x)$ and $g(x)$, respectively. But then $\big(2, f(2) \big)$ is a point on the graph of $f(x)$ and $\big(2, g(2) \big)$ is a point on the graph of $g(x)$. But because $f(2)= g(2)$, we know that $\big(2, f(2) \big)= \big(2, g(x) \big)$. Therefore, if $f(2)= g(2)$, the graphs of $f(x)$ and $g(x)$ intersect when $x= 2$. \pspace

\item If $f(x)$ and $g(x)$ intersect at the point $(-1, 7)$, then we know that $(-1, 7)$ is a point on the graph of $f(x)$ and $g(x)$. But then when $x= -1$, we know that $y= 7$. Therefore, $f(-1)= 7$ and $g(-1)= 7$. But then we know that $f(-1)= g(-1)$. 
\end{enumerate}


\end{document}