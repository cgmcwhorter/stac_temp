\documentclass[11pt,letterpaper]{article}
\usepackage[lmargin=1in,rmargin=1in,tmargin=1in,bmargin=1in]{geometry}
\usepackage{../style/homework}
\usepackage{../style/commands}
\setbool{quotetype}{true} % True: Side; False: Under
\setbool{hideans}{false} % Student: True; Instructor: False

% -------------------
% Content
% -------------------
\begin{document}

\homework{3: Due 09/26}{I'm never going to financially recover from this.}{Joe Exotic, Tiger King}

% Problem 1
\problem{10} Write the following decimal numbers in scientific notation:
	\begin{enumerate}[(a)]
	\item 1\,541\,000
	\item 40\,000
	\item 0.6
	\item 4
	\item 0.00000008
	\end{enumerate} \pspace

\sol
\begin{enumerate}[(a)]
\item 
	\[
	1\,541\,000= 1.541 \cdot 10^6
	\] \pspace

\item 
	\[
	40\,000= 4.0 \cdot 10^4
	\] \pspace

\item 
	\[
	0.6= 6 \cdot 10^{-1}
	\] \pspace

\item 
	\[
	4= 4 \cdot 10^0
	\] \pspace

\item 
	\[
	0.00000008= 8.0 \cdot 10^{-8}
	\]
\end{enumerate}



\newpage



% Problem 2
\problem{10} Write the following numbers in scientific notation as decimal numbers:
	\begin{enumerate}[(a)]
	\item $5.2 \cdot 10^0$
	\item $1.7 \cdot 10^5$
	\item $6.4 \cdot 10^{-1}$
	\item $9.1 \cdot 10^{-5}$
	\item $7.6 \cdot 10^1$
	\end{enumerate} \pspace

\sol
\begin{enumerate}[(a)]
\item 
	\[
	5.2 \cdot 10^0= 5.2
	\] \pspace

\item 
	\[
	1.7 \cdot 10^5= 170,000
	\] \pspace

\item 
	\[
	6.4 \cdot 10^{-1}= 0.64
	\] \pspace

\item 
	\[
	9.1 \cdot 10^{-5}= 0.000091
	\] \pspace

\item
	\[
	7.6 \cdot 10^1= 76
	\]
\end{enumerate}



\newpage



% Problem 3
\problem{10} Showing all your work, compute the following:
	\begin{enumerate}[(a)]
	\item 46\% of 1320
	\item 90\% of 40
	\item 39.6\% of 8476
	\item 1\% of 19.5
	\item 265\% of 210
	\end{enumerate} \pspace

\sol
\begin{enumerate}[(a)]
\item 
	\[
	1320(0.46)= 607.2
	\] \pspace

\item
	\[
	40(0.90)= 36
	\] \pspace

\item
	\[
	8476(0.396)= 3356.5 
	\] \pspace

\item	
	\[
	19.5(0.01)= 0.195
	\] \pspace

\item
	\[
	210(2.65)= 556.5
	\]
\end{enumerate}



\newpage



% Problem 4
\problem{10} Showing all your work, compute the following:
	\begin{enumerate}[(a)]
	\item 750 increased by 15\%
	\item 60 decreased by 33\%
	\item 840 increased by 92\%
	\item 431 decreased by 99\%
	\item 15 increased by 170\%
	\end{enumerate} \pspace

\sol
\begin{enumerate}[(a)]
\item 
	\[
	750(1 + 0.15)= 750(1.15)= 862.5
	\] \pspace

\item
	\[
	60(1 - 0.33)= 60(0.67)= 40.2
	\] \pspace

\item
	\[
	840(1 + 0.92)= 840(1.92)= 1612.8
	\] \pspace

\item	
	\[
	431(1 - 0.99)= 431(0.01)= 4.31
	\] \pspace

\item
	\[
	15(1 + 1.70)= 15(2.70)= 40.5
	\]
\end{enumerate}

	

\newpage



% Problem 5
\problem{10} A laptop from Macrosoft is advertised as being \$999. You plan on ordering this computer online. You know that you will be charged 7\% sales tax. How much will the computer then cost? If you find out that the laptop costs Macrosoft \$89 to produce, what is the percent markup that Macrosoft puts on this laptop? How much profit do they make per laptop? \pspace

\sol The 7\% sales tax increases the cost of the laptop by 7\%. Therefore, the final price of the laptop is\dots 
	\[
	\$999(1 + 0.07)= \$999(1.07)= \$1,068.93.
	\] 
The percent markup will be the percentage increase that turns the \$89 production price to the \$999 sales price. Let $\%_d$ denote the percentage increase, written as a decimal. Then we have\dots \pspace
	\[
	\begin{aligned}
	\$89(1 + \%_d)&= \$999 \\[0.3cm]
	1 + \%_d&= 11.224719 \\[0.3cm]
	\%_d&= 10.224719
	\end{aligned}
	\] \pspace
Therefore, the percent markup is 1,022.4719\%. Equivalently, we can also compute this via\dots \pspace
	\[
	\%_d= \dfrac{\text{new price} - \text{org. price}}{\text{org. price}}= \dfrac{999 - 89}{89}= \dfrac{910}{89}= 10.224719.
	\] \pspace
This again leads to determining that the percent markup is 1,022.4719\%. The profit is the difference between the sales price (the revenue) and the cost of production. Therefore, the profit on each laptop is $\$999 - \$89= \$910$. 



\newpage



% Problem 6
\problem{10} Jeff is arguing with Carol. Over the last 7~months, the cost of products has risen 4\% each month. Jeff then argues that goods now cost $7 \cdot 4\%= 28\%$ more now than they cost 4~months ago. Carol claims that this is not correct and that goods now cost a little more than 30\% more than they did 7~months ago. Who is correct? Be sure to fully justify your response. \pspace

\sol Carol is correct. To see why, let us apply both Jeff's and Carol's logic. According to Jeff, the prices have gone up 4\% each month for a total of $4\% + 4\% + 4\% + 4\% + 4\% + 4\% + 4\%= 7 \cdot 4\%= 28\%$ compared to 7~months ago. Therefore, a good that costs \$1 three years ago will now cost $\$1(1 + 0.28)= \$1(1.28)= \$1.28$. According to Lydia, the costs of goods have gone up by a little more than 30\% compared to 7~months ago. Therefore, an item that cost \$1 three years ago will now cost more than $\$1(1 + 0.30)= \$1(1.30)= \$1.30$. \pspace

Now let us compute the cost of a good directly. Seven months ago, the good cost \$1. After one month, the good costs $\$1(1+0.04)= \$1(1.04)= \$1.04$. After another month, the good costs $\$1.04(1+0.04)= \$1.04(1.04)= \$1.0816$. After another month, the good costs $\$1.0816(1+0.04)= \$1.0816(1.04)= \$1.12486$. After another month, the good costs $\$1.12486(1+0.04)= \$1.12486(1.04)= \$1.16985$. After another month, the good costs $\$1.16985(1+0.04)= \$1.16985(1.04)= \$1.21664$. After another month, the good costs $\$1.21664(1+0.04)= \$1.21664(1.04)= \$1.26531$. After the final month, the good costs $\$1.26531(1+0.04)= \$1.26531(1.04)= \$1.31592$. We could compute this directly by applying a 4\% interest seven times: $\$1(1 + 0.04)^7= \$1 (1.04)^7= \$1 (1.31593) \approx \$1.32$. \pspace

Clearly, Carol's value is closer to reality than Jeff's value. The reason for this is that percentages do not add in this way. Percentages only add if they are applied to a static (unchanging) value. Whereas here, the 4\% increase is being applied to a new value each month. If the price is $P$~dollars now and goes up by \% each month, then the cost after $t$ months will be $P(1 + \%_d)^t$, where $\%_d$ is the percentage written as a decimal. Applying this to the current scenario, we have\dots \pspace
	\[
	P(1 + 0.04)^7= P(1.04)^7= P(1.31593).
	\] \pspace
We can recognize this product as representing a 31.593\% increase to $P$. Because $31.593\% \approx 31.6\%$ (which is a little more than 30\%), we can see that Carol's interpretation is correct. \pspace


\end{document}