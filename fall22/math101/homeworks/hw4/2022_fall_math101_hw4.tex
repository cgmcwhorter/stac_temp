\documentclass[11pt,letterpaper]{article}
\usepackage[lmargin=1in,rmargin=1in,tmargin=1in,bmargin=1in]{geometry}
\usepackage{../style/homework}
\usepackage{../style/commands}
\setbool{quotetype}{false} % True: Side; False: Under
\setbool{hideans}{true} % Student: True; Instructor: False

% -------------------
% Content
% -------------------
\begin{document}

\homework{4: Due 09/28}{We need to do what any animal in nature does when it's cornered: act erratically and blindly lash out at everything around us.}{Erlich Bachman, Silicon Valley}

% Problem 1
\problem{10} Suppose there are three exams in a course: two midterms and a final. A student receives the exam scores given below. \par
	\begin{table}[!ht]
	\centering
	\begin{tabular}{|c|c|c|c|} \hline
	& Midterm~I & Midterm~II & Final \\ \hline
	Score & 72 & 41 & 117 \\ \hline
	Out of\dots & 85 & 76 & 120 \\ \hline
	\end{tabular}
	\end{table} \par
Find the student's exam average if\dots
	\begin{enumerate}[(a)]
	\item each exam is weighted equally.
	\item the exam average is measured by the percentage of points earned.
	\item the final exam is worth twice as much as any midterm.
	\item the final exam is worth twice as much as the midterms combined. 
	\end{enumerate}



\newpage



% Problem 2
\problem{10} Vinyl Richie is taking an English course that has the following grade components: \par
	\begin{table}[!ht]
	\centering
	\begin{tabular}{lrclr}
	Participation & 10\% && Research Paper~II & 10\% \\
	Quizzes & 5\% && Research Paper~III & 15\% \\
	Homework & 20\% && Presentation & 5\% \\
	Research Paper~I & 5\% && Final Paper & 30\%
	\end{tabular}
	\end{table} \par
If he has a 100\% participation average, 90\% quiz average, 84\% homework average, a 75\%, 91\%, and 88\% on research papers I, II, and III, respectively, a 92\% on his presentation, and a 83\% on his final paper, what is his final average in the course?



\newpage



% Problem 3
\problem{10} D.J. Salinger is finishing his undergraduate at TACS. Thus far, he has taken 112~credits and has a 3.415 GPA. This last semester grades are found below: [The college's letter grade scheme is show above on the right.] \par
	\begin{table}[!ht]
	\centering
	\begin{tabular}{lrl}
	Course & Credits & Grade \\ \hline
	Women in Music & 3 & A$-$ \\
	Marketing in Industry & 3 & B \\
	Desktop Music Production & 4 & B+ \\
	DIY Music Marketing & 3 & A \\
	Music, Turmoil and Nationalism & 3 & C+ \\
	Music Supervision & 1 & A
	\end{tabular} \hspace{1cm}
        \begin{tabular}{|l||c|l||c|} \hline
        A & 4.0 & C+ & 2.3 \\ \hline
        A-- & 3.7 & C & 2.0 \\ \hline
        B+ & 3.3 & C-- & 1.7 \\ \hline
        B & 3.0 & D & 1.0 \\ \hline
        B-- & 2.7 & F & 0.0 \\ \hline
        \end{tabular}
	\end{table} \par
What is D.J.'s final GPA? Does he finish cum laude (GPA at least 3.5), magna cum laude (a GPA at least 3.65), or even summa cum laude (GPA 3.8 or higher)?



\newpage



% Problem 4
\problem{10} 
Restauranteurs purchase olive oil by the gallon. Because of the expense of olive oil, they often track its price volatility. They do not measure the average price per gallon. Instead, they measure the average gas price per gallon weighted by the volumes purchased from various regions. Suppose one restauranteur is measuring the average olive oil price per gallon in a region. Over the course of a week, the price and volumes of olive oil purchased in three regions are given below: \par
	\begin{table}[!ht]
	\centering
        \begin{tabular}{lrr} \hline
	Station & Price (\$) & Volume (Gallons) \\ \hline
	Northern & \$51.71 & 27,600 \\
	Southern & \$43.25 & 18,200 \\
	Eastern & \$156.45 & 8,700
        \end{tabular}
	\end{table} \par
Find the average olive oil price per gallon `normally' and then the average olive oil price per gallon weighted by volume. Explain the differences between the two. 


\end{document}