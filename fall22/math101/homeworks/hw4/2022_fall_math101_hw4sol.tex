\documentclass[11pt,letterpaper]{article}
\usepackage[lmargin=1in,rmargin=1in,tmargin=1in,bmargin=1in]{geometry}
\usepackage{../style/homework}
\usepackage{../style/commands}
\setbool{quotetype}{false} % True: Side; False: Under
\setbool{hideans}{false} % Student: True; Instructor: False

% -------------------
% Content
% -------------------
\begin{document}

\homework{4: Due 09/28}{We need to do what any animal in nature does when it's cornered: act erratically and blindly lash out at everything around us.}{Erlich Bachman, Silicon Valley}

% Problem 1
\problem{10} Suppose there are three exams in a course: two midterms and a final. A student receives the exam scores given below. \par
	\begin{table}[!ht]
	\centering
	\begin{tabular}{|c|c|c|c|} \hline
	& Midterm~I & Midterm~II & Final \\ \hline
	Score & 72 & 41 & 117 \\ \hline
	Out of\dots & 85 & 76 & 120 \\ \hline
	\end{tabular}
	\end{table} \par
Find the student's exam average if\dots
	\begin{enumerate}[(a)]
	\item each exam is weighted equally.
	\item the exam average is measured by the percentage of points earned.
	\item the final exam is worth twice as much as any midterm.
	\item the final exam is worth twice as much as the midterms combined. 
	\end{enumerate} \pspace

\sol Note that as percentages, the exam scores were: $72/85 \approx 84.71\%$, $41/76 \approx 53.95\%$, and $117/120 \approx 97.5\%$. 
\begin{enumerate}[(a)]
\item An average where everything is weighted equally, i.e. has weight $1/n$, is simply a `normal' average:
	\[
	\text{Exam Average}= \dfrac{\sum \text{scores}}{\# \text{ scores}}= \dfrac{84.71\% + 53.95\% + 97.5\%}{3}= \dfrac{236.16\%}{3} \approx 78.72\%
	\]

\item The percentage of points earned is\dots
	\[
	\text{Exam Average}= \dfrac{\sum \text{ points earned}}{\text{total possible points}}= \dfrac{72 + 41 + 117}{85 + 76 + 120}= \dfrac{230}{281} \approx 81.85\%
	\]

\item Suppose you weighted the exam average such that the final exam is worth twice as much as any midterm. If we assume that the first midterm is worth one part of the exam grade, then because the second midterm is worth as much as the first, the second midterm is worth one part of the exam grade. Because the final exam is worth twice as much as the first midterm and the second midterm, it must be worth two parts. But then we have split the exam grade into $1 + 1 + 2= 4$ parts. If we then weight each exam average by its weight, i.e. part/total parts, we have\dots
	\[
	\begin{aligned}
	\text{Exam Average}&= \sum \text{avg} \cdot \text{weight} \\
	&= \dfrac{72}{85} \cdot \dfrac{1}{4} + \dfrac{41}{76} \cdot \dfrac{1}{4} + \dfrac{117}{120} \cdot \dfrac{2}{4} \\
	&\approx 84.71\% (0.25) + 53.95\% (0.25) + 97.5\% (0.50) \\
	&= 21.17\% + 13.49\% + 48.75\% \\
	&= 83.41\%
	\end{aligned}
	\]

\item Suppose you weighted the exam average such that the final exam is worth twice as much both of the midterms combined. If we assume that the first midterm is worth one part of the exam grade, then because the second midterm is worth as much as the first, the second midterm is worth one part of the exam grade. Because the final exam is worth twice as much as the other two midterms combined. Combined, they make up $1 + 1= 2$ parts of the exam grade. Then the final exam is worth $2 \cdot 2= 4$ parts of the exam grade. But then we have split the exam grade into $1 + 1 + 4= 6$ parts. If we then weight each exam average by its weight, i.e. part/total parts, we have\dots
	\[
	\begin{aligned}
	\text{Exam Average}&= \sum \text{avg} \cdot \text{weight} \\
	&= \dfrac{72}{85} \cdot \dfrac{1}{6} + \dfrac{41}{76} \cdot \dfrac{1}{6} + \dfrac{117}{120} \cdot \dfrac{4}{6} \\
	&\approx 84.71\% (0.1667) + 53.95\% (0.1667) + 97.5\% (0.6667) \\
	&= 14.12\% + 8.99\% + 65\% \\
	&= 88.11\%
	\end{aligned}
	\] 
\end{enumerate}



\newpage



% Problem 2
\problem{10} Vinyl Richie is taking an English course that has the following grade components: \par
	\begin{table}[!ht]
	\centering
	\begin{tabular}{lrclr}
	Participation & 10\% && Research Paper~II & 10\% \\
	Quizzes & 5\% && Research Paper~III & 15\% \\
	Homework & 20\% && Presentation & 5\% \\
	Research Paper~I & 5\% && Final Paper & 30\%
	\end{tabular}
	\end{table} \par
If he has a 100\% participation average, 90\% quiz average, 84\% homework average, a 75\%, 91\%, and 88\% on research papers I, II, and III, respectively, a 92\% on his presentation, and a 83\% on his final paper, what is his final average in the course? \pspace

\sol The course grade is a weighted average of each of the grade components by what percentage was actually earned. We can compute this directly: 
	\[
	\begin{aligned}
	\hspace{-1cm}\text{Course Average}&= \sum \text{values} \cdot \text{weights} \\[0.3cm]
	&= 10\% (1.00) + 5\% (0.90) + 20\% (0.84) + 5\% (0.75) + 10\% (0.91) + 15\% (0.88) + 5\% (0.92) + 30\% (0.83) \\[0.3cm]
	&= 10\% + 4.5\% + 16.8\% + 3.75\% + 9.1\% + 13.2\% + 4.6\% + 24.9\% \\[0.3cm]
	&= 86.85\% 
	\end{aligned}
	\]



\newpage



% Problem 3
\problem{10} D.J. Salinger is finishing his undergraduate at TACS. Thus far, he has taken 112~credits and has a 3.615 GPA. This last semester grades are found below: [The college's letter grade scheme is show above on the right.] \par
	\begin{table}[!ht]
	\centering
	\begin{tabular}{lrl}
	Course & Credits & Grade \\ \hline
	Women in Music & 3 & A$-$ \\
	Marketing in Industry & 3 & B \\
	Desktop Music Production & 4 & B+ \\
	DIY Music Marketing & 3 & A \\
	Music, Turmoil and Nationalism & 3 & C+ \\
	Music Supervision & 1 & A
	\end{tabular} \hspace{1cm}
        \begin{tabular}{|l||c|l||c|} \hline
        A & 4.0 & C+ & 2.3 \\ \hline
        A-- & 3.7 & C & 2.0 \\ \hline
        B+ & 3.3 & C-- & 1.7 \\ \hline
        B & 3.0 & D & 1.0 \\ \hline
        B-- & 2.7 & F & 0.0 \\ \hline
        \end{tabular}
	\end{table} \par
What is D.J.'s final GPA? Does he finish cum laude (GPA at least 3.5), magna cum laude (a GPA at least 3.65), or even summa cum laude (GPA 3.8 or higher)? \pspace

\sol GPA is a weighted average, where the course value, i.e. course grade (given by its numerical equivalent), is weighted by the percentage of the total credits. This can be written and computed more simply as follows:
	\[
	\begin{aligned}
	\text{Semester GPA}&= \dfrac{\sum \text{value} \cdot \text{credit}}{\text{total credits}} \\[0.3cm]
	&= \dfrac{3.7(3) + 3.0(3) + 3.3(4) + 4.0(3) + 2.3(3) + 4.0(1)}{3 + 3 + 4 + 3 + 3 + 1} \\[0.3cm]
	&= \dfrac{11.1 + 9 + 13.2 + 12 + 6.9 + 4}{17} \\[0.3cm]
	&= \dfrac{56.2}{17} \\[0.3cm]
	&\approx 3.306
	\end{aligned}
	\] 
Therefore, D.J.~Salinger received a GPA of 3.306 this past semester. \pspace

We now combine this with his previous GPA to find his overall GPA. To combine these GPAs, we compute their weighted average, where we weight by the amount of credits corresponding to each GPA:
	\[
	\begin{aligned}
	\text{Final GPA}&= \dfrac{\text{Previous GPA} \cdot \text{Previous Credits} + \text{Current GPA} \cdot \text{Semester Credits}}{\text{Previous Credits} + \text{Semester Credits}} \\[0.3cm]
	&= \dfrac{3.615(112) + 3.306(17)}{112 + 17} \\[0.3cm]
	&= \dfrac{404.88 + 56.202}{112 + 17} \\[0.3cm]
	&= \dfrac{461.082}{129} \\[0.3cm]
	&\approx 3.574
	\end{aligned}
	\]
Therefore, D.J.~Salinger graduates cum laude but not magna cum laude or summa cum laude. 



\newpage



% Problem 4
\problem{10} 
Restauranteurs purchase olive oil by the gallon. Because of the expense of olive oil, they often track its price volatility. They do not measure the average price per gallon. Instead, they measure the average gas price per gallon weighted by the volumes purchased from various regions. Suppose one restauranteur is measuring the average olive oil price per gallon in a region. Over the course of a week, the price and volumes of olive oil purchased in three regions are given below: \par
	\begin{table}[!ht]
	\centering
        \begin{tabular}{lrr} \hline
	Station & Price (\$) & Volume (Gallons) \\ \hline
	Northern & \$51.71 & 27,600 \\
	Southern & \$43.25 & 18,200 \\
	Eastern & \$156.45 & 8,700
        \end{tabular}
	\end{table} \par
Find the average olive oil price per gallon `normally' and then the average olive oil price per gallon weighted by volume. Explain the differences between the two. \pspace

\sol In a normal average, we add all the possible values and divide by the total number of values. So the `normal' average is\dots 
	\[
	\text{Average}= \dfrac{\$51.71 + \$43.25 + \$156.45}{3}= \dfrac{\$251.41}{3} \approx \$83.80
	\]
Recall that a weighted average is given by\dots 
	\[
	\sum \text{value} \cdot \text{weight},
	\] 
where $\sum$ stands for the sum of the values value~$\cdot$~weight. The values here are the prices and the weight factor is the volume sold at that price level divided by the total volume of oil sold. There was a total of $27600 + 18200 + 8700= 54500$~gallons of oil sold. Then the weighted average is\dots
	\[
	\begin{aligned}
	\text{Weighted Average}&= \sum \text{value} \cdot \text{weight} \\[0.3cm]
	&= \$51.71 \cdot \dfrac{27600}{54500} + \$43.25 \cdot \dfrac{18200}{54500} + \$156.45 \cdot \dfrac{8700}{54500} \\[0.3cm]
	&= \$51.71 (0.506422) + \$43.25 (0.333945) + \$156.45 (0.159633) \\[0.3cm]
	&= \$26.1871 + \$14.4431 + \$24.9746 \\[0.3cm]
	&= \$65.6048 \\[0.3cm]
	&\approx \$65.60
	\end{aligned}
	\]
The difference between a `normal' average and a weighted average is that a weighted average considers what are the `typical' values, whereas a `normal' average weights everything equally. For instance, if 99\% of people buy a product for \$100 and 1\% of people buy the same product for \$10, a `normal' average would say the product costs $(\$100 + \$10)/2= \$55$. However, no one pays a price near that value. In fact, the overwhelming majority of people pay far more than that. Whereas the weighted average considers what is `typical': $\$100(0.99) + \$10(0.01)= \$99.10$. `Normal' averages are `good' when all the values occur essentially equal frequency; otherwise, one should consider using a weighted average. Notice here, because `most' people are paying \$51.71 for oil and not the maximum price of \$156.45, the weighted average is a better measure of `average' oil price for this area. 


\end{document}