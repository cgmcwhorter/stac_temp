\documentclass[11pt,letterpaper]{article}
\usepackage[lmargin=1in,rmargin=1in,tmargin=1in,bmargin=1in]{geometry}
\usepackage{../style/homework}
\usepackage{../style/commands}
\setbool{quotetype}{true} % True: Side; False: Under
\setbool{hideans}{true} % Student: True; Instructor: False

% -------------------
% Content
% -------------------
\begin{document}

\homework{10: Due 10/24}{VIP is always better, Vivian.}{Anna Delvey (Sorokin), Inventing Anna}

% Problem 1
\problem{10} A function $f(x)$ has a table of values given below. Using this table, explain why $f^{-1}(x)$ cannot exist. \par
	\begin{table}[!ht]
	\centering
	\begin{tabular}{|r||c|c|c|c|c|} \hline
	$x$ & $1$ & $2$ & $3$ & $4$ & $5$ \\ \hline
	$f(x)$ & $6$ & $3$ & $9$ & $6$ & $1$ \\ \hline
	\end{tabular}
	\end{table}



\newpage



% Problem 2
\problem{10} Let $f(x)= 4x + 3$ and $g(x)= \frac{1}{4}(x - 3)$. Show that $g(x)$ is the inverse of $f(x)$ by showing that $(f \circ g)(x)= f(g(x))= x$ and $(g \circ f)(x)= x$. 



\newpage



% Problem 3
\problem{10} Let $y= \frac{1}{3}\,x + 5$.
	\begin{enumerate}[(a)]
	\item By interchanging the roles of $y$ and $x$, find the inverse to the function $f(x)= \frac{1}{3}\,x + 5$.
	\item Use the answer from (a) to find $f^{-1}(-2)$. 
	\end{enumerate}


\end{document}