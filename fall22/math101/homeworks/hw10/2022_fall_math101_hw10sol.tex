\documentclass[11pt,letterpaper]{article}
\usepackage[lmargin=1in,rmargin=1in,tmargin=1in,bmargin=1in]{geometry}
\usepackage{../style/homework}
\usepackage{../style/commands}
\setbool{quotetype}{true} % True: Side; False: Under
\setbool{hideans}{false} % Student: True; Instructor: False

% -------------------
% Content
% -------------------
\begin{document}

\homework{10: Due 10/24}{VIP is always better, Vivian.}{Anna Delvey (Sorokin), Inventing Anna}

% Problem 1
\problem{10} A function $f(x)$ has a table of values given below. Using this table, explain why $f^{-1}(x)$ cannot exist. \par
	\begin{table}[!ht]
	\centering
	\begin{tabular}{|r||c|c|c|c|c|} \hline
	$x$ & $1$ & $2$ & $3$ & $4$ & $5$ \\ \hline
	$f(x)$ & $6$ & $3$ & $9$ & $6$ & $1$ \\ \hline
	\end{tabular}
	\end{table} \pspace

\sol Because $f(1)= 6$, we know that $f^{-1}(6)= 1$. But we also have $f(4)= 6$, so that $f^{-1}(6)= 4$. But we cannot have both $f^{-1}(6)= 1$ and $f^{-1}(6)= 4$. Therefore, $f^{-1}(6)$ is not well defined so that $f^{-1}(x)$ does not exist. 



\newpage



% Problem 2
\problem{10} Let $f(x)= 4x + 3$ and $g(x)= \frac{1}{4}(x - 3)$. Show that $g(x)$ is the inverse of $f(x)$ by showing that $(f \circ g)(x)= f(g(x))= x$ and $(g \circ f)(x)= x$. \pspace

\sol We have\dots
	\[
	\begin{aligned}
	(f \circ g)(x)&= f \big( g(x) \big) &\hspace{1cm}&& (g \circ f)(x)&= g \big( f(x) \big) \\[0.3cm]
	&= f \left( \frac{1}{4} (x - 3) \right) &&& &= g(4x + 3) \\[0.3cm]
	&= 4 \cdot \frac{1}{4} (x - 3) + 3 &&& &= \frac{1}{4} \, \big( (4x + 3) - 3 \big) \\[0.3cm]
	&= (x - 3) + 3 &&& &= \frac{1}{4} \cdot 4x \\[0.3cm]
	&= x &&& &=x
	\end{aligned}
	\] \pspace
Therefore, because $(f \circ g)(x)= x$ and $(g \circ f)(x)= x$, we know that $g(x)= f^{-1}(x)$. 



\newpage



% Problem 3
\problem{10} Let $y= \frac{1}{3}\,x + 5$.
	\begin{enumerate}[(a)]
	\item By interchanging the roles of $y$ and $x$, find the inverse to the function $f(x)= \frac{1}{3}\,x + 5$.
	\item Use the answer from (a) to find $f^{-1}(-2)$. 
	\end{enumerate} \pspace

\sol 
\begin{enumerate}[(a)]
\item We can write $f(x)= \frac{1}{3}\,x + 5$ as $y= \frac{1}{3}\,x + 5$. Interchanging the roles of $x$ and $y$, we have $x= \frac{1}{3}\,y + 5$. But then\dots
	\[
	\begin{aligned}
	x&= \frac{1}{3}\,y + 5 \\[0.3cm]
	x - 5&= \frac{1}{3}\,y \\[0.3cm]
	y&= 3(x - 5)
	\end{aligned}
	\] \pspace
But then we have $f^{-1}(x)= 3(x - 5)$. \pspace

\item We have\dots
	\[
	f^{-1}(-2)= 3(-2 - 5)= 3(-7)= -21
	\]
\end{enumerate}


\end{document}