\documentclass[11pt,letterpaper]{article}
\usepackage[lmargin=1in,rmargin=1in,tmargin=1in,bmargin=1in]{geometry}
\usepackage{../style/homework}
\usepackage{../style/commands}
\setbool{quotetype}{false} % True: Side; False: Under
\setbool{hideans}{false} % Student: True; Instructor: False

% -------------------
% Content
% -------------------
\begin{document}

\homework{19: Due 11/16}{There is no branch of mathematics, however abstract, which may not some day be applied to phenomena of the real world.}{Nikolai Lobachevsky}

% Problem 1
\problem{10} An oil company is selling off one of its oil reserves. The amount of oil left in the storage tank in tens of thousands of gallons, $O$, after $d$ days is given by $O(d)= 1600 - 133.4d$.
	\begin{enumerate}[(a)]
	\item Find and interpret the slope of $O(d)$ in the context of the problem.
	\item Find and interpret the $y$-intercept of $O(d)$ in the context of the problem.
	\item Find how long it will take the company to sell all the oil in the tank. 
	\item If the company sells the oil for \$2.085/gallon, how much money do they make selling this reserve oil?
	\end{enumerate} \pspace

\sol 
\begin{enumerate}[(a)]
\item The slope of $O(d)$ is $m= -133.4= \frac{-133.4}{1}$. Interpreting $m= \frac{-133.4}{1}$ as $\frac{\Delta O}{\Delta d}$, we can see that for every additional day, the amount of oil decreases by $133.4$~tens of thousands of gallons. Therefore, the company sells 133.4~tens of thousands of gallons of oil per day. \pspace

\item The $y$-intercept of $O(d)$ is $b= 1600$. The $y$-intercept occurs when $d= 0$, i.e. the first day. Therefore, the oil reserve contains 1,600~tens of thousands of gallons of oil on the first day of the sales. \pspace

\item If the company has sold all the oil in the tank, the amount left in the tank, $O(d)$, is 0. But then we have\dots
	\[
	\begin{aligned}
	O(d)&= 0 \\[0.3cm]
	1600 - 133.4d&= 0 \\[0.3cm]
	133.4d&= 1600 \\[0.3cm]
	d&= 11.994
	\end{aligned}
	\] 
Therefore, they have sold all the oil in the tank after 11.994~days. \pspace

\item By (b), we know that the reserve contains 1,600~tens of thousands of gallons of oil, i.e. $1600 \cdot 10000= 16000000$~gallons of oil. Each gallon is sold for \$2.085/gallon. Therefore, the amount of revenue from selling this oil reserve is $\$2.085 \cdot 16,000,000= \$33,360,000$. 
\end{enumerate}



\newpage



% Problem 2
\problem{10} Suppose that last year, the demand for a certain good was 185~thousand units. It is estimated that next year, the demand will be for 221~thousand units. 
	\begin{enumerate}[(a)]
	\item Assuming that the change in demand is constant, find a linear function predicting the level of demand $t$~years from now.
	\item Interpret the slope and $y$-intercept from your function in (a).
	\item What is your prediction for the level of demand in 5~years?
	\item Predict how many years until the demand is 400~thousand units. 
	\end{enumerate} \pspace

\sol 
\begin{enumerate}[(a)]
\item We know that the demand, $D(t)$, after $t$ years has the form $D(t)= mt + b$. We know that last year, i.e. $t= -1$~years ago, the level of production was 185~thousand units. Next year, i.e. $t= 1$~years from now, the demand will be 221~thousand units. Therefore, $(-1, 185)$ and $(1, 221)$ are points on the graph of $D(t)$. But then the slope of $D(t)$ is\dots
	\[
	m= \dfrac{\Delta D}{\Delta t}= \dfrac{221 - 185}{1 - (-1)}= \dfrac{36}{1 + 1}= \dfrac{36}{2}= 18
	\]
But then $D(t)= 18t + b$. Because the graph of $D(t)$ contains the point $(1, 221)$, when $t= 1$, $D(1)= 221$, i.e. $t= 1$ and $D= 221$ satisfy the equation for $D(t)$. But then\dots
	\[
	\begin{aligned}
	D(t)&= 18t + b \\[0.3cm]
	221&= 18(1) + b \\[0.3cm]
	221&= 18 + b \\[0.3cm]
	b&= 203
	\end{aligned}
	\]
Therefore, $D(t)= 18t + 203$. \pspace

\item The slope of $D(t)= 18t + 203$ is $m= 18$. Interpreting the slope $m= 18$ as $m= \dfrac{\Delta D}{\Delta t}$, we see that for each additional year, the demand increases by 18~thousand units; that is, there is a yearly increase in demand of 18,000~units. The $y$-intercept of $D(t)= 18t + 203$ is $b= 203$. The $y$-intercept occurs when $t= 0$, i.e. this current year. Therefore, the current demand is 203~thousand units. \pspace

\item We have\dots
	\[
	D(5)= 18(5) + 203= 90 + 203= 293
	\]
Therefore, if the rate of change in demand is constant, the demand in 5~years will be 293~thousand units. \pspace

\item We have\dots
	\[
	\begin{aligned}
	D(t)&= 18t + 203 \\[0.3cm]
	400&= 18t + 203 \\[0.3cm]
	18t&= 197 \\[0.3cm]
	t&= 10.94
	\end{aligned}
	\]
Therefore, the demand will be 400~thousand units in 10.94~years, i.e. 10~years, 11~months, and 9~days. 
\end{enumerate}


\end{document}