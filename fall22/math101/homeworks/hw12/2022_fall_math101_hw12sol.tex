\documentclass[11pt,letterpaper]{article}
\usepackage[lmargin=1in,rmargin=1in,tmargin=1in,bmargin=1in]{geometry}
\usepackage{../style/homework}
\usepackage{../style/commands}
\setbool{quotetype}{true} % True: Side; False: Under
\setbool{hideans}{false} % Student: True; Instructor: False

% -------------------
% Content
% -------------------
\begin{document}

\homework{12: Due 11/07}{The good thing about science is that it's true whether or not you believe it.}{Neil deGrasse Tyson}

% Problem 1
\problem{10} Consider the line $2x - 5y= 4$.
	\begin{enumerate}[(a)]
	\item Is $(-2, 0)$ on the line? Explain.
	\item Is $(-3, -2)$ on the line? Explain.
	\item Showing all your work, find two points, distinct from $(-2, 0)$ and $(-3, -2)$, on the given line. 
	\end{enumerate} \pspace

\sol 
\begin{enumerate}[(a)]
\item We have\dots
	\[
	\begin{aligned}
	2x - 5y&= 4 \\[0.3cm]
	2(-2) - 5(0)&\stackrel{?}{=} 4 \\[0.3cm]
	-4 - 0&\stackrel{?}{=} 4 \\[0.3cm]
	-4 &\neq 4 \\
	&\;\;\text{\xmark}
	\end{aligned}
	\]
Therefore, $(-2, 0)$ is not on the line $2x - 5y= 4$. \pspace

\item We have\dots
	\[
	\begin{aligned}
	2x - 5y&= 4 \\[0.3cm]
	2(-3) - 5(-2)&\stackrel{?}{=} 4 \\[0.3cm]
	-6 + 10&\stackrel{?}{=} 4 \\[0.3cm]
	4&= 4 \\
	&\;\;\text{\cmark}
	\end{aligned}
	\]
Therefore, $(-3, -2)$ is on the line $2x - 5y= 4$. \pspace

\item Any $(x, y)$ which satisfy the equation $2x - 5y= 4$ are on the line. If $x= 0$, we have $-5y= 4$ so that $y= -\frac{4}{5}$. Therefore, $(0, -\frac{4}{5})$ is a point on the line. Similarly, if $y= 0$, we have $2x= 4$ so that $x= \frac{4}{2}= 2$. Therefore, $(2, 0)$ is on the line. 
\end{enumerate}



\newpage



% Problem 2
\problem{10} Consider the line $-3x - 5y= 10$.
	\begin{enumerate}[(a)]
	\item Find the slope of the line.
	\item Find the $y$-intercept of the line.
	\item Find this line as a linear function, $f(x)$.
	\item Using your $f(x)$ from (c), find a point on the line distinct from the $y$-intercept.	
	\end{enumerate} \pspace

\sol First, observe that\dots
	\[
	\begin{aligned}
	-3x - 5y&= 10 \\[0.3cm]
	-5y&= 3x + 10 \\[0.3cm]
	y&= -\frac{3}{5}\,x - 2
	\end{aligned}
	\]

\begin{enumerate}[(a)]
\item Because $y= -\frac{3}{5}\,x - 2$ is of the form $f(x)= mx + b$ with $m= -\frac{3}{5}$ and $b= -2$, the slope of the line is $-\frac{3}{5}$. \pspace

\item Because $y= -\frac{3}{5}\,x - 2$ is of the form $f(x)= mx + b$ with $m= -\frac{3}{5}$ and $b= -2$, the $y$-intercept of the line is $-2$, i.e. $(0, -2)$. \pspace

\item From the work above, we have $f(x)= -\frac{3}{5}\,x - 2$. \pspace

\item If $x= x_0$, then $\big(x_0, f(x_0) \big)$ is a point on the line. For instance, suppose that $x= 5$, then we have\dots
	\[
	f(5)= -\dfrac{3}{5} \cdot 5 - 2= -3 - 2= -5
	\]
Therefore, $(5, -5)$ is a point on the line. 
\end{enumerate}


\end{document}