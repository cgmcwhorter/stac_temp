\documentclass[11pt,letterpaper]{article}
\usepackage[lmargin=1in,rmargin=1in,tmargin=1in,bmargin=1in]{geometry}
\usepackage{../style/homework}
\usepackage{../style/commands}
\setbool{quotetype}{true} % True: Side; False: Under
\setbool{hideans}{false} % Student: True; Instructor: False

% -------------------
% Content
% -------------------
\begin{document}

\homework{18: Due 11/14}{Poetry is a precise a thing as geometry.}{Gustave Flaubert}

% Problem 1
\problem{10} Solve the following equation and verify your solution:
	\[
	4(x - 2)= 5 - 9x
	\] \pspace

\sol We have\dots
	\[
	\begin{aligned}
	4(x - 2)&= 5 - 9x \\[0.3cm]
	4x - 8&= 5 - 9x \\[0.3cm]
	13x - 8&= 5 \\[0.3cm]
	13x&= 13 \\[0.3cm]
	x&= 1
	\end{aligned}
	\] 
We can check/verify the solution:
	\[
	\begin{aligned}
	4(x - 2)&= 5 - 9x \\[0.3cm]
	4(1 - 2)&\stackrel{?}{=} 5 - 9(1) \\[0.3cm]
	4(-1)&\stackrel{?}{=} 5 - 9 \\[0.3cm]
	-4&= -4 \\[0.3cm]
	&\;\text{\cmark}
	\end{aligned}
	\]



\newpage



% Problem 2
\problem{10} Solve the following equation and check your solution:
	\[
	\dfrac{x - 3}{1 - x}= 5
	\] \pspace

\sol We have\dots
	\[
	\begin{aligned}
	\dfrac{x - 3}{1 - x}&= 5 \\[0.3cm]
	x - 3&= 5(1 - x) \\[0.3cm]
	x - 3&= 5 - 5x \\[0.3cm]
	6x - 3&= 5 \\[0.3cm]
	6x&= 8 \\[0.3cm]
	x&= \dfrac{8}{6} \\[0.3cm]
	x&= \dfrac{4}{3}
	\end{aligned}
	\] 
We can check/verify this solution:
	\[
	\begin{aligned}
	\dfrac{x - 3}{1 - x}&= 5 \\[0.3cm]
	\dfrac{\frac{4}{3} - 3}{1 - \frac{4}{3}}&\stackrel{?}{=} 5 \\[0.3cm]
	\dfrac{\frac{4}{3} - \frac{9}{3}}{\frac{3}{3} - \frac{4}{3}}&\stackrel{?}{=} 5 \\[0.3cm]
	\dfrac{-\frac{5}{3}}{-\frac{1}{3}}&\stackrel{?}{=} 5 \\[0.3cm]
	-\dfrac{5}{3} \cdot -\dfrac{3}{1}&\stackrel{?}{=} 5 \\[0.3cm]
	5&= 5 \\[0.3cm]
	&\;\text{\cmark}
	\end{aligned}
	\] 



\newpage



% Problem 3
\problem{10} If $f(x)= 5 - 3x$ and $g(x)= -3(x + 8)$, will there be a solution to $f(x)= g(x)$? Explain. \pspace

\sol There is a solution of an equation $f(x)= g(x)$ if and only if the graphs of $f(x)$ and $g(x)$ intersect at the point $\big(x, f(x) \big)= \big(x, g(x) \big)$. Both $f(x)$ and $g(x)$ are linear functions. Observe that $f(x)= 5 - 3x$ is a linear function with slope $m= -3$. The function $g(x)= -3(x + 8)= -3x - 24$ is a linear function with slope $m= -3$. But then the linear functions $f(x)$ and $g(x)$ have the same slope so that they are parallel. But then the lines do not intersect so that the equation $f(x)= g(x)$ cannot have a solution. 


\end{document}