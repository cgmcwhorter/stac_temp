\documentclass[11pt,letterpaper]{article}
\usepackage[lmargin=1in,rmargin=1in,tmargin=1in,bmargin=1in]{geometry}
\usepackage{../style/homework}
\usepackage{../style/commands}
\setbool{quotetype}{true} % True: Side; False: Under
\setbool{hideans}{false} % Student: True; Instructor: False

% -------------------
% Content
% -------------------
\begin{document}

\homework{22: Due 12/07}{Try to learn something about everything and everything about something.}{Thomas Huxley}

% Problem 1
\problem{10} Factor each of the following quadratic functions:
	\begin{enumerate}[(a)]
	\item $x^2 - 64$
	\item $2x^2 - 4x - 6$
	\item $x^2 - 2x - 35$
	\item $x^2 + 8x + 16$
	\item $27 + 6x - x^2$
	\end{enumerate} \pspace

\sol 
\begin{enumerate}[(a)]
\item 
	\[
	x^2 - 64= (x - 8)(x + 8)
	\] \pspace

\item 
	\[
	2x^2 - 4x - 6= 2(x^2 - 2x - 3)= 2(x - 3)(x + 1)
	\] \pspace

\item 
	\[
	x^2 - 2x - 35= (x - 7)(x + 5)
	\] \pspace

\item 
	\[
	x^2 + 8x + 16= (x + 4)(x + 4)= (x + 4)^2
	\] \pspace

\item 
	\[
	27 + 6x - x^2= -(x^2 - 6x - 27)= -(x - 9)(x + 3)
	\]
\end{enumerate}



\newpage



% Problem 2
\problem{10} Showing all your work, use the discriminant to show that  $10x^2 + 33x - 7$ factors `nicely' then factor the polynomial. \pspace

\sol The discriminant of a quadratic function $f(x)= ax^2 + bx + c$ is $D= b^2 - 4ac$. The function $f(x)= 10x^2 + 33x - 7$ is quadratic with $a= 10$, $b= 33$, and $c= -7$. But then $D= b^2 - 4ac= 33^2 - 4(10)(-7)= 1089 + 280= 1369= 37^2$. A quadratic function factors `nicely' if and only if the discriminant is a square. Because $D= 1369= 37^2$, the polynomial $10x^2 + 33x - 7$ factors `nicely.' To factor this polynomial, we find the roots of $10x^2 + 33x - 7$ using the quadratic formula, i.e. the solutions to $10x^2 + 33x - 7= 0$:
	\[
	\begin{aligned}
	x&= \dfrac{-b \pm \sqrt{b^2 - 4ac}}{2a} \\
	&= \dfrac{-33 \pm \sqrt{33^2 - 4(10)(-7)}}{2(10)} \\
	&= \dfrac{-33 \pm \sqrt{1089 + 280}}{20} \\
	&= \dfrac{-33 \pm \sqrt{1369}}{20} \\
	&= \dfrac{-33 \pm 37}{20} 
	\end{aligned}
	\]
Therefore, the roots are $x= \frac{-33 - 37}{20}= \frac{-70}{20}= -\frac{7}{2}$ and $x= \frac{-33 + 37}{20}= \frac{4}{20}= \frac{1}{5}$. Now recall given a quadratic function $ax^2 + bx + c$ with roots $r_1$ and $r_2$, the function factors as $a(x - r_1)(x - r_2)$. But then we have\dots
	\[
	10x^2 + 33x - 7= 10 \left(x - \dfrac{-7}{2} \right) \left(x - \dfrac{1}{5} \right)= 10 \left(x + \dfrac{7}{2} \right) \left(x - \dfrac{1}{5} \right)= 2 \left(x + \dfrac{7}{2} \right) \cdot 5 \left(x - \dfrac{1}{5} \right)= (2x + 7)(5x - 1)
	\]



\newpage



% Problem 3
\problem{10} Showing all your work, solve the following equation using factoring and then verify your solution:
	\[
	x= 30 - x^2
	\] \pspace

\sol We have\dots
	\[
	\begin{aligned}
	x= 30 &- x^2 \\[0.3cm]
	x^2 + x - 30&= 0 \\[0.3cm]
	(x + 6)(x - 5)&= 0 
	\end{aligned}
	\]
But then either $x + 6= 0$, which implies $x= -6$, or $x - 5= 0$, which implies $x= 5$. Therefore, the solutions are $x= -6, 5$. We can verify the solution $x= -6$:
	\[
	\begin{aligned}
	x&= 30 - x^2 \\
	-6 &\stackrel{?}{=} 30 - (-6)^2 \\
	-6 &\stackrel{?}{=} 30 - 36 \\
	-6&= -6 \\
	&\;\text{\cmark}
	\end{aligned}
	\]
and the solution $x= 5$:
	\[
	\begin{aligned}
	x&= 30 - x^2 \\
	5 &\stackrel{?}{=} 30 - 5^2 \\
	5 &\stackrel{?}{=} 30 - 25 \\
	5&= 5 \\
	&\;\text{\cmark}
	\end{aligned}
	\]



\end{document}