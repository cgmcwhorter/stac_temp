\documentclass[11pt,letterpaper]{article}
\usepackage[lmargin=1in,rmargin=1in,tmargin=1in,bmargin=1in]{geometry}
\usepackage{../style/homework}
\usepackage{../style/commands}
\setbool{quotetype}{true} % True: Side; False: Under
\setbool{hideans}{false} % Student: True; Instructor: False

% -------------------
% Content
% -------------------
\begin{document}

\homework{14: Due 11/07}{If you want to have good ideas, you must have many ideas.}{Linus Pauling}

% Problem 1
\problem{10} Showing all your work and explaining your reasoning, answer the following:
	\begin{enumerate}[(a)]
	\item Find the equation of the line through $(-5, 9)$ with slope $-\frac{3}{5}$.
	\item Find the equation of the line through $(0, -4)$ and $(-6, -11)$. 
	\end{enumerate} \pspace

\sol 
\begin{enumerate}[(a)]
\item Because the line is not vertical, we know that it has the form $y= mx + b$ for some $m, b$. Because the slope is $-\frac{3}{5}$, we know that $m= -\frac{3}{5}$. But then $y= -\frac{3}{5}\,x + b$. But as $(-5, 9)$ is on the line, we know\dots
	\[
	\begin{aligned}
	y&= -\frac{3}{5}\,x + b \\[0.3cm]
	9&= -\frac{3}{5} \cdot -5 + b \\[0.3cm]
	9&= 3 + b \\[0.3cm]
	b&= 6
	\end{aligned}
	\]
Therefore, the equation of the line is $y= -\frac{3}{5}\,x + 6$. \pspace

\item Because the line is not vertical, we know that it has the form $y= mx + b$ for some $m, b$. We know that the slope is\dots
	\[
	m= \dfrac{\Delta y}{\Delta x}= \dfrac{-11 - (-4)}{-6 - 0}= \dfrac{-7}{-6}= \dfrac{7}{6}
	\]
Therefore, $m= \frac{7}{6}$. Then $y= \frac{7}{6}\,x + b$. Because the line contains the point $(0, -4)$, we have\dots
	\[
	\begin{aligned}
	y&= \dfrac{7}{6}\,x + b \\[0.3cm]
	-4&= \dfrac{7}{6} \cdot 0 + b \\[0.3cm]
	b&= -4
	\end{aligned}
	\]
Therefore, the equation of the line is $y= \frac{7}{6}\,x - 4$. 
\end{enumerate}



\newpage



% Problem 2
\problem{10} Find the equation of the line with $x$-intercept $-7$ and $y$-intercept 3. \pspace

\sol Because the line is not vertical, the line has the form $y= mx + b$. If the $x$-intercept of the line is $-7$, then the line contains the point $(-7, 0)$. If the $y$-intercept of the line is 3, the line contains the point $(0, 3)$. The slope is\dots
	\[
	m= \dfrac{\Delta y}{\Delta x}= \dfrac{0 - 3}{-7 - 0}= \dfrac{-3}{-7}= \dfrac{3}{7}
	\]
Therefore, $y= \frac{3}{7}\,x + b$. Because the line contains the point $(0, 3)$, we have $3= \frac{3}{7} \cdot 0 + b$ so that $b= 3$. Therefore, the equation of the line is $y= \frac{3}{7}\,x + 3$. 


\end{document}