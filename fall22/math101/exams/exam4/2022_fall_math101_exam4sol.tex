\documentclass[12pt,letterpaper]{exam}
\usepackage[lmargin=1in,rmargin=1in,tmargin=1in,bmargin=1in]{geometry}
\usepackage{../style/exams}

% -------------------
% Course & Exam Information
% -------------------
\newcommand{\course}{MAT 101: Exam 4}
\newcommand{\term}{Fall -- 2022}
\newcommand{\examdate}{12/14/2022}
\newcommand{\timelimit}{85 Minutes}

\setbool{hideans}{false} % Student: True; Instructor: False

\DeclareMathOperator{\disc}{disc}

% -------------------
% Content
% -------------------
\begin{document}

\examtitle
\instructions{Write your name on the appropriate line on the exam cover sheet. This exam contains \numpages\ pages (including this cover page) and \numquestions\ questions. Check that you have every page of the exam. Answer the questions in the spaces provided on the question sheets. Be sure to answer every part of each question and show all your work.} 
\scores
%\bottomline
\newpage

% ---------
% Questions
% ---------
\begin{questions}

% Question 1
\newpage
\question[10] Sketch the function $y= (x + 6)^2 - 7$ as accurately as possible on the graph below. Your sketch should include the vertex and axis of symmetry. 
	\[
	\fbox{
	\begin{tikzpicture}[scale=2,every node/.style={scale=0.5}]
	\begin{axis}[
	grid=both,
	axis lines=middle,
	ticklabel style={fill=blue!5!white},
	xmin= -10.5, xmax=10.5,
	ymin= -10.5, ymax=10.5,
	xtick={-10,-8,-6,-4,-2,0,2,4,6,8,10},
	ytick={-10,-8,-6,-4,-2,0,2,4,6,8,10},
	minor tick = {-10,-9,...,10},
	xlabel=\(x\),ylabel=\(y\),
	]
	\draw[line width= 0.03cm, dotted] (-6,-10.5) -- (-6,10.5);
	\addplot[samples= 100, domain= -10.5:0, line width= 0.03cm] ({x}, {(x + 6)^2 - 7});
	\draw[fill=black] (-6,-7) circle (0.2);
	\end{axis}
	\end{tikzpicture}
	}
	\] \pspace

\sol Recall the vertex form of a quadratic function is $f(x)= a(x - P)^2 + Q$, where $(P, Q)$ is the vertex of the quadratic function $a$ is the $a$ in the standard form $ax^2 + bx + c$ of the quadratic function. We have $y= (x + 6)^2 - 7= 1 \big(x - (-6) \big)^2 + (-7)$. But then the vertex of this quadratic function is $(-6, -7)$. Furthermore, the axis of symmetry is $x= -6$. Because $a= 1 > 0$, the quadratic function opens upwards, i.e. is convex. Using this gives the sketch above. 



% Question 2
\newpage
\question[10] A quadratic function $f(x)= ax^2 + bx + c$ is plotted below. Find $a$, $b$, and $c$ for this function. 
	\[
	\fbox{
	\begin{tikzpicture}[scale=1,every node/.style={scale=0.5}]
	\begin{axis}[
	grid=both,
	axis lines=middle,
	ticklabel style={fill=blue!5!white},
	xmin= -10.5, xmax=10.5,
	ymin= -10.5, ymax=10.5,
	xtick={-10,-8,-6,-4,-2,0,2,4,6,8,10},
	ytick={-10,-8,-6,-4,-2,0,2,4,6,8,10},
	minor tick = {-10,-9,...,10},
	xlabel=\(x\),ylabel=\(y\),
	]
	\addplot[samples= 100, domain= -6:10, line width= 0.03cm] ({x}, {-1/4*(x - 2)^2 + 4});
	\end{axis}
	\end{tikzpicture}
	}
	\] \pspace

\sol The vertex form of a quadratic function is $f(x)= a(x - P)^2 + Q$, where $(P, Q)$ is the vertex of the quadratic function $a$ is the $a$ in the standard form $ax^2 + bx + c$ of the quadratic function. From the plot, we see that the vertex is $(2, 4)$. But then $f(x)= a(x - P)^2 + Q= a(x - 2)^2 + 4$. Because the parabola opens downwards, we know $a < 0$. We see that the parabola contains the points $(-4, 5)$, $(-2, 0)$, $(0, 3)$, $(2, 4)$, $(4, 3)$, $(6, 0)$, and $(8, -5)$. Using the point $(0, 3)$, we have\dots
	\[
	\begin{gathered}
	f(x)= a(x - 2)^2 + 4 \\
	f(0)= a(0 - 2)^2 + 4 \\
	3= 4a + 4 \\
	4a= -1 \\
	a= -\dfrac{1}{4}
	\end{gathered}
	\]
Therefore, $f(x)= -\frac{1}{4} (x - 2)^2 + 4$. \pspace

Alternatively, if $f(x)$ is a quadratic function with roots $r_1, r_2$, then $f(x)= a(x - r_1)(x - r_2)$, where $a$ is the $a$ in the standard form $ax^2 + bx + c$. From the plot, we can see the parabola has $x$-intercepts $(-2, 0)$ and $(6, 0)$, i.e. the roots are $x= -2, 6$. But then $f(x)= a(x - r_1)(x - r_2)= a \big(x - (-2) \big) (x - 6)= a(x + 2)(x - 6)$. We see that the parabola contains the points $(-4, 5)$, $(-2, 0)$, $(0, 3)$, $(2, 4)$, $(4, 3)$, $(6, 0)$, and $(8, -5)$. Using the point $(0, 3)$, we have\dots
	\[
	\begin{gathered}
	f(x)= a(x + 2)(x - 6) \\
	f(0)= a(0 + 2)(0 - 6) \\
	3= a \cdot 2 \cdot -6 \\
	3= -12a \\
	a= -\dfrac{1}{4}
	\end{gathered}
	\]
Therefore, $f(x)= -\frac{1}{4} (x - 2)^2 + 4$.



% Question 3
\newpage
\question[10] Consider the quadratic function $f(x)= \dfrac{5}{3} - \left(x + \dfrac{3}{2} \right)^2$.
	\begin{enumerate}[(a)]
	\item Find the vertex and axis of symmetry for $f(x)$.
	\item Does this parabola open upwards or downwards?
	\item Is this parabola concave or convex?
	\item Does this parabola have a maximum or minimum? 
	\item Find the maximum or minimum of $f(x)$, if it exists. 
	\end{enumerate} \pspace

\sol 
\begin{enumerate}[(a)]
\item The vertex form of a quadratic function is $f(x)= a(x - P)^2 + Q$, where $(P, Q)$ is the vertex of the quadratic function $a$ is the $a$ in the standard form $ax^2 + bx + c$ of the quadratic function. This also forces the axis of symmetry to be the line $x= P$. Observe\dots
	\[
	f(x)= \dfrac{5}{3} - \left(x + \dfrac{3}{2} \right)^2= -\left(x - \left(- \dfrac{3}{2} \right) \right)^2 + \dfrac{5}{3}
	\]
Therefore, we have $(P, Q)= (-\frac{3}{2}, \frac{5}{3})$ and $a= -1$. The vertex is $(-\frac{3}{2}, \frac{5}{3})$ and the axis of symmetry is $x= -\frac{3}{2}$. \pspace

\item Because $a= -1 < 0$, the parabola opens downwards. \pspace 

\item Because $a= -1 < 0$, the parabola is convex. \pspace

\item Because $a= -1 < 0$, we know the parabola opens downwards. But then the parabola does not have a minimum value but does have a maximum value. \pspace

\item The maximum value is the $y$-value of the vertex. From (a), we know the vertex is $(-\frac{3}{2}, \frac{5}{3})$. Therefore, the maximum value for $f(x)$ is $\frac{5}{3}$. 
\end{enumerate}



% Question 4
\newpage
\question[10] Showing all your work, find the vertex form of $f(x)= 4x^2 + 4x - 6$. \pspace

\sol By completing the square, we have\dots
	\[
	\begin{gathered}
	4x^2 + 4x - 6 \\
	4 \left(x^2 + x - \dfrac{3}{2} \right) \\
	4 \left(x^2 + x + \left(\dfrac{1}{4} - \dfrac{1}{4} \right) - \dfrac{3}{2} \right) \\
	4 \left( \left(x^2 + x + \dfrac{1}{4} \right) - \dfrac{1}{4} - \dfrac{3}{2} \right) \\
	4 \left( \left(x + \dfrac{1}{2} \right)^2 - \dfrac{7}{4} \right) \\
	4 \left(x + \dfrac{1}{2} \right)^2 - 7
	\end{gathered}
	\]
Using the `evaluation method', we know the vertex occurs at $x= -\frac{b}{2a}= -\frac{4}{2(4)}= -\frac{4}{8}= -\frac{1}{2}$. We have\dots
	\[
	f\left( -\dfrac{1}{2} \right)= 4 \left( -\dfrac{1}{2} \right)^2 + 4 \left( -\dfrac{1}{2} \right) - 6= 4 \cdot \dfrac{1}{4} - 2 - 6= 1 - 2 - 6= -7
	\]
Therefore, the vertex is $(-\frac{1}{2}, -7)$. We know the vertex from of a quadratic function is $f(x)= a(x - P)^2 + Q$, where $(P, Q)$ is the vertex of the quadratic function $a$ is the $a$ in the standard form $ax^2 + bx + c$ of the quadratic function. But then $f(x)= a (x - P)^2 + Q= 4 \left(x - (-\frac{1}{2}) \right)^2 + (-7)= 4(x + \frac{1}{2})^2 - 7$. 



% Question 5
\newpage
\question[10] Find the $y$ and $x$-intercepts for the function $f(x)= x^2 + 21x - 72$. \pspace

\sol We know the $y$-intercept is the point where the graph of the function intersects the $y$-axis. But the $y$-axis is the line $x= 0$. Therefore, the $y$-intercept is the point on the graph of $f(x)$ with $x= 0$. We have $f(0)= 0^2 + 21(0) - 72= 0 + 0 - 72= -72$. Therefore, the $y$-intercept is $-72$, i.e. the point $(0, -72)$. \pspace

The $x$-intercept(s)---if they exist---is the point(s) where the graph of $f(x)$ intersects the $x$-axis. The $x$-axis is the line $y= 0$. But then the $x$-intercept(s) are the points on the graph of $f(x)$ with $y= f(x)= 0$. We have\dots	
	\[
	\begin{aligned}
	x^2 + 21x - 72= 0 \\
	(x + 24)(x - 3)= 0 
	\end{aligned}
	\]
This implies that either $x + 24= 0$, i.e. $x= -24$, or $x - 3= 0$, which implies $x= 3$. Therefore, the $x$-intercepts are $x= -24, 3$, i.e. the points $(-24, 0)$ and $(3, 0)$. \pspace

Alternatively, we could use the quadratic formula with $a= 1$, $b= 21$, and $c= -72$: 
	\[
	\begin{aligned}
	x&= \dfrac{-b \pm \sqrt{b^2 - 4ac}}{2a} \\[0.3cm]
	&= \dfrac{-21 \pm \sqrt{21^2 - 4(1)(-72)}}{2(1)} \\[0.3cm]
	&= \dfrac{-21 \pm \sqrt{441 + 288}}{2} \\[0.3cm]
	&= \dfrac{-21 \pm \sqrt{729}}{2} \\[0.3cm]
	&= \dfrac{-21 \pm 27}{2}
	\end{aligned}
	\]
Therefore, the $x$-intercepts are $x= \frac{-21 - 27}{2}= \frac{-48}{2}= -24$ and $x= \frac{-21 + 27}{2}= \frac{6}{2}= 3$, i.e. the points $(-24, 0)$ and $(3, 0)$. 



% Question 6
\newpage
\question[10] Use the discriminant of $f(x)= x^2 - 3x - 108$ to show that $f(x)$ has a `nice' factorization and then find its factorization. \pspace

\sol The discriminant of a quadratic function $ax^2 + bx + c$ is $b^2 - 4ac$. But then\dots
	\[
	\disc f(x)= b^2 - 4ac= (-3)^2 - 4(1)(-108)= 9 + 432= 441= (21)^2
	\]
We know that a quadratic function has a `nice' factorization if and only if its discriminant is a perfect square. Because $\disc f(x)= 21^2$ is a perfect square, $f(x)$ has a `nice' factorization. \pspace

To factor $f(x)$, we find factors of $108$ that sum to $-3$. Because $-108 < 0$, the factors must have opposite signs. \par
	\begin{table}[!ht]
	\centering
	\underline{\bfseries 108} \pvspace{0.2cm}
	\begin{tabular}{rr} 
	$1 \cdot -108$ & $-107$ \\
	$-1 \cdot 108$ & $107$ \\
	$2 \cdot -54$ & $-52$ \\
	$-2 \cdot 54$ & $52$ \\
	$3 \cdot -36$ & $-33$ \\
	$-3 \cdot 36$ & $33$ \\
	$4 \cdot -27$ & $-23$ \\
	$-4 \cdot 27$ & $23$ \\
	$6 \cdot -18$ & $-12$ \\
	$-6 \cdot 18$ & $12$ \\ \hline
	\multicolumn{1}{|r}{$9 \cdot -12$} & \multicolumn{1}{r|}{$-3$} \\ \hline
	$-9 \cdot 12$ & $3$
	\end{tabular}
	\end{table} \par
Therefore, we have\dots
	\[
	f(x)= (x + 9)(x - 12)
	\]



% Question 7
\newpage
\question[10] Use the discriminant of some quadratic function to show that the equation given below does not have a `nice' solution.
	\[
	31= x(14 - x)
	\] \pspace

\sol We have\dots
	\[
	\begin{gathered}
	31= x(14 - x) \\[0.3cm]
	31= 14x - x^2 \\[0.3cm]
	x^2 - 14x + 31= 0 
	\end{gathered}
	\]
Therefore, the original equation has a solution if and only if $x^2 - 14x + 31= 0$. The solutions to $x^2 - 14x + 31= 0$ correspond to roots of $f(x)$. We know that a quadratic function has `nice' roots if and only if the discriminant of the quadratic function is a perfect square. The discriminant of a quadratic function $ax^2 + bx + c$ is $b^2 - 4ac$. But then\dots
	\[
	\disc(x^2 - 14x + 31)= (-14)^2 - 4(1)31= 196 - 124= 72
	\]
Because $72$ is not a perfect square, $x^2 - 14x + 31$ does not have `nice' roots so that $31= x(14 - x)$ does not have a `nice' solution. 



% Question 8
\newpage
\question[10] Showing all your work, factor the polynomial $x^2 - 23x - 24$. Verify that your factorization is correct. \pspace

\sol To factor $x^2 - 23x - 24$, we find factors of $24$ that sum to $-23$. Because $-24 < 0$, the factors must have opposite signs. \par
	\begin{table}[!ht]
	\centering
	\underline{\bfseries 24} \pvspace{0.2cm}
	\begin{tabular}{rr} \hline
	\multicolumn{1}{|r}{$1 \cdot -24$} & \multicolumn{1}{r|}{$-23$} \\ \hline
	$-1 \cdot 24$ & $23$ \\
	$2 \cdot -12$ & $-10$ \\
	$-2 \cdot 12$ & $10$ \\
	$3 \cdot -8$ & $-5$ \\
	$-3 \cdot 8$ & $5$ \\
	$4 \cdot -6$ & $-2$ \\
	$-4 \cdot 6$ & $2$
	\end{tabular}
	\end{table} \par
Therefore, we have\dots
	\[
	x^2 - 23x - 24= (x + 1)(x - 24)
	\] \pspace
We verify this factorization is correct:
	\[
	(x + 1)(x - 24)= x^2 - 24x + x - 24= x^2 - 23x - 24
	\]



% Question 9
\newpage
\question[10] Showing all your work, factor the polynomial $x^2 + 17x - 84$. \pspace

\sol To factor $x^2 + 17x - 84$, we find factors of $84$ that sum to $17$. Because $-84 < 0$, the factors must have opposite signs. \par
	\begin{table}[!ht]
	\centering
	\underline{\bfseries 84} \pvspace{0.2cm}
	\begin{tabular}{rr} 
	$1 \cdot -84$ & $-83$ \\
	$-1 \cdot 84$ & $83$ \\
	$2 \cdot -42$ & $-40$ \\
	$-2 \cdot 42$ & $40$ \\
	$3 \cdot -28$ & $-25$ \\
	$-3 \cdot 28$ & $25$ \\
	$4 \cdot -21$ & $-17$ \\ \hline
	\multicolumn{1}{|r}{$-4 \cdot 21$} & \multicolumn{1}{r|}{$17$} \\ \hline
	$6 \cdot -14$ & $-8$ \\
	$-6 \cdot 14$ & $8$ \\
	$7 \cdot -12$ & $-5$ \\
	$-7 \cdot 12$ & $5$
	\end{tabular}
	\end{table} \par
Therefore, we have\dots
	\[
	x^2 + 17x - 84= (x - 4)(x + 21)
	\] 



% Question 10
\newpage
\question[10] Showing all your work, factor the polynomial $7x^2 + 18x - 9$. \pspace

\sol To factor $7x^2 + 18x - 9$, we find factors of $7 \cdot -9= -63$ that sum to $18$. Because $-63 < 0$, the factors must have opposite signs. \par
	\begin{table}[!ht]
	\centering
	\underline{\bfseries 63} \pvspace{0.2cm}
	\begin{tabular}{rr} 
	$1 \cdot -63$ & $-62$ \\
	$-1 \cdot 63$ & $62$ \\
	$3 \cdot -21$ & $-18$ \\ \hline
	\multicolumn{1}{|r}{$-3 \cdot 21$} & \multicolumn{1}{r|}{$18$} \\ \hline
	$7 \cdot -9$ & $-2$ \\
	$-7 \cdot 9$ & $2$
	\end{tabular}
	\end{table} \par
Therefore, we have\dots
	\[
	7x^2 + 18x - 9= 7x^2 + (-3x + 21x) - 9= (7x^2 - 3x) + (21x - 9)= x(7x - 3) + 3(7x - 3)= (7x - 3)(x + 3)
	\] 



% Question 11
\newpage
\question[10] Showing all your work, use the quadratic formula to factor the polynomial $288x^2 - 1524x + 935$. \pspace

\sol If a quadratic function $ax^2 + bx + c$ has roots $r_1, r_2$, then it factors as $a(x - r_1)(x - r_2)$. We then use the quadratic function to find the roots of $288x^2 - 1524x + 935$: 
	\[
	\begin{aligned}
	x&= \dfrac{-b \pm \sqrt{b^2 - 4ac}}{2a} \\[0.3cm]
	&= \dfrac{-(-1524) \pm \sqrt{(-1524)^2 - 4(288)935}}{2(288)} \\[0.3cm]
	&= \dfrac{1524 \pm \sqrt{2322576 - 1077120}}{576} \\[0.3cm]
	&= \dfrac{1524 \pm \sqrt{1245456}}{576} \\[0.3cm]
	&= \dfrac{1524 \pm 1116}{576}
	\end{aligned}
	\]
Therefore, the roots are $r_1= \frac{1524 - 1116}{576}= \frac{408}{576}= \frac{17}{24}$ and $r_2= \frac{1524 + 1116}{576}= \frac{2640}{576}= \frac{55}{12}$. Therefore, we have\dots \pspace
	\[
	\begin{gathered}
	288x^2 - 1524x + 935 \\[0.3cm]
	288 \left(x - \dfrac{17}{24} \right) \left(x - \dfrac{55}{12} \right) \\[0.3cm]
	(24 \cdot 12) \left(x - \dfrac{17}{24} \right) \left(x - \dfrac{55}{12} \right) \\[0.3cm]
	24 \left(x - \dfrac{17}{24} \right) \cdot 12 \left(x - \dfrac{55}{12} \right) \\[0.3cm]
	(24x - 17)(12x - 55)
	\end{gathered}
	\]



% Question 12
\newpage
\question[10] Showing all your work, solve the equation below then verify that your solution(s) are correct:
	\[
	x^2 + 9= 10x
	\] \pspace

\sol By factoring, we have\dots
	\[
	\begin{gathered}
	x^2 + 9= 10x \\[0.3cm]
	x^2 - 10x + 9= 0 \\[0.3cm]
	(x - 1)(x - 9)= 0 
	\end{gathered}
	\]
This implies $x - 1= 0$, which implies $x= 1$, or $x - 9= 0$, which implies $x= 9$. Alternatively, the equation has a solution if and only if $x^2 - 10x + 9= 0$. We can then use the quadratic formula to solve this equation:
	\[
	\begin{aligned}
	x&= \dfrac{-b \pm \sqrt{b^2 - 4ac}}{2a} \\[0.3cm]
	&= \dfrac{-(-10) \pm \sqrt{(-10)^2 - 4(1)9}}{2(1)} \\[0.3cm]
	&= \dfrac{10 \pm \sqrt{100 - 36}}{2} \\[0.3cm]
	&= \dfrac{10 \pm \sqrt{64}}{2} \\[0.3cm]
	&= \dfrac{10 \pm 8}{2}
	\end{aligned}
	\]
Therefore, the solutions are $x= \frac{10 - 8}{2}= \frac{2}{2}= 1$ and $x= \frac{10 + 8}{2}= \frac{18}{2}= 9$. \pspace

We now verify these solutions:
	\[
	\begin{aligned}
	x^2 + 9&= 10x &\qquad\qquad x^2 + 9&= 10x \\
	1^2 + 9&\stackrel{?}{=} 10(1) & 9^2 + 9&\stackrel{?}{=} 10(9) \\
	1 + 9&\stackrel{?}{=} 10 & 81 + 9&\stackrel{?}{=} 90 \\
	10&= 10 & 90&= 90 \\
	&\text{\cmark} & &\text{\cmark}
	\end{aligned}
	\]



% Question 13
\newpage
\question[10] Showing all your work, solve the equation below:
	\[
	6x^2= 5 - 7x
	\] \pspace

\sol We have\dots
	\[
	\begin{gathered}
	6x^2= 5 - 7x \\[0.3cm]
	6x^2 + 7x - 5= 0
	\end{gathered}
	\]
Therefore, the original equation has a solution if and only if $6x^2 + 7x - 5= 0$. Solving this using factoring, we have\dots
	\[
	\begin{gathered}
	6x^2 + 7x - 5= 0 \\[0.3cm]
	(3x + 5)(2x - 1)= 0 
	\end{gathered}
	\]
But then either $3x + 5= 0$, which implies $x= -\frac{5}{3}$, or $2x - 1=0$, which implies $x= \frac{1}{2}$. \pspace

Alternatively, using the quadratic formula, we have\dots
	\[
	\begin{aligned}
	x&= \dfrac{-b \pm \sqrt{b^2 - 4ac}}{2a} \\[0.3cm]
	&= \dfrac{-7 \pm \sqrt{7^2 - 4(6)(-5)}}{2(6)} \\[0.3cm]
	&= \dfrac{-7 \pm \sqrt{49 + 120}}{12} \\[0.3cm]
	&= \dfrac{-7 \pm \sqrt{169}}{12} \\[0.3cm]
	&= \dfrac{-7 \pm 13}{12}
	\end{aligned}	
	\]
Therefore, the solutions are $x= \frac{-7 - 13}{12}= \frac{-20}{12}= -\frac{5}{3}$ and $x= \frac{-7 + 13}{12}= \frac{6}{12}= \frac{1}{2}$. 



% Question 14
\newpage
\question[10] Showing all your work, solve the equation below:
	\[
	-x^2= 2(23 - 7x)
	\] \pspace

\sol We have\dots
	\[
	\begin{gathered}
	-x^2= 2(23 - 7x) \\[0.3cm]
	-x^2= 46 - 14x \\[0.3cm]
	0= x^2 - 14x + 46
	\end{gathered}
	\]
Observe that $b^2 - 4ac= (-14)^2 - 4(1)46= 196 - 184= 12$ is not a perfect square. Therefore, $x^2 - 14x + 46$ does not factor `nicely.' Therefore, we use the quadratic formula to solve the equation:
	\[
	\begin{aligned}
	x&= \dfrac{-b \pm \sqrt{b^2 - 4ac}}{2a} \\[0.3cm]
	&= \dfrac{-(-14) \pm \sqrt{(-14)^2 - 4(1)46}}{2(1)} \\[0.3cm]
	&= \dfrac{14 \pm \sqrt{196 - 164}}{2} \\[0.3cm]
	&= \dfrac{14 \pm \sqrt{12}}{2} \\[0.3cm]
	&= \dfrac{14 \pm \sqrt{4 \cdot 3}}{2} \\[0.3cm]
	&= \dfrac{14 \pm 2 \sqrt{3}}{2} \\[0.3cm]
	&= 7 \pm \sqrt{3}
	\end{aligned}	
	\]
Therefore, the solutions are $x= 7 - \sqrt{3} \approx 5.268$ and $x= 7 + \sqrt{3} \approx 8.732$. 


\end{questions}
\end{document}