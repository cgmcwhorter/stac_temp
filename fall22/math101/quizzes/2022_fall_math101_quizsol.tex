\documentclass[11pt,letterpaper]{article}
\usepackage[lmargin=1in,rmargin=1in,bmargin=1in,tmargin=1in]{geometry}
\usepackage{style/quiz}
\usepackage{style/commands}

% -------------------
% Content
% -------------------
\begin{document}
\thispagestyle{title}


% Quiz 1
\quizsol \textit{True/False}: Both $12= 3 \cdot 4$ and $12= 2^2 \cdot 3$ are prime factorizations of 12.  \pspace

\sol The statement is \textit{false}. A factorization of an integer $n$ is a product of integers that yields $n$. For instance, if $n= 100$, then $n= 1 \cdot 100, 10 \cdot 10, 5 \cdot 20, \ldots$ are all factorizations of 100. A prime factorization is a factorization where all the numbers in the product are primes or powers of primes. [If $n$ is prime, we allow $n= n$ to be the prime factorization, i.e. the `empty' product.] Then in the instance of $n= 100$, the factorizations $5 \cdot 20$ cannot be a prime factorization because 20 is not prime. In the given problem, $12= 3 \cdot 4$ is \textit{not} a prime factorization because 4 is not prime ($4= 2 \cdot 2$), while $12= 2^2 \cdot 3$ is a prime factorization because we have written 12 as a product of (powers of) primes. By the Fundamental Theorem of Arithmetic, every integer greater than 1 is either prime or can be written uniquely (up to order, e.g. $6= 2 \cdot 3= 3 \cdot 2$) as a product of primes. \pvspace{1.5cm}



% Quiz 2
\quizsol \textit{True/False}: $\gcd(2^{50} \cdot 3^{60} \cdot 7^{40}, 2^{30} \cdot 3^{70} \cdot 5^{90})= 2^{30} \cdot 3^{60} \cdot 5^{90} \cdot 7^{40}$ \pspace

\sol The statement is \textit{false}. If one wishes to compute $\gcd(a, b)$, one can compute the prime factorizations of $a, b$ and find the product of the primes appearing in \textit{both} prime factorizations of $a, b$, each to the smaller of the prime powers involved in the factorizations of $a, b$. For instance, if we wanted to compute $\gcd(2520, 74844)= \gcd(2^3 \cdot 3^2 \cdot 5^1 \cdot 7, 2^2 \cdot 3^5 \cdot 7 \cdot 11)$, observe that the primes occurring both are $2, 3, 7$. The smallest power for each is 2, 2, 1, respectively. Therefore, $\gcd(2520, 74844)= \gcd(2^3 \cdot 3^2 \cdot 5^1 \cdot 7, 2^2 \cdot 3^5 \cdot 7 \cdot 11)= 2^2 \cdot 3^2 \cdot 7^1= 252$. In the given problem, while the smallest power of each prime was chosen, \textit{every} prime was used rather than just the primes both factorizations have in common. \pvspace{1.5cm}



% Quiz 3
\quizsol \textit{True/False}: We compute the reduced result of $(3/7)/(45/56)$ as follows: 
	\[
	\dfrac{\;\;\dfrac{3}{7}\;\;}{\dfrac{45}{56}}= \dfrac{3}{7} \cdot \dfrac{56}{45}= \dfrac{3 \cdot 56}{7 \cdot 45}= \dfrac{168}{315}
	\] \pspace

\sol The statement is \textit{false}. While the given answer is equal to the reduced answer, the given answer is not reduced (for instance, the numerator and denominator are both divisible by 3). Therefore, the statement is false. Remember, one should always cancel wherever possible \textit{before} `multiplying straight across' in multiplication of rational numbers:
	\[
	\dfrac{\;\;\dfrac{3}{7}\;\;}{\dfrac{45}{56}}= \dfrac{3}{7} \cdot \dfrac{56}{45}= \dfrac{\cancel{3}^1}{\cancel{7}^1} \cdot \dfrac{\cancel{56}^{\,8}}{\cancel{45}^{\,15}}= \dfrac{1}{1} \cdot \dfrac{8}{15}= \dfrac{1 \cdot 8}{1 \cdot 15}= \dfrac{8}{15}
	\] 





\newpage





% Quiz 4
\quizsol \textit{True/False}: 
	\[
	\dfrac{x y^4 z^3}{z^{-3} \sqrt{x^2 y^4}}= \dfrac{1}{\sqrt{x}}
	\] \pspace

\sol The statement is \textit{false}. We can simplify this expression as follows:
	\[
	\dfrac{x y^4 z^3}{z^{-3} \sqrt{x^2 y^4}}= \dfrac{x y^4 z^3}{z^{-3} (x^{2/2} y^{4/2})}= \dfrac{x y^4 z^3}{z^{-3} x y^2}= \dfrac{x y^4 z^3 z^3}{x y^2}= \dfrac{y^4 z^6}{y^2}= y^2 z^6
	\]


% To compute 560 increased by 160\%, one computes $560(2.60)= 1456$. 


% Suppose a course has three exams: two prelims and a final exam. The final exam is worth twice as much as any of the other exams. If a student receives an 85\%, 89\%, and 81\% on the prelims and final exam, respectively, then their exam average is $85\% \left( \frac{1}{4} \right) + 89\%  \left( \frac{1}{4} \right) +  81\% \left( \frac{1}{2} \right)= 84\%$.  









\end{document}